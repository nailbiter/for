\documentclass[11pt]{article} % use larger type; default would be 10pt
%Problem 14
%Problem 17
%Problem 24
%Problem 27
%Problem 33
\usepackage[10pt]{type1ec}          % use only 10pt fonts
\usepackage[T1]{fontenc}
\usepackage{graphicx}
\usepackage{enumerate}
\usepackage{float}
\usepackage{CJKutf8}
\usepackage{subfig}
\usepackage{amsmath}
\usepackage{listings}
\usepackage{amsfonts}
\usepackage{hyperref}
\newtheorem{prob}{Problem}

\newenvironment{solution}%
{\par\textbf{Solution}\space }%
{\par}

\title{Introduction to Networks\\Homework 1}
\author{歐立思\\
9822058\\Department of Applied Mathematics}
\begin{document}
\begin{CJK}{UTF8}{bsmi}
\maketitle
\end{CJK}
\begin{prob}
	Exercise 14
\end{prob}
\begin{solution}
\begin{enumerate}[(a)]
	\item{Since we know speed and the distance that the data will travel, we can calculate RTT simply as \[\text{RTT}=
		2\frac{55\times10^9 \text{ m}}{3 \times 10^8 \text{ m/s}}=366.67 \text{ s}\]}
	\item{The assume that "delay" here means "RTT" (otherwise, answer should be divided by two):\[\text{delay}\times\text{bandwidth}=
		128\text{ kbps} \times 366.67\text{ s}=46933.76\text{ kbits}\]
		}
	\item{We will calculate transfer time as on page 49 of handout for the chapter 1:\[\text{transmission time}=\text{RTT}+
		\frac{\text{transfer size}}{\text{bandwidth}}=366.67\text{ s}+\frac{5\times8\times10^6\text{ bits}}{128 \times 10^3\text{ bits}}=
		679.17\text{ s}\]
		}
\end{enumerate}
\end{solution}
\begin{enumerate}[(a)]
	\item{\[\text{latency}=2(10\times10^{-6}\text{ sec}+\frac{5000\text{ bits}}{10^{9}\text{ bps}})=30 \mu\text{s}\]
		}
	\item{\[\text{latency}=4(10\times10^{-6}\text{ sec}+\frac{5000\text{ bits}}{10^{9}\text{ bps}})=60 \mu\text{s}\]
		}
	\item{To simplify our understanding of the problem and facilitate computations we virtually will separate the packet on two pieces: 
		we consider time it takes for the small part of the message (say, only first bit) to reach the destination and then calculate
		time it takes from the receiving of the first piece to the receiving of the second piece. As for the second part, it is simple:
		it will take only transmission time (that is, ratio of message size to bandwidth) because the message is \textit{continuous}, that
		is when first bit have arrived, it means that second is almost arrived etc. Now, let's calculate what time it takes for the
		first bit to reach the destination. This bit will experience all the propagation delays (for 4 channels) and also will
		have to wait in the three switches - in each switch it will spend time equal to transmission time of 128 bits. Putting these
		considerations into the numbers, we get the following: \[\text{latency}=
		4\times10\times10^{-6}\text{ sec}+3\frac{128\text{ bits}}{10^{9}\text{ bps}}+\frac{5000\text{ bits}}{10^{9}\text{ bps}}=45.38\mu
		\text{s}\]
		}
\end{enumerate}

\begin{prob}
	Exercise 17
\end{prob}
\begin{solution}
	As we know in general, \[\text{data}=\text{delay}\times\text{bandwidth}=\frac{\text{distance}}{\text{propagation speed}}\times
	\text{bandwidth}\]
	Therefore, for the first case \[\text{distance}=\frac{1500\text{ bits}\times2\times10^8\text{ m/s}}{100\times10^6\text{ bps}}=
	3000\text{ m}=3\text{ km}\]
	For the second case \[\frac{\text{distance}}{\text{propagation speed}}=\frac{\text{data}}{\text{bandwidth}}\]
	\[\frac{100n\text{ m}}{2\times{10^8}\text{ m/s}}=\frac{10\times \lfloor n \rfloor+1500\text{ bits}}{100\times10^6\text{ bps}}\]
	where $\lfloor n \rfloor$ means the floor of $n$ (that is, the biggest integer number that is not bigger than $n$)
	\[ 10^{10} n=2\times10^9\lfloor n\rfloor + 3\times 10^{11}\]
	For the purpose of getting estimation for $n$, let's omit the floor on the right hand side and solve
	\[ 10^{10} n=2\times10^9 n + 3\times 10^{11}\implies n=37.5\]
	Consequently, let's try $n=37+\alpha,\;0\leq\alpha<1$
	\[ 10^{10} (37+\alpha)=2\times10^9 \times37 + 3\times 10^{11} \implies \alpha=0.4\]
	Therefore, $n=37.4$ and $\text{length}=3740\text{ m}$
\end{solution}
\begin{prob}Exercise 27\end{prob}
\begin{solution}
	In all derivations below we neglect RTT in compare with the transmission time
	\begin{enumerate}[(a)]
		\item{\[\text{bandwidth}=\frac{\text{data}}{\text{time}}=\frac{1920\times1080\times24\text{ bits}}{1/30\text{ s}}=
			1493\text{ Mbps}\]}
		\item{\[\text{bandwidth}=\frac{\text{data}}{\text{time}}=\frac{8\text{ bits}}{1/8\times10^{-3}\text{ s}}=64\text{ Kbps}\]
			}
		\item{\[\text{bandwidth}=\frac{\text{data}}{\text{time}}=\frac{260\text{ bits}}{1/50\text{ s}}=13\text{ Kbps}\]
			}
		\item{\[\text{bandwidth}=\frac{\text{data}}{\text{time}}=24\text{ bits}\times88.2\times 10^3\text{ s}^{-1}=2.1168\text{ Mbps}\]
			}
	\end{enumerate}
\end{solution}
\begin{prob}Exercise 33\end{prob}
	\begin{solution}
		Modified codes are attached and also listed below for the sake of completeness:\\
		\texttt{simplex-talk.c} (client)\\
		\lstinputlisting[language=C]{../forc/simplex-talk.c}
		\texttt{simplex-talk-server.c} (server)\\
		\lstinputlisting[language=C]{../forc/simplex-talk-server.c}
	\end{solution}
\end{document}
