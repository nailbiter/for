\documentclass{beamer}
\usetheme{default}
\usepackage{mystyle}
\begin{document}

\begin{frame}{Problem}
\begin{center}Use the definition of the limit to find \[\lim_{x\to 2}\sqrt{x}\]\end{center}
\end{frame}
\begin{frame}{Step 1 (Guess the limit)}
We need to \textit{guess} the limit. In many cases, if the direct substitution gives \textit{reasonable} value, this is the value to go for. Thus,
in our case, we shall prove
\[\lim_{x\to 2}\sqrt{x}=\sqrt{2}\]
\end{frame}
\begin{frame}{Step 2 (Prove guess was correct)}
Assume $\epsilon>0$ is given. We need to find $\delta>0$, such that \[0<\myabs{x-2}<\delta\] implies \[\myabs{\sqrt{x}-\sqrt{2}}<\epsilon\]
\end{frame}
\begin{frame}{Step 2 (Prove guess was correct)}
Let us simplify expressions first. What we \textit{have} can be written in two ways
\[0<\myabs{x-2}<\delta\iff 2-\delta<x<2+\delta\]
what we \textit{want} to prove can be written as
\[\myabs{\sqrt{x}-\sqrt{2}}=\frac{\myabs{x-2}}{\sqrt{x}+\sqrt{2}}<\epsilon,\;x>0\]
\end{frame}
\begin{frame}{We want $x>0$}
We \textit{want} \[x>0\] and what we \textit{have} is \[2-\delta<x<2+\delta\]
Naturally, let us \textit{require} \[0<2-\delta\] (this can be achieved with \textit{small} delta, so, done with this)
\end{frame}
\begin{frame}{We want $\frac{\myabs{x-2}}{\sqrt{x}+\sqrt{2}}<\epsilon$}
We \textit{want}
\[\frac{\myabs{x-2}}{\sqrt{x}+\sqrt{2}}<\epsilon\]
and we \textit{have} \[\myabs{x-2}<\delta\]
\end{frame}
\begin{frame}{We want $\frac{\myabs{x-2}}{\sqrt{x}+\sqrt{2}}<\epsilon$}
Note that
\[\frac{\myabs{x-2}}{\sqrt{x}+\sqrt{2}}<\frac{\myabs{x-2}}{\sqrt{2}}\]
and we originally \textit{wanted}
\[\frac{\myabs{x-2}}{\sqrt{x}+\sqrt{2}}<\epsilon\]
Hence, we may say that now we \textit{want}
\[\frac{\myabs{x-2}}{\sqrt{2}}<\epsilon\]
(we want the inequality we \textit{want} to involve only pieces we \textit{have})
\end{frame}
\begin{frame}{We want $\frac{\myabs{x-2}}{\sqrt{x}+\sqrt{2}}<\epsilon$}
So, at the end of a day...\\
We \textit{want}
\[\frac{\myabs{x-2}}{\sqrt{2}}<\epsilon\]
and we \textit{have}
\[\myabs{x-2}<\delta\]
Naturally, let us \textit{require}
\[\delta<\sqrt{2}\epsilon\]
\end{frame}
\begin{frame}{Step 2 (Prove guess was correct)}
At the end, we have
\begin{enumerate}
\item We \textit{require} $0<2-\delta\iff \delta<2$
\item We \textit{require} $\delta<\sqrt{2}\epsilon$
\end{enumerate}
Naturally, we may \textit{take}
\[\delta=\frac{\min\mycbra{\sqrt{2}\epsilon,2}}{2}>0\]
as a particular $\delta$ to satisfy all requirements
\end{frame}

%A displayed formula:
%
%\[
%  \int_{-\infty}^\infty e^{-x^2} \, dx = \sqrt{\pi}
%\]
%
%An itemized list:
%
%\begin{itemize}
%  \item itemized item 1
%  \item itemized item 2
%  \item itemized item 3
%\end{itemize}
%
%\begin{theorem}
%  In a right triangle, the square of hypotenuse equals
%  the sum of squares of two other sides.
%\end{theorem}
%

\end{document}
