\documentclass[8pt]{article} % use larger type; default would be 10pt

%\usepackage[utf8]{inputenc} % set input encoding (not needed with XeLaTeX)
\usepackage[10pt]{type1ec}          % use only 10pt fonts
\usepackage[T1]{fontenc}
%\usepackage{CJK}
\usepackage{graphicx}
\usepackage{float}
\usepackage{CJKutf8}
\usepackage{subfig}
\usepackage{amsmath}
\usepackage{amsfonts}
\usepackage{hyperref}
\usepackage{enumerate}
\usepackage{enumitem}

\newtheorem{prob}{Problem}

\newenvironment{solution}%
{\par\textbf{Solution}\space }%
{\par}

\newenvironment{switch_table}[1]
{Switch #1\\
\begin{tabular}{|l|l|l|l|}
\hline
Incoming Interface & Incoming VC & Outgoing Interface & Outgoing VC\\
\hline
}
{
\hline
\end{tabular}
}

\title{Introduction to Networks\\Homework 3}
\author{歐立思\\
9822058\\Department of Applied Mathematics}

\begin{document}
\begin{CJK}{UTF8}{bsmi}
\maketitle
\end{CJK}
\begin{prob}Problem 2\end{prob}
\begin{solution}
\begin{enumerate}[label=(\alph*)]
\item{
\begin{switch_table}{1}
0 & 0 & 1 & 0\\
\end{switch_table}
\\\begin{switch_table}{2}
3 & 0 & 1 & 0\\
\end{switch_table}
\\Switch 3 so far does not have any records in virtual circuit table.
\\\begin{switch_table}{4}
3 & 0 & 0 & 0\\
\end{switch_table}
}
\item{
\begin{switch_table}{1}
0 & 0 & 1 & 0\\
\end{switch_table}
\\\begin{switch_table}{2}
3 & 0 & 1 & 0\\
\hline
0 & 1 & 1 & 1\\
\end{switch_table}
\\\begin{switch_table}{3}
3 & 0 & 0 & 1\\
\end{switch_table}
\\\begin{switch_table}{4}
3 & 0 & 0 & 0\\
\hline
3 & 1 & 1 & 0\\
\end{switch_table}
}
\item{
\begin{switch_table}{1}
0 & 0 & 1 & 0\\
\end{switch_table}
\\\begin{switch_table}{2}
3 & 0 & 1 & 0\\
\hline
0 & 0 & 1 & 1\\
\end{switch_table}
\\\begin{switch_table}{3}
3 & 0 & 0 & 1\\
\end{switch_table}
\\\begin{switch_table}{4}
3 & 0 & 0 & 0\\
\hline
3 & 1 & 1 & 0\\
\end{switch_table}
}
\end{enumerate}
\end{solution}
\end{document}
%2, 13, 15, 52, 56,68
%Incoming Interface & Incoming VC & Outgoing Interface & Outgoing VC\\
