\documentclass[8pt]{article} % use larger type; default would be 10pt

%\usepackage[utf8]{inputenc} % set input encoding (not needed with XeLaTeX)
\usepackage[10pt]{type1ec}          % use only 10pt fonts
\usepackage[T1]{fontenc}
%\usepackage{CJK}
\usepackage{graphicx}
\usepackage{float}
\usepackage{CJKutf8}
\usepackage{subfig}
\usepackage{amsmath}
\usepackage{amssymb}
\usepackage{amsfonts}
\usepackage{hyperref}
\usepackage{enumerate}
\usepackage{enumitem}

\newcommand{\mynorm}[1]{\left|\left|#1\right|\right|}
\newcommand{\myabs}[1]{\left|#1\right|}
\newcommand{\myset}[1]{\left\{#1\right\}}
\let\oldsum\sum
\renewcommand*{\sum}{\displaystyle\oldsum}

\title{Advanced Calculus, Exercise 20}
\begin{document}
\maketitle
\begin{enumerate}
\item{To begin with, $\forall x$ sequence $\sum_{n=1}^\infty c_nI(x-x_n)$ converges absolutely (for $\sum_{n=1}^\infty\myabs{c_nI(x-x_n)}\leq \sum_{n=1}^\infty\myabs{c_n}
	<\infty$) and hence is convergent. Furthermore, convergence is continuous, for $\forall \epsilon>0,\;\exists N\in\mathbb{N}\mid \sum_{n=N+1}^\infty \myabs{c_n}<\epsilon$
	(because $\sum_{n=1}^\infty\myabs{c_n}<\infty$), thus $\forall x\forall n>N\;\myabs{f(x)-f_n(x)}=\myabs{\sum_{n=N+1}^\infty c_nI(x-x_n)}\leq\sum_{n=N+1}^\infty\myabs{c_n}<
	\epsilon$, which is the very definition of the uniform convergence. Finally, to prove that $f$ is continuous on $[a,b]\setminus \{x_n\}_{n=1}^\infty$, assume that 
	arbitrary $\epsilon>0$ is given and take $N\in\mathbb{N}$, so that $\forall x\forall n>N,\;\myabs{f(x)-f_n(x)}<\epsilon$, as above. Fix 
	$x_0\in[a,b]\setminus \{x_n\}_{n=1}^\infty$. Then, function $f_{N+1}$ is a
	step function, which is constant on any interval, disjoint with $\{x_n\}_{n=1}^N$. Since $x\notin\{x_n\}_{n=1}^\infty\supset\{x_n\}_{n=1}^N$ and latter is a compact
	set, $\exists \delta>0$, such that $(x_0-\delta,x_0+\delta)\cap\{x_n\}_{n=1}^N=\emptyset$. Hence, $f_{N+1}$ is constant on $(x_0-\delta,x_0+\delta)$. Therefore,
	$\forall y\in(x_0-\delta,x_0+\delta),\; \myabs{f(x_0)-f(y)}\leq \myabs{f(x_0)-f_{N+1}(x_0)}+\myabs{f_{N+1}(x_0)-f_{N+1}(y)}+\myabs{f_{N+1}(y)-f(y)}\leq
	\epsilon+0+\epsilon=2\epsilon$. Since $\epsilon$ and $x_0$ were arbitrary, this finishes the proof.
}
\item{\begin{enumerate}[label=(\alph*)]
		\item{Indeed, $\forall x\in[1,2]$ we have $1+x>1$, hence geometric series $\sum_{n=1}^\infty\frac{1}{(1+x)^n=1/x}$ converges,
			hence $\sum_{n=1}^\infty f_n(x)=\sum_{n=1}^\infty\frac{x}{(1+x)^n}=1$ exists $\forall x\in[1,2]$.}
		\item{Yes, the convergence is uniform. To see this, we shall establish now the result, so-called Weierstrass M-test, which states that if $f_n:A\to\mathbb{R}$, such that
			$\forall x\in A,\;\myabs{f_n(x)}\leq c_n$ and $\sum_{n=1}^\infty c_n<\infty$, sum $\sum_{n=1}^\infty$ converges uniformly on $A$. 
			To show this, note that 
			as $\sum_{n=1}^\infty c_n<\infty$, $\forall \epsilon>0\exists N\in\mathbb{N},\;\mid \sum_{n=N+1}^\infty c_n<\epsilon$. Then, $\forall x\in A\forall m>n>N$ we have
			$\myabs{\sum_{k=1}^m f_k(x)-\sum_{k=1}^n f_k(x)}\leq \sum_{k=n+1}^m\myabs{f_k(x)}\leq \sum_{k=n+1}^m c_n\leq\sum_{k=n+1}^\infty c_n<\epsilon$. Hence, 
			$\forall x\in A$ partial sums of a series $\sum_{n=1}^\infty f_n(x)$ form a Cauchy sequence, which is thus convergent, due to completeness of $\mathbb{R}$. This
			shows convergence. Uniform convergence follows, for $\forall x\in A\forall n>N,\; \myabs{\sum_{k=1}^\infty f_k(x)-\sum_{k=1}^n f_k(x)}\leq
			\sum_{k=n+1}^\infty \myabs{f_k(x)}\leq \sum_{k=n+1}^\infty c_k\leq \sum_{k=N+1}^\infty c_k<\epsilon$. This proves the statement of Weierstrass M-test.

			Now, consider functions $f_n$ given. $f_n'(x)=\frac{(1+x)^n-xn(1+x)^{n-1}}{(1+x)^{2n}}=\frac{1-(n-1)x}{(1+x)^{n+1}}<0$ on $[1,2]$. Hence, 
			$\forall x\in[1,2],\; \myabs{f_n(x)}\leq \myabs{f_n(1)}=1/2^n$. As $\sum\frac{1}{2^n}$ converges, Weierstrass M-test guarantees uniform convergence.
			}
		\item{Yes, these two will be equal, due to uniform convergence}
	\end{enumerate}
	}
\item{Note that $\myabs{\frac{x^2/n^2}\leq 1/n^2}$ on $[0,1]$ and $\sum_{n=1}^\infty 1/n^2<\infty$. Hence, series converges uniformly due to M-test. Furthermore,
	as each summand is continuous, their sum is also continuous due to the uniformity of convergence.
	}
\item{Fix any $x_0\in\mathbb{R}$. We shall show that $f$ exists and
	is continuous at $x_0$, and this will imply the desired statement due to the arbitrary choice of $x_0$. Let
	$M:=\myabs{x_0}+1$. Consider $f_n(x):=\frac{\sin(nx) x^3}{n^2}$, defined on $[-M,M]\ni x_0$. Now, for $\forall x\in[-M,M]$ we have $\myabs{f_n(x)}=
	\frac{\myabs{\sin(nx)} \myabs{x}^3}{n^2}\leq \frac{M^3}{n^2}$. As $\sum_{n=1}^\infty M^3/n^2<\infty$, $\sum_{n=1}^\infty f_n$ converges uniformly on $[-M,M]$,
	and since each $f_n$ is continuous on $[-M,M]$, their sum is also continuous on $[-M,M]$ and in particular at $x_0\in[-M,M]$. Therefore, $f(x_0)$ exists
	(series converges) and continuous at $x_0$.
	}
\end{enumerate}
\end{document}
