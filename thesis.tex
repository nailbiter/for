%\documentclass[10pt]{article} % use larger type; default would be 10pt
\documentclass[14pt]{extarticle} % use larger type; default would be 10pt

\usepackage{mathtext}                 % підключення кирилиці у математичних формулах
                                          % (mathtext.sty входить в пакет t2).
\usepackage[T1,T2A]{fontenc}         % внутрішнє кодування шрифтів (може бути декілька);
                                          % вказане останнім діє по замовчуванню;
                                          % кириличне має співпадати з заданим в ukrhyph.tex.
\usepackage[utf8]{inputenc}       % кодування документа; замість cp866nav
                                          % може бути cp1251, koi8-u, macukr, iso88595, utf8.
\usepackage[english,ukrainian]{babel} % національна локалізація; може бути декілька
                                          % мов; остання з переліку діє по замовчуванню. 

%\usepackage{sectsty}   %in order to make chapter headings and title centered
%\chapterfont{\centering}

\usepackage{amsthm}
\usepackage{amsmath}
\usepackage{amsfonts}
\usepackage{graphicx}
\usepackage[pdftex]{hyperref}
\usepackage{caption}
\usepackage{subfig}
\usepackage{fancyhdr}

%custom command for title
\newcommand{\HRule}{\rule{\linewidth}{0.5mm}}

%for Re and Im like in the book
\renewcommand\Re{\operatorname{Re}}
\renewcommand\Im{\operatorname{Im}}

%put subscript under lim and others
\let\oldlim\lim
\renewcommand{\lim}{\displaystyle\oldlim}
\let\oldmin\min
\renewcommand{\min}{\displaystyle\oldmin}
\let\oldmax\max
\renewcommand{\max}{\displaystyle\oldmax}

%more space after \forall and \exists
\let\oldforall\forall
\renewcommand{\forall}{\oldforall\;}
\let\oldexists\exists
\renewcommand{\exists}{\oldexists\;}

%custom commands to save typing
\newcommand{\mynorm}[1]{\left|\left|#1\right|\right|}
\newcommand{\myabs}[1]{\left|#1\right|}
\newcommand{\myset}[1]{\left\{#1\right\}}

%custom footer
%\pagestyle{myheadings}
%\markright{\hfill \textcopyleft\;Т. А. Мельник, Курс лекцій з комплексного аналізу, Київ-- 2004\hfill}

%custom theorem environments
\newtheorem{definition}{Означення}[section]
\renewcommand{\thedefinition}{\arabic{definition}}
\newtheorem{example}{\indent Приклад}[section]
\renewcommand{\theexample}{\arabic{example}}
\newtheorem{exercise}{Вправа}
\newtheorem{theorem}{Теорема}
\newtheorem{proposition}{Твердження}[section]
\newtheorem{remark}{Зауваження}

\begin{document}
\begin{titlepage}
	\addtolength{\voffset}{-3cm}
	\setlength{\footskip}{5.5cm}
	\thispagestyle{fancy}
	\fancyfoot[C]{м. Київ -- 2013}
	\begin{center}
		\textsc{\Large Київський Національний Університет імені Тараса Шевченка}\\[0.5cm]
		\textsc{\Large Механіко-математичний факультет}\\[0.5cm]
		\textsc{\Large Кафедра інтегральних та диференціальних рівнянь}\\[1.5cm]

		\textsc{\Large курсова робота}\\
		\textsc{\Large на тему:}\\[0.5cm]

		% Title
		\HRule \\[0.4cm]
		{ \huge \bfseries Експоненціально дихотомічні лінійні системи диференціальних рівнянь}\\[0.4cm]

		\HRule \\[1.5cm]

		% Author and supervisor
		\begin{minipage}{0.4\textwidth}
			\begin{flushleft} \large
				\emph{Студент:}\\
				\textsc{Леонтьєв} Олексій Костянтинович
			\end{flushleft}
		\end{minipage}
		\begin{minipage}{0.4\textwidth}
			\begin{flushright} \large
				\emph{Керівник:} \\
				\textsc{Фекета} Петро Володимирович
			\end{flushright}
		\end{minipage}

		\vfill

		% Bottom of the page
		{\large \today}
	\end{center}
\end{titlepage}
\tableofcontents
\section*{Вступ}
\addcontentsline{toc}{section}{Вступ}
Експоненціальна дихотомія є явищем, що має місце в системах однорідних лінійних диференціальних рівнянь. Вивчення цього явища
закладено ще роботами Ж. Адамара та O. Перрона. В той час як дихотомія при першому погляді на означення здається ненатуральним
і складним концептом, вона тісно пов’язана з експоненціальною поведінкою розв’язків рівняння і дає інформацію про існування рівномірно обмежених
в розв’язків однорідних систем із збуреннями, якщо останні також рівномірно обмежені.

Попри всю свою цікавість і корисність, автору невідома жодна монографія, що вивчала б дихотомію, більшість вводить її просто як допоміжне
означення для встановлення інших результатів. Це робить роботу з пошуку інформації дещо більш важкою, адже необхідні факти і леми
розподілені по великій кількості літератури, більшість з якої, у свою чергу, невідома автору цієї роботи.

У даній роботі використовувалася література: Ю. Л. Далецкий, М. Г. Крейн
\emph{Устойчивость решений дифференциальных уравнений в банаховом пространстве} та Митропольський Ю. А., Самойленко А. М., Кулик В. Л.
\emph{Исследование дихотомии линейных систем дифференциальных уравнений с помощью функций Ляпунова}. З них перша, як видно з назви,
досліджує широкий спектр явищ, пов’язаних із стійкістю розв’язків диференціальних рівнянь (в тому числі, і дихотомію) в дещо
більш загальному контексті банахових просторів, в той час як друга використовує дихотомію для дослідження інваріантних торів.

Варто зазначити, що ця робота носить поки що суто реферативних характер і автор вважає її метою скоріше зібрати докупи розрізнені факти, пов’язані
з дихотомією і перекласти їх на більш "сучасну{}"{} мову, а також навести декілька прикладів.
\section{Базові поняття}
%definition
У роботі досліджено спеціальний вид систем диференційних рівнянь - так звані, експоненціально дихотомічні системи.
Ми спробуємо спершу пояснити чим цікаві та корисні такі системи; як маючи дану систему рівнянь відповісти на питання, чи
є вона експоненціально дихотомічною і наведемо декілька (сподіваємось) нетривіальних прикладів. Це буде складати матеріал
відповідно секцій "Вступ"{}, "Результати" і "Приклади". 
Почнемо ж ми з означення.

Експоненціально дихотомічні системи являють собою особливий тип однорідних систем виду
\equation\label{LinHomSysDef}\frac{dx}{dt}=A(t)x,\;x(t)\in\mathbb{R}^n\endequation

де $t\in\mathbb{R},\; A(t)$ - обмежена неперервна матрична функція на $\mathbb{R}$. 

Перш ніж продовжити, варто зробити декілька зауважень стосовно систем такого типу загалом:
%remark all norms are equiv (in particular, operator norms) => no need to mention the norm
\begin{remark}\label{AllNormsAreEqRemark}
	Ми вважаємо відомим і використовуємо без доведення той факт, що всі норми на $\mathbb{R}^n$ є еквівалентними,
	тобто якщо $\mynorm{\cdot}$ і $\mynorm{\cdot}'$ є двома нормами на $\mathbb{R}^n$, то існує пара $m,M>0$, залежна
	лише від $n,\;\mynorm{\cdot}$ та $\mynorm{\cdot}'$ і така, що виконуються нерівності $\forall x\in\mathbb{R}^n,\;m\mynorm{x}\leq
	\mynorm{x}'\leq M\mynorm{x}$. Таким чином, усюди, де ми будемо казати що певний об’єкт лінійної алгебри (матриця, вектор тощо) або ж
	функція, значеннями якої є такі об’єкти є обмеженою, ми не будемо уточнювати, яка норма мається на увазі і в доведеннях будемо 
	використовувати ті норми, які є доречними в даному контексті
	, особливо не промовляючи це. Окремо зауважимо, що у випадку операторів так звані "операторні норми" також
	є нормами, якщо ми використаємо стандартне ототожнення оператора на $\mathbb{R}^n$ з вектором в $\mathbb{R}^{n\times n}$.
\end{remark}
\begin{remark}\label{SolsExistAndUniqRemark}
	Для довільних $t_0\in\mathbb{R},\;x_0\in\mathbb{R}^n$ існує єдиний визначений на 
$\mathbb{R}$ розв'язок $y:\mathbb{R}\mapsto\mathbb{R}^n$, що задовольняє початковим умовам $y(t_0)=x_0$.
Це випливає з теореми Піка, яка стверджує, що функція $F(t,x)$, що набуває значень в $\mathbb{R}^n$, визначена на 
$\Omega=\left\{(t,x) \mid \myabs{t-t_0}\leq a;\;\mynorm{x-x_0}_1:=\max_{1\leq i\leq n}\myabs{x^i-x^i_0}\leq b\right\}$ і відповідає умові Ліпшиця по другому аргументу (тобто 
$\exists M,\;\forall (t,x),\;(t,y)\in\Omega\;\mynorm{F(t,x)-F(t,y)}\leq M\mynorm{x-y}$), то задача Коші $\frac{dx}{dt}=F(t,x),\;x(t_0)=x_0$
має єдиний розв’язок принаймні на інтервалі $\myabs{t-t_0}\leq h:=\min\left\{a;\frac{b}{M}\right\}$. Таким чином, нам лише потрібно показати,
що $F(t,x):=A(t)x$ задовольняє умові Ліпшиця по другому аргументу при даній гіпотезі стосовно $A(t)$. Оскільки $\forall t\in\mathbb{R}$
операторна норма матриці $A(t)$, узгоджена з нормою $\mynorm{\cdot}_2$ на $\mathbb{R}^n$ не перевищує певне $M>0$
(див. попереднє зауваження \ref{AllNormsAreEqRemark}), маємо 
$\forall (t,x),\;(t,y)\in\mathbb{R}\times\mathbb{R}^n,\;\mynorm{F(t,x)-F(t,y)}_2=\mynorm{A(t)(x-y)}\leq M\mynorm{x-y}_2$ (помітимо,
що через еквівалентність норм неважливо, яку конкретно норму ми використовуємо в умові Ліпшиця в даний момент)
\end{remark}
\begin{remark}
	Ми позначатимемо матриціант системи \ref{LinHomSysDef} як $\Omega_a^b(A)$. За означенням це лінійний
	оператор на $\mathbb{R}$ і $\Omega_a^b(A)(x_0):=x(b)$, де $x(t)$ - це розв’язок задачі Коші $\frac{dx}{dt}=
	A(t)x,\;x(a)=x_0$. Через однорідність задачі, матриціант є дійсно лінійним (тому надалі ми
	писатимемо $\Omega_a^b(A)x_0$ замість $\Omega_a^b(A)(x_0)$) і попереднє зауваження \ref{SolsExistAndUniqRemark}
	гарантує, що він завжди визначений однозначно. За означенням, $Omega^t_t(A)=I_n$.
\end{remark}
\begin{remark}
Наведемо декілька властивостей матриціанта, які будуть корисні у подальшому разом із стислими доведеннями:
\begin{itemize}
	\item{Знову ж таки, безпосередньо з означення і єдиності розв’язку маємо $\Omega_b^c(A)\cdot\Omega_a^b(A)=\Omega_a^c(A)$}
	\item{З попередньої властивості $\Omega_a^b(A)\cdot\Omega_b^a(A)=\Omega_b^b(A)=I_n$, тому $\Omega_b^a(A)=\left(\Omega_a^b(A)\right)^{-1}$
		і ми бачимо, що $\Omega_a^b(A)$ є автоморфізмом $\mathbb{R}^n$}
\end{itemize}
\end{remark}

%definition of dichotomy
Таким чином, система виду \ref{LinHomSysDef} називається \textbf{експоненціально дихотомічною} на $A\subset\mathbb{R}$
(або просто "дихотомічною" в подальшому) якщо векторний простір $\mathbb{R}^n$ розкладається
у векторну суму двох підпросторів $E^+\oplus E^-$ і для довільних $x^+\in E^+,\; x^-\in E^-$ і $a,b\in A$ мають місце наступні оцінки:
\begin{equation}\begin{aligned}\label{DichotomyDef}
	\mynorm{\Omega^b_0(A)x^+}\leq K\mynorm{\Omega_0^a(A)x^+}\exp\left\{-\gamma\myabs{a-b}\right\},\;a\leq b\\
	\mynorm{\Omega^b_0(A)x^-}\leq K\mynorm{\Omega_0^a(A)x^-}\exp\left\{-\gamma\myabs{a-b}\right\},\;b\leq a\\
\end{aligned}\end{equation}
для сталих $K,\;\gamma>0$, що не залежать від $x^+,x^-,a,b$

На перший погляд означення здається безнадійно заплутаним, але ми сподіваємось, що наступні пояснення внесуть певне полегшення.
\begin{remark}
	Через причини, пояснені в зауваженні \ref{AllNormsAreEqRemark}, ми не уточнюємо, яка норма використовується в означенні
	\ref{DichotomyDef} і будемо використовувати у доведеннях ту, яка буде доречною.
\end{remark}
%remark - intuitive meaning
\begin{remark}
Немає жодної втрати загальності, якщо ми вважатимемо, що $0\in A$. Дійсно, дане довільне $o\in A$ (нас не цікавить тривіальний
випадок $A=\emptyset$) ми можемо в означенні \ref{DichotomyDef} переписати $\Omega_0^a(A)x^+$ як $\Omega_o^a(A)\Omega_0^o(A)x^+$
Таким чином, якщо ми позначимо $\Omega_0^o(A)x^+=:\tilde{x}^+\in \tilde{E}^+:=\Omega_0^o(A)E^+$. Оскільки $\Omega_0^o(A)$ є автоморфізмом,
ми все ще маємо $\mathbb{R}^n=\tilde{E}^+\oplus \tilde{E}^-$, де $\tilde{E}^-:=\Omega_0^o(A)E^-$. Ми отримаємо, після таких переозначень
\begin{equation*}
	\tag{\ref*{DichotomyDef}$'$}
	\begin{aligned}
	\mynorm{\Omega^b_o(A)\tilde{x}^+}\leq K\mynorm{\Omega_o^a(A)\tilde{x}^+}\exp\left\{-\gamma\myabs{a-b}\right\},\;a\leq b\\
	\mynorm{\Omega^b_o(A)\tilde{x}^-}\leq K\mynorm{\Omega_o^a(A)\tilde{x}^-}\exp\left\{-\gamma\myabs{a-b}\right\},\;b\leq a\\
\end{aligned}\end{equation*}

Таким чином, зсунувши все на $o$ ми можемо вважати $o\in A$ нулем або, що те ж саме, що $0\in A$.

Щоб зрозуміти означення, можна почати зі слабшої версії, тобто зафіксувати $a=0$. Це дасть нам
\begin{equation*}
	\tag{\ref*{DichotomyDef}$''$}
	\begin{aligned}
	\mynorm{\Omega^b_0(A){x}^+}\leq K\mynorm{{x}^+}\exp\left\{-\gamma\myabs{b}\right\},\; b\geq 0\\
	\mynorm{\Omega^b_0(A){x}^-}\leq K\mynorm{{x}^-}\exp\left\{-\gamma\myabs{b}\right\},\;b\leq 0\\
\end{aligned}\end{equation*}

Таким чином, якщо розв’язок починається в $E^+$, він (принаймні)
експоненціально
згасає при $b\to+\infty$ (може, швидше), в той час як розв’язки, які стартують з $E^-$,
(принаймні) експоненціально згасають на $b\to-\infty$ (можливо швидше).

З іншого боку, зафіксувавши $b=0$ отримаємо
\begin{equation*}
	\tag{\ref*{DichotomyDef}$'''$}
	\begin{aligned}
	\frac{1}{K}\exp\left\{\gamma\myabs{a}\right\}\mynorm{{x}^+}\leq \mynorm{\Omega_0^a(A){x}^+},\; a\leq 0\\
	\frac{1}{K}\exp\left\{\gamma\myabs{a}\right\}\mynorm{{x}^+}\leq \mynorm{\Omega_0^a(A){x}^-},\;a\geq 0\\
\end{aligned}\end{equation*}

Тобто розв’язки, що почалися в $E^+$ принаймні 
експоненціально зростають при $a\to-\infty$, ті, що почалися в $E^-$ проявляють аналогічну поведінку при 
$a\to+\infty$. 

Підсумовуючи вищесказане, нерівності \ref{DichotomyDef} надають нам інформацію про асимптотичну поведінку розв’язків, залежно
від їх положення в момент часу $t=0$. Разом із фактом, що увесь фазовий простір $\mathbb{R}^n=E^+\oplus E^-$ і тим, що 
розв’язки \ref{LinHomSysDef} утворюють векторний простір, це дає нам суттєву інформацію про асимптотичну поведінку усіх розв’язків
системи \ref{LinHomSysDef} і пов’язяних з нею, як ми побачимо в подальшому.
\end{remark}
%remark - arb set, but conn und has infty => only R+- are interesting
\begin{remark}
	Хоча означення дихотомії \ref{DichotomyDef} було сформульоване для довільного $A\subset\mathbb{R}$, на практиці лише три випадки є 
	найбільш цікавими: дихотомія на $A=\mathbb{R}^+:=[0,+\infty)$
	, на $A=\mathbb{R}^-:=(-\infty,0]$ і на $A=\mathbb{R}$. Нижче ми спробуємо пояснити, чому це так.

	По-перше, у випадку обмеженого $A$ дихотомія виконується тривіально для кожної системи виду \ref{LinHomSysDef} із обмеженою, неперервною
	$A$. Дійсно, ми покажемо що у випадку $A\subset[-M,M]$ дихотомія виконується для $E^+:=\mathbb{R}^n$ та $E^-:=\emptyset$. Ми маємо
	довести існування певних $K,\;\gamma>0$, таких що
	\[\forall x\in\mathbb{R}^n,\;a,b\in A,\;a\leq b\implies 
	\mynorm{\Omega^b_0(A){x}}\leq K\mynorm{\Omega_0^a(A){x}}\exp\left\{-\gamma\myabs{a-b}\right\}\\
	\]

	Ми покажемо більш сильне твердження:
	\[\tag{*}\label{InterestingARemarkBddDesiredStatement}
	\forall x\in\mathbb{R}^n,\;a,b\in [-M,M],\;
	\mynorm{\Omega^b_0(A){x}}\leq K\mynorm{\Omega_0^a(A){x}}\exp\left\{-\myabs{a-b}\right\}\\
	\]
	
	Оскільки для $x=0$ нерівність очевидно виконується незалежно від $K$ та $\gamma$, у подальшому ми припустимо $x\neq 0$. Більше того,
	оскільки норма лінійна, можна припустити $\mynorm{x}=1$. Ми введемо додатні функції
	\[m(r):=\min_{\mynorm{x}=1}\mynorm{\Omega^r_0(A)x}\]
	\[M(r):=\max_{\mynorm{x}=1}\mynorm{\Omega^r_0(A)x}\]
	
	Оскільки $\Omega_0^r(A)x$ є неперервною по $r$ і $x$, $m(r)$ та $M(r)$ є неперервними на $\mathbb{R}$. Таким чином, оскільки в 
	бажаному твердженні \ref{InterestingARemarkBddDesiredStatement} $\mynorm{\Omega^b_0(A)x}\leq M(b)$ і $\mynorm{\Omega^a_0(A)x}
	\geq m(a)$, твердження \ref{InterestingARemarkBddDesiredStatement} випливатиме з сильнішого твердження
	\[\tag{$\star$}\label{InterestingARemarkStronger}
	\forall x\neq 0,\;a,b\in [-M,M],\;
	M(b)\leq K\cdot m(a)\exp\left\{-\myabs{a-b}\right\}\\
	\]

	Оскільки $M(b)$ та $m(a)$ є неперервними і додатними, функція 
	\[F(a,b):=\frac{M(b)}{m(a)\exp\left\{-\myabs{a-b}\right\}}\]
	визначена на $[-M,M]\times[-M,M]$ є також неперервною, а отже обмежена згори, оскільки її область визначення компактна. Іншими словами,
	існує $K$ таке, що $\forall (a,b)\in[-M,M]\times[-M,M] F(a,b)\leq K$, що еквівалентно твердженню
	\ref{InterestingARemarkStronger}. Це показує, що обмежені $A$ не є цікавими для теорії експоненціальної дихотомії.

	По-друге, $A$ можна вважати закритою множиною. Дійсно, нехай $a_0,b_0\in\overline{A}$ (
	найменша закрита множина, що містить $A$) і система \ref{LinHomSysDef} є дихотомічною на $A$,
	тобто рівняння \ref{DichotomyDef} виконуються для $a,b\in A$. Ми покажемо, що ці рівняння виконуються і для $a_0,b_0$, тому останні можна
	вважати елементами $A$, а саме $A$ - закритим.

	Випадок $a_0=b_0$ тривіальний, адже рівняння \ref{DichotomyDef} очевидно виконуються якщо $K\geq 1$. Але остання умова випливає з дихотомії
	на $A$, адже ми можем взяти $a=b=0\in A$ і просто записати умови дихотомії, які в такому разі тягнуть за собою умову $K\geq 1$. Таким чином,
	можемо припустити $a_0\neq b_0$. Оскільки випадки $a_0<b_0$ та $a_0>b_0$ є симетричними, достатньо розглянути лише перший. Оскільки $a_0,
	b_0\in\overline{A}$, маємо $a_n,b_n\in A,\; a_n\to a_0,\;b_n\to b_0$. Оскільки за припущенням $a_0<b_0$, для великих $n$ маєм
	$a_n<b_n$. Переходячи для підпослідовності, можна вважати, що остання умова виконується для всіх $a_n,\;b_n$. Тому
	\[\forall x^+\in E^+,\;\mynorm{\Omega^{b_n}_0(A)x^+}\leq K\mynorm{\Omega_0^{a_n}(A)x^+}\exp\left\{-\gamma\myabs{a_n-b_n}\right\}\]
	Оскільки для $\Omega_y^z(A)x$ є неперервною в $y,\;z$ та $x$, отримуємо $\Omega^{a_n}(A)x^+\to\Omega^{a_0}(A)x^+$ i
	$\Omega^{b_n}(A)x^+\to\Omega^{b_0}(A)x^+$. Додавши до цього очевидне $\exp\left\{-\gamma\myabs{a_n-b_n}\right\}\to
	\exp\left\{-\gamma\myabs{a_0-b_0}\right\}$, маємо
	\[\forall x^+\in E^+,\;\mynorm{\Omega^{b_0}_0(A)x^+}\leq K\mynorm{\Omega_0^{a_0}(A)x^+}\exp\left\{-\gamma\myabs{a_0-b_0}\right\}\]

	Таким чином, нерівність \ref{DichotomyDef} виконується і для $a_0< b_0$, а тому їх можна вважати елементами $A$.

	По-третє, з суто технічної точки зору, умови експоненціальної дихотомії дають більший ефект у випадку зв’язної множини $A\subset\mathbb{R}$.
	Таким чином, природно зосередитися на випадку $A$ замкненого (див. вище) інтервалу. З поясненого вище, цей інтервал має бути необмеженим,
	якщо нас цікавлять нетривіальні випадки. Таким чином, інтервал має містити нуль (з попереднього зауваження) і окіл хоча б однієї
	з бескінечностей: $+\infty$ або $-\infty$ (можливо, обидва). Ці 3 випадки і приводять до трьох найбільш цікавих значень для $A$, на
	яких ми зосередимося в подальшому: дихотомія на $\mathbb{R}^+$, на $\mathbb{R}^-$ або на $\mathbb{R}$
\end{remark}
\section{Результати}
Варто помітити, що жоден з результатів, зібраних у цьому розділі не являється моїм особистим доведенням. Доведення і формулювання результатів
були взяті з посилань \cite{krein} та \cite{mitrop} літерально і ця робота, на жаль, носить суто реферативний характер.
%R+- iff at least one bdd result for each bdd fluctuation - TODO
\begin{theorem}
	Система виду \ref{LinHomSysDef} є дихотомічною на $\mathbb{R}^+$ (або $\mathbb{R}^-$) тоді і тільки тоді, коли для довільної обмеженої на
	$\mathbb{R}^+$ (або $\mathbb{R}^-$) $f:\mathbb{R}^+\mapsto\mathbb{R}^n$ (або $f:\mathbb{R}^-\mapsto\mathbb{R}^n$) рівняння 
	\equation\label{LinHomPerturbedSysDef}\frac{dx}{dt}=A(t)x+f(x),\;x(t)\in\mathbb{R}^n\endequation
	має принаймні один розв’язок, обмежений на $\mathbb{R}^+$ (або $\mathbb{R}^-$)
\end{theorem}
\begin{proof}
	Взяте з \cite{krein}.
	Ми розглянемо лише випадок дихотомії на $\mathbb{R}^+$, адже інший випадок є симетричним і почнемо з обґрунтування того, що існування
	обмеженого розв’язку при обмеженій $f$ є наслідком дихотомії. 
	
	$(\Longrightarrow)$ Ми доведемо існування цього розв’язку просто сконструювавши його.
	і для цього ми введемо функцію Гріна обмеженого розв’язку у явному вигляді:
	\[
	G(t,\tau)=
	\begin{cases}
			\Omega_0^t(A) P_{E^+} \Omega^0_{\tau}(A), & t\geq\tau,\\
			-\Omega_0^t(A) P_{E^-} \Omega^0_{\tau}(A), & t<\tau,\\
	\end{cases}
	\]
	де $t,\tau\in\mathbb{R}^+$, $P_{E^+}$ та $P_{E^-}$ є операторами проектування на $E^+$ та $E^-$ відповідно. Таким чином,
	нам потрібно продемонструвати той факт, що для довільної $f:\mathbb{R}^+\mapsto\mathbb{R}^n$ неперервної і обмеженої, вираз
	\[\tag{*}\label{XsolutionGreenInt}x(t):=\int_{\mathbb{R}^+} f(\tau)G(t,\tau)d\tau\]
	є обмеженим на $\mathbb{R}^+$ розв’язком рівняння \ref{LinHomPerturbedSysDef}. Зафіксуємо обмежену неперервну 
	вектор-функцію $f(t)$ (і отже $x(t)$, задане формулою \ref{XsolutionGreenInt}
	). По-перше, ми покажемо, що інтеграл \ref{XsolutionGreenInt} в принципі має сенс. Обидві $f(t)$ є неперервною і 
	$G(t,\tau)$ при фіксованому $t$ є кусково-неперервною як функція від $\tau$ з єдиним розривом першого роду
	в $\tau=t$. Дійсно, за неперервністю $\Omega_a^b(A)$ в $a$ і $b$ (що, як було згадано вище випливає з неперервності розв’язків 
	\ref{LinHomSysDef}, яка в свою чергу є наслідком їх диференційованості) маємо
	\[\lim_{\tau\to t-}G(t,\tau)=\lim_{\tau\to t-}(\Omega_0^t(A) P_{E^+} \Omega^0_{\tau}(A))=\Omega_0^t P_{E^+}\Omega^0_t\]
	\[\lim_{\tau\to t+}G(t,\tau)=\lim_{\tau\to t+}(-\Omega_0^t(A) P_{E^-} \Omega^0_{\tau}(A))=-\Omega_0^t P_{E^-}\Omega^0_t\]
	Таким чином, підінтегральний вираз в \ref{XsolutionGreenInt} є кусково-неперервним, тому інтегрування
	має сенс (у цій роботі ми будемо користуватися виключно інтегралом Рімана), за умови збіжності. Збіжність ми доведемо зараз.

	Збіжність \ref{XsolutionGreenInt} в свою чергу випливала б з оцінки
	\[\label{DesiredXsolutionEstim}
	\tag{$\star$}\exists N\mid \forall t,\;\tau\in\mathbb{R}^+,\;\mynorm{G(t,\tau)}\leq Ne^{-\gamma\myabs{t-\tau}}\]
	якби ми спромоглися її встановити, адже вона гарантує експоненційний спад
	. Давайте спробуємо встановити таку оцінку і без втрати загальності обмежимось випадком $t\geq\tau$ (випадок
	$t<\tau$ є симетричним). Для доведення ми виберемо норму операторною 
	і нагадаємо, що для неї виконується $\mynorm{AB}\leq\mynorm{A}\mynorm{
	B}$.
	Для довільного $x\in\mathbb{R}^n$ $P_{E^+}\Omega^0_{\tau}(A)x\in E^+$, тому згідно нерівностей \ref{DichotomyDef}
	маємо
	\[\mynorm{\Omega_0^t(A) P_{E^+} \Omega^0_{\tau}(A)x}\leq Ke^{-\gamma\myabs{t-\tau}}\mynorm{\Omega_0^{\tau}(A) P_{E^+} \Omega^0_{\tau}(A)x}\]
	щоб встановити бажану оцінку \ref{DesiredXsolutionEstim} нам залишається показати, що норма оператора 
	\[\mynorm{\Omega_0^{\tau}(A) P_{E^+} \Omega^0_{\tau}(A)}\]
	рівномірно обмежена в $\tau\in\mathbb{R}^+$. Більше того, обмеженість можна показати лише для великих $\tau$, адже 
	$F:\mathbb{R^+}\ni \tau\mapsto\Omega_0^{\tau}(A) P_{E^+} \Omega^0_{\tau}(A)\in\mathbb{R}^{n\times n}
	$ є неперервною, тому неперервною є і $\mynorm{F(\tau)}$ і
	таким чином, $\forall A>0,\;\max_{\tau\in[0,A]}\mynorm{F(\tau)}<\infty$. 

	Ми введемо позначення $P_{E^+}(\tau):=	\Omega_0^{\tau}(A) P_{E^+} \Omega^0_{\tau}(A)$ і відповідно 
	$P_{E^-}(\tau):=\Omega_0^{\tau}(A) P_{E^-} \Omega^0_{\tau}(A)$. Помітимо, що $P_{E^+}(\tau)+P_{E^-}(\tau)=
	\Omega_0^{\tau}(A) (P_{E^+}+P_{E^-}) \Omega^0_{\tau}(A)=\Omega_0^{\tau}(A)I\Omega^0_{\tau}(A)=I$. 
	%Загалом
	%\[P_{E^+}(\tau)x(\tau)=x(\tau)\iff \Omega_0^{\tau}(A)P_{E^+}x(0)=x(\tau)\iff\]\[\iff P_{E^+}x(0)=x(0)\iff x(0)\in E^+\]
	
	Ми хочемо показати,
	що $\overline{\lim_{\tau\to+\infty}}\mynorm{P_{E^+}(\tau)}<\infty$. Припустимо, щоб досягнути протиріччя, що для
	$\tau_n\to+\infty$ і $\mynorm{x_n(\tau_n)}=1$ виконується $\mynorm{P_{E^+}(\tau_n)x_n(\tau_n)}\to\infty$. Перенормувавши, можем припустити
	$\mynorm{P_{E^+}(\tau_n)x_n(\tau_n)}=1$, а $x_n(\tau_n)\to 0$. 
	Тоді $\mynorm{P_{E^-}(\tau_n)x_n(\tau_n)}=\mynorm{x_n(\tau_n)-P_{E^+}(\tau_n)x_n(\tau_n)}\to 1$. Ми також
	позначимо $P_{E^+}(\tau_n)x_n(\tau_n)=x^+_n(\tau_n)\implies x^+_n(0)\in E^+$ (імплікація слідує
	з властивостей $P_{E^+}(\tau_n)$) і аналогічно для $E^-$.

	Також помітимо, що для 
	\[\exists M\mid\forall m\in\mathbb{Z},\;\mynorm{\Omega_t^{t+m}(A)}\leq e^{mM}\]
	Оскільки для $m>0$
	\[\mynorm{\Omega_t^{t+m}}=\mynorm{\Omega_t^{t+1}(A)\cdot\Omega_{t+1}^{t+2}(A)\cdot\hdots\cdot\Omega_{m-1}^{m}(A)}\leq\]
	\[\leq \mynorm{\Omega_t^{t+1}(A)}\cdot\mynorm{\Omega_{t+1}^{t+2}(A)}\cdot\hdots\cdot\mynorm{\Omega_{m-1}^{m}(A)}\]
	і аналогічно для $m<0$ потрібно лише показати, що $\mynorm{\Omega_t^{t+1}}$ рівномірно обмежено в $t\in A$ (або,
	що те ж саме, що $\log\mynorm{\Omega_t^{t+1}}$ рівномірно обмежено в $t\in A$). Але
	\[\forall t\in A\forall x(t)\neq 0,\; \log\frac{\mynorm\sigma_t^{t+1}x(t)}{x(t)}{\mynorm{x(t)}}=\frac{1}{2}\log\frac{\mynorm{x(t+1)}^2}
	{\mynorm{x(t)}^2}=\frac{1}{2}\log\mynorm{x(y)^2}\mid_t^{t+1}=\]
	\[=\int_t^{t+1}\frac{2 x(y)\cdot\frac{dx}{dy}dy}{\mynorm{x(y)}^2}=\int_t^{t+1}\frac{2 x(y)\cdot A(y)x(y)dy}{\mynorm{x(y)}^2}\leq\]
	\[\leq\int_t^{t+1}\frac{\mynorm{A(y)}\mynorm{x(y)}^2dy}{\mynorm{x(y)}^2}=\int_t^{t+1}\mynorm{A(y)}dy\]
	Останній інтеграл є рівномірно обмеженим в $t\in A$, адже $\mynorm{A(y)}$ є рівномірно обмеженою в $y\in A$. Це показує бажане обмеження
	на $\mynorm{\Omega_t^{t+m}}$.

	З цього (адже для великих $n$ $1/2\leq\mynorm{x_n^+(\tau_n)},\;\mynorm{x_n^-(\tau_n)}\leq 2$ і $x_n^+(0)\in E^+,\;x_n^-(0)\in E^-$) 
	з нерівностей \ref{DichotomyDef} маєм
	для $m\in\mathbb{Z}$.
	\[\mynorm{x_n^+(\tau_n+m)}\leq Ke^{-\gamma m}\mynorm{x_n^+(\tau_n)}\leq 2Ke^{-\gamma m}\]
	\[\mynorm{x_n^-(\tau_n+m)}\geq K^{-1}e^{\gamma m}\mynorm{x_n^-(\tau_n)}\leq \frac{1}{2}K^{-1}e^{\gamma m}\]
	Таким чином,
	\[\mynorm{x_n(\tau_n)}=
	\mynorm{x_n^+(\tau_n)+x_n^-(\tau_n)}\geq e^{-mM}\mynorm{\Omega^{\tau_n+m}_{\tau_n}(A)x_n^+(\tau_n)+
	\Omega^{\tau_n+m}_{\tau_n}(A)x_n^-(\tau_n)}\geq\]\[\geq e^{-mM}(\mynorm{x_n^+(\tau_n+m)}-\mynorm{x_n^-(\tau_n+m)})\geq\]
	\[\geq e^{-mM}(K^{-1}e^{\gamma m}-Ke^{-\gamma_m})\]
	Оскільки остання нерівність правдива для всіх $m\in\mathbb{Z}$, а $\geq e^{-mM}(K^{-1}e^{\gamma m}-Ke^{-\gamma_m})>0$ для великих
	$m$ ми отримуємо протиріччя до припущення $\mynorm{x_n(\tau_n)}\to 0$, що і показує, що $\mynorm{P_{E^+}(\tau)}$ обмежена $\tau$.

	%R- non-trivial similar estimate comment
	Помітимо, що попередні викладки не залежали істотно від того, що $t,\;\tau\in\mathbb{R}^+$ і працювали б і у протилежному випадку $
	A=\mathbb{R}^-$. 

	Наостанок ми маємо показати, що \ref{XsolutionGreenInt} є розв’язком \ref{LinHomPerturbedSysDef}. Нагадаємо, що ми довели збіжність
	інтегралу, причому експоненційна збіжність дозволяє нам брати похідну від підінтегрального виразу. Таким чином
	\begin{gather*}
		\frac{dx}{dt}=\frac{d}{dt}(\int_0^tG(t,\tau)f(\tau)d\tau+\int_t^{\infty}G(t,\tau)f(\tau)d\tau)=\\
		=G(t,t-0)f(t)-G(t,t+0)f(t)+\int_0^t\frac{\partial}{\partial t}G(t,\tau)f(\tau)d\tau+
		\int_t^{\infty}\frac{\partial}{\partial t}G(t,\tau)f(\tau)d\tau=\\
		f(t)+\int_0^{\infty}A(t)G(t,\tau)f(\tau)d\tau=f(t)+A(t)x(t)
	\end{gather*}
	Ми використовували той факт, що
	\[\frac{\partial}{\partial t}G(t,\tau)=\frac{d}{d t}\pm\Omega_0^t(A)\cdot P_{E^{\pm}}\Omega_{\tau}^0(A)=A(t)G(t,\tau)\]
	адже \[\forall x(0)\in\mathbb{R}^n,\;(\frac{d}{dt}\Omega_0^t)x(0)=\lim_{h\to 0}\frac{\Omega_0^{t+h}x(0)-\Omega_0^tx(0)}{h}=\]
	\[\lim_{h\to 0}\frac{x(t+h)-x(t)}{h}=\frac{dx}{dt}(t)=A(t)x(t)=A(t)\Omega_0^tx(0)\implies\]
	\[\implies\frac{d}{dt}\Omega_0^t=A(t)\Omega_0^t\]
	Доведена рівність $\frac{dx}{dt}=f(t)+A(t)x(t)$ показує, що $x(t)$ є розв’язком \ref{LinHomPerturbedSysDef}.

	%TODO: disclaimer
	$(\Longleftarrow)$ Ця частина не буде доводитися у даній роботі. Доведення можна знайти у книзі \cite{krein}.
\end{proof}
%remark - in case of all R exactly one result, given by Green's function
\begin{remark}
	Як описано в \cite{mitrop},
	у випадку дихотомії на $\mathbb{R}$ існування обмеженого розв’язку для кожної обмеженої $f$ також є наслідком дихотомії, проте в цьому 
	випадку обмежений розв’язок існує лише один і задається функцією Гріна обмеженого розв’язку, наведеною в формулюванні теореми. Той факт,
	що ця ж функція Гріна буде давати обмежені розв’язки і у випадку дихотомії на $\mathbb{R}$ випливає з доведених в теоремі
	властивостей функції Гріна.

	Щоб довести єдиність, припустимо, що $x_1(t)$ і $x_2(t)$ є обмеженими розв’язками $\frac{dx}{dt}=A(t)x+f(t)$, тоді
	$x_0:=x_1-x_2\neq 0$ є ненульовим обмеженим розв’язком $\frac{dx}{dt}=A(t)x$. Тоді, оскільки розв’язок однозначно визначається
	початковими умовами, $x_0(0)\neq 0$, тому хоча б одна з проекцій на $E^+$ або ж на $E^-$ буде ненульовою і відповідна компонента
	розв’язку буде експоненціально зростати на $-\infty$ (на $+\infty$) в той час як інша компонента буде прямувати до нуля. В усякому разі,
	це суперечить обмеженості на $\mathbb{R}$ і дає протиріччя, що доводить твердження.
\end{remark}
\section{Приклади}
У цьому розділі зібрано два прості приклади експоненційної дихотомії, що носять ілюстративний характер. Вони були отримані загалом шляхом
спрощення системи \ref{LinHomSysDef} і намагання дати прості необхідні і достатні умови дихотомії у цих випадках.
%1d case
\begin{example}У випадку $n=1$ рівняння
	\[\frac{dx}{dt}=a(t)x(t),\; a(t),x(t)\in\mathbb{R}^1\tag{\#}\label{OneDLinEq}\]
	має експоненціальну дихотомію на $A$ ($A=\mathbb{R}^+$, $A=\mathbb{R}^-$ чи $A=\mathbb{R}$) тоді і лише тоді,
	коли $\int_0^t a(m)dm$ є рівномірно обмеженим у $t\in A$.
\end{example}
\begin{remark}
	Цей приклад є майже очевидним, адже у одновимірному випадку ми можемо просто розв’язати рівняння \ref{OneDLinEq}
	із збуренням, тобто
	\[\frac{dx}{dt}=a(t)x(t)+f(t)\]
	розв’язком, як відомо є $x(t)=e^{\int_0^ta(\tau)d\tau}(\int_0^tf(\tau)e^{-\int_0^{\tau}a(m)dm}d\tau+C)$

	Далі ми застосовуємо наведену теорему.
\end{remark}
\begin{example}У випадку діагоналізованої над $\mathbb{R}^n$ сталої матриці $A\in\mathbb{R}^{n\times n}$ рівняння
	\[\frac{dx}{dt}=A\cdot x(t)\]
	має експоненціальну дихотомію на $A$ ($A=\mathbb{R}^+$, $A=\mathbb{R}^-$ чи $A=\mathbb{R}$) тоді і лише тоді,
	коли $A$ не є сингулярною.
\end{example}
\section{Висновки}
Як вже стверджувалося, дана робота носить більш реферативний характер і автор усвідомлює, що вона не містить жодних нових ідей чи фактів.
Моєю метою в цій роботі було:
\begin{enumerate}
	\item{Вивчити явище дихотомії як підготовку до подальшої роботи (хочеться сподіватися, більш креативної)
		}
	\item{Сформулювати декілька необхідних і достатніх умов для дихотомії в окремих більш простих випадках.}
\end{enumerate}
\begin{thebibliography}{9}
\bibitem{krein}
Ю. Л. Далецкий, М. Г. Крейн
\emph{Устойчивость решений дифференциальных уравнений в банаховом пространстве}.
Видавництво "Наука"{}, Головна редакція фізико-математичної літератури. Москва, 1970.
\bibitem{mitrop}
Митропольський Ю. А., Самойленко А. М., Кулик В. Л.
\emph{Исследование дихотомии линейных систем дифференциальных уравнений с помощью функций Ляпунова}.
АН УССР. Ін-т математики. Київ, "Наукова думка", 1990.
\end{thebibliography}
\end{document}
%FLOW:
	%bounded -> remark -> (=>) -> (<=)stat1 -> (()stat_3_as_remark_to_stat1)
	%examples(read&input_stat_1)
	%others?? (references+links_to_them,formulas_middle,disclaim&ref_in_pf_and_chapter) <- (abstract)&intro&conclusions, titleFont,
	%thms(stat_1; ||stat<==>) 
%stat_1: R+-_dichotomy <==> at_least_one_bdd_solution_for_any_bdd_fluctuation
%stat_2: at_least_one_bdd_solution_for_any_bdd_fluctuation <==> quadr_form
%thm_1.1: quadr_nonsing_form ==> R_dichotomy
%thm_1.2: R_dichotomy ==> quadr_nonsing_form
%stat_3: R_dichotomy => unique_bdd_solution&Green_func
%thm_1.3: R_dichotomy_Green_function_bound_constants
%thm_2.1: R+-_dichotomy ==> Lyapunov_var_change
%thm_2.2: useless??
%thm_2.3: useless??
