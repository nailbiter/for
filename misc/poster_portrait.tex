\documentclass[portrait,final,paperwidth=90cm,paperheight=120cm,fontscale=0.3]{baposter}

\usepackage{tcolorbox}
\usepackage{lipsum}
\usepackage{calc}
\usepackage{graphicx}
\usepackage{amsmath}
\usepackage{amssymb}
\usepackage{relsize}
\usepackage{multirow}
\usepackage{rotating}
\usepackage{bm}
\usepackage{url}
\usepackage{mystyle}
\usepackage{enumerate}
\usepackage{geometry}
\usepackage{setspace}
\usepackage{amsmath,amssymb,bbm,xypic}
\usepackage[all,cmtip]{xy}
\usepackage{amsmath,amssymb,bbm,float,mystyle}
\usepackage[normalem]{ulem}
\usepackage{caption}
\usepackage{subcaption}
\usepackage{setspace}
\usepackage{catchfilebetweentags}
\usepackage{multirow}
\usepackage{bbm}
\usepackage{graphicx}
\usepackage{multicol}
\usepackage{array}

%\usepackage{times}
%\usepackage{helvet}
%\usepackage{bookman}
\usepackage{palatino}


\graphicspath{{images/}{../images/}}
\usetikzlibrary{calc}

\newcommand{\SET}[1]  {\ensuremath{\mathcal{#1}}}
\newcommand{\MAT}[1]  {\ensuremath{\boldsymbol{#1}}}
\newcommand{\VEC}[1]  {\ensuremath{\boldsymbol{#1}}}
\newcommand{\Video}{\SET{V}}
\newcommand{\video}{\VEC{f}}
\newcommand{\track}{x}
\newcommand{\Track}{\SET T}
\newcommand{\LMs}{\SET L}
\newcommand{\lm}{l}
\newcommand{\PosE}{\SET P}
\newcommand{\posE}{\VEC p}
\newcommand{\negE}{\VEC n}
\newcommand{\NegE}{\SET N}
\newcommand{\Occluded}{\SET O}
\newcommand{\occluded}{o}
\newcommand{\same}{\dots\mbox{(similar)}}

%%%%%%%%%%%%%%%%%%%%%%%%%%%%%%%%%%%%%%%%%%%%%%%%%%%%%%%%%%%%%%%%%%%%%%%%%%%%%%%%
%%%% Some math symbols used in the text
%%%%%%%%%%%%%%%%%%%%%%%%%%%%%%%%%%%%%%%%%%%%%%%%%%%%%%%%%%%%%%%%%%%%%%%%%%%%%%%%

%%%%%%%%%%%%%%%%%%%%%%%%%%%%%%%%%%%%%%%%%%%%%%%%%%%%%%%%%%%%%%%%%%%%%%%%%%%%%%%%
% Multicol Settings
%%%%%%%%%%%%%%%%%%%%%%%%%%%%%%%%%%%%%%%%%%%%%%%%%%%%%%%%%%%%%%%%%%%%%%%%%%%%%%%%
\setlength{\columnsep}{1.5em}
\setlength{\columnseprule}{0mm}

%%%%%%%%%%%%%%%%%%%%%%%%%%%%%%%%%%%%%%%%%%%%%%%%%%%%%%%%%%%%%%%%%%%%%%%%%%%%%%%%
% Save space in lists. Use this after the opening of the list
%%%%%%%%%%%%%%%%%%%%%%%%%%%%%%%%%%%%%%%%%%%%%%%%%%%%%%%%%%%%%%%%%%%%%%%%%%%%%%%%
\newcommand{\compresslist}{%
\setlength{\itemsep}{1pt}%
\setlength{\parskip}{0pt}%
\setlength{\parsep}{0pt}%
}
\newcommand{\assign}{:=}
\newcommand{\comma}{{,}}
\newcommand{\nin}{\not\in}
\newcommand{\tmop}[1]{\ensuremath{\operatorname{#1}}}
\newcommand{\tmtextit}[1]{{\itshape{#1}}}
\newcommand{\um}{-}

\newtheorem{theorem}{Theorem}
\newcommand{\sol}{\mathcal{S}\!{\it ol}(\R^{p,q};\lambda,\nu)}
\newcommand{\Hom}{\mbox{\normalfont Hom}}
\newcommand{\Sol}{\mathcal{S}\!{\it ol}}
\newcommand{\Ind}{\mbox{\normalfont Ind}}
\newcommand{\Supp}{\mathcal{S}\!{\it upp}}
\newtheorem{remark}[theorem]{Remark}
\newtheorem{corollary}[theorem]{Corollary}
\newtheorem{fact}{Fact}
%\newtheorem{definition}{Definition}
\theoremstyle{definition}
\newtheorem{definition}{Definition}
\newcommand{\dueto}[1]{\textup{\textbf{(#1) }}}
\newcommand{\tmrsub}[1]{\ensuremath{_{\textrm{#1}}}}
\newcommand{\tmrsup}[1]{\textsuperscript{#1}}
\newcommand{\tmtextbf}[1]{{\bfseries{#1}}}
\newtheorem{proposition}{Proposition}
\newtheorem{problem}{Problem}
\newcommand{\Op}{\mbox{\normalfont Op}}
\newcommand{\Res}{\operatorname{Res}\displaylimits}
\newcommand{\OpR}{\mbox{\it R}}
\renewcommand{\Q}{Q_{p,q}}
\newcommand{\IlambdaGprime}{I(\lambda)\kern-0.3em\mid_{G'}}
\newcommand{\SBO}{\Hom_{G'}\left(\IlambdaGprime,J(\nu) \right)}
\renewcommand{\setminus}{-}
%%%%%%%%%% End TeXmacs macros

\setlength{\parskip}{0.4em}
\setlength{\parindent}{2em}

\newcommand{\even}{2\Z}
\newcommand{\odd}{2\Z+1}
\newcommand{\teven}{\mbox{\textrm{: even}}}
\newcommand{\todd}{\mbox{\textrm{: odd}}}
\newcommand{\tevenText}[1]{\vspace{-3cm}$\begin{array}{l}\nu\teven\\\nu#1\end{array}$}
\newcommand{\toddText}[1]{\vspace{-3cm}$\begin{array}{l}\nu\todd\\\nu#1\end{array}$}
\newcommand{\mm}{\mid\mid}
\newcommand{\bb}{\backslash\backslash}
\renewcommand{\ss}{//}

%%%%%%%%%%%%%%%%%%%%%%%%%%%%%%%%%%%%%%%%%%%%%%%%%%%%%%%%%%%%%%%%%%%%%%%%%%%%%%
%%% Begin of Document
%%%%%%%%%%%%%%%%%%%%%%%%%%%%%%%%%%%%%%%%%%%%%%%%%%%%%%%%%%%%%%%%%%%%%%%%%%%%%%

\begin{document}

%%%%%%%%%%%%%%%%%%%%%%%%%%%%%%%%%%%%%%%%%%%%%%%%%%%%%%%%%%%%%%%%%%%%%%%%%%%%%%
%%% Here starts the poster
%%%---------------------------------------------------------------------------
%%% Format it to your taste with the options
%%%%%%%%%%%%%%%%%%%%%%%%%%%%%%%%%%%%%%%%%%%%%%%%%%%%%%%%%%%%%%%%%%%%%%%%%%%%%%
% Define some colors

%\definecolor{lightblue}{cmyk}{0.83,0.24,0,0.12}
\definecolor{lightblue}{rgb}{0.145,0.6666,1}

% Draw a video
\newlength{\FSZ}
\newcommand{\drawvideo}[3]{% [0 0.25 0.5 0.75 1 1.25 1.5]
   \noindent\pgfmathsetlength{\FSZ}{\linewidth/#2}
   \begin{tikzpicture}[outer sep=0pt,inner sep=0pt,x=\FSZ,y=\FSZ]
   \draw[color=lightblue!50!black] (0,0) node[outer sep=0pt,inner sep=0pt,text width=\linewidth,minimum height=0] (video) {\noindent#3};
   \path [fill=lightblue!50!black,line width=0pt] 
     (video.north west) rectangle ([yshift=\FSZ] video.north east) 
    \foreach \x in {1,2,...,#2} {
      {[rounded corners=0.6] ($(video.north west)+(-0.7,0.8)+(\x,0)$) rectangle +(0.4,-0.6)}
    }
;
   \path [fill=lightblue!50!black,line width=0pt] 
     ([yshift=-1\FSZ] video.south west) rectangle (video.south east) 
    \foreach \x in {1,2,...,#2} {
      {[rounded corners=0.6] ($(video.south west)+(-0.7,-0.2)+(\x,0)$) rectangle +(0.4,-0.6)}
    }
;
   \foreach \x in {1,...,#1} {
     \draw[color=lightblue!50!black] ([xshift=\x\linewidth/#1] video.north west) -- ([xshift=\x\linewidth/#1] video.south west);
   }
   \foreach \x in {0,#1} {
     \draw[color=lightblue!50!black] ([xshift=\x\linewidth/#1,yshift=1\FSZ] video.north west) -- ([xshift=\x\linewidth/#1,yshift=-1\FSZ] video.south west);
   }
   \end{tikzpicture}
}

\hyphenation{resolution occlusions}
%%
\begin{poster}%
  % Poster Options
  {
  % Show grid to help with alignment
  grid=false,
  % Column spacing
  colspacing=1em,
  % Color style
  bgColorOne=white,
  bgColorTwo=white,
  borderColor=lightblue,
  headerColorOne=black,
  headerColorTwo=lightblue,
  headerFontColor=white,
  boxColorOne=white,
  boxColorTwo=lightblue,
  % Format of textbox
  textborder=roundedleft,
  % Format of text header
  eyecatcher=true,
  headerborder=closed,
  headerheight=0.1\textheight,
%  textfont=\sc, An example of changing the text font
  headershape=roundedright,
  headershade=shadelr,
  headerfont=\Large\bf\textsc, %Sans Serif
  textfont={\setlength{\parindent}{1.5em}},
  boxshade=plain,
%  background=shade-tb,
  background=plain,
  linewidth=2pt
  }
  % Eye Catcher
  {\includegraphics[height=5em]{images/graph_occluded}} 
  % Title
  {\bf\textsc{Symmetry Breaking Operators of Indefinite Orthogonal Groups $O(p,q)$}\vspace{0.5em}}
  % Authors
  {\textsc{ Toshiyuki Kobayashi and Alex Leontiev}}
  % University logo
  {% The makebox allows the title to flow into the logo, this is a hack because of the L shaped logo.
    \includegraphics[height=8.0em]{images/logo}
  }

%%%%%%%%%%%%%%%%%%%%%%%%%%%%%%%%%%%%%%%%%%%%%%%%%%%%%%%%%%%%%%%%%%%%%%%%%%%%%%
%%% Now define the boxes that make up the poster
%%%---------------------------------------------------------------------------
%%% Each box has a name and can be placed absolutely or relatively.
%%% The only inconvenience is that you can only specify a relative position 
%%% towards an already declared box. So if you have a box attached to the 
%%% bottom, one to the top and a third one which should be in between, you 
%%% have to specify the top and bottom boxes before you specify the middle 
%%% box.
%%%%%%%%%%%%%%%%%%%%%%%%%%%%%%%%%%%%%%%%%%%%%%%%%%%%%%%%%%%%%%%%%%%%%%%%%%%%%%
    %
    % A coloured circle useful as a bullet with an adjustably strong filling
    \newcommand{\colouredcircle}{%
      \tikz{\useasboundingbox (-0.2em,-0.32em) rectangle(0.2em,0.32em); \draw[draw=black,fill=lightblue,line width=0.03em] (0,0) circle(0.18em);}}

  \headerbox{Setting}{name=setting,column=0,row=0}{
	  \begin{equation*}
  \Q (x) \assign \,^t \! x I_{p, q} x, \; (x \in
  \mathbbm{R}^{p + q}),
	  \end{equation*}
\begin{equation*}
	\mbox{where }I_{p, q} \assign \tmop{diag} (\underbrace{1, \ldots, 1}_p, \underbrace{-
  1, \ldots, - 1}_q).
  \vspace*{-0.8cm}
\end{equation*}
\begin{align*}
	G &\assign O (p +
	1, q + 1)\\&=\left\{ g\in GL_{n+2}\left( \R \right):\;^t\!gI_{p+1,q+1}g=I_{p+1,q+1} \right\}
\end{align*}
where $n:=p+q$.
Define
a maximal parabolic subgroup $P=MAN_{+}$ with
\begin{align*}
   M &\assign \left\{ \left( \begin{array}{ccc}
    \epsilon  &0  &0\\
    0  &A  &0\\
    0  &0  &\epsilon
  \end{array} \right)
  \right\}_{A \in O (p, q),\;
    \epsilon = \pm 1}\kern-1em\simeq O(p,q)\times \Z_2,  \\
  A &\assign \left\{a (t) \assign \left( \begin{array}{ccc}
    \cosh (t)  &0  &\sinh (t)\\
    0  &I_{p + q}  &0\\
    \sinh (t)  &0  &\cosh (t)
  \end{array} \right)\right\}_{ t \in \mathbbm{R}} \\&\simeq \mathbbm{R},\\
  N_+ &\assign\!\left\{\!  \left( \begin{array}{@{}c@{}c@{}c@{}}
    1- \frac{1}{2} \Q (b) , &\,-^t \! (I_{p, q} b),  &\frac{1}{2} \Q (b)\\
    b  &I_{n}&  - b\\
    - \frac{1}{2} \Q (b),  &- \,^t \! (I_{p, q} b),  &1+\frac{1}{2} \Q (b)
  \end{array} \right)\right\}_{b \in \mathbbm{R}^n } \\&\simeq
  \mathbbm{R}^n.
\end{align*}
{\noindent For complex parameter $\lambda\in\C$ we define (unnormalized) spherical degenerate principal series representations of $G$ as}
{\centering
	$\displaystyle
\begin{aligned}
I(\lambda)&:=\Ind_P^G(\C_\lambda)\\
&\simeq \left\{ f\in C^{\infty}(G)\mid f(gma(t)n)=e^{-\lambda t}f(g),\right.\\&\left.\quad\forall(g,ma(t)n)\in G\times P \right\}.
\end{aligned}
$}
We realize $G':=O(p,q+1)$ as the subgroup $G_{e_{p+1}}:=\mysetn{g \in G}{g \cdot e_{p + 1} = e_{p + 1}}$ of $G$. 
Then $G'$ is {\it compatible} with $P$ in the sense that 
$P':=P\cap G'$ is also a maximal parabolic subgroup
with Langlands decomposition $P'=(G'\cap M)A (G'\cap N_+)$,
because $A\subset G'$.
Similarly, we define (unnormalized) spherical degenerate principal series representations $J(\nu):=\Ind_{P'}^{G'}(\C_{\nu})$ of $G'$ for $\nu\in\C$.
 }
 \headerbox{Problem}{name=problem,column=0,below=setting}
 {
	 \begin{problem}
		 \begin{align*}
			 &\SBO = \mbox{\large\bf ?}\\
			 &\quad\uparrow\mbox{{\it symmetry breaking operators} (\textit{SBOs} for short)}
		 \end{align*}
%%		 For any given $(\lambda,\nu)\in\C^2$ find all , that is 
%%		 all members of the space $\SBO$.
	 \end{problem}
	 \begin{problem}
		 Investigate the properties of SBOs (e.g. residue formulae, functional equations, compute their images).
	 \end{problem}
 }
 \headerbox{Notations}{name=notations,column=0,below=problem}
 {
\begin{itemize}
	\item $\mid \mid \mid \assign \{ (\lambda, \nu) \in \mathbbm{C}^2 \mid \nu \in
	- 2\mathbbm{N} \cup (q + 1 + 2\mathbbm{Z}) \}$;
	\item $\backslash\backslash:=\mysetn{(\lambda,\nu)\in\C^2}{\lambda+\nu-n+1\in-2\N}$;
	\item $/ / \assign
	\{ (\lambda, \nu) \in \mathbbm{C}^2 \mid \lambda - \nu \in
	-2\N \}$;
	\item$ \mid\mid:=\C\times\left( 2\N+1 \right);\tilde{C}(s,t)$ as in \cite[(16.3)]{kobayashi2015symmetry};
	\item $A:=\ss\cap\mid\mid\mid, X:=\mm\cap\bb$
	\end{itemize}
 }
 \headerbox{References}{name=references,column=0,below=notations}
 {
    \smaller
    \nocite{kobayashi2015program}
    \nocite{kobayashi2015symmetry}
    \nocite{kobayashi2016differential1}
\bibliographystyle{mystyle}
\bibliography{todai_master}
   \vspace{0.3em}
 }
 \headerbox{Key Fact}{name=keyfact,column=1,row=0}
 {
Applying the very general result proven in \cite[Chap.\ 3]{kobayashi2015symmetry} to our particular setting, we get the following:
\begin{fact}[{\cite[Thm. 3.16]{kobayashi2015symmetry}}]\label{fact1}
Let $n:=p+q$. The following diagram commutes:
\vspace{-0.8cm}
\begin{figure}[H]
	\xymatrixcolsep{0.0pc}
	\hspace*{-1.3cm}\xymatrix{
&2^{P'\backslash G/P}\\
		\SBO\ar[r]^{\simeq} \ar[ur]^{\Supp}
	&\left( \mathcal{D}'(G/P,\mathcal{L}_{n-\lambda}) \otimes\mathbb{C}_\nu \right)^{P'}
\ar[u]_-{F\mapsto \supp(F)}\ar[dl]^{\simeq}_{\mbox{rest}}\\
{\hspace{1.65cm}\sol\subset\mathcal{D}'(\R^{p,q})}\ar[u]^{\mbox{Op}}_{\simeq}\\
}
\end{figure}
\end{fact}
 }
 \headerbox{Result \#1}{name=suppclassif,column=1,below=keyfact}
 {
\begin{theorem}[classification of closed $P'$-invariant subsets of $G/P$]
	Suppose $p,q\ge1$.
	The left $P'$-invariant closed subsets of $G/P$ are described in the following Hasse diagram. Here 
	$
	\begin{array}{l}
	        \xymatrixrowsep{0.5pc}
		\xymatrix{A\ar@{-}[d]^m\\B}
	\end{array}
	$
	means that $A\supset B$ and that the generic part of $B$ is of codimension $m$ in $A$.
	\vspace*{-0.5cm}
  \begin{figure}[H]
    \centering
    \begin{subfigure}[t]{0.3\textwidth}
	    \xymatrixrowsep{0.5pc}
	    \xymatrix{&X\ar@{-}[ld]_1\ar@{-}[rd]^1&\\Y\ar@{-}[rd]_1&&C\ar@{-}[ld]^1\\&C\cap Y\ar@{-}[dd]^{p+q-2}&\\&&\\&\{[o]\}&}
	\caption{when $p>1$}
    \end{subfigure}
    ~ %add desired spacing between images, e. g. ~, \quad, \qquad, \hfill etc. 
      %(or a blank line to force the subfigure onto a new line)
    \begin{subfigure}[t]{0.3\textwidth}
	    \xymatrixrowsep{0.5pc}
	    {\xymatrix{&X\ar@{-}[ld]_1\ar@{-}[rd]^1&\\Y\ar@{-}[rddd]_{p+q-2}&&C\ar@{-}[lddd]^{p+q-2}\\&&\\&&\\&\{[o]\}&}}
	\caption{when $p=1$}
    \end{subfigure}
\end{figure}
\end{theorem}
 }
 \headerbox{Result \#2}{name=construction,column=1,below=suppclassif}
 {
\begin{theorem}[construction of SBO]\label{thm:construction}
	For $S=X,Y,C,$ and $\left\{ o \right\}$, the following operators $R_{\lambda,\nu}^S$ and $\tilde{R}_{\lambda,\nu}^X$ are SBOs, which depend holomorphically on $(\lambda,\nu)\in D_S$. 
	
	Moreover, $\Supp(R_{\lambda,\nu}^S)\subset S$, ``='' generically and were computed explicitly.\\
	\newcommand{\mystack}[2]{\begin{array}{@{}c@{}}#1\\#2\end{array}}
	\begin{tabular}{@{}|@{}b{1.6cm}@{}|@{}l@{}|}
  \hline
  $R_{\lambda,\nu}^S$& $\tmop{Op} : 
  \Sol(\mathbbm{R}^{p, q} ; \lambda, \nu)
  \rightarrow \tmop{Hom}_{G'} (I (\lambda), J (\nu))$\\
  \hline
  $\mystack{\tilde{R}^X_{\lambda, \nu} =}{(\lambda,\nu)\in\mid\mid\mid}$ & $\frac{1}{\Gamma \left( \frac{\lambda + \nu - n + 1}{2}
  \right) \Gamma \left( \frac{1 - \nu}{2} \right)}{\tmop{Op} \left( | x_p |^{\lambda +
  \nu - n} | \Q |^{- \nu}\right)} $ \\
  \hline
  $\mystack{R_{\lambda, \nu}^Y =}{(\lambda,\nu)\in\bb}$ & ${ q_Y^X (\lambda, \nu)}{\tmop{Op} \left( \delta^{(2k)}(x_p)
  | \Q |^{- \nu}  \right)}$.\\
  \hline
  $\mystack{R_{\lambda, \nu}^C =}{(\lambda,\nu)\in\mm}$ & $q_C^X (\lambda, \nu){\tmop{Op} \left( | x_p |^{\lambda + \nu - n}\delta^{(2m)}\left( \Q \right)
    \right)}$ \\
  \hline
  $\mystack{R_{\lambda, \nu}^{\{ o \}} =}{(\lambda,\nu)\in\ss}$ & 
  $\tmop{Op} \left( \tilde{C}_{\nu -
  \lambda}^{\lambda - \frac{n - 1}{2}} \left(-\Delta_{\mathbbm{R}^{p - 1, q}}
  \delta_{\mathbbm{R}^{p + q - 1}}, \delta (x_p)\right) \right)$\\
  \hline
\end{tabular}
\end{theorem}
%We set $m:=\frac{1}{2}\left( \nu-1 \right)\in\N$ for $(\lambda,\nu)\in\mm$ and $k:=\frac{1}{2}\left( n-1-\lambda-\nu \right)\in\N$ for $(\lambda,\nu)\in\bb$.
For $p=1$ we define $q_C^X(\lambda,\nu)$ and $q_Y^X(\lambda,\nu)$ by
\[ q_C^X (\lambda, \nu) : = \left\{ \begin{array}{ll}
		\Gamma^{-1} \left(\frac{\lambda-\nu}{2} \right), & q \in
     2\mathbbm{Z}, \nu \leqslant q - \nu,\\
     \Gamma^{-1} \left( \frac{\lambda + \nu-n+1}{2} \right), & q \in 2\mathbbm{Z},
     \nu > q - \nu,\\
     \Gamma^{-1} \left( \frac{\lambda - \nu}{2} \right), & q \in
     2\mathbbm{Z}+ 1.
   \end{array} \right. \]
\[q_Y^X (\lambda, \nu) : = \same\]
}

\headerbox{Result \#3 (classif. of SBOs)}{name=classif,column=1,below=construction}
{
\begin{theorem}\label{thm:classif}
	Suppose $p,q\ge1$, let $H:=\SBO$.
\[ p = 1 \Rightarrow H = \left\{
	\begin{array}{@{}l@{}l@{}}
     \mathbbm{C}R^X_{\lambda, \nu}, & (\lambda, \nu) \notin  A\cup X
     ,\\
     \mathbbm{C} \tilde{R}^X_{\lambda, \nu} \oplus \mathbbm{C}R^{\{ o
     \}}_{\lambda, \nu}, & (\lambda, \nu) \in A -
     X,\\
     \mathbbm{C}R^P_{\lambda, \nu} \oplus \mathbbm{c}r^c_{\lambda, \nu}, &
     (\lambda, \nu) \in X - / /,\\
     \mathbbm{C}R^{\{ o \}}_{\lambda, \nu}, & (\lambda, \nu) \in \mid \mid
     \cap \backslash\backslash \cap / /.
   \end{array} \right. \]
\[ p > 1 \Rightarrow H = \left\{
   \begin{array}{ll}
     \mathbbm{C} \tilde{R}^X_{\lambda, \nu} \oplus \mathbbm{C}R^{\{ o
     \}}_{\lambda, \nu \lambda, \nu}, & (\lambda, \nu) \in / / \cap \mid \mid
     \mid,\\
     \mathbbm{C}R^X_{\lambda, \nu}, & \tmop{otherwise.}
   \end{array} \right. \]
\end{theorem}
%%\begin{corollary}\label{cor:classif}
%%	We have $\dim_{\C}\SBO\in\left\{ 1,2 \right\}$ for all $(\lambda,\nu)\in\C^2$.
%%\end{corollary}
}
\headerbox{Result \#4 }{name=spherical,column=2,row=0}
{
\begin{theorem}[spectrum for spherical vectors]\label{thm:spherical}
	Let $\mathbbm{1}_\lambda\in I(\lambda)^K,\mathbbm{1}_\nu^{K'}$ be the spherical vectors normalized so that $\mathbbm{1}_\lambda(e)=\mathbbm{1}_\nu(e)=1$.
	With $n:=p+q\;(p,q\ge1)$ we have:
\[ \OpR^X_{\lambda, \nu} \mathbbm{1}_{\lambda} =  \frac{2^{1 -
\lambda}\pi^{n / 2}}{\Gamma \left( \frac{\lambda}{2} \right)
\Gamma \left(  \frac{\lambda + 1-q}{2} \right) \Gamma \left(
\frac{q - \nu + 1}{2} \right)} \mathbbm{1}_{\nu}. \]
\end{theorem}
}
\headerbox{Result \#5 (residue formula)}{name=residue,column=2,below=spherical}
{
\begin{theorem}
	Let $n:=p+q\;(p,q\ge1)$ as before.
	For $(\lambda,\nu)\in//$, we set $l:=\frac{1}{2}\left( \nu-\lambda \right)\in\N$. Then we have
  \[\OpR_{\lambda,\nu}^X  = \frac{ (- 1)^l l!\pi^{(n - 2) / 2} 
		}{2^{ \nu + 2 l-1}}\cdot  \frac{\sin\left( \frac{1+q-\nu}{2}\pi \right)}{\Gamma\left( \frac{\nu}{2} \right)}
     \OpR_{\lambda,\nu}^{ \left\{ o \right\} },\;(\lambda,\nu)\in//. \]
	\end{theorem}
}
\headerbox{Result \#6}{name=functional,column=2,below=residue}
{
	\begin{definition}
		Similarly to the construction of Fact \ref{fact1}, for $G=O(p+1,q+1)$ we have $\Hom_G(I(\lambda),I(\nu))\simeq\Sol_G(\R^{p,q};\lambda,\nu)$
		where $\Sol_G(\R^{p,q};\lambda,\nu)\subset\mathcal{D}'(\R^{p+q})$.

		Now, the generalized function defined as
		\begin{equation*}
			\myabs{\Q}^{\lambda-n}\times\begin{cases}
				\Gamma^{-1}\left( \lambda-n/2 \right),&\min\left\{ p,q \right\}=0,\\
				\same,&\dots
			\end{cases}
		\end{equation*}
		belongs to $\Sol_G(\R^{p,q};\lambda,n-\lambda)$ and we can use it to
		define an intertwining operator of $G=O(p+1,q+1)$,
		$\tilde{\mathbb{T}}^{G}_{\lambda}:I(\lambda)\to
		I(n-\lambda)$
		(\textit{Knapp--Stein operator}).
		The result of this construction repeated with $G'=O(p,q+1)$ in place of $G$ will be denoted by $\tilde{\mathbb{T}}^{G'}_\nu:J(\nu)\to J(n-1-\nu)$.
	\end{definition}
	\begin{theorem}[functional identities]
		Let $n:=p+q\;(p,q\ge1)$ as before.
		We have:
\begin{eqnarray}
    & \tilde{\mathbbm{T}}^{G'}_{n-1 - \nu} \circ R^X_{\lambda, n' - \nu} =\pi^{\frac{n - 3}{2}} q^{T X}_X
  (\lambda, \nu) R^X_{\lambda, \nu}, &  \nonumber\\
  & R_{n - \lambda, \nu}^X \circ \tilde{\mathbbm{T}}^G_{\lambda} =2^{2\lambda-n}\pi^{-\frac{n}{2}-1} q^{X T}_X
  (\lambda, \nu) R_{\lambda, \nu}^X, &  \nonumber
  \end{eqnarray}
  where
  \begin{gather*}
  q^{T X}_X (\lambda, \nu) \assign\frac{\sin\left( \frac{p-\nu}{2} \pi\right)}{\Gamma\left( \frac{n-1-\nu}{2} \right)} \ \left\{
	  \begin{array}{@{}l@{}l}
    {\Gamma\left( \frac{n / 2 - \nu}{2} \right)}, & \frac{n - 1}{2}+p\todd,\\
    \dots&
  \end{array} \right.\\
  q^{X T}_X (\lambda, \nu) \assign\same
\end{gather*}
	\end{theorem}
}
\headerbox{Result \#7 (images of SBOs)}{name=images,column=2,below=functional}
{
	For every $(\lambda,\nu)\in\C^2$ we can determine image of every SBO constructed in Theorem \ref{thm:construction}.
}
\headerbox{test}{name=test,column=2,below=images}
{
	\begin{tcolorbox}[colback=green!5,colframe=green!40!black,title=A nice heading]
		\lipsum[2]
	\end{tcolorbox}
}

\end{poster}

\end{document}
%put the contents
%decide order
%expand/compress/rephrase
%add colors?
