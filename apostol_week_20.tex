\documentclass[10pt]{article} % use larger type; default would be 10pt
\usepackage[utf8]{inputenc}       % кодування документа; замість cp866nav
\usepackage[margin=0.7in]{geometry}
\usepackage[russian,english]{babel} % національна локалізація; може бути декілька
\usepackage{setspace}
\usepackage{CJKutf8}
\usepackage{mdframed}
\title{Divine Liturgy\\Reading from Epistles}
\date{Week 
20
 after the Pentecost\vspace{-3ex}}
\begin{document}
\pagenumbering{gobble}
\begin{otherlanguage*}{russian}
\maketitle
\end{otherlanguage*}
\large
\onehalfspacing
\framebox[\textwidth]{
\begin{minipage}[t]{0.45\textwidth}
\begin{otherlanguage*}{russian}
\textbf{Гал., 200 зач., I, 11-19.}\\
Возвещаю вам, братия, что Евангелие, которое я благовествовал, не есть человеческое,\\
ибо и я принял его и научился не от человека, но через откровение Иисуса Христа.\\
Вы слышали о моем прежнем образе жизни в Иудействе, что я жестоко гнал Церковь Божию, и опустошал ее,\\
и преуспевал в Иудействе более многих сверстников в роде моем, будучи неумеренным ревнителем отеческих моих преданий.\\
Когда же Бог, избравший меня от утробы матери моей и призвавший благодатью Своею, благоволил\\
открыть во мне Сына Своего, чтобы я благовествовал Его язычникам,- я не стал тогда же советоваться с плотью и кровью,\\
и не пошел в Иерусалим к предшествовавшим мне Апостолам, а пошел в Аравию, и опять возвратился в Дамаск.\\
Потом, спустя три года, ходил я в Иерусалим видеться с Петром и пробыл у него дней пятнадцать.\\
Другого же из Апостолов я не видел никого, кроме Иакова, брата Господня. \\
\end{otherlanguage*}
\end{minipage}
\hfill
\begin{minipage}[t]{0.45\textwidth}

\textbf{Galatians 1:11 -- 1:19.}\\
But I certify you, brethren, that the gospel which was preached of me is not after man.\\
For I neither received it of man, neither was I taught it, but by the revelation of Jesus Christ.\\
For ye have heard of my conversation in time past in the Jews' religion, how that beyond measure I persecuted the church of God, and wasted it:\\
And profited in the Jews' religion above many my equals in mine own nation, being more exceedingly zealous of the traditions of my fathers.\\
But when it pleased God, who separated me from my mother's womb, and called me by his grace,\\
To reveal his Son in me, that I might preach him among the heathen; immediately I conferred not with flesh and blood:\\
Neither went I up to Jerusalem to them which were apostles before me; but I went into Arabia, and returned again unto Damascus.\\
Then after three years I went up to Jerusalem to see Peter, and abode with him fifteen days.\\
But other of the apostles saw I none, save James the Lord's brother.\\
\end{minipage}}
\newpage\huge
\onehalfspacing
\framebox[\textwidth]{
\begin{minipage}[t]{\textwidth}
\begin{CJK}{UTF8}{bsmi}
\textbf{加拉太書 1:11 -- 1:19.}\\
弟兄們,我要你們知道,我所傳的福音不是按照人的意思;\\
因為我不是從人領受的,也不是人教導我的,而是藉着耶穌基督的啟示而來。\\
你們聽說過從前我在 猶太教中的行徑,我怎樣竭力壓迫殘害上帝的教會。\\
在 猶太教中,我比本國許多同輩的人更激進,為我祖宗的傳統更熱心。\\
然而,那位把我從母腹裏分別出來、又施恩呼召我的上帝(註),既然樂意\\
把他兒子啟示在我心裏,讓我在外邦人中傳揚他,我就沒有跟有血有肉的人商量,\\
也沒有上 耶路撒冷去見那些比我先作使徒的,惟獨到 阿拉伯去,後來又回到 大馬士革。\\
過了三年,我才上 耶路撒冷去見 磯法,和他同住了十五天。\\
至於別的使徒,除了主的兄弟 雅各,我都沒有見過。 \\
\end{CJK}
\end{minipage}}
\end{document}
