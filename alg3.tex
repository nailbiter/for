\documentclass[8pt]{article} % use larger type; default would be 10pt

%\usepackage[utf8]{inputenc} % set input encoding (not needed with XeLaTeX)
%\usepackage{CJK}
\usepackage[margin=1in]{geometry}
\usepackage{graphicx}
\usepackage{float}
\usepackage{subfig}
\usepackage{amsmath}
\usepackage{amsfonts}
\usepackage{hyperref}
\usepackage{enumerate}
\usepackage{enumitem}

\usepackage{mystyle}

\title{Homework 2, Math 5111}
\author{Alex Leontiev, 1155040702, CUHK}
\begin{document}
\maketitle
\begin{enumerate}[label=\bfseries Problem \arabic*.]
	\item{
		\begin{enumerate}[label=(\arabic*).]
			\item{Note that $(\sigma(\alpha)\alpha^{-1})^n=\frac{\sigma(\alpha^n)}{\alpha^n}=\frac{\sigma(a)}{a}$, as $\alpha^n=a$. Now, as
				$a\in F$, $\sigma\in Gal(E/F)$, $\sigma$ fixes elements of $F$, in particular $a$. Hence,
				$(\sigma(\alpha)\alpha^{-1})^n=\frac{\sigma(a)}{a}=\frac{a}{a}=1$ and thus $\sigma(\alpha)\alpha^{-1}\in\mu_n$, as it is
				the root of equation $a^n-1=0$ over $E$, but this equation splits in $F$ (as it has there $n$ distinct roots by assumption and it
				cannot have more than $n$ roots in any extension field), hence $\sigma(\alpha)\alpha^{-1}\in F$ and hence in $\mu_n$.

				Now, if $\beta$ is another root of $x^n-a$, we should have $\beta=\mu\alpha,\;\mu\in\mu_n$, as multiplying $\alpha$ with all elements
				of $\mu_n$ shall give us $n$ distinct elements (as $\myabs{\mu_n}=n$ and $a\neq 0\implies\alpha\neq 0$) all of which will satisfy
				$x^n-a$ (because $(\mu\alpha)^n=\mu^n\alpha^n=1\cdot a=a$, if $\mu\in\mu_n$) and latter equation can have no more than $n$ roots.
				Therefore, $\beta=\mu\alpha$ and hence $\frac{\beta}{\alpha}=\mu\in F$ and as $\sigma$ fixes $F$, $\sigma\left(\frac{\beta}{\alpha}
				\right)=\frac{\beta}{\alpha}\implies \sigma(\alpha)\alpha^{-1}=\sigma(\beta)\beta^{-1}$ as required.
				}
			\item{Although as computations done in a previous item imply, this definition of $\Phi$ will not depend on a particular $\alpha$ root
				of $x^n-a=0$ in $E$, for convenience we shall assume it fixed from now on. Now, given $\tau,\;\sigma\in Gal(E/F)$ arbitrary,
				not that as $\sigma(\alpha)/\alpha\in\mu_n\subset F$, we should have $\tau(\sigma(\alpha)/\alpha)=\sigma(\alpha)/\alpha$ and hence
				$\tau(\sigma(\alpha))=\tau(\alpha)\sigma(\alpha)/\alpha$ and thus $(\tau\circ\sigma)(\alpha)/\alpha=\tau(\alpha)/\alpha\cdot
				\sigma(\alpha)/\alpha$ which can be written as $\Phi(\tau\circ\sigma)=\Phi(\tau)\cdot\Phi(\sigma)$. This proves that $\Phi$ is indeed
				a group homomorphism.

				Now, if $\Phi(\sigma)=\sigma(\alpha)/\alpha=1$
				for $\sigma\in Gal(E/F)$, this means that $\forall\alpha,\;\alpha^n=a\implies\sigma(\alpha)=\alpha$, thus $\sigma$ fixes
				zeros of $x^n-a=0$. As splitting field $E$ is generated by zeros of $x^n-a$, the fact that $\sigma$ fixes these zeros as well as $F$
				implies that $\sigma$ fixes $E$, that is $\sigma=id_E$ and this shows injectiveness.
				}
			\item{As $\mbox{char }F=0$, $E/F$ is a separable extension. As $E$ is a splitting field over $F$ (of $x^n-a=0$), it is also normal extension,
				thus it is Galois and hence $[E:F]=\myabs{G}$. Now, as there is an injective mapping $\Phi:G\mapsto \mu_n$ we have validated
				in the previous item, $\myabs{G}$ is the divisor of $\myabs{\mu_n}=n$ and hence $[E:F]$ divides $n$ as well, as $\myabs{G}=[E:F]$.
				}
		\end{enumerate}
		}
	\item{
		\begin{enumerate}[label=(\arabic*).]
			\item{
				}
		\end{enumerate}
		}
	\item{
		\begin{enumerate}[label=(\arabic*).]
			\newcommand{\E}{\mathbb{C}((x^{\frac{1}{n}}))}
			\newcommand{\F}{\mathbb{C}((x))}
			\item{As $\mbox{char }\mathbb{C}((x))=\mbox{char }\mathbb{C}=0$, this extension is separable. As it is a splitting field of $t^n-x=0$ over
				$\mathbb{C}((x))$, it is normal. The last claim is true, as $t^n-x$ splits in $\mathbb{C}((x^{\frac{1}{n}}))$, with roots
				being the products $x^{\frac{1}{n}}\cdot\mu$ for all possible $\mu\in\mu_n:=\mysetn{\mu\in\mathbb{C}}{\mu^n=1}$. As
				$\myabs{\mu_n}=n$, there are $n$ such distinct products and they exhaust zero set of $t^n-x=0$, as latter equation cannot
				have more than $n$ roots. To finish the verification of a claim that $\mathbb{C}((x^{\frac{1}{n}}))$ is indeed a splitting field
				of $t^n-x=0$ over $\mathbb{C}((x))$, it is sufficient to show that $\E$ is generated over $\F$ by $x^{\frac{1}{n}}$.
				Now, given arbitrary $\sum_{i=-m}^{\infty}a_nx^{\frac{i}{n}}\in\E$, part $\sum_{i=-m}^0a_nx^{\frac{i}{n}}$ is a finite sum
				of products of elements of $\F$ with $x^{-\frac{1}{n}}$ and hence belongs to $\F(x^{\frac{1}{n}})$. Now, the remaining
				part $\sum_{i=0}^\infty a_nx^{\frac{i}{n}}$ can be realized as a finite sum
				$\sum_{k=0}^{n-1}\left((x^{\frac{1}{n}})^k\sum_{i=0}^\infty a_{in+k}x^i\right)$ and as every addend belongs to
				$\F(x^{\frac{1}{n}})$, the whole $\sum_{i=-m}^{\infty}a_nx^{\frac{i}{n}}$ belongs to $\F(x^{\frac{1}{n}})$ and as it was
				arbitrary, $E\subset\F(x^{\frac{1}{n}})$ which ends the proof of a claim.

				Thus $\E/\F$ is normal and separable, and hence Galois. 
				}
			\item{To begin with, note that upon assigning $E:=\E$ and $F:=\F$ we find ourselves completely within
				a setting of Problem 1. Applying it's results, we see that $G:=Gal(\E/\F)$ can be thought
				as a subgroup of a cyclic group $\mu_n$ of $n$-th roots of unity, and thus $G$ is cyclic. Therefore,
				it is completely determined by knowing its order and below we attempt to show that it's order is $n$,
				so it is cyclic of order $n$.

				In order to show that $\myabs{G}=n$, it is sufficient to explicitly exhibit $n$ different
				automorphisms of $\E$ that fix $\F$. We claim that $\sigma_i$ defined as $\sigma_i\mid_\mathbb{C}
				=id_\mathbb{C}$
				and $\sigma_i(x^{\frac{1}{n}})=g^ix^{\frac{1}{n}}$ (where $g$ is an arbitrary generator of $\mu_n$)
				for $1\leq i\leq n$ will do. Now, each two of them are distinct, as they all map $x^{\frac{1}{n}}$
				to distinct values. Besides, each of them is an automorphism of $\E$, as $\sigma_i(\E)\subset\E$
				and $\sigma_i$ is well-behaved under multiplication and addition and maps $1$ to $1$. Finally,
				each $\sigma_i$ fixes $\F$, as $\sigma_i$ fixes $\mathbb{C}$ and $\sigma_i(x)=(\sigma(x^{\frac{1}{n}})
				^n)=g^n(x^{\frac{1}{n}})^n=1\cdot x=x$.
				}
			\item{
				}
		\end{enumerate}
		}
\end{enumerate}
\begin{thebibliography}{9}
	\bibitem{tb} Robert B. Ash, {\em Abstract Algebra: The basic Graduate Year}, digital version provided by author can be found at
		\url{http://www.math.uiuc.edu/~r-ash/Algebra.html}
\end{thebibliography}
\end{document}
