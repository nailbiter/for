%stoic ethic
\documentclass[14pt]{extarticle} % use larger type; default would be 10pt
\usepackage{fontspec}
\usepackage{array, xcolor, lipsum, bibentry}
\usepackage[margin=3cm]{geometry}
\usepackage{hyperref}
\usepackage{fancyhdr}

 
\title{\bfseries\Huge Oleksii Leontiev}
\author{inp9822058@cs.nctu.edu.tw}
\date{}
 
\definecolor{lightgray}{gray}{0.8}
\newcolumntype{L}{>{\raggedleft}p{0.2\textwidth}}
\newcolumntype{R}{p{0.8\textwidth}}
\newcommand\VRule{\color{lightgray}\vrule width 0.5pt}
 
%font configuration
\defaultfontfeatures{Mapping=tex-text}
\setromanfont[Ligatures={Common}, Numbers={OldStyle}, Variant=01]{Times New Roman} % Main text font
%%\setromanfont[Ligatures={Common}, Numbers={OldStyle}, Variant=01]{Linux Biolinum Slanted} % Main text font
\chardef\&="E050 % Custom ampersand character

\begin{document}
\begin{titlepage}
	\addtolength{\voffset}{-2cm}
	%\setlength{\footskip}{5.5cm}
	\thispagestyle{fancy}
	\fancyfoot[C]{м. Київ -- 2013}
	\begin{center}
		\newcommand{\HRule}{\rule{\linewidth}{0.5mm}}
		\textsc{\Large Київський Національний Університет імені Тараса Шевченка}\\[1.5cm]

		% Title
		\HRule \\[0.4cm]
		{ \huge \bfseries Етика стоїцизму}\\[0.4cm]

		\HRule \\[1.5cm]

		% Author 
			\begin{flushright} \large
				Робота студента\\
				4-го курсу\\
				механіко-математичного факультету\\
				заочної форми навчання\\
				\textsc{Леонтьєва} Олексія Костянтиновича
			\end{flushright}

		\vfill

		% Bottom of the page
		{\large 19 червня 2013 р.}
	\end{center}
\end{titlepage}
\section{Вступ}%1 page (1) - goal: 6 pages
Стоїцизм (грец. sto -- портик (галерея з колонами в Афінах, де Зенон, засновник Стої, навчав філософії)) — філософське вчення, згідно з
яким світ-космос перебуває в нескінченній пустоті, будучи живим сферичним тілом, розумною істотою, що організовує всі свої частини в
доцільно упорядковане ціле. Долю окремого тіла таким чином, за стоїцизмом, визначає його природа, що доцільно включається у всезагальну природу
. Найважливішим своїм завданням стоїцизм вважав обґрунтування міцної і розумної основи морального життя людини, яку вбачав у
подоланні пристрастей, силі духу, що виявляється у підпорядкуванні своїй долі. Основними чеснотами стоїка було проголошено стійкість, 
твердість у житті.

Стоїки вважали логіку, фізику і етику частинами філософії. Відоме їх порівняння філософії з фруктовим садом, де логіка — садова огорожа, фізика 
— фруктове (фруктові) дерево(а), а етика — плоди дерева, тобто результат, що базується на певних (зумовлено-визначених) принципах
і обмежений певними рамками.

Ця школа запозичувала певні ідеї в епікурейців і кініків, але мала свої особливості. Подальший розлад суспільних відносин, загроза розпаду Римської 
імперії висунули перед філософами завдання створення більш жорстких норм морально-етичного виховання громадян у суспільстві. Замість теорії 
«вільного поводження», досягнення повсякденної насолоди і необмеженого блага, потрібно було розробити основи раціональної етики, побудованої
на принципах дотримання розумних потреб. Звичайно, що така теорія має велику цінність у сьогоденних умовах.

Замість колективних форм відповідальності людей має місце індивідуалізація людини, піднесення її відповідальності за свої дії. Проповідується
фаталізм, віра в людську долю, трагічне стає героїчним. Замість альтруїзму проповідується  егоїзм, егоцентризм і аскетизм. У світі
панує невблаганна необхідність (фаталізм), вчать стоїки, і немає можливості протистояти їй, людина цілком залежить від усього, що діється
у зовнішньому світі, природі. І мудрець, і невіглас підкоряються необхідності, але «мудрого необхідність веде, дурного ж — волочить».
Мудрість дозволяє стримувати афекти (чуттєві пориви), але для цього, згідно з ученням стоїків, слід виробити в собі чотири чесноти: 
розсудливість, невибагливість, невблаганність, мужність і таким чином можна вироби­ти ідеальний спосіб ставлення до світу — апатію (відсутність
переживань, безпристрасність, загальне блаженство).
\section{Основні постулати етики стоїків}
Етика стоїків базувалася на дещо змінених і пом’якшених доктринах кінізму (таке пом’якшення відбувалося в трьох основних напрямках – по-перше, людина не має повністю придушувати будь-які емоції, а лише вміти їх контролювати; по-друге, серед «нейтральних» (не злих і не добрих) об’єктів зовнішнього світу стоїки почали виокремлювати ті, яким варто надавати перевагу, яким не варто і зовсім нейтральні; по-третє, для Стої люди вже не поділялися на «абсолютно досконалих» і «абсолютно поганих», а мали закладений всередині потенціал до розвитку)
\subsection{Поняття про свободу}%2 pages
Головна роль у стоїцизмі відводилась етиці, якій підпорядковувалися натурфілософія (філософія природи) і логіка. Згідно зі стоїцизмом у світі існує жорстка, однозначна, невблаганна необхідність, яка не має жодних винятків. У цей фатальний, заздалегідь визначений потік буття включена і людина, тому безглуздими є розмірковування про її свободу, наявність у неї можливості вибору (а отже, й моралі, якщо послідовно дотримуватися цієї точки зору), про сенс людського буття, оскільки життєвий шлях кожної людини залежить не від неї. Однак вона має розум, а тому може усвідомити невідворотність долі. В цьому й полягає її свобода, яка є джерелом внутрішньої незворушності, спокою, стійкості духу людини. Отже, кожен має прагнути й намагатися жити згідно із природою з її невблаганною необхідністю.

Стоїки уявляли світ як єдине тіло, в якому діє всезагальний закон, який надає всьому сущому в ньому доцільний смисл: «Закон цей — правильний розум». Жити згідно з природою означає керуватися розумом, що править світом, а він є природною характеристикою індивіда. Проте від природи розум існує лише як можливість. Доброчесність, людська досконалість полягають у тому, щоб розвинути його до рівня світового розуму. «Сама природа, — вважав Зенон, — веде нас до доброчесності», тобто життя згідно з природою тотожне життю розумному, доброчесному.

Щастя людини, відповідно до вчення стоїцизму, - у свідомому виборі відповідності власній природі. Але ж природа дана нам від народження.
Отож, свобода стоїків – не свобода в сучасному розумінні: вона обмежена фатумом як першопричиною всіх речей. Фатум постає як наслідок необхідності:
«Добровільно віддай себе ткалі Клото й доручай їй впрясти тебе в будь-яку пряжу» (кн.IV, 34); але, крім вчення про нього, стоїки обґрунтували ще
й Провіденцію – поняття, що більше апелює до сутності кожної окремої людини: кожна окрема істота вже мислилась у світовому цілому як поняття
(т.зв. «сім’яний логос»), і, оскільки весь світ прямує до розкриття своєї природи, то й ця частина його за підтримки першопричини долучається до
основної мети: «Що від богів, повне промислу; що від випадку – теж не проти природи або сплетено з тим, чим керує промисл. Все тече – звідти;
і тут же невідворотність і користь того світового цілого, якого ти частина» (кн.ІІ,3). Оскільки «пневма» в різних областях діє різними
способами, між фатумом і провіденцією стоїки помістили природу. В цій концепції виражається теодицея – виправдання Бога за зло у
світі: те, що здається нам злим, може бути зумовленим природними потребами.

Стоїки визнавали, що найвища мета людського життя – щастя: «Всі, брате Галліоне, бажають жити щасливо…мети, яку сама природа зробила для нас такою бажаною» (Бл. ж. І). За Зеноном, щастя – в «узгодженому» житті (він мав на увазі узгодженість думок і почуттів); Клеанф додає, що воно мусить бути узгодженим з природою, і це твердження не втрачає актуальності й для Сенеки. Природа покладає щастя для людини в самозбереженні (а не в задоволенні,
як вчили епікурейці), як індивідуальному, так і колективному (турбота про дітей, батьків, «громадянство світу»). Отже, найпершою серед
чеснот для стоїків виступає мудрість; вона лежить в основі інших чеснот – помірності, мужності, самоконтролю (які об’єднує термін апатія)
і справедливості (соціальна цінність). Чесноти – шлях до атараксії, стану повної безтурботності мудреця. З усіх живих істот мудрості
здатна досягти лише людина, і то не кожна; на відміну від тварини, яка у своїх діях керується природою несвідомо, людина має робити усвідомлений
вибір на користь розуму: «… можу ще назвати блаженним того, хто завдяки розуму нічого не бажає й нічого не боїться. Правда, камені теж не
відають ні страху, ні печалі, як і скоти; однак їх не можна назвати щасливими, бо в них нема поняття про щастя…» (Бл. ж. V,1).
Благом можна назвати річ, лише якщо вона корисна для вищої мети життя. Стоїки розробили цілу класифікацію речей за етичним критерієм.

При цьому, застерігають стоїки, чесноти варто досягати безкорисливо, лише задля її самої, тобто не плутати благо з задоволенням: «якщо чеснота
і приносить насолоду, то досягти її прагнуть не заради цього» (Бл. ж. ІХ,1). Марк Аврелій інтерпретує цю безкорисливість дуже
близько до євангельського тексту: «Інший, коли зробить кому щось путнє, не забариться натякнути йому, що той відтепер заборгував…А
ще інший якось навіть і не пам’ятає, що зробив, а подібний до лози, яка принесла свій плід і нічого не чекає понад це» (кн..V,6). Важливо,
що чеснота не вважалася такою, якщо її не було досягнуто остаточно: «Жодна складова честі не може бути безчесною, й вище благо втратить свою 
істинність, якщо в ньому виявиться щось не цілком найкраще» (Бл. ж. XV,1). 

Логічно, що вади стоїки бачили як антитезу чеснотам; оскільки «сукупна чеснота» - розвинутий логос, то «сутність вади полягає в незнанні
чи недостачі або недосконалості мистецтва» (логосу). Порушення міри втілювалося в понятті афекту- неправильної думки, неодмінно пов’язаної з
сильним емоційним потягом. Називали 4 основних афекти – 2 в теперішньому (печаль і задоволення) і 2 в майбутньому (страх і сильна хіть). Стоїки
мали досить тонке розуміння афектів: для них злочинним був не тільки гнів, але й, скажімо, надмірне співчуття; були зроблені спроби створити 
ієрархію вад – Марк Аврелій так прокоментував Теофраста: «правильно й гідно філософії він стверджував, що гріхи, вчинені в насолоді, заслуговують 
на більший осуд, ніж коли з печаллю» (кн..2,11). Афекти як одиничні стани можуть розвинутися в постійні – душевні хвороби («безсилля 
душі», які стоїки, продовжуючи традицію Арістотеля, відрізняють від вродженої схильності (hexis) до певних емоцій (pathos). 
\subsection{Космополітизм стоїків}%1 page
В етиці стоїцизм близький киникам, не розділяючи презирливого ставлення останніх до культури. Всі люди - громадяни космосу як світового держави;
стоїчний космополітизм зрівнював (в теорії) перед особою світового закону всіх людей: вільних і рабів, греків і варварів, чоловіків і жінок.

Космополітизм стоїків безпосередньо випливає з шанованої ними чесноти -- справедливості. Справжній мудрець, наслідуючи принцип самозбереження,
вище блага одного (себе) ставить благо багатьох. «Розширюючи цю точку зору за межі рідного полісу, стоїки приходили до космополітизму… Цей 
космополітизм був баченням, характерним для громадян епохи утворення світової імперії, що поглинула Грецію з малими полісами». Отже,
з одного боку, такий підхід був вигідний Римській імперії. Але в той же час він сформував принципово нову етичну доктрину: відтепер людину
справедливо оцінювати – і то не завжди – тільки за етичним критерієм, який не враховує її бідності чи багатства, національності, статі: «Азія,
Європа – закапелки світу. Ціле море – для світу краплина. Афон – грудочка в ньому. Кожне теперішнє в часі – крапка для вічності.
Мале все, непостійне, зникоме...» (кн.. 6, 36); «Я називаю черню й тих, що носять лахміття, й вінценосців; я не дивлюсь на колір одягу, 
що вкриває тіла, і не вірю очам своїм, коли мова йде про людину... тільки дух може відкрити, що доброго є в іншому дусі» (Бл.ж.ІІ,2) 
\subsection{Етичний ідеал}%1 page
«Бути схожим на скелю, об яку постійно б’ється хвиля; вона стоїть – і розпалена волога затихає довкола неї. Нещасний я, таке зі
мною сталося! – Ні! Щасливий я, що зі мною таке сталося, а я все ще безпечальний, теперішнім не вражений, майбутнього не боюся» (кн..4,49).
Крім непохитності, мудрець – етичний ідеал стоїків – мав жити, керуючись розумом і природою (що описано вище), а також був людиною богорівною,
наділеною особливими знаннями та розумінням: «…своїми здоровими очима він зуміє побачити красу і певний розквіт у старенької чи старого, і 
привабливість новонародженого; йому зустрінеться багато такого, що зрозуміло не кожному, а тільки тому, хто душею звернений до природи та її справ»
(кн..ІІІ,2); «І ось така вже людина… є деякий жрець і посібник богів…» (кн..ІІІ,4). Для мудреця характерна чистота думок: «Привчати себе
треба тільки таке мати на думці, щоб, лишень тебе спитають: «Про що зараз думаєш?» - відповідати щиро й відразу, що й як» (кн..ІІІ,4). 
\section{Внесок в сучасну етику}%1 page (2)
Стоїцизм був впливовим філософським напрямком від епохи раннього еллінізму аж до кінця античного світу. Свій вплив ця школа залишала і
на подальші філософські епохи. В кінцевому підсумку відбулося зближення стоїцизму з неоплатонізмом, а потім розчинення його в останньому.
Також, безсумнівно вплив стоїцизму на гностичні вчення аскетичної спрямованості (валентініанская і маркіонітская школи). 

Інша їх доктрина в теорії пізнання була більш впливовою, хоча й більш сумнівною. Це була віра у вроджені ідеї та принципи. Грецька логіка
була повністю дедуктивно, і це ставило питання про вихідних посилках. Вихідні посилки повинні були бути, хоча б частково, загальними,
і не існувало способу довести їх. Стоїки вважали, що є певні принципи, які абсолютно очевидні і визнані всіма; вони можуть стати,
як у «Началах» Евкліда, основою дедукції. Вроджені ідеї, подібно до цього, можуть бути використані як відправна точка для дефініцій. Ця точка
зору була прийнята протягом середніх віків, і її поділяв навіть Декарт.

На думку деяких авторів, стоїки – не менш самостійна щодо попередників школа, ніж системи Арістотеля та Платона. Вона мала кілька здобутків
світового значення. По-перше, концепція «єдиного громадянства» і космополітизму підготувала ґрунт для появи поняття природного права 
(один із сучасних типів праворозуміння: право, що базується на загальному для всієї природи розумі), яке в ідеалі має втілюватися в позитивному 
праві – законодавстві держави, наскільки воно узгоджене з природним правом. По-друге, поєднання в етиці принципів свободи і необхідності згодом 
відбилось у філософії Гоббса і Спінози.
Нарешті, попри всі суперечності в цьому питанні, стоїки намагались утвердити автономність моралі в суспільстві щодо фізіологічних і соціальних потреб людини.
Сенека і Марк Аврелій також були моралістами-новаторами: зокрема, Сенека ввів в етику поняття сумління («Будь-якої миті, коли природа
вимагатиме повернути їй моє дихання...я піду, підтвердивши під присягою, що все життя любив
чисте сумління і гідні заняття...», а Марк Аврелій розмежував людську душу (пневма) та інтелект (нус).

Доктрина природного права, якою вона була в XVI, XVII, XVIII століттях, є відродження доктрини стоїків, хоча і з важливими змінами. Саме стоїки 
відрізняли \textit{jus naturale} від \textit{jus gentium}.
Природне право випливало з перших таких принципів, які вживалися для обґрунтування всякого загального 
знання. За природою, вчили стоїки, всі людські істоти рівні. Марк Аврелій у своїх «Роздумах» хвалить «політику, в якій існує один і той самий
закон для всіх, - політику керовану, що бере до уваги рівні права і рівну свободу слова, - і царське правління, яке поважає найбільше свободу
керованих». Це був ідеал, який не міг бути відповідно здійснено в Римській імперії, але який впливав на законодавство, зокрема в сенсі
поліпшення становища жінок і рабів. Християнство перейняло цю частину вчення стоїків разом з багатьма іншими. І коли, нарешті,
в XVII столітті настав час ефективно боротися проти деспотизму, вчення стоїків про природне право і природному рівність у своєму християнському 
вбранні придбало практичну силу, якій під час античності не міг йому надати навіть імператор.
\begin{thebibliography}{9}
\bibitem{levchuk}
Л. Т. Левчук, Д. Ю. Кучерюк, В. І. Панченко; За заг, ред. Л. Т. Левчук.
{\em Естетика: Підручник}.
К-Вища шк., 1997.— 399 с ISBN 5-11-004388-4.
\bibitem{osvita_ua}
{\em Сюрреалізм як модерністський напрямок мистецтва ХХ століття} - \url{http://osvita.ua/vnz/reports/culture/10890/}
\bibitem{redfox}
{\em Етика стоїцизму} - реферат на \url{http://redfox.if.ua/page/jetika-stojicizmu-11263.html}
\bibitem{coolref}
{\em Етика стоїцизму} - реферат на \url{http://ua.coolreferat.com/}
\end{thebibliography}
\end{document}
