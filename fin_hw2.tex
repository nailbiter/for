\documentclass[8pt]{article} % use larger type; default would be 10pt

%\usepackage[utf8]{inputenc} % set input encoding (not needed with XeLaTeX)
\usepackage[10pt]{type1ec}          % use only 10pt fonts
\usepackage[T1]{fontenc}
%\usepackage{CJK}
\usepackage{graphicx}
\usepackage{float}
\usepackage{CJKutf8}
\usepackage{subfig}
\usepackage{amsmath}
\usepackage{amsfonts}
\usepackage{hyperref}
\usepackage{enumerate}
\usepackage{enumitem}
\usepackage{harpoon}

\usepackage{mystyle}

\title{Math 1540\\University Mathematics for Financial Studies\\2013-14 Term 1\\Suggested solutions for\\HW problems 12.3-12.5}
\begin{document}
\maketitle
	\renewcommand{\v}{\mathbf{v}}
	\renewcommand{\u}{\mathbf{u}}
	\renewcommand{\i}{\mathbf{i}}
	\renewcommand{\j}{\mathbf{j}}
	\renewcommand{\k}{\mathbf{k}}
	\newcommand{\myproj}{\mbox{proj}_{\v}{\u}}
\begin{description}
%12.3: #5, 8, 29
\section{Section 12.3}
\item[\# 5.]{
	{\it 
	Find
		\begin{enumerate}[label=\bfseries\alph*.]
			\item{$\mathbf{v}\cdot \mathbf{u},\;\myabs{\v},\;\myabs{\u}$}
			\item{the cosine of angle between $\v$ and $\u$}
			\item{the scalar component of $\u$ in the direction of $\v$}
			\item{the vector projection $\mbox{proj}_{\v}{\u}$}
		\end{enumerate}
		where $\v=5\j -\k,\quad \u=\i+\j+\k$.}
	This is straightforward computational problems, which should pose no difficult once you remember the formulas
	\begin{gather*}
		\v\cdot\u=xx'+yy'+zz',\;\mbox{where }\v=x\i+y\j+z\k,\;\u=x'\i+y'\j+z'\k\\
		\myabs{v}=\sqrt{x^2+y^2+z^2}\\
		\cos\left(\angle(\v,\u)\right)=\frac{\v\cdot\u}{\myabs{\u}\myabs{\v}}\\
		\mbox{the scalar component of $\u$ in the direction of $\v$ }=\frac{\u\cdot\v}{\myabs{\v}}\\
		\mbox{proj}_{\v}{\u}=\left(\frac{\u\cdot\v}{\myabs{\v}^2}\right)\v
	\end{gather*}
	and can work out the algebra without a mistakes! The answers are
	\begin{gather*}
		\v\cdot\u=2\\
		\myabs{v}=\sqrt{33}\\
		\myabs{u}=\sqrt{3}\\
		\cos\left(\angle(\v,\u)\right)=\frac{2}{3\sqrt{11}}\\
		\mbox{the scalar component of $\u$ in the direction of $\v$ }=\frac{2}{\sqrt{33}}\\
		\mbox{proj}_{\v}{\u}=\left<0,\frac{10}{33},\frac{-2}{11}\right>
	\end{gather*}
}
\item[\# 5.]{
	{\it 
	Find
		\begin{enumerate}[label=\bfseries\alph*.]
			\item{$\mathbf{v}\cdot \mathbf{u},\;\myabs{\v},\;\myabs{\u}$}
			\item{the cosine of angle between $\v$ and $\u$}
			\item{the scalar component of $\u$ in the direction of $\v$}
			\item{the vector projection $\mbox{proj}_{\v}{\u}$}
		\end{enumerate}
		where $\v=5\j -\k,\quad \u=\i+\j+\k$.}
	Again, we just applying formulas outlined in previous problem.
	\begin{gather*}
		\v\cdot\u=\frac{1}{6}\\
		\myabs{v}=\myabs{u}=\sqrt{\frac{5}{6}}\\
		\cos\left(\angle(\v,\u)\right)=\frac{1}{5}\\
		\mbox{the scalar component of $\u$ in the direction of $\v$ }=\frac{1}{\sqrt{30}}\\
		\mbox{proj}_{\v}{\u}=\left<\frac{1}{5\sqrt{2}},\frac{1}{5\sqrt{3}}\right>
	\end{gather*}
	}
\item[\# 29.]{{\it Using the definition of the projection of $\u$ onto $\v$, show by direction computation that
	$(\u-\myproj)\cdot\myproj=0$.
	}
	Indeed,
	\begin{gather*}
	(\u-\myproj)\cdot\myproj=(\u-\left(\frac{\u\cdot\v}{\myabs{\v}^2}\right)\v)\cdot\left(\frac{\u\cdot\v}{\myabs{\v}^2}\right)\v=\\
	=\left(\frac{\u\cdot\v}{\myabs{\v}^2}\right)\u\cdot\v-\left(\frac{\u\cdot\v}{\myabs{\v}^2}\right)^2\v\cdot\v=\\
	=\left(\frac{\u\cdot\v}{\myabs{\v}^2}\right)(\u\cdot\v-\left(\frac{\u\cdot\v}{\myabs{\v}^2}\right)\v\cdot\v)=\\
	=\left(\frac{\u\cdot\v}{\myabs{\v}^2}\right)(\u\cdot\v-\u\cdot\v)=0
	\end{gather*}
	}
\end{description}
%12.4: #7, 15, 50
\section{Section 12.4}
\begin{description}
	\item[\# 7.]{{\it Find the length and direction (where defined) of $\u\times\v$ and $\v\times\u$, where $\u=
		-8\i-2\j-4\k,\;v=2\i+2\j+\k$.
		}
		Just by definition,
		\[\myvecprodexpanded{-8}{-2}{-4}{2}{2}{1}=6\i-12\k\]
		Hence \[\myabs{\u\times\v}=\myabs{\v\times\u}=6\sqrt{5}\]
		By "find the direction" it is means to find a unit vector, that points in the same direction as the original one, so the
		directions are
		\[\frac{\u\times\v}{\myabs{\u\times\v}}=\left<\frac{1}{\sqrt{5}},0,\frac{-2}{\sqrt{5}}\right>\]
		\[\frac{\v\times\u}{\myabs{\v\times\u}}=\frac{-\u\times\v}{\myabs{-\u\times\v}}=\left<\frac{-1}{\sqrt{5}},0,
		\frac{2}{\sqrt{5}}\right>\]
		}
	\item[\# 15.]{{\it \begin{enumerate}[label=\bfseries\alph*.]
			\item{Find the area of $\triangle PQR$.
				}
			\item{Find a unit vector, perpendicular to plane $PQR$.}
			\end{enumerate}
			where $P=P(1,-1,2)$, $Q=Q(2,0,-1)$ and $R=R(0,2,1)$.
		}
		\begin{enumerate}[label=\bfseries\alph*.]
			\item{The area of triangle is given as $\frac{1}{2}\myabs{\myvec{PQ}\times\myvec{PR}}$ and therefore we first
				need to compute the vector product of $\myvec{PQ}=\left<1,1,-3\right>$ and $\myvec{PR}=\left<
				-1,3,-1\right>$
				\[
				\myvecprodexpanded{1}{1}{-3}{-1}{3}{-1}=8\i+4\j+4\k
				\]
				And thus
				\[\frac{1}{2}\myabs{\myvec{PQ}\times\myvec{PR}}=\frac{1}{2}\sqrt{96}=\sqrt{24}=2\sqrt{6}\]
				}
			\item{The vector parallel to the plane $PQR$ is given by $\myvec{PQ}\times\myvec{PR}$ and thus we just need
				to normalize what we have got during the computations above
				\[\mathbf{n}=\frac{\left<8,4,4\right>}{\sqrt{96}}=\left<\frac{2}{\sqrt{6}},\frac{1}{\sqrt{6}},\frac{1}{\sqrt{6}}
				\right>\]
				}
		\end{enumerate}
		}
\end{description}
%12.5: #10, 23, 31, 39
\section{Section 12.5}
\begin{description}
	\item[\# 10.]{{\it Find parametric equation for the line through $(2,3,0)$, perpendicular to the vectors $\u=\i+
		2\j+3\k$ and $\v=3\i+4\j+5\k$.
		}
		To begin with, we need a direction, that is a vector, that would be perpendicular to both $\u$ and $\v$. Such vector
		can be easily computed via the vector product
		\[\mathbf{n}=\myvecprodexpanded{1}{2}{3}{3}{4}{5}=-2\i+4\j-2\k\]
		Knowing the point on a line and direction, parametric equation is easily written
		\[x=2-2t,\;y=3+4t,\;z=-2t,\quad-\infty<t<\infty\]
		}
	\item[\# 23.]{{\it Find equation of the plane passing through the $A(1,1,-1)$, $B(2,0,2)$ and $C(0,-2,1)$.
		}
		As in the textbook's example, we find a vector normal to the plane and use it with one of the points (it does
		not matter which) to write an equation for the plane.
		The cross product
		\[\myvec{AB}\times\myvec{AC}=\myvecprodexpanded{1}{-1}{3}{-1}{-3}{2}=7\i-5\j-4\k\]
		is normal to the plane. Hence, knowing the perpendicular to the plane vector and point on it, planar equation is easily written as
		\[7(x-1)-5(y-1)-4(x+1)=0\]or equivalently as\[7x-5y-4x=6\]
		}
	\item[\# 31.]{{\it Find a plane through $P_0(2,1,-1)$ and perpendicular to the line of intersection of planes $2x+y-z=3,\;
		x+2y+z=2$.
		}
		This problem in fact consists of two parts which are better to be handled as separate ones. First, let us find an equation of a line
		of intersection of $2x+y-z=3$ and $x+2y+z=2$. The problem similar to this one was done in textbook, so we just outline the steps.
		To begin with, we find a direction of a line, by noticing that it should be perpendicular to $\left<2,1,-1\right>$ and
		$\left<1,2,1\right>$ (for these vectors are perpendicular to the first and the second planes
		respectively, while intersection line's direction is parallel to both of them, as line itself lies in both of them). The
		direction thus can be found as a vector product of two vectors mentioned
		\[\myvecprodexpanded{2}{1}{-1}{1}{2}{1}=3\i-3\j+3\k\]
		To find some (arbitrary would suffice) common point of both planes we substitute $z=0$ in both planar equations and solve for $x$
		and $y$ which gives
		\[\begin{cases}2x+y=3\\x+2y=2\end{cases}\implies x=\frac{4}{3},\;y=\frac{1}{3}\]
		Thus $\left(\frac{4}{3},\frac{1}{3},0\right)$ lies on both planes. Then, knowing direction and point on a line, its equation is
		easily seen to be
		\[x=\frac{4}{3}+3t,\;y=\frac{1}{3}-3t,\;z=3t,\quad -\infty<t<\infty\]

		Now, knowing the equation of a line, we have to find a plane which is perpendicular to it and passes through a given point. 
		This is also the standard problem, equation is immediately seen to be
		\[3(x-2)-3(y-1)+3(z+1)=0\] or equivalently\[3x-3y+3z=0\]

		Note 
		that not all the computations above are in fact necessary for this problem: we are not interested in \textit{all} information about
		the intersection
		line, we just needed the \textit{direction}. Therefore, computation of the point on a line (the point that belonged to both planes)
		could be omitted, if we would have to save time (during the test, say). 
		}
	\item[\# 39.]{{\it Find the distance from the point $S(2,-3,4)$ to the plane $x+2y+2z=13$.}
		We just substitute the concrete values in the known formula in this problem
		\[d=\myabs{\myvec{PS}\cdot\frac{\mathbf{n}}{\myabs{\mathbf{n}}}}=\frac{\left<1,-6,1\right>\cdot\left<1,2,2\right>}{3}=\myabs{-3}=3\]
		note, that point $P(1,3,3)$, that we have taken as a point on a plane, indeed lies on a plane. 
		}
\end{description}
\end{document}
