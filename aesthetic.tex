%surreal
\documentclass[14pt]{extarticle} % use larger type; default would be 10pt
\usepackage{fontspec}
\usepackage{array, xcolor, lipsum, bibentry}
\usepackage[margin=3cm]{geometry}
\usepackage{hyperref}
\usepackage{fancyhdr}

 
\title{\bfseries\Huge Oleksii Leontiev}
\author{inp9822058@cs.nctu.edu.tw}
\date{}
 
\definecolor{lightgray}{gray}{0.8}
\newcolumntype{L}{>{\raggedleft}p{0.2\textwidth}}
\newcolumntype{R}{p{0.8\textwidth}}
\newcommand\VRule{\color{lightgray}\vrule width 0.5pt}
 
%font configuration
\defaultfontfeatures{Mapping=tex-text}
\setromanfont[Ligatures={Common}, Numbers={OldStyle}, Variant=01]{Times New Roman} % Main text font
%%\setromanfont[Ligatures={Common}, Numbers={OldStyle}, Variant=01]{Linux Biolinum Slanted} % Main text font
\chardef\&="E050 % Custom ampersand character

\begin{document}
\begin{titlepage}
	\addtolength{\voffset}{-2cm}
	%\setlength{\footskip}{5.5cm}
	\thispagestyle{fancy}
	\fancyfoot[C]{м. Київ -- 2013}
	\begin{center}
		\newcommand{\HRule}{\rule{\linewidth}{0.5mm}}
		\textsc{\Large Київський Національний Університет імені Тараса Шевченка}\\[1.5cm]

		% Title
		\HRule \\[0.4cm]
		{ \huge \bfseries Мистецтво сюрреалізму}\\[0.4cm]

		\HRule \\[1.5cm]

		% Author 
			\begin{flushright} \large
				Робота студента\\
				4-го курсу\\
				механіко-математичного факультету\\
				заочної форми навчання\\
				\textsc{Леонтьєва} Олексія Костянтиновича
			\end{flushright}

		\vfill

		% Bottom of the page
		{\large 19 червня 2013 р.}
	\end{center}
\end{titlepage}
\section{Вступ}
Сюрреалізм (від фр. surreealisme — надреалізм) існує вже понад піввіку і сьогодні належить до популярних та впливових течій. 
Це модерністський напрямок у мистецтві ХХ століття,
який сформувався на початок 1920-х у Франції і відрізняється використанням алюзій і парадоксальних поєднань форм. Він проголосив джерелом мистецтва
сферу підсвідомості, а його методом - розриви
логічних зв'язків, замінених суб'єктивними асоціаціями. Головними рисами сюрреалізму є протиприродність сполучення, що лякає, предметів 
і явищ, яким надається видима вірогідність.

Сюрреалізм сформувався як модерний художньо-літературний рух, гаслом якого стала тріада: "кохання, краса, бунт". За словами Бретона (1935 рік), два заклики стали основоположними у формуванні сюрреалістичного канону: "переробити світ" (Маркс) і "змінити життя" (Рембо).

Це явище літературно-мистецького життя, що зародилося в атмосфері розчарування, характерної для французького суспільства після Першої світової війни, справило потужний вплив не тільки на сучасні йому мистецво та літературу, а й на подальший розвиток всіх сфер життя, що мають відношення до художньої творчості. Сюрреалізм можна назвати революційним перетворенням модерного мистецтва, яке, образно кажучи, відкрило нові двері сприйняття, ставши джерелом натхнення для багатьох мистецьких рухів. Спадщина сюрреалізму знайшла втілення в прозі (наприклад, "Новий роман", леттризм) і поезії, живопису та скульптурі, в театральному та кіномистецтві, в графіці та графіті. Ідеї та гасла, висунуті в сюрреалістичному русі, вплинули на пізніші мистецькі течії: абстракціонізм, експресіонізм, брутизм, перформанс-арт, неодадаїзм, поп-арт і концептуалізм.

До тих, хто належав до сюрреалістичного руху або ж поділяв ті чи інші його ідеї, належать Марсель Дюшан, Пауль Клее, Макс Ернст, Хуан
Міро, Ганс (Жан) Арп, Сальвадор Далі, Андре Массон, Рене Магрітт, Альберто Джакометті, Мерет Оппенгейм, Доротея Таннінг, Роберто Матта,
Пабло Пікассо, Ман Рей, Луї Арагон, Поль Елюар, Робер Деснос, Жорж Батай, Жан Кокто, Луїс Бунюель, Ів Тангі, Жак Превер та інші. 

\section{Передумови і поява}%TODO: freud, dada
Хоча засновником та ідеологом сюрреалізму вважається письменник і поет Андре Бретон, він не був першим, хто використав цей термін.
Під заголовком "сюрреалістична драма" позначив у 1917 одну зі своїх п'єс Гійом Аполлінер. 

Навесні 1917 Гійом Аполлінер придумав і вперше вжив термін "сюрреалізм" у своєму маніфесті "Новий дух", написаному до скандально гучного
балету " Парад ". Цей балет був спільною роботою композитора Еріка Саті, художника Пабло Пікассо, сценариста Жана Кокто, і балетмейстера Леоніда Мясіна : "У цьому новому союзі нині створюються декорації та костюми, з одного боку, і хореографія - c інший, і ніяких фіктивних накладень не
відбувається. В" Параді ", як у вигляді зверхреалізм (сюрреалізму), я бачу вихідну точку для цілого ряду нових досягнень цього Нового духу ". 

Сюрреалізм відокремився від дадаїзму, твори якого були своєрідною спробою художників і поетів «жити й творити, керуючись дитячою психологією». 
Виник дадаїзм у 1916— 1918 pp. в Швейцарії, Німеччині, Франції. Ця течія склалася як позанаціональне об'єднання молодих митців, переконаних, що можна
створити «утопічне, міжкласове товариство експлуатованих», яке за допомогою мистецтва звільнить світ: те, що зіпсувала політика, виправить мистецтво.

Мине час, і колишні дадаїсти Д. де Кіріко, К- Карра та інші будуть говорити про дадаїзм та його пізнішу трансформацію у сюрреалізм як
про «божевілля 1914 року». Аналізуючи морально-психологічний стан молодих митців того періоду, можна констатувати, що,не приєднавшись
до прогресивного мистецтва, налякані зростанням революційної боротьби народних мас, модерністи все ж не приймають агресивних ідеалістичних ідей.
Виникає мрія про надкласову позицію, про «всесильне» мистецтво, яке допоможе розв'язати всі складні життєві ситуації.

Першу світову війну ці митці сприйняли як переконливу демонстрацію божевілля, породженого інтелектом, як «божевільну і звірячу виставу».
Перебуваючи у полоні поверхового, примітивного підходу до аналізу причинно-наслідкових зв'язків, вони винуватцем усього, що відбувалося,
проголошують розум і шукають порятунку в «анти-розумі». За образним висловом німецького мистецтвознавця Й. Шварца, починається повстання проти 
раціональності й інтелекту. Це своєрідне «повстання» і стало поштовхом до відокремлення сюрреалістів від дадаїстів. Якщо дадаїсти виступали
«проти всього і всіх», то сюрреалісти звузили поняття «все», «всі» до понять «раціональне», «розум», «інтелект».

Починаючи з 1921 р. більшість французьких дадаїстів перейшли на позиції сюрреалізму. Його теоретиком і «духовним» батьком став А. Бретон.
До групи Бретона входили М. Рей, М. Ернст, Ж. Барон, І. Тангі, Ж- Пре-вер. До цієї ж групи, що рік від року збільшувалася, хоч 
і не мала глибоких внутрішніх зв'язків та справжньої творчої єдності, належали «Чарівник з Бельгії» — Р. Магрітт і каталонець С. Далі.
Ці два художники згодом стали всесвітньовідомими, і саме іхній творчості сюрреалізм багато в чому завдячує своєю популярністю.
Товариство сюрреалістів організаційно утвердилося між 1921 і 1924 pp.

Теоретична програма сюрреалізму формувалася при безпосередній участі 3. Фрейда. Відомі листування Фрейда з А. Бретоном — фактичним
засновником цього напряму мистецтва, зустріч Фрейда з С. Далі — найбільш визначним практиком сюрреалізму. Не буде перебільшенням сказати, що 
сюрреалізм — це також своєрідна художня ілюстрація психоаналізу. Сюрреалісти повністю підтримали думку Фрейда про невичерпність позасвідомого, 
його активний вплив на життя кожної людини. Бретон розглядає творчість як стан «позасвідомих спонтанних процесів». Процес творчості, на його
думку,— процес загадковий і не піддається логічному осмисленню. Тому справжній художник працює лише асоціативним, алогічним методом, спираючись
на власні сновидіння.

По визнанню Далі, для нього світ ідей Фрейда означав стільки ж, скільки світ Священного Писання означав для середньовічних художників чи світ
античної міфології - для художників Відродження.

Ідеї віденського психолога і мислителя мали особливий сенс для цих людей, тому що фрейдизм був життєво
важливий для сюрреалістів і був, бути може, одним з головних факторів підйому й успіху їхньої доктрини. Однак було би спрощенням
думати, що сюрреалісти працювали по рецептах чи Фрейда "ілюстрували" його ідеї. Фрейдизм допомагав їм в іншому плані.

Самі концепції сюрреалістів одержували могутню підтримку з боку психоаналізу й інших відкриттів фрейдизму. І перед собою і перед
іншими вони одержували вагомі підтвердження правильності своїх устремлінь. Вони не могли не помітити, що "випадкові" методи раннього сюрреалізму 
(напр. фроттаж) відповідали фрейдовській методиці "вільних асоціацій", що вживалася при вивченні внутрішнього світу людини. Коли
пізніше в мистецтві художників сюрруалізму стверджувався принцип ілюзіоністського "фотографування несвідомого", те не можна не згадати про те,
що психоаналіз виробив техніку "документального реконструювання " сновидінь.

Європейське людство вже давно було занурено в суперечки про сутність, межах і необхідності моральності: імморалізм Фрідріха Ніцше мало кого
залишив байдужим. Але фрейдизм викликав більш широкий резонанс. Він не був просто філософською тезою. Він був більш-менш науковим
плином, він пропонував і прогнозував досвідчену перевірюваність своїх постулатів і висновків, він розробляв клінічні методи впливу на психіку
- методи, що давали безсумнівний успіх. Фрейдизм укоренився не тільки на університетських кафедрах і у свідомості інтелектуалів, вона нестримно
завойовував собі місце в більш широких сферах суспільного буття. І він виключав мораль і розум із самих основ життєдіяльності людини, вважаючи
їх вторинними і багато в чому навіть обтяжними утвореннями цивілізації.
\section{Основні концепції}
\section{Відомі представники}
Якось Далі сказав: "Сюрреалізм - це я!". Дійсно, він вважається "королем сюрреалізму" багатьма критиками і істориками мистецтва. Сон і уява, 
марення і дійсність перемішані і нерозрізнені, так що не зрозуміти, де вони самі по собі злилися, а де були ув'язані між собою вмілою рукою 
художника. Фантастичні сюжети, дивовижні галюцинації, гротеск у сполученні з віртуозною мальовничою технікою - саме це залучає у творах Далі.

Швидше за все, маючи справу з цим художником і людиною, треба виходити з того, що буквально усе, що його характеризує 
(картини, літературні твори, суспільні акції і навіть життєві звички) варто було б розуміти як сюрреалістичну діяльність. Далі дуже цілісний
у всіх своїх проявах. 
Читаючи про Далі, про його витівки, що шокують публіку, розумієш, що і все його життя було життям сюрреаліста. Якщо вірити його
"Щоденнику", він навіть хотів перетворити шматок реального світу у своїй каталонській резиденції в подобу своєї сюрреалістичної картини, засіявши 
берег безліччю слонячих черепів, спеціально виписаних для цієї мети з тропічних країн.

Далі дійсно був сюрреалістом до мозку кіст. У сюрреалістичні образи перетворювалося усе, що він чи робив говорив. Далі всерйоз пестував 
і культивував своє сюрреалістичне "Я" тими самими способами, що особливо цінувалися і шанувалися всіма сюрреалістами. В очах "розумних
і моральних" людей радикальна філософія сюрреалізму, узята зовсім серйозно і без усяких застережень (як у Далі), викликає протест.

І безглуздо дорікати Далі в непослідовності, тому що алогізм і ірраціональність - його програма і його стихія. Саме такий був 
метод творчості Далі й у житті, і в мистецтві. Він схожий на ризикований експеримент зі змістами і цінностями європейської традиції. Далі немов 
випробує їх на міцність, зіштовхуючи між собою і вигадливо з'єднуючи непоєднуване. Але в результаті створення цих дивовижних образних і значеннєвих
амальгам явно розпадається сама матерія, з якої вони складаються. Далі небезпечні для тихого і затишного пристрою людських справ, для людського
"добробуту", тому що він дискредитує зміст і цінності культури. Він дискредитує і релігію і безбожництво, і нацизм і антифашизм, і 
поклоніння традиціям мистецтва, і авангардний бунт проти них, і віру в людину і невір'я в нього.

Принципи сюрреалізму істотно вплинули також на розвиток західного театрального мистецтва. В ЗО—40-і роки французький театральний критик
і драматург А. Арто намагався створити «театр жорстокості», суть якого полягала у відмові від зображення об'єктивної дійсності. Усупереч класичному
Арто мріяв створити театр, на виставах якого глядач подорожуватиме у світ підсвідомого. Така подорож повинна була показати безперспективність 
людського існування, приреченість усіх людських зусиль. Арто намагався «правдиво зобразити сни», де тяжіння до злочину, еротичні думки,
варварство, химери, утопічний смисл бут->гя, навіть канібалізм виявлялися б не у вигляді ілюзій та припущень, а в їх прихованому внутрішньому світі.

Після другої світової війни в Арто з'явилися численні послідовники — А. Адамов, С. Беккет, Е. Іонеско та ін.
Ці «реформатори» театру зробили чимало для того, шоб знищити в театральних виставах усе розумне, реальне, пропагували філософію приреченості, відчаю
, людської самотності.

У п'єсі «Носорог» 5 Іонеско демонструє приголомшеному глядачеві перетворення людей на носорогів. Крізь дивну, штучно ускладнену символіку
драматург намагається донести до глядача більш ніж конкретну ідею: служіння громадським цілям та інтересам робить з людини однотипний,
тупоголовий табун. Цікаво, що відома художниця-сюрреалістка С. Кан у картині «Пам'яті Іонеско» (1970 р.) зобразила істоту у вишуканому світлому
вбранні, зі спокійно складеними людськими руками (в одній з них — трефова карта) та з головою носорога. 
Таке осмислення «образу Іонеско» — данина п'єсі «Носорог», яка більшістю європейських критиків визнається вершиною творчості драматурга.
\section{Сюрреалізм сьогодні}
В даний час напрямок помітно комерціалізувалося. Продовжуючи традицію Далі, Тангі, Дельво, Ернста, художники запозичили у них головним чином 
зовнішню сторону напрямки - фантасмагоричность сюжету. Глибинно-психологічна сторона сюрреалізму, експресія, вираз своїх несвідомих фантазій,
сексуальних страхів і комплексів, передача мовою іносказань елементів власного дитинства, особистого життя - цей внутрішній аспект,
що вважався в 1920-30-х рр.. титульним, визначальною якістю сюрреалізму, найчастіше ігнорується. Джим Уоррен наповнює полотна барвистими,
життєствердними сюжетами, вважаючи головною метою підняти настрій кожного окремого глядача, вселити йому інтерес до життя і благоговіння
перед природою. Картини Любові Зубової можна розділити на "теплі" і "холодні": вона насичує їх то приємними, ласкавими квітами заходу і золотистого
моря, то прохолодними фарбами ночі або раннього ранку. Деякі полотна поєднують в собі і тепло і холод. Картини немов спонукають глядача
просто милуватися красою, кришталевого повітря, безтурботністю моря, не опосередковуючи все це будь-якими смислами. Таким чином, сюрреалізм
еволюціонував, і в даний час важко говорити про нього як про чистий напрямок: він відчув сильний вплив фентазі-арту та класичного мистецтва. 
\begin{thebibliography}{9}
\bibitem{levchuk}
Л. Т. Левчук, Д. Ю. Кучерюк, В. І. Панченко; За заг, ред. Л. Т. Левчук.
{\em Естетика: Підручник}.
К-Вища шк., 1997.— 399 с ISBN 5-11-004388-4.
\bibitem{osvita_ua}
{\em Сюрреалізм як модерністський напрямок мистецтва ХХ століття} - \url{http://osvita.ua/vnz/reports/culture/10890/}
\bibitem{znaimo}
{\em Сюрреалізм} - стаття на \url{http://znaimo.com.ua/%D0%A1%D1%8E%D1%80%D1%80%D0%B5%D0%B0%D0%BB%D1%96%D0%B7%D0%BC}
\end{thebibliography}
\end{document}
%TODO: literature->read->gathher->proofread
%TODO: 8~10 pages, each chapter: 1 page
