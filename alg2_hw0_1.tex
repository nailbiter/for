\documentclass[8pt]{article} % use larger type; default would be 10pt

%\usepackage[utf8]{inputenc} % set input encoding (not needed with XeLaTeX)
\usepackage[10pt]{type1ec}          % use only 10pt fonts
\usepackage[T1]{fontenc}
%\usepackage{CJK}
\usepackage{graphicx}
\usepackage{float}
\usepackage{CJKutf8}
\usepackage{subfig}
\usepackage{amsmath}
\usepackage{amsfonts}
\usepackage{hyperref}
\usepackage{enumerate}
\usepackage{enumitem}

%theorem environments configuration
\newtheorem{theorem}{Theorem}
\newtheorem{claim}{Claim}
\newtheorem{lemma}[theorem]{Lemma}
\newtheorem{proposition}[theorem]{Proposition}
\newtheorem{corollary}[theorem]{Corollary}
\newenvironment{proof}[1][Proof]{\begin{trivlist}
\item[\hskip \labelsep {\bfseries #1}]}{\qed\end{trivlist}}
\newenvironment{definition}[1][Definition]{\begin{trivlist}
\item[\hskip \labelsep {\bfseries #1}]}{\end{trivlist}}
\newenvironment{example}[1][Example]{\begin{trivlist}
\item[\hskip \labelsep {\bfseries #1}]}{\end{trivlist}}
\newenvironment{remark}[1][Remark]{\begin{trivlist}
\item[\hskip \labelsep {\bfseries #1}]}{\end{trivlist}}
\newcommand{\qed}{\nobreak \ifvmode \relax \else
\ifdim\lastskip<1.5em \hskip-\lastskip
\hskip1.5em plus0em minus0.5em \fi \nobreak
  \vrule height0.75em width0.5em depth0.25em\fi}

\newtheorem{prob}{Problem}

\newenvironment{solution}%
{\par\textbf{Solution}\space }%
{\par}

\title{Algebra II\\Homework 0.1}
\author{Oleksii (Alex) Leontiev\\歐立思\\3035078276\\Exchange Student (BSc)\\4th grade\\
Original School: \href{http://www.nctu.edu.tw/}{NCTU}, Taiwan}

\begin{document}
\begin{CJK}{UTF8}{bsmi}
\maketitle
\end{CJK}
\begin{claim}
	Degree of field extension $[\mathbb{Q}(\sqrt{2},\sqrt[3]{3}):\mathbb{Q}]=6$
\end{claim}
\begin{proof}We divide proof into two steps
	\begin{enumerate}[label=\textbf{Step }\bfseries\arabic*.]
		\item{First, we shall show that order of element $\alpha:=\sqrt{2}+\sqrt[3]{3}$ over $\mathbb{Q}$ is 6. To do this
			it is enough to construct irreducible polynomial, that has $\alpha$ as its zero.
			Consider polynomial, defined as
			\[p(x)=(x-\sqrt{2}-\nu^0\sqrt[3]{3})(x-\sqrt{2}-\nu^1\sqrt[3]{3})(x-\sqrt{2}-\nu^2\sqrt[3]{3})\]
			\[(x+\sqrt{2}-\nu^0\sqrt[3]{3})(x+\sqrt{2}-\nu^1\sqrt[3]{3})(x+\sqrt{2}-\nu^2\sqrt[3]{3})\]
			where
			\[\nu=\cos(\pi/3)+i\sin(\pi/3)=\frac{1}{2}+i\frac{\sqrt{3}}{2}\in\sqrt[3]{-1}\]
			It obviously has $\alpha$ as its zero. Moreover, it incidentally happens after carrying out multiplication that
			\[p(x)=1-36x+12x^2-6x^3-6x^4+x^6\]
			That is, $p(x)\in\mathbb{Q}[x]$. Finally, we shall show that $p(x)$ is irreducible. By well-known theorem, it is enough
			to show that it is irreducible over $\mathbb{Z}$, since $p(x)\in\mathbb{Z}[x]$. Consider factorizations possible.\\ 
			Products definitely cannot be linear factors - $p(x)$ has no zeros over $\mathbb{Z}$. Similarly, products
			cannot be quadratic polynomials, since otherwise sum of some two roots would give natural number, but this is not true.
			Hence, factorization is of the form
			\[x^6-6x^4-6x^3+12x^2-36x+1=(x^3+ax^2+bx+1)(x^3+cx^2+dx+1)\text{ or }(x^3+ax^2+bx-1)(x^3+cx^2+dx-1)\]
			First possibility yields contradictory requirements $a+c=0,\;b+ac+d=-6,\;b+d=-36\implies a+c=0$ and $ac=30$. Similarly,
			second yields $a+c=0$ and $ac=-42$ and thus is also impossible. Hence, no factorization is possible, and $p(x)$ is
			irreducible over $\mathbb{Z}$ and hence over $\mathbb{Q}$. Order of $\alpha$ over $\mathbb{Q}$ is $6$. Hence,
			in particular, $[\mathbb{Q}(\alpha):\mathbb{Q}]=6$.
			}
		\item{We shall show that $\mathbb{Q}(\sqrt{2},\sqrt[3]{3})=\mathbb{Q}(\alpha)$ and thus finish the proof of the claim.
			It is only necessary to show that $\sqrt{2},\sqrt[3]{2}\in\mathbb{Q}(\alpha)$, since other inclusion is obvious.\\
			Indeed, for $i=1,2,\ldots,5$, $\alpha^i$ can be expressed as linear combination of five numbers $\sqrt{2},3^{1/3},3^{2/3},
			\sqrt{2}3^{1/3},\sqrt{2}3^{2/3}$ with coefficients in $\mathbb{Q}$. Naturally, this yields us
			five equations for these 5 numbers considered as variables and with coefficients in $\mathbb{Q}(\alpha))$. If these
			equations are all linearly independent, then they can be solved in $\mathbb{Q}(\alpha)$ and these 5 numbers
			hence belong to $\mathbb{Q}(\alpha)$ with $\sqrt{2}$ and $\sqrt[3]{2}$ in particular.\\
			But equations are in fact linearly independent. If they would not be so, some linear combination (with rational
			coefficients) of $\alpha^1,\alpha^2,\alpha^3,\alpha^4,\alpha^5$ would equate to zero, contradicting the fact
			that order of $\alpha$ over $\mathbb{Q}$ is 6.
			}
	\end{enumerate}
\end{proof}
\end{document}
