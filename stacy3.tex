\documentclass[12pt]{article} % use larger type; default would be 10pt

\usepackage{textcomp} %for copyleft symbol
\usepackage{mathtext}                 % підключення кирилиці у математичних формулах
                                          % (mathtext.sty входить в пакет t2).
\usepackage[T1,T2A]{fontenc}         % внутрішнє кодування шрифтів (може бути декілька);
                                          % вказане останнім діє по замовчуванню;
                                          % кириличне має співпадати з заданим в ukrhyph.tex.
\usepackage[utf8]{inputenc}       % кодування документа; замість cp866nav
                                          % може бути cp1251, koi8-u, macukr, iso88595, utf8.
\usepackage[english,ukrainian]{babel} % національна локалізація; може бути декілька
                                          % мов; остання з переліку діє по замовчуванню. 

\usepackage{sectsty}   %in order to make chapter headings and title centered
\chapterfont{\centering}

\usepackage{amsthm}
\usepackage{amsmath}
\usepackage{amssymb}
\usepackage{amsfonts}
\usepackage{graphicx}
\usepackage[pdftex]{hyperref}
\usepackage{caption}
\usepackage{subfig}
\usepackage{fancyhdr}
\usepackage{enumerate}
\usepackage{enumitem}

%custom commands to save typing
\newcommand{\mynorm}[1]{\left|\left|#1\right|\right|}
\newcommand{\myabs}[1]{\left|#1\right|}
\newcommand{\myset}[1]{\left\{#1\right\}}

%put subscript under lim and others
\let\oldlim\lim
\renewcommand{\lim}{\displaystyle\oldlim}
\let\oldmin\min
\renewcommand{\min}{\displaystyle\oldmin}
\let\oldmax\max
\renewcommand{\max}{\displaystyle\oldmax}

\newtheorem{prob}{Завдання}

\title{
Асимптотичні методи в теорії диференційних рівнянь\\
Контрольна робота (8 семестр)\\
No. 7}
\begin{document}
\maketitle
\begin{prob}Знайти розв’язок, що задовольняє задані початкові умови, у вигляді степеневого ряду. Обчислити декілька перших
коефіцієнтів ряду (до коефіцієнту $x^4$ включно).\end{prob}%TODO: prove smoothness
Для початку, оскільки права частина гладка, існує розв’язок задачі Коші $y_0$, визначений принаймні в околі одиниці. Оскільки таке $y_0$ буде
задовольнять рівнянню, воно автоматично буде диференційованим на якомусь околі. Більше того, його похідна ($y_0'=y_0^2+x$) є композицією
диференційованих функцій ($y_0(x)$ диференційоване, як ми вже показали), що показує, що $y_0'$ також є диференційованою. Відповідно,
$y_0''=2y_0y_0'+1$ є композицією диференційованих функцій, тому $y_0''$ також диференційоване і так далі. Відповідно, $y_0$ можна вважати
диференційованою довільну наперед задану кількість разів (на відповідному достатньо малому околі одиниці). Відповідно, до неї застосовується
формула Тейлора
\[y(x)=c_1(x-1)+c_2(x-1)^2+c_3(x-1)^3+c_4(x-1)^4+o((x-1)^4)\]
Підставимо це в наше рівняння і отримаємо
\[c_1+2c_2(x-1)+3c_3(x-1)^2+4c_4(x-1)^3+5c_5(x-1)^4+o((x-1)^4)=\]
\[=(c_1(x-1)+c_2(x-1)^2+c_3(x-1)^3+c_4(x-1)^4+o((x-1)^4))^2+x\]
Розкриваємо дужки
\[c_1+2c_2(x-1)+3c_3(x-1)^2+4c_4(x-1)^3+5c_5(x-1)^4=\]\[=c_1^2(x-1)^2+c_2^2(x-1)^4+2c_1c_2(x-1)^3+2c_1c_3(x-1)^4
+(x-1)+1+o((x-1)^4)\]
а отже прирівнюючи коефіцієнти маємо 
\[\begin{cases}
	c_1=1\\
	2c_2=1\implies c_2=\frac{1}{2}\\
	3c_3=c_1^2\implies c_3=\frac{1}{3}\\
	4c_4=2c_1c_2\implies c_4=\frac{1}{4}\\
	5c_5=c_2^2+2c_1c_3
\end{cases}\]
і тому
\[y(x)=(x-1)+\frac{(x-1)^2}{2}+\frac{(x-1)^3}{3}+\frac{(x-1)^4}{4}+o((x-1)^4)\]

\begin{prob}Використовуючи інтегрування за допомогою рядів, знайти розв’язок рівняння в околі точки  $x_0=0$. Там, де це можливо,%no elem talk
виразити розв’язок через елементарні функції. Знайти область збіжності одержаних рядів. Знайти загальний розв’язок рівняння.\end{prob}
Використовуючи теорему 2 з конспекту ($p(x)=-3-x$ і $q(x)=4$), розв’язок починаємо з характеристичного рівняння
\[\lambda(\lambda-1)-3\lambda +4=0\implies \lambda=2\mbox{ подвійний корінь}\]
Таким чином, один із розв’язків можна записати у вигляді
\[y_2(x)=\alpha y_1(x)\ln x+\sum_{k=0}^{\infty}b_kx^{k+2} \]
де $y_1(x)=\sum_{k=0}^{\infty}c_kx^{k+2}$ розв’язок рівняння. Підставимо його в рівняння
	\[x^2\sum_{k=0}^{\infty}c_k(k+2)(k+1)x^{k+2-2}-(3+x)x\sum_{k=0}^{\infty}c_k(k+2)x^{k+2-1}+
	4\sum_{k=0}^{\infty}c_kx^{k+2}=0\]
скоротимо на $x^2$ і отримаємо
\[\sum_{k=0}^{\infty}c_k(k+2)(k+1)x^k+\sum_{k=0}^{\infty}(-2-3k)c_kx^k-\sum_{k=1}^{\infty}c_{k-1}(k+1)x^k=0\]
Таким чином,
\[2c_0-2c_0=0\]
\[c_k\cdot k^2-c_{k-1}(k+1)=0\implies c_k=\frac{k+1}{k^2}c_{k-1}\implies c_k=\frac{k+1}{k!}\]
Цей ряд збіжний на усій числовій осі.
Таким чином,
\[y_1(x)=\sum_{k=0}^{\infty}\frac{k+1}{k!}x^k=e^x+\sum_{k=1}^{\infty}\frac{k}{k!}x^k=e^x+xe^x\]
Тепер щоб знайти $y_2$ ми підставимо $y_1(x)\ln x$ у рівняння.
\begin{prob}Знайти три члени розкладу розв’язку за степенями малого параметру $\mu$.\end{prob}
	Оскільки і права частина рівняння і початкові умови є гладкими в околі $(x_0,y_0,\mu_0)=(1,1,0)$, за теоремою
	про диференційованість за початковими даними та параметрами, можем припустити, що до розв’язка $y(x,\mu)$ можна застосувать
	формулу Тейлора $y(x,\mu)=y_0(x)+\mu y_1(x)+\mu^2y_2(x)+o(\mu^2)$. Підставляючи це в рівняння маємо
	\[y_0'(x)+\mu y_1'(x)+\mu^2y_2'(x)+o(\mu^2)=
	\frac{3}{y_0(x)+\mu y_1(x)+\mu^2y_2(x)+o(\mu^2)}-\mu x
	\]
	а також
	\[y_0(1)+\mu y_1(1)+\mu^2 y_2(1)+\dots=1+\mu\cdot 0+\mu^2\cdot0+\dots\]
	Таким чином
	\[(y_0'+y_1'\mu+y_2'\mu^2+o(\mu^2))(y_0+y_1\mu+y_2\mu^2+o(\mu^2))=3-\mu x(y_0+y_1\mu+y_2\mu^2+o(\mu^2))\]
	і отже
		\[y_0y_0'+(y_1y_0'+y_1'y_0)\mu+(y_2'y_0+y_1'y_1+y_2y_0')\mu^2=3-\mu xy_0-\mu^2xy_1+o(\mu^2)\]
	тепер ми прирівняємо коефіцієнти при $\mu$ і отримаєм
	\[\begin{cases}
		y_0'(x)=\frac{3}{y_0(x)}\implies y_0(x)=\sqrt{6x-5}\\
		y_1(x)y_0'(x)+y_1'(x)y_0(x)=-xy_0(x)\Rightarrow 3y_1(x)+y_1'(x)(6x-5)=-x(6x-5)\Rightarrow\\
		 \Rightarrow y_1(x)=-\frac{2}{5}x^2+\frac{x}{9}+\frac{5}{27}+\frac{14}{135}\frac{1}{\sqrt{6x-5}}\\
		y_2'(x)y_0(x)+y_1'(x)y_1(x)+y_2(x)y_0'(x)=-xy_1(x)
	\end{cases}
	\]
\end{document}
