\documentclass[8pt]{article} % use larger type; default would be 10pt

\usepackage[margin=1in]{geometry}
\usepackage{graphicx}
\usepackage{float}
\usepackage{subfig}
\usepackage{amsmath}
\usepackage{amsfonts}
\usepackage{hyperref}
\usepackage{enumitem}
\usepackage[neverdecrease]{paralist}

\usepackage{mystyle}
\newcommand{\dx}{\;dx}
\newcommand{\dy}{\;dy}
\newcommand{\dr}{\;d\rho}
\newcommand{\dph}{\;d\phi}

\title{Math 1540\\University Mathematics for Financial Studies\\2013-14 Term 1\\Suggested solutions for\\
Sec. 15.4--15.8}
\begin{document}
\maketitle
\section{Section 15.4}
\begin{description}
	\item[\# 22.]{{\it Change the Cartesian integral into an equivalent polar integral. Then evaluate the integral.}
		\[\int_1^2\int_0^{\sqrt{2x-x^2}}\frac{1}{(x^2+y^2)^2}\dx\dy\]
		Note, that the region of integration is the "triangle" with vertices $(1,0)$, $(2,0)$ and $(1,1)$. In polar 
		coordinates, they have their respective coordinates $(\phi,\rho)$ equal to $(0,1)$, $(0,2)$ and $(\pi/4,\sqrt{2})$
		respectively. The curves joining $(1,1)$ and $(2,0)$ has Cartesian equation $y=\sqrt{2x-x^2}$ and thus in
		polar coordinates can be written as $\rho=2\cos\phi$, while line joining $(1,0)$ and $(1,1)$ is
		$x=1$ in Cartesian coordinates, hence $\rho\cos\phi=1$ in polar. This allows us to rewrite integral as
		\[\int_1^2\int_0^{\sqrt{2x-x^2}}\frac{1}{(x^2+y^2)^2}\dx\dy=\int_0^{\pi/4}\int_{1/\cos\phi}^{2/\cos\phi}\frac{1}
		{\rho^4}\dr\dph=\int_0^{\pi/4}=\int_0^{\pi/4}\frac{\rho^{-3}}{-4}\bigg|_{1/\cos\phi}^{2/\cos\phi}\dph=\]
		\[=\frac{7}{32}\int_0^{\pi/4}\cos^3\phi\dph\]
		}
\end{description}
\end{document}
%Sec 15.4: #22, 25, 30
%Sec 15.5: #22, 29, 44
%Sec 15.7: #34, 47, 58
%Sec 15.8: #8, 16, 24
%hw->topo->bookmark
