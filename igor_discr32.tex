\documentclass[8pt]{article} % use larger type; default would be 10pt

%\usepackage[utf8]{inputenc} % set input encoding (not needed with XeLaTeX)
\usepackage[10pt]{type1ec}          % use only 10pt fonts
\usepackage[T1]{fontenc}
%\usepackage{CJK}
\usepackage{graphicx}
\usepackage{float}
\usepackage{CJKutf8}
\usepackage{subfig}
\usepackage{amsmath}
\usepackage{amsfonts}
\usepackage{hyperref}
\usepackage{enumerate}
\usepackage{enumitem}

\newcommand{\mynorm}[1]{\left|\left|#1\right|\right|}
\newcommand{\myabs}[1]{\left|#1\right|}
\newcommand{\myset}[1]{\left\{#1\right\}}
\let\oldsum\sum
\renewcommand*{\sum}{\displaystyle\oldsum}

\title{Problem 32}
\begin{document}
\maketitle
First, let $n=2m$ be even. Let us analyze the proof of Sperner Theorem, Thm. 5.3.3 in textbook. As we are looking for an antichain of size $\binom{n}{m}$, all inequalities should become equalities in proof of Sperner Theorem. In 
particular, $\sum_{k=0}^n \alpha_k/\binom{n}{k}=1$ and $\sum_{k=0}^n \alpha_{k}/\binom{n}{m}=1$. Now, recall that for $n=2m$ we have
$\binom{n}{k}<\binom{n}{m}$, unless $k=m$. Thus, if $\alpha_k>0,\;k\neq m$, we would have
$\alpha_k/\binom{n}{k}<\alpha_k/\binom{n}{m}$, which would imply $1=\sum_{k=0}^n \alpha_k/\binom{n}{k}>\sum_{k=0}^n \alpha_{k}/\binom{n}{m}=1$, contradiction. Therefore, an antichain of size $\binom{n}{m}$ (if exists) cannot contain
sets of any size, other than $m$. As the number of sets of size $m$ is $\binom{n}{m}$, and size of antichain is $\binom{n}{m}$ as well, it therefore have no other option that contain all sets of size $m$ and only them. Finally, it
the family of all sets of size $m$ is antichain.

Now, let $n=2m-1$ be odd. The reasonoting is similar to that of above, except that now $\binom{n}{k}<\binom{n}{m}$, unless $k=m$ or $k=m-1$. Therefore, antichain of size $\binom{n}{m}$ can contain only sets of size $m$ or $m-1$.
To finish the proof, we just need to show that no antichain of size $\binom{n}{m}$ can contain both sets of size $m$ and $m-1$ at the same time, for families consisting of all the sets of size $m$ or of all the sets of size $m-1$ are
antichains.

Therefore, assume $\mathcal{A}$ is an antichain of size, $\binom{n}{m}$, $X,Y\in\mathcal{A}$ and $\myabs{X}=m$, whereas $\myabs{Y}=m-1$. As in a first sentence of proof of Sperner's theorem, we should have $\beta=n!$,
which means that every maximal chain occurs as second element of pair $(A,C)$, where $A\in\mathcal{A},\;A\in C$. To get contradiction, we will build a chain that does not contain any of sets in $\mathcal{A}$. Without loss of
generality, we may assume that $\myabs{X\cap Y}$ is maximal possible. As $X$ and $Y$ are not comparable, $Y\cap X\neq Y\implies s<m-1$. Therefore, $Y\setminus \left(Y\cap X\right)
\neq \emptyset\implies \exists y_0\in Y,\;y\notin Y\cap X$.
Now, take $Y' \subset X$, such that $\myabs{Y'}=m-1,\;Y' \subset X\cap Y$. Note, that as $Y'\subset X,\;Y'\notin\mathcal{A}$ (for $Y'$ and $X$ are comparable). Finally, take $X':=Y'\cup\left\{y_0\right\}\implies\myabs{X'}=m$. Similarly,
as $X'\cap Y=\left(X\cap Y\right)\cup \{y_0\}\implies \myabs{X'\cap Y}>\myabs{X\cap Y}\implies X'\notin\mathcal{A}$ (otherwise, this would contradict our maximality assumption on $\myabs{X\cap Y}$. Thus, $Y'\subset X'$ and neither
of them is member of $\mathcal{A}$. Hence, as $\mathcal{A}$ contain only sets of size $m$ and $m-1$, maximal chain containing $Y',\;X'$ would not intersect $\mathcal{A}$ (otherwise, it would contain two sets of size $m$ or two of
size $m-1$). This contradiction ends the proof.
\end{document}
