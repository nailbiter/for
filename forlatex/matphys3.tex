\documentclass[12pt]{article} % use larger type; default would be 10pt

\usepackage{mathtext}                 % підключення кирилиці у математичних формулах
                                          % (mathtext.sty входить в пакет t2).
\usepackage[T1,T2A]{fontenc}         % внутрішнє кодування шрифтів (може бути декілька);
                                          % вказане останнім діє по замовчуванню;
                                          % кириличне має співпадати з заданим в ukrhyph.tex.
\usepackage[utf8]{inputenc}       % кодування документа; замість cp866nav
                                          % може бути cp1251, koi8-u, macukr, iso88595, utf8.
\usepackage[english,russian,ukrainian]{babel} % національна локалізація; може бути декілька
                                          % мов; остання з переліку діє по замовчуванню. 
\usepackage{mystyle}

\newtheorem{prob}{Завдання}
\newcommand{\ds}{\;ds}
\newcommand{\dt}{\;dt}
\newcommand{\dx}{\;dx}
\newcommand{\dta}{\;d\tau}
\newcommand{\extr}{\mbox{\normalfont extr}}

\newtheorem{myulem}[mythm]{Лема}

\renewenvironment{myproof}[1][Доведення]{\begin{trivlist}
\item[\hskip \labelsep {\bfseries #1}]}{\myqed\end{trivlist}}

\title{Рівняння математичної фізики (10 семестр)}
\author{Олексій Леонтьєв}

\begin{document}
\def\dx{\Delta x}
\def\dt{\Delta t}
\def\dX{\frac{\partial}{\partial x}}
\maketitle
\begin{prob}Звести до канонічного вигляду запропоноване рівняння
	\[u_{xx}-4u_{xy}+2u{xz}+4u_{yy}+u_{zz}=0\]
\end{prob}
Ця задача еквівалентна задачі приведення квадратичної форми до канонічного виду. Маємо
\[x^2-4xy+2xz+4y^2+z^2=(x-2y+z)^2+4yz=(x-2y+z)^2+(y+z)^2-(y-z)^2\]
і таким чином, це рівняння гіперболічного типу.
\setcounter{prob}{6}
\begin{prob}Використовуючи формулу д’Аламбера, знайти розв’язок задачі Коші для одновимірного хвильового рівняння
	\[u_{tt}=4u_{xx},\qquad\mybra{x\in\mathbb{R},\;t>0};\qquad u(x,0)=x^2;\quad u_t(x,0)=x.\]
\end{prob}
За формулою д’Аламбера, для $\phi(x):=x^2,\;\psi(x):=x,\;a=2$ маємо
\[u(t,x)=\frac{\phi(x-at)+\phi(x+at)}{2}+\frac{1}{2a}\int_{x-at}^{x+at}\psi(z)\;dz=\]
\[=\frac{(x-2t)^2+(x+2t)^2}{2}+\frac{1}{4}\int_{x-2t}^{x+2t}z\;dz=x^2+4t^2+{xt}\]
\setcounter{prob}{8}
\begin{prob}Методом характеристик знайти розв’язок задачі Коші для одновимірного хвильового рівняння
	\[u_{tt}=u_{xx}+\sin x\qquad(x\in\mathbb{R},\;t>0);\qquad u(x,0)=\sin x;\quad u_t(x,0)=0.\]
\end{prob}
Зробимо заміну
\[\xi=t+x,\;\eta=t-x\]
матимемо
\[4u_{\xi\eta}=\sin\mybra{\frac{\xi-\eta}{2}}=\sin\mybra{\frac{\xi}{2}}\cos\mybra{\frac{\eta}{2}}-\sin\mybra{\frac{\eta}{2}}\cos\mybra{
\frac{\xi}{2}}\]
інтегруючи маємо
\[4u_\xi=\sin\mybra{\frac{\xi}{2}}2\sin\mybra{\frac{\eta}{2}}+2\cos\mybra{\frac{\eta}{2}}\cos\mybra{\frac{\xi}{2}}+C(\xi)\]
або еквівалентно
\[2u_\xi=\sin\mybra{\frac{\xi}{2}}\sin\mybra{\frac{\eta}{2}}+\cos\mybra{\frac{\eta}{2}}\cos\mybra{\frac{\xi}{2}}+C(\xi)\]
\[u=-\cos\mybra{\frac{\xi}{2}}\sin\mybra{\frac{\eta}{2}}+\cos\mybra{\frac{\eta}{2}}\sin\mybra{\frac{\xi}{2}}+C(\xi)+D(\eta)\]
граничні умови дають нам
\[-\cos\mybra{\frac{x}{2}}\sin\mybra{\frac{-x}{2}}+\cos\mybra{\frac{-x}{2}}\sin\mybra{\frac{x}{2}}+C(x)+D(-x)=\sin x\implies
C(x)+D(-x)=0\]
\[C'(x)+D'(-x)=0\]
і таким чином, можемо вважати $C=D=0$ а розв’язок записується як
\[u(x,y)=-\cos\mybra{\frac{t+x}{2}}\sin\mybra{\frac{t-x}{2}}+\cos\mybra{\frac{t-x}{2}}\sin\mybra{\frac{t+x}{2}}=\sin x\]
\setcounter{prob}{11}
\begin{prob}Розв’язати методом Фур’є мішану задачу
	\[
	\begin{array}{cl}
		u_t=u_{xx}, &x\in(0,1)\;t>0;\\
		u_x(0,t)=u(1,t)=0,&t>0;\\
		u(x,0)=\cos\frac{\pi x}{2}+\cos\frac{3\pi x}{2},&x\in(0,1).
	\end{array}
	\]
\end{prob}
	Будемо шукати розв’язок у вигляді
	\[u(x,t)=T(t)X(x)\neq0\]
	тоді перша і друга умови задачі запишуться як
	\[T'+\lambda T=0\]
	\[X''+\lambda X=0\implies X=\alpha\cos\lambda x+\beta\sin\lambda x\]
	\[X'(0)=0\implies\beta=0\]
	\[X(1)=0\implies\cos\lambda=0\implies\lambda=\frac{\pi}{2}+\pi k\]
	і відповідно
	\[u(x,t)=\sum_{n=1}^\infty a_ne^{-(\pi n+\pi/2)^2t}\cos(\pi n+\pi/2)x\]
	і враховуючи третю умову задачі
	\[u(x,t)=e^{-(\pi/2)^2t}\cos\frac{\pi x}{2}+e^{-(3\pi/2)^2t}\cos\frac{3\pi x}{2}\]
\end{document}
