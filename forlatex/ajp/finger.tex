%        File: \cygwin\home\AJP\for\forlatex\ajp\finger.tex
%     Created: Tue May 09 01:00 PM 2017 ���� (�W����)
% Last Change: Tue May 09 01:00 PM 2017 ���� (�W����)
%
\documentclass[a4paper]{article}
%%\usepackage[T1,T2A]{fontenc}
%%\usepackage[utf8]{inputenc}
%%\usepackage[english,ukrainian]{babel}
\usepackage{enumerate}
\usepackage{xeCJK}
\usepackage{ruby}
\usepackage[]{amsmath}

\newcommand{\tname}{{\ttfamily AnimeSetName.xlsl}}

%\renewcommand\rubysep{-4ex}
\newcommand{\kana}[2]{\ruby{#1}{#2}}
\newcommand{\fbxname}{{\ttfamily plant.fbx}}
\newcommand{\name}{ {\ttfamily ''plant''}}
\setCJKmainfont{MS PGothic} %AJP windows
%\setCJKmainfont[AutoFakeBold=true]{Hiragino Mincho Pro} %my Mac
\title{\#188 アニメーションがモデルの手に反映されない}
\begin{document}
\maketitle
{\ttfamily Assets/Resources/Models/Charactor/}、{\ttfamily Assets/Resources/Models/Animation/Motion/}
と{\ttfamily Assets/Resources/Prefabs/Debug/original model/}に入っている全てのFBXに対する以下のようなことをする:
\begin{enumerate}
	\item Inspectorタブ上で、RigでConfigure\dots を押す
	\item Left Handを押して、{\ttfamily l\_finger}を絵の{\ttfamily Middle Proximal}にドラッグする
	\item Right Handを押して、{\ttfamily r\_finger}を絵の{\ttfamily Middle Proximal}にドラッグする
	\item Doneを押す; ポップアップが出てきたら、これにApplyを押す
\end{enumerate}
全部終わったらアニメーターを再設定し(MakeAnimator>Build)、とAssetBundleを更新すること(AssetBundles>BuildAssetBundles)。
\end{document}
