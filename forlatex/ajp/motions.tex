%        File: \cygwin\home\AJP\for\forlatex\ajp\motions.tex
%     Created: Fri Apr 21 01:00 PM 2017 ???? (?W????)
% Last Change: Fri Apr 21 01:00 PM 2017 ???? (?W????)
%
\documentclass[a4paper]{report}
%%\usepackage[T1,T2A]{fontenc}
%%\usepackage[utf8]{inputenc}
%%\usepackage[english,ukrainian]{babel}
\usepackage{enumerate}
\usepackage{xeCJK}
\usepackage{ruby}
\usepackage[]{amsmath}

\newcommand{\tname}{{\ttfamily AnimeSetName.xlsl}}

%\renewcommand\rubysep{-4ex}
\newcommand{\kana}[2]{\ruby{#1}{#2}}
\newcommand{\fbxname}{{\ttfamily plant.fbx}}
\newcommand{\name}{ {\ttfamily ''plant''}}
\setCJKmainfont{MS PGothic} %AJP windows
%\setCJKmainfont[AutoFakeBold=true]{Hiragino Mincho Pro} %my Mac
\title{モーションデータの追加ための手続き}
\begin{document}
\maketitle
追加されている動作について以下のような仮定する:
\begin{itemize}
	\item この動作が{\ttfamily AnimeSetName.xlsl}表に載っている
	\item 追加されている動作で道具が\textbf{使われていない}\footnote{道具が使う動作の手続き次のバーションで記述する}
\end{itemize}
例のため、追加されている動作に対応する\tname のデーテが以下のようする:
\begin{itemize}
	\item {\ttfamily name=''plant''}
	\item {\ttfamily ID=035}
\end{itemize}
\begin{enumerate}
	\item \tname で追加される動作に対応するの行で:\begin{itemize}
			\item u8 miscDebugを1にする
			\item u8hideOnEndを0にする
		\end{itemize}
	\item Mayaファイルから{\ttfamily plant.fbx}へ書き出す\begin{enumerate}
			\item サバーの{\ttfamily /Public/STONE2/design/受け渡し/STN2\_物量リスト(Steam,iOS)\_20170417.xlsx}を参考して、
				サバーの{\ttfamily /Public/STONE2/model/motion\_ma}で適当なMayaファイルを見つける
			\item Mayaで\begin{equation*}
					\mbox{\ttfamily File}\to\mbox{\ttfamily Export All\dots}
				\end{equation*}をクリックする
			\item 書き出す場所を
				\begin{equation*}
					\kern-5cm\mbox{\ttfamily
C:/cygwin/home/AJP/stone2/Develop/Projects/Stone/Assets/Resources/Prefabs/Debug/original model/anima}
				\end{equation*}にする
			\item 書き出すファイルの名前\fbxname にする
		\end{enumerate}
	\item 読み込まれた \fbxname を Unityで設定する:
	\begin{enumerate}
		\item 読み込まれた\fbxname をUnityで選択する
		\item Inspector窓で以下のような設定する:
			\begin{enumerate}
				\item \underline{Rig}: Animation TypeをHumanoidにする; {下右のApplyをクリックして忘れず!}
				\item Left Handを押して、{\ttfamily l\_finger}を絵の{\ttfamily Middle Proximal}にドラッグする
				\item Right Handを押して、{\ttfamily r\_finger}を絵の{\ttfamily Middle Proximal}にドラッグする
				\item \underline{Animation}:
					\begin{enumerate}
						\item Animationの名前をplantにする
						\item Root Transform RotationでBake Into Poseをチェックする
						\item Root Transform Position (Y)で:\begin{itemize}
								\item Bake Into Poseをチェックする
								\item Based Upon (at Start)でFeetを選択する
							\end{itemize}
					\end{enumerate}
			\end{enumerate}
	\end{enumerate}
\item \fbxname のアニメーションをアニメータに追加する:\begin{enumerate}
		\item {\ttfamily C:/cygwin/home/AJP/stone2/Develop/Projects/Stone/Assets/}\\{\ttfamily Resources/Prefabs/Debug/animator}をUnityで選択する
		\item \fbxname をAnimator窓へドラッグする
		\item Animator窓で新しいステートが出てきて、このステートの名前が\name と等しくと確認する
		\item 新しいステートにクリックして、Inspector窓で\begin{equation*}
				\mbox{\ttfamily Add Behaviour}\to\mbox{\ttfamily MotionSE}
			\end{equation*}
			をクリックする
		\item Animatorで新しい\name というトリガを追加する
		\item Any Stateから新しい \name ステートへトランジションを追加する
			\begin{itemize}
				\item このトランジションのConditionに\name トリガを追加する
			\end{itemize}
		\item 新しい \name ステートからidle へトランジションを追加する
	\end{enumerate}
\end{enumerate}
\end{document}


