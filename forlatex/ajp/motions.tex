%        File: \cygwin\home\AJP\for\forlatex\ajp\motions.tex
%     Created: Fri Apr 21 01:00 PM 2017 ???? (?W????)
% Last Change: Fri Apr 21 01:00 PM 2017 ???? (?W????)
%
\documentclass[a4paper]{report}
%%\usepackage[T1,T2A]{fontenc}
%%\usepackage[utf8]{inputenc}
%%\usepackage[english,ukrainian]{babel}
\usepackage{enumerate}
\usepackage{xeCJK}
\usepackage{ruby}

\newcommand{\tname}{{\ttfamily AnimeSetName.xlsl}}

%\renewcommand\rubysep{-4ex}
\newcommand{\kana}[2]{\ruby{#1}{#2}}
\newcommand{\fbxname}{\ttfamily plant.fbx}
\setCJKmainfont{MS PGothic} %AJP windows
%\setCJKmainfont[AutoFakeBold=true]{Hiragino Mincho Pro} %my Mac
\title{モーションデータの追加ための手続き}
\begin{document}
\maketitle
追加されている動作について以下のような仮定する:
\begin{itemize}
	\item この動作が{\ttfamily AnimeSetName.xlsl}表に載っている
	\item 追加されている動作で道具が\textbf{使われていない}\footnote{道具が使う動作の手続き次のバーションで記述する}
\end{itemize}
例のため、追加されている動作に対応する\tname のデーテが以下のようする:
\begin{itemize}
	\item {\ttfamily name=''plant''}
	\item {\ttfamily ID=035}
\end{itemize}
\begin{enumerate}
	\item 
	\item Mayaファイルから{\ttfamily plant.fbx}へ書き出す\begin{enumerate}
			\item \fbxname を {\ttfamily 
				C:/cygwin/home/AJP/stone2/Develop/Projects/Stone/}\\{\ttfamily Assets/Resources/Prefabs/Debug/original model/anima
				}に置く
		\end{enumerate}
\end{enumerate}
\end{document}


