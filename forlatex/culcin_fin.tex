\documentclass[10pt]{article}
\usepackage{fontspec}
\usepackage{array, xcolor, lipsum, bibentry}
\usepackage[margin=3cm]{geometry}
\usepackage{sectsty} % Allows changing the font options for sections in a document
\usepackage{hyperref}
\usepackage{xeCJK}

 
\title{\bfseries\Huge Oleksii Leontiev}
\author{inp9822058@cs.nctu.edu.tw}
\date{}
 
\definecolor{lightgray}{gray}{0.8}
\newcolumntype{L}{>{\raggedleft}p{0.2\textwidth}}
\newcolumntype{R}{p{0.8\textwidth}}
\newcommand\VRule{\color{lightgray}\vrule width 0.5pt}
 
%font configuration
\setCJKmainfont{AR PL KaitiM Big5}
\defaultfontfeatures{Mapping=tex-text}
\setromanfont[Ligatures={Common}, Numbers={OldStyle}, Variant=01]{Linux Libertine O} % Main text font
%\setromanfont[Ligatures={Common}, Numbers={OldStyle}, Variant=01]{Linux Biolinum Slanted} % Main text font
\sectionfont{\mdseries\upshape\Large} % Set font options for sections
\subsectionfont{\mdseries\scshape\normalsize} % Set font options for subsections
\subsubsectionfont{\mdseries\upshape\large} % Set font options for subsubsections
\chardef\&="E050 % Custom ampersand character
%TODO: no less than 2500 chars

\title{Literature and Films\\Final Report\\Caesar Must Die}
\author{歐立思\\9822058\\Department of Applied Mathematics}
 
\begin{document}
\maketitle
Paolo and Vittorio Taviani, who are among the best noted Italian film directors and screenwriters, are now in the beginning of their eighties. They began their careers as journalists, became interested in cinema after seeing
the war trilogy "Paisan"(1946) and then soon after making together short films, came to the cinema in sixties. Since then,
they've managed to do quite a remarkable path, going from the documentary films (which probably was suitable at the beginning to their background), then slowly shifting to the drama (so-called documentary drama).
To make an example, their film "Padre padrone"(1977), about the struggle of a Sardinian boy, who cannot complete the education due to the will of his domineering father in that patriarchal rural society, have been praised by the critics
internationally and received awards at the Berlin International Film Festival and at the Cannes Film Festival. Afterwards, brothers switched to the literary dramas. Unfortunately, they were unable to achieve the success of "Padre
Padrone" again. The Taviani brothers, despite remaining respected authors were quite out of the limelight in the end of 20th century. In the 2000 they seemingly returned back to the familiar to them realm of TV media, as they started
to successfully direct miniseries and films on TV. However, their 2012 film "Caesar Must Die" ("Cesare deve morire" being the original name in Italian),
made in the same genre as the triumphant "Padre padrone" (that is, closer to the documentary drama, than
to the literature adaptation, that brothers followed before) was once again a breakthrough.\\
Years ago, the two of brothers went to the Rebibbia high-security prison, in order to see the all-inmate performance of parts of a Dante Alighieri's "Inferno". It is known, that what they have seen inside the walls of Rebibbia made
them cry and wipe more than any performance in a professional theatre. Probably, it was difficult to find more appropriate setting for a trip to Hell, than it was in there. And presumably, no one (among the ones who we can find on
under the moon) can tell this tale better, than the killers, gangsters, drug dealers and mafiosi kept in there. Apparently, Taviani brothers, a professional film-makers, had realised at this point, that this material, despite being raw,
is fresh and can make a base for something innovative. These people know about death and pain far more than audience, perhaps even more than the authors of the dialogs they are performing did. These people spoken their own
rough dialect, badly suited to the elegant verse of Dante or Shakespeare, but their energy and nativeness did more work, than could be achieved by training sometimes. After all,
who on Earth can know more about the pain, than the murderer? Who can better tell about the tragedy of impossible love, than life sentenced? Who knows more about the war and treachery, than the real mafiosi and gangsters? 
Henceforth, it was decided that Paolo and Vittorio Taviani will shoot their next film behind the bars in the Rebibbia Prison.\\
%500 words
"Caesar Must Die" won the Golden Bear at the 62nd Berlin International Film Festival in February 2012, but what seems fresh and unusual to the audience often does not seem to be so to the much more sophisticated taste of a critic.
As a result, the choice of the judge committee (led by British director Mike Leigh) was not shared by many. "Cesare deve morire" was banned as "a very conservative selection" by Der Spiegel (German weekly news magazine, weekly 
circulation around one million), jury was accused in that they "shunned almost all the contemporary films that were admired or hotly debated at an otherwise pretty remarkable festival" by Der Tagesspiegel(classical liberal
German newspaper, weekly circulation around 148,000), film was called "a major upset" by the Hollywood Reporter (American weekly magazine, weekly circulation around 72,000). The film was being compared to other works of much younger
directors, such as "Tabu" - melodrama by the Portugese director Miguel Gomes. Difficult to say, who deserves the award more - the extremely talented and innovative young prodigy, or eighty-year old genteel film creators, who have been
out of (deserved) limelight, besides making interesting and humanistic cinema for a half of the century. It is probably always difficult to select among the great ones. Apart from the Golden Bear, "Caesar Must Die" was nominated as
an Italian entry for the Best Foreign Film Oscar, but unfortunately did not get the latter award in the end. However, it is still a mystery for me, why would such film could be accused in being "conservative". It is not political,
it does not explicitly refer to any of the acute problems of nowadays, but yet it would be a mistake to treat it as merely one more reinterpretation of a Shakespeare's masterpiece. Let us do not forget, that Taviani brothers have been
shooting political cinema for at least 20 years. Is that possible that they gave up this topic so easily? Taviani brothers, besides being elderly and worldly wise people, are well known as the masters of allusions and omissions.
It is not a coincidence that film is said to bring higher intensity to the peace. There is a remarkable tension under the surface of the film - the inmates obviously have do something to say about the inadequacy of life, unfair
political system, inhuman living conditions, grief for not be able to see relatives and outer world, yet Paolo and Vittorio Taviani do not give them the opportunity to give this off their chests. Instead they give us allusions. "Rome...
a city with no shame" says one prisoner in the first half of the movie, and adds meditatively after a few moments: "Excuse me, Fabio, but it seems to me as though this Shakespeare has lived on the streets of my city". 
"I don't know what I'll do if I don't get a single cell tomorrow morning." says other one in the middle, and adds decisively and desperately: "It's my right". "I want to confess, I want to speak" tell us prisoners in the night with
their voices melted together. This is not conservative, in my unprofessional opinion.\\
%500 words
It is particularly interesting, that exactly the Shakespearian play was chosen to be performed by all-inmates theatre. Clearly, almost any other
would suffice. Is that because Shakespeare is appreciated, well-known and relatively easy to play? Or there is some deeper logic behind directors'
choice? The plot of the film does not endow us with much clues. From the short theatre director's speech at the beginning of the film we may only
guess that play was chosen due to it being story about the Rome. Curious enough, most of my personal encounters with Shakespear's plays were
in the form similar to present, someone uses performing Shakespeare as part of another story. It may be impolite, but I think Shakespeare's plays
are pretty good at providing background to the main action, at the same time working as an axis to it. Long time ago, while preparing report for
the class, devoted to Shakespeare's plays, I've read somewhere a line that Shakespeare's plays may be so famous and live throughout the ages because
they contain "energy between the lines". Shakespeare plays somehow mystically interact with our unconscious, bypassing our mind. What is more
important, I think, is that Shakespeare's plays also charge surrounding action with this mystical energy. Another example of similar encounter
with Shakespeare for me was a play performed by Foreign Language Department of our university, "Six Characters in Search of an Author". Again, it may
be interesting to notice that that play and "Caesar must die" share quite some similarities. "Six Characters" also involves play rehearsal as
a part of action, and for the role of the inner play, the role for which virtually any other play would work (and indeed, "Six Characters" author
Luigi Pirandello chose another one), Shakespeare was chosen. Without making any speculations, I just want to illustrate that Shakespeare's plays
incidentally provide new dimension to action around and outside of them. Yet, "Julius Caesar" is violent and harsh as all other Shakespeare plays.
It does not get old - it talks about psychological issues of betrayal and consciousness and therefore is timeless. Being itself part of film,
Shakespeare's "Julius Caesar" melts surrounding reality, so the film essentially becomes not rehearsal, but sincere play inmates perform with their
own lives. Play melts surrounding settings, engulfing actors in its own reality and restoring itself to actors' reality, making the latter stage.
Indeed, this play about betrayal, envy and guilty (and not that much about Rome, after all -- Shakespeare is timeless) is performed by each
of us on a daily basis. Everybody can. As in the film the Italian mafioso and criminals declare that "will this glorious scene of ours be acted
over'\textbackslash\textbackslash In kingdoms, not yet born, in languages not yet invented", the Shakespeare on his own probably would be quite
surprised to see who is telling these words this time. 
However, these people in some way are best suited (to the point of being indigenous)
for their role - they not only live in Rome and are direct ancestors of Rome citizens, they are criminals and know about treason firsthand.\\
%500
Yes, as one of reviews said "The success of the Taviani's latest endeavour hinges upon the exceptionally powerful turns of its
non-professional cast, who only occasionally lift the veil behind which lie their actual personalities". The idea of filming not a professional
actors, but sincere inhabitants of Rebibbia high-security prison, added to the film. About the halfway through the film you notice that it's
not finishing play (ironically, shown at the beginning of the film) that entertains you the most, but rather rehearsals. It was mentioned, that
on of the most vivid and full of action scenes in the film is casting scene, where those willing to join the play crew should tell their particulars
first pretending to be in despair from separation, second pretending being pissed off. Besides being very fun, this five minute scene introduces
one observation, that is recurrent throughout the course of the film: they \textit{want} to play. Play for them is not something that they can 
easily go without, it is important for them. They are clearly happy hearing that all of them will be able to join, they spend a lot of their spare
time on rehearsals, they memorize text be hearts and are ready and happy to make avid improvisations. Why? Throughout the film we are constantly
led to think about how playing changes actor's identity, how play can influence actor. Art changes soul, and in contrast to other films criminals
(despite being real, again in contrast to other films) do not look dangerous. It takes us armored guardians and 
annotations cut in the film to suddenly realize, that
Salvatore Striano (depicting Brutus) is mafioso, Cosimo Rega (depicting Cassius) is murderer and Giovanni Arcuri (depicting Caesar)
is a drug dealer. Authors play a fair game, they do not use flashbacks too much (which certainly would be able to add to film a violence,
action or anything). Maybe, this mirrors the fact that actors are so engaged perhaps in order to forget about their own pain, that they certainly
don't want to pull above. They constantly live under the pressure of guiltiness, and look for escape in the space of play. From this particular
prospective, "Julius Caesar" is particularly bad choice - it is a play about guilty (among other things), it is full of it -- that's why Brutus
experiencing so hard times during rehearsals. These people want to become cleaner and better through acting as Noble Romans, through melting
in characters they play. But as they succeed in this, acting so sincerely and with engagement they inevitably interpret the story, saying
wrong lines (messengers of unconscious, thanks to Freud). "Brutus is inside of him". They want to act so much, that they even suppress their
own emotions and conflicts just not to ruin the project (as with Decius and Caesar). The real tragedy is that as they succeed in becoming
better and more cultured, as they come in touch with the art, they realize more sharply that they are not part of it, that they were excluded
from society, abandoned. And nobody to blame. My (not only, I believe) favourite piece of movie is the final one, where Rega says
"Ever since I became acquainted with art, this cell turned into a prison". This naturally crowns the film-long journey from the outer (final
performance) through the rehearsal and inmates' flashbacks to their real feelings: it simply does not help, pain only gets worse.\\
%500
Many of objections raised by critics after the arguable decision of judging panel on 62nd Berlin Festival could be reduced to saying that film
was too realistic, almost documentary. On the one hand, yes, film is realistic - Taviani brothers here fully take advantage of their 50 year
(with gaps) experience in making documentary films. After all, let's not forget that their first film back in 1960 was documentary 
"L'Italia non è un paese povero" ("Italy is not a poor country").
On the other hand, it is advisedly filmed in black and white, signature to the world of Big
And Real Art. If we however, will stop messing with taxonomy questions at this point, the interaction of life and art is one of the main themes
of the whole picture. Naturally, to explore this question the movie itself should be between the units of classification. Looking on this a bit
more closer, are art and life so clearly separated? Is that always that we can draw a line between them? Or is that they always together separated
by us rather by themselves, constantly imitating one another? Is that true that art imitates life, trying to build itself to resemble
the nature, thought as perfect model, as Aristotle claimed? Or in fact it is life that imitates art, while latter stipulates aesthetic
standards that life tries to adhere to, as Oscar Wilde proposed in 1889? This relations between life and art all seen well and from a fresh
prospective in this tape made by Italian directors. On the one hand, art imitates life, as Shakespeare's play is based on historical storyline --
the etalon of realism. The trick is that play, being a symbol, going through the centuries is attached to new meanings and starts bearing
its own life, often unrelated to the original source. This happens when we read the play. But when real people, acting as Actors, try
to put life into the play, they inevitable bring also something additional, the piece of themselves, deforming original idea, giving it new
direction and dimension, filling it with new energy. On the other hand, inmates in Rebibbia are very good example of how life tries to imitate
art -- they play because they want to become better and to escape terrifying reality around. With infinite surprise they discover another type of
human and etalon of human behaviour, that ironically enough, was not available to them outside the jail. Inhabitants of Italy high security jail
require art probably more than it was ever required by most fine-grained intellectuals: they cannot live without it. For them art is window in other
world, so different from cell. Catharsis is the way for them to clean their soul and forget for a moment (just for a moment, unfortunately)
about what they did and what they are sentenced for because of it. Finally, art changes people, as the closing credits show: upon leaving
the prison, members of theatrical crew write books about their experience and becoming an actors.\\
At the beginning of the previous paragraph it was noted, that whether it was planned or not, semi-documentary genre allowed Taviani brothers to
leverage their journalist and documentary movie directors experience. 
There is yet one more big layer of Taviani directing activity, that incidentally
also gets used to full extent in "Caesar must die": their experience as creators of a literary costume drama. The latter is a complicated
genre appreciated mostly by insiders, and it may be responsible for the fact that Taviani brothers were not noticed by critics for such a long time.
Indeed, Taviani brothers' first work in literary drama genre may probably be traced as far back as to 1971, with their 
"San Michele aveva un gallo" (1971, an adaptation of Tolstoy's novel "The Divine and the Human"), presenting some quite revolutionary ideas
and shifting well-known story depicted by Russian classic to the new realm -- political and existential conflict between utopian and scientific 
socialism. Their subsequent works were often derived from literature, such as well-received "Padre padrone" (1977, taken from a novel by Gavino
Ledda), beatiful "Kaos" (1984, based on short stories by Luigi Pirandello, incidentally, also an author of "Six Characters in Search of an
Author", mentioned above). Finally, their "Good morning Babylonia" (1987)
also has to do with theatre and cinema, but in a bit different way -- it depicts
story of two brothers, who work as set designers for a silent film. As we see from this, Taviani brothers have around 20 years of time to think
about relationships between cinema and theater. They have something to say. While theatre and cinema are often depicted as "cousins", theatre
certainly has smaller range of available technical means to entertain the public and in general is characterized as being more "static". It strives
to compensate for this with being more arguably more pure and natural. What it cannot achieve with technical means of cinema, it tries to get
with performance of human actors. As was briefly noticed above, Taviani brothers in principle could use wide range of cinema to make film far more
dynamic, and not only in principle - as creator of documentary video they indeed know how bring life to the screen. Instead, however, they chose
to film theatrical performance. This makes film more static obviously, it does not enjoy viewer with sudden changes of plan or location - everything
happens inside the relatively small space of prison. Broad technical arsenal of cinema is used merely for modest purpose of extending stage
size to that of the whole Rebibbia, dissolving stage in bigger space of prison.
But with acquiring the weakness of cinema, this film surprisingly enough acquires its main
strength as well -- expression of actors looks especially vivid and bold on such a pale background. In style reminiscent of Shakespeare, whose
"play within a play" in "A Midsummer Night's Dream" - "Pyramus and Thisbe", attracted viewers attention of amount comparable with "outer"
play itself, Taviani brothers play Shakespeare within their movie and use this activity as center of axis, to represent ideas related and unrelated
to original play.\\
To summarize, "Cesare deve morire" while at the first sight pale and not entertaining, brings together different topics, such as relationship
of cinema and theatre, relationship of live and art, Shakespeare plays and psychology of a convict and looks on them from different angle. It
entertains intelligent public not with blows on a screen, but rather with after-thoughts.
\end{document}
