%platex
%%--------------- Text starts from here ----------- %%

%%%%%%%%%%%%%%%2016年度FMSP Annual Report用スタイルファイル%%%%%%%%%%%%%%%%%%%%%%%%

%%%%%%%%%%%%%%%%%%%%%%%%%%%%%%%%%%%%%%%
%留学生などはこのレポートを英文で作成して結構です。
%This report can be written in English as well.
%%%%%%%%%%%%%%%%%%%%%%%%%%%%%%%%%%%%%%%

%%%%%%%%%%%%%%%%%%%%%%%%%%%%%%%%%%%%%%%
%このFormatはpLaTex を使用しています。
%
%以下に報告書の基本形が示してありますので、参考にしてお書き下さい。 
%数字は2桁以上は全て半角で書いて下さい。 
%文末の空白は必ず半角でお願いします。全角の空白は TeX では特殊文字と 
%判断して問題を起こすことがあります。
%数式はかならずmath mode でお願いいたします。 
%事務局では校正をせずにコンパイルしたファイルをそのまま印刷に出しますので、 
%一度コンパイルして、スペルチェック、校正は必ず行なって下さい。 
%まとめの編集の都合上、\newcommand, \renewcommand, \def の追加等はさけて下さいま
%すようお願いいたします。
%参考文献を入れる場合は \cite などは用いず、文献番号を手で入れて下さい。
%%%%%%%%%%%%%%%%%%%%%%%%%%%%%%%%%%%%%%%

%%%%%%%%%%%%%%%%%%%%%%%%%%%%%%%%%%%%%%%%%
%このファイルは \input によってまとめてコンパイルします。
%ファイル名は次の例のように
% 学年_lastname_firstname.tex
% として下さい。
%(例) m1_suri_taro.tex
%該当する事項がない部分は項目ごと削除して下さい。
%%%%%%%%%%%%%%%%%%%%%%%%%%%%%%%%%%%%%%%%%%


\documentclass[a4j,twocolumn]{jarticle}

\usepackage{amssymb,amsmath}
\textheight=25cm
\textwidth=15cm
\parskip=0mm
\parindent=0mm
\topmargin=-1cm
\oddsidemargin=5mm

\begin{document}

%%%%%%%%%%%%%%%%%%%%%%%%%%%%%%%%%%%%%%%%%%
%%%----------- CUT HERE -----------------------------------------------------------------
%%%%%%%%%%%%%%%%%%%%%%%%%%%%%%%%%%%%%%%%%%


%%%%%%%%%%%%%%%%%%%%%%%%%%%%%%%%%%%%%%%%%%%%%%%%%%%%%%%%%%%%%%%%%%%%%%%%%%%% 
% 氏名(ローマ字綴りで名字は全て大文字,名前は最初の字だけ大文字) 
% を書いて下さい. {\bf 数理 太郎 (SURI Taro)}
%%%%%%%%%%% 
%%%%%%%%%%%%%%%%%%%%%%%%%%%%%%%%%%%%%%%

{\bf  レオンチエフ・オレクシィ  (LEONTIEV Oleksii)}


%学振DC1,国費などの留学生などに採用されている人は記載して下さい.
%(例) 学振DC1
%%%%%%%%%%%%%%%%%%%%%%%%%%%%%%%%%%%%%%%%%%%%%%%%%%%%%%%%%%%%%%%%
FMSPコース生

%%%%%%%%%%%%%%%%%%%%%%%%%%%%%%%%%%%%%%%%%%%%%%%%%%%%%%%%%%%%%%%%%%%%%%%%%%% 
% 所属専攻名と学年を記入して下さい
%%%%%%%%%%%%%%%%%%%%%%%%%%%%%%%%%%%%%%%%%%%%%%%
%(例) 数理科学専攻 修士課程1年
%物理学専攻 博士課程1年
%%%%%%%%%%%%%%%%%%%%%%%%%%% 

{数理科学 }専攻 {博士 }課程{1 }年


\vspace{0.2cm}
\noindent
{\bf 研究概要}

\vspace{0.1cm}
%%%%%%%%%%%%%%%%%%%%%%%%%%%%%%%%%%%%%%%%%%%%%%%%%%%%%%%%%%%%%%%%%%%%%%%%%% 
%研究の要約を記入してして下さい.
%留学生の人などは英文でも結構です.
%コンパイルして0.5ページ以上2ページ以内程度になるようにまとめて下さい。
%%%%%%%%%%%%%%%%%%%%%%%%%%%%%%%%%%%%%%%%%%%%%%%%%%%%%%%%%%%%%%%%%%%%%%%%%% 
$G$ を Lie 群、$G'$ を $G$ の閉部分群とする。さらに、$(\pi,V)$と$(\tau,W)$を$G$と$G'$の表現とする。
その時、$V$から$W$へ$G'$-線形作用素は対称性破れ作用素と呼ばれる。特に、$\pi$が無
限次元で、$G'$ が非コンパクトの時、対称性破れ作用素の空間$\mbox{Hom}_{G'}(\pi\big|_{G'},\tau)$を
具体的に求めるという問題はかなり難しい。しかし最近、$O(n + 1, 1) \supset O(n, 1)$ という特別な場合
に、すべての対称性の破れ作用素が2014、2015年に小林俊行氏と B. Speh 氏によって
完全に分類された。これはその問題の完全な答えとして、一番最初である。\\
私の目的は小林俊行氏と B. Speh 氏によって発展された一般な手法によって、$(G, G') = (O(p+1,q),O(p,q))$
の場合の対称性の破れ作用素を研究するということであった。
具体的には、2016--2017学年で以下の問題を考えて、完全に答えた:\\
{\noindent}\textbf{問\textbf{1}.}{与えられた $( \lambda, \nu) \in
\mathbb{C}^2$
に対して、対称性の破れ作用素の空間 $\operatorname{Hom}_{G'}(I(\lambda),J(\nu))$ を具体的に求めよ。
特に、この空間の基底を具体的に求めよ。ここで、$I(\lambda):=C^{\infty}\left(  G\times_P\mathbb{C}_\lambda\right)$
と$J(\nu):=C^{\infty}\left( G'\times_{P'}\mathbb{C}_{\nu} \right)$は$G$と
$G'$の退化主系列である。}\\
今年は否定値直交群の対称性破れ作用素を更に研究して、次の結果を得た:
\begin{enumerate}
	\item 対称性破れ作用素の間の函数等式を得た;
	\item singularと微分対称性破れ作用素をregular対称性破れ作用素の留数として表している留数公式を得た;
	\item 対称性破れ作用素の像を計算した;
	\item 求退化主系列表現のある部分表現ファミーリの対称性破れ作用素の像を計算した;
	\item 求退化主系列表現として出て来るZuckerman導来函手加群間の$G'$-普遍写像を分類した;
\end{enumerate}

\vspace{0.2cm}
\noindent
{\bf 発表論文}

\vspace{0.1cm}
%%%%%%%%%%%%%%%%%%%%%%%%%%%%%%%%%%%%%%%%%%%%%%%%%%%%%%%%%%%%%%%%%%%%%%%%%%%%%% 
% プレプリントも含めて,大学院進学後に発表したものをすべて書いて下さい。
%プレプリントarchiveに投稿したものは番号を記載して下さい。
% 様式は以下の例のように
% 著者・共著者名・ 題名・ジャーナル名・巻・年・ページの順に書いて下さい.
% タイトルの前に著者・共著者名を入れる形です。
% 共著の場合は著者名をすべて書いて下さい。
%%%%%%%%%%%%%%%%%%%%%%%%%%%%%%%%%%%%%%%%%%%%%%%%%%%%%%%%%%%%%%%%%%%%%%%%%%%%% 
\begin{enumerate}
	\item[(1)] O. Leontiev and P. Feketa, ``A new criterion for the roughness of exponential dichotomy on $\mathbb{R}$''. {\it
		Miskolc Mathematical Notes}, 16(2): 987-994, 2015;
	\item[(2)] T. Kobayashi and O. Leontiev, ``Symmetry breaking operators for representations of indefinite orthogonal groups $O(p,q)$''. Symposium on Representation Theory 2016, pp. 39--52;
	\item[(3)] T. Kobayashi and O. Leontiev, ``Symmetry breaking operators for representations of indefinite orthogonal groups $O(p,q)$'', (submitted);
\end{enumerate}

%\begin{enumerate}
%\item G. van der Geer and T. Katsura,  On a stratification of 
%the moduli of K3 surfaces,
%J.\ Eur.\ Math.\ Soc. {\bf 2} (2000) 259--290.
%\end{enumerate}

\noindent
{\bf 学位論文}

\vspace{0.1cm}
%%%%%%%%%%%%%%%%%%%%%%%%%%%%%%%%%%%%%%%%%%%%%%%%%%%%%%%%%%%%%%%%%%%%%%%%%%%%%% 
% 今年度に修士または博士の学位論文を提出した人は
% 例のように論文タイトルを記載して下さい。
%%%%%%%%%%%%%%%%%%%%%%%%%%%%%%%%%%%%%%%%%%%%%%%%%%%%%%%%%%%%%%%%%%%%%%%%%%%%%%%

%<例>(修士論文) 正標数の代数曲面の cotangent 
% bundle のstability と Bogomolov の不等式
\begin{enumerate}
	\item (修士論文) O. Leontiev , ``Study of symmetry breaking operators of indefinite orthogonal groups $O(p,q)$'', 東京大学修士論文
		(2016).
\end{enumerate}


\vspace{0.2cm}
\noindent
{\bf 口頭発表}

\vspace{0.1cm}
%%%%%%%%%%%%%%%%%%%%%%%%%%%%%%%%%%%%%%%%%%%%%%%%%%%%%%%%%%%%%%%%%%%%%%%%%%%%%%%
% 大学院進学後に行なった研究発表について
% 昨年度以前のものも含めて
% タイトル・シンポジウム(またはセミナー等)名・場所・月・年を 
% 書いて下さい.国際会議の場合は国名をお願いします.タイトルは原題で.
%%%%%%%%%%%%%%%%%%%%%%%%%%%%%%%%%%%%%%%%%%%%%%%%%%%%%%%%%%%%%%%%%%%%%%%%%%%%%%% 
%\begin{enumerate}
%\item (1) 曲面の写像類群とは, (2) 写像類群をめぐるこれまでの結果と夢,
%Encounter with Mathematics 第11回, 中央大学理工学部,
%1999年4月.
%\end{enumerate}
\begin{enumerate}
	\item[(1)] 
2017年3月26日, 日本数学会 2017年度年会, 共形変換群 $O( p;q )$ に関する対称性破れ作用素 ,首都大学,東京 ;
	\item[(2)] 
2016年11月30日, Symposium on Representation Theory 2016, 不定値直交群 $O ( p; q )$ の対称性破れ作用素 , Grand Mer Resort, 沖縄;
	\item[(3)] 
2016年11月19日, 日本数学会 異分野・異業種研究交流会 2016, Symmetry breaking operators of indefinite orthogonal groups $O(p,q)$, 明治大学;
	\item[(4)] 
2016年10月7日, 広島幾何学研究集会2016, Symmetry breaking operators of indefinite orthogonal groups $O(p,q)$, 広島大学;
	\item[(5)] 
2016年9月18日, 日本数学会2016年度秋季総合分科会, 関西大学;
	\item[(6)] 
2016年8月11日, Workshop on “Actions of Reductive Groups and Global Analysis”, ”Discrete decomposability of the restriction of $A_q(\lambda)$ with respect to reductive subgroups and its applications (T. Kobayashi, Invent Math) の紹介, 東京大学 玉原国際セミナーハウス;
	\item[(7)] 
2016年7月19日, 広島大学幾何セミナー, 不定値直交群 O(p,q) の対称性破れ作用素, 広島大学;
%%(1) 曲面の写像類群とは, (2) 写像類群をめぐるこれまでの結果と夢,
%%Encounter with Mathematics 第11回, 中央大学理工学部,
%%1999年4,5月.
\end{enumerate}

\vspace{0.2cm}
\noindent
{\bf FMSPの活動への参加}

\vspace{0.1cm}
%%%%%%%%%%%%%%%%%%%%%%%%%%%%%%%%%%%%%%%%%%%%%%%%%%%%%%%%%%%%%%%%%%%%%%%%%%%%%% 
%本年度のFMSPが主催、共催する研究会、ワークショップ、FMSP Lectures、社会数理コロキウムなどへの参加を記入して下さい。また、それ以外に本年度FMSPから旅費等の補助を得て参加した国内外の研究会、セミナーなどの参加も記載して下さい。このような活動への参加によって何が得られたかを簡潔に書いて下さい。
%本年度スタディグループワークショップ、社会数理実践研究に参加した人
%はどのような課題を扱ったか、参加によって何が得られたか、自身がどのような貢献をしたかを書いて下さい。
%FMSPプログラムにおいて、長期海外渡航やインターンシップを行った
%場合は、2016年度だけではなく、すべて記載して下さい。また、それらについて、具体的にどのような研究活動を行ったか、どのような成果が得られたかをなるべく詳しく記述して下さい。
%%%%%%%%%%%%%%%%%%%%%%%%%%%%%%%%%%%%%%%%%%%%%%%%%%%%%%%%%%%%%%%%%%%%%%%%%%%%%%%

\begin{enumerate}
	\item [(1)] 2016年7月6日、NEC中央研究所見学、参考者;
	\item [(2)] 2017年2月22日、FMSP PO現地訪問、参考者;
\end{enumerate}
TODO

\vspace{0.2cm}
\noindent
{\bf 受賞}

\vspace{0.1cm}
%%%%%%%%%%%%%%%%%%%%%%%%%%%%%%%%%%%%%%%%%%%%%%%%%%%%%%%%%%%%%%%%%%%%%%%%%% 
% 修士課程進学以降にありましたら書いて下さい. 研究科長賞などを含みます.
% 受賞年度を記入して下さい。
%%%%%%%%%%%%%%%%%%%%%%%%%%%%%%%%%%%%%%%%%%%%%%%%%%%%%%%%%%%%%%%%%%%%%%%%
\begin{enumerate}
	\item[(1)] 平成27年度学生表彰「数理科学研究科長賞」;
\end{enumerate}

\vspace{0.4cm}

%%%%%%%%%%%%%%%%%%%%%%%%%%%%%%%%%%%%%%%%%%
%%%----------- CUT HERE -----------------------------------------------------------------
%%%%%%%%%%%%%%%%%%%%%%%%%%%%%%%%%%%%%%%%%%

\end{document} 
