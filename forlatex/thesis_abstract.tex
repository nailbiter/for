\documentclass[8pt]{article} % use larger type; default would be 10pt

%\usepackage[utf8]{inputenc} % set input encoding (not needed with XeLaTeX)
\usepackage[10pt]{type1ec}          % use only 10pt fonts
\usepackage[T1]{fontenc}
%\usepackage{CJK}
\usepackage{graphicx}
\usepackage{float}
\usepackage{CJKutf8}
\usepackage{subfig}
\usepackage{amsmath}
\usepackage{amssymb}
\usepackage{amsthm}
\usepackage{amsfonts}
\usepackage{hyperref}
\usepackage{enumerate}
\usepackage{enumitem}

%custom commands to save typing
\newcommand{\mynorm}[1]{\left|\left|#1\right|\right|}
\newcommand{\myabs}[1]{\left|#1\right|}
\newcommand{\myset}[1]{\left\{#1\right\}}

%put subscript under lim and others
\let\oldlim\lim
\renewcommand{\lim}{\displaystyle\oldlim}
\let\oldmin\min
\renewcommand{\min}{\displaystyle\oldmin}
\let\oldmax\max
\renewcommand{\max}{\displaystyle\oldmax}

\newtheorem*{prob}{Question}

\title{Exponentially dichotomous linear systems of differential equations}
\author{Author: Oleksii Leontiev}
\date{}
\begin{document}
\maketitle
The exponential dichotomy is a phenomenon which occurs in systems of linear homogeneous
differential equations $\dot{x}=A(t)x$ on $\mathbb{R}^n$. What it essentially means is that
phase space decomposes into two subspaces (hence the name "dichotomy"), and the solutions that start in one of these subspaces
very clear asymptotic behaviour: they grow exponentially as $t\to+\infty$ and decay exponentially as $t\to-\infty$, while those starting
in the other one exhibit opposite behaviour (grow as $t\to-\infty$ and decay as $t\to+\infty$).

Dichotomy thus is a natural extension of the idea of hyperbolicity to non-autonomous systems and the current work aimed at collecting together
in one place fundamental properties related to exponential dichotomy, that are currently spread in the literature around. Besides, there
was some work done in simplifying the proofs of results and expressing them in strict logical order to facilitate understanding. Unfortunately,
essentially no new results were found (besides several almost obvious remarks)
 and this work is mostly reviewing. However, the author did construct two equivalent criteria for dichotomy in simpler cases (such as when our
phase space is $\mathbb{R}^1$ or $A(t)$ does not change with time). This work is the author's first step in researching dichotomous systems.
\end{document}
