\documentclass[11pt]{book}
%\documentclass[8pt]{article} % use larger type; default would be 10pt

%\usepackage[utf8]{inputenc} % set input encoding (not needed with XeLaTeX)
\usepackage[10pt]{type1ec}          % use only 10pt fonts
\usepackage[T1]{fontenc}
%\usepackage{CJK}
\usepackage{graphicx}
\usepackage{float}
\usepackage{CJKutf8}
\usepackage{subfig}
\usepackage{amsmath}
\usepackage{amsfonts}
\usepackage{hyperref}
\title{Introduction to the modern Japan}
\begin{document}
\maketitle
\section*{Preface}
\subsection*{Part One}
In the spring of 2009 I was sitting in the train going from the Narita Airport to Tokyo. At the time when the train have entered the city, it was
already dark. In the twilight I have spotted another train that was carrying workers going home. The light in the train was quite bright and I
passenger have been either sitting or standing and looked very organized. For a moment I seemed to fall into the realm of Kenji Miyazawa's "Night
on the Galactic Railroad" - in the middle of the night two trains are slowly leaving towards the Milky Way. In the orange light going through the windows, the passengers in the other train have been peeling apples,
joking. And Miyazawa's Giovanni met with injured children from the Titanic in a train...\\
Saying about "Night on the Galactic Railroad", I have prepared an another book: "The Cultural History of Japan - Conflict and Agreement" (the name is tentative) to discuss this. It is worth to mention, that the railroad is
the symbol of Japan's modernization and Japanese began to rule Manchuria exactly by building a railroad. Secondly, whether he reads the book, or writing a message on a mobile phone, neatly clothed Japanese always silently staying or
sitting, and talk only a bit. Also, Japanese trains and trams are very precise in their timing, which is extremely rare in Asia.\\
Starting from the Meiji Restoration there is a strong wave of rapid Westernization in Japan. But while for the other Asian countries this time
can be said to bring the beginning of setbacks and humiliation, lost wars, loss of territory, compensation payment, becoming colonies and
massacres, Japan became rich and built its military power, thus becoming a nightmare for each other Asian country. The influence of Meiji period
on Japanese far exceeds their own understanding. For example, there is a lot of things that are considered "traditional" by insightful people,
yet were established as late as during the Meiji. For example, the requirement that only the male can be the Emperor, the association of emperror's
rule with the name of the period in history, same surname shared by husband and wife etc. After the end of the World War II even despite the
disadvantageous peace treaty imposed by USA, Japan has unexpectedly transformed into strong economic power (in fact, into strong military power
as well). At that time all over Asian countries have said again: Japanese make invasion by the means of economics.\\
The period when I have been studying in  Japan (1977-1984), it has already passed through the peak of economic growth and was coming into the 
bluffing period of a bubble economics. At that time Ezra V. Fogel has already declared: "Japan is number one" (1979),
and many Japanese also were full
of self-confidence than they have been thinking about their country. They thought Japan is already staying on the top of the world. Wang Chenchih
have broken the world home run record (1977), adventurer Naomi Uemura reached the North Polo on a dog sled solo (1978), the Disneyland in Tokyo
started its operation and NHK broadcaster aired television drama "Oshin" (1983). In the entertainment industry there were
Pink Lady's "UFO", the retirement of Momoe Yamaguchi, debut of Seiko Matsuda, Ken Shimura joined the comedy show "Hachiji Da Yo! Zen'in Shugo!"
(\textit{jap.}
\begin{CJK*}{UTF8}{min}"8時だョ!全員集合"\end{CJK*}
) etc. In arts and culture there were Hiromatsu Wataru's "Overcoming Modernity" thesis (\textit{jap.}
\begin{CJK*}{UTF8}{min}"近代の超克"\end{CJK*}) (1980), Haruki Murakami's "Hear the Wind Sing" (1978), "Pinball, 1973" (1980), "A Wild Sheep Chase"
(1982) were published. What about anime and manga, there were "Gundam Series" (serial TV show at 1979-1980 and movie version in 1981) 
released, Hayao Miyazaki's "Nausica\"a of the Valley of the Wind" (manga) series started (1982), and not to forget about Akio Nakamori
who published his research about otaku - this is generally considered to be the beginning of modern colloquial usage of the word "otaku".\\
Now, in the 21st century, Japan is suddenly suffering from both internal and external problems. Japan has big internal debt, high unemployment
rate did not decrease, contrary to expectations, Japan Airlines declared bankruptcy, after long-period rule Liberal Democratic Party gave up to
the Democratic Party. Externally Japan faced challenge from the Chinese factories, Korean cars, LCD TV, soap operas, movies etc. All of these
brought upon serious threat. Besides there were also rise of Indonesia etc. Therefore, today's Japan does not have even a single point, that would
be worth to mention, isn't it?\\
As it was said above, which country in Asia has such precise trains and trams, who's precision even makes detective novelists even relate train
schedules with killings? Streets in Japan are clean, citizens obey traffic regulations, pedestrians are dressed clean and tidy. Except of the famous
for its strict laws city-country Singapore, where else in Asia can you find a similar place? Number of Japanese Nobel award winners (15 people)
is also difficult to expected to be surpassed by any other Asian country. And there is one more thing that is worth to mention in the context
of countries in Asia. In Japan, well known as a patriarchal country, there is an outstanding number of woman-writers, to name a few: "Oshin"'s 
creator Sugako Hashida, "The Hospital"'s creator Toyoko Yamasaki, "Freezing Point"'s creator Ayako Miura, Kuniko Muk\=oda, Sawako Ariyoshi,
Ayako Sono, Fumiko Enchi, Seiko Tanabe, Kazuki Sakuraba, Jakucho Setouchi, Mariko Hayashi, Yoshimoto Banana, "Chibi Maruko-chan"'s creator
Sakura Momoko, Miyabe Miyuki, "Nodame Cantabile"'s creator Tomoko Ninomiya etc., all of them are appreciated woman-writers.
\subsection*{Part Two}
This book can be roughly divided into four parts: Japan prior to the World War II, Japanese society, economics and politics. In fact, the 
structure of the main thesis of this book could be ready as early as ten years ago, but as the book claims to reflect the "present potential"
of Japan, which changes dramatically and moreover has a wide range, requires substantial revision each year. A considerable amount of time and
energy should be spent solely on information update and supplement. Hence, it was difficult to publish the book immediately, most of the time
was spent considering the format in which information should be presented to the readers.\\
This sort of books have never been abundant in Taiwan. After all, if there is no long-term concerns about detailed investigation and analysis
of Japan, we may not hope to get the comprehensive understanding of Japanese society, politics or economics. Therefore, the author has decided
to try to expose broad perspective of modern reality in Japan, while still keeping reasonably deep historical account in striving to achieve
twofold goal. Therefore, clever reader perhaps will be aware of some differences between the present book and some more traditional textbooks.
Presentation of the history of life is often intermixed with the presentation of the relevant historical events. Some of these are
quite important - for example, the discussion of the establishment of "Meiji Constitution", is not mentioned in the first part of the book
(which may seem a bit strange at first), but rather is postponed till "Constitution" chapter in the "Japanese Politics" part, in order
to deepen reader's appreciation. Similarly, the discussion of Japanese economics prior to World War II also partly resides at the third chapter,
"Japanese Economics".\\
In principle, the present book does not touch upon cultural aspects, since culture is very important, diverse and (most importantly) complicated
field, author intends to explore it in depth in the next book. However, the influence of culture can never be completely avoided, hence
there is some place allocated in second part ("Japanese Society") in order to very briefly discuss the culture of life in Japan after the World War II
and how Japanese value the sustainment of their native culture.\\
Basically, Japan now is in front of a turning point, since not only the political regime has changed, but also the so-called Japanese corporate management faces changes.
Before if one was about to do the research in Japanese politics, one could safely restrict himself solely to the studying of Liberal Democratic Party,
while now the role of the ruling party is played by Democratic Party. At the same time the leading car manufacturer in Japan, and at the same time the representative
of the Japanese corporations - Toyota Corporation, recalled its vehicles due to the problems with brakes and acceleration pedal and suffered from the reputation damage. This made
people to put the rejection of old good management system into question.\\
The readers perhaps may be interested to know whether there are some peculiarities of Japan that will not change for a long time,
for example devotion to various groups, strict hierarchy etc. Although the present book discusses this aspect in some depth, but I still cannot
ignore the fact that no things in the world are eternal or subject to no change. Especially with the advent of the Internet that changed not
only Japan, but the world in general and moreover provoked various extensive multi-dimensional changes. After that in a short time after the
publication of this book there will be huge changes in Japan. The author can only hope that his imagination will be broad enough to embrace
these changes and make more precise prognosis.
\begin{table}[H]
\caption{Short reference for Japan}
\centering
\begin{tabular}{|l|l|}
	\hline
	Area (as of October 2008) & 377,944 square kilometers\\
	\hline
	Population (as of October 2008) & 127,692,273 people\\
	\hline
	GDP per capita (as of 2008, data from IMF) & 38,457 USD (23rd in the world)\\
	\hline
	Life expectancy (as of 2009) & \parbox{6cm}{79.59 for men (fifth in the world)\\86.44 for women (first in the world)}\\
	\hline
	Import/Export (as of 2008) & 789,547 M JPY / 810,181 M JPY\\
	\hline
\end{tabular}
\end{table}
\part{Japan prior to the World War II}
\chapter{The cultural heritage of Edo Period}
\section{The reasons for the success of the Meiji Restoration}
At 19th century under the massive colonization of the East by the Western countries, China, India ant other eastern countries have gradually became
either colonies, or half-colonies. As the sole exception, the Japan did not became a colony, and because of the success of the Meiji Restoration
reforms, have smoothly set foot on the way of modernization. And even till 20th century, there are still a lot of countries suffer from the late
modernization. As an example, in Iran king Pahlavi (Mohammad Rez\={a} Sh\={a}h Pahlavi, 1919-1980)
while trying to imitate Japanese Meiji Restoration, have sent a large number of people
to study abroad, thus causing revolution. At the end, Pahlavi was forced to flee overseas.\\
So, what are the reasons of the success of the Meiji Restoration? This is not an easy question, although in accordance with the present thoughts,
we may summarize as follows:
External reasons:
\begin{enumerate}
	\item{Balance in strength - one country was seldom able to make strong moves}
	\item{England did not value the importance of the Japanese market, when compared to that of China or India. Japan was though as minor
		in diplomatic sense}
\end{enumerate}
Internal reasons:
\begin{enumerate}
	\item{Empowered by the Taika reforms, Japan have accepted western culture with ease, as it did not have a feeling of strong cultural
		superiority}
	\item{Due to the system of the private schools developed during the Edo period illiteracy was low, and the standards of the national
		education were high}
	\item{Confrontation between the hans (\textit{jap.} \begin{CJK}{UTF8}{bsmi}藩\end{CJK} - vassal states) gave momentum to the reforms}
	\item{Merchants of the Edo period have raised their heads, business activity have flourished; moreover under the rule of the Shogunate
		system, the national domestic market have already been formed.}
	\item{The middle class, that was ranked lower than the samurai, became the main politic force, and this was helpful for the establishment
		of the modern-type national country}
\end{enumerate}
\section{The basis for the success of Meiji Restoration - mass education of the Edo Period}
Japanese scholar Kuwabara Takeo (1904-1988) have incisively pointed out: the reasons for the success of Meiji Restoration are in the high reading
and writing ability of the Japanese people of that time. The literacy of the Meiji Restoration period is approximately 43\% for men; whereas
in the France during the Revolution literacy was lower than 30 \%; at the time of the founding of the People Republic of China (1949) about 
15\%; when the India got the independence only 10\%. By comparing these numbers one sees 
that when compared with other societies enduring great revolutions, the highest literacy rate was probably achieved in Japan during Meiji
Reconstruction. This surely can be traced back to the early built strong system of education of the Edo period - educational 
contribution of "terakoya" (\textit{jap.}
\begin{CJK}{UTF8}{bsmi} 寺子屋\end{CJK}).\\
"Terakoya" (picture 1) have originated in the late Muromachi period (15th century), and become prominent public education institutions
during the Edo period. The name "terakoya", was originally used for the education centers in monasteries where monks have educated secular
people. But when it came to the Edo period, there were already only few terakoya operated by monasteries. According to statistics, in 1875
in Japan there were opened 15600 terakoya in total.\\
The school enrollment rate for men in schools at that time was 43\% and 10\% for woman which was better than in Europe at that time. Young students
have enrolled in the school at age from 6 to 8 and have been studying from 3 to 5 years. The curriculum of terakoya consisted of three practical
disciplines, namely reading, writing and calculations with the abacus. Such promotion of knowledge have mutually interfered with the society, 
economics and culture.
\section{Confrontation between the hans}
Satsuma (modern Kagoshima prefecture), Choshu (modern Yamaguchi prefecture), Tosa (modern Kochi prefecture) and other hans on the south-west 
of Japan have already carried out an agriculture reform, because of the farmer riots in 19th century. Afterwards, under the pressure of the
foreign powers and high sensing of crisis, they have actively implemented the political reforms. These hans, had strong army, and therefore
high political power, and were known as \begin{CJK}{UTF8}{bsmi}雄藩\end{CJK} during the late Edo period. Among these hans Satsuma
and Choshu were 300 years ago defeated by the Tokugawa shogunate in the Battle of Sekigahara during the Sengoku period, therefore they were
strongly against the Tokugawa shogunate. These hans not only strengthened minister's influence to confront Tokugawa Shogunate, but also later
various senior positions in the newly established government were occupied by the samurai from these hans.\\
The warlord of the Choshu han, M\=ori Motonari (1497-1571). When he has been dying at the age of seventy, he already has been controlling 11 areas
in San'y\=o Region (\textit{jap.} \begin{CJK}{UTF8}{bsmi}山陽\end{CJK}) and
	San'in Region (\textit{jap.} \begin{CJK}{UTF8}{bsmi}山陰\end{CJK}) which constituted his kingdom.
However, at that time the founder of Tokugawa shogunate Tokugawa Ieyasu (1543-1616) was only 30 years old, and his power was far below
that of M\=ori. When Toyotomi Hideyoshi, who have been holding ultimate power in Japan at that time, have been preparing to become "Taiko" (title,
that is given to a retired regent in Japan), M\=ori Motonari together with Tokugawa Ieyasu were included in the \textit{Order of Five Elders}
(\textit{jap.} \begin{CJK}{UTF8}{bsmi}五大老\end{CJK}), 
formed by Toyotomi, with him included as well. Immediately after Toyotomi Hideyoshi has passed away, forces of the M\=ori on the west of Japan
were defeated by Tokugawa. As Tokugawa Ieyasu was afraid of the military power of M\=ori, he reduced the land they owned in five times. However,
M\=ori family did not dismiss its servants. In order to support all of them on such a small land, M\=ori generals had to administer changes
aimed towards the more industrial form of manufacturing. At the time when Tokugawa and other warlords still lived in economy that
was dependent on rice cropping, M\=ori produced paper, wax and other products of consumer industry. Moreover, they assimilated new land.
Therefore, when Tokugawa shogunate came to an end, while other hans were in relatively backward and poor agriculture-based phase, Choshu was
already quite a rich region, with the ability to support an army of a Western type. The 300 years long resentment that M\=ori felt towards
Tokugawa was ignited by Yoshida Sh\=oin (1830-1859).\\
When it comes to the agriculture, the impact of Edo period is also quite remarkable. The capital earned by successful businesses and
agricultural sector, partly taken by the government tax, was used to nourish ill-functioning political system. Moreover, these businesses and farms
gave impetus to the industry. Moreover, the high number of skilled blacksmith and laundry masters bolstered the industrialization in Japan. All 
these may be seen as a heritage inherited by Meiji government from the Edo.
\chapter{Meiji Restoration and the amelioration of the culture}
\section{Collapse of the Edo Period Shogunate}
\subsection{Opening to the West and Unfair Treaties}
On June 3, 1853 (by Lunar calendar, July 8 in Gregorian calendar) at the evening, the four battleships from the East India Squadron of the America
(1794-1858) arrived at the Uraga harbour. At the time, these probably could be claimed as the most technologically advanced warships in the world. 
Prior to beaching, Perry already completed an occupation of Ryuku and Ogasawara islands. Uraga is now the part of the city of Yokosuka,
Kanagawa Prefecture, the city which after the end of the Second World War became the port for American's military forces located in Japan and
an important base of Japan's Self-Defence Forces. Because at that time the boats were painted black, Japanese subsequently referred to them
as to "Black Ships" (picture 2).\\
Due to that time United States' ambitions to overthrown the Britain's position as the "queen of the seas", the former hoped to make Japan
their base for operations on a Pacific Ocean, and moreover to strengthen trade relationships with China; furthermore, to protect Americans
whale hunters near the coasts of Japan and let them get fuel and food safely, Perry's forces were dispatched to Uraga.\\
After Perry's black ships stopped at the Uraga, the representatives of Tokugawa Shogunate arrived demanding Perry to move his ships to Nagasaki, the
only port in Japan at that time opened for foreigners (Dutch and Chinese). But Perry refused and in turn submitted to Japanese the letter from
the American President, together with the order to Japanese to open their country for the trade with United States. Perry also notified
Tokugawa government, that he will come back at the spring of next year, and he hopes that at that time government will prepare an answer. After that,
Black Ships left the Uraga harbour. That year was the eleventh after the end of First Opium War (1842), three years after the rise of Taiping
Heavenly Kingdom (1850) and the influential painter Vincent van Gogh was also born on that year.\\
On the January of next year, Perry came to the bay of the Edo (Tokyo now), in order to sign Japan-American treaties. United Stated were treated
as best is possible in these agreements, which included not only opening Shimoda (Shizuoka county) and Hakodate (Hokkaido) for the mooring of
American ships. Besides that, the Consul General of United States Townsend Harris (1804-1878) who came to Japan later, was able to convince
Tokugawa Shogunate to sign stronger mutual goodwill treaties. According to these, prior to signing commerce treaties with the England,
Japan should make sure that relations with United States will not be harmed from this. Furthermore, five ports: Yokohama (Kanagawa prefecture),
Kobe, Hakodate, Nagasaki and Kobe (Hy\=ogo prefecture) were opened to trade with Americans, this being the end to more than 200 years long isolation
of Japan. The treaties signed were very similar to those unequal agreements signed by China and Western powers, enforcing the recognition
of American jurisdiction, and moreover do not recognizing Japan's freedom in stipulating tariffs. Furthermore, Japan shortly signed similar
unfair treaties with Holland, Russia, England and France.\\
Although Japan ended with isolation, there was no way for it to become a power of the day, due to treaties signed with Western countries. What
we mean here
by "unfair treaties", are exactly these, signed with foreign countries. At that time, Japan, like the whole of Asia and Africa, had to face
the fate of descent towards the status of "colony" or "protectorate".\\
The sincerest hope of that time Japanese was "keep independence" and "abolish unfair treaties, catch up the Western countries", and Meiji
Restoration was largely an attempt to achieve these two goals.
\section{Meiji Restoration}%page 23
In March 1868 (first year of Meiji), emperor
\begin{CJK}{UTF8}{bsmi}
\section{Meiji Government's religious policy}%small
\subsection{Repressions on the Christians}
\subsection{Haibitsu Kishaki (\textit{jap.} 廃仏毀釈)}
\end{CJK}
\section{New government and the world}

\section{New academic system}%small
\section{Drastic changes in social structure}%small
\section{Strong army of rich country and the amelioration of the culture}%small
\section{Women's fate in the "amelioration of the culture"}%small
\end{document}
