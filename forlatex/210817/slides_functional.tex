\setbeamercovered{transparent}
\begin{frame}{函数等式}
	\vspace{0.5em}
%%  \begin{figure}[H]
%%%%    \begin{subfigure}[t]{0.3\textwidth}
%%    \centering
	\centerline{\xymatrix{I(\lambda)\ar[d]^{\tilde{\mathbb{T}}_{\lambda}^G}\ar@{-->}[rd]^{R_{\lambda,\nu}^X}&\\I(n-\lambda)\ar[r]^{R^X_{n-\lambda,\nu}}&J(\nu)}}
%%    \end{subfigure}
%%    ~ %add desired spacing between images, e. g. ~, \quad, \qquad, \hfill etc. 
%%      %(or a blank line to force the subfigure onto a new line)

%%    \begin{subfigure}[t]{0.3\textwidth}
%%			\xymatrix{I(\lambda)\ar@{-->}[rd]^{R_{\lambda,\nu}^X}\ar[r]^{R_{\lambda,n-1-\nu}^X}&J(n-1-\nu)\ar[d]^{\tilde{\mathbb{T}}_{n-1-\nu}^{G'}}\\&J(\nu)}
%%    \end{subfigure}
%%    \end{figure}
    \begin{theorem}[函数等式]
	$n=p+q\;(p,q\ge1)$。
	このとき、
	\begin{equation*}
		\begin{array}[]{c}
\tilde{R}_{n - \lambda, \nu}^X \circ
\tilde{\mathbbm{T}}_{\lambda}^G = \pause q_X^{X T} (\lambda, \nu)
\tilde{R}^X_{\lambda, \nu} \mbox{、ここで}\\
%%\tilde{\mathbbm{T}}_{n - 1 - \nu}^{G'} \circ
%%\tilde{R}_{\lambda, n - 1 - \nu}^X = q_X^{T X} (\lambda, \nu)
%%\tilde{R}_{\lambda, \nu}^X,\\
\uncover<2>{
  q^{X T}_X (\lambda, \nu) \assign
  \frac{2^{2\lambda-n}\pi^{-\frac{n}{2}-1}\sin\left( \frac{p-\lambda+1}{2}\pi \right)}{\Gamma\left( \frac{n-\lambda}{2} \right)}\times
   \left\{ \begin{array}{ll}
    2^{1 - \lambda} \sqrt{\pi}, & n \in 2\mathbbm{Z}+ 1,\\
    \Gamma \left( \frac{\lambda - n / 2 + 1}{2} \right), & \frac{n}{2} + p \in
    2\mathbbm{Z},\\
    \Gamma \left( \frac{\lambda - n / 2}{2} \right), & \frac{n}{2} + p \in
    2\mathbbm{Z}+ 1.
  \end{array} \right.
}
		\end{array}
	\end{equation*}
	\end{theorem}
	\begin{remark}
		函数等式の原型
		は\cite[Thm. 12.6]{kobayashi2015symmetry} ($q=0$)で証明された。
	\end{remark}
	\note<1>{
		\mytiming{funct1}
		比例定数も計算できて、極点と霊点を分
		けるように$\Gamma$関数の積で表せます。

		前のように、函数等式の原型として $q = 0$ の場合はKobayashi--Spehの論文で得られている。
	}
	\note<2>{
		\mytiming{funct2}
		比例定数も計算できて、極点と霊点を分
		けるように$\Gamma$関数の積で表せます。

		前のように、函数等式の原型として $q = 0$ の場合はKobayashi--Spehの論文で得られている。
	}
\end{frame}
