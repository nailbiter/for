	以下の図の説明をする。
	\begin{itemize}
		\item %page10
$(\lambda,\nu)\in//$に対して、
	$l:=\frac{1}{2}\left( \nu-\lambda \right)\in\N$、$(\lambda,\nu)\in\backslash\backslash$に対して、$k:=\frac{1}{2}\left( n-1-\lambda-\nu \right)
	\in\N$ とおく{。}
		\item 灰色と白色のみで描れている図は
			$\SBO=\C R^X_{\lambda,\nu}$の場合に相当し、灰色部分は$G$加群$I(\lambda)$の$R_{\lambda,\nu}^X$に{よる}像が$G'$加群$J(\nu)$のとのような部分加群になっている
			かを与える。
		\item 灰色の代わりに
			\begin{equation*}
				\begin{array}[]{l}
					\mbox{緑色(右上がり斜線)}\\
					\mbox{茶色網かけ$=$緑色(右上がり斜線)$+$オレンジ色(右下がり斜線)}
				\end{array}
			\end{equation*}
			で描かれている図は
			\begin{equation*}
				R_{\lambda,\nu}^X=0\quad\mbox{かつ}\quad\SBO=\C R_{\lambda,\nu}^{ \left\{ o \right\}}\oplus \C \tilde{R}_{\lambda,\nu}^X
			\end{equation*}の場合に相当する(定理\ref{thm:classif}を参照)。
			この場合は
		\item 微分対称性破れ作用素 $R_{\lambda,\nu}^{ \left\{ o \right\}}$の像は緑色(右上がり斜線)
			と茶色網かけの合併。
		\item 再正規化した対称性破れ作用素$\tilde{R}_{\lambda,\nu}^X$の像は茶色網かけ。
	\end{itemize}
%p_bigger_than_one_image_diagram_explanation.tex
