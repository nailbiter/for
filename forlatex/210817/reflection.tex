\documentclass[12pt]{article} % use larger type; default would be 10pt

\usepackage{enumerate}
\usepackage{mystyle}
\usepackage{amsthm}
\usepackage{xeCJK}

%%\usepackage{fancyhdr}
%%\pagestyle{fancy}
%%\fancyfoot[C]{text me at \href{mailto:leontiev@ms.u-tokyo.ac.jp}{leontiev@ms.u-tokyo.ac.jp} if there are mistakes/obscurities}
%%\fancyhead{}

\theoremstyle{theorem}
\newtheorem{problem}{Problem}
\newtheorem{question}{Question}
\theoremstyle{definition}
\newtheorem{answer}{My Answer}
\newtheorem{reason}{Reasons}
\theoremstyle{remark}
\newtheorem{countermeasure}{Counter-measures}
\newtheorem{remark}{Remark}
\newtheorem*{remark*}{Remark}

\setCJKmainfont{Hiragino Mincho Pro}

\title{Reflection}
\begin{document}

	\maketitle

\section{Trivia}
\begin{center}
	\begin{tabular}[]{l|l}
		Date:&August 21, 2017\\
		Conference/Seminar's name:& 第56回実函数論・函数解析学合同シンポジウム\\
		Place:& お茶の水女子大学\\
		Title:&{不定値直交群$O(p,q)$の対称性破れ作用素}\\
		Expected Duration:&60 min\\
		Real Duration:& \textasciitilde 57 min\\
		Self-evaluation:& 8 out of 10\\
	\end{tabular}
\end{center}
\section{Questions asked during the talk and my answers}
\begin{question}[Professor Kogiso]
    The results are symmetric for interchanging $p$ and $q$. Was that easy to see from the start?
\end{question}
\begin{answer}
    This was rather a coincidence, we did not predict it from the start.
\end{answer}
\begin{remark}
    I think this answer is terrible. First of all, results are \textit{not} symmetric
    with respect to $p\leftrightarrow q$ interchanging: case $p=1$ is special.

    Moreover, I said that this was easy to see when I started answering, and then changed my answer to an opposite.
\end{remark}
\begin{question}
    Can differential SBO be defined from the beginning as a residue of regular SBO?
\end{question}
\begin{answer}
    It depends on a viewpoint. We could define it directly, or see it as a residue of the regular SBO.
\end{answer}
\begin{remark}
    I talked to this person after the end of the talk to convey my answer better.
\end{remark}
\section{Problems, their reasons and counter-measures}
\begin{problem}
    I finished reached the last slide of the presentation in 50 minutes.
    After that, I incoherently jumped to various slides of the talk and talked about:\begin{enumerate}
        \item 5 equivalent definitions of Zuckerman derived functors appearing in Kobayashi--Orsted;
        \item how only 3 families of constructed five appear in classification of SBOs for $p>1$ case;
        \item how method of proof in our case is different from Kobayashi--Speh:\begin{itemize}
                \item geometry becomes more difficult: $C$
            appears;
                \item  how distribution product and wavefront considerations did not appear in Kobayashi--Speh;
                \item how normalizing regular SBO becomes more difficult;
                \item how normalization was computed via expressing ``generalized eigenvalues'' via an integral
                    and expressing the latter in Gamma functions;
                \item how current optimization of regular and singular SBOs is optimal;
            \end{itemize}
    \end{enumerate}
\end{problem}
\begin{reason}[with reference to {\ttfamily timing.pdf}]
    The main reason was that I messed up the checkpoints: I was sure that the checkpoint I marked as ``40 min''
    in {\ttfamily timing.pdf} should be reached in 30 minutes (in fact,
    I did not have the ``30 minutes'' checkpoint), and then I spend 10 more minutes on every next checkpoint,
    so I finished in 50 minutes in total.
\end{reason}
\begin{countermeasure}
    Be more careful about the checkpoints! Make sure that they are distributed with distance no longer then
    one sixth of length of the talk.
\end{countermeasure}
\section{Miscellaneous comments}
\begin{enumerate}
    \item Before my talk, I was lucky to be suggested by one of the speakers (Sobajima-san: young student 
        working in PDE) to change 
於お茶の水大学 in my slide to 
於お茶の水女子大学. I was said that it is important to get the name of the school which hosts the meeting correctly.
I should be more careful next time. Nevertheless, this time I was lucky enough to fix it before the talk.
\item As was noted by one of the participants, there is a typo in normalization of regular SBO on slide 18:\begin{equation*}
        \begin{array}[]{c}
				\frac{1}{
                    \boxed{\Gamma
                        \left( \frac{\lambda - \nu}{2} \right)} \Gamma \left( \frac{\lambda - \nu}{2}
				\right) \Gamma \left( \frac{\lambda + \nu - n + 1}{2} \right)} | x_p
                |^{\lambda + \nu - n} | Q_{p, q} |^{- \nu}\mbox{ should be }\\
				\frac{1}{\boxed{\Gamma
                    \left( \frac{1 - \nu}{2} \right)} \Gamma \left( \frac{\lambda - \nu}{2}
				\right) \Gamma \left( \frac{\lambda + \nu - n + 1}{2} \right)} | x_p
                |^{\lambda + \nu - n} | Q_{p, q} |^{- \nu}
        \end{array}
    \end{equation*}
\item I overheard the talk between two participants, who claimed that Kubo--san's talk made 3 years ago on this symposium
    was extremely easy to understand, as in particular he did not use representation theory concepts at all.
    This made me think that it might be better to define $I(\lambda),J(\nu)$ not as spherical degenerate principal
    series representations, but rather as homogeneous functions on the cone to make the picture more concrete and accessible
    for non-specialists. Therefore, before my talk I wrote this fact on the whiteboard and mentioned it during the talk.
\item At least one of the participants (Sobajima--san)
    mentioned the talk as ``difficult'' when talking to me after the talk. He is a young
    student working in PDE.
\end{enumerate}
%%\begin{thebibliography}{9}
%%\bibitem{gelbaum}Gelbaum, B.R. and Olmsted, J.M.H.. Counterexamples in Analysis. Dover Publications. 2003
%%\end{thebibliography}
\end{document}


