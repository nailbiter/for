この場合は$J(\nu)$の$K$タイプは無重複かつ$1$次元のパラメータ$b\in\N$で記述できるので、
以下の図では$\N$の部分集合を用いて$J(\nu)$の既約部分商を記述する。その他、下記で用いる記法を説明する。
	\begin{itemize}
		\item 灰色と白色のみで描か
			れている図は
			$\SBO=\C R^X_{\lambda,\nu}$の場合に相当し、灰色部分は$G$加群$I(\lambda)$の$R_{\lambda,\nu}^X$に{よる}像が$G'$加群$J(\nu)$のどのような部分加群になっている
			かを記述している。
		\item それ以外の図では
			\begin{equation*}
				\begin{array}[]{l}
					\mbox{緑色(右下がり{斜}線)}\\
					\mbox{茶色網かけオレンジ色(右下がり斜線)$+$緑色(右上がり斜線)}\\
					\mbox{赤色(右下がり斜線)}\\
					\mbox{紫色網かけ$=$赤色(右下がり斜線)$+$青色(右上がり斜線)}
				\end{array}
			\end{equation*}
			の色のいくつかが現れる。これらの
			いずれの場合も$R_{\lambda,\nu}^X=0$である。
		\item 微分作用素$R_{\lambda,\nu}^{ \left\{ o \right\}}$の像は緑色(右上がり斜線)と茶色網かけの合併。
		\item 再正規化した対称性破れ作用素$\tilde{R}_{\lambda,\nu}^X$の像は茶色網かけ。
		\item $R_{\lambda,\nu}^C$の像は赤色(右下がり斜線)と紫色網かけの合併。
		\item $R_{\lambda,\nu}^Y$の像は紫色網かけ。
		\end{itemize}
%%p_equal_to_one_image_diagram_explanation.tex
