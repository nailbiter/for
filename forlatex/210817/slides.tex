%make suugakkai_16_aux/Makefile
\documentclass[notes,notheorems]{beamer}
\mode<presentation>{\usetheme[secheader]{Boadilla}}
\usepackage{mystyle}
\usepackage{geometry}
\usepackage{amsmath}
\usepackage{xeCJK}
\usepackage{ruby}
\usepackage{enumerate}
\usepackage{setspace}
\usepackage{xypic}
\usepackage[all,cmtip]{xy}
\usepackage{bbm,ulem,float,mystyle}
\usepackage{caption}
\usepackage{subcaption}
\usepackage{setspace}
\usepackage{tikz-cd,array}
\usepackage{catchfilebetweentags}

\def\same{\;''\;}
\newcommand{\Supp}{\mathcal{S}\!{\it upp}}
\newcommand{\red}[1]{{\color[rgb]{0.6,0,0}#1}}
\newcommand{\Sol}{\mathcal{S}\mbox{ol}}
\newcommand{\Hom}{\mbox{\normalfont Hom}}
\newcommand{\D}{\mathcal{D}}
\newcommand{\A}{\mathcal{A}}
\newcommand{\Co}{\mathbb{C}}
\newcommand{\X}{\mathbb{X}}
\renewcommand{\setminus}{\backslash}
\newcommand{\nin}{\not\in}
\newcommand{\tmop}[1]{\ensuremath{\operatorname{#1}}}
\newcommand{\tmtextbf}[1]{{\bfseries{#1}}}
\newcommand{\tmtextit}[1]{{\itshape{#1}}}
\newcommand{\mss}{//}
\newcommand{\mbb}{\backslash\backslash}
\newcommand{\mmm}{\mid\mid}
\catcode`\<=\active \def<{
\fontencoding{T1}\selectfont\symbol{60}\fontencoding{\encodingdefault}}
\catcode`\>=\active \def>{
\fontencoding{T1}\selectfont\symbol{62}\fontencoding{\encodingdefault}}
\newcommand{\assign}{:=}
\newcommand{\comma}{{,}}
\newcommand{\um}{-}
\newcommand{\sol}{\mathcal{S}ol(\R^{p,q};\lambda,\nu)}
\newcommand{\Op}{\mbox{\normalfont Op}}
\newcommand{\Res}{\operatorname{Res}\displaylimits}
\newcommand{\OpR}{\mbox{\it R}}

\setbeamertemplate{theorem}[ams style]
\setbeamertemplate{theorems}[numbered]

\newtheorem{theorem}{定理}
\newtheorem{corollary}{{系}}
\newtheorem{fact}{{Fact}}
\newtheorem{lemma}{{命題}}
\newtheorem{problem}{{問題}}
\newtheorem{solution}{{Solution}}
\theoremstyle{definition}
\newtheorem{definition}{{定義}}
\theoremstyle{example}
\newtheorem{example}{{例}}
\theoremstyle{remark}
\newtheorem*{remark}{注意}

\setCJKmainfont{Hiragino Mincho Pro}
\renewcommand{\thefootnote}{\fnsymbol{footnote}}
\hypersetup{colorlinks=true,urlcolor=blue}
\urlstyle{same}

%%%%%%%%%% Start TeXmacs macros
\newcommand{\nobracket}{}
%%\newcommand{\tmop}[1]{\ensuremath{\operatorname{#1}}}
%%\newcommand{\tmtextbf}[1]{{\bfseries{#1}}}
%%\newcommand{\tmtextit}[1]{{\itshape{#1}}}
\newcommand{\tmtextmd}[1]{{\mdseries{#1}}}
\newcommand{\tmtextrm}[1]{{\rmfamily{#1}}}
\newcommand{\tmtextup}[1]{{\upshape{#1}}}
%%%%%%%%%% End TeXmacs macros

\newenvironment{setting}{\begin{exampleblock}{Setting.}\it}{\end{exampleblock}}
\newenvironment{question}{\begin{block}{Problem.}\it}{\end{block}}
\newenvironment{prop}[1][]{\begin{block}{Proposition#1.}\it}{\end{block}}
\makeatletter
\def\th@mystyle{%
    \normalfont % body font
    \setbeamercolor{block title example}{bg=orange,fg=white}
    \setbeamercolor{block body example}{bg=orange!20,fg=black}
    \def\insertpropblockenv{exampleblock}
  	}
\makeatother
\theoremstyle{mystyle}

%custom formatting settings
\setlength{\parskip}{1em}

%%\newcommand{\yipx}{Y^{p+1, q + 1}_{+, x}}
%%\newcommand{\yipy}{Y^{p, q + 1}_{+, y}}
%%\newcommand{\yimx}{Y^{p+1, q + 1}_{-, x}}
%%\newcommand{\yimy}{Y^{p, q + 1}_{-, y}}
%%\newcommand{\pipx}{\pi^{p+1, q + 1}_{+, x}}
%%\newcommand{\pipy}{\pi^{p, q + 1}_{+, y}}
%%\newcommand{\pimx}{\pi^{p+1, q + 1}_{-, x}}
%%\newcommand{\pimy}{\pi^{p, q + 1}_{-, y}}
%%\newcommand{\tisevenjapb}{\mbox{は偶数の場合}}
%%\newcommand{\tisoddjapb}{\mbox{は奇数の場合}}
%%\newcommand{\mystack}[2]{\begin{array}{c}#1,\\#2\end{array}}
%%%%\newcommand{\pipxStack}[1][]{\mystack{\pipx}{x\d{\in} \Azeven(p\d{+}1,q\d{+}1)#1}}
%%%%\newcommand{\pimxStack}[1][]{\mystack{\pimx}{x\d{\in} \Azeven(q\d{+}1,p\d{+}1)#1}}
%%%%\newcommand{\pipyStack}[1][]{\mystack{\pipy}{y\d{\in} \Azeven(p,q\d{+}1)#1}}
%%%%\newcommand{\pimyStack}[1][]{\mystack{\pimy}{y\d{\in} \Azeven(q\d{+}1,p)#1}}
%%\newcommand{\tzo}{2\Z+1}
%%\newcommand{\tz}{2\Z}
%%\newcommand{\tno}{2\N+1}
\newcommand{\Azeven}{A_0^{\rm\footnotesize even}}

\title{不定値直交群$O(p,q)$の対称性破れ作用素}

% A subtitle is optional and this may be deleted

\author[小林、\underline{レオンチエフ}]{小林俊行\inst{1} \and \underline{レオンチエフ アレックス}\inst{2}}

\institute[東大数理] % (optional, but mostly needed)
{
  \inst{1}%
  東京大学\\
  大学院数理科学研究科{・}\\カブリ数物連携宇宙研究機構
  \and
  \inst{2}%
  東京大学\\
  大学院数理科学研究科
  }
% - Use the \inst command only if there are several affiliations.
% - Keep it simple, no one is interested in your street address.

  \date[第56合同シンポジウム]{第56回実函数論・函数解析学 合同シンポジウム講演集\\2017年8月21日--8月23日\\
於お茶の水大学}
% - Either use conference name or its abbreviation.
% - Not really informative to the audience, more for people (including
%   yourself) who are reading the slides online

\subject{表現論}

\begin{document}
\section{}
\begin{frame}\titlepage\end{frame}

\section{分岐則の問題}

\begin{frame}{}
	$G$と$G'$を簡約群とする。
一般には、$G$の既約表現$\pi$を部分群に制限すると、最早既約にならない\\
	\[
	\xymatrixrowsep{10pt}
	\xymatrixcolsep{50pt}
	\xymatrix{
		\pi:&G\ar[r]&GL_{\mathbb{C}}(V)&\left( \dim V=\infty \right)\\
	&\bigcup&&\\
	&G'\ar@{-->}[uur]_{\pi\mid_{G'}}
	}
\]
\begin{block}{(広い意味での)\underline{分岐則の問題}}
	\centerline{\large $\pi\kern-0.1cm\mid_{G'}$を理解する。}
\end{block}

$\pi$が有限次元表現、$G$がコンパクトの場合は古くから多くの研究があり、組合せ論的なアルゴリズムさえ知られている。
この設定の上で、$\pi$は$G'$の既約表現の直和へに分解する
%%These are well-studied (i.e. combinatorial algorithm) for $\pi$
%%finitely-dimensional and $G$:compact. In this setting, $\pi$ always splits
%%into a direct sum
\begin{equation*}
	\pi\kern-0.1cm\mid_{G'} =  \bigoplus_{\tau \in\widehat{G'}} m (\pi, \tau) \tau
\end{equation*}
of irreducibles $\tau$ of $G'$.
\end{frame}
\begin{frame}{}
	ただ、
\begin{equation*}
	\begin{array}[]{c}
		\dim\pi=\infty,\mbox{ および}\\
		G,G'\mbox{: nonコンパクトならば,}
	\end{array}
\end{equation*}
この問題は極めて難しくなり、
1990年代から
ようやく本格的な研究が始まりました。

%%In particular, several examples that show
%%that behaviour is very wild in general were constructed
特にwildな行動を出展する具体的な例も作られた(\cite{Kobayashi2005}).

%% An idea to understand restriction $\pi\kern-0.1cm \mid_{G'}$ that is not
%%discretely decomposable: compare it with irreducible representations $\tau$ of
%%the subgroup $G'$, i.e. to study the space{
無限次元表現の制限が離散的でない場合に制限の問題を理解する1つ考え方として、$G'$の既約表現$\tau$と比較するというものがある。

つまり、対称性破れ作用素(SBO)の空間
\begin{equation*}
	\tmop{Hom}_{G'} (\pi \mid_{G'}, \tau)\quad\mbox{を理解する。}
\end{equation*}
\end{frame}

\section{$\mathcal{A}\mathcal{B}\mathcal{C}$プログラム}

\begin{frame}{}
	\quad \cite{kobayashi2015program}で小林俊行先生は岐則問題を深く研究するため$\mathcal{ABC}$プログラムを提唱した。
	このプログラムは以下のステップを含む:{
		
}\qquad$(\mathcal{A})$\quad$\mathcal{A}$bstract features: 分岐則の抽象的な様相(ようそう)を研究する
;{

}\qquad$(\mathcal{B})$\quad$\mathcal{B}$ranching law: 分岐則を具体的に決定する; {

}\qquad$(\mathcal{C})$\quad$\mathcal{C}${onstruction}: 対称性破れ作用素を具体的に構成する。
\end{frame}
\begin{frame}{}
	今回の主テーマは``標準表現''に対するプログラム$\mathcal{C}$である。特に、以下の問題を考る:
\vspace{-1em}{

}\qquad$(\mathcal{C}1)$\quad 対称性破れ作用素を構成する;{

	\vspace{-0.4em}
}\qquad$(\mathcal{C}2)$\quad 対称性破れ作用素を分類する;{

	\vspace{-0.4em}
}\qquad$(\mathcal{C}3)$\quad 対称性破れ作用素と普通のintertwining operatorの間の函数等式を決定する;{

	\vspace{-0.6em}
}\qquad$(\mathcal{C}4) \quad$対称性破れ作用素の間の相互関係をoperator valueの有理型な関数の留数の形で整理する;{

	\vspace{-0.6em}
}\qquad$(\mathcal{C}5)$\quad 対称性破れ作用素の像を決定する。{

}\vspace{-1em}\quad $(\mathcal{C}1) - (\mathcal{C}5)$の五つの問題はKobayashi先生とSpeh先生がrank 1の場合に提唱し、
\vspace{-0.6em}
\begin{equation*}
	(G, G') = (O (n + 1, 1), O (n, 1)). \quad\mbox{\cite{kobayashi2015symmetry}}
\vspace{-0.6em}
\end{equation*}
\vspace{-1.5em}
\begin{block}{\underline{目標}:}
	\cite{kobayashi2015symmetry}の結果を高階場合
\vspace{-1em}
	\begin{equation*}
		(G, G') = (O (p + 1, q + 1), O (p, q + 1)).
\vspace{-1em}
	\end{equation*}
	に拡張する
\end{block}
\end{frame}
\begin{frame}{}
``標準表現''として \underline{球退化主系列表現
}を扱う:\begin{equation*}
	\begin{array}[]{c}
		I(\lambda)=\tmop{Ind}_P^G(\mathbb{C}_{\lambda}),\quad \lambda\in\mathbb{C},\\
		I(\nu)=\tmop{Ind}_{P'}^{G'}(\mathbb{C}_{\nu}),\quad \nu\in\mathbb{C},
	\end{array}
\end{equation*}
ここで $P \subset G$は Levi部分
\begin{equation*}
{MA} \simeq O (p, q) \times \{ \pm 1 \}
\times \mathbbm{R},
\end{equation*}
を持つ極大放物型部分群である。
そうすると、$P' = P \cap G'$ は $G'$の極大放物型部分群になる。
\end{frame}
\begin{frame}{}
	\begin{block}{共形幾何の見方}
		幾何の言葉で言うと、$I(\lambda)=C^\infty(X,\mathcal{L}_{\lambda})$は共形幾何から出て来る:
		\centerline{\scalebox{0.8}{
		\newdir{:=}{{}}
		\xymatrix{
			& \mathcal{L}_\lambda\mbox{ :共形等質ライン束},\lambda\in\mathbb{C}
			\ar[d]\\
  		G=O(p+1,q+1)
		\ar@/^2pc/[r] &G/P\simeq (\Sp^p\times\Sp^q)/\left\{ \pm I \right\}\\
		P=MAN\ar@{:=}[u]_{\hspace{-0.25cm}\bigcup}
		\ar@/^2pc/[rd]^{{\begin{array}{c}\; \\\mbox{共形変換}\end{array}}}
		%\mbox\newline oeueou}\vspace{0.8cm}}
		&\\
	M_+N=O(p,q)\ltimes \mathbb{R}^{p,q}
	\ar@{:=}[u]_{\hspace{-0.25cm}\bigcup}
	\ar@/^2pc/[r]^{\mbox{等長}}&
	\mathbb{R}^{p,q}=\left( \mathbb{R}^{p+q},ds^2=dx_1^2+\ldots+dx_p^2-dx_{p+1}^2-\ldots-dx_{p+q}^2 \right)\ar@{^{(}->}[uu]
	_{\mbox{共形コンパクト化}}}
}}
	\end{block}
\begin{remark}
プロガラム Aにおけるアプリオリ評価として(\cite{kobayashi2013finite}、\cite{kobayashi2014classification})、対称性破れ作用素空間の次元
\vspace{-1.2em}
\begin{equation*}
\dim \tmop{Hom}_{G'} (I (\lambda) \mid_{G'}, J
\nobracket (\nu)
\vspace{-0.8em}
\end{equation*}
が表現のパラーメタ$(\lambda,\nu)\in\mathbb{C}^2$よらずに一様に押さえられている。
\end{remark}
\end{frame}
\section{対称性破れ作用素の分類}
\begin{frame}{}
%%	Use the strategy developed in \cite{kobayashi2015symmetry}. 
	基本的に、\cite{kobayashi2015symmetry}の作戦を使う。
%%	The geometry is a little more complicated.
	しかし、高階の設定で幾何が少し複雑になる。
	\begin{fact}[対称性破れ作用素の積分核]
		\cite{kobayashi2015symmetry}の一般論を
今の特別な設定に適
用すると、以下の図式を得る:
%%Applying the general statement of
%%\cite{kobayashi2015symmetry} to our concrete setting we get:
	\centerline{
		\scalebox{0.9}{
		\xymatrixcolsep{5pc}
		\xymatrix{\Hom_{G'}(I(\lambda),J(\nu))\ar[r]^{\simeq} \ar@/^2pc/[rr]^{\mathcal{S}}
		&\left( \mathcal{D}'(G/P,\mathcal{L}_{n-\lambda}) \otimes\mathbb{C}_\nu \right)^{P'}
	\ar[r]_-{F\mapsto \supp(F)}\ar[d]^{\simeq}_{\mbox{rest}}
	&2^{P'\backslash G/P}\\
	&\sol\subset\mathcal{D}'(\R^{p,q})\ar[lu]^{\mbox{Op}}_{\simeq}&
	}
	}
}
	\end{fact}
\end{frame}
\begin{frame}{}
	\begin{equation*}
		\begin{array}[]{l}
X \assign G / P \simeq (\mathbbm{S}^p \times
\mathbbm{S}^q) / \{ \pm I \}, \\
Y \assign \{ [\xi, \eta] \in G / P \mid \xi_{p +
1} = 0 \} \simeq (\mathbbm{S}^{p - 1} \times \mathbbm{S}^q) / \{ \pm I \}, \\
C \assign \{ [\xi, \eta] \in G / P \mid \xi_1 =
\eta_{q + 1} \},\\
\left\{ o \right\}:=\left\{ \left[ 1,0_{p+q}1 \right] \right\}\mbox{とする}.
		\end{array}
	\end{equation*}
	\begin{theorem}[両側剰余空間$P'
\backslash G / P$の分類]
$P' \backslash G / P$は以下のようになる:
  \begin{figure}[H]
    \centering
    \begin{subfigure}[t]{0.3\textwidth}
	    \xymatrixrowsep{0.5pc}
	    \xymatrix{&X\ar@{-}[ld]_1\ar@{-}[rd]^1&\\Y\ar@{-}[rd]_1&&C\ar@{-}[ld]^1\\&Y\cap C\ar@{-}[dd]^{p+q-2}&\\&&\\&\{[0]\}&}
	\caption{$p>1$のとき}
    \end{subfigure}
    ~ %add desired spacing between images, e. g. ~, \quad, \qquad, \hfill etc. 
      %(or a blank line to force the subfigure onto a new line)
    \begin{subfigure}[t]{0.3\textwidth}
	    \xymatrixrowsep{0.5pc}
	    {\xymatrix{&X\ar@{-}[ld]_1\ar@{-}[rd]^1&\\Y\ar@{-}[rddd]_{p+q-2}&&C\ar@{-}[lddd]^{p+q-2}\\&&\\&&\\&\{[0]\}&}}
	    \vspace{0.7em}
	\caption{$p=1$のとき}
    \end{subfigure}
\vspace{-0.8em}
\end{figure}
	\end{theorem}
\end{frame}
\begin{frame}{微分対称性破れ作用素}
	\begin{fact}[微分対称性破れ作用素の分類 (\cite{kobayashi2015branching})]
$(\lambda, \nu) \in / / \assign \{
(\lambda, \nu) \in \mathbbm{C}^2 \mid \nu - \lambda \in 2\mathbbm{N} \}$に対し、
\begin{equation*}
	{R}_{\lambda, \nu}^{\{ o \}} =
\tmop{Rest}_{x_p = 0} \circ \tilde{C}_{\nu - \lambda}^{\lambda - \frac{n -
1}{2}} \left( - \Delta_{\mathbbm{R}^{p - 1, q}}, \frac{\partial}{\partial x_p}
\right)を定める。
\end{equation*}
ここで$\tilde{C}(s,t)$は\cite[(6.5)]{Kobayashi2016}で定義された2変数多項式であり、正規化された
	Gegenbauer多項式から導かれるものである。
	$R^{ \left\{ o \right\}}_{\lambda,\nu}$は微分対称性破れ作用素になり(つまり、$\mathcal{S}\mbox{upp} ({R}_{\lambda,
	\nu}^{\{ o \}}) = \{ 0 \}$)、全て微分対称性破れ作用素は$R^{ \left\{ o \right\}}_{\lambda,\nu}$に非礼である。
	\end{fact}
\end{frame}
\begin{frame}
	\begin{theorem}[regular 対称性破れ作用素の構成]
For
$(\lambda, \nu) \in \mathbbm{C}^2$ および 
$\tmop{Re} (\nu) < 0$ and $\tmop{Re}
(\lambda + \nu - n) > 0$ を満たす$(\lambda,\nu)\in\mathbb{C}^2$に対し、関数
\vspace{-0.6em}
\begin{equation*}
	| x_p |^{\lambda + \nu - n} | Q_{p, q} |^{- \nu}
\end{equation*}
は連続であるし、
$\mathcal{S} \mbox{ol} (\mathbbm{R}^{p, q} ; \lambda, \nu)$の元になる。
\vspace{-0.5em}
\begin{equation*}
	\tilde{R}_{\lambda, \nu}^X \assign \tmop{Op} (| x_p
	|^{\lambda + \nu - n} | Q_{p, q} |^{- \nu}) \mbox{とする}.
\end{equation*}
	\end{theorem}
\vspace{-0.4em}
	\begin{theorem}[regular 対称性破れ作用素の正規化]
		$\tilde{R}_{\lambda, \nu}^X$ は $(\lambda, \nu) \in
\mathbbm{C}^2$に有理型に依存する対称性破れ作用素の族に拡張できる。更に、
\vspace{-1.1em}
\begin{equation*}
	{R}_{\lambda, \nu}^X \assign \frac{1}{\Gamma
\left( \frac{\lambda - \nu}{2} \right) \Gamma \left( \frac{\lambda - \nu}{2}
\right) \Gamma \left( \frac{\lambda + \nu - n + 1}{2} \right)} \tilde{R}_{\lambda,
\nu}^X
\vspace{-0.6em}
\end{equation*}
のように正規化すると、$R_{\lambda,\nu}^X$は$(\lambda,\nu)\in\mathbb{C}^2$に正則に依存する
対称性破れ作用素になる。
また、$R^X_{\lambda,\nu}=0\iff(\lambda,\nu)$
がある離散集合に属する。
	\end{theorem}
\end{frame}
\begin{frame}
	\begin{theorem}[singular 対称性破れ作用素の構成]
	$S=X,Y,C,$ $\left\{ o \right\}$に対して、
	パラメータ集合$D_S$を以下の表のように定めると、作用素
	$R_{\lambda,\nu}^S$および$\tilde{R}_{\lambda,\nu}^X$は
	$I(\lambda)$から$J(\nu)$への対称性破れ作用素であり、
$(\lambda,\nu)\in D_S$に正則に依存する。
それぞれ$R_{\lambda,\nu}^S$および$\tilde{R}_{\lambda,\nu}^X$はregular対称性破れ作用素$\tilde{R}_{\lambda,\nu}^X$の再正規化である。
	全て$(\lambda,\nu)\in D_S$に対して、$\Supp(R_{\lambda,\nu}^S)\subset S$である(更に、一般の位置に{ある}$(\lambda,\nu)$に対しては{等号}が成り立つ)。
	全て$(\lambda,\nu)$とそれぞれ$R_{\lambda,\nu}^S$および$\tilde{R}_{\lambda,\nu}^X$に対し、
	台を具体的に決定することが出来る。\\
\centerline{\begin{tabular}{|l|l|}
  \hline
  ${R}_{\lambda, \nu}^S$ & $D_S$\\
  \hline
  ${\tilde{R}}_{\lambda, \nu}^X$ & $\mid \mid \mid$\\
  \hline
  ${R}_{\lambda, \nu}^Y$ & \textbackslash\textbackslash\\
  \hline
  ${R}_{\lambda, \nu}^C$ & $\mid \mid \mid$\\
  \hline
\end{tabular}}
\vspace{-1.2em}
表の記号を説明する:\vspace{-0.5em}
	\begin{equation*}
		\begin{array}[]{l}
		\mid \mid \mid \assign \left\{ (\lambda, \nu) \in \mathbbm{C}^2 \mid
\nu \in - 2\mathbbm{N} \; \tmop{or} \; \nu \equiv q + 1 \; \tmop{mod} \; 2
\right\},\\
\backslash\backslash \assign \{ (\lambda, \nu) \in \mathbbm{C}^2 \mid
\lambda + \nu - n + 1 \in - 2\mathbbm{N} \},\\
\mid \mid \assign \{ (\lambda, \nu) \in \mathbbm{C}^2 \mid \nu \in 1 +
2\mathbbm{N} \} . 
		\end{array}
	\end{equation*}
	\end{theorem}
\end{frame}
\begin{frame}{}
	\begin{theorem}[対称性破れ作用素の空間の次元]
$(\lambda, \nu) \in \mathbbm{C}^2$に対し、
\begin{equation*}
\dim \tmop{Hom}_{G'} (I (\lambda) \mid_{G'}, J
(\nu)) \in \{ 1, 2 \}.
\end{equation*}
	\end{theorem}
	\begin{theorem}[対称性破れ作用素の分類]
		\begin{equation*}
			\begin{array}[]{l}
				p=1\Rightarrow \Hom_{G'}\left( I(\lambda)\kern-0.1cm\mid_{G'},J(\nu) \right)=\\
				\left\{  \begin{array}[]{ll}
  \mathbbm{C} \tilde{R}_{\lambda, \nu}^X, & (\lambda, \nu) \nin (/ / \cap \mid
  \mid \mid) \cup (\mid \mid \cap \backslash\backslash),\\
  \mathbbm{C} \widetilde{\tilde{R}}_{\lambda, \nu}^X \oplus \mathbbm{C}
  \tilde{R}_{\lambda, \nu}^{\{ o \}}, & (\lambda, \nu) \in (/ / \cap \mid \mid
  \mid) - (\mid \mid \cap \backslash\backslash),\\
  \mathbbm{C} \tilde{R}_{\lambda, \nu}^C \oplus \mathbbm{C}
  \tilde{R}^Y_{\lambda, \nu}, & (\lambda, \nu) \in (\mid \mid \cap
  \backslash\backslash) - / /,\\
  \mathbbm{C} \widetilde{\tilde{R}}_{\lambda, \nu}^X, & (\lambda, \nu) \in
  \mid \mid \cap \backslash\backslash \cap / /,
				\end{array}
					\right.\\
				p>1\Rightarrow \Hom_{G'}\left( I(\lambda)\kern-0.1cm\mid_{G'},J(\nu) \right)=\\
				\left\{  \begin{array}[]{ll}
  \mathbbm{C} \tilde{R}_{\lambda, \nu}^X, & (\lambda, \nu) \nin / / \cap \mid
  \mid \mid,\\
  \mathbbm{C} \widetilde{\tilde{R}}_{\lambda, \nu}^X \oplus \mathbbm{C}
  \tilde{R}_{\lambda, \nu}^{\{ o \}}, & (\lambda, \nu) \in / / \cap \mid \mid
  \mid.
				\end{array}
					\right.\\
			\end{array}
		\end{equation*}
	\end{theorem}
\end{frame}

\section{$(\mathcal{C}3) - (\mathcal{C}5)$にの答え}
\begin{frame}
	\begin{theorem}[$K$不変ベクトルのスペクトラム]
	$n=p+q\;(p,q\ge1)$という記法は前述の通りとする。この{とき、}
	\begin{equation*}
		\tilde{R}_{\lambda, \nu}^X 1_{\lambda} =
\frac{2^{1 - \lambda} \pi^{n / 2}}{\Gamma \left( \frac{\lambda}{2} \right)
\Gamma \left( \frac{\lambda + 1 - q}{2} \right) \Gamma \left( \frac{q - \nu +
1}{2} \right)} 1_{\nu}
	\end{equation*}
	\end{theorem}
	\begin{remark}
%%This Theorem was known in
%%Bernstein--Reznikov 2004 for $p = q = 1$ (i.e. $G' \simeq \tmop{SL} (2,
%%\mathbbm{R})$) and in \cite{kobayashi2015symmetry} for $q = 0$.
	$SL(2,\R)$の表現のテンソル積の場合、すなわち、
	定理\ref{thm:spherical}\;の$p=q=1$の特別な場合はBernstein--Reznikov\cite[Lem. A.5]{bernstein2004estimates} によって得られ、
	また$q=0$の場合は\cite[Prop.\ 7.4]{kobayashi2015symmetry}で得られている。
	\end{remark}
\end{frame}
\begin{frame}{留数公式}
	$(\lambda,\nu)\in\mathbb{C}^2\setminus//$に対して、
\begin{equation*}
K_{\lambda, \nu}^X \assign \frac{| x_p |^{\lambda +
\nu - n}}{\Gamma \left( \frac{\lambda + \nu - n + 1}{2} \right)} \cdot \frac{|
Q |^{- \nu}}{\Gamma \left( \frac{1 - \nu}{2} \right)} \in \mathcal{S}
\mbox{ol} (\mathbbm{R}^n ; \lambda, \nu)\mbox{ を定める}.
\end{equation*}
このとき、$\tilde{R}_{\lambda, \nu}^X = \frac{1}{\Gamma \left( \frac{\lambda -
\nu}{2} \right)} \tmop{Op} (K_{\lambda, \nu})$は$(\lambda,\nu)\in\mathbb{C}^2$に正則に依存する対称性破れ作用素の族に拡張できる
\begin{theorem}[留数公式]
	$(\lambda,\nu)\in//$に対して、$l:=\frac{1}{2}\left( \nu-\lambda \right)\in\N$とおく。このとき
\begin{equation*}
	\tilde{R}_{\lambda, \nu}^X = \frac{(- 1)^l l!
\pi^{(n - 2) / 2}}{2^{\nu + 2 l - 1}} \cdot \frac{\sin \left( \frac{1 + q -
\nu}{2} \pi \right)}{\Gamma \left( \frac{\nu}{2} \right)} \tilde{R}_{\lambda,
\nu}^{\{ o \}}, \quad \mbox{for }(\lambda, \nu) \in //.
\end{equation*}
\end{theorem}
\begin{remark}
		留数公式の原型
		として$q=0$の場合は\cite[Thm. 12.2]{kobayashi2015program}で得られた。
\end{remark}
\end{frame}
\begin{frame}{Knapp--Stein 作用素}
	\begin{definition}
		\underline{Knapp--Stein 作用素}は以下のように定義された $G$絡作用素である\begin{equation*}
			\begin{array}[]{l}
				\tilde{\mathbb{T}}_\lambda^G:I(\lambda)\to I(n-\lambda)\\
				f\mapsto q_T(\lambda)\left(\myabs{Q_{p,q}}^{\lambda-n}\star f  \right)
			\end{array}
		\end{equation*}
		ここで
		$q_T (\lambda)$ は具体的に$\Gamma$関数の積に述べられた。
	\end{definition}
		$\tilde{R}_{n - \lambda, \nu}^X \circ\tilde{\mathbbm{T}}_{\lambda}^G$ と対称性破れ作用素 $\tilde{R}_{\lambda, \nu}^X$を比較する。両方は比例になる。 
\end{frame}
\begin{frame}{}
	合成
		$\tilde{R}_{n - \lambda, \nu}^X \circ\tilde{\mathbbm{T}}_{\lambda}^G$ と対称性破れ作用素 $\tilde{R}_{\lambda, \nu}^X$を比較する。両方は比例になる
		\centerline{\xymatrix{I(\lambda)\ar[d]^{\tilde{\mathbb{T}}_{\lambda}^G}\ar@{-->}[rd]^{R_{\lambda,\nu}^X}&\\
		I(n-\lambda)\ar[r]^{R^X_{n-\lambda,\nu}}&J(\nu)}}
		$G' = O (p, q + 1)$に対し、同様に$\tilde{\mathbbm{T}}_{\nu}^{G'}$を定める:
		\centerline{
			\xymatrix{
				I(\lambda)\ar@{-->}[rd]^{R_{\lambda,\nu}^X}\ar[r]^{R_{\lambda,n-1-\nu}^X}&J(n-1-\nu)\ar[d]^{\tilde{\mathbb{T}}_{n-1-\nu}^{G'}}\\
				&J(\nu)
			}}
\end{frame}
\begin{frame}
  \begin{figure}[H]
    \centering
    \begin{subfigure}[t]{0.3\textwidth}
\xymatrix{I(\lambda)\ar[d]^{\tilde{\mathbb{T}}_{\lambda}^G}\ar@{-->}[rd]^{R_{\lambda,\nu}^X}&\\I(n-\lambda)\ar[r]^{R^X_{n-\lambda,\nu}}&J(\nu)}
    \end{subfigure}
    ~ %add desired spacing between images, e. g. ~, \quad, \qquad, \hfill etc. 
      %(or a blank line to force the subfigure onto a new line)
    \begin{subfigure}[t]{0.3\textwidth}
			\xymatrix{I(\lambda)\ar@{-->}[rd]^{R_{\lambda,\nu}^X}\ar[r]^{R_{\lambda,n-1-\nu}^X}&J(n-1-\nu)\ar[d]^{\tilde{\mathbb{T}}_{n-1-\nu}^{G'}}\\&J(\nu)}
    \end{subfigure}
    \end{figure}
    \vspace{-1em}
    \begin{theorem}[函数等式]
	$n=p+q\;(p,q\ge1)$で{あった}ことを思い出そう。
	このとき、
	\begin{equation*}
		\begin{array}[]{l}
\tilde{R}_{n - \lambda, \nu}^X \circ
\tilde{\mathbbm{T}}_{\lambda}^G = q_X^{X T} (\lambda, \nu)
\tilde{R}^X_{\lambda, \nu} \\
\tilde{\mathbbm{T}}_{n - 1 - \nu}^{G'} \circ
\tilde{R}_{\lambda, n - 1 - \nu}^X = q_X^{T X} (\lambda, \nu)
\tilde{R}_{\lambda, \nu}^X,\\
	\mbox{ここで}\\
q_X^{X T} (\lambda, \nu) = \ldots
q_X^{T X} (\lambda, \nu) = \frac{\pi^{\frac{n -
2}{2}} \sin \left( \frac{p - \nu}{2} \pi \right) 2^{1 - n - \nu}}{\Gamma
\left( \frac{n - 1 - \nu}{2} \right)} \left\{ \begin{array}{ll}
  \Gamma \left( \frac{1 - \nu}{2} \right), & p = 1\\
  1, & n \in 2\mathbbm{Z}\\
  \ldots, & \ldots
\end{array} \right.
		\end{array}
	\end{equation*}
	\end{theorem}
\vspace{-0.5em}
	\begin{remark}
		函数等式の原型
		として$q=0$の場合は\cite[Thm. 12.6]{kobayashi2015program}で得られた。
	\end{remark}
\end{frame}
\begin{frame}{対称性破れ作用素の像}
	\begin{theorem}
%%		We can compute images of every SBOs constructed above for every 
		全て$(\lambda, \nu) \in \mathbbm{C}^2$に対し、上記で構成されたそれぞれの対称性破れ作用素の像を具体的に決定出来る。
	\end{theorem}
\end{frame}
\begin{frame}{}
	最後に、上記の結果の応用としてZuckerman導来函手加群の間の対称性破れ作用素の問題を論じる。
\cite[(5.1.1)]{KO2}にg{倣}って$p>1${かつ}$q\ge1$のときに\begin{equation*}
	A_0(p,q):=\left\{ \lambda\in\Z+\frac{p+q}{2}\;:\;\lambda>-1 \right\}
\end{equation*}とおくと、
\cite{KO2}で
%16-b
示したように、$\lambda\in A_0(p,q)$に対して
$O(p,q)$の既約ユニタリ表現$\pi_{\pm,\lambda}^{p,q}$
%%16-c
が定まる。(\cite{KO2}では$\pi_{\pm,\lambda}^{p,q}$を定義するにあたって5種類の
特徴づけが与えら{れ}、それらは互いに
同値であることが示されている。その特徴づけの1つは
Zuckerman導来函手加群で記載される。)
以下では、この加群の間に対称性破れ作用素がある{か}について論じる。簡単のため、$A_0^{\mbox{\scriptsize even}}(p,q):=\left\{ \lambda\in A_0(p,q)\mid \lambda-\frac{p-q}{2}+1\in2\Z \right\}$とおく。
%%16-a
以下では\begin{equation*}
	g(t):=\left\{ \begin{array}[]{ll}
		1&\left(t\in2\N+\frac{1}{2}  \right)\\[10pt]
		0&t\nin \left(2\N+\frac{1}{2}  \right)
	\end{array}\right.,\quad h(t):=\left\{ \begin{array}[]{ll}
		1&\left(  t<\frac{q}{2}\right)
		\\[10pt]
		0&\left( t\ge\frac{q}{2} \right)
\end{array}\right.
\end{equation*}とおく。
\end{frame}
\begin{frame}
\begin{theorem}[Zuckerman導来加群函手{$\pi_{\pm,\lambda}^{p,q}$}間の対称性破れ作用素の存在問題]
	$n=p+q\;(p,q\ge1),\;n':=n-1$とする。
	以下では
	\vspace{-1em}
\begin{equation*}
                \begin{array}[]{c}
                        x\in\left\{\begin{array}[]{ll}
                                \Azeven(p+1,q+1),&\delta=+\mbox{ のとき}\\
                                \Azeven(q+1,p+1),&\delta=-\mbox{ のとき}\\
                        \end{array}\right.\\
                        y\in\left\{\begin{array}[]{ll}
                                \Azeven(p,q+1),&\varepsilon=+\mbox{ のとき}\\
                                \Azeven(q+1,p),&\varepsilon=-\mbox{ のとき}\\
                        \end{array}\right.
		\end{array}
	\vspace{-1em}
	\end{equation*}
	と仮定する。このとき{、}
	$\Hom_{G'}\left(\pi_{\delta,x}^{p+1,q+1}\kern-0.3em\mid_{G'} ,\pi_{\varepsilon,y}^{p,q+1} \right)$の次元と基底を具体的に決定出来る。
%%
%%%%	%%%%%%%%%%%%%%%%%%%%%%%%%%%%%%%%
\renewcommand{\mystack}[2]{\begin{array}{c}#1,\\#2\end{array}}
\newcommand{\mytable}[9]{{\centering
$\begin{array}{|@{}c@{}|@{}c@{}|@{}c@{}|}
  \hline
	#1& #2&#3\\
  \hline
	#4& #5&#6\\
  \hline
	#7& #8&#9\\
  \hline
\end{array} \newline$
}}
\newcommand{\mytableThreeTwo}[6]{\begin{center}
$\begin{array}{|@{}c@{}|@{}c@{}|}
  \hline
	#1& #2\\
  \hline
	#3& #4\\
  \hline
	#5& #6\\
  \hline
\end{array} \newline$
\end{center}}
\newcommand{\commonShift}{\hspace*{-0.0cm}}
%%%%%%%%%%%%%%%%%%%%%%%%%%%%%%%%%%%%
\begin{enumerate}[(1)]
	\item $p=1,q\in2\Z$
		\\
\hspace*{0cm}\commonShift\mytableThreeTwo	%#1
{}		{\pimyStack[\mid y\geq q/2]}
{\pipx}		{0}
{\pimx}		{g\kern-0.05cm\left( {y-x} \right)}
	\item $p=1,q\in2\Z+1$\\
\hspace*{0cm}\commonShift\mytableThreeTwo	%#2
{}		{\pimyStack[\mid y\geq q/2]}
{\pipx}		{0}
{\pimx}		{g\kern-0.05cm\left( {y-x} \right)}
	\item $p,q\in2\Z$\\
\hspace*{-0cm}\commonShift\mytable	%3
{}	{\pipy}				{\pimy}
{\pipx}	{g\kern-0.05cm\left({x-y}\right)} 	{0}
{\pipx}	{0} 				{g\kern-0.05cm\left( {y-x} \right)}
\item $p\in2\Z,q\in2\Z+1$\\
\commonShift\mytable	%4
{}	{\pipy}	{\pimy}
{\pipx} {0}	{g\kern-0.05cm\left( {-x-y} \right)}
{\pimx} {0} 	{g\kern-0.05cm\left( {y-x} \right)}
\item $p\in2\Z+1,q\in2\Z$\\
\commonShift\mytable	%5
{}			{\pipy}		{\pimy}
{\pipx}			{0} 		{g\kern-0.05cm\left( {-x-y} \right)}	
{\pimx} 		{0} 		{g\kern-0.05cm\left( {y-x} \right)}
\item $p,q\in2\Z+1$\\
\commonShift\mytable	%6
{}		{\pipy}				{\pimy}
{\pipx}		{g\kern-0.05cm\left( {x-y} \right)}	{0}
{\pimx}		{0}				{g\kern-0.05cm\left( {y-x} \right)}	
\end{enumerate}

%%	$p=1,q\tisevenjapb$
%%		\\
%%\hspace*{0cm}\commonShift\mytableThreeTwo	%#1
%%{}		{\pimy}
%%{\pipx}		{h(y)(1-g(x-y))}
%%{\pimx}		{g\kern-0.05cm\left( {y-x} \right)}
\end{theorem}
\begin{remark}
	\begin{enumerate}[(1)]
		\item この定理では分岐則が離散分解する場合(一般論は\cite{10.2307/120963})とそうでない場合の両方が含まれている。
			分岐則が離散分解する場合、上記の分岐則は\cite[Thm. 3.3]{kobayashi1993}によって得られた公式と一致する。
		\item $q=0$の場合の類似の結果は\cite[Thms. 12.1 and 1.3]{kobayashi2015symmetry}で得られている。
	\end{enumerate}
	\vspace{-0.8em}
\end{remark}
\end{frame}

\begin{frame}[allowframebreaks]{References}
	
\begin{thebibliography}{KØSS15}

\bibitem[KP16b]{Kobayashi2016}
T.~Kobayashi and M.~Pevzner.
\newblock Differential symmetry breaking operators: {I}{I}. {R}ankin--{C}ohen
  operators for symmetric pairs.
\newblock \emph{Selecta Mathematica}, \textbf{22}(2), (2016), pp. 847--911.
Available at \url{http://dx.doi.org/10.1007/s00029-015-0208-8}.

\bibitem[K93]{kobayashi1993}
T.~Kobayashi.
\newblock The restriction of ${A}_q \left( \lambda \right)$ to reductive
  subgroups.
\newblock \emph{Proc. Japan Acad. Ser. A Math. Sci.}, \textbf{69}(7), (1993),
  pp. 262--267.
Available at \url{http://dx.doi.org/10.3792/pjaa.69.262}.
  \bibitem[KM14]{kobayashi2014classification}T.~Kobayashi  and 
  T.~Matsuki.{\newblock} Classification of finite-multiplicity symmetric
  pairs.{\newblock} In \tmtextit{\tmtextrm{\tmtextup{\tmtextmd{Special Issue
  in honour of Professor Dynkin for his 90th birthday}}}},  volume~19,  pages 
  457--493. Springer, 2014.{\newblock}
  
  \bibitem[KO13]{kobayashi2013finite}T.~Kobayashi  and  T.~Oshima.{\newblock}
  Finite multiplicity theorems for induction and restriction.{\newblock}
  \tmtextit{Advances in Mathematics}, 248:921--944, 2013.{\newblock}
  
  \bibitem[Kob05]{Kobayashi2005}Toshiyuki Kobayashi.{\newblock}
  \tmtextit{Restrictions of Unitary Representations of Real Reductive Groups},
  pages  139--207.{\newblock} Birkh{\"a}user, 2005.{\newblock}
  
  \bibitem[Kob15]{kobayashi2015program}T.~Kobayashi.{\newblock} A program for
  branching problems in the representation theory of real reductive
  groups.{\newblock} In \tmtextit{\tmtextrm{\tmtextup{\tmtextmd{Special issue
  in honor of Vogan's 60th years birthday}}}},  volume  312,  pages  277--322.
  Birkh{\"a}user, 2015.{\newblock}
  
  \bibitem[KS15]{kobayashi2015symmetry}T.~Kobayashi  and  B.~Speh.{\newblock}
  \tmtextit{Symmetry Breaking for Representations of Rank One Orthogonal
  Groups},  volume \tmtextbf{238} of \tmtextit{Memoirs of the Amer. Math.
  Soc}.{\newblock} 2015.{\newblock}
  
  \bibitem[K{\O}SS15]{kobayashi2015branching}T.~Kobayashi, B.~{\O}rsted,
  P.~Somberg, and  V.~Sou{\v c}ek.{\newblock} Branching laws for verma
  modules and applications in parabolic geometry. I.{\newblock}
  \tmtextit{Advances in Mathematics}, 285:1796--1852, 2015.{\newblock}

\bibitem[K{\O}03]{KO2}
T.~Kobayashi and B.~{\O}rsted.
\newblock Analysis on the minimal representation of\/ {$\mbox{\rm O}(p,q)$}.{$\;$}{{\rm{II}}}. {B}ranching laws.
\newblock \emph{Adv. Math.}, \textbf{180}(2), (2003), pp. 513--550.
Available at \url{https://doi.org/10.1016/S0001-8708(03)00013-6}.

\bibitem[BR04]{bernstein2004estimates}
J.~Bernstein and A.~Reznikov.
\newblock Estimates of automorphic functions.
\newblock \emph{Mosc. Math. J}, \textbf{\textbf{4}}(1), (2004), pp. 19--37.
Available at \url{http://mi.mathnet.ru/eng/mmj141}.
\end{thebibliography}
\end{frame}

\end{document}
%talk is 15 minutes long
