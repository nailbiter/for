%make suugakkai_16_aux/Makefile
\documentclass[notes,notheorems]{beamer}
\mode<presentation>{\usetheme[secheader]{Boadilla}}
\usepackage{mystyle}
\usepackage{geometry}
\usepackage{amsmath}
\usepackage{xeCJK}
\usepackage{ruby}
\usepackage{enumerate}
\usepackage{setspace}
\usepackage{xypic}
\usepackage[all,cmtip]{xy}
\usepackage{bbm,ulem,float,mystyle}
\usepackage{caption}
\usepackage{subcaption}
\usepackage{setspace}
\usepackage{tikz-cd,array}
\usepackage{catchfilebetweentags}

\newcommand{\red}[1]{{\color[rgb]{0.6,0,0}#1}}
\newcommand{\Sol}{\mathcal{S}\mbox{ol}}
\newcommand{\Hom}{\mbox{\normalfont Hom}}
\newcommand{\D}{\mathcal{D}}
\newcommand{\A}{\mathcal{A}}
\newcommand{\Co}{\mathbb{C}}
\newcommand{\X}{\mathbb{X}}
\renewcommand{\setminus}{\backslash}
\newcommand{\nin}{\not\in}
\newcommand{\tmop}[1]{\ensuremath{\operatorname{#1}}}
\newcommand{\tmtextbf}[1]{{\bfseries{#1}}}
\newcommand{\tmtextit}[1]{{\itshape{#1}}}
\newcommand{\mss}{//}
\newcommand{\mbb}{\backslash\backslash}
\newcommand{\mmm}{\mid\mid}
\catcode`\<=\active \def<{
\fontencoding{T1}\selectfont\symbol{60}\fontencoding{\encodingdefault}}
\catcode`\>=\active \def>{
\fontencoding{T1}\selectfont\symbol{62}\fontencoding{\encodingdefault}}
\newcommand{\assign}{:=}
\newcommand{\comma}{{,}}
\newcommand{\um}{-}
\newcommand{\sol}{\mathcal{S}ol(\R^{p,q};\lambda,\nu)}
\newcommand{\Op}{\mbox{\normalfont Op}}
\newcommand{\Res}{\operatorname{Res}\displaylimits}
\newcommand{\OpR}{\mbox{\it R}}

\setbeamertemplate{theorem}[ams style]
\setbeamertemplate{theorems}[numbered]

\makeatletter
    \ifbeamer@countsect
      \newtheorem{theorem}{\translate{Theorem}}[section]
    \else
      \newtheorem{theorem}{\translate{Theorem}}
    \fi
    \newtheorem{corollary}{\translate{Corollary}}
    \newtheorem{fact}{\translate{Fact}}
    \newtheorem{lemma}{\translate{Lemma}}
    \newtheorem{problem}{\translate{Problem}}
    \newtheorem{solution}{\translate{Solution}}

    \theoremstyle{definition}
    \newtheorem{definition}{\translate{Definition}}
    \newtheorem{definitions}{\translate{Definitions}}

    \theoremstyle{example}
    \newtheorem{example}{\translate{Example}}
    \newtheorem{examples}{\translate{Examples}}


    % Compatibility
    \newtheorem{Beispiel}{Beispiel}
    \newtheorem{Beispiele}{Beispiele}
    \theoremstyle{plain}
    \newtheorem{Loesung}{L\"osung}
    \newtheorem{Satz}{Satz}
    \newtheorem{Folgerung}{Folgerung}
    \newtheorem{Fakt}{Fakt}
    \newenvironment{Beweis}{\begin{proof}[Beweis.]}{\end{proof}}
    \newenvironment{Lemma}{\begin{lemma}}{\end{lemma}}
    \newenvironment{Proof}{\begin{proof}}{\end{proof}}
    \newenvironment{Theorem}{\begin{theorem}}{\end{theorem}}
    \newenvironment{Problem}{\begin{problem}}{\end{problem}}
    \newenvironment{Corollary}{\begin{corollary}}{\end{corollary}}
    \newenvironment{Example}{\begin{example}}{\end{example}}
    \newenvironment{Examples}{\begin{examples}}{\end{examples}}
    \newenvironment{Definition}{\begin{definition}}{\end{definition}}
\makeatother

\setCJKmainfont{Hiragino Mincho Pro}
\renewcommand{\thefootnote}{\fnsymbol{footnote}}
\hypersetup{colorlinks=true,urlcolor=blue}
\urlstyle{same}

%%%%%%%%%% Start TeXmacs macros
\newcommand{\nobracket}{}
%%\newcommand{\tmop}[1]{\ensuremath{\operatorname{#1}}}
%%\newcommand{\tmtextbf}[1]{{\bfseries{#1}}}
%%\newcommand{\tmtextit}[1]{{\itshape{#1}}}
\newcommand{\tmtextmd}[1]{{\mdseries{#1}}}
\newcommand{\tmtextrm}[1]{{\rmfamily{#1}}}
\newcommand{\tmtextup}[1]{{\upshape{#1}}}
%%%%%%%%%% End TeXmacs macros

\newenvironment{setting}{\begin{exampleblock}{Setting.}\it}{\end{exampleblock}}
\newenvironment{question}{\begin{block}{Problem.}\it}{\end{block}}
\newenvironment{prop}[1][]{\begin{block}{Proposition#1.}\it}{\end{block}}
\makeatletter
\def\th@mystyle{%
    \normalfont % body font
    \setbeamercolor{block title example}{bg=orange,fg=white}
    \setbeamercolor{block body example}{bg=orange!20,fg=black}
    \def\insertpropblockenv{exampleblock}
  	}
\makeatother
\theoremstyle{mystyle}
\newtheorem*{remark}{Remark.}

%custom formatting settings
\setlength{\parskip}{1em}

%%\newcommand{\yipx}{Y^{p+1, q + 1}_{+, x}}
%%\newcommand{\yipy}{Y^{p, q + 1}_{+, y}}
%%\newcommand{\yimx}{Y^{p+1, q + 1}_{-, x}}
%%\newcommand{\yimy}{Y^{p, q + 1}_{-, y}}
%%\newcommand{\pipx}{\pi^{p+1, q + 1}_{+, x}}
%%\newcommand{\pipy}{\pi^{p, q + 1}_{+, y}}
%%\newcommand{\pimx}{\pi^{p+1, q + 1}_{-, x}}
%%\newcommand{\pimy}{\pi^{p, q + 1}_{-, y}}
%%\newcommand{\tisevenjapb}{\mbox{は偶数の場合}}
%%\newcommand{\tisoddjapb}{\mbox{は奇数の場合}}
%%\newcommand{\mystack}[2]{\begin{array}{c}#1,\\#2\end{array}}
%%%%\newcommand{\pipxStack}[1][]{\mystack{\pipx}{x\d{\in} \Azeven(p\d{+}1,q\d{+}1)#1}}
%%%%\newcommand{\pimxStack}[1][]{\mystack{\pimx}{x\d{\in} \Azeven(q\d{+}1,p\d{+}1)#1}}
%%%%\newcommand{\pipyStack}[1][]{\mystack{\pipy}{y\d{\in} \Azeven(p,q\d{+}1)#1}}
%%%%\newcommand{\pimyStack}[1][]{\mystack{\pimy}{y\d{\in} \Azeven(q\d{+}1,p)#1}}
%%\newcommand{\tzo}{2\Z+1}
%%\newcommand{\tz}{2\Z}
%%\newcommand{\tno}{2\N+1}
\newcommand{\Azeven}{A_0^{\rm\footnotesize even}}

\title{不定値直交群$O(p,q)$の対称性破れ作用素}

% A subtitle is optional and this may be deleted

\author[小林、\underline{レオンチエフ}]{小林俊行\inst{1} \and \underline{レオンチエフ アレックス}\inst{2}}

\institute[東大数理] % (optional, but mostly needed)
{
  \inst{1}%
  東京大学\\
  大学院数理科学研究科{・}\\カブリ数物連携宇宙研究機構
  \and
  \inst{2}%
  東京大学\\
  大学院数理科学研究科
  }
% - Use the \inst command only if there are several affiliations.
% - Keep it simple, no one is interested in your street address.

  \date[第56合同シンポジウム]{第56回実函数論・函数解析学 合同シンポジウム講演集\\2017年8月21日--8月23日\\
於お茶の水大学}
% - Either use conference name or its abbreviation.
% - Not really informative to the audience, more for people (including
%   yourself) who are reading the slides online

\subject{表現論}

\begin{document}
\section{}
\begin{frame}\titlepage\end{frame}

\section{Branching problem}

\begin{frame}{}
Suppose $G \supset G'$ are reductive groups and $\pi$ is an irreducible
representation of $G$. If we restrict $\pi$ to $G'$, it is no longer
irreducible in general.\\
	\[
	\xymatrixrowsep{10pt}
	\xymatrixcolsep{50pt}
	\xymatrix{
		\pi:&G\ar[r]&GL_{\mathbb{C}}(V)&\left( \dim V=\infty \right)\\
	&\bigcup&&\\
	&G'\ar@{-->}[uur]_{\pi\mid_{G'}}
	}
\]
\begin{block}{\underline{Branching problem}  (in a wider sense)}
	\centerline{\large Understand $\pi\kern-0.1cm\mid_{G'}$.}
\end{block}

These are well-studied (i.e. combinatorial algorithm) for $\pi$
finitely-dimensional and $G$:compact. In this setting, $\pi$ always splits
into a direct sum
\begin{equation*}
	\pi\kern-0.1cm\mid_{G'} =  \bigoplus_{\tau \in\widehat{G'}} m (\pi, \tau) \tau
\end{equation*}
of irreducibles $\tau$ of $G'$.
\end{frame}
\begin{frame}{}
However, when
\begin{equation*}
	\begin{array}[]{c}
		\dim\pi=\infty,\mbox{ and}\\
		G,G'\mbox{: not compact,}
	\end{array}
\end{equation*}
the
situation becomes much more involved and was not studied seriously before
Kobayashi's theory appeared in 90s.

In particular, several examples that show
that behaviour is very wild in general were constructed
\cite{Kobayashi2005}.

 An idea to understand restriction $\pi\kern-0.1cm \mid_{G'}$ that is not
discretely decomposable: compare it with irreducible representations $\tau$ of
the subgroup $G'$, i.e. to study the space{

}{\hspace{0.5\columnwidth}}$\tmop{Hom}_{G'} (\pi \mid_{G'}, \tau)${

}of \underline{symmetry breaking operators} (SBOs, for short).
\end{frame}

\section{$\mathcal{A}\mathcal{B}\mathcal{C}$ program for branching}

\begin{frame}{}
\quad In \cite{kobayashi2015program} T. Kobayashi introduced the
far-reaching program for studying branching of noncompact groups, which can be
summarized as follows:{

}\qquad$(\mathcal{A})$\quad$\mathcal{A}$bstract features of the
representation;{

}\qquad$(\mathcal{B})$\quad$\mathcal{B}$ranching law of $\pi\kern-0.15cm\mid_{G'}$; {

}\qquad$(\mathcal{C})$\quad$\mathcal{C}${onstruction} {of}
{SBOs}.
\end{frame}
\begin{frame}{}
	
The main theme of this work is Program $\mathcal{C}$ for ``standard
representations'' with focus on:\vspace{-1em}{

}\qquad$(\mathcal{C}1)$\quad Construct SBOs;{

}\qquad$(\mathcal{C}2)$\quad Classify all SBOs;{

}\qquad$(\mathcal{C}3)$\quad Study functional equations among SBOs;{

}\qquad$(\mathcal{C}4) \quad$Find residue formulae for SBOs;{

}\qquad$(\mathcal{C}5)$\quad Find images of SBOs.{

}\vspace{-1em}\quad The subprogram $(\mathcal{C}1) - (\mathcal{C}5)$ was proposed by
Kobayashi-Speh in their book \cite{kobayashi2015symmetry} for real rank 1
pair
\begin{equation*}
(G, G') = (O (n + 1, 1), O (n, 1)).
\end{equation*}
\vspace{-1.5em}
\begin{block}{\underline{Goal}:}
	Extend this to higher rank case:
\vspace{-1em}
	\begin{equation*}
		(G, G') = (O (p + 1, q + 1), O (p, q + 1)).
	\end{equation*}
\vspace{-2em}
\end{block}
\end{frame}
\begin{frame}{}
For ``standard representations'' we work with \underline{spherical degenerate principal series representations
}:\begin{equation*}
	\begin{array}[]{c}
		I(\lambda)=\tmop{Ind}_P^G(\mathbb{C}_{\lambda}),\quad \lambda\in\mathcal{C},\\
		I(\nu)=\tmop{Ind}_{P'}^{G'}(\mathbb{C}_{\nu}),\quad \nu\in\mathcal{C},
	\end{array}
\end{equation*}
where $P \subset G$ is the maximal parabolic subgroup with the Levi part:
\begin{equation*}
{MA} \simeq O (p, q) \times \{ \pm 1 \}
\times \mathbbm{R},
\end{equation*}
$P' = P \cap G'$ is a maximal parabolic of $G'$.
\end{frame}
\begin{frame}{}
	\begin{block}{Conformal viewpoint}
		Geometrically $I(\lambda)=C^\infty(X,\mathcal{L}_{\lambda})$ arise from conformal geometry:
		\centerline{\scalebox{0.8}{
		\newdir{:=}{{}}
		\xymatrix{
			& \mathcal{L}_\lambda\mbox{ :conformally equivariant line bundle},\lambda\in\mathbb{C}
			\ar[d]\\
  		G=O(p+1,q+1)
		\ar@/^2pc/[r] &G/P\simeq (\Sp^p\times\Sp^q)/\left\{ \pm I \right\}\\
		P=MAN\ar@{:=}[u]_{\hspace{-0.25cm}\bigcup}
		\ar@/^2pc/[rd]^{{\begin{array}{c}\; \\\mbox{conformal transformations}\end{array}}}
		%\mbox\newline oeueou}\vspace{0.8cm}}
		&\\
	M_+N=O(p,q)\ltimes \mathbb{R}^{p,q}
	\ar@{:=}[u]_{\hspace{-0.25cm}\bigcup}
	\ar@/^2pc/[r]^{\mbox{isometries}}&
	\mathbb{R}^{p,q}=\left( \mathbb{R}^{p+q},ds^2=dx_1^2+\ldots+dx_p^2-dx_{p+1}^2-\ldots-dx_{p+q}^2 \right)\ar@{^{(}->}[uu]
	_{\mbox{conformal 
	compactification}}}
}}
	\end{block}
\begin{remark}
\quad Works
\cite{kobayashi2013finite},\cite{kobayashi2014classification}
regarding the Program $\mathcal{A}$ (a priori estimate) for this setting,
imply that
\begin{equation*}
\dim \tmop{Hom}_{G'} (I (\lambda) \mid_{G'}, J
\nobracket (\nu)
\end{equation*}
is uniformly bounded in $(\lambda, \nu) \in \mathbbm{C}^2$.

\end{remark}
\end{frame}
\section{Classification of SBOs}
\begin{frame}{}
	Use the strategy developed in \cite{kobayashi2015symmetry}. The geometry is a little
	more complicated.
	\begin{fact}[Distribution kernel for SBOs]
Applying the general statement of
\cite{kobayashi2015symmetry} to our concrete setting we get:
	\centerline{
		\scalebox{0.9}{
		\xymatrixcolsep{5pc}
		\xymatrix{\Hom_{G'}(I(\lambda),J(\nu))\ar[r]^{\simeq} \ar@/^2pc/[rr]^{\mathcal{S}}
		&\left( \mathcal{D}'(G/P,\mathcal{L}_{n-\lambda}) \otimes\mathbb{C}_\nu \right)^{P'}
	\ar[r]_-{F\mapsto \supp(F)}\ar[d]^{\simeq}_{\mbox{rest}}
	&2^{P'\backslash G/P}\\
	&\sol\subset\mathcal{D}'(\R^{p,q})\ar[lu]^{\mbox{Op}}_{\simeq}&
	}
	}
}
	\end{fact}
\end{frame}
\begin{frame}{}
	We set
	\begin{equation*}
		\begin{array}[]{l}
X \assign G / P \simeq (\mathbbm{S}^p \times
\mathbbm{S}^q) / \{ \pm I \}, \\
Y \assign \{ [\xi, \eta] \in G / P \mid \xi_{p +
1} = 0 \} \simeq (\mathbbm{S}^{p - 1} \times \mathbbm{S}^q) / \{ \pm I \}, \\
C \assign \{ [\xi, \eta] \in G / P \mid \xi_1 =
\eta_{q + 1} \},\\
\left\{ o \right\}:=\left\{ \left[ 1,0_{p+q}1 \right] \right\}.
		\end{array}
	\end{equation*}
	\begin{theorem}[description of double coset space $P'
\backslash G / P$]
We have $P' \backslash G / P$ being equal to
  \begin{figure}[H]
    \centering
    \begin{subfigure}[t]{0.3\textwidth}
	    \xymatrixrowsep{0.5pc}
	    \xymatrix{&X\ar@{-}[ld]_1\ar@{-}[rd]^1&\\Y\ar@{-}[rd]_1&&C\ar@{-}[ld]^1\\&Y\cap C\ar@{-}[dd]^{p+q-2}&\\&&\\&\{[0]\}&}
	\caption{when $p>1$}
    \end{subfigure}
    ~ %add desired spacing between images, e. g. ~, \quad, \qquad, \hfill etc. 
      %(or a blank line to force the subfigure onto a new line)
    \begin{subfigure}[t]{0.3\textwidth}
	    \xymatrixrowsep{0.5pc}
	    {\xymatrix{&X\ar@{-}[ld]_1\ar@{-}[rd]^1&\\Y\ar@{-}[rddd]_{p+q-2}&&C\ar@{-}[lddd]^{p+q-2}\\&&\\&&\\&\{[0]\}&}}
	    \vspace{0.7em}
	\caption{when $p=1$}
    \end{subfigure}
\vspace{-0.8em}
\end{figure}
	\end{theorem}
\end{frame}
\begin{frame}{Differential SBO}
	\begin{fact}[classification of differential SBO, see \cite{kobayashi2015branching}]
For $(\lambda, \nu) \in / / \assign \{
(\lambda, \nu) \in \mathbbm{C}^2 \mid \nu - \lambda \in 2\mathbbm{N} \}$, let
\begin{equation*}
	\tilde{R}_{\lambda, \nu}^{\{ o \}} =
\tmop{Rest}_{x_p = 0} \circ \tilde{C}_{\nu - \lambda}^{\lambda - \frac{n -
1}{2}} \left( - \Delta_{\mathbbm{R}^{p - 1, q}}, \frac{\partial}{\partial x_p}
\right),
\end{equation*}
where $\tilde{C} (s, t)$ is a polynomial of two-variables, obtained by
inflation of the renormalized Gegenbauer polynomial, defined as in
[\ref{kobayashi2015symmetry}, (16.3)].{

}It is a differential SBO (i.e. $\mathcal{S}\mbox{upp} (\tilde{R}_{\lambda,
\nu}^{\{ [o] \}}) = \{ 0 \}$) and any differential SBO is proportional to it.
	\end{fact}
\end{frame}
\begin{frame}{Regular SBO}
	\begin{theorem}[construction of regular SBO]
For
$(\lambda, \nu) \in \mathbbm{C}^2$ with $\tmop{Re} (\nu) < 0$ and $\tmop{Re}
(\lambda + \nu - n) > 0$ the continuous function
\begin{equation*}
	| x_p |^{\lambda + \nu - n} | Q_{p, q} |^{- \nu}
\end{equation*}
is a member of $\mathcal{S} \mbox{ol} (\mathbbm{R}^{p, q} ; \lambda, \nu)$
and we let
\vspace{-0.5em}
\begin{equation*}
	R_{\lambda, \nu}^X \assign \tmop{Op} (| x_p
|^{\lambda + \nu - n} | Q_{p, q} |^{- \nu}) .
\end{equation*}
	\end{theorem}
\vspace{-0.4em}
	\begin{theorem}[normalization of regular SBO]
$R_{\lambda, \nu}^X$ can be meromorphically extnended to $(\lambda, \nu) \in
\mathbbm{C}^2$. Moreover,
\vspace{-1.1em}
\begin{equation*}
	\tilde{R}_{\lambda, \nu}^X \assign \frac{1}{\Gamma
\left( \frac{\lambda - \nu}{2} \right) \Gamma \left( \frac{\lambda - \nu}{2}
\right) \Gamma \left( \frac{\lambda + \nu - n + 1}{2} \right)} R_{\lambda,
\nu}^X
\end{equation*}
becomes an SBO depending holomorphically on $(\lambda, \nu)$. It then
vanishes only on a discrete set of $\mathbbm{C}^2$ which we can determine.
	\end{theorem}
\end{frame}
\begin{frame}{Singular SBOs}
	\begin{theorem}[construction of singular SBOs]
$S
= X, Y, S$, and the following operators $R_{\lambda, \nu}^S$ and
$\tilde{R}_{\lambda, \nu}^X$ are symmetry breaking operators from $I (\lambda)
\mid_{G'}$ to $J (\nu)$, which depend holomorphically on $(\lambda, \nu) \in
D_S$ and are renormalizations of $\tilde{R}_{\lambda, \nu}^X$. Moreover,
$\mathcal{S} \tmop{upp} (R_{\lambda, \nu}^S) \subseteq S$ with ``='' holding
generically and are determined explicitly.\\
\centerline{\begin{tabular}{|l|l|}
  \hline
  $\tilde{R}_{\lambda, \nu}^S$ & $D_S$\\
  \hline
  $\widetilde{\tilde{R}}_{\lambda, \nu}^X$ & $\mid \mid \mid$\\
  \hline
  $\tilde{R}_{\lambda, \nu}^Y$ & \textbackslash\textbackslash\\
  \hline
  $\tilde{R}_{\lambda, \nu}^C$ & $\mid \mid \mid$\\
  \hline
\end{tabular}}

Let us explain the notation in the table:
	\begin{equation*}
		\begin{array}[]{l}
		\mid \mid \mid \assign \left\{ (\lambda, \nu) \in \mathbbm{C}^2 \mid
\nu \in - 2\mathbbm{N} \; \tmop{or} \; \nu \equiv q + 1 \; \tmop{mod} \; 2
\right\},\\
\backslash\backslash \assign \{ (\lambda, \nu) \in \mathbbm{C}^2 \mid
\lambda + \nu - n + 1 \in - 2\mathbbm{N} \},\\
\mid \mid \assign \{ (\lambda, \nu) \in \mathbbm{C}^2 \mid \nu \in 1 +
2\mathbbm{N} \} . 
		\end{array}
	\end{equation*}
	\end{theorem}
\end{frame}
\begin{frame}{}
	\begin{theorem}[dimension of SBO space]
		For all
$(\lambda, \nu) \in \mathbbm{C}^2$, 
\begin{equation*}
\dim \tmop{Hom}_{G'} (I (\lambda) \mid_{G'}, J
(\nu)) \in \{ 1, 2 \}.
\end{equation*}
	\end{theorem}
	\begin{theorem}[classification of SBOs]
		\begin{equation*}
			\begin{array}[]{l}
				p=1\Rightarrow \Hom_{G'}\left( I(\lambda)\kern-0.1cm\mid_{G'},J(\nu) \right)=\\
				\left\{  \begin{array}[]{ll}
  \mathbbm{C} \tilde{R}_{\lambda, \nu}^X, & (\lambda, \nu) \nin (/ / \cap \mid
  \mid \mid) \cup (\mid \mid \cap \backslash\backslash),\\
  \mathbbm{C} \widetilde{\tilde{R}}_{\lambda, \nu}^X \oplus \mathbbm{C}
  \tilde{R}_{\lambda, \nu}^{\{ o \}}, & (\lambda, \nu) \in (/ / \cap \mid \mid
  \mid) - (\mid \mid \cap \backslash\backslash),\\
  \mathbbm{C} \tilde{R}_{\lambda, \nu}^C \oplus \mathbbm{C}
  \tilde{R}^Y_{\lambda, \nu}, & (\lambda, \nu) \in (\mid \mid \cap
  \backslash\backslash) - / /,\\
  \mathbbm{C} \widetilde{\tilde{R}}_{\lambda, \nu}^X, & (\lambda, \nu) \in
  \mid \mid \cap \backslash\backslash \cap / /,
				\end{array}
					\right.\\
				p>1\Rightarrow \Hom_{G'}\left( I(\lambda)\kern-0.1cm\mid_{G'},J(\nu) \right)=\\
				\left\{  \begin{array}[]{ll}
  \mathbbm{C} \tilde{R}_{\lambda, \nu}^X, & (\lambda, \nu) \nin / / \cap \mid
  \mid \mid,\\
  \mathbbm{C} \widetilde{\tilde{R}}_{\lambda, \nu}^X \oplus \mathbbm{C}
  \tilde{R}_{\lambda, \nu}^{\{ o \}}, & (\lambda, \nu) \in / / \cap \mid \mid
  \mid.
				\end{array}
					\right.\\
			\end{array}
		\end{equation*}
	\end{theorem}
\end{frame}

\section{Solutions to $(\mathcal{C}3) - (\mathcal{C}5)$}
\begin{frame}
	\begin{theorem}[spectrum for spherical vectors]
	Let $n
	\assign p + q \hspace{0.27em} (p, q \ge 1)$ as before.
	\begin{equation*}
		\tilde{R}_{\lambda, \nu}^X 1_{\lambda} =
\frac{2^{1 - \lambda} \pi^{n / 2}}{\Gamma \left( \frac{\lambda}{2} \right)
\Gamma \left( \frac{\lambda + 1 - q}{2} \right) \Gamma \left( \frac{q - \nu +
1}{2} \right)} 1_{\nu}
	\end{equation*}
	\end{theorem}
	\begin{remark}
This Theorem was known in
Bernstein--Reznikov 2004 for $p = q = 1$ (i.e. $G' \simeq \tmop{SL} (2,
\mathbbm{R})$) and in \cite{kobayashi2015symmetry} for $q = 0$.
	\end{remark}
\end{frame}
\begin{frame}{Residue formulae}
For $(\lambda,\nu)\nin //$, we set
\begin{equation*}
K_{\lambda, \nu}^X \assign \frac{| x_p |^{\lambda +
\nu - n}}{\Gamma \left( \frac{\lambda + \nu - n + 1}{2} \right)} \cdot \frac{|
Q |^{- \nu}}{\Gamma \left( \frac{1 - \nu}{2} \right)} \in \mathcal{S}
\mbox{ol} (\mathbbm{R}^n ; \lambda, \nu).
\end{equation*}
Then $\tilde{R}_{\lambda, \nu}^X = \frac{1}{\Gamma \left( \frac{\lambda -
\nu}{2} \right)} \tmop{Op} (K_{\lambda, \nu})$ and we recall that the
left-hand side extends to a family of SBOs with holomorphic parameter
$(\lambda, \nu) \in \mathbbm{C}^2.$
\begin{theorem}[residue formula]
Suppose $(\lambda,\nu)\in //$, namely $l \assign \frac{1}{2}  (\nu - \lambda) \in
\mathbbm{N}$. Then we have
\begin{equation*}
	\tilde{R}_{\lambda, \nu}^X = \frac{(- 1)^l l!
\pi^{(n - 2) / 2}}{2^{\nu + 2 l - 1}} \cdot \frac{\sin \left( \frac{1 + q -
\nu}{2} \pi \right)}{\Gamma \left( \frac{\nu}{2} \right)} \tilde{R}_{\lambda,
\nu}^{\{ o \}}, \quad \mbox{for }(\lambda, \nu) \in //.
\end{equation*}
\end{theorem}
\begin{remark}
	This was shown for $q = 0$ in
\cite{kobayashi2015symmetry}.
\end{remark}
\end{frame}
\begin{frame}{Knapp--Stein intertwining operators}
	\begin{definition}
		The \underline{Knapp--Stein operator} is a $G$-intertwining operator defined as\begin{equation*}
			\begin{array}[]{l}
				\tilde{\mathbb{T}}_\lambda^G:I(\lambda)\to I(n-\lambda)\\
				f\mapsto q_T(\lambda)\left(\myabs{Q_{p,q}}^{\lambda-n}\star f  \right)
			\end{array}
		\end{equation*}
		where $q_T (\lambda)$ is explicitly given by Gamma factors. 
	\end{definition}
		We compare the
		composition $\tilde{R}_{n - \lambda, \nu}^X \circ\tilde{\mathbbm{T}}_{\lambda}^G$ and $\tilde{R}_{\lambda, \nu}^X$. They should
		be proportional to each other.
\end{frame}
\begin{frame}{}
		We compare the
		composition $\tilde{R}_{n - \lambda, \nu}^X \circ\tilde{\mathbbm{T}}_{\lambda}^G$ and $\tilde{R}_{\lambda, \nu}^X$. They should
		be proportional to each other.
		\centerline{\xymatrix{I(\lambda)\ar[d]^{\tilde{\mathbb{T}}_{\lambda}^G}\ar@{-->}[rd]^{R_{\lambda,\nu}^X}&\\
		I(n-\lambda)\ar[r]^{R^X_{n-\lambda,\nu}}&J(\nu)}}
		For $G' = O (p, q + 1)$ we similarly define $\tilde{\mathbbm{T}}_{\nu}^{G'}$:
		\centerline{
			\xymatrix{
				I(\lambda)\ar@{-->}[rd]^{R_{\lambda,\nu}^X}\ar[r]^{R_{\lambda,n-1-\nu}^X}&J(n-1-\nu)\ar[d]^{\tilde{\mathbb{T}}_{n-1-\nu}^{G'}}\\
				&J(\nu)
			}}
\end{frame}
\begin{frame}
  \begin{figure}[H]
    \centering
    \begin{subfigure}[t]{0.3\textwidth}
\xymatrix{I(\lambda)\ar[d]^{\tilde{\mathbb{T}}_{\lambda}^G}\ar@{-->}[rd]^{R_{\lambda,\nu}^X}&\\I(n-\lambda)\ar[r]^{R^X_{n-\lambda,\nu}}&J(\nu)}
    \end{subfigure}
    ~ %add desired spacing between images, e. g. ~, \quad, \qquad, \hfill etc. 
      %(or a blank line to force the subfigure onto a new line)
    \begin{subfigure}[t]{0.3\textwidth}
			\xymatrix{I(\lambda)\ar@{-->}[rd]^{R_{\lambda,\nu}^X}\ar[r]^{R_{\lambda,n-1-\nu}^X}&J(n-1-\nu)\ar[d]^{\tilde{\mathbb{T}}_{n-1-\nu}^{G'}}\\&J(\nu)}
    \end{subfigure}
    \end{figure}
    \vspace{-1em}
	\begin{theorem}[functional identities]
	Let $n = p +
	q, (p, q \geqslant 1)$ as before. We have
	\begin{equation*}
		\begin{array}[]{l}
\tilde{R}_{n - \lambda, \nu}^X \circ
\tilde{\mathbbm{T}}_{\lambda}^G = q_X^{X T} (\lambda, \nu)
\tilde{R}^X_{\lambda, \nu} \\
\tilde{\mathbbm{T}}_{n - 1 - \nu}^{G'} \circ
\tilde{R}_{\lambda, n - 1 - \nu}^X = q_X^{T X} (\lambda, \nu)
\tilde{R}_{\lambda, \nu}^X,\\
	\mbox{where}\\
q_X^{X T} (\lambda, \nu) = \ldots
q_X^{T X} (\lambda, \nu) = \frac{\pi^{\frac{n -
2}{2}} \sin \left( \frac{p - \nu}{2} \pi \right) 2^{1 - n - \nu}}{\Gamma
\left( \frac{n - 1 - \nu}{2} \right)} \left\{ \begin{array}{ll}
  \Gamma \left( \frac{1 - \nu}{2} \right), & p = 1\\
  1, & n \in 2\mathbbm{Z}\\
  \ldots, & \ldots
\end{array} \right.
		\end{array}
	\end{equation*}
	\end{theorem}
	\begin{remark}
		This was shown in
		\cite{kobayashi2015symmetry} for $q = 0$.
	\end{remark}
\end{frame}
\begin{frame}{Images of SBOs}
	\begin{theorem}
		We can compute images
		of every SBOs constructed above for every $(\lambda, \nu) \in \mathbbm{C}^2$.
	\end{theorem}
\end{frame}
\begin{frame}{}
	最後に、上記の結果の応用としてZuckerman導来函手加群の間の対称性破れ作用素の問題を論じる。
\cite[(5.1.1)]{KO2}にg{倣}って$p>1${かつ}$q\ge1$のときに\begin{equation*}
	A_0(p,q):=\left\{ \lambda\in\Z+\frac{p+q}{2}\;:\;\lambda>-1 \right\}
\end{equation*}とおくと、
\cite{KO2}で
%16-b
示したように、$\lambda\in A_0(p,q)$に対して
$O(p,q)$の既約ユニタリ表現$\pi_{\pm,\lambda}^{p,q}$
%%16-c
が定まる。(\cite{KO2}では$\pi_{\pm,\lambda}^{p,q}$を定義するにあたって5種類の
特徴づけが与えら{れ}、それらは互いに
同値であることが示されている。その特徴づけの1つは
Zuckerman導来函手加群で記載される。)
以下では、この加群の間に対称性破れ作用素がある{か}について論じる。簡単のため、$A_0^{\mbox{\scriptsize even}}(p,q):=\left\{ \lambda\in A_0(p,q)\mid \lambda-\frac{p-q}{2}+1\in2\Z \right\}$とおく。
%%16-a
以下では\begin{equation*}
	g(t):=\left\{ \begin{array}[]{ll}
		1&\left(t\in2\N+\frac{1}{2}  \right)\\[10pt]
		0&t\nin \left(2\N+\frac{1}{2}  \right)
	\end{array}\right.,\quad h(t):=\left\{ \begin{array}[]{ll}
		1&\left(  t<\frac{q}{2}\right)
		\\[10pt]
		0&\left( t\ge\frac{q}{2} \right)
\end{array}\right.
\end{equation*}とおく。
\end{frame}
\begin{frame}
\begin{theorem}[Zuckerman導来加群函手{$\pi_{\pm,\lambda}^{p,q}$}間の対称性破れ作用素の存在問題]
	$n=p+q\;(p,q\ge1),\;n':=n-1$とする。
	以下では
	\vspace{-1em}
\begin{equation*}
                \begin{array}[]{c}
                        x\in\left\{\begin{array}[]{ll}
                                \Azeven(p+1,q+1),&\delta=+\mbox{ のとき}\\
                                \Azeven(q+1,p+1),&\delta=-\mbox{ のとき}\\
                        \end{array}\right.\\
                        y\in\left\{\begin{array}[]{ll}
                                \Azeven(p,q+1),&\varepsilon=+\mbox{ のとき}\\
                                \Azeven(q+1,p),&\varepsilon=-\mbox{ のとき}\\
                        \end{array}\right.
		\end{array}
	\vspace{-1em}
	\end{equation*}
	と仮定する。このとき{、}
	$\Hom_{G'}\left(\pi_{\delta,x}^{p+1,q+1}\kern-0.3em\mid_{G'} ,\pi_{\varepsilon,y}^{p,q+1} \right)$の次元と基底を具体的に決定出来る。
%%
%%%%	%%%%%%%%%%%%%%%%%%%%%%%%%%%%%%%%
\renewcommand{\mystack}[2]{\begin{array}{c}#1,\\#2\end{array}}
\newcommand{\mytable}[9]{
$\begin{array}{|@{}c@{}|@{}c@{}|@{}c@{}|}
  \hline
	#1& #2&#3\\
  \hline
	#4& #5&#6\\
  \hline
	#7& #8&#9\\
  \hline
\end{array} \newline$
}
\newcommand{\mytableFourTwo}[8]{
$\begin{array}{|@{}c@{}|@{}c@{}|}
  \hline
	#1& #2\\
  \hline
	#3& #4\\
  \hline
	#5& #6\\
  \hline
	#7& #8\\
  \hline
\end{array} \newline$
}
\newcommand{\mytableThreeTwo}[6]{
$\begin{array}{|@{}c@{}|@{}c@{}|}
  \hline
	#1& #2\\
  \hline
	#3& #4\\
  \hline
	#5& #6\\
  \hline
\end{array} \newline$
}
\newcommand{\commonShift}{\hspace*{-0.5cm}}
%%%%%%%%%%%%%%%%%%%%%%%%%%%%%%%%%%%%
\begin{enumerate}[(1)]
	\item $p=1,q\in2\Z$
		\\
\hspace*{0cm}\commonShift\mytableThreeTwo	%#1
{}		{\pimyStack[\mid y\geq q/2]}
{\pipx}			{0}
{\pimx}			{h\left( \frac{y-x-{1}/{2}}{2} \right)}
	\item $p=1,q\in2\Z+1$\\
\hspace*{0cm}\commonShift\mytableThreeTwo	%#2
{}		{\pimyStack[\mid y\geq q/2]}
{\pipx}			{0}
{\pimx}			{h\left( \frac{y-x-{1}/{2}}{2} \right)}
	\item $p,q\in2\Z$\\
\hspace*{-0cm}\commonShift\mytable	%3
{}		{\pipy}				{\pimy}
{\pipx}	{h\left(\frac{x-y-1/2}{2}\right)} 	{0}
{\pipx}	{0} 					{h\left( \frac{y-x}{2}-\frac{1}{4} \right)}
\item $p\in2\Z,q\in2\Z+1$\\
\commonShift\mytable	%4
{}		{\pipy}{\pimy}
{\pipx} {0}		{h\left( \frac{-1/2-x-y}{2} \right)}
{\pimx} {0} {h\left( \frac{y-x-1/2}{2} \right)}
\item $p\in2\Z+1,q\in2\Z$\\
\commonShift\mytable	%5
{}		{\pipy}		{\pimy}
{\pipx}			{0} 			{h\left( \frac{-1/2-x-y}{2} \right)}	
{\pimx} 			{0} 			{h\left( \frac{y-x-1/2}{2} \right)}
\item $p,q\in2\Z+1$\\
\commonShift\mytable	%6
{}		{\pipyStack}	{\pimyStack}
{\pipx}		{h\left( \frac{x-y-1/2}{2} \right)}			{0}
{\pimx}		{0}	{h\left( \frac{y-x-1/2}{2} \right)}	
\end{enumerate}

%%	$p=1,q\tisevenjapb$
%%		\\
%%\hspace*{0cm}\commonShift\mytableThreeTwo	%#1
%%{}		{\pimy}
%%{\pipx}		{h(y)(1-g(x-y))}
%%{\pimx}		{g\kern-0.05cm\left( {y-x} \right)}
\end{theorem}
\begin{remark}
	\begin{enumerate}[(1)]
		\item この定理では分岐則が離散分解する場合(一般論は\cite{10.2307/120963})とそうでない場合の両方が含まれている。
			分岐則が離散分解する場合、上記の分岐則は\cite[Thm. 3.3]{kobayashi1993}によって得られた公式と一致する。
		\item $q=0$の場合の類似の結果は\cite[Thms. 12.1 and 1.3]{kobayashi2015symmetry}で得られている。
	\end{enumerate}
	\vspace{-0.8em}
\end{remark}
\end{frame}

\begin{frame}[allowframebreaks]{References}
	
\begin{thebibliography}{KØSS15}
\bibitem[K93]{kobayashi1993}
T.~Kobayashi.
\newblock The restriction of ${A}_q \left( \lambda \right)$ to reductive
  subgroups.
\newblock \emph{Proc. Japan Acad. Ser. A Math. Sci.}, \textbf{69}(7), (1993),
  pp. 262--267.
Available at \url{http://dx.doi.org/10.3792/pjaa.69.262}.
  \bibitem[KM14]{kobayashi2014classification}T.~Kobayashi  and 
  T.~Matsuki.{\newblock} Classification of finite-multiplicity symmetric
  pairs.{\newblock} In \tmtextit{\tmtextrm{\tmtextup{\tmtextmd{Special Issue
  in honour of Professor Dynkin for his 90th birthday}}}},  volume~19,  pages 
  457--493. Springer, 2014.{\newblock}
  
  \bibitem[KO13]{kobayashi2013finite}T.~Kobayashi  and  T.~Oshima.{\newblock}
  Finite multiplicity theorems for induction and restriction.{\newblock}
  \tmtextit{Advances in Mathematics}, 248:921--944, 2013.{\newblock}
  
  \bibitem[Kob05]{Kobayashi2005}Toshiyuki Kobayashi.{\newblock}
  \tmtextit{Restrictions of Unitary Representations of Real Reductive Groups},
  pages  139--207.{\newblock} Birkh{\"a}user, 2005.{\newblock}
  
  \bibitem[Kob15]{kobayashi2015program}T.~Kobayashi.{\newblock} A program for
  branching problems in the representation theory of real reductive
  groups.{\newblock} In \tmtextit{\tmtextrm{\tmtextup{\tmtextmd{Special issue
  in honor of Vogan's 60th years birthday}}}},  volume  312,  pages  277--322.
  Birkh{\"a}user, 2015.{\newblock}
  
  \bibitem[KS15]{kobayashi2015symmetry}T.~Kobayashi  and  B.~Speh.{\newblock}
  \tmtextit{Symmetry Breaking for Representations of Rank One Orthogonal
  Groups},  volume \tmtextbf{238} of \tmtextit{Memoirs of the Amer. Math.
  Soc}.{\newblock} 2015.{\newblock}
  
  \bibitem[K{\O}SS15]{kobayashi2015branching}T.~Kobayashi, B.~{\O}rsted,
  P.~Somberg, and  V.~Sou{\v c}ek.{\newblock} Branching laws for verma
  modules and applications in parabolic geometry. I.{\newblock}
  \tmtextit{Advances in Mathematics}, 285:1796--1852, 2015.{\newblock}

\bibitem[K{\O}03]{KO2}
T.~Kobayashi and B.~{\O}rsted.
\newblock Analysis on the minimal representation of\/ {$\mbox{\rm O}(p,q)$}.{$\;$}{{\rm{II}}}. {B}ranching laws.
\newblock \emph{Adv. Math.}, \textbf{180}(2), (2003), pp. 513--550.
Available at \url{https://doi.org/10.1016/S0001-8708(03)00013-6}.
\end{thebibliography}
\end{frame}

\end{document}
%talk is 15 minutes long
