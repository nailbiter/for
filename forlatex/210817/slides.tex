%make suugakkai_16_aux/Makefile
\documentclass[notheorems]{beamer}
\mode<presentation>{\usetheme[secheader]{Boadilla}}
\usepackage{mystyle}
\usepackage{xcolor}
\usepackage{geometry}
\usepackage{amsmath}
\usepackage{xeCJK}
\usepackage{ruby}
\usepackage{ruby}
\usepackage{enumerate}
\usepackage{setspace}
\usepackage{xypic}
\usepackage[pdftex,color,all,cmtip]{xy}
\usepackage{bbm,ulem,float,mystyle}
\usepackage{caption}
\usepackage{subcaption}
\usepackage{setspace}
\usepackage{tikz-cd,array}
\usepackage{catchfilebetweentags}
\usepackage{color,soul}

\definecolor{red}{rgb}{1,0,0}
\setulcolor{red}

\newcommand{\kana}[2]{\ruby{#1}{#2}}
\newcommand{\doubt}[1]{\fbox{#1}}
\newcommand{\mynum}{}
\newcommand{\pause}{$\bullet$}
\newcommand{\slowly}[1]{\dashuline{#1}}
\newcommand{\continuously}[1]{\underline{#1}}
\newcommand{\badword}[1]{\uwave{#1}}
\renewcommand{\C}{\mathbb{C}}

\renewcommand{\Q}{Q_{p,q}}
\newcommand{\Ind}{\mbox{\normalfont Ind}}
\def\same{\;''\;}
\newcommand{\sbo}{\tmop{Hom}_{G'} (I(\lambda)\kern-0.1cm\mid_{G'}, J(\nu))}
\newcommand{\Supp}{\mathcal{S}\!{\it upp}}
\newcommand{\red}[1]{{\color[rgb]{0.6,0,0}#1}}
\newcommand{\Sol}{\mathcal{S}\mbox{ol}}
\newcommand{\D}{\mathcal{D}}
\newcommand{\A}{\mathcal{A}}
\newcommand{\Co}{\mathbb{C}}
\newcommand{\X}{\mathbb{X}}
\renewcommand{\setminus}{\backslash}
\newcommand{\nin}{\not\in}
\newcommand{\tmop}[1]{\ensuremath{\operatorname{#1}}}
\newcommand{\tmtextbf}[1]{{\bfseries{#1}}}
\newcommand{\tmtextit}[1]{{\itshape{#1}}}
\newcommand{\mss}{//}
\newcommand{\mbb}{\backslash\backslash}
\newcommand{\mmm}{\mid\mid}
\catcode`\<=\active \def<{
\fontencoding{T1}\selectfont\symbol{60}\fontencoding{\encodingdefault}}
\catcode`\>=\active \def>{
\fontencoding{T1}\selectfont\symbol{62}\fontencoding{\encodingdefault}}
\newcommand{\assign}{:=}
\newcommand{\comma}{{,}}
\newcommand{\um}{-}
\newcommand{\sol}{\mathcal{S}ol(\R^{p,q};\lambda,\nu)}
\newcommand{\Op}{\mbox{\normalfont Op}}
\newcommand{\Res}{\operatorname{Res}\displaylimits}
\newcommand{\OpR}{\mbox{\it R}}

\setbeamertemplate{theorem}[ams style]
\setbeamertemplate{theorems}[numbered]

\newtheorem{theorem}{定理}
\newtheorem{corollary}{{系}}
\newtheorem{fact}{{Fact}}
\newtheorem*{fact*}{{Fact}}
\newtheorem{lemma}{{命題}}
\newtheorem{problem}{{問題}}
\newtheorem{solution}{{Solution}}
\theoremstyle{definition}
\newtheorem{definition}{{定義}}
\theoremstyle{example}
\newtheorem{example}{{例}}
\newtheorem*{example*}{{例}}
\theoremstyle{remark}
\newtheorem*{remark}{注意}

%\setbeameroption{show only notes}

\setCJKmainfont{Hiragino Mincho Pro}
\renewcommand{\thefootnote}{\fnsymbol{footnote}}
\hypersetup{colorlinks=true,urlcolor=blue}
\urlstyle{same}

%%%%%%%%%% Start TeXmacs macros
\newcommand{\nobracket}{}
%%\newcommand{\tmop}[1]{\ensuremath{\operatorname{#1}}}
%%\newcommand{\tmtextbf}[1]{{\bfseries{#1}}}
%%\newcommand{\tmtextit}[1]{{\itshape{#1}}}
\newcommand{\tmtextmd}[1]{{\mdseries{#1}}}
\newcommand{\tmtextrm}[1]{{\rmfamily{#1}}}
\newcommand{\tmtextup}[1]{{\upshape{#1}}}
%%%%%%%%%% End TeXmacs macros

\newenvironment{setting}{\begin{exampleblock}{Setting.}\it}{\end{exampleblock}}
\newenvironment{question}{\begin{block}{Problem.}\it}{\end{block}}
\newenvironment{prop}[1][]{\begin{block}{Proposition#1.}\it}{\end{block}}
\makeatletter
\def\th@mystyle{%
    \normalfont % body font
    \setbeamercolor{block title example}{bg=orange,fg=white}
    \setbeamercolor{block body example}{bg=orange!20,fg=black}
    \def\insertpropblockenv{exampleblock}
  	}
\makeatother
\theoremstyle{mystyle}

%custom formatting settings
\setlength{\parskip}{1em}

\newcommand{\mystack}[2]{$\begin{array}{l}#1\\#2\end{array}$}
%%\newcommand{\yipx}{Y^{p+1, q + 1}_{+, x}}
%%\newcommand{\yipy}{Y^{p, q + 1}_{+, y}}
%%\newcommand{\yimx}{Y^{p+1, q + 1}_{-, x}}
%%\newcommand{\yimy}{Y^{p, q + 1}_{-, y}}
\newcommand{\pipx}{\pi^{p+1, q + 1}_{+, x}}
\newcommand{\pipy}{\pi^{p, q + 1}_{+, y}}
\newcommand{\pimx}{\pi^{p+1, q + 1}_{-, x}}
\newcommand{\pimy}{\pi^{p, q + 1}_{-, y}}
\newcommand{\tisevenjap}{\mbox{は偶数}}
\newcommand{\tisoddjap}{\mbox{は奇数}}
%%\newcommand{\tisevenjapb}{\mbox{は偶数の場合}}
%%\newcommand{\tisoddjapb}{\mbox{は奇数の場合}}
%%\newcommand{\mystack}[2]{\begin{array}{c}#1,\\#2\end{array}}
%%%%\newcommand{\pipxStack}[1][]{\mystack{\pipx}{x\d{\in} \Azeven(p\d{+}1,q\d{+}1)#1}}
%%%%\newcommand{\pimxStack}[1][]{\mystack{\pimx}{x\d{\in} \Azeven(q\d{+}1,p\d{+}1)#1}}
%%%%\newcommand{\pipyStack}[1][]{\mystack{\pipy}{y\d{\in} \Azeven(p,q\d{+}1)#1}}
%%%%\newcommand{\pimyStack}[1][]{\mystack{\pimy}{y\d{\in} \Azeven(q\d{+}1,p)#1}}
%%\newcommand{\tzo}{2\Z+1}
%%\newcommand{\tz}{2\Z}
%%\newcommand{\tno}{2\N+1}
\newcommand{\Azeven}{A_0^{\rm\footnotesize even}}

\newcommand\Myref[1]{%
  \begingroup
  \usebeamerfont*{item projected}%
  \usebeamercolor[bg]{item projected}%
  \begin{pgfpicture}{-1ex}{0ex}{1ex}{2ex}
    \pgfpathcircle{\pgfpoint{0pt}{.75ex}}{1.2ex}
    \pgfusepath{fill}
    \pgftext[base]{\color{fg}\ref{#1}}
  \end{pgfpicture}%
  \endgroup
}

\title{不定値直交群$O(p,q)$の対称性破れ作用素}

% A subtitle is optional and this may be deleted

\author[{レオンチエフ}]{{レオンチエフ アレックス}}

\institute[東大数理] % (optional, but mostly needed)
{
  東京大学\\
  大学院数理科学研究科
  }
% - Use the \inst command only if there are several affiliations.
% - Keep it simple, no one is interested in your street address.

  \date[第56合同シンポジウム]{第56回実函数論・函数解析学 合同シンポジウム講演集\\2017年8月21日--8月23日\\
於お茶の水大学}
% - Either use conference name or its abbreviation.
% - Not really informative to the audience, more for people (including
%   yourself) who are reading the slides online

\subject{表現論}

\begin{document}
\section{}

\begin{frame}\titlepage
	\note{
	初めまして。東京大学博士1年生レオンチエフ・アレックスと申します。
	今回の第56回実函数論・函数解析学 合同シンポジウムで発表させて頂き、誠にありがとうございます。
	今日は{不定値直交群$O(p,q)$の対称性破れ作用素}をテーマにお話ししたいと思います。
	}\end{frame}
\begin{frame}
	\begin{center}
		\huge この研究は\textbf{小林俊行}先生(東京大学)との共同研究である
	\end{center}
	\note{この研究は東京大学の小林俊行先生との\kana{共同}{キョウドウ}研究です。}
\end{frame}
\section{目標}
\begin{frame}{主テーマ -- 対称性破れ作用素}
この講演での主テーマは{、}
群$G$の表現$I(\lambda)$から部分群$G'$の表現$J(\nu)$への$G'$絡作用素(\textbf{対称性破れ作用素}、symmetry breaking operator)である。
%%対称性破れ作用素全体のなす空間を$$\sbo\mbox{と表記す{る}。}$$
\centerline{
	\xymatrixcolsep{1.8cm}
	\xymatrixrowsep{1.8cm}
	\xymatrix{
		I(\lambda)\ar[rr]_{\mbox{対称性破れ作用素}}&&J(\nu)\\
	G\ar@/^1pc/[u]&\supset&G'\ar@/_1pc/[u]}
}
\vspace{2em}
\begin{equation*}
	\lambda,\nu\mbox{: 表現のパラメータ}
\end{equation*}
\note{
この講演での主テーマは{、}
群$G$の球退化主表現$I(\lambda)$から部分群$G'$の球退化主表現$J(\nu)$への連続な$G'$絡作用素(対称性破れ作用素、{symmetry breaking operator})です。
\slowly{symmetry breaking operator}の
\kana{頭文字}{カシラモジ}を{とって}SBOと\kana{略記}{リャクキ}することにします。
}
\end{frame}
\section{設定}
\begin{frame}{扱う群の設定 -- 不定値直交群}
	  $n=p+q$と{し}、$x\in\R^n$上の符号$(p,q)$をもつ標準二次形式を
	  \begin{equation*}
		  \begin{array}[]{c}
			  Q_{p,q} (x) \assign \,^t \! x I_{p, q} x=x_1^2+\cdots+x_p^2-x_{p+1}^2-\cdots-x_{p+q}^2\mbox{と定める。}\\
		  \end{array}
	  \end{equation*}

	  ここで$
		I_{p, q} \assign \tmop{diag} (\underbrace{1, \ldots, 1}_p, \underbrace{-
		1, \ldots, - 1}_q)\in GL(p+q,\R)
	  $。
	  この二次形式$Q_{p,q}(x)$を保つ不定値直交群を
\begin{equation*}
	 O (p
, q)=\left\{ g\in GL\left( p+q,\R \right):\;^t\!gI_{p,q}g=I_{p,q} \right\}
\end{equation*}
と定義する。

以下では$G=O(p+1,q+1)$とおく。
\note{
	$n=p+q$と{し}、$\R^n$上の符号$(p,q)$をもつ標準二次形式$Q_{p,q}$を定めます。
	標準座標でこの二次形式は\kana{対角行列}{タイカギョウレツ}$I_{p,q}$で表されます。
	二次形式$Q_{p+1,q+1}(x)$を保つ不定値直交群$O(p+1,q+1)$を$G$と\kana{表記}{ヒョウキ}します。
}
\end{frame}
\begin{frame}{扱う表現の設定 -- 球退化主系列表現}
極大放物型部分群
$P=MAN_{+}$を以下のように定める、
  \begin{equation*}
	  \kern-0.2cm
	  \begin{array}{@{}l@{}ll}
   M &\assign {\left\{ \left( \begin{array}{ccc}
    \epsilon & 0 & 0\\
    0 & A & 0\\
    0 & 0 & \epsilon
  \end{array} \right) \middle| 
    A \in O (p, q),\;
    \epsilon = \pm 1
  \right\}}&\simeq O(p,q)\times \Z/{2\Z},  \\
  A&=\left\{ a(t)=\exp\left( t\left( E_{1,p+q+2}+E_{p+q+2,1}:\footnote[frame]{I think we should replace the colon $:$ with bar $\mid$ here,
  in order to maintain consistency in set notation.}t\in\R \right) \right) \right\}&\simeq \R,\\
  N_+&&:\mbox{可換な冪零部分群}
%%  A &\assign \left\{a(t)\assign\left( \begin{array}{ccc}
%%    \cosh (t) & 0 & \sinh (t)\\
%%    0 & I_{p + q} & 0\\
%%    \sinh (t) & 0 & \cosh (t)
%%  \end{array} \right)\middle| t \in \mathbbm{R}\right\} &\simeq \mathbbm{R},\\
%%   N_+ &\assign\kern-0.1cm\left\{  I_{n + 2}\kern-0.05cm+\kern-0.15cm\left( \begin{array}{ccc}
%%    - \frac{1}{2} Q_{p,q} (b) & - \,^t \! (I_{p, q} b) & \frac{1}{2} Q_{p,q} (b)\\
%%    b & 0 & - b\\
%%    - \frac{1}{2} Q_{p,q} (b) & - \,^t \! (I_{p, q} b) & \frac{1}{2} Q_{p,q} (b)
%%  \end{array} \right) \kern-0.1cm\middle| b \in \mathbbm{R}^{p + q} \right\} &\simeq
%%  \mathbbm{R}^{p + q}.\\
  \end{array}
  \end{equation*}
放物型部分群$P$からの誘導表現として、
複素数パラメータ$\lambda\in\C$に対して、
$G$の球退化主系列表現$I(\lambda)$Яを定義する。\footnote[frame]{I think this slide somehow violates the "1 slide 1 topic" policy:
	it introduces both $P$'s Langlands decomposition, and $I(\lambda)$.
}
\begin{align*}
I(\lambda)&:=\Ind_P^G(\C_\lambda)\\
&\simeq \left\{ f\in C^{\infty}(G)\mid f(gma(t)n)=e^{-\lambda t}f(g),\;\forall(g,ma(t)n)\in G\times P \right\}
\end{align*}
\begin{fact*}
$\lambda\notin\Z$(整数)ならば$I(\lambda)$は$G$の既約表現である。
\end{fact*}
\note{
$G$の極大放物型部分群$P=MAN_{+}$を以下のように定めます。

	ここでの$M,A$と$N_+$は$P$のラングランズ分解です。
	$A$はPを定めた
	1次元可換群であり、$N_+$は$p+q$次元\kana{冪零}{ベキレイ}部分群です。今の場合で$N_+$も可換です。
放物型部分群$P$からの誘導表現として、以下のように
$G$の球退化主系列表現を定めます。この表現を$I(\lambda)$と表記します。
}
\end{frame}
\begin{frame}{部分群$G'$とその表現$J(\nu)$の定義}
	群$G$の部分群$G'$を以下のように定める
	\begin{equation*}
		\begin{array}[]{l}
			G=O(p+1,q+1)\\
			\cup\\
			\mbox{\ul{$G':=O(p,q+1)$}}
		\end{array}
	\end{equation*}
	\newcommand{\snip}{$G'$の球退化主系列表現 $J(\nu)$}
	$G$の球退化主系列表現$I(\lambda)$と同様に、
	$\color{red}\underline{{\color{black}\mbox{\snip}}}$
	($\nu$は
	複素数)を定める。
	\note{TODO}
\end{frame}
\begin{frame}{部分群とその表現の具体的な設定}
	座標で計算する場合は$G=O(p+1,q+1)$の部分群$G'$を
\begin{equation*}
	G':=\mysetn{g \in G}{g \cdot e_{p + 1} = e_{p + 1}}\subset G\quad\mbox{と実現する。}
\end{equation*}
%%\begin{equation*}
%%	\underbrace{A}_{\mbox{$P$を定めたsplitな可換群}}\subset G'\Longrightarrow\mbox{$P=MAN$は$G'$と適合する。}
%%\end{equation*}
%%$G$の放物型部分群$A$が$G'$に含まれているため、
%%$P=MAN$は$G'$と適合する。
%%すなわち、
%%\vspace{-1em}
$G$の
極大放物型部分群$P=MAN_+$を以下のようにとっておく
\begin{enumerate}
	\item $P':=P\cap G'$は$G'$の極大放物型部分群であり、
	{\item {$P'$のLanglands分解は$P'=(G'\cap M)A(G'\cap N_+)$で与えられる。}}
\end{enumerate}
部分群$P'$から誘導表現として
、複素数パラメータ$\nu\in\mathbb{C}$に対して\\
$G'$の球退化主系列表現$J(\nu):=\Ind_{P'}^{G'}(\mathbb{C}_{\nu})$を定める。
\note{
複素数パラメータ
$\lambda\in\C$に対して
$I(\lambda)$という$G$の球退化主系列表現を定義します。
次に第$p$座標の固定部分群を
\begin{equation*}
	G'
\end{equation*}
とすると、$G'$は$G$の簡約部分群となる。
$G$の放物型部分群$P$を定めたsplitな可換群$A$が$G'$に含まれているため、
$P=MAN$は$G'$と適合します。すなわち、$P':=P\cap G'$は$G'$の極大放物型部分群であり、そのラングランズ分解は
$P'=(G'\cap M)A(G'\cap N_+)$で与えられます。
$I(\lambda)$と同様
に、複素数パラメータ$\nu\in\mathbb{C}$に対して$G'$の球退化主系列表現$J(\nu):=\Ind_{P'}^{G'}(\mathbb{C}_{\nu})$を定めます。
}
\end{frame}
\begin{frame}{ここまでの記号のまとめ}
	\centerline{
		\xymatrixcolsep{0.01cm}
		\xymatrixrowsep{0.5cm}
		\xymatrix{
			P&\subset&G=O(p+1,q+1)&\curvearrowright&I(\lambda)=\Ind_P^G(\C_\lambda)\\
			&\mbox{極大放物型部分群}&&&\mbox{球退化主系列表現}\ar@[red]@{.>}[dd]\\
			\cup&&\cup&&\\
			P'&\subset&G'=O(p,q+1)&\curvearrowright&J(\nu)=\Ind_{P'}^{G'}(\C_\nu)\\
			&\mbox{極大放物型部分群}&&&\mbox{球退化主系列表現}\\
%%		G\ar@/_1pc/[d]&\supset&G'\ar@/^1pc/[d]\\
		}}
	\note{NB: talk at least 1min:\\
	}
\end{frame}
\begin{frame}{小林--大島{利雄}による一般理論(アプリオリ評価)}
	$(G,G')=\left( O(p+1,q+1),O(p,q+1) \right)$の複素化は$$(G_\C,G'_\C)=(O(n+2,\mathbb{C}),O(n+1,\C))\mbox{である。}$$
ここで$n=p+q$。$\left( G_\C,G'_\C \right)$は
強Gelfand対となるので、小林--大島{利雄}
\cite{kobayashi2013finite}の一般理論によって以下のようなアプリオリ評価が得られる。
\begin{fact}
	対称性破れ作用素のなす空間$\sbo$の次元は表現のパラーメタ$(\lambda,\nu)$よらずに一様に押さえられている。
\end{fact}
\note{
%%対称性破れ作用素全体のなす空間を$\sbo$と表記す{る}。

$(G,G')$の複素化は$(O(n+2,\mathbb{C}),O(n+1,\C))$であり、これは強Gelfand対となるので、小林--大島
の一般理論によって以下のようなアプリオリ評価が得られます。
}
\end{frame}
\begin{frame}{目標 -- 対称性破れ作用素の構成と分類}
\begin{block}{\underline{目標}:}
対称性破れ作用素のなす線型空間
$\sbo$について、
\begin{enumerate}[1.]
	\item 次元を決定する。
	\item その基底を具体的に構成する。
	\item 関数等式を決定する。
\end{enumerate}

\end{block}
この方向での最初の結果$\cdots$ Kobayashi--Speh, Memoirs of AMS 2015.
\note{
対称性破れ作用素のなす線型空間
$\sbo$の基底を具体的に決定しましょう。
}
\end{frame}
\begin{frame}{対称性破れ作用素の超函数核}
	Kobayashi--Speh (2015) の戦略(一般理論)
	\vspace{-1em}
	\begin{enumerate}[1.]
		\item 対称性破れ作用素をその超函数核で特徴{づける}
		\item その台によるフィルトレーション($P'\backslash G/P$の幾何)による帰納法
	\end{enumerate}
	対称性破れ作用素の超関数核に注目し、 以下の可換図式を得る
	\begin{fact}[対称性破れ作用素の積分核,{\cite[Thm. 3.16]{kobayashi2015symmetry}}]
%%		\cite{kobayashi2015symmetry}の一般論を今の特別な設定に適用すると、以下の図式を得る:
%%Applying the general statement of
%%\cite{kobayashi2015symmetry} to our concrete setting we get:
	\centerline{
		\scalebox{0.9}{
		\xymatrixcolsep{5pc}
		\xymatrix{\Hom_{G'}(I(\lambda),J(\nu))\ar[r]^{\simeq} \ar@/^2pc/[rr]^{\mathcal{S}}
		&\left( \mathcal{D}'(G/P,\mathcal{L}_{n-\lambda}) \otimes\mathbb{C}_\nu \right)^{P'}
		\ar[r]_-{F\mapsto \supp(F)}\ar[d]^{\rotatebox{90}{$\simeq$}}_{\operatorname{rest}}
	&2^{P'\backslash G/P}\\
	&*+[r]{\kern0.0cm\sol\subset\mathcal{D}'(\R^{p,q})}\ar[lu]^{\operatorname{Op}}_{\simeq}&
	}
	}
}
	\end{fact}
		\vspace{-2em}
	\begin{equation*}
		\begin{array}[]{ll}
			\operatorname{Op}:&\mbox{超函数核}\rightsquigarrow\mbox{対称性破れ作用素}\\
			\mathcal{S}:&\mbox{超函数核の台(support)}
		\end{array}
	\end{equation*}
	\note{\tiny
		今回の結果の手法は、小林先生および、Kobayashi-Spehによって開発された手法を主に使います。
		一方、\kana{高階}{コウカイ}の群では、幾何的な状況や解析的な部分がもう少し複雑になります。
		最初に、対称性破れ作用素を、シュワルツの\kana{核}{かく}定理によって、
		超関数核で表せることを紹介したいと思います。
		Kobayashi-Spehの本で証明された一般理論を今の特別な設定に\kana{適
		用}{てきよう}すると、
		以下の図式を得ます。

%%この\kana{図}{ズ}はKobayashi-Spehの\kana{一般論}{イッパンロン}を\kana{今}{イマ}\kana{考}{カンガ}{え}ようとしている\kana{場合}{バアイ}に\kana{当}{ア}てはめたものです。
下の$S\mbox{ol}\left(
\mathbb{R}^{p,q}
\right)$という空間はある$(\lambda,\nu)$に依存する偏微分方程式を満たす超関数空間です。ここで、左から真ん中への同型写像はSchwartzの\kana{核}{カク}
定理を\kana{用}{モチ}います、真ん中から下へのrestという写像は、open Bruhat cellに\kana{積分核}{セキブンカク}
を制限することによって\kana{導}{ミチビ}かれます。この2つの写像は線形空間の同型写像になります。
なので、抽象的な対称性破れ作用素空間の代わりに具体的なEuclid空間上の\kana{連立}{レンリツ}偏微分方程式の解空間を研究に\kana{置}{オ}き\kana{換}{カ}えることができます。
ポイントはもう1つあります。真ん中の空間は$G/P$上の$P'$不変超関数空間なので、それぞれの元のサーポトが$P'$不変$G/P$閉部分集合になります。
つまり、真ん中のベクトル空間から$G$の\kana{両側剰余}{リョウガワジョウヨ}空間へのが\kana{定}{サダ}{まった}ということです。Target spaceであるこの\kana{両側剰余}
{リョウガワジョウヨ}空間は有限なので、$P'$不変$G/P$閉部分集合が大切なinvariantになります。
	}
\end{frame}
\begin{frame}{Fact 2の記号$\sol$の定義}
\begin{definition}\label{def2}
	群$O(p-1,q)$を$\R^n$ $(n=p+q)$に第$p$座標を固定する形で作用させる。
	以下のような条件を満たす超関数$F\in\mathcal{D}'(\R^n)${のなす}空間を$\sol$と表記する。
	\begin{enumerate}
		\item $F (x) = F (- x)$;
		\item $F$は$O(p-1,q)$不変である;
		\item $F$は$(\lambda-\nu-n)$次の斉次性を持つ; 
		\item \label{item:4} $F$は$\R^{p,q}$上に$N_+'$不変である。
	\end{enumerate}
\end{definition}
\begin{remark}
	$N_+'$は$\R^{p,q}$の共形コンパクト化にしか作用しないので、
	\scalebox{1.6}{\Myref{item:4}}の定義には注意を要\doubt{する}。
\end{remark}
\note{
	TODO
}
\end{frame}
\section{対称性破れ作用素の分類}
\begin{frame}{実旗多様体の軌道分解$P'\backslash G/P$}
	\newcommand{\setinvThmName}{$G/P$の$P'$軌道の分類とその閉包関係}
	\begin{equation*}
		\begin{array}[]{l}
X \assign G / P \simeq (\mathbbm{S}^p \times
\mathbbm{S}^q) / \{ \pm I \} \mbox{:実旗多様体}\\
Y \assign \{ [\xi, \eta] \in G / P : \xi_{p +
1} = 0 \} \simeq G'/P'\simeq(\mathbbm{S}^{p - 1} \times \mathbbm{S}^q) / \{ \pm I \}, \\
C \assign \{ [\xi, \eta] \in G / P : \xi_1 =
\eta_{q + 1} \},\\
\left\{ o \right\}:=\left\{ \left[ 1,0_{p+q}1 \right] \right\}\mbox{とおく。}
		\end{array}
	\end{equation*}

	\begin{theorem}[\setinvThmName]
		  \begin{figure}[H]
    \centering
    \begin{subfigure}[t]{0.3\textwidth}
	    \xymatrixrowsep{0.5pc}
	    \xymatrix{&X\ar@{-}[ld]_1\ar@{-}[rd]^1&\\Y\ar@{-}[rd]_1&&C\ar@{-}[ld]^1\\&Y\cap C\ar@{-}[dd]^{p+q-2}&\\&&\\&\{[0]\}&}
	\caption{$p>1$のとき}
    \end{subfigure}
    ~ %add desired spacing between images, e. g. ~, \quad, \qquad, \hfill etc. 
      %(or a blank line to force the subfigure onto a new line)
    \begin{subfigure}[t]{0.3\textwidth}
	    \xymatrixrowsep{0.5pc}
	    {\xymatrix{&X\ar@{-}[ld]_1\ar@{-}[rd]^1&\\Y\ar@{-}[rddd]_{p+q-2}&&C\ar@{-}[lddd]^{p+q-2}\\&&\\&&\\&\{[0]\}&}}
	    \vspace{0.7em}
	\caption{$p=1$のとき}
    \end{subfigure}
\vspace{-0.8em}
\end{figure}

	\end{theorem}
	\note{
小林--Spehに述べられたストラテジー
を$G$が不定値直交群の\kana{場合}{バアイ}に\kana{用}{モチ}いる
\kana{第一歩}{ダイイッポ}として、両側\kana{剰余類}{ジョウヨルイ}とその\slowly{closure relations}、\kana{閉包}{ヘイホウ}関係の\kana{記述}{キジュツ}を述べます。
今の設定では、両側\kana{剰余類}{ジョウヨルイ}の\kana{個数}{コスウ}は、$p$は1以上ならば5つあり、$p$が1に等しいならば、4つにな
ることが証明されます。図式では、$X$が\slowly{\kana{全体}{ゼンタイ}}、すなわち実旗多様体を表し、edgeのとなりの数字がgenericなcodimensionを表します。\\
	}
\end{frame}
\begin{frame}{微分対称性破れ作用素}
	\begin{tabular}[]{cl}
		\underline{ポイント}&対称性破れ作用素の内\\
		& あるものは積分で、\\
		& あるものは特異積分で、\\
		& また、あるものは微分で\\
		&表すことができる。
	\end{tabular}
	\begin{definition}
		$X\supset Y$を$C^\infty$多様体とする。線形写像$T:C^\infty(X)\to C^\infty(Y)$は\begin{equation*}
			T=\operatorname{Rest}_{Y}\circ X\mbox{上の微分作用素}
		\end{equation*}と表せるとき、$T$を(広い意味で)\underline{微分作用素}という。
	\end{definition}
	\note{TODO}
\end{frame}
\begin{frame}{微分対称性破れ作用素$T:I(\lambda)\to J(\nu)$の分類}
	\begin{fact}[微分対称性破れ作用素の分類 ({\cite[Thm. 4.3]{kobayashi2015branching}})]
%%$(\lambda, \nu) \in / / \assign \{
%%(\lambda, \nu) \in \mathbbm{C}^2 \mid \nu - \lambda \in 2\mathbbm{N} \}$に対し、
		\begin{enumerate}[(1)]
			\item 零でない微分対称性破れ作用素が存在する$\iff\nu-\lambda\in2\N$。
			\item $\ell:=\frac{1}{2}(\nu-\lambda)\in\N$のとき、
\begin{equation*}
	\kern-0.2cm\begin{array}[]{@{}r@{}l}
	{R}_{\lambda, \nu}^{\{ o \}} &
	=\kern-0.2cm
		\sum_{j=0}^\ell\frac{(-1)^j2^{2\ell-2j}}{j!(2\ell-2j)!}\kern-0.0cm\prod_{i=1}^{\ell-j}\kern-0.2cm\left( \frac{n
		+1}{2}+\frac{\nu+\lambda}{2}
		+i \right)\kern-0.1cm\left(- \Delta_{\mathbbm{R}^{p-1,q}} \right)^j\kern-0.1cm\left( \frac{\partial}{\partial x_p} \right)^{\kern-0.1cm 2\ell-2j}
\\
	&
	=
\tmop{Rest}_{x_p = 0} \circ \tilde{C}_{2\ell}^{\lambda - \frac{n -
1}{2}} \left( - \Delta_{\mathbbm{R}^{p - 1, q}}, \frac{\partial}{\partial x_p}
\right)
	\end{array}
\end{equation*}
	微分対称性破れ作用素になる。
	ここで$\tilde{C}^\alpha_{m}(s,t)$はGegenbauer多項式
	を2変数化した多項式である。[Kobayashi--Pevzner (2016)]。
\item 逆に全ての微分対称性破れ作用素は$R^{ \left\{ o \right\}}_{\lambda,\nu}$の定数倍である。
		\end{enumerate}
	\end{fact}
	\note{\tiny
		\doubt{第二歩}は\kana{実}{ジツ}旗多様体$G/P$における$P'$-invariantなclosed subsetそれぞれに対し、
対称性破れ作用素のfamilyを構成し、更に、そのmeromorphic continuationを証明するというステップです。\slowly{構成をした\kana{後}{アト}}は分類になります。
これが3つ目のステップです。Closed setが小さいものから\kana{帰納的}{キノウテキ}に\kana{順}{ジュン}に\kana{分類}{ブンルイ}します。
そのstarting pointは、closed setが1点の\kana{場合}{バアイ}{で}、
このときは対称性破れ作用素は微分作用素で表せ\doubt{ます}。
このときは新しい手法であるF-methodが使えます。

次の定理は、微分対称性破れ作用素についてです。
微分対称性破れ作用素というのは、積分核となる超関数の言葉では、その台が原点に集中し
ているということと同値
になることが小林ーPevznerによって証明されています。不定値直交群の微分対称性破れ作用素は2015
年の結果によって、分類されています。具体的な形はrenormalized 
Gegenbauer多項式を2変数化して、そこにラプラシアンと法微分を代入して得られるものです。
parameter $(\lambda,\nu)$がこの//(ナナメのニ\kana{重線}{ジュウセン}集合に入っていると、微分対称性破れ作用素空間が一次元になり、そうでないと微分対称性破れ作用素は存在しません。
	}
\end{frame}
\begin{frame}{}
	{超函数}$\myabs{x_p}^{a}$と$\myabs{x_1^2+\cdots+x_p^2-x_{p+1}^2-\cdots-x_{p+q}^2}^b$の積
	\begin{itemize}
		\item $\frac{1}{\Gamma\left( \frac{a+1}{2} \right)}\myabs{x_p}^a$
		\item $\frac{1}{\Gamma(*)}\myabs{Q_{p,q}(x)}^b=\frac{1}{\Gamma(*)}\myabs{x_1^2+\cdots+x_p^2-x_{p+1}^2-\cdots-x^2_{p+q}}^b$
	\end{itemize}
	はそれぞれ個別に$a,b\in\C$に正則 (holomorphic)に依存する$\R^{p+q}$上の超函数である。

	しかし、これらの積は well-defined ではない
	\begin{example*}
		$\delta(x_p)\times\delta(x_1,\cdots,x_{p+q})$は well-defined ではない
	\end{example*}
	$\implies$ $\myabs{x_p}^a\myabs{Q_{p,q}(x)}^b$の正規化には注意が必要である。
	\note{TODO}
\end{frame}
\begin{frame}{正則な(regular)対称性破れ作用素の{{構成}}}
%%For
%%$(\lambda, \nu) \in \mathbbm{C}^2$ および 
%%$\tmop{Re} (\nu) < 0$ and $\tmop{Re}
%%(\lambda + \nu - n) > 0$ を満たす$(\lambda,\nu)\in\mathbb{C}^2$に対し、
		\begin{equation*}
			(\lambda,\nu)\in\C^2\mbox{ が }\begin{array}[]{l}
			\Re(\nu)<0\mbox{、および}\\
			\Re(\lambda+\nu-n)>0
		\end{array}\mbox{を決定するとき、}
		\end{equation*}
\begin{equation*}
	\quad| x_p |^{\lambda + \nu - n} | Q_{p, q} |^{- \nu}\mbox{
		は$L^1_{\operatorname{loc}}\left( \R^{p+q} \right)$あり、
$\mathcal{S} \mbox{ol} (\mathbbm{R}^{p, q} ; \lambda, \nu)$の元になる。
	}
\end{equation*}
これを核超函数\doubt{とする}。対称性破れ作用素を
\begin{equation*}
	\tilde{R}_{\lambda, \nu}^{\R^{p,q}} \assign \tmop{Op} (| x_p
	|^{\lambda + \nu - n} | Q_{p, q} |^{- \nu}) \mbox{とする。}
\end{equation*}
\vspace{-1em}
	\begin{theorem}[regular 対称性破れ作用素の構成]
		$\tilde{R}_{\lambda, \nu}^{\R^{p,q}}$ は $(\lambda, \nu) \in
\mathbbm{C}^2$に有理型に依存する対称性破れ作用素の族に拡張できる。
	\end{theorem}
\vspace{-1em}
\note{TODO}
\end{frame}
\begin{frame}{正規化}
		 $(\lambda, \nu) \in
\mathbbm{C}^2$に有理型に依存する対称性破れ作用素の族
\begin{equation*}
	\tilde{R}_{\lambda, \nu}^{\R^{p,q}} \assign \tmop{Op} (| x_p
	|^{\lambda + \nu - n} | Q_{p, q} |^{- \nu})
\end{equation*}
	\begin{theorem}[regular 対称性破れ作用素の正規化]
\begin{equation*}
	{R}_{\lambda, \nu}^X \assign \frac{1}{\Gamma
\left( \frac{\lambda - \nu}{2} \right) \Gamma \left( \frac{\lambda - \nu}{2}
\right) \Gamma \left( \frac{\lambda + \nu - n + 1}{2} \right)} \tilde{R}_{\lambda,
\nu}^{\R^{p,q}}\mbox{のように正規化すると、}
\vspace{-0.6em}
\end{equation*}
$R_{\lambda,\nu}^X$は$(\lambda,\nu)\in\mathbb{C}^2$に正則に依存する
対称性破れ作用素になる。
また、
\vspace{-1em}
$$R^X_{\lambda,\nu}=0\iff(\lambda,\nu)\mbox{がある離散集合に属する。}$$
	\end{theorem}
	\note{
		その次に、regular対称性破れ作用素を作ります。微分対称性破れ作用と正反対で、
		超函数の台が全体になるものを、Kobayashi-Spehの用語で regular な対称性破れ作用素と言います。
		まずパラメータが十分良いときは、連続函数を積分核として対称性作用素を構成
		できます。
		パラメータが全空間で有理型になるような形で、この連続関数を超函数として拡
		張出来ます。
		更に、このようにnormalizationすると、パラメータにholomorphic(正則)に依存
		する作用素が得られます。
		この作用素は零点はcodimension 1ではなく、codimension 2、すなわち離散集合
		となります。
		genericには積分核の台は全体になりますが、パラメータが特異のときも、積分核の超函数としての台を正確に
		決定することができます。
	}
\end{frame}
\section{設定}
\begin{frame}{実旗多様体の軌道分解$P'\backslash G/P$ (再揚)}
\newcommand{\setinvThmName}{$G/P$の$P'$軌道の分類とその閉包関係}
	\begin{equation*}
		\begin{array}[]{l}
X \assign G / P \simeq (\mathbbm{S}^p \times
\mathbbm{S}^q) / \{ \pm I \}, \\
%%Y \assign \{ [\xi, \eta] \in G / P \mid \xi_{p +
%%1} = 0 \} \simeq \mbox{実旗多様体}G'/P'\simeq(\mathbbm{S}^{p - 1} \times \mathbbm{S}^q) / \{ \pm I \}, \\
%%C \assign \{ [\xi, \eta] \in G / P \mid \xi_1 =
%%\eta_{q + 1} \},\\
%%\left\{ o \right\}:=\left\{ \left[ 1,0_{p+q}1 \right] \right\}\mbox{とおく。}
		\end{array}
	\end{equation*}
	\begin{block}{定理 1 (\setinvThmName)}
		  \begin{figure}[H]
    \centering
    \begin{subfigure}[t]{0.3\textwidth}
	    \xymatrixrowsep{0.5pc}
	    \xymatrix{&X\ar@{-}[ld]_1\ar@{-}[rd]^1&\\Y\ar@{-}[rd]_1&&C\ar@{-}[ld]^1\\&Y\cap C\ar@{-}[dd]^{p+q-2}&\\&&\\&\{[0]\}&}
	\caption{$p>1$のとき}
    \end{subfigure}
    ~ %add desired spacing between images, e. g. ~, \quad, \qquad, \hfill etc. 
      %(or a blank line to force the subfigure onto a new line)
    \begin{subfigure}[t]{0.3\textwidth}
	    \xymatrixrowsep{0.5pc}
	    {\xymatrix{&X\ar@{-}[ld]_1\ar@{-}[rd]^1&\\Y\ar@{-}[rddd]_{p+q-2}&&C\ar@{-}[lddd]^{p+q-2}\\&&\\&&\\&\{[0]\}&}}
	    \vspace{0.7em}
	\caption{$p=1$のとき}
    \end{subfigure}
\vspace{-0.8em}
\end{figure}

	\end{block}
	\note{TODO}
\end{frame}
\section{対称性破れ作用素の分類}
\begin{frame}
	微分--、積分作用素以外の対称性破れ作用素
	\begin{theorem}[特異な対称性破れ作用素の構成]
		実旗多様体$X=G/P$の$P'$不変な閉集合
	$S=X,Y,C$に対して、
	$I(\lambda)$から$J(\nu)$への対称性破れ作用素
	であり、
	以下の性質をみたすものが存在する。

	今までに述べた微分対称性破れ作用素でも正則な対称性破れ作用素で
	もない。
	その積分核は台が$S$に含まれる超
	函数である。
%%	$\Supp(R_{\lambda,\nu}^S)\subset S$
%%	をみたす
%%	(更に、一般の位置に{ある}$(\lambda,\nu)$に対しては{等号}が成り立つ)。
	\begin{equation*}
		\begin{array}[]{@{}l@{}l@{}l}
			\mbox{対称性破れ作用素}&&\mbox{定義域}\\
			{\tilde{R}}_{\lambda, \nu}^X&\mbox{ on }&\mid \mid \mid \assign \left\{ (\lambda, \nu) \in \mathbbm{C}^2 \mid
\nu \in - 2\mathbbm{N} \; \tmop{or} \; \nu \equiv q + 1 \; \tmop{mod} \; 2
\right\},\\
{R}_{\lambda, \nu}^Y&\mbox{ on }&\backslash\backslash \assign \{ (\lambda, \nu) \in \mathbbm{C}^2 \mid
\lambda + \nu - n + 1 \in - 2\mathbbm{N} \},\\
{R}_{\lambda, \nu}^C&\mbox{ on }&\mid \mid \assign \{ (\lambda, \nu) \in \mathbbm{C}^2 \mid \nu \in 1 +
2\mathbbm{N} \} . 
		\end{array}
	\end{equation*}
	\end{theorem}
	\note{
		regular作用素が消えてしまう離散集合に対しては、renormalizeという手続きに
		よって
		新しい対称性破れ作用素ファミリ三つを構成することができます。
		それぞれのファミリが自分
		全てのパラメータに対してではなくて、$\mathbb{C}^2$のaffne subspace の可算和で\doubt{定義することができ、それは正確に決定することができます}。
	}
\end{frame}
\begin{frame}{今までに構成した対称性破れ作用素$I(\lambda)\to J(\nu)$のまとめ}
	\vspace{-2em}
	\begin{equation*}
		\begin{array}[]{l}
			R_{\lambda,\nu}^{ \left\{ o \right\}}\mbox{ on $//$}:\mbox{微分対称性破れ作用素}\\
			\left.
			\begin{array}[]{l}
				\tilde{R}_{\lambda,\nu}^X\mbox{ on $\mid\mid\mid$}\\
				R^Y_{\lambda,\nu}\mbox{ on $\backslash\backslash$}\\
				R^C_{\lambda,\nu}\mbox{on $\mid\mid$}
			\end{array}
			\right\}
			\mbox{特異対称性破れ作用素}\\
			R^X_{\lambda,\nu}\mbox{ on $\C^2$: 正則 (regular)な対称性破れ作用素}
		\end{array}
	\end{equation*}
	\note{TODO}
\end{frame}
\begin{frame}
	\begin{theorem}[対称性破れ作用素の分類]
\begin{enumerate}[(1)]
	\item $p=1$のとき、
		\vspace{-2em}
		\begin{equation*}
			\kern-0.8cm\sbo = \left\{
				\begin{array}{@{}l@{}l@{}}
     \mathbbm{C}R^X_{\lambda, \nu}, & (\lambda, \nu) \in \mathbbm{C}^2 - (/ /
     \cap \mid \mid \mid) - (\mid \mid \cap \backslash\backslash),\\
     \mathbbm{C} \tilde{R}^X_{\lambda, \nu} \oplus \mathbbm{C}R^{\{ o
     \}}_{\lambda, \nu}, & (\lambda, \nu) \in (/ / \cap \mid \mid \mid) -
     (\mid \mid \cap \backslash\backslash),\\
     \mathbbm{C}R^Y_{\lambda, \nu} \oplus \mathbbm{C}R^C_{\lambda, \nu}, &
     (\lambda, \nu) \in (\mid \mid \cap \backslash\backslash) - / /,\\
     \mathbbm{C}R^{\{ o \}}_{\lambda, \nu}, & (\lambda, \nu) \in \mid \mid
     \cap \backslash\backslash \cap / /.
   \end{array} \right.
		\end{equation*}
	\item $p>1$のとき、
		\vspace{-2em}
		\begin{equation*}
\kern-1.9cm\sbo = \left\{
   \begin{array}{ll}
     \mathbbm{C} \tilde{R}^X_{\lambda, \nu} \oplus \mathbbm{C}R^{\{ o
     \}}_{\lambda, \nu \lambda, \nu}, & (\lambda, \nu) \in / / \cap \mid \mid
     \mid,\\
     \mathbbm{C}R^X_{\lambda, \nu}, & \mbox{それ以外の場合。}
   \end{array} \right.
   \vspace{-1em}
		\end{equation*}
\end{enumerate}
	\end{theorem}
	\note{TODO}
\end{frame}
\begin{frame}{対称性破れ作用素の次元(再揚)}
	一般論(小林--大島{利雄})$(G_\C,G'_\C)=\left( O(n+2,\C),O(n+1,\C) \right)$
	の任意の実形$(G,G')$に対して、\\
$\implies$ある定数$C>0$が存在{して}\begin{equation*}
	\dim_{\C}\sbo\le C
\end{equation*}
	\begin{corollary}[対称性破れ作用素の{{存在}}定理と次元の上限]
		どんな$(\lambda,\nu)\in\mathbb{C}^2$に対しても
		$$\dim_{\mathbb{C}}\sbo\in\left\{ 1,2 \right\}\mbox{が成り立つ。}$$
	\end{corollary}
	\note{
		このように構成した対称性破れ作用素は全ての対称性破れ作用素を生成することが証明できます。
		ここまでの結果をまとめると、次元公式、
		より\kana{精密}{セイミツ}にはそれぞれ具体的な基底がスライドに述べたようにわかります。

		次に、問題(C3)から(C5)までについての結果をご紹介します。
	}
\end{frame}

\section{$(\mathcal{C}3) - (\mathcal{C}5)$にの答え}
\begin{frame}
	\begin{theorem}[$K$不変ベクトルのスペクトラム]
		\label{thm:spherical}
%%	$n=p+q\;(p,q\ge1)$という記法は前述の通りとする。この{とき、}
		正則な(regular)対称性破れ作用素${R}_{\lambda,\nu}^X:I(\lambda)\to J(\nu)$は
		$K$不変ベクトル$1_\lambda$%\footnote{should I perhaps change to $\mathbbm{1}_\lambda$ and $\mathbbm{1}_\nu$, as in the abstract?}
		を$1_\nu$の定数倍に移す。
	\begin{equation*}
		{R}_{\lambda, \nu}^X 1_{\lambda} =
		\frac{2^{1 - \lambda} \pi^{\frac{p+q}{2}}}{\Gamma \left( \frac{\lambda}{2} \right)
\Gamma \left( \frac{\lambda + 1 - q}{2} \right) \Gamma \left( \frac{q - \nu +
1}{2} \right)} 1_{\nu}.
	\end{equation*}
	\end{theorem}
	\begin{remark}
%%This Theorem was known in
%%Bernstein--Reznikov 2004 for $p = q = 1$ (i.e. $G' \simeq \tmop{SL} (2,
%%\mathbbm{R})$) and in \cite{kobayashi2015symmetry} for $q = 0$.
	$SL(2,\R)$の表現のテンソル積の場合、すなわち、
	定理\ref{thm:spherical}\;の$p=q=1$の特別な場合はBernstein--Reznikov\cite[Lem. A.5]{bernstein2004estimates} によって得られ、
	また$q=0$の場合は\cite[Prop.\ 7.4]{kobayashi2015symmetry}で得られている。
	\end{remark}
	\note{
		$G$ の球退化主系列表現 $I(\lambda)$ は $K$ 不変ベクトルを含む。部分群 $G'$ の表現 $J(\nu)$ も同様である。いずれの場合も $K$ 不変ベクトルのなす空間は1次元である。
		次の結果によると、I(lambda)のK-finiteベクトルの像がJ(nu)の$K'$-
		finiteベクトルと定数の倍になります。これらのベクトルを具体的に正規化した時の
		比例定数も計算できて、極点と霊点を分
		けるように積公式で表せます。
		これは、Bernstein-Reznikovが$SL(2,\R)$に対しての計算した結果を$p=q=1$の場合と
		して含みます。
	}
\end{frame}
\begin{frame}
%%	$(\lambda,\nu)\in\mathbb{C}^2\setminus//$に対して、
%%\vspace{-1em}
\begin{equation*}
	 \frac{| x_p |^{\lambda +
\nu - n}}{\Gamma \left( \frac{\lambda + \nu - n + 1}{2} \right)} \cdot \frac{|
	Q_{p,q} |^{- \nu}}{\Gamma \left( \frac{1 - \nu}{2} \right)} 
\end{equation*}
を積分核{とする}対称性破れ作用素
${R}_{\lambda, \nu}^X$は
%%$(\lambda,\nu)\in\mathbb{C}^2$全体に正則(holomorphic)に拡張できる。
%%以下では作用素$\Op\left( \tilde{K}^{\R^{p,q}}_{\lambda,\nu} \right)$の留数を求める。
%%\vspace{-0.5em}
	$\ell:=\frac{1}{2}\left( \nu-\lambda \right)\in\N$となるとき、
	そのときに限り$\tilde{R}_{\lambda,\nu}^X$は微分対称性破れ作用素となり、
\begin{theorem}[留数公式]
	\vspace{-1em}
\begin{equation*}
	{R}_{\lambda, \nu}^X = \frac{(- 1)^\ell \ell!
\pi^{(n - 2) / 2}}{2^{\nu + 2 \ell - 1}} \cdot \frac{\sin \left( \frac{1 + q -
\nu}{2} \pi \right)}{\Gamma \left( \frac{\nu}{2} \right)}\cdot \underbrace{{R}_{\lambda,
\nu}^{\{ o \}}}_{\mbox{微分対称性破れ作用素}}, (\lambda, \nu) \in //.
\end{equation*}
\end{theorem}
\vspace{-0.5em}
\begin{remark}
		対称性破れ作用素の
		留数公式の原型は$q=0$の場合で、このときは小林 \cite{KOBAYASHI2014272}、小林--Speh \cite[Thm. 12.2]{kobayashi2015symmetry}で証明されている。
\end{remark}
\note{
	次の定理はregular対称性破れ作用素の
	\slowly{\kana{留数}{リュウスウ}定理}です。ここに\kana{記}{シル}した2つの超関数はそれぞれ正則パラメータをもつ超関数ですが、そのwavefront setが重なりをもつので、
	\kana{超関数}{チョウカンスウ}の
	\slowly{\kana{積}{セキ}}はwell-defined
	とは限りません。well-definedでない\kana{場所}{トコロ}にまた新しいpoleが\kana{生}{ショウ}じます。このpoleのresidueは微分対称性破れ作用素
	になります。より\kana{精密}{セイミツ}に、比例\kana{定数}{テイスウ}を
	\slowly{\kana{初等}{ショトウ}}\kana{関数}{カンスウ}の{\kana{積}{セキ}}として具体的に\kana{表}{アラワ}すことができました。
}
\end{frame}
\begin{frame}{Knapp--Stein 作用素(古典理論)}
	$G=O(p+1,q+1),n=p+q$とする。
	\begin{definition}[$\tilde{\mathbb{T}}_\lambda^G:I(\lambda)\to I(n-\lambda)$]
		\underline{Knapp--Stein 作用素}は以下の定義
		を拡張に得られる
		\doubt{さ} $G$絡作用素である\begin{equation*}
			\begin{array}[]{l}
				f\mapsto q(\lambda)\left(\myabs{x_1^2+\cdots+x_p^2-x_{p+1}^2-\cdots-x_{p+q}^2}^{\lambda-n}\ast f  \right).\\
				C^\infty_c(\R^n)\to C^\infty(\R^n),
			\end{array}
		\end{equation*}
		ここで
		$q (\lambda)$は
		正規化のための関数で$\Gamma$関数の積で表される。
	\end{definition}
%%		$\tilde{R}_{n - \lambda, \nu}^X \circ\tilde{\mathbbm{T}}_{\lambda}^G$ と対称性破れ作用素 $\tilde{R}_{\lambda, \nu}^X$を比較する。両方は比例になる。 
	\note{
次の結果を紹介する前に、古典的なKnapp-Stein作用素の定義を復習したいと思い
ます。
$G = G'$ の場合は、古くから多く研究があり、この場合は$G$の球退化主系列表現の間の対称性破れ作用素は\kana{容易}{ヨウイ}に求めることができます。

Knapp-Stein作用素というのは、パーラメータに正則に依存する、
$I(\lambda)$から$I(n-\lambda)$へのintertwining operatorです。
	}
\end{frame}
\begin{frame}{函数等式の考え方}
	\centerline{
	\xymatrixcolsep{1.0cm}
	\xymatrixrowsep{0.8cm}
		\xymatrix{
		G\ar@/_1pc/[d]&\supset&G'\ar@/^1pc/[d]\\
		I(\lambda)\ar@<-3pt>@[red][d]\ar[rr]\ar[drr]&&J(\nu)\ar@<-3pt>@[red][d]\\
		I(n-\lambda)\ar@<-3pt>@[red][u]\ar[rr]\ar[urr]&&J(n-1-\nu)\ar@<-3pt>@[red][u]
		}
	}
	\vspace{-1em}
	\begin{tabular}[]{ll}
	$\rightarrow$ $\searrow$ $\nwarrow$ $\rightarrow$& はいずれも対称性破れ作用素である\\
	$\color{red}{\downarrow\uparrow}$&は古典的な絡作用素(Knapp--Stein)である
	\end{tabular}
	\begin{block}{Observation}
		$\color{red}{\uparrow}\color{black}{\searrow}$や$\searrow\color{red}{\uparrow}$などの合成写像も対称性破れ作用素となる。
	\end{block}
	\vspace{-1em}
	これらは既存の対称性破れ作用素の定数倍となる(関数等式)。
	\begin{block}{{ポイント}}
		この定数が特別な$\lambda,\nu$で零となる$\implies$これを決定したい
	\end{block}
	\note{TODO}
\end{frame}
\begin{frame}{函数公式と比例定数}
	合成
		${R}_{n - \lambda, \nu}^X \circ\tilde{\mathbbm{T}}_{\lambda}^G$ と対称性破れ作用素 ${R}_{\lambda, \nu}^X$を比較する。
		\doubt{${R}_{n - \lambda, \nu}^X \circ\tilde{\mathbbm{T}}_{\lambda}^G$は${R}_{\lambda, \nu}^X$の定数倍である。}
	\centerline{\xymatrix{I(\lambda)\ar[d]^{\tilde{\mathbb{T}}_{\lambda}^G}\ar@{-->}[rd]^{R_{\lambda,\nu}^X}&\\I(n-\lambda)\ar[r]^{R^X_{n-\lambda,\nu}}&J(\nu)}}
		\begin{equation*}
			\boxed{R^X_{n-\lambda,\nu}\circ\tilde{\mathbb{T}}_\lambda^G=c(\lambda,\nu)R_\lambda^X}
		\end{equation*}定数$c(\lambda,\nu)$を決定したい。

		特に、いつ$c(\lambda,\nu)=0$となるかを知りたい。
	\note{
		以下のdiagramのように、$R_{n-\lambda,\nu}^X$ regular対称性破れ作用素と$\mathbb{T}_
	\lambda^G$ Knapp-Stein作用素の組み合わせと$R_{\lambda,\nu}^X$対称性破れ作用素が
両方は$I(lambda)$から$J(nu)$へ$G'$作用を絡み合う作用素
で、対称性破れ作用素の次元がgenericにはであることを既に証明してあるので、これらは互いに
比例していることがわかります。
その比例乗数を決定することもできました。
GとG’の役割を入れ替えたものが下の公式になります。
			}
\end{frame}
\setbeamercovered{transparent}
\begin{frame}{函数等式}
	\vspace{0.5em}
%%  \begin{figure}[H]
%%%%    \begin{subfigure}[t]{0.3\textwidth}
%%    \centering
	\centerline{\xymatrix{I(\lambda)\ar[d]^{\tilde{\mathbb{T}}_{\lambda}^G}\ar@{-->}[rd]^{R_{\lambda,\nu}^X}&\\I(n-\lambda)\ar[r]^{R^X_{n-\lambda,\nu}}&J(\nu)}}
%%    \end{subfigure}
%%    ~ %add desired spacing between images, e. g. ~, \quad, \qquad, \hfill etc. 
%%      %(or a blank line to force the subfigure onto a new line)

%%    \begin{subfigure}[t]{0.3\textwidth}
%%			\xymatrix{I(\lambda)\ar@{-->}[rd]^{R_{\lambda,\nu}^X}\ar[r]^{R_{\lambda,n-1-\nu}^X}&J(n-1-\nu)\ar[d]^{\tilde{\mathbb{T}}_{n-1-\nu}^{G'}}\\&J(\nu)}
%%    \end{subfigure}
%%    \end{figure}
    \begin{theorem}[函数等式]
	$n=p+q\;(p,q\ge1)$。
	このとき、
	\begin{equation*}
		\begin{array}[]{c}
\tilde{R}_{n - \lambda, \nu}^X \circ
\tilde{\mathbbm{T}}_{\lambda}^G = \pause q_X^{X T} (\lambda, \nu)
\tilde{R}^X_{\lambda, \nu} \mbox{、ここで}\\
%%\tilde{\mathbbm{T}}_{n - 1 - \nu}^{G'} \circ
%%\tilde{R}_{\lambda, n - 1 - \nu}^X = q_X^{T X} (\lambda, \nu)
%%\tilde{R}_{\lambda, \nu}^X,\\
\uncover<2>{
  q^{X T}_X (\lambda, \nu) \assign
  \frac{2^{2\lambda-n}\pi^{-\frac{n}{2}-1}\sin\left( \frac{p-\lambda+1}{2}\pi \right)}{\Gamma\left( \frac{n-\lambda}{2} \right)}\times
   \left\{ \begin{array}{ll}
    2^{1 - \lambda} \sqrt{\pi}, & n \in 2\mathbbm{Z}+ 1,\\
    \Gamma \left( \frac{\lambda - n / 2 + 1}{2} \right), & \frac{n}{2} + p \in
    2\mathbbm{Z},\\
    \Gamma \left( \frac{\lambda - n / 2}{2} \right), & \frac{n}{2} + p \in
    2\mathbbm{Z}+ 1.
  \end{array} \right.
}
		\end{array}
	\end{equation*}
	\end{theorem}
	\begin{remark}
		函数等式の原型
		は\cite[Thm. 12.6]{kobayashi2015symmetry} ($q=0$)で証明された。
	\end{remark}
	\note<1>{
		\mytiming{funct1}
		比例定数も計算できて、極点と霊点を分
		けるように$\Gamma$関数の積で表せます。

		前のように、函数等式の原型として $q = 0$ の場合はKobayashi--Spehの論文で得られている。
	}
	\note<2>{
		\mytiming{funct2}
		比例定数も計算できて、極点と霊点を分
		けるように$\Gamma$関数の積で表せます。

		前のように、函数等式の原型として $q = 0$ の場合はKobayashi--Spehの論文で得られている。
	}
\end{frame}

\begin{frame}
	\centerline{\xymatrix{I(\lambda)\ar[d]_{\tilde{\mathbb{T}}_{\lambda}^G}\ar@{-->}[rd]^{R_{\lambda,\nu}^X}&\\I(n-\lambda)\ar[r]_{R^X_{n-\lambda,\nu}}&J(\nu)}}
	の代わりに
	\centerline{\xymatrix{I(\lambda)\ar@{-->}[rd]_{R_{\lambda,\nu}^X}\ar[r]^{R_{\lambda,n-1-\nu}^X}&J(n-1-\nu)\ar[d]^{\tilde{\mathbb{T}}_{n-1-\nu}^{G'}}\\&J(\nu)}}
	を用いた関数等式も同様に決定できる。
	\note{TODO}
\end{frame}
\begin{frame}{対称性破れ作用素の像}
	\begin{block}{問題}
		全て$(\lambda, \nu) \in \mathbbm{C}^2$に対し、上記で構成されたそれぞれの対称性破れ作用素の像を具体的に決定せよ。
	\end{block}
	\vspace{1em}
	\centerline{
		\kern2cm\xymatrix{\;\ar@{=>}^{\mbox{函数等式\footnote[frame]{in fact, as far as I remember, we \textbf{did not} use functional identities to derive the images of SBOs}}}[d]\\\;}
	}
	\vspace{1em}
	\begin{theorem}
		それぞれの対称性破れ作用素の像が決定できる。
	\end{theorem}
	\note{
		最後は、これらの\kana{緒}{しょ}結果をプラスアルファの結果から
		上に構成された対称性破れ作用素の像がパラメータによって、どのように変わるかについてもの全て決定することが
		できます。
	}
\end{frame}
\begin{frame}{Zuckerman導来函手加群}
	最後に、上記の結果の応用としてZuckerman導来函手加群の間の対称性破れ作用素の問題を論じる。
	\cite[(5.1.1)]{KO2}に{倣}って$p>1${かつ}$q\ge1$のときに\begin{equation*}
	A_0(p,q):=\left\{ \lambda\in\Z+\frac{p+q}{2}\;:\;\lambda>-1 \right\}
\end{equation*}とおくと、
\cite{KO2}で
示したように、$\lambda\in A_0(p,q)$に対して
$O(p,q)$の既約ユニタリ表現$\pi_{\pm,\lambda}^{p,q}$
が定まる。(\cite{KO2}では$\pi_{\pm,\lambda}^{p,q}$を定義するにあたって5種類の
特徴づけが与えら{れ}、それらは互いに
同値であることが示されている。その特徴づけの1つは
Zuckerman導来函手加群で記載される。)
以下では、この加群の間に対称性破れ作用素がある{か}について論じる。簡単のため、$A_0^{\mbox{\scriptsize even}}(p,q):=\left\{ \lambda\in A_0(p,q)\mid \lambda-\frac{p-q}{2}+1\in2\Z \right\}$とおく。
\note{
	最後に、上記の結果の応用としてZuckerman導来函手加群の間の対称性破れ作用素の問題を論じましょう。
	KobayashiとOrstedの2003年の論文
$O(p,q)$の既約ユニタリ表現$\pi_{\pm,\lambda}^{p,q}$
が定まりました。
この論文では$\pi_{\pm,\lambda}^{p,q}$を定義するにあたって5種類の
特徴づけが与えら{れ}、それらは互いに
同値であることが示されています。その特徴づけの1つは
Zuckerman導来函手加群で記載されます。
%%以下では、この加群の間に対称性破れ作用素がある{か}について論じる。簡単のため、$A_0^{\mbox{\scriptsize even}}(p,q):=\left\{ \lambda\in A_0(p,q)\mid \lambda-\frac{p-q}{2}+1\in2\Z \right\}$とおく。
%%以下では\begin{equation*}
%%	g(t):=\left\{ \begin{array}[]{ll}
%%		1&\left(t\in2\N+\frac{1}{2}  \right)\\[10pt]
%%		0&t\nin \left(2\N+\frac{1}{2}  \right)
%%	\end{array}\right.,\quad h(t):=\left\{ \begin{array}[]{ll}
%%		1&\left(  t<\frac{q}{2}\right)
%%		\\[10pt]
%%		0&\left( t\ge\frac{q}{2} \right)
%%\end{array}\right.
%%\end{equation*}とおく。
}
\end{frame}
\begin{frame}{}
	\begin{theorem}[Zuckerman導来加群函手{$\pi_{\pm,\lambda}^{p,q}$}間の対称性破れ作用素]
	$n=p+q\;(p,q\ge1),\;n':=n-1$とする。
	以下では
	\vspace{-1em}
\begin{equation*}
                \begin{array}[]{c}
                        x\in\left\{\begin{array}[]{ll}
                                \Azeven(p+1,q+1),&\delta=+\mbox{ のとき}\\
                                \Azeven(q+1,p+1),&\delta=-\mbox{ のとき}\\
                        \end{array}\right.\\
                        y\in\left\{\begin{array}[]{ll}
                                \Azeven(p,q+1),&\varepsilon=+\mbox{ のとき}\\
                                \Azeven(q+1,p),&\varepsilon=-\mbox{ のとき}\\
                        \end{array}\right.
		\end{array}
	\vspace{-1em}
	\end{equation*}
	と仮定する。このとき{、}
	$\Hom_{G'}\left(\pi_{\delta,x}^{p+1,q+1}\kern-0.3em\mid_{G'} ,\pi_{\varepsilon,y}^{p,q+1} \right)$の次元は以下のようになる。
%%
\end{theorem}
\note{TODO}
\end{frame}
\begin{frame}{$p,q{\mbox{は奇数の場合}}$}
\begin{center}
$\begin{array}{|@{}c@{}|@{}c@{}|@{}c@{}|}
  \hline
	& \pipy&\pimy\\
  \hline
	\pipx& {\begin{cases}
	1,&x-y\in2\N+\frac{1}{2}\\
	0,&\mbox{それ以外の場合}
\end{cases}}&0\\
  \hline
	\pimx& 0&{\begin{cases}
	1,&y-x\in2\N+\frac{1}{2}\\
	0,&\mbox{それ以外の場合}
\end{cases}}\\
  \hline
\end{array} \newline$
\end{center}
\begin{remark}
	$p,q$が奇数出ない場合も、同様に。
\end{remark}
\note{TODO}
\end{frame}
\begin{frame}
\begin{remark}
	\begin{enumerate}[(1)]
		\item この定理では分岐則が
			連続スペクトルムを持たない場合、すなわち、
			離散分解する場合(一般論は\cite{10.2307/120963})とそうでない場合の両方が含まれている。
		\item 分岐則が離散分解する場合、上記の分岐則は\cite[Thm. 3.3]{kobayashi1993}によって得られた公式と一致する。
		\item 
			離散的に分解{しない}場合の最初の結果はKobayashi--Speh
			\cite[Thms. 12.1 and 1.3]{kobayashi2015symmetry}
			が証明した ($q=0$の場合)
			ている。
	\end{enumerate}
	\vspace{-0.8em}
\end{remark}
\note{
	上記に得られた対称性破れ作用素の分類を用い、
	Zuckerman導来加群函手{$\pi_{\pm,\lambda}^{p,q}$}間の対称性破れ作用素も決定できます。

	簡単のため、具体的な分類情報を省略します。ご興味がある方は、アブストラクトをご参考下さい。

	 この定理では分岐則が離散分解する場合とそうでない場合の両方が含まれています。
		分岐則が離散分解する場合、上記の分岐則は小林先生の1993年の論文によって得られた公式と一致します。
	 前のように、$q=0$の場合の類似の結果はKobayashi--Spehで得られています。
}
\note{TODO}
\end{frame}

\begin{frame}[allowframebreaks]{参考文献}
	\begin{thebibliography}{CK{\O}P11}
\expandafter\ifx\csname urlstyle\endcsname\relax
  \providecommand{\doi}[1]{doi:\discretionary{}{}{}#1}\else
  \providecommand{\doi}{doi:\discretionary{}{}{}\begingroup
  \urlstyle{rm}\Url}\fi

\bibitem[BR04]{bernstein2004estimates}
J.~Bernstein and A.~Reznikov.
\newblock Estimates of automorphic functions.
\newblock \emph{Mosc. Math. J}, \textbf{\textbf{4}}(1), (2004), pp. 19--37.
Available at \url{http://mi.mathnet.ru/eng/mmj141}.

\bibitem[CK{\O}P11]{clerc2011generalized}
J.-L. Clerc, T.~Kobayashi, B.~{\O}rsted and M.~Pevzner.
\newblock Generalized {B}ernstein--{R}eznikov integrals.
\newblock \emph{Math. Ann.}, \textbf{349}(2), (2011), pp. 395--431.
Available at \url{https://doi.org/10.1007/s00208-010-0516-4}.

\bibitem[HT93]{howe1993homogeneous}
R.~E. Howe and E.-C. Tan.
\newblock Homogeneous functions on light cones: the infinitesimal structure of
  some degenerate principal series representations.
\newblock \emph{Bull. Amer. Math. Soc.}, \textbf{28}(1), (1993), pp. 1--74.
Available at \url{http://www.ams.org/journals/bull/1993-28-01/S0273-0979-1993-00360-4/S0273-0979-1993-00360-4.pdf}.

\bibitem[J09]{juhl2009families}
A.~Juhl.
\newblock \emph{Families of {C}onformally {C}ovariant {D}ifferential
  {O}perators, {Q}-curvature and {H}olography}, \emph{Progr. Math,}
  \textbf{275}.
\newblock Springer Science \& Business Media (2009).
Available at \url{http://www.springer.com/in/book/9783764398996}.

\bibitem[K15]{kobayashi2015program}
T.~Kobayashi.
\newblock A program for branching problems in the representation theory of real
  reductive groups.
\newblock \emph{Progr. Math.}, \textbf{312}, (2015), pp. 277--322.
\newblock In: \emph{{\normalfont Special issue in honor of Vogan's 60th years
  birthday}}.
Available at \url{https://doi.org/10.1007/978-3-319-23443-4_10}.

\bibitem[K{\O}03]{KO2}
T.~Kobayashi and B.~{\O}rsted.
\newblock Analysis on the minimal representation of\/ {$\mbox{\rm O}(p,q)$}.{$\;$}{{\rm{II}}}. {B}ranching laws.
\newblock \emph{Adv. Math.}, \textbf{180}(2), (2003), pp. 513--550.
Available at \url{https://doi.org/10.1016/S0001-8708(03)00013-6}.

\bibitem[KO13]{kobayashi2013finite}
T.~Kobayashi and T.~Oshima.
\newblock Finite multiplicity theorems for induction and restriction.
\newblock \emph{Adv. Math.}, \textbf{248}, (2013), pp. 921--944.
Available at \url{http://dx.doi.org/10.1016/j.aim.2013.07.015}.

\bibitem[K93]{kobayashi1993}
T.~Kobayashi.
\newblock The restriction of ${A}_q \left( \lambda \right)$ to reductive
  subgroups.
\newblock \emph{Proc. Japan Acad. Ser. A Math. Sci.}, \textbf{69}(7), (1993),
  pp. 262--267.
Available at \url{http://dx.doi.org/10.3792/pjaa.69.262}.

\bibitem[K98]{10.2307/120963}
T.~Kobayashi.
\newblock Discrete decomposability of the restriction of ${A}_q(\lambda)$ with
  respect to reductive subgroups {I}{I}: Micro-local analysis and asymptotic
  {K}-support.
\newblock \emph{Annals of Mathematics}, \textbf{147}(3), (1998), pp. 709--729.
Available at \url{http://dx.doi.org/10.2307/120963}.

\bibitem[K14]{KOBAYASHI2014272}
T.~Kobayashi.
\newblock F-method for symmetry breaking operators.
\newblock \emph{Differential Geometry and its Applications}, \textbf{33},
  (2014), pp. 272 -- 289.
Available at \url{http://dx.doi.org/10.1016/j.difgeo.2013.10.003}.

\bibitem[K16]{kobayashi16birth}
T.~Kobayashi.
\newblock \emph{Birth of new branching problems}.
\newblock
  日本数学会70年記念 総合講演・企業特別講演アブストラクト, pp. 65--92,
  日本数学会, 2016.
Available at \url{http://www.ms.u-tokyo.ac.jp/~toshi/texpdf/tk2016p-msj70.pdf}.

\bibitem[K{\O}SS15]{kobayashi2015branching}
T.~Kobayashi, B.~{\O}rsted, P.~Somberg and V.~Sou{\v{c}}ek.
\newblock Branching laws for verma modules and applications in parabolic
  geometry. {I}.
\newblock \emph{Adv. Math.}, \textbf{285}, (2015), pp. 1796--1852.
Available at \url{http://dx.doi.org/10.1016/j.aim.2015.08.020}.

\bibitem[KP16a]{kobayashi2016differential1}
T.~Kobayashi and M.~Pevzner.
\newblock Differential symmetry breaking operators: I. {G}eneral theory and
  {F}-method.
\newblock \emph{Selecta Mathematica}, \textbf{22}(2), (2016), pp. 801--845.
Available at \url{http://dx.doi.org/10.1007/s00029-015-0207-9}.

\bibitem[KP16b]{Kobayashi2016}
T.~Kobayashi and M.~Pevzner.
\newblock Differential symmetry breaking operators: {I}{I}. {R}ankin--{C}ohen
  operators for symmetric pairs.
\newblock \emph{Selecta Mathematica}, \textbf{22}(2), (2016), pp. 847--911.
Available at \url{http://dx.doi.org/10.1007/s00029-015-0208-8}.

\bibitem[KS15]{kobayashi2015symmetry}
T.~Kobayashi and B.~Speh.
\newblock \emph{Symmetry {B}reaking for {R}epresentations of {R}ank {O}ne
  {O}rthogonal {G}roups}, \emph{Memoirs of the Amer. Math. Soc,} \textbf{238}.
\newblock American Mathematical Society (2015).
Available at \url{http://dx.doi.org/10.1090/memo/1126}.

\end{thebibliography}

\end{frame}

\begin{frame}{Fact 2の記号$\sol$の説明(準備)}
%%この空間を分析するために、まず以下のような定義する:
\begin{definition} \label{def1}
	$\R^n\times\R^n$上の関数$h(\cdot,\cdot)$を
	$h(b,x):=1-2\,^t\!bI_{p,q}x+\Q(b)\Q(x)$と定める (ここで、$n=p+q$)。 
	$b_p=0$を満たすよう{な}各$b\in\R^n$に対し、開集合$\mysetn{x\in\R^{p,q}}{h(b,x)\neq0}$上で
	\vspace{-1em}
  \begin{equation*}
    \label{eq-Nequiv} | h(b,x) |^{\lambda - n} F \left(
    \frac{x - \Q (x) b}{h(b,x)} \right) = F (x)
	\vspace{-1em}
  \end{equation*}
  が成り立つとき、
	超関数$F \in \mathcal{D}' (\R^{p,q})$を
	{$N_+'$不変}と呼ぶ。(ここで、$N_+'$は$\R^{p,q}$の共形コンパクト化には作用するが、$\R^{p,q}$には作用していないことに注意する)。
\end{definition}
\end{frame}
\end{document}
%talk is 15 minutes long
