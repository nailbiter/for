\begin{frame}{Zuckerman導来函手加群}
	最後に、上記の結果の応用としてZuckerman導来函手加群の間の対称性破れ作用素の問題を論じる。

	$G=O(p+1,q+1)$は$\R^{p+q+2}$の超曲面\begin{equation*}
		X_{p,q}^\pm=\left\{ (x,y)\in\R^{(p+1)+(q+1)}:\sum_{i=1}^{p+1}x_i^2-\sum_{j=1}^{q+1}y_j^2=\pm1 \right\}
	\end{equation*}
	に作用し、さらにヒルベルト空間
	$L^2(X_{p,q}^\pm)$にユニタリ表現を定める。$p,q>0$のとき、このユニタリ表現は
	加算無限個の既約ユニタリ
	表現($X_{p,q}^\pm$の\textbf{離散系列表現})を含む。
	これを$\Pi_{\pm,a}$と書く。パラメータ$a$は\begin{equation*}
		a\in\Z,a>\pm\frac{q-p}{2}+1
	\end{equation*}を動く。既約ユニタリ表現$\Pi_{\pm,a}$はZuckerman導来函手加群を用いて代数的に{も}構成できる。
以下では、この加群の間に対称性破れ作用素がある{か}について論じる。
\note{
	\mytiming{zuckermandef}
	最後に、上記の結果の応用としてZuckerman導来函手加群の間の対称性破れ作用素の問題を論じましょう。
	不定値直交群$G$は$\R^{p+q+2}$の超曲面$X_{p,q}^\pm$に作用し、さらにヒルベルト空間
	$L^2(X_{p,q}^\pm)$にユニタリ表現を定めます。$p,q>0$のとき、このユニタリ表現は
	加算無限個の既約ユニタリ
	表現($X_{p,q}^\pm$の\textbf{離散系列表現})を含みます。
	これを$\Pi_{\pm,a}$と書きます。既約ユニタリ表現$\Pi_{\pm,a}$はZuckerman導来函手加群を用いて代数的に{も}構成できます。
%%	KobayashiとOrstedの2003年の論文
%%	$O(p,q)$の既約ユニタリ表現$\pi_{\pm,\lambda}^{p,q}$
%%	が定まりました。
%%	この論文では$\pi_{\pm,\lambda}^{p,q}$を定義するにあたって5種類の
%%	特徴づけが与えら{れ}、それらは互いに
%%	同値であることが示されています。その特徴づけの1つは
%%	Zuckerman導来函手加群で記載されます。
%%
	以下では、この加群の間に対称性破れ作用
	素があるかについて論じます。
}
\end{frame}
\newcommand{\zuk}{Zuckerman導来加群函手加群}
\begin{frame}{Zuckerman導来加群函手記号のまとめ}
	\begin{equation*}
		\begin{array}[]{c}
			\Pi_{\pm,a}:\mbox{$G=O(p+1,q+1)$の既約ユニタリ表現}\\
			\R^{p+q+2}\mbox{の$X_{p,q}^\pm$の上の$L^2$関数の空間に実現できる表現}\\
			\mbox{パラメータ$a\in\Z,a>\pm\frac{q-p}{2}+1$}
		\end{array}
	\end{equation*}
	\pause
	\begin{equation*}
		\begin{array}[]{c}
			\pi_{\pm,b}:\mbox{$G'=O(p,q+1)$の既約ユニタリ表現}\\
			\R^{p+q+1}\mbox{の$X_{p-1,q}^\pm$の上の$L^2$関数の空間に実現できる表現}\\
			\mbox{パラメータ$b\in\Z,b>\pm\frac{q-p+1}{2}+1$}
		\end{array}
	\end{equation*}
	\note<1>{\mytiming{zkigou}
			$\Pi_{\pm,a}:\mbox{$G=O(p+1,q+1)$の既約ユニタリで、}$
			$\R^{p+q+2}\mbox{の$X_{p,q}^\pm$の上の$L^2$関数の空間に実現できる表現です。}$
			整数の$a$は$\Pi_{\pm,a}$のパラメータでございます。
%%	便利のため、Kobayashi--Orstedのnotationをそのままにお使いではなく、パラメータをちょっと変えます。
%%	アブストラクトで元々のKobayashi--Orstedの記号を用い、全ての結果を述べます。

	なお、このスライドのnotationはアブストラクトのnotationと少し異なります。これは、結果をシンプルに述べる
	ための\kana{変更}{ヘンコウ}です。
	}
	\note<2>{\mytiming{zkigou2}
	$G$とその部分群$G'$の\zuk を区別するため、$G$の表現を大文字の$\Pi$で表し、部分群$G'$の表現を小文字の$\pi$で表します。
	}
\end{frame}
\begin{frame}{Zuckerman導来加群函手{$\Pi_{\pm,a}$と$\pi_{\pm,b}$}の間の対称性破れ作用素}
	\begin{theorem}
		$p,q{\mbox{は奇数の場合}}$
\begin{center}
$\begin{array}{|@{}c@{}|@{}c@{}|@{}c@{}|}
  \hline
  & \pi_{+,b}&\pi_{-,b}\\
  \hline
  \Pi_{+,a}& {\begin{cases}
	1,&a-b\in2\N\\
	0,&\mbox{それ以外の場合}
\end{cases}}&0\\
  \hline
  \Pi_{-,b}& 0&{\begin{cases}
	1,&b-a\in2\N\\
	0,&\mbox{それ以外の場合}
\end{cases}}\\
  \hline
\end{array} \newline$
\end{center}
\end{theorem}
\begin{remark}
	$p,q$が奇数でない場合も、同様{の}結果が成り立つ(アブストラクトに記載)。
\end{remark}
\note{\mytiming{zuckrem}
	上記に得られた対称性破れ作用素の分類を用い、
	\zuk の間の対称性破れ作用素も決定できます。

	簡単のため、$p$と$q$は両方奇数である場合だけ述べます。
	他の例を省略させて頂きました。
	アブストラクト
	の\kana{方}{ホウ}には$p,q$のparity conditionの全ての場合に記述しておきました。
}
\end{frame}
\begin{frame}
\begin{remark}
	\begin{enumerate}[(1)]
		\item この定理では分岐則が
			連続スペクトルムを持たない場合、すなわち、
			離散分解する場合(一般論は\cite{10.2307/120963})とそうでない場合の両方が含まれている。
		\item 分岐則が離散分解する場合、上記の分岐則は\cite[Thm. 3.3]{kobayashi1993}によって得られた公式と一致する。
		\item 
			離散的に分解{しない}場合の最初の結果はKobayashi--Speh
			\cite[Thms. 12.1 and 1.3]{kobayashi2015symmetry}
			が証明した ($q=0$の場合)у
			。
	\end{enumerate}
	\vspace{-0.8em}
\end{remark}
\note{\mytiming{zuckrem2}
	 この定理では分岐則が離散分解する場合とそうでない場合の両方が含まれています。
		分岐則が離散分解する場合、上記の分岐則は小林先生の1993年の論文によって得られた公式と一致します。
	 前のように、$q=0$の場合の類似の結果はKobayashi--Spehで得られています。
}
\end{frame}
