%texmacs ~/for/fortexmacs/master_extract.tm
\documentclass[10pt]{article} % use larger type; default would be 10pt

\usepackage{enumerate}
\usepackage[left=0.6in]{geometry}
\usepackage{setspace}
\usepackage{amsmath,amssymb,bbm,xypic}
\usepackage[all,cmtip]{xy}
\usepackage{amsmath,amssymb,bbm,ulem,float,mystyle}
\usepackage{caption}
\usepackage{subcaption}
\usepackage{setspace}
\usepackage{comment}
\usepackage{catchfilebetweentags}
%\excludecomment{versiona}
\includecomment{versiona}

%%%%%%%%%% Start TeXmacs macros
\catcode`\<=\active \def<{
\fontencoding{T1}\selectfont\symbol{60}\fontencoding{\encodingdefault}}
\catcode`\>=\active \def>{
\fontencoding{T1}\selectfont\symbol{62}\fontencoding{\encodingdefault}}
\newcommand{\assign}{:=}
\newcommand{\comma}{{,}}
\newcommand{\nin}{\not\in}
\newcommand{\tmop}[1]{\ensuremath{\operatorname{#1}}}
\newcommand{\tmtextit}[1]{{\itshape{#1}}}
\newcommand{\um}{-}
\newtheorem{theorem}{Theorem}
\newcommand{\sol}{\mathcal{S}ol(\R^{p,q};\lambda,\nu)}
\newcommand{\Hom}{\mbox{\normalfont Hom}}
\newcommand{\Sol}{\mathcal{S}ol}
\newtheorem{remark}{Remark}
\newtheorem{fact}{Fact}
\newtheorem{definition}{Definition}

\catcode`\<=\active \def<{
\fontencoding{T1}\selectfont\symbol{60}\fontencoding{\encodingdefault}}
\catcode`\>=\active \def>{
\fontencoding{T1}\selectfont\symbol{62}\fontencoding{\encodingdefault}}
\newcommand{\dueto}[1]{\textup{\textbf{(#1) }}}
\newcommand{\tmrsub}[1]{\ensuremath{_{\textrm{#1}}}}
\newcommand{\tmrsup}[1]{\textsuperscript{#1}}
\newcommand{\tmtextbf}[1]{{\bfseries{#1}}}
\newtheorem{proposition}{Proposition}
\newcommand{\Op}{\mbox{\normalfont Op}}
\newcommand{\Res}{\operatorname{Res}\displaylimits}
\newcommand{\OpR}{\mbox{\it R}}
%%%%%%%%%% End TeXmacs macros

\setlength{\parskip}{0.4em}
\setlength{\parindent}{2em}
%%%%%%%%%% End TeXmacs macros

\begin{document}

\title{Symmetry breaking operators of indefinite orthogonal groups $O(p,q)$}

  %%%% 講演者1
  \author{Toshiyuki Kobayashi (The University of Tokyo, Kavli IPMU)\\
  Alex Leontiev (The University of Tokyo)}

  %%%% 講演者2

  %%%% 日付
%  \date{2012年3月26日}

  %%%% 謝辞、キーワード、MSCコード  

  \maketitle

We fix $p, q \in \mathbbm{Z}_{\geq 1}$, $n \assign p + q$, $G \assign O (p +
1, q + 1)$ and $G' \assign O (p + 1, q + 1)_{e_{p + 1}} \assign \mysetn{g \in
G}{g \cdot e_{p + 1} = e_{p + 1}} \simeq O (p, q + 1)$. We shall be interested
in the following
\ExecuteMetaData[.master_extract.tex]{tagsetting}

Here $Q (\cdot)$ denotes the \ $(p, q)$-quadratic form and $J$ is the
diagonal\,matrix with $p$ ``+1'' entries and $q$ ``-1'' entries. We also let
$M' \assign M \cap G'$, $N_{\pm}' \assign N_{\pm} \cap G' \simeq
\mathbbm{R}^{p - 1, q}$, $A' \assign A \cap G' = A$. Then $G$ is a reductive
group (of non-inner type) with $P \assign M A N_+ \subset G$ being a maximal
parabolic subgroup of $G$. $P = M A N_+$ is a Langlands decomposition. In
particular, multiplication $M \times A \times N_+ \rightarrow P$ is a
diffeomorphism. Similarly $P' \assign M' A' N_+' = P \cap G'$ is maximal
parabolic of $G'$.

\begin{definition}
	\label{def-n-nots:def-n+invar}For $F \in \mathcal{D}' (\R^{p,q})$
  we say that $F$ is
  \tmtextbf{$N_+'$-invariant} if $\forall b \in \mathbbm{R}^{p, q}$
  with $b_p = 0$ and $x_0 \in \R^{p,q}$ such that $\frac{x_0 - Q (x_0) b}{1 - 2 Q
  (x_0, b) + Q (x_0) Q (b)} \in \R^{p,q}$ and the expression makes sense (i.e. the
  denominator is non-zero) we have
  \begin{equation*}
    \label{eq-Nequiv} | 1 - 2 Q (b, x) + Q (x) Q (b) |^{\lambda - n} F \left(
    \frac{x - Q (x) b}{1 - 2 Q (x, b) + Q (x) Q (b)} \right) = F (x)
  \end{equation*}
  equality holding for $x$ near $x_0$.
\end{definition}

\begin{definition}
	\label{sol:def-sol}For $F \in \mathcal{D}' (\R^{p,q})$
	we say that $F \in \sol$ if the
  following holds:
  \begin{enumerate}
    \item if $x_0 \in \R^{p,q}$ and $- x_0 \in \R^{p,q}$, then $F (x) = F (- x)$ for $x$
    near $x_0$;
    
    \newcommand{\Stab}{O(p,q)_{e_p}}
    \item if $(m, x_0, m \cdot x_0) \in \Stab \times \R^{p,q} \times \R^{p,q}$, then $F (x)
    = F (m \cdot x)$ for $x$ near $x_0$, where $\Stab \assign \{g \in O (p, q)
    |g \cdot e_p = e_p \}$;
    
    \item if $(\alpha, x_0, \alpha x_0) \in \mathbbm{R}_{> 0} \times \R^{p,q} \times
    \R^{p,q}$, then $\alpha^{\lambda - \nu - n} F (x) = F (\alpha x)$ for $x$ near
    $x_0$;
    
    \item $F$ is $N_+'$-invariant on $\R^{p,q}$.{
    
    }
  \end{enumerate}
\end{definition}
\begin{remark}
    For a closed set $S \subset \R^{p,q}$ we will also use the notation $\mathcal{S}
    \tmop{ol}_S (\R^{p,q} ; \lambda, \nu) \assign \{ u \in \mathcal{S} \tmop{ol} (\R^{p,q} ;
    \lambda, \nu) | \tmop{supp} (u) \subset S \}$
\end{remark}

\begin{fact}[Kobayashi-Speh]
Let $n:=p+q$. The following diagram commutes:
\begin{figure}[H]
\centerline{
	\xymatrixcolsep{5pc}
	\xymatrix{\Hom_{G'}(I(\lambda),J(\nu))\ar[r]^{\simeq} \ar@/^2pc/[rr]^{\mathcal{S}}
	&\left( \mathcal{D}'(G/P,\mathcal{L}_{n-\lambda}) \otimes\mathbb{C}_\nu \right)^{P'}
\ar[r]_-{F\mapsto \supp(F)}\ar[d]^{\simeq}_{\mbox{rest}}
&2^{P'\backslash G/P}\\
&\sol\subset\mathcal{D}'(\R^{p,q})\ar[lu]^{\mbox{Op}}_{\simeq}&
}
}
\end{figure}
\end{fact}

Note that $G$ acts on $\Xi^{p+1,q+1}:=\mysetn{(x,y)\in\R^{p+1,q+1}\setminus\left\{ 0 \right\}}{\myabs{x}^2=\myabs{y}^2}$ and on its quotient space
$X^{p,q}:=\Xi^{p+1,q+1}/\R^{\times}\simeq G/P$. Let
\[
	X:=G/P\simeq X^{p,q},\quad Y:=\mysetn{[\xi:\eta]\in G/P\simeq X^{p,q}}{\xi_{p}=0}\simeq X^{p-1,q}\]
	\[C:=\mysetn{[\xi:\eta]\in G/P\simeq X^{p,q}}{\xi_{0}=\eta_q}\simeq X^{p-1,q-1}\cup\Xi^{p,q},\quad\left\{ [0] \right\}:=\left\{ [1,0_{p+q},1] \right\}\]
\begin{theorem}[classification of closed $P'$-invariant subsets of $G/P$]
	The left $P'$-invariant closed subspaces of $G/P$ are as follows (numbers indicate codimension):\\
  \begin{figure}[H]
    \centering
    \begin{subfigure}[t]{0.3\textwidth}
	    \xymatrixrowsep{0.5pc}
	    \xymatrix{&X\ar@{-}[ld]_1\ar@{-}[rd]^1&\\Y\ar@{-}[rd]_1&&C\ar@{-}[ld]^1\\&Y\cap C\ar@{-}[dd]^{p+q-2}&\\&&\\&\{[0]\}&}
	\caption{when $p>1$}
    \end{subfigure}
    ~ %add desired spacing between images, e. g. ~, \quad, \qquad, \hfill etc. 
      %(or a blank line to force the subfigure onto a new line)
    \begin{subfigure}[t]{0.3\textwidth}
	    \xymatrixrowsep{0.5pc}
	    {\xymatrix{&X\ar@{-}[ld]_1\ar@{-}[rd]^1&\\Y\ar@{-}[rddd]_{p+q-2}&&C\ar@{-}[lddd]^{p+q-2}\\&&\\&&\\&\{[0]\}&}}
	\caption{when $p=1$}
    \end{subfigure}
\end{figure}
\end{theorem}
\begin{theorem}
We can construct the following families of SBOs which holomorphically depend on parameters:\\
\ExecuteMetaData[.master_extract.tex]{table}

Here:
\begin{itemize}
	\item $\mid \mid \mid \assign \{ (\lambda, \nu) \in \mathbbm{C}^2 \mid \nu \in
	- 2\mathbbm{Z}_{\geqslant 0} \cup (q + 1 + 2\mathbbm{Z}) \}$ \item $/ / \assign
\{ (\lambda, \nu) \in \mathbbm{C}^2 \mid \lambda - \nu \in
2\mathbbm{Z}_{\leqslant 0} \}$;
\item $\mid\mid:=\mysetn{(\lambda,\nu)\in\C^2}{\nu\in1+2\Z_{\ge0}}$
\item $\backslash\backslash:=\mysetn{(\lambda,\nu)\in\C^2}{\lambda+\nu-n+1\in-2\Z_{\ge0}}$
\item $Q:=x_1^2+\cdots+x_p^2-x_{p+1}^2-\cdots-x_{p+q}^2$;
\item $\tilde{C}(s,t)$ is a two-variable inflation of renormalized Gegenbauer polynomial, defined as in [Kobayashi-Speh]
\end{itemize}
Moreover,\\
\ExecuteMetaData[.master_extract.tex]{residue}
\end{theorem}
\begin{theorem}
		We can find basis for $\Hom_{G'}(I(\lambda),J(\nu))$ for every $(\lambda,\nu)\in \mathbb{C}^2$. More concretely,
		\ExecuteMetaData[.master_extract.tex]{classification}
\end{theorem}
\begin{theorem}
	Let $1_\lambda\in I(\lambda)^K,1_\nu\in J(\nu)^{K'}$ be the spherical vectors. We then have
\[ \OpR^X_{\lambda, \nu} 1_{\lambda} = 2^{1 -
\lambda} \frac{\pi^{n / 2}}{\Gamma \left( \frac{\lambda}{2} \right)
\Gamma \left(  \frac{\lambda + 1-q}{2} \right) \Gamma \left(
\frac{q - \nu + 1}{2} \right)} 1_{\nu}. \]
\end{theorem}
	\begin{theorem}
		The distribution
		\[K_{\lambda,\nu}^{\mathbb{R}^{p,q}}:=\frac{\myabs{x_p}^{\lambda+\nu-n}}{\Gamma\left( \frac{\lambda+\nu-n+1}{2} \right)}\times
		\frac{\myabs{Q}^{-\nu}}{\Gamma\left( \frac{1-\nu}{2} \right)}\]
		has the pole at $(\lambda,\nu)\in//$ with the residue given by
		\[\Res_{(\lambda,\nu)\in//}K_{\lambda,\nu}^{\mathbb{R}^{p,q}}=\frac{K_{\lambda,\nu}^{\mathbb{R}^{p,q}}}{\Gamma\left( \frac{\lambda-\nu}{2} \right)}
			=\frac{ (- 1)^k k!\pi^{(n - 2) / 2} 
		}{2^{ \nu + 2 k-1}}\cdot  \frac{\sin\left( \frac{1+q-\nu}{2}\pi \right)}{\Gamma\left( \frac{\nu}{2} \right)}
	\tilde{C}_{\nu - \lambda}^{\lambda - \frac{n
  	- 1}{2}} ({\Delta}_{\mathbb{R}^{p-1,q}} {\delta}_{\mathbb{R}^{p+q-1}}, \delta (x_p))
		\]
		where $k:=\frac{\nu-\lambda}{2}$.
		Hence taking $\Op(\cdot)$ on both sides we get
  \[\OpR_{\lambda,\nu}^X  = \frac{ (- 1)^k k!\pi^{(n - 2) / 2} 
		}{2^{ \nu + 2 k-1}}\cdot  \frac{\sin\left( \frac{1+q-\nu}{2}\pi \right)}{\Gamma\left( \frac{\nu}{2} \right)}
     \OpR_{\lambda,\nu}^{ \left\{ 0 \right\} },\quad(\lambda,\nu)\in// . \]
	\end{theorem}
	\begin{theorem}
		Let $\tilde{\mathbb{T}_{\lambda}}:I(\lambda)\to I(n-\lambda)$ be $G$-intertwining Knapp-Stein operator, and similarly for $\nu$. Let $n':=n-1$. We then have:
\ExecuteMetaData[.master_extract.tex]{functional}
	\end{theorem}
	Let:\\
\ExecuteMetaData[.master_extract.tex]{Add}
\begin{theorem}
	The image of regular SBO $R_{\lambda,\nu}^X$ is full when unless $(\lambda,\nu)\in\Z$. In the latter case, they are as follows
	(here for $(\lambda,\nu)\in//$ we set $l:=\frac{\nu-\lambda}{2}$ and for $(\lambda,\nu)\in\backslash\backslash$ we set
	$k:=-\frac{\lambda+\nu-n+1}{2}$):\\
\ExecuteMetaData[.master_extract.tex]{imagesX}
\end{theorem}
\begin{remark}
	Similarly we can compute images of $R_{\lambda,\nu}^{ \left\{ [0] \right\}}$ and $\tilde{R}_{\lambda,\nu}^{X}$
\end{remark}
\begin{theorem}
	The dimension of $G'$-intertwining operator space between certain irreducible representations of $G,G'$ are as follows:\\
\ExecuteMetaData[.master_extract.tex]{Aq}
\end{theorem}
\end{document}

%FIXME
%	beautiful Sol
%	define L_\lambda
%	compute support of R^Y
%	frac fix
%	names for theorems

%TODO:
	%images
	%Aq

%ask:
	%compute images of Y, C?
