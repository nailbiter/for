%mypipes
%texmacs ~/for/fortexmacs/master_extract.tm
\documentclass[12pt]{article} % use larger type; default would be 10pt

\usepackage{enumerate}
\usepackage{geometry}
\usepackage{setspace}
\usepackage{amsmath,amssymb,bbm,xypic}
\usepackage[all,cmtip]{xy}
\usepackage{amsmath,amssymb,bbm,float,mystyle}
\usepackage[normalem]{ulem}
\usepackage{caption}
\usepackage{subcaption}
\usepackage{setspace}
\usepackage{comment}
\usepackage{catchfilebetweentags}
\usepackage{multirow}
\usepackage[table]{xcolor}
\includecomment{versiona}
\usepackage{tikz}
\usepackage{bashful}
\usetikzlibrary{patterns}
\usepackage{bbm}

%%%%%%%%%% Start TeXmacs macros
\catcode`\<=\active \def<{
\fontencoding{T1}\selectfont\symbol{60}\fontencoding{\encodingdefault}}
\catcode`\>=\active \def>{
\fontencoding{T1}\selectfont\symbol{62}\fontencoding{\encodingdefault}}
\newcommand{\assign}{:=}
\newcommand{\comma}{{,}}
\newcommand{\nin}{\not\in}
\newcommand{\tmop}[1]{\ensuremath{\operatorname{#1}}}
\newcommand{\tmtextit}[1]{{\itshape{#1}}}
\newcommand{\um}{-}

\newtheorem{theorem}{Theorem}
\newcommand{\sol}{\mathcal{S}\!{\it ol}(\R^{p,q};\lambda,\nu)}
\newcommand{\Hom}{\mbox{\normalfont Hom}}
\newcommand{\Sol}{\mathcal{S}\!{\it ol}}
\newcommand{\Ind}{\mbox{\normalfont Ind}}
\newcommand{\Supp}{\mathcal{S}\!{\it upp}}
\newtheorem{remark}[theorem]{Remark}
\newtheorem{corollary}[theorem]{Corollary}
\newtheorem{fact}{Fact}
%\newtheorem{definition}{Definition}
\theoremstyle{definition}
\newtheorem{definition}{Definition}

\makeatletter
\newtheoremstyle{exampstyle}
  {\topsep} % Space above
  {\topsep} % Space below
  { {\addtolength{\@totalleftmargin}{3.5cm}
     \addtolength{\linewidth}{-3.5cm}
        \parshape 1 3.5em \linewidth}} % Body font
  {-2.5cm} % Indent amount
  {\bfseries} % Theorem head font
  {.} % Punctuation after theorem head
  {.5em} % Space after theorem head
  {} % Theorem head spec (can be left empty, meaning `normal')
\makeatother

\theoremstyle{exampstyle} \newtheorem{examp}[theorem]{Theorem}

\catcode`\<=\active \def<{
\fontencoding{T1}\selectfont\symbol{60}\fontencoding{\encodingdefault}}
\catcode`\>=\active \def>{
\fontencoding{T1}\selectfont\symbol{62}\fontencoding{\encodingdefault}}
\newcommand{\dueto}[1]{\textup{\textbf{(#1) }}}
\newcommand{\tmrsub}[1]{\ensuremath{_{\textrm{#1}}}}
\newcommand{\tmrsup}[1]{\textsuperscript{#1}}
\newcommand{\tmtextbf}[1]{{\bfseries{#1}}}
\newtheorem{proposition}{Proposition}
\newcommand{\Op}{\mbox{\normalfont Op}}
\newcommand{\Res}{\operatorname{Res}\displaylimits}
\newcommand{\OpR}{\mbox{\it R}}
\renewcommand{\Q}{Q_{p,q}}
\newcommand{\IlambdaGprime}{I(\lambda)\kern-0.3em\mid_{G'}}
\newcommand{\SBO}{\Hom_{G'}\left(\IlambdaGprime,J(\nu) \right)}
\renewcommand{\setminus}{-}
%%%%%%%%%% End TeXmacs macros

\setlength{\parskip}{0.4em}
\setlength{\parindent}{2em}

\newcommand{\even}{2\Z}
\newcommand{\odd}{2\Z+1}
\newcommand{\bb}{\backslash\backslash}
\renewcommand{\ss}{//}
%%%%%%%%%% End TeXmacs macros

\begin{document}

\title{Symmetry breaking operators of indefinite orthogonal groups $O(p,q)$}

  %%%% 講演者1
\author{Toshiyuki Kobayashi\thanks{Partially supported by Grant-in-Aid for Scientific
	Research (A) (25247006), Japan Society for the Promotion of Science.} (The University of Tokyo, Kavli IPMU)\\
  Alex Leontiev (The University of Tokyo)}

  %%%% 講演者2

  %%%% 日付
%  \date{2012年3月26日}

  %%%% 謝辞、キーワード、MSCコード  

  \maketitle
\begin{abstract}
For the pair $(G, G') =(O(p+1, q+1), O(p,q+1))$, we construct and classify
all symmetry breaking operators from spherical, most degenerate 
principal series representions of 
$G$ to those of the subgroup $G'$, extending the results of Kobayashi--Speh in the $q=0$ 
case [Memoirs of Amer. Math. Soc. 2015].
Functional identities, residue formul\ae, and the images of the regular symmetry breaking operators are also provided 
explicitly.
The results contribute to ``stage C'' of the branching program suggested by the first author [Progr. Math. 2015].
\end{abstract}

  \begin{versiona}
	  Let us set up some notation. We define the standard quadratic form
	  $\Q$ on $\R^n$ ($n:=p+q$) of signature $(p,q)$ by
	  \begin{equation*}
  \Q (x) \assign \,^t \! x I_{p, q} x, \; (x \in
  \mathbbm{R}^{p + q}),
	  \end{equation*}
	  where
\begin{equation*}
   I_{p, q} \assign \tmop{diag} (\underbrace{1, \ldots, 1}_p, \underbrace{-
  1, \ldots, - 1}_q).
\end{equation*}
We set $G \assign O (p +
1, q + 1)=\left\{ g\in GL\left( p+q+2,\R \right):\;^t\!gI_{p+1,q+1}g=I_{p+1,q+1} \right\}$, and define
a maximal parabolic subgroup $P=MAN_{+}$ with
\ExecuteMetaData[.master_extract.tex]{tagsetting}
For complex parameter $\lambda\in\C$ we define (unnormalized) spherical degenerate principal series representations of $G$ as
\begin{align*}
I(\lambda)&:=\Ind_P^G(\C_\lambda)\\
&\simeq \left\{ f\in C^{\infty}(G)\mid f(gma(t)n)=e^{-\lambda t}f(g),\;\forall(g,ma(t)n)\in G\times P \right\}.
\end{align*}

We realize $G':=O(p,q+1)$ as the subgroup $G_{e_{p+1}}:=\mysetn{g \in G}{g \cdot e_{p + 1} = e_{p + 1}}$ of $G$. 
Then $G'$ is {\it compatible} with $P$ in the sense that 
$P':=P\cap G'$ is also a maximal parabolic subgroup
with Langlands decomposition $P'=(G'\cap M)A (G'\cap N_+)$,
because $A\subset G'$.
Similarly, we define (unnormalized) spherical degenerate principal series representations $J(\nu):=\Ind_{P'}^{G'}(\C_{\nu})$ of $G'$ for $\nu\in\C$.

The objects of this study are then \textit{symmetry breaking operators} (\textit{SBOs} for short),
that is, $G'$-intertwining operators between the $G$-module $I(\lambda)$ regarded as a $G'$-module by restriction and the $G'$-module $J(\nu)$. We denote by $\SBO$ the totality
of such operators.

The general theory \cite{kobayashi2013finite,kobayashi2014classification} implies the following {\it a priory} estimate of its dimension in our particular setting.
\begin{fact}
	The dimension of $\SBO$ is uniformly bounded in $(\lambda,\nu)\in\C^2$.
\end{fact}
We shall find an explicit basis of $\SBO$ in our setting in Theorem \ref{thm:classif}, and in particular, its dimension in Corollary \ref{cor:classif}.

In order to analyze the space $\SBO$ of symmetry breaking operators, we begin with:
\begin{definition} \label{def1}
	Let $h(b,x):=1-2\,^t\!bI_{p,q}x+\Q(b)\Q(x)$ for $b,x\in\R^{n}\;(n=p+q)$. A distribution
	$F \in \mathcal{D}' (\R^{p,q})$ is said to be
  \tmtextit{$N_+'$-invariant} if 
  for any $b\in\R^n$ with $b_p=0$
  \begin{equation*}
    \label{eq-Nequiv} | h(b,x) |^{\lambda - n} F \left(
    \frac{x - \Q (x) b}{h(b,x)} \right) = F (x)
  \end{equation*}
  holds in the open set of $x\in\R^{p,q}$ satisfying $h(b,x)\neq0$.
  
\end{definition}

\begin{definition}\label{def2}We let $O(p-1,q)$ act on $\R^n$ $(n=p+q)$ by leaving $x_p$ invariant. Let $\sol$ 
	denote the space of distributions $F\in\mathcal{D}'(\R^n)$ satisfying the following four conditions:
\begin{enumerate}[(1)]
    \item $F (x) = F (- x)$;
    \item $F$ is $O(p-1,q)$-invariant;
    \item $F$ is homogeneous of degree $\lambda-\nu-n$;
    \item $F$ is $N_+'$-invariant on $\R^{p,q}$.
  \end{enumerate}
\end{definition}
\end{versiona}

Applying the very general result proven in \cite[Chap.\ 3]{kobayashi2015symmetry} to our particular setting, we get the following:
\begin{fact}[{\cite[Thm. 3.16]{kobayashi2015symmetry}}]\label{fact1}
Let $n:=p+q$. The following diagram commutes:
\begin{figure}[H]
\centerline{
	\xymatrixcolsep{5pc}
	\xymatrix{\SBO\ar[r]^{\simeq} \ar@/^2pc/[rr]^{\Supp}
	&\left( \mathcal{D}'(G/P,\mathcal{L}_{n-\lambda}) \otimes\mathbb{C}_\nu \right)^{P'}
\ar[r]_-{F\mapsto \supp(F)}\ar[d]^{\simeq}_{\mbox{rest}}
&2^{P'\backslash G/P}\\
&{\hspace{1.65cm}\sol\subset\mathcal{D}'(\R^{p,q})}\ar[lu]^{\mbox{Op}}_{\simeq}&
}
}
\end{figure}
\end{fact}

In particular, for $T\in\SBO$, $\Supp(T)$ is a closed subset of $P'\backslash G/P$.
Thus one sees that closed subsets of the finite double coset space $P'\backslash G/P$ provide an important invariant of the symmetry breaking operators. Therefore,
the first step to classify SBOs is to describe explicitly the double coset space $P'\backslash G/P$ together with its closure relations.

The natural action of $G=O(p+1,q+1)$ on $\R^{p+1,q+1}$ leaves
$\Xi^{p+1,q+1}:=\mysetn{(x,y)\in\R^{p+1,q+1}\setminus\left\{ 0 \right\}}{\myabs{x}^2=\myabs{y}^2}$ invariant, and thus $G$ acts naturally on its quotient space
$X^{p,q}:=\Xi^{p+1,q+1}/\R^{\times}$. 
Geometrically, $X^{p,q}$ is identified with the direct product manifold $\Sp^p\times\Sp^q$ equipped with the pseudo-Riemannian metric $g_{\Sp^p}\oplus \left( -g_{\Sp^q} \right)$,
modulo the direct product of antipodal maps, and $G$ is the group of conformal transformations of $X^{p,q}$.
We set
\[
	X:=G/P\simeq X^{p,q},\quad Y:=\mysetn{[\xi:\eta]\in G/P\simeq X^{p,q}}{\xi_{p}=0}\simeq X^{p-1,q}\]
	\[C:=\mysetn{[\xi:\eta]\in G/P\simeq X^{p,q}}{\xi_{0}=\eta_q}\simeq X^{p-1,q-1}\cup\Xi^{p,q},\quad\left\{ [o] \right\}:=\left\{ [1:0_{p+q}:1] \right\}.\]
\begin{theorem}[classification of closed $P'$-invariant subsets of $G/P$]
	Suppose $p,q\ge1$.
	The left $P'$-invariant closed subsets of $G/P$ are described in the following Hasse diagram. Here 
	$
	\begin{array}{l}
	        \xymatrixrowsep{0.5pc}
		\xymatrix{A\ar@{-}[d]^m\\B}
	\end{array}
	$
	means that $A\supset B$ and that the generic part of $B$ is of codimension $m$ in $A$.\\
  \begin{figure}[H]
    \centering
    \begin{subfigure}[t]{0.3\textwidth}
	    \xymatrixrowsep{0.5pc}
	    \xymatrix{&X\ar@{-}[ld]_1\ar@{-}[rd]^1&\\Y\ar@{-}[rd]_1&&C\ar@{-}[ld]^1\\&C\cap Y\ar@{-}[dd]^{p+q-2}&\\&&\\&\{[o]\}&}
	\caption{when $p>1$}
    \end{subfigure}
    ~ %add desired spacing between images, e. g. ~, \quad, \qquad, \hfill etc. 
      %(or a blank line to force the subfigure onto a new line)
    \begin{subfigure}[t]{0.3\textwidth}
	    \xymatrixrowsep{0.5pc}
	    {\xymatrix{&X\ar@{-}[ld]_1\ar@{-}[rd]^1&\\Y\ar@{-}[rddd]_{p+q-2}&&C\ar@{-}[lddd]^{p+q-2}\\&&\\&&\\&\{[o]\}&}}
	\caption{when $p=1$}
    \end{subfigure}
\end{figure}
\end{theorem}
Now, for each closed subset $S$ of $P'\backslash G/P$, we construct a family of SBOs, to be denote by $R^S_{\lambda,\nu}$, such that:
\begin{itemize}
	\item $R_{\lambda,\nu}^S$ is defined for $(\lambda,\nu)\in D_S$, where $D_S$ is the subset of $\C^2$ (more precisely, it is either the whole $\C^2$, or is a countable
		union of one-dimensional complex affine spaces);
	\item $R_{\lambda,\nu}^S$ depends holomorphically on $(\lambda,\nu)\in D_S$;
	\item for every $(\lambda,\nu)\in D_S$ we have $\Supp(R_{\lambda,\nu}^S)\subset S$ and the equality holds for generic $(\lambda,\nu)$.
\end{itemize}
These operators may vanish at special values of $(\lambda,\nu)$ (see Remark \ref{rmk:thm:construction}). Correspondingly, we shall also define
a family of SBOs as a renormalization, to be denoted by $\tilde{R}^X_{\lambda,\nu}$. On the other hand, we shall omit the case when $S=C\cap Y$ for $p=1$;
$S=C$ or $Y$ for $p>1$, as those are not used for the classification below (see Theorem \ref{thm:classif}).
\newpage
\begin{theorem}[construction of SBO]\label{thm:construction}
	For $S=X,Y,C,$ and $\left\{ o \right\}$, the following operators $R_{\lambda,\nu}^S$ and $\tilde{R}_{\lambda,\nu}^X$ are symmetry breaking operators from $\IlambdaGprime$ to $J(\nu)$, which depend holomorphically on $(\lambda,\nu)\in D_S$. Moreover, $\Supp(R_{\lambda,\nu}^S)\subset S$, and are given explicitly as follows.\\
\ExecuteMetaData[.master_extract.tex]{table}\vspace{\baselineskip}
Let us explain the notation in the table.
\begin{itemize}
	\item $\mid \mid \mid \assign \{ (\lambda, \nu) \in \mathbbm{C}^2 \mid \nu \in
	- 2\mathbbm{N} \cup (q + 1 + 2\mathbbm{Z}) \},\quad \backslash\backslash:=\mysetn{(\lambda,\nu)\in\C^2}{\lambda+\nu-n+1\in-2\N}$;
\item $/ / \assign
\{ (\lambda, \nu) \in \mathbbm{C}^2 \mid \lambda - \nu \in
-2\N \},\quad \mid\mid:=\mysetn{(\lambda,\nu)\in\C^2}{\nu\in1+2\N}$;
\item $\tilde{C}(s,t)$ is a polynomial of two-variable's, which obtained by inflation of the renormalized Gegenbauer polynomial, defined as in \cite[(16.3)]{kobayashi2015symmetry}.
\end{itemize}
\end{theorem}
For $p=1$ we define $q_C^X(\lambda,\nu)$ and $q_Y^X(\lambda,\nu)$ bf
\ExecuteMetaData[.master_extract.tex]{residue}
\begin{remark}
	$R_{\lambda,\nu}^S$ is a differential operator if $S=\left\{ o \right\}$ owing to the general theory of differential SBOs established in \cite[Chap.\ 2]{kobayashi2016differential1}.
	By definition, $R_{\lambda,\nu}^{ \left\{ o \right\}}$ in Theorem \ref{thm:construction} amounts to
	\begin{equation*}
		R_{\lambda,\nu}^{ \left\{ o \right\}}=
		\sum_{j=0}^{\frac{\nu-\lambda}{2}}\frac{(-1)^j2^{\nu-\lambda-2j}}{j!(\nu-\lambda-2j)!}\prod_{i=1}^{\frac{\nu-\lambda}{2}-j}\left( \frac{n
		+1}{2}+\frac{\nu+\lambda}{2}
		+i \right)\left(- \Delta_{\mathbbm{R}^{p-1,q}} \right)^j\left( \frac{\partial}{\partial x_p} \right)^{\nu-\lambda-2j}.
	\end{equation*}
	This formula was previously found in \cite[Thms. 5.1.1 and 5.2.1]{juhl2009families}, \cite[(10.1)]{kobayashi2015symmetry} for $q=0$ and in \cite[Thm.\ 4.3]{kobayashi2015branching}
	for general $p,q$.
\end{remark}
\begin{remark}\label{rmk:thm:construction}
	The rightmost column in Theorem \ref{thm:construction} implies $R_{\lambda,\nu}^{ \left\{ o \right\}},R_{\lambda,\nu}^Y,R_{\lambda,\nu}^C\neq0$
	for every $(\lambda,\nu)\in\C^2$, while
	$R^X_{\lambda,\nu}=0$ iff $(\lambda,\nu)$ belongs to the following discrete set
	\[\begin{cases}
			//\cap\mid\mid\mid,&p>1,\\
			\mybra{//\cap\mid\mid\mid} \cup \mybra{\backslash\backslash\cap\mid\mid},&p=1,
		\end{cases}
	\]
	and $\tilde{R}_{\lambda,\nu}^X=0$ iff $p=1$ and $(\lambda,\nu)$ is in the discrete set $\backslash\backslash\cap \mid\mid$.
\end{remark}
The SBOs in Theorem \ref{thm:construction} are not always linearly independent, but exhaust all SBOs. We provide explicit
basis for $\SBO$ for every $(\lambda,\nu)\in \mathbb{C}^2$ as follows:
\begin{theorem}[classification of SBOs]\label{thm:classif}
	Suppose $p,q\ge1$.
		\ExecuteMetaData[.master_extract.tex]{classification}
\end{theorem}
\begin{corollary}\label{cor:classif}
	We have $\dim_{\C}\SBO\in\left\{ 1,2 \right\}$ for all $(\lambda,\nu)\in\C^2$.
\end{corollary}
The degenerate principal series representation $I(\lambda)$ of $G$ contains the one-dimensional subspace of spherical vectors (i.e. $K$-fixed vectors), and likewise $J(\nu)$ of $G'$.
Let $\mathbbm{1}_\lambda\in I(\lambda)^K,\mathbbm{1}_\nu\in J(\nu)^{K'}$ be the spherical vectors normalized so that $\mathbbm{1}_\lambda(e)=\mathbbm{1}_\nu(e)=1$. With this normalization, we have:
\begin{theorem}[spectrum for spherical vectors]\label{thm:spherical}
	Let $n:=p+q\;(p,q\ge1)$ as before.
\[ \OpR^X_{\lambda, \nu} \mathbbm{1}_{\lambda} =  \frac{2^{1 -
\lambda}\pi^{n / 2}}{\Gamma \left( \frac{\lambda}{2} \right)
\Gamma \left(  \frac{\lambda + 1-q}{2} \right) \Gamma \left(
\frac{q - \nu + 1}{2} \right)} \mathbbm{1}_{\nu}. \]
\end{theorem}
\begin{remark}
	Theorem \ref{thm:spherical} was known in \cite[Lem. A.5]{bernstein2004estimates} for $p=q=1$, which was extended in \cite[Thm. 1.1]{clerc2011generalized} for higher dimensional cases.
	See also \cite[Prop.\ 7.4]{kobayashi2015symmetry} for $q=0$ case.
\end{remark}
For $(\lambda,\nu)\in\C^2\setminus//$, we set 
\[K_{\lambda,\nu}^{\mathbb{R}^{p,q}}:=\frac{\myabs{x_p}^{\lambda+\nu-n}}{\Gamma\left( \frac{\lambda+\nu-n+1}{2} \right)}\times
\frac{\myabs{\Q}^{-\nu}}{\Gamma\left( \frac{1-\nu}{2} \right)}\in\sol.\]
Then $R_{\lambda,\nu}^X=\frac{1}{\Gamma\left( \frac{\lambda-\nu}{2} \right)}\Op\left( K_{\lambda,\nu}^X \right)$. We recall that the left-hand side extends to a family of SBOs with holomorphic
parameter $(\lambda,\nu)\in\C^2$.
\begin{theorem}[residue formula]
	Let $n:=p+q\;(p,q\ge1)$ as before.
	For $(\lambda,\nu)\in//$, we set $l:=\frac{\nu-\lambda}{2}\in\N$. Then we have
  \[\OpR_{\lambda,\nu}^X  = \frac{ (- 1)^l l!\pi^{(n - 2) / 2} 
		}{2^{ \nu + 2 l-1}}\cdot  \frac{\sin\left( \frac{1+q-\nu}{2}\pi \right)}{\Gamma\left( \frac{\nu}{2} \right)}
     \OpR_{\lambda,\nu}^{ \left\{ o \right\} },\quad(\lambda,\nu)\in// . \]
	\end{theorem}
	\begin{remark}
		The residue formula in the case $q=0$ was given in \cite[Thm. 12.2]{kobayashi2015symmetry}.
	\end{remark}
	\begin{definition}
		Similarly to the construction of Fact \ref{fact1}, for $G=O(p+1,q+1)$ we have $\Hom_G(I(\lambda),I(\nu))\simeq\Sol_G(\R^{p,q};\lambda,\nu)$
		where $\Sol_G(\R^{p,q};\lambda,\nu)\subset\mathcal{D}'(\R^{p+q})$ is defined to be the space of generalized functions on $\R^{p+q}$ that satisfy
		the four items in Definition \ref{def2}, except that in second item $O(p,q)_{e_p}$ is replaced by $O(p,q)$ and the fourth item is replaced by $N_+$-invariance
		on $\R^{p,q}$, which in turn is defined as in Definition \ref{def1}, with the only difference that we do not assume $b_p=0$ anymore.

		Now, the generalized function defined as
		\begin{equation*}
			\myabs{\Q}^{\lambda-n}\times\begin{cases}
				\Gamma^{-1}\left( \lambda-n/2 \right),&\min\left\{ p,q \right\}=0,\\
				\Gamma^{-1}\left( \frac{\lambda-n+1}{2} \right)\Gamma^{-1}\left( \lambda-n/2 \right),&\min\left\{ p,q \right\}>0,n\in2\Z+1,\\
  \Gamma^{-1} \left( \frac{\lambda-n + 1}{2} \right) \Gamma ^{-1}\left( \frac{\lambda-n/2+
  1}{2} \right), &\min\left\{ p,q \right\}>0, n / 2 + p \in 2\mathbbm{Z}+ 1,\\
  \Gamma^{-1} \left( \frac{\lambda-n + 1}{2} \right) \Gamma ^{-1}\left( \frac{\lambda-n/2}{2}
  \right), & \min\left\{ p,q \right\}>0,n / 2 + p \in 2\mathbbm{Z}
			\end{cases}
		\end{equation*}
		belongs to $\Sol_G(\R^{p,q};\lambda,n-\lambda)$ and we can use it to
		define an intertwining operator of $G=O(p+1,q+1)$,
		$\tilde{\mathbb{T}}^{G}_{\lambda}:I(\lambda)\to
		I(n-\lambda)$
		(\textit{Knapp--Stein operator}).
		The result of this construction repeated with $G'=O(p,q+1)$ in place of $G$ will be denoted by $\tilde{\mathbb{T}}^{G'}_\nu:J(\nu)\to J(n-1-\nu)$.
	\end{definition}
	\begin{theorem}[functional identities]
		Let $n:=p+q\;(p,q\ge1)$ as before.
		We have:
\ExecuteMetaData[.master_extract.tex]{functional}
	\end{theorem}
	\begin{remark}
		The functional identities in the case $q=0$ were proven in \cite[Thm. 12.6]{kobayashi2015program}.
	\end{remark}
	Since the representation $J(\nu)$ of $G'=O(p,q+1)$ is multiplicity-free as a $K'$-module, we can describe its $(\mathfrak{g}',K')$-submodules by means of subsets of $\N_{+}$
	for $p>1$, which parametrize the $K'$-structure of $J(\nu)$ by the spherical harmonics $\mathcal{H}^a(\Sp^{p-1})\boxtimes\mathcal{H}^b(\Sp^q)$.
	As in \cite{howe1993homogeneous}, we also indicate the Jordan--H\"older series (socle filtrations) of $J(\nu)$ by using arrows.
	We then have:
\begin{theorem}[images of SBOs]
	The regular SBO $R_{\lambda,\nu}^X:I(\lambda)\to J(\nu)$ is surjective,
	unless $\nu\in\Z$. In the latter case, the images of the underlying $(\mathfrak{g},K)$-module $I(\lambda)_K$ under
	$R_{\lambda,\nu}^X$ are given as follows (here for $(\lambda,\nu)\in//$ we set $l:=\frac{\nu-\lambda}{2}\in\N$ and for $(\lambda,\nu)\in\backslash\backslash$ we set
	$k:=-\frac{\lambda+\nu-n+1}{2}\in\N$; the barriers $A^{\pm\pm}$ are defined as in \cite{howe1993homogeneous}): \\
	for $p>1$:
\end{theorem}
\begin{enumerate}[(1)]
	\item Suppose $p\in2\N_++1$ and $q\in2\Z$. Then, if $\nu\in2\Z,0<\nu<n-1$, $R_{\lambda,\nu}^X$ is surjective. Otherwise,
		\hspace*{-1cm}\begin{figure}[H]
			\noindent\begin{tabular}{m{3.2cm}p{3.5cm}p{3.5cm}p{3.5cm}}
	      $(\lambda,\nu)\in$&$\mybra{//\cup\backslash\backslash}^c$ & $\backslash\backslash-//$  & $//\cap\backslash\backslash,k> l$\\[15pt]
	      {\vspace{-3cm} $ \even\ni\nu\leq0$}:&\input{|"guile  -e mp $HOME/for/forscheme/ma.scm '((App 0 1 0))' '((App 0 1 0))' '((App 0 3 0 K0)(App 0 1 0 Kt))'"}\\[15pt]
	      \vspace{-3cm}$\odd\ni\nu\leq\frac{n-3}{2}$:&\input{|"guile -e mp $HOME/for/forscheme/ma.scm '((Apm 0 1 0)(Amp 0 2 0))' '((Apm 0 1 0)(Amp 0 0 0))' '((Apm 0 11 0 Kt)(Amp 0 3 0 K0))'"}\\[15pt]
	      $(\lambda,\nu)\in$&$\mybra{//\cup\backslash\backslash}^c$ && $//\cap\backslash\backslash,k=l$\\[15pt]
	      \vspace{-3cm}$\odd\ni\nu=\frac{n-1}{2}:$&\input{|"guile  -e mp $HOME/for/forscheme/ma.scm '((Apm 0 11 0)(Apm 1 0 0))' '()' '((Apm 0 3 0 K0)(Apm 1 11 0 Kt))'"}\\[15pt]
	      $(\lambda,\nu)\in$&$\mybra{//\cup\backslash\backslash}^c$ & $//-\backslash\backslash$  & $//\cap\backslash\backslash,k< l$\\[15pt]
	      \vspace{-3cm}$\even\ni\nu\ge n-1$:&\input{|"guile  -e mp $HOME/for/forscheme/ma.scm '((App 1 3 1))' '((App 1 3 1))' '((App 1 2 1))'"}\\[15pt]
	      \vspace{-3cm}$\odd\ni\nu\ge\frac{n+1}{2}$:&\input{|"guile -e mp $HOME/for/forscheme/ma.scm '((Apm 1 0 1)(Amp 1 2 1))' '((Apm 1 3 1 K0)(Amp 1 2 1 Kt))' '((Apm 1 2 1 K0)(Amp 1 2 1 K0)(Amp 1 2 1 Kt))'"}\\[25pt]
	    \end{tabular}
	  \end{figure}
	\item Suppose $p,q\in\odd$ and $p>1$. Then,
		\begin{figure}[H]
			\noindent\begin{tabular}{m{3.2cm}p{3.5cm}p{3.5cm}p{3.5cm}}
			$(\lambda,\nu)\in$&$\mybra{\ss\cup\bb}^c$ & $\bb-\ss$  & $\ss-\bb$\\[15pt]
			\vspace{-3cm}$\odd\ni\nu\le0$:&\input{|"guile -e mp $HOME/for/forscheme/ma.scm '((App 0 1 0)(Apm 0 0 0))' '((App 0 1 0)(Apm 0 0 0))' '((App 0 1 0 Kt)(Apm 0 3 0 K0))'"}\\[15pt]
			\vspace{-3cm}$\odd\ni\nu\le n-3$:&\input{|"guile -e mp $HOME/for/forscheme/ma.scm '((Amp 0 3 0))' '((Amp 0 11 0))' '((Amp 0 3 0))'"}\\[15pt]
			\vspace{-3cm}$\even\ni\nu>0$:&\input{|"guile -e mp $HOME/for/forscheme/ma.scm '((Apm 0 11 0))' '((Apm 0 11 0))' '((Apm 0 11 0 Kt)(Apm 0 3 0 K0))'"}\\[15pt]
			\vspace{-3cm}$\odd\ni\nu>n-3$:&\input{|"guile -e mp $HOME/for/forscheme/ma.scm '((App 1 3 1)(Apm 1 0 1))' '((App 1 0 1)(Apm 1 2 1))' '((App 1 3 1)(Apm 1 0 1))'"}\\[15pt]
			  
		\end{tabular}
		\end{figure}
	\item Suppose $p,q\in\even$. Then,
		\begin{figure}[H]
			\noindent\begin{tabular}{m{3.2cm}p{3.5cm}p{3.5cm}p{3.5cm}}
			$(\lambda,\nu)\in$&$\mybra{\ss\cup\bb}^c$ & $\bb-\ss$  & $\ss-\bb$\\[15pt]
			\vspace{-3cm}$\odd\ni\nu\le0$:&\input{|"guile -e mp $HOME/for/forscheme/ma.scm '((App 0 1 0)(Amp 0 0 0))' '((App 0 1 0)(Amp 0 0 0))' '((App 0 1 0 Kt)(Amp 0 3 0 K0))'"}\\[15pt]
			\vspace{-3cm}$\odd\ni\nu\le n-3$:&\input{|"guile -e mp $HOME/for/forscheme/ma.scm '((Apm 0 3 0))' '((Apm 0 11 0))' '((Apm 0 3 0 K0)(Apm 0 11 0 Kt))'"}\\[15pt]
			\vspace{-3cm}$\even\ni\nu>0$:&\input{|"guile -e mp $HOME/for/forscheme/ma.scm '((Amp 0 11 0))' '((Amp 0 11 0))' '((Amp 0 3 0))'"}\\[15pt]
			\vspace{-3cm}$\odd\ni\nu>n-3:$&\input{|"guile -e mp $HOME/for/forscheme/ma.scm '((App 1 0 1)(Amp 1 2 1))' '((App 1 0 1)(Amp 1 2 1))' '((App 1 3 1 K0)(Amp 1 2 1 Kt))'"}\\[15pt]
		\end{tabular}
		\end{figure}
	\item Suppose $p\in\even,q\in\odd$. Then for $\nu\in\odd$ we have $R_{\lambda,\nu}^X$ being surjective. Otherwise (for $\nu\in\even$) we have,
	  \begin{figure}[H]
			\noindent\begin{tabular}{m{3.2cm}p{3.5cm}p{3.5cm}p{3.5cm}}
	      $(\lambda,\nu)\in$&$\mybra{//\cup\backslash\backslash}^c$ & $\backslash\backslash-//$  & $//\cap\backslash\backslash,k> l$\\[15pt]
	      \vspace{-3cm}$\nu\leq0$:&\input{|"guile -e mp $HOME/for/forscheme/ma.scm '((App 0 1 0)(Apm 0 0 0)(Amp 0 0 0))' '((App 0 1 0)(Apm 0 0 0)(Amp 0 0 0))' '((App 0 1 0 Kt)(Apm 0 3 0 K0)(Amp 0 0 0))'"}\\[15pt]
	      \vspace{-3cm}$0<\nu\leq\frac{n-3}{2}$:&\input{|"guile -e mp $HOME/for/forscheme/ma.scm '((Apm 0 1 0)(Amp 0 2 0))' '((Apm 0 1 0)(Amp 0 0 0))' '((Apm 0 11 0 Kt)(Amp 0 3 0 K0))'"}\\[25pt]
              $(\lambda,\nu)\in$&$\mybra{//\cup\backslash\backslash}^c$ && $//\cap\backslash\backslash,k=l$\\[15pt]
	      \vspace{-3cm}$\odd\ni\nu=\frac{n-1}{2}:$&\input{|"guile  -e mp $HOME/for/forscheme/ma.scm '((Apm 0 11 0)(Apm 1 0 0))' '()' '((Apm 0 11 0 Kt)(Apm 1 3 0 K0))'"}\\[25pt]
	      $(\lambda,\nu)\in$&$\mybra{//\cup\backslash\backslash}^c$ & $//-\backslash\backslash$  & $//\cap\backslash\backslash,k< l$\\[15pt]
	      \vspace{-3cm}$\frac{n+1}{2}\le\nu\le n-3$:&\input{|"guile -e mp $HOME/for/forscheme/ma.scm '((Apm 1 0 1)(Amp 1 2 1))' '((Apm 1 3 1 K0)(Amp 1 2 1 Kt))' '((Apm 1 2 1 K0)(Amp 1 2 1 Kt)(Amp 1 2 1 K0))'"}\\[25pt]
	      \vspace{-3cm}$\nu>n-3$:&\input{|"guile -e mp $HOME/for/forscheme/ma.scm '((App 1 0 1)(Apm 1 0 1)(Amp 1 2 1))' '((App 1 0 1)(Apm 1 3 1 K0)(Amp 1 2 1 Kt))' '((App 1 0 1)(Apm 1 2 1 K0)(Amp 1 2 1 Kt)(Amp 1 2 1 K0))'"}\\[15pt]
	    \end{tabular}
	  \end{figure}
	\end{enumerate}
	In the diagrams above some of them are filled not with gray, but with colored diagonal lines. This means that the image of the regular
	SBO $R_{\lambda,\nu}^X$ is zero and the (green/purple)
	ascending/descending diagonal lines show the images of its residues $R_{\lambda,\nu}^{ \left\{ o \right\}}$ and $\tilde{R}_{\lambda,\nu}^X$ respectively.

	For $p=1$ we have:\\
	\begin{figure}[H]
		\begin{tabular}{p{3.2cm}p{2.0cm}p{2.0cm}p{2.0cm}p{2.3cm}p{2.3cm}}
		$(\lambda,\nu)\in$ & $\mybra{\ss\cup\bb}^c$ & $\ss-\bb$ & $\bb-\ss$ & $\ss\cap\bb,k<l$ & $\ss\cap\bb,k\geq l$\\
		\vspace{-0.7cm}$\even\ni\nu\le0$:&\input{|"guile -e mp1 $HOME/for/forscheme/ma.scm 		'((def 0 1))' '((Kt 0 1)(K0 0 2))' '((def 0 1))' '()' '((K0 0 2)(Kt 0 1))'"}\\
		\vspace{-0.5cm}$\nu,q\in2\Z,0<\nu<q$:&\input{|"guile -e mp1 $HOME/for/forscheme/ma.scm 		'((def 1))' '((def 1))' '((def 1))' '((def 1))' '((def 1))'"}\\
		\vspace{-0.5cm}$\nu\in2\Z,q\in2\Z+1,0<\nu<q$:&\input{|"guile -e mp1 $HOME/for/forscheme/ma.scm 	'((def 1))' '((Kt 1)(K0 1))' '((def 1))' '((def 1))' '((def 1))'"}\\
		\vspace{-0.7cm}$q\in\even,\even\ni\nu\ge q$:&\input{|"guile -e mp1 $HOME/for/forscheme/ma.scm 	'((def 1 2))' '((def 1 2))' '((def 1 1))' '((def 1 1))' '()'"}\\
		\vspace{-0.7cm}$q\in\odd,\even\ni\nu\ge q$:&\input{|"guile -e mp1 $HOME/for/forscheme/ma.scm 	'((def 1 1))' '((Kt 1 1)(K0 1 2))' '((def 1 1))' '((def 1 1))' '()'"}\\
		\vspace{-0.7cm}$q\in\even,\odd\ni\nu\le0$:&\input{|"guile -e mp1 $HOME/for/forscheme/ma.scm 	'((def 0 2))' '((Kt 0 2)(K0 0 2))' '((def 0 2))' '()' '((K0 0 2)(Kt 0 2))'"}\\
		\vspace{-0.7cm}$q\in\odd,\odd\ni\nu\le0$:&\input{|"guile -e mp1 $HOME/for/forscheme/ma.scm 	'((def 0 2))' '((def 0 2))' '((def 0 2))' '()' '((def 0 2))'"}\\
		\vspace{-0.5cm}$q\in\even,\nu\in\odd,0<\nu<q$:&\input{|"guile -e mp1 $HOME/for/forscheme/ma.scm	'((def 1))' '((Kt 1)(K0 1))' '((KC 1)(KY 1))' '((Kt 1)(K0 1))' '((def 1))'"}\\
		\vspace{-0.5cm}$q,\nu\in\odd,0<\nu<q$:&\input{|"guile -e mp1 $HOME/for/forscheme/ma.scm 	'((def 1))' '((def 1))' '((KC 1)(KY 1))' '((Kt 1)(K0 1))' '((def 1))'"}\\
		\vspace{-0.7cm}$q\in\even,\odd\ni\nu\ge q$:&\input{|"guile -e mp1 $HOME/for/forscheme/ma.scm 	'((def 1 1))' '((Kt 1 1)(K0 1 2))' '((KY 1 1)(KC 1 1))' '((Kt 1 1)(K0 1 1))' '()'"}\\
		\vspace{-0.7cm}$q\in\odd,\odd\ni\nu\ge q$:&\input{|"guile -e mp1 $HOME/for/forscheme/ma.scm 	'((def 1 2))' '((def 1 2))' '((KY 1 1)(KC 1 2))' '((K0 1 2)(Kt 1 2))' '()'"}\\
	\end{tabular}\end{figure}
	In the diagrams above some of them are filled not with gray, but with colored diagonal lines. This means that the image of the regular SBO $R_{\lambda,\nu}^X$ is zero and:
	\begin{itemize}
		\item For $(\lambda,\nu)\in\ss$ the (green/purple)
			ascending/descending diagonal lines show the images of its residues $R_{\lambda,\nu}^{ \left\{ o \right\}}$ and $\tilde{R}_{\lambda,\nu}^X$ 
			respectively.
		\item For $(\lambda,\nu)\in\ss$ the (blue/red) ascending/descending diagonal lines show the images of its residues $R_{\lambda,\nu}^{Y}$ and ${R}_{\lambda,\nu}^C$ 
			respectively.
	\end{itemize}
\begin{remark}
	Although we omit the corresponding diagrams for brevity, the images of the other SBO constructed in Theorem 2, we can find their images as well. Note that
	the proof of this theorem is performed \textit{independent of} of \cite{howe1993homogeneous}.
\end{remark}
Now, we recall from \cite{KO2} the five equivalent definitions of the
irreducible unitary representations $\pi_{\pm,\lambda}^{p,q}$ of $O(p,q)$.
\begin{theorem}[$G'$-invariant maps between Zuckerman modules $\pi_{\pm,\lambda}^{p,q}$]\label{thm:Aq}
	Let $n=p+q,\;(p,q\ge1)$ and $n':=n-1$.
	The dimensions of $\Hom_{G'}\left(\pi_{\pm,{n}/{2}-\lambda}^{p+1,q+1}\kern-0.3em\mid_{G'} ,\pi_{\pm,\nu-{n'}/{2}}^{p,q+1} \right)$
	are as follows:\newline
\ExecuteMetaData[.master_extract.tex]{Aq}\\\vspace{\baselineskip}
\vspace{\baselineskip}Theorem \ref{thm:Aq} generalizes \cite[Thms. 12.1 and 1.3]{kobayashi2015symmetry}.
\end{theorem}
\nocite{kobayashi2015program}
\small
\bibliography{todai_master}
\bibliographystyle{mystyle}
\end{document}
