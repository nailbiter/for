%mypipes
%texmacs ~/for/fortexmacs/master_extract.tm
\documentclass[12pt]{article} % use larger type; default would be 10pt

\usepackage{enumerate}
\usepackage{setspace}
\usepackage{amsmath,amssymb,bbm,xypic}
\usepackage[all,cmtip]{xy}
\usepackage{amsmath,amssymb,bbm,float,mystyle}
\usepackage[normalem]{ulem}
\usepackage{caption}
\usepackage{subcaption}
\usepackage{setspace}
\usepackage{comment}
\usepackage{catchfilebetweentags}
\usepackage{multirow}
\usepackage[table]{xcolor}
\includecomment{versiona}
\usepackage{tikz}
\usepackage{bashful}
\usetikzlibrary{patterns}
\usepackage{bbm}

%%%%%%%%%% Start TeXmacs macros
\catcode`\<=\active \def<{
\fontencoding{T1}\selectfont\symbol{60}\fontencoding{\encodingdefault}}
\catcode`\>=\active \def>{
\fontencoding{T1}\selectfont\symbol{62}\fontencoding{\encodingdefault}}
\newcommand{\assign}{:=}
\newcommand{\comma}{{,}}
\newcommand{\nin}{\not\in}
\newcommand{\tmop}[1]{\ensuremath{\operatorname{#1}}}
\newcommand{\tmtextit}[1]{{\itshape{#1}}}
\newcommand{\um}{-}
\newtheorem{theorem}{Theorem}
\newtheorem*{corollary}{Corollary}
\newcommand{\sol}{\mathcal{S}\!{\it ol}(\R^{p,q};\lambda,\nu)}
\newcommand{\Hom}{\mbox{\normalfont Hom}}
\newcommand{\Sol}{\mathcal{S}\!{\it ol}}
\newcommand{\Ind}{\mbox{\normalfont Ind}}
\newcommand{\Supp}{\mathcal{S}\!{\it upp}}
\newtheorem*{remark}{Remark}
\newtheorem{fact}{Fact}
%\newtheorem{definition}{Definition}
\theoremstyle{definition}
\newtheorem{definition}{Definition}

\catcode`\<=\active \def<{
\fontencoding{T1}\selectfont\symbol{60}\fontencoding{\encodingdefault}}
\catcode`\>=\active \def>{
\fontencoding{T1}\selectfont\symbol{62}\fontencoding{\encodingdefault}}
\newcommand{\dueto}[1]{\textup{\textbf{(#1) }}}
\newcommand{\tmrsub}[1]{\ensuremath{_{\textrm{#1}}}}
\newcommand{\tmrsup}[1]{\textsuperscript{#1}}
\newcommand{\tmtextbf}[1]{{\bfseries{#1}}}
\newtheorem{proposition}{Proposition}
\newcommand{\Op}{\mbox{\normalfont Op}}
\newcommand{\Res}{\operatorname{Res}\displaylimits}
\newcommand{\OpR}{\mbox{\it R}}
\renewcommand{\Q}{Q_{p,q}}
%%%%%%%%%% End TeXmacs macros

\setlength{\parskip}{0.4em}
\setlength{\parindent}{2em}

\newcommand{\even}{2\Z}
\newcommand{\odd}{2\Z+1}
\newcommand{\bb}{\backslash\backslash}
\renewcommand{\ss}{//}
%%%%%%%%%% End TeXmacs macros

\begin{document}

\title{Symmetry breaking operators of indefinite orthogonal groups $O(p,q)$}

  %%%% 講演者1
  \author{Toshiyuki Kobayashi (The University of Tokyo, Kavli IPMU)\\
  Alex Leontiev (The University of Tokyo)}

  %%%% 講演者2

  %%%% 日付
%  \date{2012年3月26日}

  %%%% 謝辞、キーワード、MSCコード  

  \maketitle
\begin{abstract}
For the pair $(G, G') =(O(p+1, q+1), O(p,q+1))$, we classify
all symmetry breaking operators from spherical, most degenerate 
principal series representions of 
$G$ to those of $G'$, extending the results of Kobayashi--Speh in the $q=0$ 
case (\cite{kobayashi2015symmetry}).
The images of the regular symmetry breaking operators are also provided 
explicitly.
\end{abstract}

  \begin{versiona}
We fix $p, q \in \mathbbm{N}_+$, $n \assign p + q$, $G \assign O (p +
1, q + 1)$. We shall be interested
in the following
\ExecuteMetaData[.master_extract.tex]{tagsetting}
Then $P:=MAN_{+}$ is a Langlands decomposition of a maximal parabolic subgroup of $G$.
For complex parameter $\lambda\in\C$ we also define spherical degenerate principal series representations of $G$ as
\begin{equation*}
I(\lambda):=\Ind_P^G(\C_\lambda)\simeq \left\{ f\in C^{\infty}(G)\mid f(gma(t)n)=e^{-\lambda t}f(g),\;\forall(g,ma(t)n)\in G\times P \right\}
\end{equation*}
We realize $G':=O(p,q)$ as the subgroup $G_{e_{p+1}}:=\mysetn{g \in G}{g \cdot e_{p + 1} = e_{p + 1}}$ of $G$. 
Note that $P':=P\cap G'$ is a maximal parabolic subgroup
with Langlands decomposition $P'=(G'\cap M)A (G'\cap N_+)$ (note that in this case
$A=A\cap G'$). We similarly define spherical degenerate principal series representations $J(\nu):=\Ind_{P'}^{G'}(\C_{\nu})$ of $G'$ corresponding to complex parameter $\nu\in\C$.

The objects of this study are then the \textit{symmetry breaking operators} (or, as we will briefly call them, \textit{SBOs}),
that is the members of the space of $G'$-intertwining $I(\lambda)\to J(\nu)$ operators
$\Hom_{G'}(I(\lambda),J(\nu))$.
\begin{fact}[\cite{kobayashi2014classification,kobayashi2015classification}]
	Results of \cite{kobayashi2014classification,kobayashi2015classification} imply that the dimension of $\Hom_{G'}(I(\lambda),J(\nu))$ is uniformly bounded in $(\lambda,\nu)\in\C^2$. That is,
	\begin{equation*}
		\exists C,\;\forall(\lambda,\nu)\in\C^2,\quad \dim\Hom_{G'}(I(\lambda),J(\nu))\le C.
	\end{equation*}
\end{fact}

\begin{definition}
	\label{def1}For $F \in \mathcal{D}' (\R^{p,q})$
  we say that $F$ is
  \tmtextit{$N_+'$-invariant} if $\forall b \in \mathbbm{R}^{p, q}$
  with $b_p = 0$ and $x_0 \in \R^{p,q}$ such that $\frac{x_0 - \Q (x_0) b}{1 - 2 \Q
  (x_0, b) + \Q (x_0) \Q (b)} \in \R^{p,q}$ and the expression makes sense (i.e. the
  denominator is non-zero) we have
  \begin{equation*}
    \label{eq-Nequiv} | 1 - 2 \Q (b, x) + \Q (x) \Q (b) |^{\lambda - n} F \left(
    \frac{x - \Q (x) b}{1 - 2 \Q (x, b) + \Q (x) \Q (b)} \right) = F (x)
  \end{equation*}
  equality holding for $x$ near $x_0$.
\end{definition}

\begin{definition}
	\label{def2}For $F \in \mathcal{D}' (\R^{p,q})$
	we say that $F \in \sol$ if the
  following holds:
  \begin{enumerate}
    \item if $x_0 \in \R^{p,q}$ and $- x_0 \in \R^{p,q}$, then $F (x) = F (- x)$ for $x$
    near $x_0$;
    
    \newcommand{\Stab}{O(p,q)_{e_p}}
    \item if $(m, x_0, m \cdot x_0) \in \Stab \times \R^{p,q} \times \R^{p,q}$, then $F (x)
    = F (m \cdot x)$ for $x$ near $x_0$, where $\Stab \assign \{g \in O (p, q)
    |g \cdot e_p = e_p \}$;
    
    \item if $(\alpha, x_0, \alpha x_0) \in \mathbbm{R}_{> 0} \times \R^{p,q} \times
    \R^{p,q}$, then $\alpha^{\lambda - \nu - n} F (x) = F (\alpha x)$ for $x$ near
    $x_0$;
    
    \item $F$ is $N_+'$-invariant on $\R^{p,q}$.{
    
    }
  \end{enumerate}
\end{definition}
\end{versiona}

Applying the very general result proven in \cite[chapter 3]{kobayashi2015symmetry} to our particular setting we get the following:
\begin{fact}[{\cite[Theorem 3.16]{kobayashi2015symmetry}}]\label{fact1}
Let $n:=p+q$. The following diagram commutes:
\begin{figure}[H]
\centerline{
	\xymatrixcolsep{5pc}
	\xymatrix{\Hom_{G'}(I(\lambda),J(\nu))\ar[r]^{\simeq} \ar@/^2pc/[rr]^{\Supp}
	&\left( \mathcal{D}'(G/P,\mathcal{L}_{n-\lambda}) \otimes\mathbb{C}_\nu \right)^{P'}
\ar[r]_-{F\mapsto \supp(F)}\ar[d]^{\simeq}_{\mbox{rest}}
&2^{P'\backslash G/P}\\
&\sol\subset\mathcal{D}'(\R^{p,q})\ar[lu]^{\mbox{Op}}_{\simeq}&
}
}
\end{figure}
\end{fact}

In particular, for $T\in\Hom_{G'}(I(\lambda),J(\nu))$ $\Supp(T)$ is a closed subset of $P'\backslash G/P$.
Hence, one sees that closed subsets of the finite double coset space $P'\backslash G/P$ provide an important invariant of the symmetry breaking operators. Therefore,
the first step to classify SBOs is to classify explicitly the double coset space $P'\backslash G/P$ together with the closure relations.

Note that $G$ acts on $\Xi^{p+1,q+1}:=\mysetn{(x,y)\in\R^{p+1,q+1}\setminus\left\{ 0 \right\}}{\myabs{x}^2=\myabs{y}^2}$ and on its quotient space
$X^{p,q}:=\Xi^{p+1,q+1}/\R^{\times}\simeq G/P$. Let
\[
	X:=G/P\simeq X^{p,q},\quad Y:=\mysetn{[\xi:\eta]\in G/P\simeq X^{p,q}}{\xi_{p}=0}\simeq X^{p-1,q}\]
	\[C:=\mysetn{[\xi:\eta]\in G/P\simeq X^{p,q}}{\xi_{0}=\eta_q}\simeq X^{p-1,q-1}\cup\Xi^{p,q},\quad\left\{ [0] \right\}:=\left\{ [1,0_{p+q},1] \right\}\]
\begin{theorem}[classification of closed $P'$-invariant subsets of $G/P$]
	The left $P'$-invariant closed subspaces of $G/P$ are as follows (numbers indicate codimension):\\
  \begin{figure}[H]
    \centering
    \begin{subfigure}[t]{0.3\textwidth}
	    \xymatrixrowsep{0.5pc}
	    \xymatrix{&X\ar@{-}[ld]_1\ar@{-}[rd]^1&\\Y\ar@{-}[rd]_1&&C\ar@{-}[ld]^1\\&Y\cap C\ar@{-}[dd]^{p+q-2}&\\&&\\&\{[0]\}&}
	\caption{when $p>1$}
    \end{subfigure}
    ~ %add desired spacing between images, e. g. ~, \quad, \qquad, \hfill etc. 
      %(or a blank line to force the subfigure onto a new line)
    \begin{subfigure}[t]{0.3\textwidth}
	    \xymatrixrowsep{0.5pc}
	    {\xymatrix{&X\ar@{-}[ld]_1\ar@{-}[rd]^1&\\Y\ar@{-}[rddd]_{p+q-2}&&C\ar@{-}[lddd]^{p+q-2}\\&&\\&&\\&\{[0]\}&}}
	\caption{when $p=1$}
    \end{subfigure}
\end{figure}
\end{theorem}
Now, for each closed subset $S$ of $P'\backslash G/P$ (except for $Y\cap C$ when $p>1$, which we omit) we construct the family of SBOs, to be denote by $R^S_{\lambda,\nu}$, such that:
\begin{itemize}
	\item $R_{\lambda,\nu}^S$ is defined for $(\lambda,\nu)\in D_S$, where $D_S$ is the subset of $\C^2$ (more precisely, it is either the whole $\C^2$, or is a countable
		union of one-dimensional complex affine spaces);
	\item $R_{\lambda,\nu}^S$ depends holomorphically on $(\lambda,\nu)\in D_S$;
	\item for every $(\lambda,\nu)\in D_S$ we have $\Supp(R_{\lambda,\nu}^S)\subset S$ and the equality is holding for generic $(\lambda,\nu)$.
\end{itemize}
\begin{remark}
	$\R_{\lambda,\nu}^S$ is a differential operator of $S=\left\{ 0 \right\}$. More general theory of differential SBOs appears in \cite[chap 2]{kobayashi2015differential1}.
\end{remark}
\begin{theorem}[construction of SBOs]
	\quad\\
\ExecuteMetaData[.master_extract.tex]{table}

\hspace*{-3.5cm}Here:
\begin{itemize}
	\item $\mid \mid \mid \assign \{ (\lambda, \nu) \in \mathbbm{C}^2 \mid \nu \in
	- 2\mathbbm{N} \cup (q + 1 + 2\mathbbm{Z}) \},\quad \backslash\backslash:=\mysetn{(\lambda,\nu)\in\C^2}{\lambda+\nu-n+1\in-2\N}$;
\item $/ / \assign
\{ (\lambda, \nu) \in \mathbbm{C}^2 \mid \lambda - \nu \in
-2\N \},\quad \mid\mid:=\mysetn{(\lambda,\nu)\in\C^2}{\nu\in1+2\N}$;
\item as defined in \cite[(16.2)]{kobayashi2015symmetry},
	\begin{gather*}
		a_j:=a_j\left( \frac{\nu-\lambda}{2};\lambda-\frac{n-1}{2} \right)\\
		a_j(l;\mu):=\frac{(-1)^j2^{2l-2j}}{j!(2l-2j)!}\prod_{i=1}^{l-j}\left( \mu+l+i-1 \right).
	\end{gather*}
	This formula for differential SBO was previously found in \cite{juhl2009families,kobayashi2015symmetry} for $q=0$ and in \cite{kobayashi2015branching} for general $p,q$.
\end{itemize}
Moreover,\\
\ExecuteMetaData[.master_extract.tex]{residue}
\end{theorem}
\begin{remark}
	Note that information in the rightmost column implies that for every $(\lambda,\nu)\in\C^2$ we have $R_{\lambda,\nu}^{ \left\{ 0 \right\}},R_{\lambda,\nu}^Y,R_{\lambda,\nu}^C\neq0$, while
	$R^X_{\lambda,\nu}=0$ iff $(\lambda,\nu)$ belongs to a discrete set
	\[\begin{cases}
			//\cap\mid\mid\mid,&p>1\\
			\mybra{//\cap\mid\mid\mid} \cup \mybra{\backslash\backslash\cap\mid\mid},&p=1,
		\end{cases}
	\]
	and $\tilde{R}_{\lambda,\nu}^X=0$ iff $p=1$ and $(\lambda,\nu)$ is in a discrete set $\backslash\backslash\cap \mid\mid$.
\end{remark}
\begin{theorem}[classification of SBOs]
		We can find basis for $\Hom_{G'}(I(\lambda),J(\nu))$ for every $(\lambda,\nu)\in \mathbb{C}^2$. More concretely,
		\ExecuteMetaData[.master_extract.tex]{classification}
\end{theorem}
\begin{corollary}
	We have $0<\dim_{\C}\Hom_{G'}(I(\lambda),J(\nu))\le2$ for all $(\lambda,\nu)\in\C^2$.
\end{corollary}
\begin{theorem}[spectrum for spherical vectors]
	Let $\mathbbm{1}_\lambda\in I(\lambda)^K,\mathbbm{1}_\nu\in J(\nu)^{K'}$ be the spherical vectors normalized so that $\mathbbm{1}_\lambda(e)=\mathbbm{1}_\nu(e)=1$. We then have
\[ \OpR^X_{\lambda, \nu} \mathbbm{1}_{\lambda} = 2^{1 -
\lambda} \frac{\pi^{n / 2}}{\Gamma \left( \frac{\lambda}{2} \right)
\Gamma \left(  \frac{\lambda + 1-q}{2} \right) \Gamma \left(
\frac{q - \nu + 1}{2} \right)} \mathbbm{1}_{\nu}. \]
\end{theorem}
\begin{remark}
	This formula was known in \cite{bernstein2004estimates,clerc2011generalized} for $p=q=1$ and in \cite{kobayashi2015symmetry} for $q=0$.
\end{remark}
\begin{theorem}[residue formula]
		The distribution
		\[K_{\lambda,\nu}^{\mathbb{R}^{p,q}}:=\frac{\myabs{x_p}^{\lambda+\nu-n}}{\Gamma\left( \frac{\lambda+\nu-n+1}{2} \right)}\times
		\frac{\myabs{Q}^{-\nu}}{\Gamma\left( \frac{1-\nu}{2} \right)}\]
		has the pole at $(\lambda,\nu)\in//$ with the residue given by
		\[\Res_{(\lambda,\nu)\in//}K_{\lambda,\nu}^{\mathbb{R}^{p,q}}=\frac{K_{\lambda,\nu}^{\mathbb{R}^{p,q}}}{\Gamma\left( \frac{\lambda-\nu}{2} \right)}
			=\frac{ (- 1)^k k!\pi^{(n - 2) / 2} 
		}{2^{ \nu + 2 k-1}}\cdot  \frac{\sin\left( \frac{1+q-\nu}{2}\pi \right)}{\Gamma\left( \frac{\nu}{2} \right)}
	\tilde{C}_{\nu - \lambda}^{\lambda - \frac{n
  	- 1}{2}} ({\Delta}_{\mathbb{R}^{p-1,q}} {\delta}_{\mathbb{R}^{p+q-1}}, \delta (x_p))
		\]
		where $k:=\frac{\nu-\lambda}{2}$.
		Hence taking $\Op(\cdot)$ on both sides we get
  \[\OpR_{\lambda,\nu}^X  = \frac{ (- 1)^k k!\pi^{(n - 2) / 2} 
		}{2^{ \nu + 2 k-1}}\cdot  \frac{\sin\left( \frac{1+q-\nu}{2}\pi \right)}{\Gamma\left( \frac{\nu}{2} \right)}
     \OpR_{\lambda,\nu}^{ \left\{ 0 \right\} },\quad(\lambda,\nu)\in// . \]
	\end{theorem}
	\begin{definition}
		Similarly to the construction of Fact \ref{fact1}, for $G=O(p+1,q+1)$ we have $\Hom_G(I(\lambda),I(\nu))\simeq\Sol_G(\R^{p,q};\lambda,\nu)$
		where the $\Sol_G(\R^{p,q};\lambda,\nu)\subset\mathcal{D}'(\R^{p,q})$ is defined to be the space of generalized functions on $\R^{p,q}$ that satisfy
		the four items in definition \ref{def2}, except that in second item $O(p,q)_{e_p}$ is replaced by $O(p,q)$ and the fourth item is replaced by $N_+$-invariance
		on $\R^{p,q}$, which in turn is defined as in definition \ref{def1}, with the only difference that we do not assume $b_p=0$ anymore.

		Now, the generalized function defined as
		\begin{equation*}
			\myabs{Q}^{\lambda-n}\times\begin{cases}
				\Gamma^{-1}\left( \lambda-n/2 \right),&\min\left\{ p,q \right\}=0\\
				\Gamma^{-1}\left( \frac{\lambda-n+1}{2} \right)\Gamma^{-1}\left( \lambda-n/2 \right),&\min\left\{ p,q \right\}>0,n\in2\Z+1\\
  \Gamma^{-1} \left( \frac{\lambda-n + 1}{2} \right) \Gamma ^{-1}\left( \frac{\lambda-n/2+
  1}{2} \right), &\min\left\{ p,q \right\}>0, n / 2 + p \in 2\mathbbm{Z}+ 1\\
  \Gamma^{-1} \left( \frac{\lambda-n + 1}{2} \right) \Gamma ^{-1}\left( \frac{\lambda-n/2}{2}
  \right), & \min\left\{ p,q \right\}>0,n / 2 + p \in 2\mathbbm{Z}
			\end{cases}
		\end{equation*}
		belongs to $\Sol_G(\R^{p,q},\lambda,n-\lambda)$ and we shall denote the corresponding member of $\Hom_{G}(I(\lambda),I(n-\lambda))$ by $\tilde{\mathbb{T}}_{\lambda}:I(\lambda)\to
		I(n-\lambda)$
		(\textit{Knapp-Stein operator}).
		The result of this construction repeated with $G'$ in place of $G$ will be denoted by $\tilde{\mathbb{T}}_\nu:J(\nu)\to J(n-1-\nu)$.
	\end{definition}
	\begin{theorem}[functional identities]
		Let $n':=n-1$. We then have:
\ExecuteMetaData[.master_extract.tex]{functional}
	\end{theorem}
%%	When considering degenerate principal series of $O(p,q)$ with $q=1$ we will use the notation $T(\cdot)$ for irreducible infinite-dimensional subquotient (or subrepresentation) of it
%%	as in \cite[p.19]{kobayashi2015symmetry}.
%%	Furthermore, let:\\
%%\ExecuteMetaData[.master_extract.tex]{Add}
%%Note that as a $K'$-module we have
%%\[\mybra{J(\nu)}_{K'}\simeq\sum_{(a,b)\in\mathfrak{I}}\mathcal{H}^a(\Sp^{p-1})\otimes \mathcal{H}^{b}(\Sp^q),\]
%%%Note that \cite[fact 3.14]{kobayashi2015symmetry} 
%%which allows us to consider $(\mathfrak{g}',K')$-submodules of $\mybra{J(\nu)}_{K'}$ as subsets of $\mathfrak{I}$. 
%%%%In turn,
%%%%due to the old result of Harish-Chandra (cf. \cite{howe1993homogeneous}) allows us to identify $(\mathfrak{g}',K')$-submodules of $\mybra{J(\nu)}_{K'}$ with $G'$-submodules of $J(\nu)$ (as Frechet 
%%%%representation).
%%Using this identification, we have:
%%	As in \cite{howe1993homogeneous}, we represent $K'$-types in $(\mathfrak{g}',K')$ module $J(\nu)_{K'}$ as points in $\R^2$ plane (with $K'$-type $\mathcal{H}^{x}\left( \Sp^{p-1} \right)
%%	\mathcal{H}^j\left( \Sp^q \right)$ represented by point $(i,j)$). With this identification we have
	Since the representation $J(\nu)$ of $G'=O(p,q+1)$ is multiplicity-free as a $K'$-module, we can describe its $(\mathfrak{g}',K')$-submodules by means of subsets of $\N_{+}$
	for $p>1$, which parametrize the $K'$-structure of $J(\nu)$ by the spherical harmonics $\mathcal{H}^a(\Sp^{p-1})\boxtimes\mathcal{H}^b(\Sp^q)$.
	As in \cite{howe1993homogeneous}, we also indicate the Jordan--Holder series (socle filtrations) of $J(\nu)$ by using arrows.
	We then have:
\begin{theorem}[images of SBOs]
	The (renormalized) regular SBO $R_{\lambda,\nu}^X:I(\lambda)\mapsto J(\nu)$ is surjective,
	unless $\nu\in\Z$. In the latter case, the images of the underlying $(\mathfrak{g},K)$ module $I(\lambda)_K$ under the
	regular symmetry breaking operator $R_{\lambda,\nu}^X$ are given as follows (
	here for $(\lambda,\nu)\in//$ we set $l:=\frac{\nu-\lambda}{2}$ and for $(\lambda,\nu)\in\backslash\backslash$ we set
	$k:=-\frac{\lambda+\nu-n+1}{2}$; the barriers $A^{\pm\pm}$ are defined as in \cite{howe1993homogeneous}): \\
%%	The tables below are to be read as follows:
%%	\begin{itemize}
%%		\item The leftmost column shows the assumptions on $\nu$ (the composition series of $J(\nu)_{K'}$ are determined by it);
%%		\item every cell of table shows the corresponding composition series of $J(\nu)_{K'}$ (in the bottom-to-top fashion) and the items colored in blue are precisely those
%%			that are included into the image of corresponding $R_{\lambda,\nu}^X$;
%%	\end{itemize}
	for $p>1$:
	\begin{enumerate}
	\item Suppose $p\in2\N_++1$ and $q\in2\Z$. Then, if $\nu\in2\Z,0<\nu<n-1$, $R_{\lambda,\nu}^X$ is onto. Otherwise,
	  \begin{figure}[H]
	    \hskip-3.6cm\noindent\begin{tabular}{m{3.5cm}ccc}
	      $(\lambda,\nu)\in$&$\mybra{//\cup\backslash\backslash}^c$ & $\backslash\backslash-//$  & $//\cap\backslash\backslash,k> l$\\[15pt]
	      {\vspace{-3cm} $ \even\ni\nu\leq0$}:&\input{|"guile  -e mp $HOME/for/forscheme/ma.scm '((App 0 1 0))' '((App 0 1 0))' '((App 0 3 0 K0)(App 0 1 0 Kt))'"}\\[15pt]
	      \vspace{-3cm}$\odd\ni\nu\leq\frac{n-3}{2}$:&\input{|"guile -e mp $HOME/for/forscheme/ma.scm '((Apm 0 1 0)(Amp 0 2 0))' '((Apm 0 1 0)(Amp 0 0 0))' '((Apm 0 11 0 Kt)(Amp 0 3 0 K0))'"}\\[25pt]
	      $(\lambda,\nu)\in$&$\mybra{//\cup\backslash\backslash}^c$ && $//\cap\backslash\backslash,k=l$\\[15pt]
	      \vspace{-3cm}$\odd\ni\nu=\frac{n-1}{2}:$&\input{|"guile  -e mp $HOME/for/forscheme/ma.scm '((Apm 0 11 0)(Apm 1 0 0))' '()' '((Apm 0 3 0 K0)(Apm 1 11 0 Kt))'"}\\[25pt]
	      $(\lambda,\nu)\in$&$\mybra{//\cup\backslash\backslash}^c$ & $//-\backslash\backslash$  & $//\cap\backslash\backslash,k< l$\\[15pt]
	      \vspace{-3cm}$\even\ni\nu\ge n-1$:&\input{|"guile  -e mp $HOME/for/forscheme/ma.scm '((App 1 3 1))' '((App 1 3 1))' '((App 1 2 1))'"}\\[25pt]
	      \vspace{-3cm}$\odd\ni\nu\ge\frac{n+1}{2}$:&\input{|"guile -e mp $HOME/for/forscheme/ma.scm '((Apm 1 0 1)(Amp 1 2 1))' '((Apm 1 3 1 K0)(Amp 1 2 1 Kt))' '((Apm 1 2 1 K0)(Amp 1 2 1 K0)(Amp 1 2 1 Kt))'"}\\[25pt]
	    \end{tabular}
	  \end{figure}
	\item Suppose $p,q\in\odd$ and $p>1$. Then,
		\begin{figure}[H]
			\hskip-3.6cm\noindent\begin{tabular}{m{3.5cm}ccc}
			$(\lambda,\nu)\in$&$\mybra{\ss\cup\bb}^c$ & $\bb-\ss$  & $\ss-\bb$\\[15pt]
			\vspace{-3cm}$\odd\ni\nu\le0$:&\input{|"guile -e mp $HOME/for/forscheme/ma.scm '((App 0 1 0)(Apm 0 0 0))' '((App 0 1 0)(Apm 0 0 0))' '((App 0 1 0 Kt)(Apm 0 3 0 K0))'"}\\[15pt]
			\vspace{-3cm}$\odd\ni\nu\le n-3$:&\input{|"guile -e mp $HOME/for/forscheme/ma.scm '((Amp 0 3 0))' '((Amp 0 11 0))' '((Amp 0 3 0))'"}\\[15pt]
			\vspace{-3cm}$\even\ni\nu>0$:&\input{|"guile -e mp $HOME/for/forscheme/ma.scm '((Apm 0 11 0))' '((Apm 0 11 0))' '((Apm 0 11 0 Kt)(Apm 0 3 0 K0))'"}\\[15pt]
			\vspace{-3cm}$\odd\ni\nu>n-3$:&\input{|"guile -e mp $HOME/for/forscheme/ma.scm '((App 1 3 1)(Apm 1 0 1))' '((App 1 0 1)(Apm 1 2 1))' '((App 1 3 1)(Apm 1 0 1))'"}\\[15pt]
			  
		\end{tabular}
		\end{figure}
	\item Suppose $p,q\in\even$. Then,
		\begin{figure}[H]
			\hskip-3.6cm\noindent\begin{tabular}{m{3.5cm}ccc}
			$(\lambda,\nu)\in$&$\mybra{\ss\cup\bb}^c$ & $\bb-\ss$  & $\ss-\bb$\\[15pt]
			\vspace{-3cm}$\odd\ni\nu\le0$:&\input{|"guile -e mp $HOME/for/forscheme/ma.scm '((App 0 1 0)(Amp 0 0 0))' '((App 0 1 0)(Amp 0 0 0))' '((App 0 1 0 Kt)(Amp 0 3 0 K0))'"}\\[15pt]
			\vspace{-3cm}$\odd\ni\nu\le n-3$:&\input{|"guile -e mp $HOME/for/forscheme/ma.scm '((Apm 0 3 0))' '((Apm 0 11 0))' '((Apm 0 3 0 K0)(Apm 0 11 0 Kt))'"}\\[15pt]
			\vspace{-3cm}$\even\ni\nu>0$:&\input{|"guile -e mp $HOME/for/forscheme/ma.scm '((Amp 0 11 0))' '((Amp 0 11 0))' '((Amp 0 3 0))'"}\\[15pt]
			\vspace{-3cm}$\odd\ni\nu>n-3:$&\input{|"guile -e mp $HOME/for/forscheme/ma.scm '((App 1 0 1)(Amp 1 2 1))' '((App 1 0 1)(Amp 1 2 1))' '((App 1 3 1 K0)(Amp 1 2 1 Kt))'"}\\[15pt]
		\end{tabular}
		\end{figure}
	\item Suppose $p\in\even,q\in\odd$. Then for $\nu\in\odd$ we have $R_{\lambda,\nu}^X$ being onto. Otherwise (for $\nu\in\even$) we have,
	  \begin{figure}[H]
	    \hskip-3.6cm\noindent\begin{tabular}{m{3.5cm}ccc}
	      $(\lambda,\nu)\in$&$\mybra{//\cup\backslash\backslash}^c$ & $\backslash\backslash-//$  & $//\cap\backslash\backslash,k> l$\\[15pt]
	      \vspace{-3cm}$\nu\leq0$:&\input{|"guile -e mp $HOME/for/forscheme/ma.scm '((App 0 1 0)(Apm 0 0 0)(Amp 0 0 0))' '((App 0 1 0)(Apm 0 0 0)(Amp 0 0 0))' '((App 0 1 0 Kt)(Apm 0 3 0 K0)(Amp 0 0 0))'"}\\[15pt]
	      \vspace{-3cm}$0<\nu\leq\frac{n-3}{2}$:&\input{|"guile -e mp $HOME/for/forscheme/ma.scm '((Apm 0 1 0)(Amp 0 2 0))' '((Apm 0 1 0)(Amp 0 0 0))' '((Apm 0 11 0 Kt)(Amp 0 3 0 K0))'"}\\[25pt]
              $(\lambda,\nu)\in$&:$\mybra{//\cup\backslash\backslash}^c$ && $//\cap\backslash\backslash,k=l$\\[15pt]
	      \vspace{-3cm}$\odd\ni\nu=\frac{n-1}{2}:$&\input{|"guile  -e mp $HOME/for/forscheme/ma.scm '((Apm 0 11 0)(Apm 1 0 0))' '()' '((Apm 0 11 0 Kt)(Apm 1 3 0 K0))'"}\\[25pt]
	      $(\lambda,\nu)\in$&$\mybra{//\cup\backslash\backslash}^c$ & $//-\backslash\backslash$  & $//\cap\backslash\backslash,k< l$\\[15pt]
	      \vspace{-3cm}$\frac{n+1}{2}\le\nu\le n-3$:&\input{|"guile -e mp $HOME/for/forscheme/ma.scm '((Apm 1 0 1)(Amp 1 2 1))' '((Apm 1 3 1 K0)(Amp 1 2 1 Kt))' '((Apm 1 2 1 K0)(Amp 1 2 1 Kt)(Amp 1 2 1 K0))'"}\\[25pt]
	      \vspace{-3cm}$\nu>n-3$:&\input{|"guile -e mp $HOME/for/forscheme/ma.scm '((App 1 0 1)(Apm 1 0 1)(Amp 1 2 1))' '((App 1 0 1)(Apm 1 3 1 K0)(Amp 1 2 1 Kt))' '((App 1 0 1)(Apm 1 2 1 K0)(Amp 1 2 1 Kt)(Amp 1 2 1 K0))'"}\\[15pt]
	    \end{tabular}
	  \end{figure}
	\end{enumerate}
	In the diagrams above some of them are filled not with gray, but with colored diagonal lines. This means that the image of the regular
	SBO $R_{\lambda,\nu}^X$ is zero and the (green/purple)
	ascending/descending diagonal lines show the images of its residues $R_{\lambda,\nu}^{ \left\{ 0 \right\}}$ and $\tilde{R}_{\lambda,\nu}^X$ respectively.

	For $p=1$ we have:\\
	\begin{figure}[H]
		\hskip-3.6cm\begin{tabular}{p{4.5cm}p{2.5cm}p{2.5cm}p{2.5cm}p{2.5cm}p{2.5cm}}
		$(\lambda,\nu)\in$ & $\mybra{\ss\cup\bb}^c$ & $\ss-\bb$ & $\bb-\ss$ & $\ss\cap\bb,k<l$ & $\ss\cap\bb,k\geq l$\\
		\vspace{-0.7cm}$\even\ni\nu\le0$:&\input{|"guile -e mp1 $HOME/for/forscheme/ma.scm 		'((def 0 1))' '((Kt 0 1)(K0 0 2))' '((def 0 1))' '()' '((K0 0 2)(Kt 0 1))'"}\\
		\vspace{-0.5cm}$\nu,q\in2\Z,0<\nu<q$:&\input{|"guile -e mp1 $HOME/for/forscheme/ma.scm 		'((def 1))' '((def 1))' '((def 1))' '((def 1))' '((def 1))'"}\\
		\vspace{-0.5cm}$\nu\in2\Z,q\in2\Z+1,0<\nu<q$:&\input{|"guile -e mp1 $HOME/for/forscheme/ma.scm 	'((def 1))' '((Kt 1)(K0 1))' '((def 1))' '((def 1))' '((def 1))'"}\\
		\vspace{-0.7cm}$q\in\even,\even\ni\nu\ge q$:&\input{|"guile -e mp1 $HOME/for/forscheme/ma.scm 	'((def 1 2))' '((def 1 2))' '((def 1 1))' '((def 1 1))' '()'"}\\
		\vspace{-0.7cm}$q\in\odd,\even\ni\nu\ge q$:&\input{|"guile -e mp1 $HOME/for/forscheme/ma.scm 	'((def 1 1))' '((Kt 1 1)(K0 1 2))' '((def 1 1))' '((def 1 1))' '()'"}\\
		\vspace{-0.7cm}$q\in\even,\odd\ni\nu\le0$:&\input{|"guile -e mp1 $HOME/for/forscheme/ma.scm 	'((def 0 2))' '((Kt 0 2)(K0 0 2))' '((def 0 2))' '()' '((K0 0 2)(Kt 0 2))'"}\\
		\vspace{-0.7cm}$q\in\odd,\odd\ni\nu\le0$:&\input{|"guile -e mp1 $HOME/for/forscheme/ma.scm 	'((def 0 2))' '((def 0 2))' '((def 0 2))' '()' '((def 0 2))'"}\\
		\vspace{-0.5cm}$q\in\even,\nu\in\odd,0<\nu<q$:&\input{|"guile -e mp1 $HOME/for/forscheme/ma.scm	'((def 1))' '((Kt 1)(K0 1))' '((KC 1)(KY 1))' '((Kt 1)(K0 1))' '((def 1))'"}\\
		\vspace{-0.5cm}$q,\nu\in\odd,0<\nu<q$:&\input{|"guile -e mp1 $HOME/for/forscheme/ma.scm 	'((def 1))' '((def 1))' '((KC 1)(KY 1))' '((Kt 1)(K0 1))' '((def 1))'"}\\
		\vspace{-0.7cm}$q\in\even,\odd\ni\nu\ge q$:&\input{|"guile -e mp1 $HOME/for/forscheme/ma.scm 	'((def 1 1))' '((Kt 1 1)(K0 1 2))' '((KY 1 1)(KC 1 1))' '((Kt 1 1)(K0 1 1))' '()'"}\\
		\vspace{-0.7cm}$q\in\odd,\odd\ni\nu\ge q$:&\input{|"guile -e mp1 $HOME/for/forscheme/ma.scm 	'((def 1 2))' '((def 1 2))' '((KY 1 1)(KC 1 2))' '((K0 1 2)(Kt 1 2))' '()'"}\\
	\end{tabular}\end{figure}
	In the diagrams above some of them are filled not with gray, but with colored diagonal lines. This means that the image of the regular SBO $R_{\lambda,\nu}^X$ is zero and:
	\begin{itemize}
		\item For $(\lambda,\nu)\in\ss$ the (green/purple)
			ascending/descending diagonal lines show the images of its residues $R_{\lambda,\nu}^{ \left\{ 0 \right\}}$ and $\tilde{R}_{\lambda,\nu}^X$ 
			respectively.
		\item For $(\lambda,\nu)\in\ss$ the (blue/red) ascending/descending diagonal lines show the images of its residues $R_{\lambda,\nu}^{Y}$ and ${R}_{\lambda,\nu}^C$ 
			respectively.
	\end{itemize}
	
	
\end{theorem}
\begin{remark}
	Although we omit the corresponding diagrams for brevity, the images of the other SBO constructed in Theorem 2, we can find their images as well. Note that
	the proof of this theorem was done \textit{independent of} the information of \cite{howe1993homogeneous}.
\end{remark}
Now, we recall from \cite{KO2} the five equivalent definitions of the
irreducible representations $\pi_{\pm,\lambda}^{p,q}$ of $O(p,q)$. Comparing the definitions, one sees that as 
representations of $O(p+1,q+1)$ we have for $p>0$:
\begin{equation*}
	\lambda+q-1\in2\Z\implies\pi_{+,\lambda+\frac{n}{2}}^{p+1,q+1}\simeq A^1\left(\lambda  \right),\quad\lambda-p+1\in2\Z\implies\pi_{-,\lambda+\frac{n}{2} }^{p+1,q+1}\simeq A^2\left(\lambda\right)
\end{equation*}
and for $q>0$ (which is our constant assumption in this work) we have $T(\nu-q)=\pi_{-,\nu-q/2}^{1,q+1}$.

\begin{theorem}[$G'$-invariant maps between Zuckerman modules $\pi_{\pm,\lambda}^{p,q}$]
	The dimensions of $\Hom_{G'}\left(\pi_{\pm,{n}/{2}-\lambda}^{p+1,q+1} ,\pi_{\pm,\nu-{n'}/{2}}^{p,q+1} \right)$
	are as follows:\newline
\ExecuteMetaData[.master_extract.tex]{Aq}
\end{theorem}
\bibliography{todai_master}
\bibliographystyle{alpha}
\end{document}
