\documentclass[8pt]{article} % use larger type; default would be 10pt

%\usepackage[utf8]{inputenc} % set input encoding (not needed with XeLaTeX)
\usepackage[10pt]{type1ec}          % use only 10pt fonts
\usepackage[T1]{fontenc}
%\usepackage{CJK}
\usepackage{graphicx}
\usepackage{float}
\usepackage{CJKutf8}
\usepackage{subfig}
\usepackage{amsmath}
\usepackage{amssymb}
\usepackage{amsthm}
\usepackage{amsfonts}
\usepackage{hyperref}
\usepackage{enumerate}
\usepackage{enumitem}

%for Re and Im like in the book
\renewcommand\Re{\operatorname{Re}}
\renewcommand\Im{\operatorname{Im}}

%custom commands to save typing
\newcommand{\mynorm}[1]{\left|\left|#1\right|\right|}
\newcommand{\myabs}[1]{\left|#1\right|}
\newcommand{\myset}[1]{\left\{#1\right\}}

%put subscript under lim and others
\let\oldlim\lim
\renewcommand{\lim}{\displaystyle\oldlim}
\let\oldmin\min
\renewcommand{\min}{\displaystyle\oldmin}
\let\oldmax\max
\renewcommand{\max}{\displaystyle\oldmax}

\newtheorem*{prob}{Question}

\title{Problem Sets}
\author{Alex Leontiev}
\begin{document}
\maketitle
\begin{prob}\end{prob}
	The most obvious examples I can think of are $\mathbb{R}^n$ (these are not compact). Somehow less trivial ones are $\mathbb{S}^n:=
	\{x\in\mathbb{R}^{n+1}\mid \sum_{i=1}^{n+1}x_i^2=1\}$ (these are compact).
\begin{prob}\end{prob}%TODO
	Recall that for $A\in M(2,\mathbb{R}$ we define $e^A$ as
	\[e^A:=I+\frac{A}{1!}+\frac{A^2}{2!}+\frac{A^3}{3!}+\dots\]
	and moreover for any $A\in M(2,\mathbb{R}$ we define $D(e^A)\in Hom(T_A X,T_{e^A}X)$ - the derivative and say that $e^x$ is singular
	at $x\in X$ if the linear transformation $D(e^x)$ is not onto. Now, as $X\simeq\mathbb{R}^4$ is flat, $T_A X=T_{e^A}X=X$, and for 
	$H\in T_A X$ we define
	\[D(e^A)(H):=\lim_{\lambda\to 0}\frac{e^{A+\lambda H}-e^A}{\lambda}\]
	Now, if $A,\;H\in X$ commute, we have $e^{A+H}=e^Ae^H$ and we can write
	\[D(e^A)(H):=\lim_{\lambda\to 0}e^A\frac{e^{\lambda H}-I}{\lambda}=e^A\lim_{\lambda\to 0}\frac{\frac{\lambda
	B}{1!}+\frac{\lambda^2B^2}{2!}+\dots}{\lambda}=e^A B\]
	Now, if we take
	\[x_0:=\begin{bmatrix}1&0\\0&0\end{bmatrix}\]
	\[H:=\begin{bmatrix}0&0\\0&1\end{bmatrix}\neq 0\]
		They clearly commute (as both are diagonal) and according to previous computations $D(e^{x_0})(H)=x\cdot H=0$. Therefore,
		operator $D(e^{x_0})$ on $\mathbb{R}^4$ has non-trivial kernel and therefore cannot be onto, thus yielding singularity.

	\textit{I hope, I've interpreted "singular" correctly.
	}
\begin{prob}\end{prob}%FAIL
	\textit{I cannot do this problem, for I do not know the definition of the Lorentz and Riemannian metric. I've heard about the latter
	one on my last semester "Introduction to Differentiable Manifolds" class, but we did not have sufficient time to cover the topic at that
	time.}
\begin{prob}\end{prob}
	Clearly, this is not true, the function $f(x):=\frac{1}{1+x^2}$ being the simple counterexample. On the one, hand, it is real analytic
	on the whole $\mathbb{R}$, for it is rational and denominator does not turn to zero. On the other hand, for any $x_0\in\mathbb{R}$ expansion
	$\sum_{i=0}^{\infty}c_n(x-x_0)^n$ of $f(x)$ around $x_0$ cannot have radius of the convergence infinity, for this would imply that
	function $\frac{1}{1+x^2}$ is analytic on the whole $\mathbb{C}$, when seen as complex-valued and this is clearly not so, for it
	has pole at $x=\pm i$.
\begin{prob}\end{prob}
	Almost (in measure-theoretic sense) the whole torus can be parametrised via the map
	\[P:(0,2\pi)^2\ni(\phi,\theta)\mapsto (a\cos\phi+b\cos\theta\cos\phi
	,a\sin\phi+{b}\cos\theta\sin\phi,b\sin\theta)\in\mathbb{R}^3,\;a>b>0\]
	Then, using this parametrization we can compute the surface area via the integral
	\[S=\int_0^{2\pi}\int_0^{2\pi}\mynorm{\frac{\partial P}{\partial \phi}\times\frac{\partial P}{\partial \theta}}d\phi d\theta\]
	this should give us $S=4\pi^2ab$.

	\textit{I did not leave sufficient time to work on this (computational) problem.}
\begin{prob}\end{prob}
	Hausdorff means that any two distinct points can be separated by disjoint neighborhoods and no, $Z$ is not Hausdorff for $[(1,0)]\neq
	[(0,1)]$ yet they cannot be separated by disjoint neighborhoods in $Z$. Let us argue by contradiction and assume that $[(1,0)]\in U_1,\;
	[(0,1)]\in U_2$, $U_1$ and $U_2$ are disjoint open sets in $Z$. Then, if we denote $\pi:\mathbb{R}^2\setminus\{(0,0)\}\mapsto Z$
	the projection map, $V_1:=\pi^{-1}(U_1)$ and $V_2:=\pi^{-1}(U_2)$
	should be open in $\mathbb{R}^2\setminus\{(0,0)\}$. Moreover, $V_1$ and $V_2$ should be disjoint for if $x\in V_1\cap V_2$ we would
	have $\pi(x)\in U_1\cap U_2$.

	As $(1,0)\in V_1$ if we take $x_n\to 0+$ we would have $(1,x_n)\to (1,0)$, so for big $n$ $(1,x_n)\in V_1$. Then, (by dropping
	finitely many terms, if necessary) we might assume $\forall n\in\mathbb{N},\;(1,x_n)\in V_1$. Now, as $(1,x_n)\sim (x_n,1)$ ($a=x_n>0$)
	we have $V_1\ni (x_n,1)\to(0,1)$, hence for big $n$ $(x_n,1)\in V_2$, thus contradicting disjointness.
\begin{prob}\end{prob}
To begin with, it is not specified explicitly on which domain we are to prove meromorphness.

Second, we may prove even analyticity when restricted to the half-plane $\{\Re z>-1\}$. Indeed, as $f(x)$ is supported on a compact domain,
we have for some $A\geq 0$ $\int_0^{\infty}f(x)x^{\lambda}dx=\int_0^Af(x)x^{\lambda}dx$ and $f(A)=0$. Hence, $P(\lambda)$ is defined
on a half-plane (see previous paragraph) and then, we can interchange derivative and integral (since $\frac{t^{\lambda+z}-t^{\lambda}}{z}\to\ln t 
\cdot t^{\lambda}$ as $z\to 0+$ uniformly on $[0,A]$) and see that $P(\lambda)$ is differentiable on the half-plane discussed.
Now, assuming that $\lim_{\lambda\to -1}(\lambda+1)P(\lambda)$ exists, it should be equal to $\lim_{\lambda\to -1+}(\lambda+1)P(\lambda)$ and then
\[\lim_{\lambda\to-1+}(\lambda+1)P(\lambda)=\lim_{\lambda\to-1+}\left(x^{\lambda+1}f(x)\mid_0^A-\int_0^A f'x^{\lambda+1}dx\right)\]
if $\lambda>-1$ we have $0^{\lambda+1}=0$ and then
\[\lim_{\lambda\to-1+}\left(x^{\lambda+1}f(x)\mid_0^A-\int_0^A f'x^{\lambda+1}dx\right)=\lim_{\lambda\to-1+}-\int_0^Af'x^{\lambda+1}dx\]
Finally, as $x^{\lambda+1}\to 1$ as $\lambda\to-1+$ uniformly on $[0,A]$, we have  
\[\lim_{\lambda\to-1+}-\int_0^Af'x^{\lambda+1}dx=f(0)-f(A)=f(0)\implies\lim_{\lambda\to-1+}(\lambda+1)P(\lambda)=f(0)\]
Similarly, under the appropriate assumptions on how fast $f(0+)$ tends to zero it can be shown that $\lim_{\lambda\to-k}(\lambda+k)P(\lambda)$
is proportional to $f^{(k)}(0)$.
\textit{Clearly, I did not get exactly on which domain we have to prove meromorphness. Probably, meromorphness can be established
on the whole $\mathbb{C}$ (with possible poles in negative integers), but I do not know about the sufficient conditions to prove the possibility
of meromorphic extension.}
\end{document}
