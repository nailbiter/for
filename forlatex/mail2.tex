
\documentclass[10pt]{article} % use larger type; default would be 10pt

%%\usepackage[T1,T2A]{fontenc}
%%\usepackage[utf8]{inputenc}
%%\usepackage[english,ukrainian]{babel} % може бути декілька мов; остання з переліку діє по замовчуванню. 
\usepackage{enumerate}
\usepackage{CJKutf8}
\usepackage{mystyle}

%%\usepackage{fancyhdr}
%%\pagestyle{fancy}
%%\fancyfoot[C]{text me at \href{mailto:leontiev@ms.u-tokyo.ac.jp}{leontiev@ms.u-tokyo.ac.jp} if there are mistakes/obscurities}
%%\fancyhead{}

\title{}
\author{Alex Leontiev}
\begin{document}
\maketitle
Consider the action of $\R$ on $\Sp^1$ defined as follows. We let $\psi\ni\R\ni x\mapsto\left( \frac{1-x^2}{1+x^2},\frac{2x}{1+x^2} 
\right)\in\Sp^1$
be a conformal compactification. Having $h\in\R$ fixed, we let $\varphi_h(\psi(x)):=\psi(x+h)$ for $\psi(x)\in\psi(\R)
$ and let $\varphi_h(-1,0)=(-1,0)$ (so that $f_h$ fixes the sole point of $\Sp^1$ that is not in the image of $\psi$).
Such $f_h:\Sp^1\to\Sp^1$ gives an action $\R\curvearrowright\Sp^1$.

Now, the space of functions on $\Sp^1$ that are invariant w.r.t. this action has dimension 2. If we replace the action by
its differential answer will be the same.

Nevertheless, if we restrict ourselves to $U:=\Sp^1\setminus\left\{ (1,0) \right\}$, then the dimension of invariant functions
is still 2. However, if now we replace the action by its differential, the dimension will become 3.
%%\begin{thebibliography}{9}
%%\bibitem{gelbaum}Gelbaum, B.R. and Olmsted, J.M.H.. Counterexamples in Analysis. Dover Publications. 2003
%%\end{thebibliography}
\end{document}
