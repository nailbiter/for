\documentclass[12pt]{article} % use larger type; default would be 10pt

\usepackage{mathtext}                 % підключення кирилиці у математичних формулах
                                          % (mathtext.sty входить в пакет t2).
\usepackage[T1,T2A]{fontenc}         % внутрішнє кодування шрифтів (може бути декілька);
                                          % вказане останнім діє по замовчуванню;
                                          % кириличне має співпадати з заданим в ukrhyph.tex.
\usepackage[utf8]{inputenc}       % кодування документа; замість cp866nav
                                          % може бути cp1251, koi8-u, macukr, iso88595, utf8.
\usepackage[english,russian,ukrainian]{babel} % національна локалізація; може бути декілька
                                          % мов; остання з переліку діє по замовчуванню. 
\usepackage{empheq}
\usepackage{mystyle}
\usepackage{comment}

\newtheorem{prob}{Завдання}
\newcommand{\ds}{\;ds}
\newcommand{\dt}{\;dt}
\newcommand{\dx}{\;dx}
\newcommand{\dy}{\;dy}
\newcommand{\dta}{\;d\tau}
\newcommand{\extr}{\mbox{\normalfont extr}}
\newcommand{\Var}{\mbox{Var}}

\newtheorem{myulem}[mythm]{Лема}

\renewenvironment{myproof}[1][Доведення]{\begin{trivlist}
\item[\hskip \labelsep {\bfseries #1}]}{\myqed\end{trivlist}}

\title{Математична статистика (10 семестр)}
\author{}

\begin{document}
\maketitle
\section{Глава 1}%17,18
\setcounter{prob}{15}
\begin{prob}\end{prob}
Як відомий загальний факт про рівномірний розподіл, маємо
\[\mathbb{E}\hat{\theta}_1=\theta-h_0+\frac{2h_0}{n+1}\to\theta-h_0\]
\[\mathbb{E}\hat{\theta}_2=\theta-h_0+\frac{2nh_0}{n+1}\to\theta+h_0\]
Відповідно, оскільки
\[\mathbb{E}\hat{\theta}_3=\mathbb{E}\xi_1=\theta\]
Таким чином, $\hat{\theta}_3$ є незміщеною оцінкою параметра $\theta$. Вона також є конзистентною, що випливає з нерівності Чебишева, адже
$\Var\hat{\theta}_3=\frac{1}{n}\Var\xi_1\to0$. Оцінки $\hat{\theta}_1$ та $\hat{\theta}_2$ не можуть бути незміщеними, адже їх середні значення
змінюються при кожному $n$. $\hat{\theta}_4$. Залишається дослідити $\hat{\theta}_4$. Її розподіл можна записати як
\[\hat{\theta}_4=\hat{\theta}_2-h_0+U(\hat{\theta}_1-\hat{\theta}_2+2h_0)\]
де $U$ має рівномірний на $[0,1]$ розподіл (незалежний від $\xi_i$). Таким чином,
\[\mathbb{E}\hat{\theta}_4=\theta-h_0+\frac{2nh_0}{n+1}-h_0+\frac{1}{2}\mycbra{\frac{2h_0}{n+1}-\frac{2nh_0}{n+1}+2h_0}=\theta\]
і $\hat{\theta}_4$ є незміщеною оцінкою $\theta$. Вона є конзистентною, адже оскільки $\hat{\theta}_1+h_0\to\theta$ і $\hat{\theta}_2-h_0\to\theta$,
при великих $n$ усі елементи інтервалу $[\hat{\theta}_2-h_0;\hat{\theta}_1+h_0]$ будуть близько від $\theta$.
\begin{prob}\end{prob}
	Дійсно,\[m_1(\hat{a},\hat{\theta}^2)=\hat{a}\]\[m_2(\hat{a},\hat{\theta}^2)=\hat{a}^2+\hat{\theta}^2\]
	Таким чином,
	\[\hat{a}=\frac{1}{n}\sum\xi_i\]
	\[\hat{\theta}^2=\frac{1}{n}\sum\xi_i^2-\mybra{\frac{1}{n}\sum\xi_i}^2\]
	є оцінками. Перша з них є незміщеною оцінкою $a$, друга не є незміщеною, адже для $n=1$, $\hat{\theta}^2=0$. $\hat{a}$ також є конзистентною,
	адже $\Var\hat{a}=\frac{1}{n}\Var\xi_1\to0$. Нижче ми покажемо, що $\hat{\theta}^2$ також є конзистентною. За теоремою Слуцького,
	$\hat{m}_1^2$ є конзистентною оцінкою $\theta^2$. Оскільки лінійна комбінація конзистентних оцінок є конзистентною, достатньо показати
	конзистентність $\hat{m}_2:=1/n\sum\xi_i^2$ як оцінки $\theta^2+\sigma^2$. Остання є незміщеною, як видно із розрахунків. До того ж,
	\[\Var\hat{m}_2=\frac{1}{n}\Var\xi_1\to0\]
	і конзистентність слідує з нерівності Чебишева.
\section{Глава 2}
\setcounter{prob}{21}
\begin{prob}\end{prob}
	В нашому випадку, розподіл величини $\xi$ тривалості роботи має вигляд $Ae^{-at}$, а нам потрібно оцінити величину $\min\mycbra{\xi,100}$.
	Оскільки
	\[G=\int_{\mathbb{R}^+}\min\mycbra{t,100}Ae^{-at}\;dt=\underbrace{\int_0^{100}tAe^{-at}\;dt}_{<\infty}+\underbrace{\int_{100}^\infty
	100Ae^{-at}\;dt}_{<\infty}<\infty\]
	То, незміщеною і конзистентною оцінкою шуканої величини буде
	\[\hat{G}_n=\frac{1}{n}\sum_{i=1}^n\min(\xi_i,100)=75.05\]
\begin{prob}\end{prob}
	Оскільки треба оцінити два параметри, нам знадобляться два перші моменти
	\[m_1(k;p,m)=\int_{\theta-\sigma\sqrt{3}}^{\theta+\sigma\sqrt{3}}\frac{t}{2\sigma\sqrt{3}}dt=\frac{2\theta\sigma\sqrt{3}}{2\sigma\sqrt{3}}=
	\theta\]
	\[m_2(k;p,m)=\int_{\theta-\sigma\sqrt{3}}^{\theta+\sigma\sqrt{3}}\frac{t^2}{2\sigma\sqrt{3}}dt=\theta^2+\sigma^2\]
	і розв’язуючи цю систему рівнянь, маємо
	\[\hat{{\theta}}=\hat{m_1}\]
	\[\hat{\sigma}^2=(\hat{m_2}-\hat{m_1}^2)\]
	де
	\[\hat{m_1}:=\frac{1}{n}\sum_{i=1}^n\xi_i\]
	\[\hat{m_2}:=\frac{1}{n}\sum_{i=1}^n\xi^2_i\]
	Через незалежність членів виборки, $\mathbb{E}\hat{\theta}=\mathbb{E}\hat{m_1}=\mathbb{E}\xi_1=\theta$, тому ця оцінка є незміщеною. Вона
	також є конзистентною, адже $\Var\hat{\theta}=\frac{1}{n}\Var\xi_1\to0$ при $n\to\infty$ і конзистентність слідує із нерівності Чебишева.

	Оцінка $\hat{\theta}^2$ не є незміщеною, адже вона тотожно рівна нулю при $n=1$, проте вона є конзистентною. За теоремою Слуцького,
	$\hat{m}_1^2$ є конзистентною оцінкою $\theta^2$. Оскільки лінійна комбінація конзистентних оцінок є конзистентною, достатньо показати
	конзистентність $\hat{m}_2$ як оцінки $\theta^2+\sigma^2$. З огляду на нерівність Чебишева, для цього, в свою чергу, достатньо показати,
	що дисперсія прямує до нуля і незміщеність. Почнемо з останнього. $\mathbb{E}\hat{m}_2=\mathbb{E}\xi_1^2=\theta^2+\sigma^2$.
	Залишається оцінити дисперсію. 
	\[\Var(\hat{m}_2)=\frac{1}{n}\Var\xi_1\to0\]
	\begin{prob}\end{prob}
	При фіксованих $\xi_i$ функція максимальної правдоподібності має вигляд
	\[L(\theta)=I_{\theta>0}\frac{\prod{\xi_i}}{\theta^n}\exp\mycbra{-\frac{\sum\xi_i^2}{2\theta}}\]
	і таким чином, оскільки ми шукаємо найбільше значення, можна автоматично вважати, що $\theta>0$ і нам,
	таким чином, потрібно мінімізувати
	\[\frac{1}{\theta^n}\exp\mycbra{-\frac{\sum\xi_i^2}{2\theta}}\]
	за умови $\theta>0$. Прирівнюючи до нуля похідну, отримуєм
	\[0=\frac{d}{d\theta}\mybra{\frac{1}{\theta^n}\exp\mycbra{-\frac{\sum\xi_i^2}{2\theta}}}=-n\theta^{-n-1}\exp\mycbra{\dots}+\theta^{-n}
	\frac{\sum\xi_i^2}{2\theta^2}\exp\mycbra{\dots}\implies\]
	\[\implies\theta=\frac{\sum\xi_i^2}{2n}=:\hat{\theta}\]
	Ця оцінка є незміщеною, оскільки
	\[\mathbb{E}\hat{\theta}=\frac{1}{2}2\theta=\theta\]
	Відповідно, за нерівністю Чебишева, конзистентність (яку ми хочемо довести наступною) випливатиме із того, що дисперсія збіжна до нуля.
	Справді,
	\[\Var\hat{\theta}=\frac{1}{4n}\Var\xi_1^2\to0\]
	Наприкінець, покажемо ефективність оцінки за нерівністю Крамера-Рао. Перевіримо умови регулярності
	\begin{enumerate}
		\item \[\frac{\partial}{\partial\theta}\mybra{\frac{x}{\theta}\exp\mycbra{-\frac{x^2}{\theta}}}=
			\mybra{-\frac{x}{\theta^2}+\frac{x^2}{\theta^2}}\exp\mycbra{-\frac{x^2}{\theta}}\] та 
			\[\frac{\partial^2}{\partial\theta^2}\mybra{\frac{x}{\theta}\exp\mycbra{-\frac{x^2}{\theta}}}=
			\mybra{\frac{2x}{\theta^3}-\frac{2x^2}{\theta^3}-\frac{x^3}{\theta^4}+\frac{x^4}{\theta^4}}
			\exp\mycbra{-\frac{x^2}{\theta}}\] існують для
			$\theta>0$.
		\item Покажемо, що для кожного $\theta_0>0$ має місце
			\[\frac{\partial}{\partial\theta}\mysbra{\int T(x)f(x;\theta)\dx}=\int T(x)\mysbra{\frac{\partial}{\partial
			\theta}f(x;\theta)}\dx\]
			де $T(x)=T(x_1,x_2,\hdots,x_n)$ статистика така, що $\forall\theta\;\mathbb{E}_\theta T(x),\;\mathbb{E}_\theta
			T^2(x)<+\infty$.
			Достатньо перевірити рівномірну на околі $\theta_0$
			збіжність інтегралу в правій частині. Маємо
			\[\int_a^\infty T(x)\mysbra{\frac{\partial}{\partial\theta}f(x;\theta)}\dx=
			\int_a^\infty T(x)\mybra{-\frac{x}{\theta^2}+\frac{x^2}{\theta^2}}\exp\mycbra{-\frac{x^2}{\theta}}\dx=\]
			\[=\int_a^\infty T(x)\sqrt{\frac{x}{\theta}}\exp\mycbra{-\frac{x^2}{2\theta}}\frac{(x-1)\sqrt{x}}{{\theta}^{
			\mysfrac{3}{2}}}\exp\mycbra{-\frac{x^2}{2\theta}}\dx\leq\]
			\[\leq\sqrt{\int_a^\infty T^2(x)f(x;\theta)\dx}\sqrt{\int_a^\infty\frac{(x-1)^2x}{\theta^3}
			\exp\mycbra{-\frac{x^2}{\theta}}\dx}\]
			підкореневий вираз з першого множника не може перевищувати $\mathbb{E}_\theta T^2$ і отже логічно припустити,
			що як і досліджувана статистика $\hat{\theta}$ має обмежений другий момент на околі $\theta_0$ (в іншому
			разі, його варіація буде очевидно більшою за варіацію $\hat{\theta}$). Таким чином, залишається
			довести, що підкореневий вираз в другому множнику прямую до нуля рівномірно на околі $\theta_0$. Надалі
			припускатимемо $\theta$ належить $0<b<\theta<c$ малому околі $\theta_0$. Тоді
			\[\int_a^\infty\frac{(x-1)^2x}{\theta^3}\exp\mycbra{-\frac{x^2}{\theta}}\dx=
			\int_{a/\sqrt{\theta}}^\infty\frac{(y\sqrt{\theta}-1)^2y\sqrt{\theta}}{\theta^3}
			\exp\mycbra{-y^2}\sqrt{\theta}\;dy\leq\]
			\[\leq\int_{a/\sqrt{c}}\frac{(y\sqrt{c}-1)^2y}{b^2}\exp\mycbra{-y^2}\;dy\to0,\;\mbox{ при }a\to0\]
	\end{enumerate}
	Таким чином, маємо
	\[D\hat{\theta}\geq\frac{1}{I(\theta)}\]
	де
	\[I(\theta)=M\mybra{\frac{\partial\ln f(\xi;\theta)}{\partial\theta}}^2=
	\int_{\mathbb{R}^n_{>0}}\mycbra{\sum_i\frac{\partial}{\partial\theta}f(x_i;\theta)}
	\Pi_i f(x_i;\theta)\;dx_1\hdots dx_{n-1}dx_n=
	\]\[=
	\int_{\mathbb{R}^n_{>0}}\mycbra{-\frac{n}{\theta}+\frac{1}{2\theta^2}\sum_ix_i^2
	}^2
	\frac{\Pi_i x_i}{\theta^n}\exp\mycbra{-\frac{\sum_ix_i^2}{2\theta}}\;dx_1\hdots dx_{n-1}dx_n=\]
	\[=\int_{\mathbb{R}^n_{>0}}\mysbra{\frac{n^2}{\theta^2}-\frac{n}{\theta^3}\sum_ix_i^2+\frac{1}{4\theta^4}\mysbra{
	\sum_ix_i^2}^2}
	\frac{\Pi_i x_i}{\theta^n}\exp\mycbra{-\frac{\sum_ix_i^2}{2\theta}}\;dx_1\hdots dx_{n-1}dx_n=\]
	\[=\frac{n^2}{\theta^2}-\frac{n}{\theta^3}2n\theta+\frac{1}{4\theta^4}\mysbra{n(n-1)4\theta^2+n\underbrace{\mathbb{E}\xi^4}_
	{8\theta^2}}=n/\theta^2\]
	і $\hat{\theta}$ є ефективною оцінкою, адже
	\[Var(\hat{\theta})=\frac{1}{4n^2}nVar\xi^2=\frac{1}{4n}\mybra{\mathbb{E}\xi^4-\mathbb{E}^2\xi^2}=
	\frac{1}{4n}\mybra{8\theta^2-4\theta^2}=\frac{\theta^2}{n}=\frac{1}{I(\theta)}\]
\section{Глава 4}%4.16, 4.17, 4.18, 
\setcounter{prob}{15}
\begin{prob}\end{prob}
	У термінах перевірки статистичних гіпотез задачу можна сформулювати так. Маємо 12 незалежних спостережень випадкової величини $\xi_i$
	-- реалізацію вибірки з нормального розподілу $N_{a;\sigma^2}$, але параметри розподілу невідомо. Відносно параметра
	$a$ висувається гіпотеза $H_0\;:\;a=a_0:=2000$. Це гіпотеза про відсутність істотного відхилення від номіналу.
	Природно визнати двобічну альтернативу. Відхилення гіпотези на користь альтернативи будемо інтерпретувати як наявність істотної різниці.

	Згідно з критерієм Стьюдента для перевірки $H_0\;:\;a=a_0$ при двобічній альтернативі треба порівняти значення 
	\[\myabs{t}=\myabs{\bar{\xi}-a_0}\left/s\sqrt{n}\right.,\mbox{ де }s:=\frac{1}{n+m-2}\mybra{(n-1)s_\xi^2+(m-1)s_\eta^2}\]
	з $t_{\alpha;(n-1)}$ -- верхньою $\alpha$-границею $t_{n-1}$-розподілу і відхиляти $H_0$ тоді і лише тоді, коли $\myabs{t}\geq t_{
	\alpha;(n-1)}$ (рівень значущості критерію становить $2\alpha$). Розрахунок показує, що
	\[\myabs{t}=1.02514<2.201 =t_{0.025;11}\]
	і таким чином на 5\%-рівні гіпотеза $H_0$ не відхиляється.

	Цей результат можна трактувати так. Припущення (гіпотеза) про те, що опір резисторів неістотно відрізняється від номіналу,
	не суперечить експериментальним даним, тобто іншими словами, такі покази типові для випадку відсутності
	відмінностей. Отже можна вважати, що істотних відмінностей немає.
\begin{prob}\end{prob}
	У термінах перевірки статистичних гіпотез задачу можна сформулювати так. Маємо 10 незалежних спостережень для сталі з конвертора I $\xi_i$
	та стільки ж для сталі з конвертора II $\eta_j$ з розподілів $N_{a_\xi;\sigma^2}$ та $N_{a_\eta;\sigma^2}$ відповідно. Відносно параметрів
	$a_\xi$ та $a_\eta$ висувається гіпотеза $H_0\;:\;a_\xi=a_\eta$. Це гіпотеза про те, що вміст марганцю в сталі виплавленій в обох 
	конверторах є приблизно однаковим.
	Природно визнати двобічну альтернативу. Відхилення гіпотези на користь альтернативи будемо інтерпретувати як наявність
	різниці.

	Згідно з критерієм Стьюдента для перевірки $H_0\;:\;a_\xi=a_\eta$ при двобічній альтернативі треба порівняти значення 
	\[\myabs{t}=\myabs{\bar{\xi}-\bar{\eta}}\left/s\sqrt{\frac{n+m}{nm}}\right.,\mbox{ де }s:=\frac{1}{n+m-2}\mybra{(n-1)s_\xi^2+(m-1)s_\eta^2}\]
	з $t_{\alpha;(n+m-2)}$ -- верхньою $\alpha$-границею $t_{n+m-2}$-розподілу і відхиляти $H_0$ тоді і лише тоді, коли $\myabs{t}\geq t_{
	\alpha;(n+m-2)}$ (рівень значущості критерію становить $2\alpha$). Розрахунок показує, що
	\[\myabs{t}=8.5657>2.101=t_{0.025;18}\]
	і таким чином на 5\%-рівні гіпотеза $H_0$ відхиляється.

	Цей результат можна трактувати так. Припущення (гіпотеза) про відсутність істотних відмінностей в показниках марганцю в сталі з обох
	конвеєрів суперечить експериментальним даним, тобто іншими словами, такі покази типові для випадку наявності
	відмінностей. Отже можна вважати, що відмінності є.
\begin{prob}\end{prob}%F_{\alpha;(n-1);(m-1)} /\in (1/F,F) => ne vidhil; 18
	У термінах перевірки статистичних гіпотез цю задачу можна сформулювати так. Маємо дві реалізації незалежних вибірок із нормальних розподілів
	$N_{a_\xi;\sigma_\xi^2}$ та $N_{a_\eta;\sigma_\eta^2}$. Відносно невідомих параметрів $\sigma_\xi^2$ та $\sigma_\eta^2$ висувається
	гіпотеза $H_0\;:\;\sigma_\xi^2/\sigma_\eta^2=1$ (гіпотеза про незміну дисперсії відбивної здатності фарби). Оскільки очікується, що нова
	технологія зменшить дисперсію, ми розглядаємо однобічну альтернативу $\sigma_\xi^2/\sigma_\eta^2>1$ (тобто, що дисперсія зменшилась).
	Відхилення гіпотези $H_0$ на користь цієї альтернативи трактуватимемо як те, що нова технологія дає ефект.

	Згідно з критерієм для перевірки гіпотези $H_0\;:\;\sigma_\xi^2/\sigma_\eta^2=1$ при однобічній альтернативі $H_0$ відхиляємо, якщо
	\[\frac{s_\xi^2}{s_\eta^2}>F_{\alpha;(n-1);(m-1)}\]
	і не відхилятимемо в противному разі. 

	У розглядуваному випадку
	\[\frac{s_\xi^2}{s_\eta^2}=4.7437<16F_{0.01;4;4}\]
	і тому на 1\%-му рівні значущості гіпотеза $H_0\;:\;\sigma_\xi^2/\sigma_\eta^2=1$ не відхиляється.

	Цей результат можна інтерпретувати так. Припущення, що технологічні зміни не дають зменшення дисперсії, не суперечить експериментальним 
	даним. Таким чином можна вважати, що зміни немає.
\section{Глава 5}%5.21, 
%5.23, 5.24
\setcounter{prob}{21}
\begin{prob}\end{prob}%22
	Розглянемо кількість колоній у квадраті як випадкову величину $\xi$. Щодо її розподілу висувається гіпотеза
	\[H_0\;:\:P\mycbra{\xi=k}=\frac{\lambda^ke^{-\lambda}}{k!},\;k=0,1,2,\hdots\]
	Треба перевірити цю гіпотезу.

	Параметр $\lambda$ гіпотетичного розподілу невідомий. Його необхідно оцінити за вибіркою. Як відомо, оцінкою максимальної правдоподібності
	параметра $\lambda$ розподілу Пуассона за вибіркою $\xi_1,x_2,\hdots,\xi_n$ є
	\[\hat{\lambda}=\frac{1}{n}\sum_{k=1}^n\xi_k\]
	Зокрема, у розглядуваному прикладі значення оцінки рівне $\hat{\lambda}=2.93$. Для того, щоб виконувався критерій "$np_i\geq10$",
	вибираємо розбиття $X_0:=\mycbra{0}$, $X_1:=\mycbra{1}$, $X_2:=\mycbra{2}$,
	$X_3:=\mycbra{3}$, $X_4:=\mycbra{4}$, $X_5:=\mycbra{5}$ і $X_6:=\mycbra{6,7,\hdots}$.
	Відповідні ймовірності попадання є рівними 0.053;0.156;0.229;0.223;0.163;0.096;0.04.

	Обчислимо значення відхилення Пірсона (нагадаємо, що ми оцінюємо один параметр і ділимо вибірковий простір на 7 підмножин)
	\[D(\hat{F}_n,G)=3.001<12.592=\chi^2_{0.05;6}\]
	Згідно з критерієм $\chi^2$ гіпотеза про Пуассонівський розподіл кількості колоній у квадраті не відхиляється. Припущення,
	що кількість колоній у квадраті має розподіл Пуассона, не суперечить даним.
\begin{prob}\end{prob}%23
	У термінах перевірки статистичних гіпотез поставлена задача формулюється як задача перевірки гіпотези про незалежність ознак
	(розумові здібності та якість одягу). У розглядуваному випадку $s=6$, $k=4$, $n=1725$, а значення $\nu_{ij}$, $\nu_{i\cdot}$,
	$\nu_{\cdot j}$; $i,j=1,2$ наведено в таблиці спряженості ознак. Щоб забезпечити виконання умови
	"${\nu_{i\cdot}\nu_{\cdot j}/n\geq10}$", ми зіллємо два перших і два останніх рядки в один ($s$ стає 4)
	. Тепер можна застосувати критерій.
	
	Маємо
	\[D(\hat{F}_n,G)=155.71\]
	і таким чином
	\[D(\hat{F}_n,G)=155.71>16.919=\chi^2_{0.05;9}=\chi^2_{\alpha;(s-1)(k-1)}\]
	Отже, згідно з критерієм $\chi^2$ гіпотеза про незалежність ознак (розумові здібності та якість одягу) відхиляється; вона
	суперечить наведеним даним. Останнє можна інтерпретувати як той факт, що люди із високими розумовими здібностями намагаються
	одягатися краще.
\begin{prob}\end{prob}
	Ми природно вважатимемо, що кожен компонент пари розподілений рівномірно від 0 до 9 і таким чином, в кожному окремому
	експерименті з вибору 100 пар, кількість пар (1,1) можна буде вважати розподіленою із розподілом Пуассона, адже ймовірність
	того, що окрема пара буде (1,1) мала, але ми робимо багато експериментів (100).

	Ми використали всю таблицю 8.10.1, якої вистачило на 21 експеримент. Результати наведено в таблиці
	\begin{center}
	\begin{tabular}{|c|ccccc|}
		\hline
		$i$&0&1&2&3&$\hdots$\\\hline
		$n_i$&5&12&2&2&0\\\hline
	\end{tabular}
	\end{center}

	Таким чином, розглянемо кількість появ пари (1;1) серед 100 як випадкову величину $\xi$. Щодо її розподілу висувається гіпотеза
	\[H_0\;:\:P\mycbra{\xi=k}=\frac{\lambda^ke^{-\lambda}}{k!},\;k=0,1,2,\hdots\]
	Треба перевірити цю гіпотезу.

	Параметр $\lambda$ гіпотетичного розподілу невідомий. Його необхідно оцінити за вибіркою. Як відомо, оцінкою максимальної правдоподібності
	параметра $\lambda$ розподілу Пуассона за вибіркою $\xi_1,x_2,\hdots,\xi_n$ є
	\[\hat{\lambda}=\frac{1}{n}\sum_{k=1}^n\xi_k\]
	Зокрема, у розглядуваному прикладі значення оцінки рівне $\hat{\lambda}=1.1428$. Через малу кількість експериментів, ми не будемо
	перевіряти умову. "$np_i\geq10$",
	Вибираємо розбиття $X_0:=\mycbra{0}$, $X_1:=\mycbra{1}$, $X_2:=\mycbra{2}$ та
	$X_3:=\mycbra{3,4,\hdots}$.
	Відповідні ймовірності попадання є рівними 0.318;0.3644;0.2082;0.0793.

	Обчислимо значення відхилення Пірсона (нагадаємо, що ми оцінюємо один параметр і ділимо вибірковий простір на 4 підмножини)
	\[D(\hat{F}_n,G)=4.2531<7.815=\chi^2_{0.05;3}\]
	Згідно з критерієм $\chi^2$ гіпотеза про Пуассонівський розподіл не відхиляється. Припущення,
	що кількість появ пари (1;1) серед 100 має розподіл Пуассона, не суперечить даним.
\section{Глава 6}
\setcounter{prob}{15}
\begin{prob}\end{prob}
	Покази 20 годинників можна розглядати як реалізацію вибірки обсягом 20 з деякого неперервного на відрізку $[0;12)$ розподілу $F$. Щодо
	розподілу $F$ випадкової величини $\xi$ (показів годинників) висувається гіпотеза $H_0:F$ є рівномірним розподілом на відрізку $[0;12)$,
	тобто щільність $f(x)$ розподілу $F$ має вигляд
	\[f(x)=\left\{\begin{array}{cr}1/12,&\mbox{якщо }x\in[0;12);\\0,&\mbox{якщо }x\notin[0;12);\end{array}\right.\]
	Користуючись методом
	Колмогорова, перевіримо нульову гіпотезу про те, що одержана вибірка є реалізацією вибірки з рівномірного на $[0,12]$ розподілу. Для початку
	ми пронормуємо дані, так щоб $[0,12]$ стало $[0,1]$, відповідно змінивши дані і гіпотезу.
	Далі цього обчислимо
	\[D(\hat{F}_n,G)=\sup_x\myabs{G(x)-\hat{F}_n(x)},\]
	де $G(x):=\min\mycbra{\max\mycbra{x,0},1}$ -- функція рівномірного на $[0,1]$ розподілу, а $\hat{F}_n(x)$ -- (нормована) 
	реалізація емпіричної функції розподілу, побудована за вибіркою. Маємо
	\[D(\hat{F}_n,G)=\sup_x\myabs{G(x)-\hat{F}_n(x)}=0.14722\]

	Отже,
	\[D(\hat{F}_n,G)=0.14722<0.2941=\varepsilon_{0.05;20}=\varepsilon_{\alpha;n}\]
	Таким чином, згідно з критерієм Колмогорова гіпотеза про те, що одержана вибірка є вибіркою з рівномірного на $[0,12]$ розподілу,
	не відхиляється. Відхилення емпіричної функції розподілу $\hat{F}_n(x)$ від $G(x)$ не є великим, для вибірки такого розміру воно
	цілком нормальне.
\end{document}
