\documentclass{article}
\usepackage{geometry,amsmath,amssymb,graphicx,bbm,theorem,xr}
\usepackage[american]{babel}
\geometry{letterpaper}

%%%%%%%%%% Start TeXmacs macros
\newcommand{\Nu}{\mathrm{N}}
\newcommand{\assign}{:=}
\newcommand{\comma}{{,}}
\newcommand{\longhookrightarrow}{{\lhook\joinrel\relbar\joinrel\rightarrow}}
\newcommand{\nequiv}{\not\equiv}
\newcommand{\nin}{\not\in}
\newcommand{\nni}{\not\ni}
\newcommand{\nobracket}{}
\newcommand{\nocomma}{}
\newcommand{\nosymbol}{}
\newcommand{\tmmathbf}[1]{\ensuremath{\boldsymbol{#1}}}
\newcommand{\tmop}[1]{\ensuremath{\operatorname{#1}}}
\newcommand{\tmtextbf}[1]{{\bfseries{#1}}}
\newcommand{\tmtextit}[1]{{\itshape{#1}}}
\newcommand{\tmtextrm}[1]{{\rmfamily{#1}}}
\newcommand{\tmtextup}[1]{{\upshape{#1}}}
\newcommand{\um}{-}
\newcommand{\upl}{+}
\newenvironment{itemizedot}{\begin{itemize} \renewcommand{\labelitemi}{$\bullet$}\renewcommand{\labelitemii}{$\bullet$}\renewcommand{\labelitemiii}{$\bullet$}\renewcommand{\labelitemiv}{$\bullet$}}{\end{itemize}}
\newenvironment{proof}{\noindent\textbf{Proof\ }}{\hspace*{\fill}$\Box$\medskip}
\newtheorem{definition}{Definition}
\numberwithin{definition}{section}
\newtheorem{lemma}{Lemma}
\numberwithin{lemma}{section}
\newtheorem{proposition}{Proposition}
\numberwithin{proposition}{section}
{\theorembodyfont{\rmfamily}\newtheorem{remark}{Remark}
\numberwithin{remark}{section}
}
%%%%%%%%%% End TeXmacs macros

\newcommand{\D}{\mathcal{D}} \newcommand{\supp}{supp}
\newcommand{\proofexplanation}[1]{(#1)}
\newcommand{\C}{{\mathbbm{C}}}\newcommand{\Z}{{\mathbbm{Z}}}
\newcommand{\Sp}{{\mathbbm{S}}} \newcommand{\R}{{\mathbbm{R}}}
\newcommand{\mybra}[1]{(#1)} \newcommand{\mysbra}[1]{\left[#1\right]}
\newcommand{\mycbra}[1]{\left\{#1\right\}}

\externaldocument{master_master1}
\externaldocument{master_master2}

\begin{document}

\title{Study of symmetry breaking operators of indefinite orthogonal groups $O( p, q)$.\\ 
III. Structural results.}
\author{T. Kobayashi, O. Leontiev}
\maketitle

{\tableofcontents}

\setcounter{section}{15}
\section{Introduction}

\subsection{Motivation and Background}

Given $G$ a Lie group and $G'$ its closed subgroup, a large amount of research
in representation theory can very roughly be said to be associated with the
problem of relating representations of $G$ with representations of $G'$. It is
interesting that this problem is ``two-sided'': that is, one can be interested
in constructing representations of $G$ starting with that of $G'$, or one can
be interested in decomposing given representation of $G$ into representations
of $G'$.

Regarding the former, the standard way of obtaining representations of $G$
from that of $G'$ is a so-called induction. In particular, if one starts with
the trivial representation of $G'$, proceeding in this way he arrives at the
representation of $G$ on $L^2 ( G / G')$. Trying to understand the latter
space, one essentially finds himself in the realm of harmonic analysis,
deriving Plancherel formulas etc. Another important special case that
attracted considerable attention is the case when one start with the trivial
representation of $G'$ and inducts to $G' \times G'$ (we see $G'$ as a
subgroup of $G' \times G'$ via the diagonal embedding). It was worked upon by
Gelfand and his students {\cite{gelfand1966generalized}} in '50s ,
Harish-Chandra {\cite{harishchandra1978harmonic}} \ in '70s, T. Oshima
{\cite{oshima1984description}}, P. Delorme {\cite{delorme1998plancherel}}, T.
Kobayashi
{\cite{kobayashi1994discrete1}},{\cite{kobayashi1998discrete2}},{\cite{kobayashi1998discrete3}}
and many others in '80s-'90s.

On the other hand, the latter problem (decomposing given
infinitely-dimensional representation $\pi$ of $G$ into representations $\tau$
of noncompact $G'$) appears to be more formidable with systematic study
beginning no earlier than in '90s. The simplest setting is that of
representation $\pi$ of $G$ decomposing as a discrete sum of irreducible
representations $\tau$ of $G'$. One is then interested in computing
multiplicities (i.e. number of times given $\tau$ appears in a sum) and this
can be done via combinatorial techniques. In general, however, infinitely
dimensional representation of $G$ \tmtextit{cannot} be decomposed into the
direct sum of irreducible representations of $G'$ and one should restrict
himself to a particularly ``nice'' settings (see {\cite{kobayashi2015program}}
for a thorough discussion). What is more, one finds himself in a situation
when he has several definitions for ``multiplicity'' and these in general are
\tmtextit{not} equal. One particularly well-behaved is $m ( \pi, \tau) \assign
\dim \tmop{Hom}_{G'} ( \pi |_{G'}, \tau)$ the dimension of $G'$-intertwining
operators or, as one sometimes calls them, \tmtextit{symmetry breaking
operators}. Finally, it \tmtextit{does not} happen in general that $\dim
\tmop{Hom}_{G'} ( \pi |_{G'}, \tau)$ are all finite, so one should further
restrict himself (see {\cite{kobayashi2014classification}} for classification
of pairs $( G, G')$ when multiplicities \tmtextit{are} always finite).

One particularly nice setting is that of $( G, G') = ( O ( p + 1, q), O ( p,
q))$. On the one hand, this setting is well-behaved in the sense that one does
not face none of the anomalies mentioned in the previous paragraph. In
particular, for $O ( n + 1, 1) \supset O ( n, 1)$ the complete classification
and explicit construction of all symmetry breaking operators between
degenerate principal series was accomplished in {\cite{kobayashi2015symmetry}}
by T. Kobayashi and B. Speh recently.

On the other hand, this setting is important to number theory (cf.
Gross-Prasad conjecture {\cite{gan2011symplectic}}, Rankin-Cohen bracket
{\cite{kobayashi2015differential1}},{\cite{kobayashi2015differential2}}),
conformal geometry (cf. Juhl conformal differential operators
{\cite{juhl2009families}}) and, of course, representation theory. Already the
special cases are very important. In particular, the latter sections of
{\cite{kobayashi2015symmetry}} contain some of the applications for $O ( n +
1, 1) \supset O ( n, 1)$ case. Another particular case $O ( 2, 2) \supset O (
2, 1)$ (see {\cite{clerc2011generalized}}) is equivalent to the problem of
finding invariant trilinear forms on representations of $\tmop{SL}_2 (
\mathbbm{R})$.

Based on the (quite general) techniques introduced in
{\cite{kobayashi2015symmetry}}, we attempt to investigate the symmetry
breaking for $( G, G') = ( O ( p + 1, q), O ( p, q))$ case.

\subsection{Main results}

Letting $p, q \geqslant 1$ and $( G, G') : = ( O ( p + 1, q + 1), O ( p, q +
1))$, the purpose of this paper is to answer the following

{\noindent}\tmtextbf{Question. }\tmtextit{For given $( \lambda, \nu) \in
\mathbbm{C}^2$ explicitly describe the vector space of $\tmop{Hom}_{G'} ( I (
\lambda), J ( \nu))$ of $G'$-intertwining operators between degenerate
principal series $I ( \lambda)$ and $J ( \nu)$ of $G$ and $G'$ respectively.
In particular, find the explicit form of a basis.}{\hspace*{\fill}}{\medskip}

{\noindent}This task is made possible by the following result, which is
essentially {\cite[thm. 3.16]{kobayashi2015symmetry}}

{\noindent}\tmtextbf{Proposition. }\tmtextit{(prop. \ref{sol:prop-sol} below)
For every $( \lambda, \nu) \in \mathbbm{C}^2$ we have $\tmop{Hom}_{G'} ( I (
\lambda), J ( \nu)) \simeq \mathcal{S} \tmop{ol} ( \mathbbm{R}^{p, q} ;
\lambda, \nu)$, where $\mathcal{S} \tmop{ol} ( \mathbbm{R}^{p, q} ; \lambda,
\nu)$ denotes the space of generalized function $F \in \mathcal{D}' (
\mathbbm{R}^{p, q})$, that:
\begin{enumerate}
  \item are homogeneous of degree $\lambda - \nu - n$;
  
  \item even on $\mathbbm{R}^{p, q}$;
  
  \item satisfy $F ( m \cdot) = F ( \cdot)$ for every $m \in O ( p, q)_{e_p}
  \assign \{ m \in O ( p, q) | m \cdot e_p = e_p \}$;
  
  \item for every $b, x_0 \in \mathbbm{R}^{p, q}$ such that $b_p = 0$ and $c_b
  ( x_0) : = 1 - 2 Q ( b, x_0) + Q ( x_0) Q ( b) \neq 0$ we have
  \begin{eqnarray}
    & | c_b ( \cdot) |^{\lambda - n} F ( \psi_b ( \cdot)) = F (\cdot)
    \hspace{1em} \tmop{near} \hspace{1em} x_0, &  \nonumber\\
    & \psi_b ( x) \assign \frac{x - Q (x) b}{c_b ( x)} . &  \nonumber
  \end{eqnarray}
\end{enumerate}}{\hspace*{\fill}}{\medskip}

{\noindent}Now, the main results of the paper are as follows:

{\noindent}\tmtextbf{Proposition. }\tmtextit{(see sec. \ref{sec:doublePGP})
Let $Q$ be $( p, q)$-quadratic form and $P, P'$ denote the maximal parabolic
subgroups of $G, G'$ respectively. Then,
\begin{enumerate}
  \item one can explicitly list the $P' \backslash G / P$ coset (see prop.
  \ref{doublePGP:prop-orbitdeco});
  
  \item pullback of $P' \backslash G / P$ cosets under the $\mathbbm{R}^{p, q}
  \simeq N_- \hookrightarrow G / P$ embedding are as follows:
  \[ \left\{ \begin{array}{ll}
       \{ x_p \neq 0, Q \neq 0 \} \sqcup \{ x_p \neq 0, Q = 0 \} \sqcup \{ x_p
       = 0, Q \neq 0 \} \sqcup \{ 0 \}, & p = 1\\
       \{ x_p \neq 0, Q \neq 0 \} \sqcup \{ x_p \neq 0, Q = 0 \} \sqcup \{ x_p
       = 0, Q \neq 0 \} \sqcup \{ x_p = 0, Q = 0 \} \backslash \{ 0 \} \sqcup
       \{ 0 \}, & p > 1
     \end{array} \right. \]
  where, say $\{ x_p \neq 0, Q \neq 0 \} \assign \{ x \in \mathbbm{R}^{p, q} |
  Q ( x) \neq 0, x_p \neq 0 \}$;
  
  \item $P' N_- P = G$.
\end{enumerate}}{\hspace*{\fill}}{\medskip}

{\noindent}\tmtextbf{Remark. }In particular, the third item supplies necessary
hypothesis for the proof of $\tmop{Hom}_{G'} ( I ( \lambda), J ( \nu)) \simeq
\mathcal{S} \tmop{ol} ( \mathbbm{R}^{p, q} ; \lambda, \nu)$ isomorphism, while
the second item implies that an element of $\mathcal{S} \tmop{ol} (
\mathbbm{R}^{p, q} ; \lambda, \nu)$ can have its support equal only to: $\{ 0
\}$, $P \assign \{ x \in \mathbbm{R}^{p, q} | x_p = 0 \}$, $C \assign \{ x \in
\mathbbm{R}^{p, q} | Q ( x) = 0 \}$, $P \cap C$, $C \cup P$ or
$\mathbbm{R}^{p, q}$.

We will therefore use the notation $\mathcal{S} \tmop{ol}_S ( \mathbbm{R}^{p,
q} ; \lambda, \nu)$ to denote elements of $\mathcal{S} \tmop{ol} (
\mathbbm{R}^{p, q} ; \lambda, \nu)$ which are supported inside some fixed
$S$.{\hspace*{\fill}}{\medskip}

{\noindent}\tmtextbf{Proposition. }\tmtextit{We have the following:
\begin{enumerate}
  \item (prop. \ref{supp-R:prop-3}) for $\lambda, \nu \in \mathbbm{C}$ with
  $\lambda - \nu \nin -\mathbbm{Z}_{\geqslant 0}$ one can extend the
  well-defined product of generalized functions
  \[ \frac{| x_p |^{\lambda + \nu - n}}{\Gamma ( ( \lambda + \nu - n + 1) /
     2)} \cdot \frac{| Q |^{- \nu}}{\Gamma ( ( 1 - \nu) / 2)} \in \mathcal{D}'
     ( \mathbbm{R}^{p, q} \backslash \{ 0 \}) \]
  to an element $K_{\lambda, \nu}^{\mathbbm{R}^n}$ of $\mathcal{S} \tmop{ol} (
  \mathbbm{R}^{p, q} ; \lambda, \nu)$. One can also explicitly determine its
  support (see prop. \ref{KR-normalization-recur:prop-supp}), which is
  generically equal to $\mathbbm{R}^{p, q}$;
  
  \item (prop. \ref{supp-Q:prop-sol-extending}) For $\nu \nin
  2\mathbbm{Z}_{\geqslant 0} + 1$ $\mathcal{S} \tmop{ol}_{\{ 0 \}} (
  \mathbbm{R}^{p, q} ; \lambda, \nu) =\mathcal{S} \tmop{ol}_C (
  \mathbbm{R}^{p, q} ; \lambda, \nu)$, while for $\nu \in
  2\mathbbm{Z}_{\geqslant 0} + 1$ fixed and $\lambda \in \{ \lambda \in
  \mathbbm{C} | \lambda - \nu \in -\mathbbm{Z}_{\geqslant 0} \}$ one can
  extend the well-defined product of generalized functions
  \[ \left\{ \begin{array}{ll}
       \delta^{( \nu - 1)} ( Q) \cdot | x_p |^{\lambda + \nu - n}, & p = 1\\
       \delta^{( \nu - 1)} ( Q) \cdot \frac{| x_p |^{\lambda + \nu -
       n}}{\Gamma ( ( \lambda + \nu - n + 1) / 2)}, & p > 1
     \end{array} \right. \]
  to an element $K_{\lambda, \nu}^C$ of $\mathcal{S} \tmop{ol}_C (
  \mathbbm{R}^{p, q} ; \lambda, \nu) \backslash\mathcal{S} \tmop{ol}_{\{ 0 \}}
  ( \mathbbm{R}^{p, q} ; \lambda, \nu)$. One can also explicitly determine its
  support (see prop. \ref{supp-Q:prop-supp-xnoq0}), which is generically equal
  to $C$;
  
  \item (sec. \ref{sec:supp-P}) For $\lambda + \nu - n = - 1 - 2 k, \; k \in
  \mathbbm{Z}_{\geqslant 0}$ the distribution
  \begin{eqnarray}
    & K^P_{\lambda, \nu} \assign \sum_{i = 0}^k \frac{(- 1)^i (2 k) !
    (\nu)^{}_i}{(2 k - 2 i) !i!} \delta^{(2 k - 2 i)} (x_p) \otimes
    \tilde{Q}_i \in \mathcal{D}' ( \mathbbm{R}^{p, q} ; \lambda, \nu), & 
    \nonumber\\
    & (\nu)^{}_i \assign \nu (\nu + 1) \ldots (\nu + i - 1), &  \nonumber\\
    & \tilde{Q}_i \assign \left\{ \begin{array}{ll}
      \tilde{Q}_+^{- \nu - i} + \tilde{Q}_-^{- \nu - i}, & i \in 2 \Z_{\ge 0},
      p \geqslant 2\\
      \tilde{Q}_+^{- \nu - i} - \tilde{Q}_-^{- \nu - i}, & i \in 2 \Z_{\ge 0}
      + 1, p \geqslant 2\\
      | \tilde{Q} |^{- \nu - i}, & i \in 2 \Z_{\ge 0}, p = 1\\
      - | \tilde{Q} |^{- \nu - i}, & i \in 2 \Z_{\ge 0} + 1, p = 1
    \end{array} \right. &  \nonumber
  \end{eqnarray}
  (with $\tilde{Q}$ being $( p - 1, q)$-quadratic form) is an element of
  $\mathcal{S} \tmop{ol}_P ( \mathbbm{R}^{p, q} ; \lambda, \nu)$ well-defined
  for $\nu \in \mathbbm{C}$ outside some discrete subset. Again, support is
  generically equal to $P$ and can be explicitly determined (see sec.
  \ref{sec:KP-normalization}).
\end{enumerate}}{\hspace*{\fill}}{\medskip}

{\noindent}\tmtextbf{Remark. }Three families $K_{\lambda,
\nu}^{\mathbbm{R}^n}$, $K_{\lambda, \nu}^C$ and $K_{\lambda, \nu}^P$ are thus
constructed for parameters outside some codimension one subsets of
$\mathbbm{C}^2$, $\mathbbm{C}$ and $\mathbbm{C}$ respectively. It turns out
that they have poles at these subsets, hence in order to extend these families
to the whole $\mathbbm{C}^2$, $\mathbbm{C}$ and $\mathbbm{C}$ respectively,
one should ``normalize'' them, that is, divide by some meromorphic function in
order to eliminate poles. This situation is much similar to the way one
normalizes generalized function $x_+^{\lambda}$ to show that $x_+^{\lambda} /
\Gamma ( \lambda + 1)$ is holomorphic (see {\cite[sec.
1.3.5]{gelfand1980distribution}}).{\hspace*{\fill}}{\medskip}

{\noindent}\tmtextbf{Proposition. }\tmtextit{We have the following:
\begin{enumerate}
  \item (sec. \ref{sec:KC-normalization}) Depending on $p, q \in
  \mathbbm{Z}_{\geqslant 1}$ and $\nu \in 2\mathbbm{Z}_{\geqslant 0} + 1$, one
  can find meromorphic function $N$ so that $K_{\lambda, \nu}^C / N$
  holomorphically extends to all $\lambda \in \mathbbm{C}$ to give a nonzero
  element of $\mathcal{S} \tmop{ol}_C ( \mathbbm{R}^{p, q} ; \lambda, \nu)$.
  Its support is explicitly determined.
  
  \item (sec. \ref{sec:KP-normalization}) Depending on $p, q \in
  \mathbbm{Z}_{\geqslant 1}$ and $k \assign - ( \lambda + \nu - n + 1) / 2$,
  one can find meromorphic function $N$ so that $K_{\lambda, \nu}^P / N$
  holomorphically extends to all $\nu \in \mathbbm{C}$ ($\lambda$ is
  determined by $\nu$ subject to $\lambda + \nu - n = - 1 - 2 k$ condition) to
  give a nonzero element of $\mathcal{S} \tmop{ol}_P ( \mathbbm{R}^{p, q} ;
  \lambda, \nu)$. Its support is explicitly determined.
  
  \item (sec. \ref{sec:KR-normalization-even}) For $q \in 2\mathbbm{Z}$ and
  depending on $p, q \in \mathbbm{Z}_{\geqslant 1}$, one can find meromorphic
  function $N$ so that $K_{\lambda, \nu}^{\mathbbm{R}^n} / N$ holomorphically
  extends to all $ ( \lambda, \nu) \in \mathbbm{C}^2$ to give an element of
  $\mathcal{S} \tmop{ol}_{} ( \mathbbm{R}^{p, q} ; \lambda, \nu)$ which is
  nonzero outside the discrete subset of $\mathbbm{C}^2$.
\end{enumerate}}{\hspace*{\fill}}{\medskip}

\subsection{Organization of the paper}

Following the style of {\cite{kobayashi2015symmetry}} it was attempted to make
the exposition detailed, elementary and as self-contained as possible.
Regarding the latter, the effort was made to clearly state all the statements
that we use without proof as ``facts'' and give clear references. The notable
exception to this are the statements from {\cite{kobayashi2015symmetry}}: we
use these within our proofs, still, hopefully, with clear indication of which
statements and where were used.

Now, the remaining sections are organized as follows:
\begin{itemizedot}
  \item in sections \ref{sec:holomorphicity-preserving} to
  \ref{sec:pull-tensor-mult} miscellaneous results are collected. These will
  be used in latter sections, mostly in and after the section \ref{sec:lem67};
  they are independent of other sections and (largely) of each other; we
  suggest to omit them on the first reading and return to them later as
  necessary;
  
  \item in section \ref{sec:def-n-nots} objects that we are dealing with are
  defined (the most importantly, groups $G \supset G'$ and their degenerate
  principal series $I ( \lambda)$ and $J ( \nu)$ respectively), the main goal
  of the paper is also stated;
  
  \item in section \ref{sec:doublePGP} two crucial results are stated and
  proven. Of these the first one is that $P' N_- P = G$ equality holds. It
  provides necessary hypothesis to apply {\cite[thm
  3.16]{kobayashi2015symmetry}} which sets up the isomorphism \
  $\tmop{Hom}_{G'} ( I ( \lambda), J ( \nu)) \simeq \mathcal{S} \tmop{ol} (
  \mathbbm{R}^{p, q} ; \lambda, \nu)$. The second one is the description of
  $P' \backslash G / P$ double coset space and description of pullback of
  these cosets under the $N_- \hookrightarrow G / P$ embedding. It allows us
  to predict what supports elements of $\mathcal{S} \tmop{ol} (
  \mathbbm{R}^{p, q} ; \lambda, \nu)$ may have;
  
  \item in section \ref{sec:sol} $\mathcal{S} \tmop{ol} ( \mathbbm{R}^{p, q} ;
  \lambda, \nu)$ is formally defined and it is shown to be isomorphic with the
  space of SBOs $\tmop{Hom}_{G'} ( I ( \lambda), J ( \nu))$ for every fixed
  pair of parameters $( \lambda \comma \nu) \in \mathbbm{C}^2$;
  
  \item in section \ref{sec:n-nonequiv} a counter-example related to the
  notions introduced in section \ref{sec:sol} is constructed; it can be
  omitted without the loss of continuity;
  
  \item in \ref{sec:lem67} system of equations that define $\mathcal{S}
  \tmop{ol} ( \mathbbm{R}^{p, q} ; \lambda, \nu)$ is explicitly solved, but on
  a smaller set $\{ x \in \mathbbm{R}^{p, q} | Q ( x) \neq 0 \}$ (where $Q$ is
  a $( p, q)$-quadratic form);
  
  \item in sections \ref{sec:supp-R}, \ref{sec:supp-P} and \ref{sec:supp-Q}
  elements of $\mathcal{S} \tmop{ol} ( \mathbbm{R}^{p, q} ; \lambda, \nu)$ are
  constructed, that are supported within $\mathbbm{R}^{p, q}$, $\{ x \in
  \mathbbm{R}^{p, q} | x_p = 0 \}$ and $\{ x \in \mathbbm{R}^{p, q} | Q ( x) =
  0 \}$ respectively for $( \lambda, \nu)$ lying in some open subsets of
  $\mathbbm{C}^2$ (different for all three cases). These elements are
  constructed, so that they depend on $( \lambda, \nu)$ holomorphically;
  
  \item in section \ref{sec:diffSBO} for all $( \lambda, \nu) \in
  \mathbbm{C}^2$ elements of $\mathcal{S} \tmop{ol} ( \mathbbm{R}^{p, q} ;
  \lambda, \nu)$ that are supported within $\{ 0 \}$ are classified; these
  correspond to differential SBO under the correspondence $\tmop{Hom}_{G'} ( I
  ( \lambda), J ( \nu)) \simeq \mathcal{S} \tmop{ol} ( \mathbbm{R}^{p, q} ;
  \lambda, \nu)$;
  
  \item in section \ref{sec:k-finite} a method to analytically continuate
  families of elements of $\mathcal{S} \tmop{ol} ( \mathbbm{R}^{p, q} ;
  \lambda, \nu)$ to a larger parameter sets is introduced;
  
  \item in sections \ref{sec:KC-normalization}, \ref{sec:KP-normalization} and
  \ref{sec:KR-normalization-even} definitions of $K_{\lambda, \nu}^C$,
  $K_{\lambda, \nu}^P$ and $K_{\lambda, \nu}^{\mathbbm{R}^n}$ made in sections
  \ref{sec:supp-Q}, \ref{sec:supp-P} and \ref{sec:supp-R} respectively are
  extended to all parameters $( \lambda, \nu) \in \mathbbm{C}^2$, so that
  holomorphic dependence is preserved. For $K_{\lambda, \nu}^{\mathbbm{R}^n}$
  this is done under the assumption $q \in 2\mathbbm{Z}$;
  
  \item section \ref{sec:knappstein} is an application of techniques
  introduced in {\cite{kobayashi2015symmetry}} and used in this paper, to the
  problem of finding a concrete form of $G$-invariant $I ( \lambda)
  \rightarrow I ( n - \lambda)$ operator in a relatively elementary and
  straightforward way; the obtained results parallel some of that of
  {\cite{KO1}}.
\end{itemizedot}
Starting from section \ref{sec:doublePGP} all sections have the same
structure: first, every section opens with a brief introduction. Next, three
subsections follow. The first one, ``Main results'' contains the statements of
main propositions and definitions this section contains. Next, subsection
``Auxiliary lemmas'' follow, containing stated and immediately proven lemmas
that we use to prove the main results of the section. Finally, in the
subsection ``Proofs'' main results of the section are proven.

\subsection{Acknowledgments}

I'd like to thank my supervisor Toshiyuki Kobayashi who brought this topic to
my attention and taught me all the techniques necessary.

\section{Differential symmetry breaking operators}\label{sec:diffSBO}

\subsection{Main results}

\begin{proposition}
  \label{diffSBO:prop-main}For $( \lambda, \nu) \in \mathbbm{C}^2$ we have
  \[ \mathcal{S} \tmop{ol}_{\{ 0 \}} ( \mathbbm{R}^{p, q} ; \lambda, \nu) =
     \left\{ \begin{array}{ll}
       \mathbbm{C} \cdot \tilde{C}_{\nu - \lambda}^{\lambda - ( n - 1) / 2}
       \left( \tilde{\Delta}, \frac{\partial}{\partial x_p} \right), & \lambda
       - \nu \in - 2\mathbbm{Z}_{\geqslant 0}\\
       0, & \tmop{otherwise}
     \end{array} \right. \]
  where $\tilde{\Delta}$ is a $( p - 1, q)$-Laplacian and $C^{\alpha}_k (
  \cdot, \cdot)$ is the 2-variable inflation of renormalized Gegenbauer
  polynomial, as in {\cite[(16.3)]{kobayashi2015symmetry}}. For $\lambda - \nu
  \in - 2\mathbbm{Z}_{\geqslant 0}$ we will use the notation
  \[ \tilde{K}_{\lambda, \nu}^{\{ 0 \}} \assign \tilde{C}_{\nu -
     \lambda}^{\lambda - ( n - 1) / 2} \left( \tilde{\Delta},
     \frac{\partial}{\partial x_p} \right) . \]
\end{proposition}

\subsection{Auxiliary results}

\begin{lemma}
  \label{diffSBO:lem-aux}For $f \in \mathcal{D}'_{\{ 0 \}} ( \mathbbm{R}^{p,
  q}) \assign \{ f \in \mathcal{D}' ( \mathbbm{R}^{p, q}) | \tmop{supp} ( f)
  \subset \{ 0 \} \}$ we have $f$ is $N_+'$-invariant on $\mathbbm{R}^{p, q}$
  (def. \ref{def-n-nots:def-n+invar}) iff it satisfies equations (\ref{Ndiff})
\end{lemma}

\begin{proof}
  As ``$\Rightarrow$'' part is clear, we just prove the converse. But it is
  readily implied by lemma \ref{supp-Q:lem-sing-q-4}.
\end{proof}

\subsection{Proofs}

\begin{proof}
  (of prop. \ref{diffSBO:prop-main}) We apply algebraic Fourier transform $F_c
  : \mathcal{D}'_{\{ 0 \}} ( \mathfrak{n}_-) \rightarrow \tmop{Pol} (
  \mathfrak{n}_-^{\ast})^{}$ defined by formula $( F_c f) ( \xi) \assign
  \hat{F} ( \xi) \assign \int_{\mathfrak{n}_-}^{} f ( x) \exp \langle x, \xi
  \rangle d x$. We further identify $\mathfrak{n}_-^{\ast}$ dual of
  $\mathfrak{n}_-$ with $\mathfrak{n}_+$, identification given via the pairing
  $\frac{1}{4} ( X, Y) \mapsto \tmop{tr} ( X Y)$.
  
  Lemmas \ref{diffSBO:lem-aux} and \ref{lem67:lem-homogImpliesE} tell us that
  $F \in \mathcal{D}'_{\{ 0 \}} ( \mathbbm{R}^{p, q})$ belongs to $\mathcal{S}
  \tmop{ol}_{\{ 0 \}} ( \mathbbm{R}^{p, q} ; \lambda, \nu)$ iff it satisfies
  \begin{eqnarray}
    & E F = ( \lambda - \nu - n) F &  \nonumber\\
    & F ( - x) = F ( x) &  \nonumber\\
    & F ( g \cdot) = F ( \cdot), \hspace{1em} \forall g \in O ( p, q)_{e_p}
    \assign \{ g \in O ( p, q) | g \cdot x_p = x_p \} \simeq O ( p - 1, q) & 
    \nonumber\\
    & \left[ (\lambda - n) \varepsilon_j x_j - \varepsilon_j x_j E +
    \frac{1}{2} Q (x) \frac{\partial}{\partial x_j} \right] F = 0,
    \hspace{1em} 1 \leqslant j \leqslant n, j \neq p &  \nonumber
  \end{eqnarray}
  Under the algebraic Fourier transform $F_c$ these change as follows (we note
  that $E$ under $F_c$ becomes $- E - n$, multiplication by $x_j$ becomes
  $\partial / \partial \zeta_j$ and $\partial / \partial x_j$ becomes
  multiplication by $- \zeta_j$):
  \begin{eqnarray}
    & E \hat{F} = ( \nu - \lambda) \hat{F} &  \nonumber\\
    & \hat{F} ( - x) = \hat{F} ( x) &  \nonumber\\
    & \hat{F} ( g \cdot) = \hat{F} ( \cdot), \hspace{1em} \forall g \in O (
    p, q)_{e_p} \assign \{ g \in O ( p, q) | g \cdot x_p = x_p \} \simeq O ( p
    - 1, q) &  \nonumber\\
    & \left[ - \frac{1}{2} \varepsilon_j \zeta_j \tilde{\Delta} + ( \lambda +
    E) \frac{\partial}{\partial \zeta_j} \right] \hat{F} = 0, \hspace{1em} 1
    \leqslant j \leqslant n, j \neq p, \hspace{1em} \tilde{\Delta} \assign
    \sum_{i = 1, i \neq p}^n \varepsilon_j \frac{\partial}{\partial \zeta_j} .
    &  \nonumber
  \end{eqnarray}
  Now, we make the following observations:
  \begin{enumerate}
    \item As $\hat{F}$ satisfies $E \hat{F} = ( \nu - \lambda) \hat{F}
    \nocomma$, it should be homogeneous of degree $\lambda - \nu$. But as it
    is a polynomial, it's homogeneity degree can only be positive integer, so
    we immediately see that $\mathcal{S} \tmop{ol}_{\{ 0 \}} ( \mathbbm{R}^{p,
    q} ; \lambda, \nu) = 0$ if $\lambda - \nu \nin -\mathbbm{Z}_{\geqslant
    0}$;
    
    \item Moreover, as $\hat{F}$ satisfies $\hat{F} ( - x) = \hat{F} ( x)$, it
    should be even and hence $\mathcal{S} \tmop{ol}_{\{ 0 \}} (
    \mathbbm{R}^{p, q} ; \lambda, \nu) = 0$ if $\lambda - \nu \nin -
    2\mathbbm{Z}_{\geqslant 0}$;
    
    \item The requirement $\hat{F} ( g \cdot) = \hat{F} ( \cdot), \hspace{1em}
    \forall g \in O ( p, q)_{e_p}$ implies that $\hat{F}$ is polynomial in
    $\tilde{Q} \assign \sum_{i = 1 ; i \neq p}^n \varepsilon_i x_i^2$ and
    $x_p$. Hence, using homogeneity with degree $a \assign \nu - \lambda \in
    2\mathbbm{Z}_{\geqslant 0}$ we can write then $\hat{F} = \tilde{Q}^{a / 2}
    g \left( x_{p - 1} / \sqrt{\tilde{Q}} \right)$ with $g ( \cdot)$ being a
    polynomial;
    
    \item Substituting this into the $\left[ - \frac{1}{2} \varepsilon_j
    \zeta_j \tilde{\Delta} + ( \lambda + E) \frac{\partial}{\partial \zeta_j}
    \right] \hat{F} = 0$ gives
    \[ g ( t) ( n - 1 - a - 2 \lambda) a - g' ( t) \times ( n - 2 - 2
       \lambda) t + ( t^2 + 1) g'' ( t) = 0 ; \]
    \item Substituting $t = i s$ into the previous equation gives (we abuse
    notation and write $g ( s)$ in place of $g ( i s)$)
    \[ g ( s) ( n - 1 - 2 \lambda) a - g' ( s) ( n - 2 - 2 \lambda) s + ( 1 -
       s^2) g'' ( s) = 0 ; \]
    hence $g ( s)$ should be proportional to $C^{\lambda - ( n - 1) / 2}_a (
    s)$ (see {\cite[thm. 11.4]{kobayashi2015differential2}}). Indeed, other
    polynomial solution (say, $\tilde{C}$) of this equation arises only when
    $\alpha \assign \lambda - ( n - 1) / 2 \in \mathbbm{Z}+ 1 / 2$ and $1 - 2
    a \leqslant 2 \alpha \leqslant - a$. But then $\tilde{C}$ has its top term
    is a non-zero multiple of $t^{- ( 2 \alpha + a)}$, but as $\alpha \in
    \mathbbm{Z}+ 1 / 2$, we have $- ( 2 \alpha + a) \in 2\mathbbm{Z}+ 1$,
    which contradicts evenness of $\hat{F}$.
  \end{enumerate}
  The result now follows by taking the inverse Fourier transform.
\end{proof}

\section{Role of $K$-finite vectors}\label{sec:k-finite}

\subsection{Main results}

\begin{proposition}
  \label{k-finite:prop-claim2}For $K \in \mathcal{S} \tmop{ol} (
  \mathbbm{R}^{p, q} ; \lambda, \nu)$ there exists unique $K^{\Xi} \in
  \mathcal{D}'_{\lambda - n} ( \Xi)^{\nu} \assign \left\{ f \in
  \mathcal{D}'_{\lambda - n} ( \Xi) | \forall p' \in P', \; f ( p' \cdot) =
  \nu ( p')^{} f ( \cdot) \right\}$ such that $\psi^{\ast} K^{\Xi} = K$ (for
  $\psi : \mathfrak{n}_- \hookrightarrow \Xi$ embedding). Moreover, if
  $K_{\mu} \in \mathcal{S} \tmop{ol} ( \mathbbm{R}^{p, q} ; \lambda ( \mu),
  \nu ( \mu))$ is holomorphic as generalized function on $\mathbbm{R}^{p, q}$
  depending on $\mu \in \Omega$ ($\Omega \subset \mathbbm{C}^n$ open) and
  $\lambda ( \cdot), \nu ( \cdot)$ holomorphic on $\Omega$, we have $(
  K_{\mu})^{\Xi}$ being holomorphic in $\mathcal{D}'_{\Gamma_{\Xi}} ( \Xi)$. 
\end{proposition}

\begin{proposition}
  \label{k-finite:prop-holo-to-holo}For $K \in \mathcal{S} \tmop{ol} (
  \mathbbm{R}^{p, q} ; \lambda, \nu)$ let $K^S \assign \iota^{\ast} K^{\Xi}$,
  where $K^{\Xi}$ is as in proposition \ref{k-finite:prop-claim2} and $\iota :
  \mathbbm{S}^p \times \mathbbm{S}^q \hookrightarrow \Xi$ is the embedding.
  Then, if $K_{\mu} \in \mathcal{S} \tmop{ol} ( \mathbbm{R}^{p, q} ; \lambda (
  \mu), \nu ( \mu))$ is holomorphic as distribution in $\mu \in \Omega$, with
  $\Omega \subset \mathbbm{C}^k$ open and $\lambda ( \cdot) \comma \nu (
  \cdot)$ holomorphic on $\Omega$, we have corresponding \ $( K_{\mu})^S \in
  \mathcal{D}' ( \mathbbm{S}^p \times \mathbbm{S}^q)$ be holomomorphic also.
  Moreover, if $\lambda ( \cdot), \nu ( \cdot)$ and $( K_{\mu})^S$
  holomorphically extend to $\Omega' \supset \Omega$, $K_{\mu}$ also extends
  to an element of $\mathcal{S} \tmop{ol} ( \mathbbm{R}^{p, q} ; \lambda (
  \mu), \nu ( \mu))$ for $\mu \in \Omega'$ and we have $( K_{\mu})^S =
  \iota^{\ast} K_{\mu}^{\Xi}$ for $\mu \in \Omega'$.
\end{proposition}

\begin{proposition}
  \label{k-finite:prop-kfinite-extension-oneparam}Suppose $\Omega \subset
  \mathbbm{C}$ is open with $D \subset \Omega$ discrete. Suppose further that
  $M$ is a smooth manifold with fixed volume form and $D_{\mu} \in
  \mathcal{D}' ( M)$ is holomorphic in $\mu \in \Omega \backslash D$. Suppose
  further that for every $f \in Z$ with $Z \subset C^{\infty}_0 ( M)$ dense
  (in the sense of Frechet topology on $C_0^{\infty} ( M)$) we have $\langle
  D_{\mu}, f \rangle$ being holomorphic in $\mu \in \Omega$. Then $D_{\mu}$
  extends to a holomorphic in $\mu \in \Omega$ generalized function on $M$. 
\end{proposition}

\begin{remark}
  Fixed volume form allows us to make sense of pairing of distribution and
  smooth function with compact support on $M$, as we \tmtextbf{do not} define
  $\mathcal{D}' ( M)$ is dual of $C_0^{\infty} ( M)$ -- see discussion in
  {\cite[sec 6.3]{hormander1983analysis}}.
\end{remark}

\begin{proposition}
  \label{k-finite:prop-KR-hook-1}For $( \lambda, \nu) \in \Omega_{- 1} \assign
  \{ \tmop{Re} ( - \nu) > 0, \tmop{Re} ( \lambda + \nu - n) > 0 \}$ we have
  $K_{\lambda, \nu} \assign | x_p |^{\lambda + \nu - n} | Q |^{- \nu} \in
  \mathcal{S} \tmop{ol} ( \mathbbm{R}^{p, q} ; \lambda, \nu)$ by proposition
  \ref{supp-R:prop-regular}. Then, the corresponding $K_{\lambda, \nu}^S$
  (given by proposition \ref{k-finite:prop-holo-to-holo}) equals to $= |
  \xi_{p + 1} |^{\lambda + \nu - n} | \xi_{p + q + 2} - \xi_1 |^{- \nu}$.
  Moreover, for every $F \in \mathcal{H}^a ( \mathbbm{S}^p) \otimes
  \mathcal{H}^b ( \mathbbm{S}^q) \subset C^{\infty} ( \mathbbm{S}^p \times
  \mathbbm{S}^q)$ with $a + b \in 2\mathbbm{Z}$ we have
  \begin{equation}
    \langle K^S_{\lambda, \nu}, F \rangle = \sum'_{N \in
    2\mathbbm{Z}_{\geqslant 0}} k_N \cdot \varphi_N [ g_N]
    \label{k-finite:eq-hookR}
  \end{equation}
  where $g_N$ are even polynomials (``even'' in the sense $g ( x, y) = g ( -
  x, - y)$), $k_N$ are some entire nonzero functions in $( \lambda, \nu) \in
  \mathbbm{C}^2$ (depending only on $F$, $p, \; q$ and $N$), $\sum'$ denotes
  finite sum and
  \begin{eqnarray}
    & \varphi_N [ g] ( \lambda, \nu) \assign \left\{ \begin{array}{ll}
      \Gamma ( \lambda + \nu - n + 1) / \Gamma \left( \frac{\lambda + \nu - n
      - N \upl 2}{2} \right) / \Gamma \left( \frac{\lambda + \nu + N - q}{2}
      \right), & p > 1\\
      1, & p = 1, N = 0\\
      0, & p = 1, N > 0
    \end{array} \right. \times &  \nonumber\\
    & \times \int_{[ - 1, 1]^2} | x - y |^{- \nu} ( 1 - x^2)^{( q - 2) / 2} (
    1 - y^2)^{( \lambda + \nu + N - q) / 2 - 1} g ( x, y) d x d y. & 
    \nonumber
  \end{eqnarray}
  Moreover, for every $N \in 2\mathbbm{Z}_{\geqslant 0}$ and even polynomial
  $g$ there exists $F \in \sum_{i, a_i + b_i \in 2\mathbbm{Z}}'
  \mathcal{H}^{a_i} ( \mathbbm{S}^p) \otimes \mathcal{H}^{b_i} (
  \mathbbm{S}^q)$ such that $\langle K_{\lambda, \nu}^S, F \rangle = k \cdot
  \varphi_N [ g]$ for some $k$ entire nonzero in $( \lambda, \nu) \in
  \mathbbm{C}^2$.
\end{proposition}

\begin{remark}
  Note that under the assumption $( \lambda, \nu) \in \Omega_{- 1}$,
  $\varphi_N [ g]$ are holomorphic.
\end{remark}

\begin{proposition}
  \label{k-finite:prop-KR-hook-2}For $R$ meromorphic nonzero in $( \lambda,
  \nu) \in \mathbbm{C}^2$ and $K_{\lambda, \nu}^{\mathbbm{R}^n} \in
  \mathcal{S} \tmop{ol} ( \mathbbm{R}^n ; \lambda, \nu)$ as defined in
  proposition \ref{supp-R:prop-3} for $( \lambda, \nu) \in \{ \lambda - \nu
  \nin -\mathbbm{Z}_{\geqslant 0} \}$ we have $K_{\lambda,
  \nu}^{\mathbbm{R}^n} / R$ extending to holomorphic in $( \lambda, \nu) \in
  \mathbbm{C}^2$ distribution if for every $N \in 2\mathbbm{Z}_{\geqslant 0}$
  and $g$:even polynomial we have $\varphi_N [ g] / \Gamma \left(
  \frac{\lambda + \nu - n + 1}{2} \right) / \Gamma \left( \frac{1 - \nu}{2}
  \right) / R$ extending to holomorphic in $( \lambda, \nu) \in \mathbbm{C}$.
  Moreover, for $( \lambda, \nu) \in \mathbbm{C}^2$ we have $K_{\lambda,
  \nu}^{\mathbbm{R}^n} / R = 0 \Leftrightarrow \forall N \in
  2\mathbbm{Z}_{\geqslant 0} \forall g, \; \left( \varphi_N [ g] / \Gamma
  \left( \frac{\lambda + \nu - n + 1}{2} \right) / \Gamma \left( \frac{1 -
  \nu}{2} \right) / R \right) ( \lambda, \nu) = 0$.
\end{proposition}

\begin{proposition}
  \label{k-finite:prop-KC-hook-kfinite}Fix $\nu \in 2\mathbbm{Z}_{\geqslant 0}
  + 1$. Then there exists $M$ such that for $\lambda \in \Omega_M \assign \{
  \lambda \in \mathbbm{C} | \tmop{Re} ( \lambda) > M \}$ such that for
  $\delta^{( \nu - 1)} ( Q) \cdot | x_p |^{\lambda + \nu - n} \in \mathcal{S}
  \tmop{ol}_{\{ Q = 0 \}} ( \mathbbm{R}^n ; \lambda, \nu)$ as in proposition
  \ref{supp-Q:prop-sol-extending}, corresponding $K^S_{\lambda, \nu}$ given by
  proposition \ref{k-finite:prop-holo-to-holo} and every $F \in \mathcal{H}^a
  ( \mathbbm{S}^p) \otimes \mathcal{H}^b ( \mathbbm{S}^q) \subset C^{\infty} (
  \mathbbm{S}^p \times \mathbbm{S}^q)$ with $a + b \in 2\mathbbm{Z}$ we have
  \begin{equation}
    \langle K^S_{\lambda, \nu}, F \rangle = \sum'_{N \in
    2\mathbbm{Z}_{\geqslant 0}} k_N \cdot \varphi_N [ g_N]
    \label{k-finite:eq-hookC}
  \end{equation}
  where $g_N$ are even polynomials (``even'' in the sense $g ( x, y) = g ( -
  x, - y)$), $k_N$ are some entire nonzero functions in $( \lambda, \nu) \in
  \mathbbm{C}^2$ (depending only on $F$, $p, \; q$ and $N$), $\sum'$ denotes
  finite sum and
  \begin{eqnarray}
    & \varphi_N [ g] ( \lambda, \nu) \assign \left\{ \begin{array}{ll}
      \Gamma ( \lambda + \nu - n + 1) / \Gamma \left( \frac{\lambda + \nu - n
      - N \upl 2}{2} \right) / \Gamma \left( \frac{\lambda + \nu + N - q}{2}
      \right), & p > 1\\
      1, & p = 1, N = 0\\
      0, & p = 1, N > 0
    \end{array} \right. \times &  \nonumber\\
    & \times \int_{- 1}^1 ( 1 - y^2)^{( \lambda + \nu + N - q) / 2 - 1}
    \frac{d^{\nu - 1}}{d x^{\nu - 1}} |_{x = y} [ ( 1 - x^2)^{( q - 2) / 2} g
    ( x, y)] d y. &  \nonumber
  \end{eqnarray}
  Moreover, for every $N \in 2\mathbbm{Z}_{\geqslant 0}$ and even polynomial
  $g$ there exists $F \in \sum_{i, a_i + b_i \in 2\mathbbm{Z}}'
  \mathcal{H}^{a_i} ( \mathbbm{S}^p) \otimes \mathcal{H}^{b_i} (
  \mathbbm{S}^q)$ such that $\langle K_{\lambda, \nu}^S, F \rangle = k \cdot
  \varphi_N [ g]$ for some $k$ entire nonzero in $( \lambda, \nu) \in
  \mathbbm{C}^2$.
\end{proposition}

\begin{proposition}
  \label{k-finite:prop-KC-hook-wrap}Fix $\nu \in 2\mathbbm{Z}_{\geqslant 0} +
  1$ and let
  \[ N_0 \assign \left\{ \begin{array}{ll}
       1, & p = 1\\
       \Gamma \left( \frac{\lambda + \nu - n + 1}{2} \right), & p > 1
     \end{array} \right. \]
  For $R$ meromorphic nonzero in $\lambda \in \mathbbm{C}$ and $K_{\lambda,
  \nu}^C \in \mathcal{S} \tmop{ol} ( \mathbbm{R}^n ; \lambda, \nu)$ as defined
  in proposition \ref{supp-Q:prop-sol-extending} for $\lambda \in \{ \lambda -
  \nu \nin -\mathbbm{Z}_{\geqslant 0} \}$ we have $K_{\lambda, \nu}^C / R$
  extending to holomorphic in $\lambda \in \mathbbm{C}$ distribution if for
  every $N \in 2\mathbbm{Z}_{\geqslant 0}$ and $g$:even polynomial we have
  $\varphi_N [ g] / N_0 / R$ extending to holomorphic in $\lambda \in
  \mathbbm{C}$. Moreover, for $\lambda \in \mathbbm{C}$ we have $K_{\lambda,
  \nu}^C / R = 0 \Leftrightarrow \forall N \in 2\mathbbm{Z}_{\geqslant 0}
  \forall g, \; ( \varphi_N [ g] / N_0 / R) ( \lambda, \nu) = 0$.
\end{proposition}

\subsection{Auxiliary lemmas}

\begin{lemma}
  \label{k-finite:lem-good-cover}Suppose $\{ p_i' \}_i$ is a finite set of
  points of $P'$, such that $\{ p'_i \mathfrak{n}_- \mathbbm{R}^{\times} \}_i$
  form an open cover of $\Xi$. Then one can find $\{ \varphi_i \}_i$ partition
  of unity subordinate to $\{ p'_i \mathfrak{n}_- \mathbbm{R}^{\times} \}_i$
  that consists of homogeneous functions of order $0$.
\end{lemma}

\begin{proof}
  As projection $\pi : \Xi \twoheadrightarrow \Xi /\mathbbm{R}^{\times} \simeq
  \mathbbm{S}^p \times \mathbbm{S}^q / \{ \pm \}$ is an open map, $U_i \assign
  \pi ( p_i' \mathfrak{n}_- \mathbbm{R}^{\times})$ will give us an open cover
  of $\mathbbm{S}^p \times \mathbbm{S}^q / \{ \pm \}$ and we can take
  partition of unity $\{ \psi_i \}_i$ subordinate to it. We can then define
  $\varphi_i ( x) \assign \psi_i ( \pi ( x))$. These $\varphi_i$ will be
  smooth on $\Xi$ and homogeneous of order 0. As $\sum_i \psi_i = 1$ on $\Xi
  /\mathbbm{R}^{\times}$, we have $\sum_i \varphi_i = 1$ on $\Xi$. Finally, as
  $\varphi_i$ is supported inside $U_i$ and we have $\pi^{- 1} ( U_i) = p_i'
  \mathfrak{n}_- \mathbbm{R}^{\times}$ ($\supseteq$ is by definition of
  $\pi^{- 1} ( \cdot)$; $\subseteq$ because, $p_i' \mathfrak{n}_-
  \mathbbm{R}^{\times}$ is $\mathbbm{R}^{\times}$-cone) $\varphi_i$ is
  supported inside $p_i' \mathfrak{n}_- \mathbbm{R}^{\times}$ and hence $\{
  \varphi_i \}$ is the partition subordinate to $\{ p'_i \mathfrak{n}_-
  \mathbbm{R}^{\times} \}_i$.
\end{proof}

\begin{lemma}
  \label{k-finite:lem-compat-A}The following holds:
  \begin{enumerate}
    \item For $a \in A'$, left multiplication by $a$ is a diffemororphism of
    $\mathfrak{n}_- \mathbbm{R}^{\times}$ with itself;
    
    \item For $K \in \mathcal{S} \tmop{ol} ( \mathbbm{R}^{p, q} ; \lambda,
    \nu)$ and $k \assign K^{\lambda - n} \in \mathcal{D}'_{\lambda - n} (
    \mathfrak{n}_- \mathbbm{R}^{\times})$ constructed as in lemma
    \ref{k-finite:lem-claim1}, we have $L_a k = \nu ( a^{- 1}) k$, where $L_a
    k ( \cdot) \assign k ( a^{- 1} \cdot)$.
  \end{enumerate}
\end{lemma}

\begin{proof}
  The first item is clear, as one observes that for $a ( t) \in A'$ as in $(
  \ref{def-n-nots:eq-A})$, we have
  \[ a ( t) \psi ( x) = e^t \cdot \psi ( e^{- t} x) . \]
  For the second item, uniqueness part of lemma \ref{k-finite:lem-claim1}
  implies that it suffices to show that for $F \in \mathcal{D}' (
  \mathbbm{R}^{p, q})$ and $a ( t) \in A'$ as above, we have
  \begin{equation}
    \psi^{\ast} ( L_{a ( t)} F^{\lambda - n}) ( \cdot) = e^{( n - \lambda) t}
    F ( e^t \cdot) \label{k-finite:eq-compat-A}
  \end{equation}
  as then $\lambda - \nu - n$-homogeneity of $K \in \mathcal{S} \tmop{ol} (
  \mathbbm{R}^{p, q} ; \lambda, \nu)$ would imply that $e^{( n - \lambda) t} F
  ( e^t \cdot) = e^{- \nu} F ( \cdot)$.
  
  In turn, as both sides $( \ref{k-finite:eq-compat-A})$ are continuous in
  $F$, we can assume $F \in C^{\infty}_0 ( \mathbbm{R}^{p, q})$ and the
  statement then becomes clear.
\end{proof}

\begin{lemma}
  \label{k-finite:lem-compat-M}The following holds:
  \begin{enumerate}
    \item For $m \in M'$, left multiplication by $m$ is a diffemororphism of
    $\mathfrak{n}_- \mathbbm{R}^{\times}$ with itself;
    
    \item For $K \in \mathcal{S} \tmop{ol} ( \mathbbm{R}^{p, q} ; \lambda,
    \nu)$ and $k \assign K^{\lambda - n} \in \mathcal{D}'_{\lambda - n} (
    \mathfrak{n}_- \mathbbm{R}^{\times})$ constructed as in lemma
    \ref{k-finite:lem-claim1}, we have $L_m k = \nu ( m^{- 1}) k$, where $L_m
    k ( \cdot) \assign k ( m^{- 1} \cdot)$.
  \end{enumerate}
\end{lemma}

\begin{proof}
  Proof is almost identical to that of lemma \ref{k-finite:lem-compat-A}.
\end{proof}

\begin{lemma}
  \label{k-finite:lem-compat}Suppose $K_{} \in \mathcal{S} \tmop{ol} (
  \mathbbm{R}^{p, q} ; \lambda, \nu)$ and let $K^{\lambda - n} \in
  \mathcal{D}'_{\lambda - n} ( \mathfrak{n}_- \mathbbm{R}^{\times})$ be as in
  lemma \ref{k-finite:lem-claim1}. Suppose further that for $U, V \subset
  \mathfrak{n}_- \mathbbm{R}^{\times}$: open $\mathbbm{R}^{\times}$-cones and
  $p' \in P'$ we have $p' U = V$. Then for $K_U \assign K^{\lambda - n} |_U$
  and similarly $K_V$ we have $K_U ( p' \cdot) = \nu ( p')^{} K_V ( \cdot)$.
\end{lemma}

\begin{proof}
  Using $P' = M' A' N_+'$ decomposition and lemmas \ref{k-finite:lem-compat-A}
  and \ref{k-finite:lem-compat-M}, we may assume that $p' \in N_+'$. The
  statement then follows directly from lemma \ref{k-finite:lem-compat-N}.
\end{proof}

\begin{lemma}
  \label{k-finite:lem-compat-N}The conclusion of lemma
  \ref{k-finite:lem-compat} holds for $p' \in N_+'$.
\end{lemma}

\begin{proof}
  We let notation be as in the statement of lemma \ref{k-finite:lem-compat}.
  We also let $U_{\psi}, \; V_{\psi}$ to be the pullbacks of $U$ and $V$
  respectively by $\psi$ -- these are open subsets of $\mathbbm{R}^{p, q}$, as
  $\psi$ is an embedding. We let $p' = : n_{} ( b)$ (with $n ( \cdot)$ as in
  $( \ref{def-n-nots:eq-N+})$) for $b \in \mathbbm{R}^{p, q}$ with $b_p = 0$.
  Hypothesis and lemma \ref{k-finite:lem-compat-N-aux} imply that $c_b (
  \cdot)$ is non-zero on $U_{\psi}$. Then lemma
  \ref{k-finite:lem-compat-N-aux} implies that $V_{\psi} = \psi_b (
  U_{\psi})$, where $\psi_b ( x) \assign ( x - Q ( x) b) / c_b ( x)$.
  
  It suffices to show that for $f_V$ being generalized function on $V_{\psi}$
  we have $L_{n ( b)} ( | c_b |^{\lambda - n} \psi_b^{\ast} f_V)^{\lambda - n}
  = ( f_V)^{\lambda - n}$ with $( \cdot)^{\lambda - n}$ as in lemma
  \ref{k-finite:lem-claim1}, as then the desired conclusion will be granted by
  item 4 of definition \ref{sol:def-sol} (note that $\nu ( n ( b)) = 1$). Due
  to the continuity of all the operations involved, we may assume that $f_V$
  is smooth and then the desired equality can be equivalently rewritten as
  \begin{equation}
    | c_b |^{\lambda - n} \cdot ( f_V \circ \psi_b^{})^{\lambda - n} = L_{n (
    - b)} ( f_V)^{\lambda - n} \label{k-finite:eq-compat-N}
  \end{equation}
  (we view $c_b$ as function on $U$). Uniqueness part of lemma
  \ref{k-finite:lem-claim1} then tells us that for $x \in \psi (
  \mathfrak{n}_-)$ and $t \in \mathbbm{R}^{\times}$ we have $( f_V \circ
  \psi_b^{})^{\lambda - n} ( t x) = | t |^{\lambda - n} f_V ( \psi_b ( x))$
  and $( f_V)^{\lambda - n} ( t x) = | t |^{\lambda - n} f_V ( x)$. Therefore
  (as both side of $( \ref{k-finite:eq-compat-N})$ are homogeneous of order
  $\lambda - n$), it suffices to show that for arbitrary $x \in U$ we have $|
  c_b ( x) |^{\lambda - n} f_V ( \psi_b ( x)) = ( f_V)^{\lambda - n} ( n ( b)
  \psi_{} ( x))$. As $n ( b) \psi ( x) = c_b ( x) \psi ( \psi_b ( x))$ and $(
  f_V)^{\lambda - n}$ is homogeneous of degree $\lambda - n$ by construction,
  the $| c_b ( x) |^{\lambda - n} f_V ( \psi_b ( x)) = ( f_V)^{\lambda - n} (
  n ( b) \psi_{} ( x))$ holds and the proof is over.
\end{proof}

\begin{lemma}
  \label{k-finite:lem-c1}Fix $\nu \in 2\mathbbm{Z}_{\geqslant 0} + 1$. Then
  there exists $M$ such that for $\lambda \in \Omega_M \assign \{ \lambda \in
  \mathbbm{C} | \tmop{Re} ( \lambda) > M \}$ such that for $\delta^{( \nu -
  1)} ( Q) \cdot | x_p |^{\lambda + \nu - n} \in \mathcal{S} \tmop{ol}_{\{ Q =
  0 \}} ( \mathbbm{R}^n ; \lambda, \nu)$ as in proposition
  \ref{supp-Q:prop-sol-extending} we have corresponding $K^{\Xi}_{\lambda,
  \nu}$ given by proposition \ref{k-finite:prop-claim2} being equal to
  $\langle K_{\lambda, \nu}^{\Xi}, \varphi \rangle \assign \langle \delta^{(
  \nu - 1)} ( \xi_{p + q + 2} - \xi_1) \nocomma, | \xi_{p + 1} |^{\lambda +
  \nu - n} \varphi \nocomma \rangle$.
\end{lemma}

\begin{proof}
  Till the end of the proof, we will call $\nu_{} \in 2\mathbbm{Z}_{\geqslant
  0} + 1$ in the statement by the name $\nu_0$. Now, application of lemma
  \ref{k-finite:lem-1} (and the observation that $| x |^{\lambda}$ enters
  arbitrary fixed $C^k ( \mathbbm{R})$ class for $\tmop{Re} ( \lambda)$ high
  enough) implies that there exists $M$ such that if $\lambda_0 \in \Omega_M$
  is fixed, then
  \[ K_{\nu} : \varphi \mapsto \left\langle \frac{| \xi_{p + q + 2} - \xi_1
     |^{- \nu}}{\Gamma ( ( 1 - \nu) / 2)} \nocomma, | \xi_{p + 1} |^{\lambda_0
     + \nu_0 - n} \varphi \nocomma \right\rangle \]
  is holomorphic in $\nu \in \Omega \assign \{ \tmop{Re} ( \nu) > \tmop{Re} (
  \nu_0) - 1 \}$. We note that $\langle K_{\nu_0}, \varphi \rangle = \langle
  \delta^{( \nu - 1)} ( \xi_{p + q + 2} - \xi_1) \nocomma, | \xi_{p + 1}
  |^{\lambda + \nu - n} \varphi \nocomma \rangle$ and thus the uniqueness part
  of proposition \ref{k-finite:prop-claim2} tells us that it suffices to show
  the following:
  \begin{enumerate}
    \item $\psi^{\ast} K_{\nu_0} = \delta^{( \nu - 1)} ( Q) \cdot | x_p
    |^{\lambda + \nu - n}$;
    
    \item $K_{\nu_0} \in \mathcal{D}'_{\lambda - n} ( \Xi)$;
    
    \item $\forall p' \in P', \; K_{\nu_0} ( p' \cdot) = \nu_0 ( p')^{}
    K_{\nu_0} ( \cdot)$.
  \end{enumerate}
  Now, we see that for $\tmop{Re} ( \nu) \gg 0$, $K_{\nu}$ becomes continuous.
  It is then easy to verify that for $\tmop{Re} ( \nu) \gg 0$ $K_{\nu}$ is
  holomorphic of degree $\lambda_0 + \nu_0 - n - \nu_{}$. Hence, by
  holomorphic extension, $K_{\nu}$ is so for $\nu \in \Omega$ as well. This
  implies the second item of the list above.
  
  Homogeneity together with lemma \ref{k-finite:lem-holo-easy} imply also that
  $K_{\nu}$ is holomorphic in $\mathcal{D}'_{\Gamma_{\Xi}} ( \Xi)$ and hence
  $\psi^{\ast} K_{\nu}$ is holomorphic as well. As for big $\nu$ $K_{\nu}$ is
  continuous, proposition \ref{holomorphicity-preserving:prop-pullback-cts}
  implies that $\psi^{\ast} K_{\nu} \simeq | Q |^{- \nu} / \Gamma ( ( 1 - \nu)
  / 2) \cdot | x_p |^{\lambda_0 + \nu_0 - n}$ and the holomorphicity of
  distributional product implies that $\psi^{\ast} K_{\nu_0} |_{\mathbbm{R}^n
  \backslash \{ 0 \}} = \delta^{( \nu_0 - 1)} ( Q) \cdot | x_p |^{\lambda_0 +
  \nu_0 - n}$ and then homogeneity (if necessary, we increase $\tmop{Re} (
  \lambda)$) implies that equality holds for $\mathbbm{R}^n$ as well. This
  shows the first item of the list above.
  
  Finally, for $\tmop{Re} ( \nu) \gg 0$ when $K_{\nu}$ becomes continuous we
  see that (for $a ( t) \in A$ and $n ( v) \in N_+$ as in $(
  \ref{def-n-nots:eq-A})$ and $( \ref{def-n-nots:eq-N+})$ respectively)
  \begin{eqnarray}
    & F_{\nu} ( \xi) : = | \xi_{p + 1} |^{\lambda_0 + \nu_0 - n} | \xi_{p + q
    + 2} - \xi_1 |^{- \nu} &  \nonumber\\
    & p' \in P', \; ( L_{p'} f) ( \cdot) \assign f ( ( p')^{- 1} \cdot) & 
    \nonumber\\
    & a ( t) \in A' \Rightarrow ( L_{a ( t)} F_{\nu}) ( \xi) = e^{- \nu t}
    F_{\nu} ( \xi) &  \nonumber\\
    & m \assign \tmop{diag} ( 1, m', 1) \in M', \; m' \in O ( p, q)_{e_p}
    \Rightarrow ( L_m F_{\nu}) ( \xi) = F_{\nu} ( \xi) &  \nonumber\\
    & m \assign \tmop{diag} ( - 1, 1, - 1) \in M' \Rightarrow ( L_m F_{\nu})
    ( \xi) = F_{\nu} ( \xi) &  \nonumber\\
    & v \in \mathbbm{R}^{p, q}, \; n ( v) \in N_+', \; v_p = 0 \Rightarrow (
    L_{n ( v)} F_{\nu}) ( \xi) = F_{\nu} ( \xi) . &  \nonumber
  \end{eqnarray}
  By analytic continuation these relations extend to $\nu \in \Omega$ and this
  proves the third item of the list above.
\end{proof}

\begin{lemma}
  \label{k-finite:lem-c2}Fix $\nu \in 2\mathbbm{Z}_{\geqslant 0} + 1$. Then
  there exists $M$ such that for $\lambda \in \Omega_M \assign \{ \lambda \in
  \mathbbm{C} | \tmop{Re} ( \lambda) > M \}$ such that for $\delta^{( \nu -
  1)} ( Q) \cdot | x_p |^{\lambda + \nu - n} \in \mathcal{S} \tmop{ol}_{\{ Q =
  0 \}} ( \mathbbm{R}^n ; \lambda, \nu)$ as in proposition
  \ref{supp-Q:prop-sol-extending} we have corresponding $K^S_{\lambda, \nu}$
  given by proposition \ref{k-finite:prop-holo-to-holo} being equal to
  $\langle K_{\lambda, \nu}^S, \varphi \rangle \assign \langle \delta^{( \nu -
  1)} ( \xi_{p + q + 2} - \xi_1) \nocomma, | \xi_{p + 1} |^{\lambda + \nu - n}
  \varphi \nocomma \rangle$.
\end{lemma}

\begin{proof}
  As in the proof of lemma \ref{k-finite:lem-c1}, we call $\nu$ in the
  statement by the name $\nu_0$ and consider $K_{\nu} \in \mathcal{D}' ( \Xi)$
  holomorphic in $\nu \in \Omega$ with all notations being as in proof of
  lemma \ref{k-finite:lem-c1}. Similarly, we see that
  \[ \mathcal{D}' ( \mathbbm{S}^p \times \mathbbm{S}^q) \ni k_{\nu} : \varphi
     \mapsto \left\langle \frac{| \xi_{p + q + 2} - \xi_1 |^{- \nu}}{\Gamma (
     ( 1 - \nu) / 2)} \nocomma, | \xi_{p + 1} |^{\lambda_0 + \nu_0 - n}
     \varphi \nocomma \right\rangle \]
  is holomorphic in $\nu \in \Omega$ and as for $\iota : \mathbbm{S}^p \times
  \mathbbm{S}^q \hookrightarrow \Xi$ embedding we have (by proposition
  \ref{holomorphicity-preserving:prop-pullback-cts}) $\iota^{\ast} K_{\nu} =
  k_{\nu}$ for $\tmop{Re} ( \nu) \gg 0$ (when both $k_{\nu}$ and $K_{\nu}$
  become continuous) and thus this holds for $\nu \in \Omega$ by analytic
  continuation (note that $\iota^{\ast} K_{\nu}$ is holo in $\nu \in \Omega$,
  since $K_{\nu}$ is holo in $\mathcal{D}'_{\Gamma_{\Xi}} ( \Xi)$, as shown in
  the proof of lemma \ref{k-finite:lem-c1}). Setting $\nu = \nu_0$ we get the
  desired result.
\end{proof}

\subsection{Proofs}

\begin{proof}
  (of prop. \ref{k-finite:prop-claim2}) As proposition
  \ref{def-n-nots:prop-ximodel} implies that $\Xi /\mathbbm{R}^{\times} \simeq
  G / P$ and we've shown in proposition \ref{doublePGP:prop-pnp} that $P' N_-
  P = G$, we see that (by compactness of $G / P$) one can find finitely many
  $p_i' \in P'$ such that $U_i \assign p_i' \cdot \mathfrak{n}_-
  \mathbbm{R}^{\times}$ cover $\Xi$. Without the loss of generality, we assume
  that $p'_0 = 1$, thus $U_0 =\mathfrak{n}_- \mathbbm{R}^{\times}$.
  
  This immediately implies for an element $F \in \mathcal{D}'_{\lambda - n} (
  \Xi)^{\nu}$ that if it vanishes on $\mathfrak{n}_- \mathbbm{R}^{\times}$, it
  vanishes everywhere. Together with the uniqueness part of lemma
  \ref{k-finite:lem-claim1}, this implies the uniqueness part.
  
  Next, we show the existence. For $p' \in P'$ and $f$ generalized function on
  open $\mathbbm{R}^{\times}$-cone $U \subset \Xi$, we let $( L_{p'} f) (
  \cdot) = f ( ( p')^{- 1} \cdot)$ be a generalized function defined on open
  cone $p' U$. Note that $L_{p'}$ preserves homogeneity and that $L_a L_b =
  L_{a b}$ for $a, b \in P'$. For $p'$ we let $L_{p'}^{\ast}$ be a
  $C_0^{\infty} ( U) \rightarrow C_0^{\infty} ( ( p')^{- 1} U)$ map defined by
  $\forall f \in \mathcal{D}' ( U), \; \langle L_{p'} f, \varphi \rangle =
  \langle f, L_{p'}^{\ast} \varphi \rangle$.
  
  Given $K \in \mathcal{S} \tmop{ol} ( \mathbbm{R}^{p, q} ; \lambda, \nu)$,
  lemma \ref{k-finite:lem-claim1} gives us $k \in \mathcal{D}'_{\lambda - n} (
  \mathfrak{n}_- \mathbbm{R}^{\times})$ which is pulled back to $K$ under
  $\psi$, so it remains to extend $k$ to an element of $\mathcal{D}'_{\lambda
  - n} ( \Xi)^{\nu}$. We do this is follows: for $\{ \varphi_i \}_i$ partition
  of unity associated to $\{ U_i \}_i$ finite open cover of $\Xi$, we set
  $\langle K^{\Xi}, \varphi \rangle \assign \sum_i \nu ( p'_i)^{} \langle
  \varphi_i L_{p'_i} k, \varphi \rangle$. It is clear that $K^{\Xi}$ is
  well-defined generalized function on $\Xi$. Moreover, as lemma
  \ref{k-finite:lem-good-cover} allows us to assume that $\varphi_i$ were
  taken homogeneous of degree 0, we have $\varphi_i L_{p_i'} k$ is homogeneous
  of degree $\lambda - n$, hence so is $K^{\Xi}$. Finally, it remains to show
  that for every $p' \in P'$ and $\varphi \in C^{\infty}_0 ( \Xi)$ we have
  $\nu ( p') \langle L_{p'} K^{\Xi}, \varphi \rangle = \langle K^{\Xi},
  \varphi \rangle$ and that $\psi^{\ast} K^{\Xi} = K$.
  
  Assuming these are shown, we note that the remaining statement
  (holomorphicity) is readily given by the construction of $K^{\Xi}$. Indeed,
  $k \in \mathcal{D}'_{\lambda - n} ( U_0)$ is holomorphic in
  $\mathcal{D}'_{\Gamma_{\Xi}}$ by lemma \ref{k-finite:lem-claim1}. As we note
  that left multiplication by $p' \in P'$ when seen as diffeomorphism of $\Xi$
  pulls $\Gamma_{\Xi}$ to itself, we see that $L_{p'_i} k ( U_i)$ is
  holomorphic in $\mathcal{D}'_{\Gamma_{\Xi}} ( \Xi)$ as well. Then, so is
  every $\varphi_i L_{p'_i} k$ and hence the whole $K^{\Xi}$.
  
  We first show that $\psi^{\ast} K^{\Xi} = K$. In the light of lemma
  \ref{k-finite:lem-claim1}, it suffices to show for this that $K^{\Xi}
  |_{\mathfrak{n}_- \mathbbm{R}^{\times}} = k$. Now, for $\varphi \in
  C_0^{\infty} ( U_0)$ we have $\langle K^{\Xi}, \varphi \rangle \assign
  \sum_i \nu ( p_i') \langle L_{p_i'} k, \varphi_{( i)} \rangle$, where we let
  $\varphi_{( i)} \assign \varphi_i \varphi$. We note that $\varphi_{( i)}$ is
  supported inside $U_i \cap U_0$ and as we have $U_i = p_i' U_0$, we have by
  lemma \ref{k-finite:lem-compat} (applied to $( U, V) \assign ( U_0 \cap (
  p_i')^{- 1} U_0, U_i \cap U_0)$) that $\nu ( p_i') \langle L_{p_i'} k,
  \varphi_{( i)} \rangle = \langle k, \varphi_{( i)} \rangle$, hence $\langle
  K^{\Xi}, \varphi \rangle = \left\langle k, \sum_i \varphi_{( i)}
  \right\rangle = \langle k, \varphi \rangle$. This shows that $\psi^{\ast}
  K^{\Xi} = K$.
  
  We note that the argument of the previous paragraph in fact also shows that
  for $\varphi \in C^{\infty}_0 ( U_i)$ we have $\langle K^{\Xi}, \varphi
  \rangle = \nu ( p_i') \langle L_{p_i'} k, \varphi \rangle$.
  
  Finally, it remains to show that for every $p' \in P'$ and $\varphi \in
  C^{\infty}_0 ( \Xi)$ we have
  \begin{equation}
    \nu ( p') \langle L_{p'} K^{\Xi}, \varphi \rangle = \langle K^{\Xi},
    \varphi \rangle . \label{k-finite:eq-claim2}
  \end{equation}
  As $( \ref{k-finite:eq-claim2})$ is linear in $\varphi$, we can assume that
  $\varphi$ is supported in $U_i \cap p' U_j = p_i' U_0 \cap p' p_j' U_0$.
  Therefore the observation of previous paragraph implies that left and right
  hand sides of $( \ref{k-finite:eq-claim2})$ are equal to $\nu ( p' p_j')
  \langle k, L_{p_j'}^{\ast} L_{p'}^{\ast} \varphi \rangle$ and $\nu ( p_i')
  \langle k, L_{p_i'}^{\ast} \varphi \rangle$ respectively with
  $L_{p_j'}^{\ast} L_{p'}^{\ast} \varphi$ and $L_{p_i'}^{\ast} \varphi$ being
  supported in $p'' U_0 \cap U_0$ and $U_0 \cap ( p_{}'')^{- 1} U_0$
  respectively for $p'' \assign ( p' p_j')^{- 1} p_i'$. Therefore, applying
  lemma \ref{k-finite:lem-compat} we see that $\nu ( p' p_j') \langle k,
  L_{p_j'}^{\ast} L_{p'}^{\ast} \varphi \rangle = \nu ( p' p_j') \langle
  L_{p'} L_{p_j'} k,^{} \varphi \rangle = \nu ( p' p_j') \nu ( p'') \langle
  L_{p'} L_{p'_j} L_{p''} k, \varphi \rangle = \nu ( p_i') \langle k,
  L_{p_i'}^{\ast} \varphi \rangle$. This proves $( \ref{k-finite:eq-claim2})$
  and ends the proof.
\end{proof}

\begin{proof}
  (of prop. \ref{k-finite:prop-holo-to-holo}) We note that the restriction
  $\iota^{\ast} K^{\Xi}$ is well-defined by lemma
  \ref{k-finite:lem-restriction-to-S}. Holomorphicity of $( K_{\mu})^S$
  follows by holomorphicity part of proposition \ref{k-finite:prop-claim2} and
  proposition \ref{holomorphicity-preserving:prop-pullback-holo}, so it
  remains to prove the part regarding the holomorphic extension. So suppose
  that $( K_{\mu})^S$ extends to $\mu \in \Omega' \supset \Omega$. Then,
  holomorphicity implies that for $\mu \in \Omega'$ $( K_{\mu})^S$ is still
  even as distribution on $\mathbbm{S}^p \times \mathbbm{S}^q$. Then, the
  converse part of lemma \ref{k-finite:lem-restriction-to-S} and lemma
  \ref{k-finite:lem-abs-is-holo} together with holomorphicity of tensor and
  pullback imply that we can construct $k_{\mu} \in \mathcal{D}'_{\lambda (
  \mu) - n} ( \Xi)$ that is holomorphic in $\mu \in \Omega'$ and such that
  $\iota^{\ast} k_{\mu} = ( K_{\mu})^S$. Then, $k_{\mu} = ( K_{\mu})^{\Xi}$
  for $\mu \in \Omega$ (as their restriction to $\mathbbm{S}^p \times
  \mathbbm{S}^q$ and homogeneity degree coincides, this is implied by lemma
  \ref{k-finite:lem-restriction-to-S}). Then, $\psi^{\ast} k_{\mu} \in
  \mathcal{D}' ( \mathbbm{R}^{p, q})$ is holomorphic in $\mu$ (by proposition
  \ref{holomorphicity-preserving:prop-pullback-holo}) and coincides with
  $K_{\mu}$ for $\mu \in \Omega$, hence proposition \ref{sol:prop-holocont}
  implies that $\psi^{\ast} k_{\mu} \in \mathcal{S} \tmop{ol} (
  \mathbbm{R}^{p, q} ; \lambda ( \mu), \nu ( \mu))$ for $\mu \in \Omega'$. It
  remains to show that we have $( K_{\mu})^S = \iota^{\ast} K_{\mu}^{\Xi}$ for
  $\mu \in \Omega'$. But this is clear, as both sides are holomorphic in $\mu
  \in \Omega'$ and they coincide for $\mu \in \Omega$.
\end{proof}

\begin{proof}
  (of prop. \ref{k-finite:prop-kfinite-extension-oneparam}) Indeed, build the
  Laurent expansion of $K_{\mu}$ at arbitrary $\mu_0 \in D$: $K_{\mu} =
  \sum_{i = - \infty}^{\infty} K_i ( \mu - \mu_0)^i$. The hypothesis then
  implies that for every $\varphi \in Z$ we have $\langle K_i, \varphi \rangle
  = 0$ for every $i \in -\mathbbm{Z}_{> 0}$ and hence we see that $K_i = 0$
  for $i < 0$ and thus $K_{\mu}$ is holomorphic at $\mu_0$. As $\mu_0$ was
  arbitrary, we are done.
\end{proof}

\begin{proof}
  (of prop. \ref{k-finite:prop-KR-hook-1}) As $K_{\lambda, \nu} \assign | x_p
  |^{\lambda + \nu - n} | Q |^{- \nu} \in \mathcal{S} \tmop{ol} (
  \mathbbm{R}^{p, q} ; \lambda, \nu)$ is continuous for $( \lambda, \nu) \in
  \Omega_0$, the uniqueness part of proposition \ref{k-finite:prop-claim2}
  (note that proposition \ref{holomorphicity-preserving:prop-pullback-cts}
  tells us that for continuous functions distributional pullback coincides
  with the usual one) implies that $K_{\lambda, \nu}^{\Xi} ( \xi) = | \xi_{p +
  1} |^{\lambda + \nu - n} | \xi_{p + q + 2} - \xi_1 |^{- \nu}$. Then, pulled
  back to $\mathbbm{S}^p \times \mathbbm{S}^q$ (again, note proposition
  \ref{holomorphicity-preserving:prop-pullback-cts}) this becomes $| \xi_{p +
  1} |^{\lambda + \nu - n} | \xi_{p + q + 2} - \xi_1 |^{- \nu}, \; \xi \in
  \mathbbm{S}^p \times \mathbbm{S}^q$. This proves the first assertion.
  
  Before going further, we need to put an element $F \in \mathcal{H}^a (
  \mathbbm{S}^p) \otimes \mathcal{H}^b ( \mathbbm{S}^q)$ into a more concrete
  form. Due to linearity, we can assume (using $( \xi, \eta)$ variable
  splitting for points of $\mathbbm{S}^p \times \mathbbm{S}^q$ with $\xi \in
  \mathbbm{S}^p$ and $\eta \in \mathbbm{S}^q$) that $F ( \xi \comma \eta) =
  h_a ( \xi) \cdot h_b ( \eta)$. Then, using the explicit form of
  $\mathcal{H}^L ( \mathbbm{S}^n) = \bigoplus_{N = 0}^L \mathcal{H}^N (
  \mathbbm{S}^{n - 1})$ branching law given in {\cite[sec
  4.2]{kobayashi2015symmetry}}, we see that we can assume (again, due to
  linearity) that
  \[ F ( \xi, \eta) = | \xi' |^N \phi \left( \frac{\xi'}{| \xi' |} \right)
     \xi_1^{m'} | \eta' |^M \phi' \left( \frac{\eta'}{| \eta' |} \right)
     \eta_{q + 1}^{n'} \]
  with $( \phi, \phi') \in \mathcal{H}^N ( \mathbbm{S}^{p - 1}) \times
  \mathcal{H}^M ( \mathbbm{S}^{q - 1})$ and $( \xi_1, \xi') \assign \xi$, $(
  \eta', \eta_{q + 1}) \assign \eta$. As we should have $F$ being even, we see
  that we should have $m' + M + N + n' \in 2\mathbbm{Z}$.
  
  Next, we consider an embedding $\psi_{N \rightarrow S} : \mathfrak{n}_-
  \simeq \mathbbm{R}^{p, q} \ni x \mapsto \pi ( 1 - Q ( x), 2 x, 1 + Q ( x)) /
  \sqrt{R} \in \mathbbm{S}^p \times \mathbbm{S}^q$ (here $C^{\infty} (
  \mathbbm{R}^{p, q}) \ni R$ is given in bipolar coordinates as $R ( r, s)
  \assign ( 1 - r^2 + s^2)^2 + 4 r^2$)). As $- \psi_{N \xrightarrow{} S} (
  \mathfrak{n}_-) \cup \psi_{N \xrightarrow{} S} ( \mathfrak{n}_-) \subset
  \mathbbm{S}^p \times \mathbbm{S}^q$ is open dense (this is implied by the
  first item of lemma \ref{k-finite:lem-compat-N-aux}) and both $F \in
  \mathcal{H}^a ( \mathbbm{S}^p) \otimes \mathcal{H}^b ( \mathbbm{S}^q)$ and
  $K_{\lambda, \nu}^S$ are even continuous on $\mathbbm{S}^p \times
  \mathbbm{S}^q$ (the former is so due to $a + b \in 2\mathbbm{Z}$
  hypothesis), we see that $\langle K_{\lambda, \nu}^S, F
  \rangle_{\mathbbm{S}^p \times \mathbbm{S}^q} = \langle \psi_{N \rightarrow
  S}^{\ast} K_{\lambda, \nu}^S, \psi_{N \rightarrow S}^{\ast} F
  \rangle_{\mathfrak{n}_-}$. Now, as one notes that canonical volume form on
  $\mathbbm{S}^p \times \mathbbm{S}^q$ is pulled back to a constant multiple
  of $| R ( r, s) |^{- n / 2}$, we have (transferring to bipolar coordinates
  on $\mathbbm{R}^{p, q}$)
  \begin{eqnarray}
    & \langle \psi_{N \rightarrow S}^{\ast} K_{\lambda, \nu}^S, \psi_{N
    \rightarrow S}^{\ast} F \rangle_{\mathfrak{n}_-} \simeq
    \int_{\mathbbm{S}^{p - 1}} \phi ( \omega) | \omega_p |^{\lambda + \nu - n}
    d \omega_{} \times \int_{\mathbbm{S}^{q - 1}} \phi' ( \omega') d \omega'
    \times &  \nonumber\\
    & \times \int_{r, s = 0}^{\infty} r^{\lambda + \nu - n} | r^2 - s^2 |^{-
    \nu} R^{( n - \lambda) / 2} R^{- n / 2} r^{p - 1} s^{q - 1} \times & 
    \nonumber\\
    & \left| \frac{r}{\sqrt{R}} \right|^N \left| \frac{s}{\sqrt{R}} \right|^M
    \left( \frac{1 - r^2 + s^2}{\sqrt{R}} \right)^{m'} \left( \frac{1 + r^2 -
    s^2}{\sqrt{R}} \right)^{n'} d r d s &  \nonumber
  \end{eqnarray}
  Now, as we have $\int_{\mathbbm{S}^{q - 1}} \phi' ( \omega') d \omega' = 0$
  unless $\phi' = \tmop{const}$, we see that $\langle \psi_{N \rightarrow
  S}^{\ast} K_{\lambda, \nu}^S, \psi_{N \rightarrow S}^{\ast} F
  \rangle_{\mathfrak{n}_-} = 0$ unless $M = 0$. Thus, in subsequent we assume
  $M = 0$.
  
  Furthermore, as explained in {\cite[lem 7.6]{kobayashi2015symmetry}},
  $\int_{\mathbb{S}^{p - 1}} \phi (\omega_{}) | \omega_p^{} |^{\lambda + \nu -
  n} d \omega_{} = 0$ if $N$ is odd or $\psi \perp \mathcal{H}^N (
  \mathbb{S}^{p - 1})^{O (p - 1)} =\mathbbm{C} \psi_N$, where $\mathcal{H}^N (
  \mathbbm{S}^{p - 1}) \ni \psi_N : \omega_{p - 1} \rightarrow
  \widetilde{\tilde{C}}_N^{p / 2 - 1} ( \omega_{p - 1}^{( p)})$, with
  $\widetilde{\tilde{C}}_N^{\mu}$ denoting the renormalized Gegenbauer
  polynomial (see {\cite[(16.4)]{kobayashi2015symmetry}}). Thus in subsequent
  we may assume $N, m' + n' \in 2\mathbbm{Z}_{\geqslant 0}$.
  
  Now, as shown in {\cite[lem. 7.6]{kobayashi2015symmetry}},
  \[ \int_{\mathbb{S}^{n - 1}} | \omega_{n - 1}^{(n)} |^{\lambda + \nu - n} 
     \widetilde{\tilde{C}}_N^{\frac{n}{2} - 1} (\omega_{n - 1}^{(n)}) d
     \omega_{n - 1} \simeq \frac{\Gamma ( \lambda + \nu - n + 1)}{\Gamma
     \left( \frac{\lambda + \nu - n - N + 2}{2} \right) \Gamma \left(
     \frac{\lambda + \nu - N}{2} \right)} . \]
  We note that it holds only for $n > 1$. For $n = 1$ {\cite[p.
  6]{howe1993homogeneous}} tells us that $\mathcal{H}^N ( \mathbbm{S}^{q - 1})
  = 0$ for $N > 0$ and for $N = 0$ the latter integral becomes independent of
  $( \lambda, \nu)$. This gives us a first factor in the formula for
  $\varphi_N [ g]$.
  
  We next compute the integral in $( r, s)$ above. Using variable change to
  coordinates $(x, y)$ given as
  \begin{eqnarray}
    & x \assign \frac{1 + (r^2 - s^2)}{\sqrt{R (r, s)}}, \hspace{1em} y
    \assign \frac{1 - (r^2 - s^2)}{\sqrt{R (r, s)}}, \hspace{1em}
    \frac{\partial ( r, s)}{\partial ( x, y)} = - \frac{1}{\sqrt{1 - s^2}
    \sqrt{1 - r^2} ( x + y)^2} &  \nonumber\\
    & r = \frac{\sqrt{1 - y^2}}{x + y}, \hspace{1em} s = \frac{\sqrt{1 -
    x^2}}{x + y} &  \nonumber
  \end{eqnarray}
  we can rewrite that integral as equal to (up to multiple inpendent of $m',
  n'$)
  \[ \int_{( x, y) \in D} x^{n'} y^{m'} ( 1 - x^2)^{( q - 2) / 2} ( 1 -
     y^2)^{( \lambda + \nu + N - q) / 2 - 1} | x - y |^{- \nu} d x d y, \]
  where $D \assign \{ ( x, y) \in ( - 1, 1)^2 | x + y > 0 \}$. But as $n' + m'
  \in 2\mathbbm{Z}_{\geqslant 0}$, this is proportional to an integral over $[
  - 1, 1]^2$.
\end{proof}

{\noindent}\tmtextbf{Fact \tmtextup{19}.
}\tmtextit{\label{k-finite:fact-hartogs}{\cite[thm.
1.2.5]{krantz1982function}} Let $\Omega \subset \mathbbm{C}^n$ be an open set
and $f : \Omega \rightarrow \mathbbm{C}$. Suppose that for each $j \in \{ 1,
2, \ldots, n \}$ and each fixed $( z_i)_{i = 1, i \neq j}^n$ we have $\{ z \in
\mathbbm{C} | ( z_1, z_2, \ldots, z_{j - 1}, z, z_{j + 1}, \ldots, z_n) \in
\Omega \} \ni z \mapsto f ( z_1, z_2, \ldots, z_{j - 1}, z, z_{j + 1}, \ldots,
z_n)$ is holomorphic as function of $z$. Then, $f$ is continuous on $\Omega$
and hence holomorphic on $\Omega$.}{\hspace*{\fill}}{\medskip}

\begin{proof}
  (of prop. \ref{k-finite:prop-KR-hook-2}) As $K_{\lambda,
  \nu}^{\mathbbm{R}^n} / R$ is holomorphic in $( \lambda, \nu) \in \{ \lambda
  + \nu \nin -\mathbbm{Z}_{\geqslant 0} \}$ and is a member of $\mathcal{S}
  \tmop{ol} ( \mathbbm{R}^{p, q} ; \lambda, \nu)$, proposition
  \ref{k-finite:prop-holo-to-holo} gives us corresponding $k_{\lambda, \nu}
  \in \mathcal{D}' ( \mathbbm{S}^p \times \mathbbm{S}^q)$. Moreover,
  proposition \ref{k-finite:prop-KR-hook-1} tells us that for $( \lambda, \nu)
  \in \Omega_{- 1}$ and $F \in \mathcal{H}^a ( \mathbbm{S}^p) \otimes
  \mathcal{H}^b ( \mathbbm{S}^q)$ with $a + b \in 2\mathbbm{Z}$ we have
  \[ \langle k_{\lambda, \nu}, F \rangle = \sum'_{N \in
     2\mathbbm{Z}_{\geqslant 0}} k_N ( \lambda, \nu) \cdot \frac{\varphi_N
     [ g_N] ( \lambda, \nu)}{\Gamma \left( \frac{1 - \nu}{2} \right) \Gamma
     \left( \frac{\lambda + \nu - n + 1}{2} \right) R ( \lambda, \nu)} . \]
  Now, as both sides of latter equality are holomorphic in $( \lambda, \nu)
  \in \{ \lambda + \nu \nin -\mathbbm{Z}_{\geqslant 0} \}$ (the right-hand
  side is so by hypothesis), it extends to $\{ \lambda + \nu \nin
  -\mathbbm{Z}_{\geqslant 0} \}$.
  
  Now, we fix $\lambda = \lambda_0 \in \mathbbm{C}$ and let $k_{\nu} \assign
  k_{\lambda_0, \nu}$. The latter is holomorphic in $\nu \in
  \mathbbm{C}\backslash D_{}$ with $D_{} \assign \{ \nu \in \mathbbm{C} |
  \lambda_0 + \nu \in -\mathbbm{Z}_{\geqslant 0} \}$. We want to show that
  $k_{\nu}$ extends to holomorphic in $\nu \in \mathbbm{C}$ distribution.
  Proposition \ref{k-finite:prop-kfinite-extension-oneparam} tells us that it
  suffices to show that for every $F \in \mathcal{H}^a ( \mathbbm{S}^p)
  \otimes \mathcal{H}^b ( \mathbbm{S}^q)$ we have $\langle k_{\nu}, F \rangle$
  being holomorphic in $\nu \in \mathbbm{C}$. Now, for $a + b \in
  2\mathbbm{Z}+ 1$ we have $k_{\nu}$ being even, while $F$ is odd and hence
  $\langle k_{\nu}, F \rangle = 0$ clearly holomorphic in $\nu$. For $a + b
  \in 2\mathbbm{Z}$ the conclusion in turn is granted by hypothesis. Repeating
  the same argument with $\nu = \nu_0$ fixed and using fact
  \ref{k-finite:fact-hartogs} tells us that $k_{\lambda, \nu}$ extends to
  holomorphic in $( \lambda, \nu) \in \mathbbm{C}^2$ distribution on
  $\mathbbm{S}^p \times \mathbbm{S}^q$ and thus proposition
  \ref{k-finite:prop-holo-to-holo} tells us that so does $K_{\lambda,
  \nu}^{\mathbbm{R}^n} / R$.
  
  It remains to show that for for $( \lambda, \nu) \in \mathbbm{C}^2$ we have
  $K_{\lambda, \nu}^{\mathbbm{R}^n} / R = 0 \Leftrightarrow \forall N \in
  2\mathbbm{Z}_{\geqslant 0} \forall g, \; \left( \varphi_N [ g] / \Gamma
  \left( \frac{\lambda + \nu - n + 1}{2} \right) / \Gamma \left( \frac{1 -
  \nu}{2} \right) / R \right) ( \lambda, \nu) = 0$. Note that proposition
  \ref{k-finite:prop-holo-to-holo} tells us that $k_{\lambda, \nu} =
  \iota^{\ast} ( K_{\lambda, \nu}^{\mathbbm{R}^n} / R)^{\Xi}$ (with $(
  \cdot)^{\Xi}$ as in proposition \ref{k-finite:prop-claim2}). We first show
  the ``$\Leftarrow$''. Assuming that $\forall N \in 2\mathbbm{Z}_{\geqslant
  0} \forall g, \; \left( \varphi_N [ g] / \Gamma \left( \frac{\lambda + \nu -
  n + 1}{2} \right) / \Gamma \left( \frac{1 - \nu}{2} \right) / R \right) (
  \lambda, \nu) = 0$ we see that $\langle k_{\lambda, \nu}, F \rangle = 0$ for
  every $F \in \mathcal{H}^a ( \mathbbm{S}^p) \otimes \mathcal{H}^b (
  \mathbbm{S}^q)$ with $a + b \in 2\mathbbm{Z}$. Arguing as in previous
  paragraph we see that $\langle k_{\lambda, \nu}, F \rangle = 0$ for $F \in
  \mathcal{H}^a ( \mathbbm{S}^p) \otimes \mathcal{H}^b ( \mathbbm{S}^q)$
  without the $a + b \in 2\mathbbm{Z}$ assumption. Hence $k_{\lambda, \nu} =
  0$. Lemma \ref{k-finite:lem-restriction-to-S} now implies that $(
  K_{\lambda, \nu}^{\mathbbm{R}^n} / R)^{\Xi} \in \mathcal{D}'_{\lambda - n} (
  \Xi)$ is zero and hence $K_{\lambda, \nu}^{\mathbbm{R}^n} / R = \psi^{\ast}
  ( K_{\lambda, \nu}^{\mathbbm{R}^n} / R)^{\Xi} = 0$.
  
  Finally, we show the ``$\Rightarrow$'' direction, so we assume $K_{\lambda,
  \nu}^{\mathbbm{R}^n} / R = 0$. This implies (by the uniqueness part of lemma
  \ref{k-finite:prop-claim2}) that $( K_{\lambda, \nu}^{\mathbbm{R}^n} /
  R)^{\Xi} = 0$, hence $k_{\lambda, \nu} = \iota^{\ast} ( K_{\lambda,
  \nu}^{\mathbbm{R}^n} / R)^{\Xi} = 0$. Proposition
  \ref{k-finite:prop-KR-hook-1} now implies that for every $N \in
  2\mathbbm{Z}_{\geqslant 0}$ and even polynomial $g$ there exists $F \in
  \sum_{i, a_i + b_i \in 2\mathbbm{Z}}' \mathcal{H}^{a_i} ( \mathbbm{S}^p)
  \otimes \mathcal{H}^{b_i} ( \mathbbm{S}^q)$ such that $\langle k_{\lambda,
  \nu}^{}, F \rangle = k \cdot \varphi_N [ g] / / \Gamma \left( \frac{\lambda
  + \nu - n + 1}{2} \right) / \Gamma \left( \frac{1 - \nu}{2} \right) / R$ for
  some $k$ entire nonzero in $( \lambda, \nu) \in \mathbbm{C}^2$ (the actual
  statement of proposition \ref{k-finite:prop-KR-hook-1} was made to hold on
  $\Omega_{- 1}$, but as both sides are analytic in $( \lambda, \nu) \in \{
  \lambda - \nu \in -\mathbbm{Z}_{\geqslant 0} \}$ it extends) and thus we
  conclude that $\varphi_N [ g] / \Gamma \left( \frac{\lambda + \nu - n +
  1}{2} \right) / \Gamma \left( \frac{1 - \nu}{2} \right) / R = 0$. As $N$ and
  $g$ were arbitrary, we are done.
\end{proof}

\begin{proof}
  (of prop. \ref{k-finite:prop-KC-hook-kfinite}) As in the proof of
  proposition \ref{k-finite:prop-KR-hook-1}, we may assume
  \[ F ( \xi, \eta) = | \xi' |^N \phi \left( \frac{\xi'}{| \xi' |} \right)
     \xi_1^{m'} | \eta' |^M \phi' \left( \frac{\eta'}{| \eta' |} \right)
     \eta_{q + 1}^{n'} \]
  with $( \phi, \phi') \in \mathcal{H}^N ( \mathbbm{S}^{p - 1}) \times
  \mathcal{H}^M ( \mathbbm{S}^{q - 1})$, $( \xi_1, \xi') \assign \xi$, $(
  \eta', \eta_{q + 1}) \assign \eta$ and $m' + M + N + n' \in 2\mathbbm{Z}$.
  
  Now, by lemma \ref{k-finite:lem-restr-opendense} for an embedding
  \begin{eqnarray}
    & \psi_{N \rightarrow S}' : \{ x \in \mathbbm{R}^{p, q} \backslash \{ 0
    \} | Q ( x) = 0 \} \hookrightarrow \{ ( \xi, \eta) \in \mathbbm{S}^p
    \times \mathbbm{S}^q | \xi_1 = \eta_{q + 1} \} &  \nonumber\\
    & I \assign \psi'_{N \rightarrow S} ( \{ x \in \mathbbm{R}^{p, q}
    \backslash \{ 0 \} | Q ( x) = 0 \}) &  \nonumber
  \end{eqnarray}
  we have $\pm I \subset \{ ( \xi, \eta) \in \mathbbm{S}^p \times
  \mathbbm{S}^q | \xi_1 = \eta_{q + 1} \}$ being dense and thus, as
  $K_{\lambda, \nu}^S$ and $F$ are even and for $\tmop{Re} ( \lambda) \gg 0$
  we have $| \xi_{p + 1} |^{\lambda + \nu - n} F$ having its $\nu - 1$'st
  normal derivative at $\{ \xi_1 = \eta_{q + 1} \}$ being continuous, we can
  restrict integration of continuous function to an open dense subset to get
  \[ \langle K_{\lambda, \nu}^S, F \rangle = \langle K_{\lambda, \nu}^S, F
     \rangle_I \]
  Now, the map
  \begin{eqnarray}
    & D : \{ ( x, y) \in ( - 1, 1)^2 | x + y > 0 \} \times \mathbbm{S}^{p -
    1} \times \mathbbm{S}^{q - 1} \ni ( x, y, \omega, \omega') \mapsto ( r
    \omega, s \omega') \in \{ ( x, y) \in \mathbbm{R}^{p, q} | x \neq 0, y
    \neq 0 \} &  \nonumber\\
    & r : = \frac{\sqrt{1 - y^2}}{x + y}, \hspace{1em} s : = \frac{\sqrt{1 -
    x^2}}{x + y} &  \nonumber
  \end{eqnarray}
  provides a diffeomorphism which pulls back $\{ x \in \mathbbm{R}^{p, q}
  \backslash \{ 0 \} | Q ( x) = 0 \}$ to $\{ x = y \} \times \mathbbm{S}^{p -
  1} \times \mathbbm{S}^{q - 1}$ and as $\{ x \in \mathbbm{R}^{p, q}
  \backslash \{ 0 \} | Q ( x) = 0 \} \subseteq \{ ( x, y) \in \mathbbm{R}^{p,
  q} | x \neq 0, y \neq 0 \}$ we may pull back the integration above by $D$ to
  get (note that canonical volume form on $\mathbbm{S}^p \times \mathbbm{S}^q$
  is pulled back to a constant multiple of $| R ( r, s) |^{- n / 2}$)
  \begin{eqnarray}
    & \langle K_{\lambda, \nu}^S, F \rangle_I \simeq \int_{\mathbbm{S}^{p -
    1}} \phi ( \omega) | \omega_p |^{\lambda + \nu - n} d \omega_{} \times
    \int_{\mathbbm{S}^{q - 1}} \phi' ( \omega') d \omega' \times & 
    \nonumber\\
    & \times \int_{\{ ( x, y) \in ( - 1, 1)^2 | x + y > 0 \}} ( 1 - y^2)^{(
    \lambda + \nu + N - q) / 2 - 1} \delta^{( \nu - 1)} ( x - y) [ ( 1 -
    x^2)^{( q - 2) / 2} g ( x, y)] d x d y. &  \nonumber
  \end{eqnarray}
  
  
  One subsequently proceeds as in the proof of proposition
  \ref{k-finite:prop-KR-hook-1}.
\end{proof}

\begin{proof}
  (of prop. \ref{k-finite:prop-KC-hook-wrap}) This result is derived from
  proposition \ref{k-finite:prop-KC-hook-kfinite} in the same way as
  proposition \ref{k-finite:prop-KR-hook-2} is derived from proposition
  \ref{k-finite:prop-KR-hook-1}.
\end{proof}

\section{Normalization of $K_{\lambda, \nu}^C$}\label{sec:KC-normalization}

Here we attempt to normalize kernel $K_{\lambda, \nu}^C$ of SBO supported on
$\{ Q = 0 \}$. We recall that for $\nu \in 2\mathbbm{Z}_{\geqslant 0} + 1$ and
$\lambda \in \{ \lambda - \nu \nin -\mathbbm{Z}_{\geqslant 0} \}$ it was
defined in proposition \ref{supp-Q:prop-sol-extending} as continuation of
product of distributions
\[ \frac{| x_p |^{\lambda + \nu - n}}{\Gamma ( ( \lambda + \nu - n \upl 1) /
   2)} \cdot \delta^{( \nu - 1)} ( Q) \in \mathcal{D}' ( \mathbbm{R}^n
   \backslash \{ 0 \}) \]
to $\mathbbm{R}^n$ given by fact \ref{holomorphicity-preserving:fact-homog},
the product being well-defined by the same reasoning as in proof of
proposition \ref{supp-R:prop-3}.

We shall also employ the notion of $\mathfrak{P} ( \cdot)$ as given in
definition \ref{def-P} and related notions.

\subsection{Main results}

\begin{proposition}
  \label{KC-normalization-2:prop-normalization}For $\nu \in
  2\mathbbm{Z}_{\geqslant 0} + 1$ fixed $\tilde{K}_{\lambda, \nu}^C \assign
  K_{\lambda, \nu} / N$ (extends to) holomorphic nonzero distribution in
  $\lambda \in \mathbbm{C}$, where
  \[ N \assign \left\{ \begin{array}{ll}
       \Gamma \left( \frac{\lambda - \nu}{2} \right), & q \in 2\mathbbm{Z}+
       1\\
       \Gamma \left( \frac{\lambda - \min \{ \nu, q - \nu \}}{2} \right) & q
       \in 2\mathbbm{Z}, \; p = 1\\
       1, & q \in 2\mathbbm{Z}, p > 1.
     \end{array} \right. \]
\end{proposition}

\begin{proposition}
  \label{KC-normalization-2:prop-supp}For $( \lambda, \nu) \in \mathbbm{C}
  \times ( 2\mathbbm{Z}_{\geqslant 0} + 1)$ and $\tilde{K}_{\lambda, \nu}^C$
  as defined in proposition \ref{KC-normalization-2:prop-normalization}
  support of $\tilde{K}_{\lambda, \nu}^C$ is given as
  \[ = \left\{ \begin{array}{lll}
       \{ 0 \}, & p = 1, q \in 2\mathbbm{Z}+ 1, & \lambda - \nu \in -
       2\mathbbm{Z}_{\geqslant 0}\\
       \{ Q = 0 \}, & p = 1, q \in 2\mathbbm{Z}+ 1, & \lambda - \nu \nin -
       2\mathbbm{Z}_{\geqslant 0}\\
       \{ 0 \}, & p = 1, q \in 2\mathbbm{Z}, & \lambda - \min \{ \nu, q - \nu
       \} \in - 2\mathbbm{Z}_{\geqslant 0}\\
       \{ Q = 0 \}, & p = 1, q \in 2\mathbbm{Z}, & \lambda - \min \{ \nu, q -
       \nu \} \nin - 2\mathbbm{Z}_{\geqslant 0}\\
       \{ 0 \}, & p > 1, q \in 2\mathbbm{Z}+ 1, & \lambda - \nu \in -
       2\mathbbm{Z}_{\geqslant 0}\\
       \{ Q = 0 \}, & p > 1, q \in 2\mathbbm{Z}_{} + 1, & \lambda - \nu \nin -
       2\mathbbm{Z}_{\geqslant 0}, \lambda + \nu - n \nin -
       2\mathbbm{Z}_{\geqslant 0} - 1\\
       \{ Q = 0 \} \cap \{ x_p = 0 \}, & p > 1, q \in 2\mathbbm{Z}_{} + 1, &
       \lambda - \nu \nin - 2\mathbbm{Z}_{\geqslant 0}, \lambda + \nu - n \in
       - 2\mathbbm{Z}_{\geqslant 0} - 1\\
       \{ Q = 0 \}, & p > 1, q \in 2\mathbbm{Z}, & \lambda + \nu - n \nin -
       2\mathbbm{Z}_{\geqslant 0} - 1\\
       \{ Q = 0 \} \cap \{ x_p = 0 \}, & p > 1, q \in 2\mathbbm{Z}, & \lambda
       + \nu - n \in - 2\mathbbm{Z}_{\geqslant 0} - 1
     \end{array} \right. \]
\end{proposition}

\subsection{Auxiliary results}

\begin{lemma}
  \label{KC-normalization-2:lem-kfin-00}For fixed $\nu \in
  2\mathbbm{Z}_{\geqslant 0} + 1$, $\lambda \in \mathbbm{C}$ with $\tmop{Re} (
  \lambda) \gg 0$ and $q \in \mathbbm{Z}_{\geqslant 0}$ we have
  \begin{eqnarray*}
    & \int_{- 1}^1 ( 1 - y^2)^{( \lambda + \nu - q) / 2 - 1} \left(
    \frac{d^{\nu - 1}}{d y^{\nu - 1}} ( 1 - y^2)^{( q - 2) / 2} \right) d y =
    \frac{\sqrt{\pi}}{2^{( \nu - 1) / 2}} \times \frac{\Gamma \left(
    \frac{\lambda - \nu}{2} \right)}{\Gamma \left( \frac{\lambda}{2} \right)}
    \times ( ( q - 2))_{( \nu - 1) / 2} \times & \\
    & \times ( ( q - \lambda - 1))_{( \nu - 1) / 2} \times ( \nu - 2)_{( \nu
    - 1) / 2}, & 
  \end{eqnarray*}
  where $( ( x))_n \assign x ( x - 2) \ldots ( x - 2 n + 2)$.
\end{lemma}

\begin{proof}
  We proceed by induction on $\nu$, case $\nu = 1$ being obvious. Using the
  equality
  \[ \frac{d^2}{d x^2} ( 1 - x^2)^l = 2 l ( 2 l - 2) ( 1 - x^2)^{l - 2} - 2 l
     ( 2 l - 1) ( 1 - x^2)^{l - 1} \]
  and integration by parts, we then have the following chain of equalities
  valid for $\tmop{Re} ( \lambda) \gg 0$
  \begin{eqnarray*}
    & \int_{- 1}^1 ( 1 - y^2)^{( \lambda + ( \nu + 2) - q) / 2 - 1} \left(
    \frac{d^{\nu + 1}}{d y^{\nu + 1}} ( 1 - y^2)^{( q - 2) / 2} \right) = & \\
    & = \int_{- 1}^1 \frac{d^2}{d \nu^2} ( 1 - y^2)^{( \lambda + ( \nu + 2) -
    q) / 2 - 1} \left( \frac{d^{\nu - 1}}{d y^{\nu - 1}} ( 1 - y^2)^{( q - 2)
    / 2} \right) = & \\
    & = ( \lambda + \nu - q) ( \lambda + \nu - q - 2) \int_{- 1}^1 ( 1 -
    y^2)^{( ( \lambda - 2) + \nu - q) / 2 - 1} \left( \frac{d^{\nu - 1}}{d
    y^{\nu - 1}} ( 1 - y^2)^{( q - 2) / 2} \right) - & \\
    & - ( \lambda + \nu - q) ( \lambda + \nu - q - 1) \int_{- 1}^1 ( 1 -
    y^2)^{( \lambda + \nu - q) / 2 - 1} \left( \frac{d^{\nu - 1}}{d y^{\nu -
    1}} ( 1 - y^2)^{( q - 2) / 2} \right) = & 
  \end{eqnarray*}
  and the induction assumption allows to rewrite this as
  \begin{eqnarray*}
    & = ( \lambda + \nu - q) ( \lambda + \nu - q - 2) \frac{\sqrt{\pi}}{2^{(
    \nu - 1) / 2}} \cdot \frac{\Gamma \left( \frac{\lambda - 2 - \nu}{2}
    \right)}{\Gamma \left( \frac{\lambda - 2}{2} \right)} ( ( q - 2))_{( \nu -
    1) / 2} ( ( q - \lambda + 1))_{( \nu - 1) / 2} ( \nu - 2)_{( \nu - 1) / 2}
    - & \\
    & - ( \lambda + \nu - q) ( \lambda + \nu - q - 1) \frac{\sqrt{\pi}}{2^{(
    \nu - 1) / 2}} \cdot \frac{\Gamma \left( \frac{\lambda - \nu}{2}
    \right)}{\Gamma \left( \frac{\lambda}{2} \right)} ( ( q - 2))_{( \nu - 1)
    / 2} ( ( q - \lambda - 1))_{( \nu - 1) / 2} ( \nu - 2)_{( \nu - 1) / 2} =
    & 
  \end{eqnarray*}
  and using the obvious equalities $( ( x))_{n + 1} / ( x - 2 n) = ( ( x))_n$,
  $( ( x))_{n + 1} / x = ( ( x - 2))_n$ and $\Gamma ( x + 1) = x \Gamma ( x)$
  this gets further rewritten as
  \begin{eqnarray*}
    & = \frac{\sqrt{\pi}}{2^{( \nu + 1) / 2}} \cdot \frac{\Gamma \left(
    \frac{\lambda - 2 - \nu}{2} \right)}{\Gamma \left( \frac{\lambda}{2}
    \right)} ( ( q - 2))_{( \nu + 1) / 2} ( ( q - \lambda - 1))_{( \nu + 1) /
    2} ( ( \nu))_{( \nu + 1) / 2} \times & \\
    & \times \frac{1}{q - \nu - 1} \cdot \frac{1}{\nu} \left( ( \lambda + \nu
    - q) ( \lambda + \nu - q - 2) 2 \frac{\lambda - 2}{2} \cdot \frac{q -
    \lambda + 1}{( q - \lambda + 1 - ( \nu - 1)) ( q - \lambda - 1 - ( \nu -
    1))} - \right. & \\
    & - \left. ( \lambda + \nu - q) ( \lambda + \nu - q - 1) 2 \frac{\lambda
    - 2 - \nu}{2} \cdot \frac{1}{q - \lambda - \nu} \right) & 
  \end{eqnarray*}
  and it suffices now to show that part after $\times$ is equal to 1. But the
  latter expression is equal to
  \begin{eqnarray*}
    & \frac{1}{( q - \nu - 1) \nu} ( ( \lambda - 2) ( q - \lambda + 1) + (
    \lambda - 2 - \nu) ( \lambda + \nu - q - 1)) = 1. & 
  \end{eqnarray*}
\end{proof}

\begin{lemma}
  \label{KC-normalization-2:lem-kfin-11}For fixed $\nu \in
  2\mathbbm{Z}_{\geqslant 0} + 1$, $\lambda \in \mathbbm{C}$ with $\tmop{Re} (
  \lambda) \gg 0$ and $q \in \mathbbm{Z}_{\geqslant 0}$ we have
  \begin{eqnarray*}
    & \int_{- 1}^1 ( 1 - y^2)^{( \lambda + \nu - q) / 2 - 1} y \left(
    \frac{d^{\nu - 1}}{d y^{\nu - 1}} [ y ( 1 - y^2)^{( q - 2) / 2}] \right) d
    y = \frac{\sqrt{\pi}}{2^{( \nu + 1) / 2}} \times \frac{\Gamma \left(
    \frac{\lambda - \nu}{2} \right)}{\Gamma \left( \frac{\lambda + 2}{2}
    \right)} \times ( ( q - 2))_{( \nu - 1) / 2} \times & \\
    & \times ( ( q - \lambda - 1))_{( \nu - 1) / 2} \times ( \nu)_{( \nu - 1)
    / 2}, & 
  \end{eqnarray*}
  where $( ( x))_n \assign x ( x - 2) \ldots ( x - 2 n + 2)$.
\end{lemma}

\begin{proof}
  The prove proceeds by induction on $\nu$, just as that of lemma
  \ref{KC-normalization-2:lem-kfin-00}, so we just outline the main
  differences. We use integration by parts, formula
  \[ \frac{d^2}{d x^2} [ x ( 1 - x^2)^l] = 2 l ( 2 l - 2) x ( 1 - x^2)^{l - 2}
     - 2 l ( 2 l + 1) x ( 1 - x^2)^{l - 1}, \]
  induction hypothesis and basic properties of $( ( x))_n$ and $\Gamma ( x)$
  to write
  \begin{eqnarray*}
    & \int_{- 1}^1 K_{\lambda, \nu + 2, q} \assign \int_{- 1}^1 ( 1 - y^2)^{(
    \lambda + ( \nu + 2) - q) / 2 - 1} \left( \frac{d^{\nu + 1}}{d y^{\nu +
    1}} [ y ( 1 - y^2)^{( q - 2) / 2}] \right) = & \\
    & = ( \lambda + \nu - q) ( \lambda + \nu - q - 2) \int_{- 1}^1 K_{\lambda
    - 2, \nu, q} - ( \lambda + \nu - q) ( \lambda + \nu - q + 1) \int_{- 1}^1
    K_{\lambda, \nu, q} = & \\
    & = ( \lambda + \nu - q) ( \lambda + \nu - q - 2) \frac{\sqrt{\pi}}{2^{(
    \nu + 1) / 2}} \cdot \frac{\Gamma \left( \frac{\lambda - 2 - \nu}{2}
    \right)}{\Gamma \left( \frac{\lambda}{2} \right)} ( ( q - 2))_{( \nu - 1)
    / 2} ( ( q - \lambda + 1))_{( \nu - 1) / 2} \times ( \nu)_{( \nu - 1) / 2}
    - & \\
    & - ( \lambda + \nu - q) ( \lambda + \nu - q + 1) \frac{\sqrt{\pi}}{2^{(
    \nu + 1) / 2}} \cdot \frac{\Gamma \left( \frac{\lambda - \nu}{2}
    \right)}{\Gamma \left( \frac{\lambda + 2}{2} \right)} ( ( q - 2))_{( \nu -
    1) / 2} ( ( q - \lambda - 1))_{( \nu - 1) / 2} \times ( \nu)_{( \nu - 1) /
    2} = & \\
    & = \frac{\sqrt{\pi}}{2^{( \nu + 3) / 2}} \cdot \frac{\Gamma \left(
    \frac{\lambda - \nu - 2}{2} \right)}{\Gamma \left( \frac{\lambda + 2}{2}
    \right)} ( ( q - 2))_{( \nu + 1) / 2} ( ( q - \lambda - 1))_{( \nu + 1) /
    2} ( \nu + 2)_{( \nu + 1) / 2} \times & \\
    & \times \frac{1}{q - \nu - 1} \cdot \frac{1}{\nu + 2} \left( 2 ( \lambda
    + \nu - q) ( \lambda + \nu - q - 2) \frac{\lambda}{2} \cdot \frac{q -
    \lambda + 1}{( q - \lambda - \nu) ( q - \lambda - \nu + 2)} - \right. & \\
    & \left. - 2 ( \lambda + \nu - q) ( \lambda + \nu - q + 1) \frac{\lambda
    - \nu - 2}{2} \cdot \frac{1}{q - \lambda - \nu} \right) & 
  \end{eqnarray*}
  and it suffices to show that part after $\times$ is equal to 1. But it can
  be rewritten as
  \[ \frac{1}{q - \nu - 1} \cdot \frac{1}{\nu + 2} ( \lambda ( q - \lambda +
     1) + ( \lambda + \nu - q + 1) ( \lambda - \nu - 2)) = \frac{( \nu + 2) (
     q - \nu - 1)}{( q - \nu - 1) ( \nu + 2)} = 1. \]
\end{proof}

\begin{lemma}
  \label{KC-normalization-2:lem-kfinite}With $\varphi_N [ g] ( \lambda)$ and
  $N_0$ as introduced in propositions \ref{k-finite:prop-KC-hook-kfinite} and
  \ref{k-finite:prop-KC-hook-wrap} respectively, we have for $N \in
  2\mathbbm{Z}_{\geqslant 0}$, $m', n' \in \mathbbm{Z}_{\geqslant 0}$ and $i =
  0, 1$
  \begin{eqnarray*}
    & \tilde{\varphi}_{m', n', N, i} : = \frac{\varphi_N [ ( x y)^i ( 1 -
    x^2)^{n'} ( 1 - y^2)^{m'}]}{N_0} \simeq R \cdot S, & \\
    & S \assign \left\{ \begin{array}{ll}
      1, & p = 1, N = 0,\\
      \Gamma \left( \frac{\lambda + \nu - n + 2}{2} \right) / \Gamma \left(
      \frac{\lambda + \nu - n - N + 2}{2} \right) / \Gamma \left(
      \frac{\lambda - q + \nu + N}{2} \right), & p > 1,\\
      0, & p = 1, N \geqslant 2,
    \end{array} \right. & \\
    & R \assign ( ( q + 2 n' - 2))_{( \nu - 1) / 2} ( ( q - 2 m' - \lambda -
    N - 1))_{( \nu - 1) / 2} \Gamma \left( \frac{\lambda + N + 2 m' + 2 n' -
    \nu}{2} \right) \times & \\
    & \times \left\{ \begin{array}{ll}
      \Gamma^{- 1} \left( \frac{\lambda + N + 2 m' + 2 n'}{2} \right), & i =
      0\\
      \Gamma^{- 1} \left( \frac{\lambda + N + 2 m' + 2 n' + 2}{2} \right), & i
      = 1,
    \end{array} \right. & 
  \end{eqnarray*}
  where $( ( x))_n \assign x ( x - 2) \ldots ( x - 2 n + 2)$.
\end{lemma}

\begin{proof}
  This follows from proposition \ref{k-finite:prop-KC-hook-kfinite} and lemmas
  \ref{KC-normalization-2:lem-kfin-00} and
  \ref{KC-normalization-2:lem-kfin-11}, as one sees that
  \begin{eqnarray*}
    & \int_{- 1}^1 (1 - y^2)^{\frac{\lambda + N + \nu - q - 2}{2}} y^i  ( 1 -
    y^2)^{m'} \left[ \frac{\partial^{\nu - 1}}{\partial x^{\nu - 1}} |_{x = y}
    x^i (1 - x^2)^{\frac{q - 2}{2}}  ( 1 - x^2)^{n'} \right] dy = & \\
    & \int_{- 1}^1 (1 - y^2)^{\frac{\lambda' + \nu - q' - 2}{2}} y^i \left[
    \frac{\partial^{\nu - 1}}{\partial x^{\nu - 1}} |_{x = y} x^i (1 -
    x^2)^{\frac{q' - 2}{2}}  \right] ; & \\
    & q' \assign q + 2 n', \hspace{1em} \lambda' \assign \lambda + N + 2 m' +
    2 n' . & 
  \end{eqnarray*}
\end{proof}

\begin{lemma}
  \label{KC-normalization-2:lem-Pcomputation}Fix $\nu \in
  2\mathbbm{Z}_{\geqslant 0} + 1$. For $( m', n', N, i) \in
  \mathbbm{Z}_{\geqslant 0}^2 \times ( 2\mathbbm{Z}_{\geqslant 0}) \times \{
  0, 1 \}$ and $\tilde{\varphi}_{m', n', N, i}$ as in lemma
  \ref{KC-normalization-2:lem-kfinite} we have
  \begin{eqnarray}
    & \mathfrak{P} ( \tilde{\varphi}_{m', n', N, i}) = &  \nonumber\\
    & \left\{ \begin{array}{ll}
      - A^{}_{m', n', N, i} - B^{}_{m', n', N, i} - & \\
      - C^{}_{m', n', N, i} - D^{}_{m', n', N, i} + E^{}_{m', n', N, i}, & p >
      1, q \in 2\mathbbm{Z}_{} + 1\\
      - C^{}_{m', n', N, i} - D^{}_{m', n', N, i} + E^{}_{m', n', N, i}, & p =
      1, \; N = 0\\
      - \infty, & p = 1, \; N > 0\\
      - A^{}_{m', n', N, i} - B^{}_{m', n', N, i} - & \\
      - C^{}_{m', n', N, i} - D^{}_{m', n', N, i} + E^{}_{m', n', N, i}, & p >
      1, \; q \in 2\mathbbm{Z}, \; n' > \frac{\nu - q - 1}{2}\\
      - \infty & p > 1, \; q \in 2\mathbbm{Z}, \; n' \leqslant \frac{\nu - q -
      1}{2}
    \end{array} \right. &  \nonumber\\
    & A^{}_{m', n', N, i} \assign -\mathfrak{P} \left( \frac{\Gamma \left(
    \frac{\lambda + \nu - n + 2}{2} \right)}{\Gamma \left( \frac{\lambda + \nu
    - n - N + 2}{2} \right)} \right) = \{ \lambda = n - \nu + 2 j \}_{j =
    0}^{( N - 2) / 2} &  \nonumber\\
    & B^{}_{m', n', N, i} \assign -\mathfrak{P} \left( \Gamma^{- 1} \left(
    \frac{\lambda - q + \nu + N}{2} \right) \right) = \{ \lambda \in q - \nu -
    N - 2\mathbbm{Z}_{\geqslant 0} \} &  \nonumber\\
    & C^{}_{m', n', N, i} \assign -\mathfrak{P} ( ( ( q - 2 m' - 2 n' -
    \lambda - N - 1))_{( \nu - 1) / 2}) = \{ \lambda = q - 2 m' - N - 1 - 2 j
    \}_{j = 0}^{( \nu - 3) / 2} &  \nonumber\\
    & D^{}_{m', n', N, i} \assign -\mathfrak{P} \left( \Gamma^{- 1} \left(
    \frac{\lambda + N + 2 m' + 2 n' + 2 i}{2} \right) \right) = \{ \lambda \in
    - N - 2 m' - 2 n' - 2 i - 2\mathbbm{Z}_{\geqslant 0} \} &  \nonumber\\
    & E_{m', n', N, i} =\mathfrak{P} \left( \Gamma \left( \frac{\lambda + N +
    2 m' + 2 n' - \nu}{2} \right) \right) = \{ \lambda \in - N - 2 m' - 2 n' +
    \nu - 2\mathbbm{Z}_{\geqslant 0} \} &  \nonumber
  \end{eqnarray}
\end{lemma}

\begin{proof}
  The lemma follows directly from the formulae in lemma
  \ref{KC-normalization-2:lem-kfinite} once one recalls that $\mathfrak{P} (
  \Gamma ( x) / \Gamma ( x - k)) = - \{ x = j \}_{j = 1}^k$ and $\mathfrak{P}
  ( ( ( x))_k) = - \{ x = 2 j \}_{j = 0}^{k - 1}$. One thing to notice is that
  \[ ( ( q + 2 n' - 2))_{( \nu - 1) / 2} = 0 \Leftrightarrow q \in
     2\mathbbm{Z}_{\geqslant 1} \hspace{1em} \tmop{and} \hspace{1em} q + 2 n'
     - \nu + 1 \leqslant 0. \]
\end{proof}

\begin{lemma}
  \label{KC-normalization-2:lem-normalization-qodd-aux-1}For $q \in
  2\mathbbm{Z}+ 1$ and $p \in 2\mathbbm{Z}_{\geqslant 1}$ with notation as in
  lemma \ref{KC-normalization-2:lem-Pcomputation} we have
  \[ \min_{( m', n', N, i) \in \mathfrak{I}} \{ C_{m', n', 0, i} + D_{m', n',
     0' i} + A_{m', n', N, i} + B_{m', n', N, i} \} = \varnothing \]
\end{lemma}

\begin{proof}
  Indeed, letting $\mathfrak{I} \assign \mathbbm{Z}_{\geqslant 0}^2 \times (
  2\mathbbm{Z}_{\geqslant 0}) \times \{ 0, 1 \}$ we have
  \begin{eqnarray}
    & \min_{( m', n', N, i) \in \mathfrak{I}} \{ C_{m', n', 0, i} + D_{m',
    n', 0' i} + A_{m', n', N, i} + B_{m', n', N, i} \} = &  \nonumber\\
    & = - \bigcap_{( m', n', N, i) \in \mathfrak{I}} ( \{ \lambda = q - 2 m'
    - N - 1 - 2 j \}_{j = 0}^{( \nu - 3) / 2} \cup \{ \lambda \in - N - 2 m' -
    2 n' - 2 i - 2\mathbbm{Z}_{\geqslant 0} \} \nobracket \cup &  \nonumber\\
    & \cup \{ \lambda \in q - \nu - N - 2\mathbbm{Z}_{\geqslant 0} \}
    \nobracket \cup \{ \lambda = n - \nu + 2 j \}_{j = 0}^{( N - 2) / 2}) = & 
    \nonumber\\
    & = \bigcap_{N \in 2\mathbbm{Z}_{\geqslant 0}} ( \{ \lambda \in q - \nu -
    N - 2\mathbbm{Z}_{\geqslant 0} \} \cup \{ \lambda = n - \nu + 2 j \}_{j =
    0}^{( N - 2) / 2} \cup \nobracket &  \nonumber\\
    & \cup \left. \bigcap_{( m', n', i) \in \mathbbm{Z}_{\geqslant 0}^2
    \times \{ 0, 1 \}} ( \{ \lambda = q - 2 m' - N - 1 - 2 j \}_{j = 0}^{( \nu
    - 3) / 2} \cup \{ \lambda \in - N - 2 m' - 2 n' - 2 i -
    2\mathbbm{Z}_{\geqslant 0} \}) \right) = &  \nonumber\\
    & = \bigcap_{N \in 2\mathbbm{Z}_{\geqslant 0}} ( \{ \lambda \in q - \nu -
    N - 2\mathbbm{Z}_{\geqslant 0} \} \cup \{ \lambda = n - \nu + 2 j \}_{j =
    0}^{( N - 2) / 2}) . &  \nonumber
  \end{eqnarray}
  Now, as we have $q - \nu < n - \nu$, the first item of lemma
  \ref{P-def:lem-threeset} implies that latter equals to
  \[ = \bigcap_{N 2\mathbbm{Z}_{\geqslant 0}} ( \{ \lambda \in q - \nu - N -
     2\mathbbm{Z}_{\geqslant 0} \}) \cup \bigcap_{\Nu 2\mathbbm{Z}_{\geqslant
     0}} \{ \lambda = n - \nu + 2 j \}_{j = 0}^{( N - 2) / 2} = \varnothing
     \cup \varnothing = \varnothing . \]
\end{proof}

\begin{lemma}
  \label{KC-normalization-2:lem-normalization-qodd}Proposition
  \ref{KC-normalization-2:prop-normalization} holds for $q \in 2\mathbbm{Z}+
  1$.
\end{lemma}

\begin{proof}
  Fix $\nu \in 2\mathbbm{Z}_{\geqslant 0} + 1$ and let $\mathfrak{I} \assign
  \mathbbm{Z}_{\geqslant 0}^2 \times ( 2\mathbbm{Z}_{\geqslant 0}) \times \{
  0, 1 \}$ and $\mathfrak{I}_0 \assign \mathbbm{Z}_{\geqslant 0}^2 \times \{
  0, 1 \}$ to facitilate the notation. As we need to show that $K_{\lambda,
  \nu}^C / N$ is holomorphic and nonzero, proposition
  \ref{k-finite:prop-KC-hook-wrap} tells us that it suffices to show that
  $\tilde{\varphi}_{m', n', N, i} / N$ are holomorphic and do not vanish
  simultaneously with $\tilde{\varphi}_{m', n', N, i}$ as in lemma
  \ref{KC-normalization-2:lem-kfinite}.
  
  Employing $\mathfrak{P} ( \cdot)$-notation, in the light of proposition
  \ref{k-finite:prop-KC-hook-wrap} it suffices to show that
  \[ \max_{( m', n', N, i) \in \mathfrak{I}} \{ \mathfrak{P} (
     \tilde{\varphi}_{m', n', N, i} / N) \}_{} = \varnothing . \]
  For this it suffices to compute
  \[ \max_{( m', n', N, i) \in \mathfrak{I}} \{ \mathfrak{P} (
     \tilde{\varphi}_{m', n', N, i}) \}_{} . \]
  Moreover, we note that lemma \ref{supp-Q:prop-supp-xnoq0} implies in
  particular that $K_{\lambda, \nu}^C$ is nonzero for $( \lambda, \nu) \in
  \mathbbm{C} \times ( 2\mathbbm{Z}_{\geqslant 0} + 1)$, hence prosition
  \ref{k-finite:prop-KC-hook-wrap} (with $R = 1$) implies that
  $\tilde{\varphi}_{m', n', N, i}$ do not vanish simultaneously and therefore
  one can compute $\mathfrak{P}_+$ instead of $\mathfrak{P}$.
  
  We first treat $p = 1$ case. With reference to formulae of lemma
  \ref{KC-normalization-2:lem-Pcomputation}, we note that under the assumption
  $q \in 2\mathbbm{Z}+ 1$ and $p = 1$ we have $\tmop{supp} ( C_{m', n', N,
  i}), \tmop{supp} ( D_{m', n', N, i}) \subset \lambda + 2\mathbbm{Z}$ and
  $\tmop{supp} ( E_{m', n', N, i}) \subset \lambda + 2\mathbbm{Z}+ 1$, hence
  we have (we use $f_+$ notation to denote positive part of $f : \mathbbm{C}
  \rightarrow \mathbbm{Z}$)
  \begin{eqnarray}
    & \max_{} \{ \mathfrak{P}_+ ( \langle K_{\lambda, \nu}^C, F_{N, 0} [
    \psi_N, 1 ; ( s t)^i ( 1 - s^2)^{n'} ( 1 - t^2)^{m'}] \rangle) \}_{( m',
    n', N, i) \in \mathfrak{I}} = &  \nonumber\\
    & = \max_{( m', n', i) \in \mathfrak{I}_0} ( - C_{m', n', 0, i} - D_{m',
    n', 0' i})_+ + \max_{( m', n', i) \in \mathfrak{I}} \{ E_{m', n', 0' i} \}
    = &  \nonumber\\
    & = \max_{( m', n', i) \in \mathfrak{I}_0} 0 + \bigcup_{( m', n', i) \in
    \mathfrak{I}_0} \{ \lambda \in - N - 2 m' - 2 n' + \nu -
    2\mathbbm{Z}_{\geqslant 0} \} = - \varnothing + \{ \lambda \in \nu -
    2\mathbbm{Z}_{\geqslant 0} \} &  \nonumber
  \end{eqnarray}
  which gives the answer in the $p = 1$ case.
  
  Next, we treat $p \in 2\mathbbm{Z}_{\geqslant 1}$ case. Similarly, to above,
  we have $C_{m', n', N, i}$, $D_{m', n', N, i}$, $A_{m', n', N, i}$ and
  $B_{m', n', N, i}$ vanishing outside $\lambda + 2\mathbbm{Z}$, while $E_{m',
  n', N, i}$ vanishes outside $\lambda + 2\mathbbm{Z}+ 1$. Hence, applying
  lemma \ref{KC-normalization-2:lem-normalization-qodd-aux-1} we have
  \begin{eqnarray}
    & \max_{} \{ \mathfrak{P}_+ ( \tilde{\varphi}_{m', n', N, i}) \}_{( m',
    n', N, i) \in \mathfrak{I}} = &  \nonumber\\
    & = \max_{( m', n', N, i) \in \mathfrak{I}} ( - C_{m', n', 0, i} - D_{m',
    n', 0' i} - A_{m', n', N, i} - B_{m', n', N, i})_+ + \max_{( m', n', i)
    \in \mathfrak{I}} \{ E_{m', n', 0' i} \} = &  \nonumber\\
    & = - \varnothing + \{ \lambda \in \nu - 2\mathbbm{Z}_{\geqslant 0} \} & 
    \nonumber
  \end{eqnarray}
  Finally, case $p \in 2\mathbbm{Z}_{\geqslant 1} + 1$ is treated similarly,
  only this time we have $A_{m', n', N, i}$ vanishing outside $\lambda +
  2\mathbbm{Z}+ 1$ and hence
  \begin{eqnarray}
    & \max_{} \{ \mathfrak{P}_+ ( \tilde{\varphi}_{m', n', N, i}) \}_{( m',
    n', N, i) \in \mathfrak{I}} = &  \nonumber\\
    & = \max_{( m', n', N, i) \in \mathfrak{I}} ( - C_{m', n', 0, i} - D_{m',
    n', 0' i} - B_{m', n', N, i})_+ + \max_{( m', n', i) \in \mathfrak{I}} \{
    E_{m', n', 0' i} - A_{m', n', N, i} \} = &  \nonumber\\
    & = - \varnothing + \{ \lambda \in \nu - 2\mathbbm{Z}_{\geqslant 0} \} .
    &  \nonumber
  \end{eqnarray}
\end{proof}

\begin{lemma}
  \label{KC-normalization-2:lem-normalization-q-even-p-1}Proposition
  \ref{KC-normalization-2:prop-normalization} holds for $q \in 2\mathbbm{Z}$
  and $p = 1$.
\end{lemma}

\begin{proof}
  We proceed as in proof of lemma
  \ref{KC-normalization-2:lem-normalization-qodd}. Under the assumptions taken
  on $( p, q)$ we have $D_{m', n', N, i} \subset \lambda + 2\mathbbm{Z}$,
  while $C_{m', n', N, i}, E_{m', n', N, i} \subset \lambda + 2\mathbbm{Z}+
  1$.
  
  Therefore, as in proof of lemma
  \ref{KC-normalization-2:lem-normalization-qodd}, it reduces to one having to
  compute for $\mathfrak{I} \assign \mathbbm{Z}_{\geqslant 0} \times \left(
  \max \left\{ \frac{\nu - q + 1}{2}, 0 \right\} +\mathbbm{Z}_{\geqslant 0}
  \right)$ and $f_+$ denoting the positive part of $f_+$, the result of
  \begin{eqnarray}
    & \max_{( m', n') \in \mathfrak{I}} ( \{ \lambda \in - 2 m' - 2 n' + \nu
    - 2\mathbbm{Z}_{\geqslant 0} \} - \{ \lambda = q - 2 m' - 1 - 2 j \}_{j =
    0}^{( \nu - 3) / 2})_+ = &  \nonumber\\
    & = \bigcup_{( m', n') \in \mathfrak{I}} ( \{ \lambda \in - 2 m' - 2 n' +
    \nu - 2\mathbbm{Z}_{\geqslant 0} \} \backslash \{ \lambda = q - 2 m' - 1 -
    2 j \}_{j = 0}^{( \nu - 3) / 2}) = &  \nonumber\\
    & = \bigcup_{m' = 0}^{\infty} \bigcup_{n' = \max \left\{ \frac{\nu - q +
    1}{2}, 0 \right\}}^{\infty} ( \{ \lambda \in - 2 m' - 2 n' + \nu -
    2\mathbbm{Z}_{\geqslant 0} \} \backslash \{ \lambda = - 2 m' + q - 1 - 2 j
    \}_{j = 0}^{( \nu - 3) / 2}) = &  \nonumber\\
    & = \bigcup_{m' = 0}^{\infty} ( \{ \lambda \in \min \{ q - \nu - 1, 0 \}
    + \nu - 2 m' - 2\mathbbm{Z}_{\geqslant 0} \} \backslash \{ \lambda = - 2
    m' + q - \nu + 2 + 2 j \}_{j = 0}^{( \nu - 3) / 2})^{} = &  \nonumber
  \end{eqnarray}
  and as $\min \{ q - \nu - 1, 0 \} + \nu \leqslant q - 1 = q - \nu + 2 + (
  \nu - 3)$, the latter equals to
  \[ = \bigcup_{m' = 0}^{\infty} \{ \lambda \in - 2 m' + \min \{ q - 1, \nu, q
     - \nu \} - 2\mathbbm{Z}_{\geqslant 0} \} = \{ \lambda \in \min \{ q - 1,
     \nu, q - \nu \} - 2\mathbbm{Z}_{\geqslant 0} \} . \]
  and as $\min \{ q - 1, \nu, q - \nu \} = \min \{ q - 1, \nu \}$, we are
  done.
\end{proof}

\begin{lemma}
  \label{KC-normalization-2:lem-normalization-q-even-p-odd}Proposition
  \ref{KC-normalization-2:prop-normalization} holds for $q \in 2\mathbbm{Z}$
  and $p > 1$.
\end{lemma}

\begin{proof}
  We first do the proof for $p \in 2\mathbbm{Z}+ 1$. The proof goes similarly
  to that of lemma \ref{KC-normalization-2:lem-normalization-q-even-p-1}. We
  have $A_{m', n', N, i}, \; D_{m', n', N, i} \subset \lambda + 2\mathbbm{Z}$,
  while $C_{m', n', N, i}, E_{m', n', N, i}, D_{m', n', N, i} \subset \lambda
  + 2\mathbbm{Z}+ 1$. Thus we need to compute for $\mathfrak{I} \assign
  \mathbbm{Z}_{\geqslant 0} \times \left( \max \left\{ \frac{\nu - q + 1}{2},
  0 \right\} +\mathbbm{Z}_{\geqslant 0} \right) \times 2\mathbbm{Z}_{\geqslant
  0}$
  \begin{eqnarray}
    & \bigcup_{( m', n', N) \in \mathfrak{I}} \{ \lambda \in - 2 n' - N + \nu
    - 2 m' - 2\mathbbm{Z}_{\geqslant 0} \} \backslash &  \nonumber\\
    & ( \{ \lambda = - 2 m' + q - N - \nu + 2 + 2 j \}_{j = 0}^{( \nu - 3) /
    2} \cup \{ \lambda \in q - \nu - N - 2\mathbbm{Z}_{\geqslant 0} \}) = & 
    \nonumber\\
    & = \bigcup_{( m', N) \in \mathbbm{Z}_{\geqslant 0} \times
    2\mathbbm{Z}_{\geqslant 0}} \{ \lambda \in - N - 2 m' + \min \{ q - 1,
    \nu, q - \nu \} - 2\mathbbm{Z}_{\geqslant 0} \} \backslash \{ \lambda \in
    q - \nu - N - 2\mathbbm{Z}_{\geqslant 0} \} = &  \nonumber\\
    & = \bigcup_{N \in 2\mathbbm{Z}_{\geqslant 0}} \{ \lambda \in - N + \min
    \{ \nu, q - \nu \} - 2\mathbbm{Z}_{\geqslant 0} \} \backslash \{ \lambda
    \in q - \nu - N - 2\mathbbm{Z}_{\geqslant 0} \} = \varnothing . & 
    \nonumber
  \end{eqnarray}
  The case $p \in 2\mathbbm{Z}$ is then handled similarly.
\end{proof}

\subsection{Proofs}

\begin{proof}
  (of prop. \ref{KC-normalization-2:prop-normalization}) Follows from lemmas
  \ref{KC-normalization-2:lem-normalization-qodd},
  \ref{KC-normalization-2:lem-normalization-q-even-p-1} and
  \ref{KC-normalization-2:lem-normalization-q-even-p-odd}.
\end{proof}

\begin{proof}
  (of prop. \ref{KC-normalization-2:prop-supp}) Directly follows from prop.
  \ref{supp-Q:prop-supp-xnoq0} and the definition of $\tilde{K}_{\lambda,
  \nu}^C$.
\end{proof}

\section{Normalization of $K_{\lambda, \nu}^P$}\label{sec:KP-normalization}

Recall that in section \ref{sec:supp-P} we constructed a member of
$\mathcal{S} \tmop{ol}_{\{ x_p = 0 \}} ( Q ; \lambda, \nu)$ for $\lambda + \nu
- n = - 1 - 2 k$ with $k \in \mathbbm{Z}_{\geqslant 0}$ holomorphically
depending on $\nu \in \mathbbm{C}\backslash \Lambda_P$ with $\Lambda_P$ being
some discrete set. It was defined as
\begin{equation}
  K_{\lambda, \nu}^P \assign \sum_{i = 0}^k \frac{( - 1)^i ( 2 k) ! (
  \nu)_i}{( 2 k - 2 i) !i!} \delta^{( 2 k - 2 i)} ( x_p) \otimes Q_i,
  \hspace{1em} k \assign \lambda + \nu - n
\end{equation}
with
\[ Q_i \assign \left\{ \begin{array}{ll}
     | \tilde{Q} |^{- \nu - i} & , i \in 2\mathbbm{Z}_{\geqslant 0}\\
     \tmop{sgn} ( \tilde{Q}) | \tilde{Q} |^{- \nu - i} & , i \in
     2\mathbbm{Z}_{\geqslant 0} + 1
   \end{array} \right. \]
and $( \nu)_i \assign \nu ( \nu + 1) \ldots ( \nu + i - 1)$. Here
$\tilde{Q}$ denotes $( p - 1, q)$-quadratic form.

In this section, we will use the data obtained in section \ref{section:q-norm}
to find the precise location of poles of $K_{\lambda, \nu}^C$. This is
equivalent to find normalization of $K_{\lambda, \nu}^C$ which makes it
holomorphic in $\nu \in \mathbbm{C}$ and nowhere vanishing. We will also
determine the support of a highest order term at every pole, this being
equivalent to find support of normalization at every $\nu \in \mathbbm{C}$.

In the light of proposition \ref{sol:prop-holocont}, the normalized kernel
will then be a member of $\mathcal{S} \tmop{ol}_{\{ x_p = 0 \}} (
\mathbbm{R}^{p, q} ; \lambda, \nu)$.

\subsection{Main results}

\begin{proposition}
  \label{KP-normalization-2:prop-1}Let $p' \assign p - 1 \geqslant 1$ and $n'
  \assign p' + q = n - 1$. For $n' \in 2\mathbbm{Z}+ 1$ we have:
  \[ \mathfrak{P} ( K_{\lambda, \nu}^P) = \left( \frac{n'}{2} - k
     +\mathbbm{Z}_{\geqslant 0} \right) \sqcup [ 1 + 2\mathbbm{Z}_{\geqslant
     0}] \]
  where $[ a]$ means that for pole at $a$, highest term has support equal to
  $\{ x_p = 0 \} \cap \{ Q = 0 \}$, while $( a)$ means that it support is
  equal to $\{ 0 \}$, and $K_{\lambda, \nu}^P$ has support $\{ x_p = 0 \}$ at
  all points $\nu \in \mathbbm{C}$ of holomorphicity.
\end{proposition}

\begin{proposition}
  \label{KP-normalization-2:prop-2}Let $p' \assign p - 1 \geqslant 1$ and $n'
  \assign p' + q = n - 1$. For $p', q, n' / 2 \in 2\mathbbm{Z}$ we have:
  \[ \mathfrak{P} ( K_{\lambda, \nu}^P) = [ 1 + 2\mathbbm{Z}_{\geqslant 0}]
     \sqcup \left\{ \begin{array}{ll}
       ( 2 + 2\mathbbm{Z}_{\geqslant 0}), & k \in 2\mathbbm{Z}_{\geqslant 0},
       k \geqslant n' / 2 - 2\\
       ( n' / 2 - k + 2\mathbbm{Z}_{\geqslant 0}), & k \in
       2\mathbbm{Z}_{\geqslant 0}, k < n' / 2 - 2\\
       ( 2 + 2\mathbbm{Z}_{\geqslant 0}), & k \in 2\mathbbm{Z}_{\geqslant 0} +
       1, ( k - 1) \geqslant n' / 2 - 2\\
       ( n' / 2 - ( k - 1) + 2\mathbbm{Z}_{\geqslant 0}), & k \in
       2\mathbbm{Z}_{\geqslant 0} + 1, ( k - 1) < n' / 2 - 2
     \end{array} \right. \]
\end{proposition}

\begin{proposition}
  \label{KP-normalization-2:prop-3}Let $p' \assign p - 1 \geqslant 1$ and $n'
  \assign p' + q = n - 1$. For $p', q, n' / 2 \in ( 2\mathbbm{Z}_{\geqslant
  0})^2 \times ( 1 + 2\mathbbm{Z}_{\geqslant 0})$ we have:
  \[ \mathfrak{P} ( K_{\lambda, \nu}^P) = [ 1 + 2\mathbbm{Z}_{\geqslant 0}]
     \sqcup \left\{ \begin{array}{ll}
       ( 2 + 2\mathbbm{Z}_{\geqslant 0}), & k \in 2\mathbbm{Z}_{\geqslant 0},
       \frac{n'}{2} + 1 - k < 2\\
       \left( \frac{n'}{2} + 1 - k + 2\mathbbm{Z}_{\geqslant 0} \right), & k
       \in 2\mathbbm{Z}_{\geqslant 0}, \frac{n'}{2} + 1 - k \geqslant 2\\
       ( 2 + 2\mathbbm{Z}_{\geqslant 0}), & k \in 2\mathbbm{Z}_{\geqslant 0} +
       1, n' / 2 - k < 2\\
       \left( \frac{n'}{2} - k + 2\mathbbm{Z}_{\geqslant 0} \right), & k \in
       2\mathbbm{Z}_{\geqslant 0} + 1, n' / 2 - k \geqslant 2
     \end{array} \right. \]
\end{proposition}

\begin{remark}
  The statement of two previous propositions can be summarized as follows:
  
  for $n', q \in 2\mathbbm{Z}$ we have
  \[ \mathfrak{P} ( K_{\lambda, \nu}^P) = [ 1 + 2\mathbbm{Z}_{\geqslant 0}]
     \sqcup \left( \max \left\{ 2, \left( \frac{n - 1}{2} - k \right)'
     \right\} + 2\mathbbm{Z}_{\geqslant 0} \right), \]
  where $\left( \frac{n - 1}{2} - k \right)'$ denotes the smallest even
  integer which is $\geqslant \left( \frac{n - 1}{2} - k \right)$.
\end{remark}

\begin{proposition}
  \label{KP-normalization-2:prop-4}Let $p' \assign p - 1 \geqslant 1$ and $n'
  \assign p' + q = n - 1$. Let also $k'$ be maximal even number that is
  $\leqslant k$. For $p', q, n' / 2 \in ( 2\mathbbm{Z}_{\geqslant 0} + 1)^3$
  we have:
  \[ \mathfrak{P} ( K_{\lambda, \nu}^P) = \left\{ \begin{array}{ll}
       \left[ 1, 3, \ldots, \frac{n'}{2} - k' - 2 \right] \sqcup \left(
       \frac{n'}{2} - k' + 2\mathbbm{Z}_{\geqslant 0} \right)^2, & k'
       \leqslant n' / 2 - 3\\
       \left( \frac{n'}{2} - k', \frac{n'}{2} - k' + 2, \ldots, - 1 \right)
       \sqcup ( 1 + 2\mathbbm{Z}_{\geqslant 0})^2, & k' > n' / 2 - 3
     \end{array} \right. \]
  where upper-index $2$ on the right-hand sides marks double poles.
\end{proposition}

\begin{proposition}
  \label{KP-normalization-2:prop-5}Let $p' \assign p - 1 \geqslant 1$ and $n'
  \assign p' + q = n - 1$. Let also $k'$ be maximal odd number that is
  $\leqslant k$. For $p', q, n' / 2 \in ( 2\mathbbm{Z}_{\geqslant 0} + 1)^{}
  \times ( 2\mathbbm{Z}_{\geqslant 0})$ we have:
  \[ \mathfrak{P} ( K_{\lambda, \nu}^P) = \left\{ \begin{array}{ll}
       \left[ 1, 3, \ldots, \frac{n'}{2} - 2 - k' \right] \sqcup \left(
       \frac{n'}{2} - k' + 2\mathbbm{Z}_{\geqslant 0} \right)^2, & k'
       \leqslant n' / 2 - 3,\\
       \left( \frac{n'}{2} - k', \frac{n'}{2} - k' + 2, \ldots, - 1 \right)
       \sqcup ( 1 + 2\mathbbm{Z}_{\geqslant 0})^2, & k' > n' / 2 - 3
     \end{array} \right. \]
  where upper-indexes $2$ on the right-hand sides indicate double poles.
\end{proposition}

\begin{proposition}
  \label{KP-normalization-2:prop-p=1}For $p = 1$ and $n' : = n - 1$ we have
  \[ \mathfrak{P} ( K_{\lambda, \nu}^P) = \left\{ \begin{array}{ll}
       ( n' / 2 - k +\mathbbm{Z}_{\geqslant 0}), & n \in
       2\mathbbm{Z}_{\geqslant 0}\\
       ( \max \{ 0, n' / 2 - k \} +\mathbbm{Z}_{\geqslant 0}), & n \in
       2\mathbbm{Z}_{\geqslant 0} + 1.
     \end{array} \right. \]
\end{proposition}

\subsection{Auxiliary results}

\begin{lemma}
  \label{KP-normalization-2:lem-holo-supp}At points of holomorphicity
  $K_{\lambda, \nu}^P$ has support equal to $\{ x_p = 0 \}$.
\end{lemma}

\begin{proof}
  In the light of lemma \ref{P-def:lem-delta-times-mero} we see that point of
  holomorphicity of $K_{\lambda, \nu}^P$ should in particular be point of
  holomorphicity of $Q_i = | \tilde{Q} |^{- \nu}$. Now, the latter is known to
  have full support at points of its holomorphicity.
\end{proof}

\begin{lemma}
  \label{KP-normalization-2:lem-sets}Suppose $\{ A_n \}_n$ be finite family of
  subsets of $\mathbbm{Z}$ of form $A_i = \{ a_i, a_i + 1, \ldots, b_i \}$ and
  such that every member of family intersects some other member. Then,
  $\bigcup_n A_n = \{ a, a + 1, \ldots, b \}$ where $a \assign \min \{ a_i
  \}_i$ and $b \assign \max \{ b_i \}_i$.
\end{lemma}

\begin{proof}
  Induction on size of family $\{ A_n \}_n$. Statement is clear when size is
  one, and when size is more, we take two members $A_0, A_1 \in \{ A_n \}_n$
  such that $A_0 \cap A_1 \neq \varnothing$ and consider the new family $\{
  A_0 \cup A_1 \} \cup \{ A_n \}_n \backslash \{ A_0, A_1 \}$.
\end{proof}

\begin{lemma}
  \label{KP-normalization-2:lem-nonzero}$K_{\lambda, \nu}^P \neq 0$ for all
  $\nu \in \mathbbm{C}$. In other words, $\mathfrak{P} ( K_{\lambda, \nu}^C)
  \geqslant 0$.
\end{lemma}

\begin{proof}
  This follows from lemma \ref{P-def:lem-delta-max} and the fact that $|
  \tilde{Q} |^{\nu} \neq 0$ for $\nu \neq 0$.
\end{proof}

\begin{lemma}
  \label{KP-normalization-2:lem-trivial}Let $A \subset 1
  +\mathbbm{Z}_{\geqslant 0}$. Then for $n \in \mathbbm{Z}_{\geqslant 0}$ we
  have $( - n + A) \backslash \{ - n + 1, - n + 2, \ldots, 0 \} = ( - n + A)
  \cap \mathbbm{Z}_{\geqslant 1}$.
\end{lemma}

\begin{proof}
  By direct computation.
\end{proof}

\subsection{Proofs}

\begin{proof}
  (of proposition \ref{KP-normalization-2:prop-1}) Till the end of proof we
  will use symbols $p, n$ for $p', n'$ respectively. Original $p, n$ will not
  appear in this proof. The last statement is readily granted by lemma
  \ref{KP-normalization-2:lem-holo-supp}, once all others will be shown.
  
  One readily sees that
  \[ \mathfrak{P} ( Q_i) = \left\{ \begin{array}{ll}
       {}[ 1 - i + 2\mathbbm{Z}_{\geqslant 0}] \sqcup \left( \frac{n}{2} - i
       +\mathbbm{Z}_{\geqslant 0} \right), & i \in 2\mathbbm{Z}\\
       {}[ 2 - i + 2\mathbbm{Z}_{\geqslant 0}] \sqcup \left( \frac{n}{2} - i
       +\mathbbm{Z}_{\geqslant 0} \right), & i \in 2\mathbbm{Z}+ 1
     \end{array} \right. \]
  \_
  
  Now, first item of lemma \ref{P-def:lem-threeset} implies immediately the
  $\left( - \frac{n}{2} - k +\mathbbm{Z}_{\geqslant 0} \right)$ term in the
  result.
  
  The first item of lemma \ref{P-def:lem-threeset} now implies that to finish
  we need to show that (we may treat $\mathfrak{P} ( ( \nu)_i^{} Q_i)$ as a
  set in accordance with definition \ref{def-P}, as $Q_i$ has only simple
  poles)
  \[ \bigcup_{i = 0}^k ( \mathfrak{P} ( ( \nu)^{}_i Q_i) \cap \mathbbm{Z}) = [
     1 + 2\mathbbm{Z}_{\geqslant 0}] \]
  But we have
  \[ \mathfrak{P} ( ( \nu)^{}_i Q_i) \cap \mathbbm{Z}= \left\{
     \begin{array}{ll}
       {}[ 1 + 2\mathbbm{Z}_{\geqslant 0}], & i \in 2\mathbbm{Z}\\
       {}[ 1 + 2\mathbbm{Z}_{\geqslant 0}], & i \in 2\mathbbm{Z}+ 1
     \end{array} \right. \]
  as $2 - i, 1 - i \geqslant 1 - i$.
\end{proof}

\begin{proof}
  (of proposition \ref{KP-normalization-2:prop-2}) Till the end of proof we
  will use symbols $p, n$ for $p', n'$ respectively. Original $p, n$ will not
  appear in this proof. Definitions imply that $\mathfrak{P} ( Q_1) = 1
  +\mathfrak{P} ( Q_0)$ (we treat $\mathfrak{P} ( | Q |^{- \nu})$ and
  $\mathfrak{P} ( | Q |^{- \nu} \tmop{sgn} ( Q))$ as sets) and hence
  \[ \mathfrak{P} ( Q_i) = - i +\mathfrak{P} ( Q_0) = \left\{
     \begin{array}{ll}
       {}[ 1 - i + 2\mathbbm{Z}_{\geqslant 0}] \sqcup \left( \frac{n}{2} - i +
       2\mathbbm{Z}_{\geqslant 0} \right), & i \in 2\mathbbm{Z}_{\geqslant
       0}\\
       {}[ 1 - ( i - 1) + 2\mathbbm{Z}_{\geqslant 0}] \sqcup \left(
       \frac{n}{2} - ( i - 1) + 2\mathbbm{Z}_{\geqslant 0} \right), & i \in
       2\mathbbm{Z}_{\geqslant 0} + 1
     \end{array} \right. \]
  Now, lemma \ref{KP-normalization-2:lem-trivial} implies that
  \[ \mathfrak{P} ( ( \nu)^{}_i Q_i) =\mathbbm{Z}_{\geqslant 1} \cap ( - i
     +\mathfrak{P} ( Q_0)) = \left\{ \begin{array}{ll}
       {}[ 1 + 2\mathbbm{Z}_{\geqslant 0}] \sqcup \left( \max \left\{ 2,
       \frac{n}{2} - i \right\} + 2\mathbbm{Z}_{\geqslant 0} \right), & i \in
       2\mathbbm{Z}_{\geqslant 0}\\
       {}[ 1 + 2\mathbbm{Z}_{\geqslant 0}] \sqcup \left( \max \left\{ 2,
       \frac{n}{2} - ( i - 1) \right\} + 2\mathbbm{Z}_{\geqslant 0} \right), &
       i \in 2\mathbbm{Z}_{\geqslant 0} + 1
     \end{array} \right. \]
  and from this answer follows, as one sees that $\mathfrak{P} ( ( \nu)^{}_i
  Q_i)$ are non-decreasing.
\end{proof}

\begin{proof}
  (of proposition \ref{KP-normalization-2:prop-3}) Till the end of proof we
  will use symbols $p, n$ for $p', n'$ respectively. Original $p, n$ will not
  appear in this proof. Definitions imply that
  \[ \mathfrak{P} ( Q_i) = \left\{ \begin{array}{ll}
       - i +\mathfrak{P} ( Q_0), & i \in 2\mathbbm{Z}\\
       - ( i - 1) +\mathfrak{P} ( Q_1), & i \in 2\mathbbm{Z}+ 1
     \end{array} \right. = \left\{ \begin{array}{ll}
       {}[ 1 - i + 2\mathbbm{Z}_{\geqslant 0}] \sqcup \left( \frac{n}{2} + 1 -
       i + 2\mathbbm{Z}_{\geqslant 0} \right), & i \in 2\mathbbm{Z}_{\geqslant
       0}\\
       {}[ 1 - ( i - 1) + 2\mathbbm{Z}_{\geqslant 0}] \sqcup \left(
       \frac{n}{2} - i + 2\mathbbm{Z}_{\geqslant 0} \right), & i \in
       2\mathbbm{Z}_{\geqslant 0} + 1
     \end{array} \right. \]
  Now, lemma \ref{KP-normalization-2:lem-trivial} implies that $\mathfrak{P} (
  ( \nu)^{}_i Q_i)$ equals to
  \[ \left\{ \begin{array}{ll}
       \mathbbm{Z}_{\geqslant 1} \cap ( - i +\mathfrak{P} ( Q_0)), & i \in
       2\mathbbm{Z}_{\geqslant 0}\\
       \mathbbm{Z}_{\geqslant 1} \cap ( - ( i - 1) +\mathfrak{P} ( Q_1)), & i
       \in 2\mathbbm{Z}_{\geqslant 0} + 1
     \end{array} \right. = \left\{ \begin{array}{ll}
       {}[ 1 + 2\mathbbm{Z}_{\geqslant 0}] \sqcup \left( \max \left\{ 2,
       \frac{n}{2} + 1 - i \right\} + 2\mathbbm{Z}_{\geqslant 0} \right), & i
       \in 2\mathbbm{Z}_{\geqslant 0}\\
       {}[ 1 + 2\mathbbm{Z}_{\geqslant 0}] \sqcup \left( \max \left\{ 2,
       \frac{n}{2} - i \right\} + 2\mathbbm{Z}_{\geqslant 0} \right), & i \in
       2\mathbbm{Z}_{\geqslant 0} + 1
     \end{array} \right. \]
  and from this answer follows, as one sees that $\mathfrak{P} ( ( \nu)^{}_i
  Q_i)$ are non-decreasing.
\end{proof}

\begin{proof}
  (of proposition \ref{KP-normalization-2:prop-4}) \ Till the end of proof we
  will use symbols $p, n$ for $p', n'$ respectively. Original $p, n$ will not
  appear in this proof. We shall also employ notation $\mathfrak{P}^e (
  \cdot)$ as in definition \ref{P-def:def-9} and a subsequent remark.
  
  Definitions imply that $\mathfrak{P}^e ( Q_1) ( \nu + 1) =\mathfrak{P}^e (
  Q_0) ( \nu)$ and hence if we let $i'$ be the biggest even number that is
  $\leqslant i$ (as $k'$ in statement), then
  \[ \mathfrak{P}^e ( Q_i) = \left[ 1 - i', 3 - i', \ldots, \frac{n}{2} - i' -
     2 \right] \sqcup \left( \frac{n}{2} - i' + 2\mathbbm{Z}_{\geqslant 0}
     \right)^2 \]
  and hence
  \[ \mathfrak{P}^e ( ( \nu)^{}_i Q_i) = \left\{ \begin{array}{ll}
       \left[ 1, 3, \ldots, \frac{n}{2} - i' - 2 \right] \sqcup \left(
       \frac{n}{2} - i' + 2\mathbbm{Z}_{\geqslant 0} \right)^2, & n / 2 - i'
       \geqslant 0\\
       \left( \frac{n}{2} - i', \frac{n}{2} - i' + 2, \ldots, - 1 \right)
       \sqcup ( 1 + 2\mathbbm{Z}_{\geqslant 0})^2, & n / 2 - i' < 0
     \end{array} \right. \]
  and the result follows as $\mathfrak{P}^e ( ( \nu)^{}_i Q_i)$ form
  nondecreasing sequence of $\mathbbm{C} \rightarrow \mathbbm{Z}$ functions
  (``nondecreasing'' in the sence of definition \ref{P-def:def-9}).
\end{proof}

\begin{proof}
  (of proposition \ref{KP-normalization-2:prop-5}) Till the end of proof we
  will use symbols $p, n$ for $p', n'$ respectively. Original $p, n$ will not
  appear in this proof. We shall also employ notation $\mathfrak{P}^e (
  \cdot)$ as defined in definition \ref{P-def:def-9} and subsequent remark.
  \[ \mathfrak{P}^e ( Q_i) = \left\{ \begin{array}{ll}
       \left[ 1 - i, 3 - i, \ldots, \frac{n}{2} - 1 - i \right] \sqcup \left(
       \frac{n}{2} + 1 - i + 2\mathbbm{Z}_{\geqslant 0} \right)^2, & i \in
       2\mathbbm{Z}_{\geqslant 0}\\
       \left[ 1 - ( i - 1), 3, \ldots, \frac{n}{2} - 3 - ( i - 1) \right]
       \sqcup \left( \frac{n}{2} - 1 - ( i - 1) + 2\mathbbm{Z}_{\geqslant 0}
       \right), & i \in 2\mathbbm{Z}_{\geqslant 0} + 1
     \end{array} \right. \]
  and hence
  \[ \mathfrak{P} ( ( \nu^{})^i Q_i) = \left\{ \begin{array}{ll}
       \left[ 1, 3, \ldots, \frac{n}{2} - 1 - i \right] \sqcup \left(
       \frac{n}{2} + 1 - i + 2\mathbbm{Z}_{\geqslant 0} \right)^2, & i \in
       2\mathbbm{Z}_{\geqslant 0}, \frac{n}{2} - 1 - i \geqslant 1\\
       \left( \frac{n}{2} + 1 - i, \frac{n}{2} + 3 - i, \ldots, - 1 \right)
       \sqcup ( 1 + 2\mathbbm{Z}_{\geqslant 0})^2, & i \in
       2\mathbbm{Z}_{\geqslant 0}, \frac{n}{2} - 1 - i < 1\\
       \left[ 1, \ldots, \frac{n}{2} - i - 2 \right] \sqcup \left( \frac{n}{2}
       - i + 2\mathbbm{Z}_{\geqslant 0} \right)^2, & i \in
       2\mathbbm{Z}_{\geqslant 0} + 1, \frac{n}{2} - 2 - i \geqslant 1\\
       \left( \frac{n}{2} - i, \ldots, - 1 \right) \sqcup ( 1 +
       2\mathbbm{Z}_{\geqslant 0})^2, & i \in 2\mathbbm{Z}_{\geqslant 0} + 1,
       \frac{n}{2} - 2 - i < 1
     \end{array} \right. \]
  and if we let $i'$ be the biggest odd number that is $\leqslant i$, we can
  rewrite this as
  \[ = \left\{ \begin{array}{ll}
       \left[ 1, 3, \ldots, \frac{n}{2} - 2 - i' \right] \sqcup \left(
       \frac{n}{2} - i' + 2\mathbbm{Z}_{\geqslant 0} \right)^2, & n / 2 - 3
       \geqslant i',\\
       \left( \frac{n}{2} - i', \frac{n}{2} - i' + 2, \ldots, - 1 \right)
       \sqcup ( 1 + 2\mathbbm{Z}_{\geqslant 0})^2, & i' > n / 2 - 3
     \end{array} \right. \]
  and the result follows as $\mathfrak{P}^e ( ( \nu)^{}_i Q_i)$ form
  nondecreasing sequence of $\mathbbm{C} \rightarrow \mathbbm{Z}$ functions
  (``nondecreasing'' in the sence of definition \ref{P-def:def-9}).
\end{proof}

\begin{proof}
  (of prop. \ref{KP-normalization-2:prop-p=1}) Follows from proposition
  \ref{q-norm:prop-pqzero} and the computations similar to above.
\end{proof}

\section{Normalization of $K_{\lambda,
\nu}^{\mathbbm{R}^n}$}\label{sec:KR-normalization-even}

In this part, we normalize the kernel of regular symmetry breaking operator.
As shown in proposition \ref{supp-R:prop-3}, for an open subset of $\{ \lambda
- \nu \nin -\mathbbm{Z}_{\geqslant 0} \} \subset \mathbbm{C}^2$ we have kernel
of regular SBO $K_{\lambda, \nu}^{\mathbbm{R}^n}$ being equal to
\begin{equation}
  K_{\lambda, \nu}^{\mathbbm{R}^n} = \frac{| x_p |^{\lambda + \nu - n} \cdot |
  Q |^{- \nu}}{\Gamma ( ( 1 - \nu) / 2) \Gamma ( ( \lambda + \nu - n \upl 1) /
  2)}
\end{equation}

\subsection{Main results}

\begin{proposition}
  \label{KR-normalization-even:prop-odd}For $K_{\lambda, \nu}^{\mathbbm{R}^n}$
  as in proposition \ref{supp-R:prop-3}, $\tilde{K}_{\lambda,
  \nu}^{\mathbbm{R}^n} \assign K_{\lambda, \nu}^{\mathbbm{R}^n} / \Gamma
  \left( \frac{\lambda - \nu}{2} \right)$ extends to holomorphic in $(
  \lambda, \nu) \in \mathbbm{C}^2$ distribution that vanishes on a discrete
  subset of $\mathbbm{C}^2$. More precisely,
  \begin{eqnarray}
    & \mathfrak{P}_- ( \tilde{K}_{\lambda, \nu}^{\mathbbm{R}^n}) = [
    \mathfrak{P}_- ( \tilde{K}_{\lambda, \nu}^{\mathbbm{R}^n}) \cap \{ \lambda
    - \nu \in - 2\mathbbm{Z}_{\geqslant 0} \}] \sqcup [ \mathfrak{P}_- (
    \tilde{K}_{\lambda, \nu}^{\mathbbm{R}^n}) \backslash \{ \lambda - \nu \in
    - 2\mathbbm{Z}_{\geqslant 0} \}], &  \nonumber\\
    & \mathfrak{P}_- ( \tilde{K}_{\lambda, \nu}^{\mathbbm{R}^n}) \backslash
    \{ \lambda - \nu \in - 2\mathbbm{Z}_{\geqslant 0} \} = \left\{
    \begin{array}{ll}
      \{ \nu \in 2\mathbbm{Z}_{\geqslant 0} + 1 \} \cap \{ \lambda + \nu - n
      \in - 1 - 2\mathbbm{Z}_{\geqslant 0} \}, & p = 1\\
      \varnothing, & p > 1
    \end{array} \right. \backslash \{ \lambda - \nu \in -
    2\mathbbm{Z}_{\geqslant 0} \}, &  \nonumber\\
    & \mathfrak{P}_- ( \tilde{K}_{\lambda, \nu}^{\mathbbm{R}^n}) \cap \{
    \lambda - \nu \in - 2\mathbbm{Z}_{\geqslant 0} \} = \{ \lambda - \nu \in -
    2\mathbbm{Z}_{\geqslant 0} \} \cap L, &  \nonumber\\
    & L \assign \left\{ \begin{array}{ll}
      \{ ( \lambda, \nu) \in \mathbbm{C}^2 | \nu\in\mathbbm{Z}_{\leqslant 0} \cup (
      2\mathbbm{Z}_{\geqslant 0} + 1) \}, & q \in 2\mathbbm{Z}\\
      \{ ( \lambda, \nu) \in \mathbbm{C}^2 | \nu \in 2\mathbbm{Z} \}, & q \in
      2\mathbbm{Z}+ 1.
    \end{array} \right. &  \nonumber
  \end{eqnarray}
\end{proposition}

\begin{remark}
  \label{KR-normalization-even:rmk-Atilde}It follows that for $( \lambda, \nu)
  \in L$ (with $L$ as in proposition \ref{KR-normalization-even:prop-odd}) one
  has that the kernel
  \[ \widetilde{\tilde{K}}^{\mathbbm{R}^n}_{\lambda, \nu} \assign \Gamma
     \left( \frac{\lambda - \nu}{2} \right) \tilde{K}_{\lambda,
     \nu}^{\mathbbm{R}^n} = \frac{| x_p |^{\lambda + \nu - n}}{\Gamma ( ( 1 -
     \nu) / 2)} \cdot \frac{| Q |^{- \nu}}{\Gamma ( ( \lambda + \nu - n \upl
     1) / 2)} \]
  is an element of $\mathcal{S} \tmop{ol} ( \mathbbm{R}^n ; \lambda \comma
  \nu)$ holomorphic in $\lambda \in \mathbbm{C}$. Proposition
  \ref{KR-normalization-recur:prop-supp} implies that support of
  $\widetilde{\tilde{K}}_{\lambda, \nu}^{\mathbbm{R}^n}$ equals to
  \[ = \left\{ \begin{array}{ll}
       \mathbbm{R}^n, & ( \lambda, \nu) \in \{ \nu \nin
       2\mathbbm{Z}_{\geqslant 0} + 1 \} \cap \{ \lambda + \nu - n \nin - 1 -
       2\mathbbm{Z}_{\geqslant 0} \}\\
       \{ Q = 0 \}, & ( \lambda, \nu) \in \{ \nu \in 2\mathbbm{Z}_{\geqslant
       0} + 1 \} \cap \{ \lambda + \nu - n \nin - 1 - 2\mathbbm{Z}_{\geqslant
       0} \}\\
       \{ x_p = 0 \}, & ( \lambda, \nu) \in \{ \nu \nin
       2\mathbbm{Z}_{\geqslant 0} + 1 \} \cap \{ \lambda + \nu - n \in - 1 -
       2\mathbbm{Z}_{\geqslant 0} \}\\
       \varnothing, & p = 1, \; ( \lambda, \nu) \in \{ \nu \in
       2\mathbbm{Z}_{\geqslant 0} + 1 \} \cap \{ \lambda + \nu - n \in - 1 -
       2\mathbbm{Z}_{\geqslant 0} \}\\
       \{ x_p = 0 \} \cap \{ Q = 0 \}, & p > 1, \; ( \lambda, \nu) \in \{ \nu
       \in 2\mathbbm{Z}_{\geqslant 0} + 1 \} \cap \{ \lambda + \nu - n \in - 1
       - 2\mathbbm{Z}_{\geqslant 0} \} .
     \end{array} \right. \]
\end{remark}

\subsection{Auxiliary results}

\begin{lemma}
  \label{KR-normalization-even:lem-gelfand}Let $\Omega \subset \mathbbm{C}$ be
  an open set and $a : \Omega \rightarrow \mathbbm{C}$ be holo with
  nonvanishing derivative. Suppose that for $D \subset \Omega$ discrete we
  have $F_{\lambda} \in \mathcal{D}' ( \mathbbm{R}^k)$ be holomorphic in
  $\lambda \in \Omega \backslash D$ and homogeneous of degree $a ( \lambda)$
  and $D = a^{- 1} ( - k -\mathbbm{Z}_{\geqslant 0})$. Suppose further that
  $F_{\lambda} |_{\mathbbm{R}^k - \{ 0 \}}$ extends to holo in $\lambda \in
  \Omega$.
  
  Then, $\tilde{F}_{\lambda} : = F_{\lambda} / \Gamma ( a ( \lambda) + k)$
  extends to holomorphic in $\lambda \in \Omega$. Moreover, for $\lambda_{}
  \in D$ we have
  \begin{eqnarray}
    & \tilde{F}_{\lambda_{}} = \sum_{| \alpha | = - a ( \lambda) - k}
    c_{\alpha} \delta^{( \alpha)} &  \nonumber\\
    & c_{\alpha} \assign \langle F_{\lambda} |_{\mathbbm{S}^k}, x^{\alpha}
    \rangle &  \nonumber
  \end{eqnarray}
\end{lemma}

\begin{remark}
  This basically is a restatement of a discussion in
  {\cite[III.{\textsection}3.5]{gelfand1980distribution}}.
\end{remark}

\begin{proof}
  We first claim that $F_{\lambda} |_{\mathbbm{S}^k}$ is well-defined and
  holomorphic in $\lambda \in \Omega$. Indeed, under the assumption that
  $F_{\lambda} |_{\mathbbm{R}^k - \{ 0 \}}$ extends to holomorphic
  distribution in $\lambda \in \Omega$, we have similarly to the proof of
  lemma \ref{k-finite:lem-holo-easy} that $F_{\lambda}$ is holomorphic in
  $\mathcal{D}'_{\Gamma} ( \mathbbm{R}^k - \{ 0 \})$ with $\Gamma$ being the
  cone $\{ ( x, \xi) \in \mathbbm{R}^k \backslash \{ 0 \} \times \mathbbm{R}^k
  | x \perp \xi \}$ and thus proposition
  \ref{holomorphicity-preserving:prop-pullback-holo} implies the claim.
  
  Discreteness of $D$ allows us to consider matters locally near every point
  of $D$, thus we may assume $D = \{ \lambda_0 \}$. Moreover, as $a (
  \lambda)$ has novanishing derivative at $\lambda_0$, we can
  biholomorphically change coordinates in parameter space and thus assume that
  $a ( \lambda) = \lambda$ and $\lambda_0 = - k - l \in - k
  -\mathbbm{Z}_{\geqslant 0}$. We next claim that near (but not equal to)
  $\lambda = \lambda_0$ we have
  \begin{eqnarray}
    & \langle \tilde{F}_{\lambda}, \varphi \rangle = \frac{1}{\Gamma (
    \lambda + k)} \langle r^{\lambda + k - 1}, u_{\lambda} \rangle, 
    \label{KR-normalization-even:eq-lemeq} & \\
    & u_{\lambda} ( r) \assign \langle F_{\lambda} |_{\mathbbm{S}^{k - 1}},
    \varphi ( r \cdot) \rangle . &  \nonumber
  \end{eqnarray}
  We note that $u \in C^{\infty}_0 ( \mathbbm{R}_{\geqslant 0})$, as is easily
  seen by direct check. Now, both sides $(
  \ref{KR-normalization-even:eq-lemeq})$ are clearly homogeneous of degree
  $\lambda$, so it suffices (by fact
  \ref{holomorphicity-preserving:fact-homog}) to show that they coincide on
  $\mathbbm{R}^k \backslash \{ 0 \}$, where the equality is clear (passing to
  polar coordinates $F_{\lambda}$ becomes tensor product).
  
  We next show that $\tilde{F}_{\lambda}$ is holomorphic at $\lambda =
  \lambda_0$. We fix $\varphi \in C^{\infty}_0 ( \mathbbm{R}^k)$ and it
  suffices to show the continuity of right-hand side of $(
  \ref{KR-normalization-even:eq-lemeq})$ in $\lambda$. So suppose that
  $\lambda_n \rightarrow \lambda_0 = - k - l$. Now, lemma
  \ref{supp-n-waves:lem-weakened-conv} implies that we just need to show that
  $u_{\lambda_n} \assign u_n \rightarrow u_0 \assign u_{\lambda_0}$ pointwise
  with all derivatives in $\mathbbm{R}_{\geqslant 0}$ and that $u_{\lambda_n}$
  have their derivatives are uniformly bounded in $n$. Now, fact
  \ref{holomorphicity-preserving:fact-basic} applied to $\psi :
  \mathbbm{R}_{\geqslant 0} \times \mathbbm{S}^{k - 1} \ni ( r, \omega)
  \mapsto \varphi ( r \omega)$ implies that (we let $f_{\lambda} \assign
  F_{\lambda} |_{\mathbbm{S}^{n - 1}}$ and $\varphi^{( \alpha)}$ denotes
  partial derivative)
  \[ \frac{d^m}{d r^m} u_{\lambda} ( r) = \left\langle f_{\lambda},
     \frac{d^m}{d r^m} \varphi ( r \cdot) \right\rangle = \left\langle
     f_{\lambda}, \sum_{| \alpha | = m} \omega^{\alpha} \varphi^{( \alpha)} (
     r \cdot) \right\rangle \]
  which readily settles the issue about the pointwise convergence of
  derivatives, so it only remains to show that these are uniformly bounded in
  $n$ and $x \in \mathbbm{R}_{\geqslant 0}$. The latter equality implies that
  it suffices to show this for the 0-th derivative. That is, we just need to
  show that $u_n$ are uniformly bounded in $r \in \mathbbm{R}_{\geqslant 0}$
  and $n$.
  
  Now, the third item of fact \ref{holomorphicity-preserving:fact-basic} tells
  us that
  \[ | \langle f_n, \varphi ( r \cdot) \rangle | \leqslant C \sum_{i = 0}^k
     \sup | \partial^i \varphi ( r \cdot) | = \sum_{i = 0}^k \sup \left(
     \sum_{| \alpha | = i} r^i \varphi^{( \alpha)} ( r \cdot) \right) \]
  and as the right-hand side is uniformly bounded in $n$ and $r$, we are done
  with showing the holomorphicity of $\tilde{F}_{\lambda}$.
  
  Finally, we need to show what $\tilde{F}_{\lambda}$ becomes for $\lambda =
  \lambda_0$. As by hypothesis $F_{\lambda} |_{\{ x \neq 0 \}}$ is holomorphic
  at $\lambda_0$, we see that $F_{\lambda} / \Gamma ( \lambda + k) |_{\{ x
  \neq 0 \}}$ necessary vanishes at $\lambda = \lambda_0$, thus
  $\tilde{F}_{\lambda_0}$ is supported only at $\{ 0 \}$ and therefore should
  be a finite sum of derivatives of delta functions. Moreover, as
  $\tilde{F}_{\lambda}$ is holomorphic in $\lambda$, it has to be homogeneous
  of degree $\lambda_0$ at $\lambda_0$ and therefore
  \[ \tilde{F}_{\lambda_0} = \sum_{| \alpha | = - a ( \lambda) - k} c_{\alpha}
     \delta^{( \alpha)} . \]
  and
  \begin{eqnarray}
    & c_{\alpha} = \frac{\langle \tilde{F}_{\lambda_0}, x^{\alpha} \rangle}{|
    \alpha | !} = \frac{\langle \delta^{- k - \lambda_0}, u \rangle}{( - k -
    \lambda_0) !} = &  \nonumber
  \end{eqnarray}
  where $u = \langle f_{\lambda_0}, x^{\alpha} \rangle = \langle
  f_{\lambda_0}, r^{- \lambda_0 - k} \omega^{\alpha} \rangle = r^{- \lambda_0
  - k} \langle f_{\lambda_0}, \omega^{\alpha} \rangle$ (where $\omega \assign
  x |_{\mathbbm{S}^{n - 1}}$) and thus we can continue
  \[ = \langle f_{\lambda_0}, \omega^{\alpha} \rangle \frac{\langle \delta^{-
     k - \lambda_0}, r^{- \lambda_0 - k} \rangle}{( - k - \lambda_0) !} =
     \langle f_{\lambda_0}, \omega^{\alpha} \rangle, \]
  and we are done.
\end{proof}

\subsection{Proofs}

\begin{proof}
  (of prop. \ref{KR-normalization-even:prop-odd}) We first show that
  $\tilde{K}_{\lambda, \nu}^{\mathbbm{R}^n}$ is indeed holomorphic. In the
  light of fact \ref{k-finite:fact-hartogs} it suffices to fix one parameter
  and show that $\tilde{K}_{\lambda, \nu}^{\mathbbm{R}^n}$ is holomorphic as a
  function of the other one. We fix $\nu = \nu_0$ (the case $\lambda =
  \lambda_0$ fixed is handled in the same way). As $K_{\lambda} \assign
  K_{\lambda, \nu_0}$ is homogeneous of degree $\lambda - \nu_0 - n$, lemma
  \ref{KR-normalization-even:lem-gelfand} allows us to conclude that
  $\widetilde{\tilde{K}}_{\lambda} \assign K_{\lambda, \nu_0}^{\mathbbm{R}^n}
  / \Gamma ( \lambda - \nu_0)$ is holomorphic. Moreover, the same lemma
  \ref{KR-normalization-even:lem-gelfand} implies that for $\lambda - \nu_0
  \in - 1 - 2\mathbbm{Z}_{\geqslant 0}$, we have
  $\widetilde{\tilde{K}}_{\lambda} = 0$, as for $| \alpha | = \nu_0 - \lambda
  \in 1 + 2\mathbbm{Z}_{\geqslant 0}$ we have $c_{\alpha} \assign \langle
  K_{\lambda, \nu_0}^{\mathbbm{R}^n} |_{\mathbbm{S}^{n - 1}}, x^{\alpha}
  \rangle = 0$, as $x^{\alpha}$ is odd, while $K_{\lambda,
  \nu_0}^{\mathbbm{R}^n} |_{\mathbbm{S}^{n - 1}}$ is even. Hence, in fact
  already $\tilde{K}_{\lambda} \assign K_{\lambda, \nu_0}^{\mathbbm{R}^n} /
  \Gamma \left( \frac{\lambda - \nu_0}{2} \right)$ is holomorphic, and from
  this the holomorphicity of $\tilde{K}_{\lambda, \nu}^{\mathbbm{R}^n}$
  function follows.
  
  From the way we did normalization it immediately follows that
  \[ \mathfrak{P}_- ( \tilde{K}_{\lambda, \nu}^{\mathbbm{R}^n}) \backslash \{
     \lambda - \nu \in - 2\mathbbm{Z}_{\geqslant 0} \} =\mathfrak{P}_- (
     K_{\lambda, \nu}^{\mathbbm{R}^n}) \backslash \{ \lambda - \nu \in -
     2\mathbbm{Z}_{\geqslant 0} \} . \]
  From this and proposition \ref{KR-normalization-recur:prop-supp} the
  expression for $\mathfrak{P}_- ( K_{\lambda, \nu}^{\mathbbm{R}^n})
  \backslash \{ \lambda - \nu \in - 2\mathbbm{Z}_{\geqslant 0} \}$ follows.
  Thus it remains to give an expression for $\mathfrak{P}_- (
  \tilde{K}_{\lambda, \nu}^{\mathbbm{R}^n}) \cap \{ \lambda - \nu \in -
  2\mathbbm{Z}_{\geqslant 0} \} = -\mathfrak{P}_{} ( \tilde{K}_{\lambda,
  \nu}^{\mathbbm{R}^n}) \cap \{ \lambda - \nu \in - 2\mathbbm{Z}_{\geqslant 0}
  \}$.
  
  Now, the formulae of lemma \ref{KR-normalization-even:lem-gelfand} imply
  that for $k \in \mathbbm{Z}_{\geqslant 0}$ we have
  \[ -\mathfrak{P}_{} ( \tilde{K}_{\lambda, \nu}^{\mathbbm{R}^n}) \cap \{
     \lambda - \nu = - 2 k \} = \max \left\{ \mathfrak{P}_{} \left(
     \frac{\langle | Q |^{- \nu} | x_p |^{\lambda + \nu - n}, x^{\gamma}
     \rangle_{\mathbbm{S}^{n - 1}}}{\Gamma \left( \frac{1 - \nu}{2} \right)
     \Gamma \left( \frac{\lambda + \nu - n + 1}{2} \right)} |_{\lambda - \nu =
     - 2 k} \right) \right\}_{| \gamma | = 2 k} \]
  Note that the holomorphicity of an argument of $\mathfrak{P}_-$ on the
  right-hand side of the latter equation is guaranteed by holomorphicity
  assertion of proposition \ref{KR-normalization-recur:prop-supp}. It suffices
  therefore to explicitly compute the $\mathfrak{P}$ of meromorphic in $(
  \lambda, \nu)$ function $\langle | Q |^{- \nu} | x_p |^{\lambda + \nu - n},
  \omega^{\alpha} \rangle_{\mathbbm{S}^{n - 1}}$ under the constraint $\lambda
  + \nu = - 2 k$. For this explicit formula for $\langle | Q |^{- \nu} | x_p
  |^{\lambda + \nu - n}, \omega^{\alpha} \rangle_{\mathbbm{S}^{n - 1}}$ would
  come handy and we now proceed with giving such a formula.
  
  One notes that for $\tmop{Re} ( - \nu), \tmop{Re} ( \lambda + \nu) \gg 0$
  one has by passing to bipolar coordinates $x = ( r \omega, s \omega') \in
  \mathbbm{R}^p \times \mathbbm{R}^q$ (and splitting multiindex $\gamma \in
  \mathbbm{Z}_{\geqslant 0}^n$ as $\gamma = : ( \alpha, \beta)$ accordingly)
  that
  \begin{eqnarray}
    & \langle | Q |^{- \nu} | x_p |^{\lambda + \nu - n}, x^{\gamma}
    \rangle_{\mathbbm{S}^{n - 1}} = \int_{\mathbbm{S}^{p - 1}} | \omega_p
    |^{\lambda + \nu - n} \omega^{\alpha_p}_p \tilde{\omega}^{\tilde{\alpha}}
    d \omega \times \int_{\mathbbm{S}^{q - 1}} ( \omega')^{\beta} d \omega'
    \times &  \nonumber
  \end{eqnarray}
  where we split multiindex $\alpha \in \mathbbm{Z}^p_{\geqslant 0}$ as
  $\alpha = : ( \tilde{\alpha}, \alpha_p)$ and split $\omega \in
  \mathbbm{S}^{p - 1}$ as $\omega = : ( \tilde{\omega}, \omega_p)$. One
  further notes that $\int_{\mathbbm{S}^{q - 1}} ( \omega')^{\beta} d \omega'
  = 0$ unless $| \beta | \assign \sum_j \beta_j \in 2\mathbbm{Z}$ and thus we
  may in subsequent assume that this is so. Hence, (as $| \gamma | = | \alpha
  | + | \beta | = 2 k$) we can restrict ourselves to situation $| \alpha | = :
  a \in 2\mathbbm{Z}_{\geqslant 0}$ and $| \beta | = : b \in
  2\mathbbm{Z}_{\geqslant 0}$. Then, we may continue as
  \begin{eqnarray}
    & \simeq \int_{\mathbbm{S}^{p - 1}} | \omega_p |^{\lambda + \nu - n}
    \omega^{\alpha_p}_p \tilde{\omega}^{\tilde{\alpha}} d \omega \int_{r^2 +
    s^2 = 1 ; r, s > 0} | r^2 - s^2 |^{- \nu} r^{\lambda + \nu - n} r^{p + a -
    1} s^{q + b - 1} d r d s \simeq &  \nonumber
  \end{eqnarray}
  Now, inspecting the expression for the volume element of $\mathbbm{S}^{p -
  1}$ sphere, one concludes that for $p > 1$
  \[ \int_{\mathbbm{S}^{p - 1}} | \omega_p |^{\lambda + \nu - n}
     \omega^{\alpha_p}_p \tilde{\omega}^{\tilde{\alpha}} d \omega =
     \int_0^{\pi} | \cos \varphi |^{\lambda + \nu - n} \cos^{\alpha_p}
     \varphi^{} \cdot \sin^{p - 2} \varphi \left[ \int_{\mathbbm{S}^{p - 2}}
     \tilde{\omega}^{\tilde{\alpha}} d \tilde{\omega} \right] d \varphi \]
  and as $\int_{\mathbbm{S}^{p - 2}} \tilde{\omega}^{\widetilde{\alpha_{}}_p}
  d \tilde{\omega} = 0$ unless $| \tilde{\alpha}_p | \in 2\mathbbm{Z}$, we can
  assume in subsequent that this is so, hence $\alpha_p \in 2\mathbbm{Z}$.
  Whereas for $p = 1$ we have
  \[ \int_{\mathbbm{S}^{p - 1}} | \omega_p |^{\lambda + \nu - n}
     \omega^{\alpha_p}_p \tilde{\omega}^{\tilde{\alpha}} d \omega = \left\{
     \begin{array}{ll}
       0, & \alpha_p \nin 2\mathbbm{Z}\\
       \tmop{const} \neq 0, & \alpha_p \in 2\mathbbm{Z}
     \end{array} \right. \]
  Moreover, using the formula for integrating along the curve $y = \sqrt{1 -
  t^2}$ in $\mathbbm{R}^2$, one sees that
  \[ \int_{r^2 + s^2 = 1 ; r, s > 0} | r^2 - s^2 |^{- \nu} r^{\lambda + \nu -
     n} r^{p + a - 1} s^{q + b - 1} d r d s = \int_0^1 | 1 - 2 t^2 |^{- \nu}
     \frac{1}{\sqrt{1 - t^2}} t^{\lambda + \nu - n + p + a - 1} \sqrt{1 -
     t^2}^{q + b - 1} d t \]
  and therefore we can continue the chain of equalities above in case $p > 1$
  (case $p = 1$ is handled similarly) as
  \begin{eqnarray}
    & \simeq \int_0^{\pi} | \cos \varphi |^{\lambda + \nu - n}
    \cos^{\alpha_p} \varphi^{} \cdot \sin^{p - 2} \varphi d \varphi \int_0^1 |
    1 - 2 t^2 |^{- \nu} \frac{1}{\sqrt{1 - t^2}} t^{\lambda + \nu + a - n + p
    - 1} \sqrt{1 - t^2}^{q + b - 1} d t = &  \nonumber\\
    & \int_{- 1}^1 | t |^{\lambda + \nu - n} t^{\alpha_p} ( 1 - t^2)^{( p -
    3) / 2} d t \int_0^1 | 1 - 2 t |^{- \nu} ( 1 - t)^{( q + b - 2) / 2} t^{(
    \lambda + a + \nu - q) / 2 - 1} d t \simeq &  \nonumber
  \end{eqnarray}
  and as we assume that $\alpha_p \in 2\mathbbm{Z}$, we have that
  \[ \int_{- 1}^1 | t |^{\lambda + \nu - n} t^{\alpha_p} ( 1 - t^2)^{( p - 3)
     / 2} d t = 2 \int_0^1 t^{\lambda + \nu - n + \alpha_p} ( 1 - t^2)^{( p -
     3) / 2} d t = 2 \int_0^1 t^{\frac{\lambda + \nu - n + \alpha_p - 1}{2}} (
     1 - t)^{\frac{p - 3}{2}} d t \]
  thus we can continue as (using the expression for beta function)
  \begin{eqnarray}
    & \simeq \int_0^1 t^{\frac{\lambda + \nu - n + \alpha_p - 1}{2}} ( 1 -
    t)^{\frac{p - 3}{2}} d t \int_0^1 | 1 - 2 t |^{- \nu} ( 1 - t)^{( q + a -
    2) / 2} t^{( \lambda + b + \nu - q) / 2 - 1} d t \simeq &  \nonumber\\
    & \simeq \frac{\Gamma \left( \frac{\lambda + \nu - n + \alpha_p + 1}{2}
    \right)}{\Gamma \left( \frac{\lambda + \nu - q + \alpha_p}{2} \right)}
    \int_{- 1}^1 | w |^{- \nu} ( 1 - w)^{( \lambda + a + \nu - q) / 2 - 1} ( 1
    + w)^{( q + b - 2) / 2} d w \simeq &  \nonumber
  \end{eqnarray}
  and accidentally the latter expression holds also for $p = 1$. Now, using
  the (valid for regular enough values of parameters) formula $\int_0^1 ( 1 +
  t)^{- a} ( 1 - t)^{c - 1} t^{b - 1} d t =_2 F_1 ( a, b ; b + c ; - 1) B ( b,
  c)$ we arrive at the expression
  \begin{eqnarray}
    & \int_{- 1}^1 | w |^{- \nu} ( 1 - w)^{( \lambda + a + \nu - q) / 2 - 1}
    ( 1 + w)^{( q + b - 2) / 2} d w = &  \nonumber\\
    & = \int_0^1 w^{- \nu} ( 1 - w)^{( \lambda + a + \nu - q) / 2 - 1} ( 1 +
    w)^{( q + b - 2) / 2} d w + \int_0^1 w^{- \nu} ( 1 + w)^{( \lambda + \nu +
    a - q) / 2 - 1} ( 1 - w)^{( q + b - 2) / 2} d w = &  \nonumber\\
    & =_2 F_1 \left( 1 - \frac{q + b}{2}, 1 - \nu ; \frac{\lambda - \nu - q +
    a}{2} + 1 ; - 1 \right) B \left( 1 \um \nu, \frac{\lambda + \nu - q +
    a}{2} \right) + &  \nonumber\\
    & +_2 F_1 \left( 1 - \frac{\lambda + \nu - q + a}{2}, 1 - \nu ; \frac{q +
    b}{2} - \nu + 1 ; - 1 \right) B \left( 1 - \nu, \frac{q + b}{2} \right) & 
    \nonumber
  \end{eqnarray}
  
  
  Now, as we assume that $\lambda - \nu = - 2 k = a + b$, we conclude that
  both hypergeometric functions in the previous expression have their first
  and third arguments being equal, and as we have an equality (seen for
  example by recalling power series expansion of $_2 F_1$), $_2 F_1 ( a, b ;
  a, - 1) = 2^{- b}$, we can continue the above chain of equalities as
  \begin{eqnarray}
    & \simeq \frac{\Gamma \left( \frac{\lambda + \nu - n + \alpha_p + 1}{2}
    \right)}{\Gamma \left( \frac{\lambda + \nu - q + \alpha_p}{2} \right)}
    \left( B \left( 1 \um \nu, \frac{\lambda + \nu - q + a}{2} \right) + B
    \left( 1 - \nu, \frac{q + b}{2} \right) \right) = &  \nonumber\\
    & = \frac{\Gamma \left( \frac{\lambda + \nu - n + \alpha_p + 1}{2}
    \right)}{\Gamma \left( \frac{\lambda + \nu - q + \alpha_p}{2} \right)}
    \Gamma ( 1 - \nu) \left\{ \frac{\Gamma \left( \frac{\lambda + \nu - q +
    a}{2} \right)}{\Gamma \left( 1 + \frac{\lambda - \nu - q + a}{2} \right)}
    + \frac{\Gamma \left( \frac{q + b}{2} \right)}{\Gamma \left( 1 - \nu +
    \frac{q + b}{2} \right)} \right\} = &  \nonumber
  \end{eqnarray}
  substituting $\lambda = \nu - 2 k$ and $b = 2 k - a$ the above gets
  rewritten as
  \[ \frac{\Gamma \left( \frac{2 \nu - 2 k - n + \alpha_p + 1}{2}
     \right)}{\Gamma \left( \frac{2 \nu - 2 k - q + \alpha_p}{2} \right)}
     \Gamma ( 1 - \nu) \left\{ \frac{\Gamma \left( \frac{2 \nu - 2 k - q +
     a}{2} \right)}{\Gamma \left( 1 + \frac{- 2 k - q + a}{2} \right)} +
     \frac{\Gamma \left( \frac{q + 2 k - a}{2} \right)}{\Gamma \left( 1 - \nu
     + \frac{q + 2 k - a}{2} \right)} \right\} . \]
  The reasoning at the beginning of the proof now suggests that it suffices to
  find
  \begin{eqnarray}
    & \max \left\{ \mathfrak{P}_{} \left( \frac{\Gamma \left( \frac{2 \nu - 2
    k - n + \alpha_p + 1}{2} \right)}{\Gamma \left( \frac{2 \nu - 2 k - q +
    \alpha_p}{2} \right)} \times \right. \right. &  \nonumber\\
    & \times \left. \left. \frac{\Gamma ( 1 - \nu)}{\Gamma \left( \frac{1 -
    \nu}{2} \right) \Gamma \left( \frac{\lambda + \nu - n + 1}{2} \right)}
    \left\{ \frac{\Gamma \left( \frac{2 \nu - 2 k - q + a}{2} \right)}{\Gamma
    \left( 1 + \frac{- 2 k - q + a}{2} \right)} + \frac{\Gamma \left( \frac{q
    + 2 k - a}{2} \right)}{\Gamma \left( 1 - \nu + \frac{q + 2 k - a}{2}
    \right)} \right\} \right) \right\}_{( \alpha_p, a) \in \mathfrak{I}} & 
    \nonumber
  \end{eqnarray}
  where $\mathfrak{I} \assign \{ ( \alpha_p, a) \in ( 2\mathbbm{Z}_{\geqslant
  0})^2 | \alpha_p \leqslant a \leqslant 2 k \}$. Now, using the formulae
  $\Gamma ( 1 - z) \Gamma ( z) = \pi / \sin ( \pi z)$ and $\sin ( \alpha) +
  \sin ( \beta) = 2 \sin ( ( \alpha + \beta) / 2) \cos ( ( \alpha - \beta) /
  2)$, we obtain
  \begin{eqnarray}
    & \frac{\Gamma \left( \frac{2 \nu - 2 k - q + a}{2} \right)}{\Gamma
    \left( 1 + \frac{- 2 k - q + a}{2} \right)} + \frac{\Gamma \left( \frac{q
    + 2 k - a}{2} \right)}{\Gamma \left( 1 - \nu + \frac{q + 2 k - a}{2}
    \right)} \simeq \frac{\sin \frac{\pi \nu}{2} \cos \left[ \pi \left( x -
    \frac{\nu}{2} \right) \right]}{\Gamma ( 1 - x) \Gamma ( 1 + x - \nu) \sin
    [ \pi ( \nu - x)] \sin ( x)} \simeq &  \nonumber\\
    & \simeq \frac{\sin \frac{\pi \nu}{2} \cos \left[ \pi \left( x -
    \frac{\nu}{2} \right) \right]}{\Gamma ( 1 + x - \nu) \sin [ \pi ( \nu -
    x)]} \simeq \sin \frac{\pi \nu}{2} \cos \left[ \pi \left( x -
    \frac{\nu}{2} \right) \right] \Gamma ( \nu - x) \hspace{1em}, \; x \assign
    \frac{q}{2} + k - \frac{a}{2} > 0. &  \nonumber
  \end{eqnarray}
  Thus we see that it is sufficient to compute
  \begin{eqnarray}
    & \max \{ A_{\alpha_p, a} + B_{\alpha_p, a} + C_{\alpha_p, a} +
    D_{\alpha_p, a} \}_{( \alpha_p, a) \in \mathfrak{I}} &  \nonumber\\
    & A_{\alpha_p, a} \assign \mathfrak{P} \left( \frac{\Gamma ( 1 -
    \nu)}{\Gamma \left( \frac{1 - \nu}{2} \right)} \right) = \{ \nu \in
    2\mathbbm{Z}_{\geqslant 0} \} &  \nonumber\\
    & B_{\alpha_p, a} \assign \mathfrak{P} \left( \frac{\Gamma \left( \frac{2
    \nu - 2 k - n + \alpha_p + 1}{2} \right)}{\Gamma \left( \frac{2 \nu - 2 k
    - n + 1}{2} \right)} \right) = - \left\{ \nu - k + \frac{1 - n}{2} = 1 -
    \frac{\alpha_p}{2}, 2 - \frac{\alpha_p}{2}, \ldots, 0 \right\} & 
    \nonumber\\
    & C_{\alpha_p, a} \assign \mathfrak{P} \left( \frac{\Gamma \left( \nu -
    \frac{q}{2} - k + \frac{a}{2} \right)}{\Gamma \left( \frac{2 \nu - 2 k - q
    + \alpha_p}{2} \right)} \right) = \left\{ \nu - \frac{q}{2} - k +
    \frac{\alpha_p}{2} = 1 - \frac{a - \alpha_p}{2}, 2 - \frac{a -
    \alpha_p}{2}, \ldots, 0 \right\} &  \nonumber\\
    & D_{\alpha_p, a} \assign \mathfrak{P} \left( \sin \frac{\pi \nu}{2} \cos
    \left[ \pi \left( \frac{q + 2 k - a}{2} - \frac{\nu}{2} \right) \right]
    \right) = - \{ \nu \in 2\mathbbm{Z} \} - \{ \nu \in 1 + q + 2 k - a +
    2\mathbbm{Z} \} &  \nonumber
  \end{eqnarray}
  hence
  \[ \max \{ A_{\alpha_p, a} + B_{\alpha_p, a} + C_{\alpha_p, a} +
     D_{\alpha_p, a} \}_{( \alpha_p, a) \in \mathfrak{I}} = - \{ \nu \in
     2\mathbbm{Z}_{< 0} \} - \{ \nu \in 1 + q + 2\mathbbm{Z} \} \]
  and this directly implies the desired result.
\end{proof}



\section{Determination of $\mathcal{S} \tmop{ol} ( \mathbbm{R}^n ; \lambda,
\nu)$}\label{sec:sol-MO}

In this section we explicitly determine the space $\mathcal{S} \tmop{ol} (
\mathbbm{R}^n ; \lambda, \nu)$ for every $( \lambda, \nu) \in \mathbbm{C}^2$.

\subsection{Main results}

\begin{proposition}
  \label{sol-MO:prop-solonnonzero}We have
  \begin{eqnarray}
    & \mathcal{S} \tmop{ol} ( \mathbbm{R}^n - \{ 0 \} ; \lambda, \nu) =
    \left\{ \begin{array}{ll}
      \mathbbm{C} | Q |^{- \nu} \delta^{( 2 k)} ( x_p) \oplus \mathbbm{C} |
      x_p |^{\lambda + \nu - n} \delta^{( \nu - 1)} ( Q), & p = 1, \; (
      \lambda, \nu) \in \mid \mid \cap \backslash\backslash, k = -
      \frac{\lambda + \nu - n + 1}{2}\\
      \mathbbm{C} \frac{| x_p |^{\lambda + \nu - n}}{\Gamma \left(
      \frac{\lambda + \nu - n + 1}{2} \right)} \cdot \frac{| Q |^{-
      \nu}}{\Gamma \left( \frac{1 - \nu}{2} \right)}, & \tmop{otherwise}
    \end{array} \right. &  \nonumber\\
    & | | \assign \{ ( \lambda, \nu) \in \mathbbm{C}^2 | \nu \in
    2\mathbbm{Z}_{\geqslant 0} + 1 \}, &  \nonumber\\
    & \backslash\backslash \assign \{ ( \lambda, \nu) \in \mathbbm{C}^2 |
    \lambda + \nu - n \in - 1 - 2\mathbbm{Z}_{\geqslant 0} \} . &  \nonumber
  \end{eqnarray}
\end{proposition}

\begin{proposition}
	\label{sol-MO:prop-main}For $p = 1$ the basis of $\mathcal{S}\tmop{ol}( \mathbbm{R}^n ; \lambda, \nu)$ is
  \[=\left\{
     \begin{array}{ll}
       \tilde{K}_{\lambda, \nu}^{\mathbbm{R}^n}, & ( \lambda, \nu) \in
       \mathbbm{C}^2 - ( / / \cap L) - ( | | \cap \backslash\backslash)\\
       \widetilde{\tilde{K}}_{\lambda, \nu}^{\mathbbm{R}^n} \oplus
       \tilde{K}^{\{ 0 \}}_{\lambda, \nu}, & ( \lambda, \nu) \in ( / / \cap L)
       - ( | | \cap \backslash\backslash)\\
       \tilde{K}_{\lambda, \nu}^P \oplus \tilde{K}_{\lambda, \nu}^C, & (
       \lambda, \nu) \in ( | | \cap \backslash\backslash) - / /\\
       \tilde{K}^{\{ 0 \}}_{\lambda, \nu}, & ( \lambda, \nu) \in | | \cap
       \backslash\backslash \cap / /
     \end{array} \right. \]
  whereas for $p > 1$ it is 
  \[\left\{
     \begin{array}{ll}
\widetilde{\tilde{K}}_{\lambda, \nu_0}^{\mathbbm{R}^n}
       \oplus \tilde{K}^{\{ 0
       \}}_{\lambda, \nu}, & ( \lambda, \nu) \in / /, \nu \in L\\
       \tilde{K}^{\mathbbm{R}^n}_{\lambda, \nu}, & \tmop{otherwise}
     \end{array} \right. \]
     where $L$ is as in proposition \ref{KR-normalization-even:prop-odd} and 
     \[//:=\left\{ \left( \lambda,\nu \right)\in \mathbb{C} \mid\lambda-\nu\in-2\Z_{\ge0}\right\}.\]
\end{proposition}

\subsection{Auxiliary results}

\begin{lemma}
  \label{sol-MO:lem-E2}Suppose $\{ 0 \} \in O \subset \mathbbm{R}^n$ is open
  and $M$ a manifold. Suppose further that for $f \in \mathcal{D}' ( O \times
  M)$ we have $f$ supported inside $\{ 0 \} \times M$ and let $E = \sum x_i
  \frac{\partial}{\partial x_i}$ be an Euler operator on $\mathbbm{R}^n$ and
  $a \in \mathbbm{C}$. Then we have
  \[ ( E - a)^2 f = 0 \Rightarrow ( E - a) f = 0 \]
\end{lemma}

\begin{remark}
  This is a straightforward generalization of {\cite[lem.
  11.11]{kobayashi2015symmetry}} with proof done essentially in the same way.
\end{remark}

\begin{proof}
  As fact \ref{fact:sing-q-3} (together with the localization argument by fact
  \ref{fact:localization}) implies,
  \[ f = \sum_{\alpha} \delta^{( \alpha)} \otimes f_{\alpha}, \; f_{\alpha}
     \in \mathcal{D}' ( M) . \]
  and hence
  \[ ( E - a)^2 f = \sum_{\alpha} \delta^{( \alpha)} \otimes ( | \alpha |^{} -
     a)^2 f_{\alpha} = 0 \]
  hence $\forall \alpha, \; ( | \alpha |^{} - a)^2 f_{\alpha} = 0$. Now, for
  every fixed $\alpha$ the latter implies that either $( | \alpha | - a) = 0$
  or $f_{\alpha} = 0$, hence in any case $( | \alpha |^{} - a)^{} f_{\alpha} =
  0$ and thus $( E - a) f = 0$.
\end{proof}

\begin{lemma}
  \label{sol-MO:lem-strangeelement}If $p = 1$ and $k \in
  \mathbbm{Z}_{\geqslant 0}$ there is a distribution that we shall denote by
  \[ | Q |^{- \nu} \delta^{( 2 k)} ( x_p) \in \mathcal{D}' ( \mathbbm{R}^n
     \backslash \{ 0 \}) \]
  such that:
  \begin{enumerate}
    \item $| Q |^{- \nu} \delta^{( 2 k)} ( x_p)$ is holomorphic in $\nu \in
    \mathbbm{C}$
    
    \item $| Q |^{- \nu} \delta^{( 2 k)} ( x_p) \in \mathcal{S} \tmop{ol} (
    \mathbbm{R}^n \backslash \{ 0 \} ; \lambda, \nu)$ if $\lambda + \nu - n =
    - 1 - 2 k$;
    
    \item When restricted to $\{ Q \neq 0 \}$ $| Q |^{- \nu} \delta^{( 2 k)} (
    x_p)$ equals to the well-defined product of $\delta^{( 2 k)} ( x_p)$ and
    $| Q |^{- \nu} \in C^{\infty} ( \{ Q \neq 0 \})$;
  \end{enumerate}
\end{lemma}

\begin{proof}
  First of all we note that under the $p = 1$ hypothesis we have $\{ Q = 0 \}$
  and $\{ x_p = 0 \}$ having no common points within $\mathbbm{R}^n \backslash
  \{ 0 \}$. Now, by fact \ref{fact:localization} we can define $| Q |^{- \nu}
  \delta^{( 2 k)} ( x_p)$ locally. Namely, we define it to be zero on $\{ x_p
  \neq 0 \}$ and as product of distribution and smooth function on $\{ Q \neq
  0 \}$. As the well-definedness on intersection is evident, this furnishes
  the definition and readily proves the third item.
  
  Now, it suffices to show the holomorphicity of restrictions of $| Q |^{-
  \nu} \delta^{( 2 k)} ( x_p)$ to $\{ x_p \neq 0 \}$ and $\{ Q \neq 0 \}$. Thi
  former is clear, while the latter follows from lemma
  \ref{KR-normalization-recur:lem-mult-smth} and holomorphicity of
  distribution multiplication. This shows the first item. Finally, in the
  light of the first item it suffices to show the second item only when
  $\tmop{Re} ( - \nu) \gg 0$, so that $| Q |^{- \nu} \delta^{( 2 k)} ( x_p)$
  can be identified with the product of distributions $| Q |^{- \nu}$ and
  $\delta^{( 2 k)} ( x_p)$ (this following again from lemma
  \ref{KR-normalization-recur:lem-mult-smth}), where the desired is granted by
  propositions \ref{supp-R:prop-3} and \ref{KR-normalization-recur:prop-supp}.
\end{proof}

\begin{lemma}
  \label{sol-MO:lem-zeromap-point}Suppose $S \subset \mathbbm{R}^n$ is closed
  such that $\dim ( \mathcal{S} \tmop{ol}_S ( \mathbbm{R}^{p, q} \backslash \{
  0 \} ; \lambda, \nu)) \leqslant 1$. Suppose further $0 \in \Omega \subset
  \mathbbm{C}$ is an open set with $\lambda ( \cdot), \nu ( \cdot)$
  holomorphic on $\Omega$ and such that $\lambda ( \mu) - \nu ( \mu) - \mu =
  \tmop{const}$ on $\Omega$. Suppose further that $K_{}^{( \mu)} \in
  \mathcal{S} \tmop{ol}_S ( \mathbbm{R}^{p, q} ; \lambda ( \mu), \nu ( \mu))$,
  $K_{}^{( 0)}$ is supported at $\{ 0 \}$ and $( d / d \mu) |_{\mu = 0}
  K_{}^{( \mu)}$ is supported at closed subset bigger than $\{ 0 \}$.
  
  Then, the restriction map $\mathcal{S} \tmop{ol}_S ( \mathbbm{R}^{p, q} ;
  \lambda ( 0), \nu ( 0)) \rightarrow \mathcal{S} \tmop{ol}_S (
  \mathbbm{R}^{p, q} \backslash \{ 0 \} ; \lambda ( 0), \nu ( 0))$ is a zero
  map.
\end{lemma}

\begin{remark}
  This is essentially a slight generalization of {\cite[lemma
  11.8]{kobayashi2015symmetry}} with tecnhique of proof being exactly the
  same.
\end{remark}

\begin{proof}
  Indeed, suppose $F \in \mathcal{S} \tmop{ol}_S ( \mathbbm{R}^{p, q} ;
  \lambda ( 0), \nu ( 0))$. We will show that $F |_{\mathbbm{R}^n \backslash
  \{ 0 \}} = 0$. Indeed, we first note that expanding $K_{}^{( \mu)}$ in
  Taylor series near $\mu = 0$ we have
  \[ K_{}^{( \mu)} = K_0 + \mu \cdot K_1 + \mu^2 \cdot K_2 + \ldots \]
  and the hypothesis now implies that $K_0$ is supported at $\{ 0 \}$ and $K_1
  |_{\mathbbm{R}^n \backslash \{ 0 \}} \neq 0$.
  
  We also note that $K_1 \in \mathcal{S} \tmop{ol}_S ( \mathbbm{R}^n
  \backslash \{ 0 \} ; \lambda, \nu)$, as $F^{( \mu)} \assign K^{( \mu)} / \mu
  |_{\mathbbm{R}^n \backslash \{ 0 \}} \in \mathcal{S} \tmop{ol}_S (
  \mathbbm{R}^{p, q} \backslash \{ 0 \} ; \lambda ( \mu), \nu ( \mu))$ for
  $\mu \neq 0$ and as $F^{( 0)} = K_1$ and $F^{( \mu)}$ is holomorphic at $\mu
  = 0$, proposition \ref{sol:prop-holocont} implies that $K_1 \in \mathcal{S}
  \tmop{ol}_{} ( \mathbbm{R}^n \backslash \{ 0 \} ; \lambda, \nu)$. The fact
  that $K_1$ vanishes outside $S$ follows by continuity.
  
  Now, as $\dim ( \mathcal{S} \tmop{ol}_S ( \mathbbm{R}^{p, q} \backslash \{ 0
  \} ; \lambda, \nu)) \leqslant 1$ we should have
  \[ F |_{\mathbbm{R}^n \backslash \{ 0 \}} = c \cdot K_1 |_{\mathbbm{R}^n
     \backslash \{ 0 \}} \]
  and it suffices to show that $c = 0$.
  
  Next, hypothesis $\lambda ( \mu) + \nu ( \mu) - \mu = \tmop{const}$ implies
  that for some $a \in \mathbbm{C}$ we have $\lambda ( \mu) - \nu ( \mu) - n =
  a + \mu$ and as member of $\mathcal{S} \tmop{ol}_S ( \mathbbm{R}^n ;
  \lambda, \nu)$ should be homogeneous of degree $\lambda + \nu - n$, we
  should have
  \[ ( E - a) K^{( \mu)} = \mu \cdot K^{( \mu)}, \]
  and therefore by {\cite[lem. 11.10]{kobayashi2015symmetry}}, we have
  \[ ( E - a) K_0 = 0, \hspace{2em} ( E - a) K_1 = K_0 . \]
  Moreover, as $F \in \mathcal{S} \tmop{ol}_S ( \mathbbm{R}^{p, q} ; \lambda (
  0), \nu ( 0))$ by assumption, we should have $( E - a) F = 0$.
  
  Now, for $h \assign F - c \cdot K_1$ supported inside $\{ 0 \}$ distribution
  we should have $( E - a)^2 h = 0$, hence lemma \ref{sol-MO:lem-E2} implies
  that $( E - a) h = 0$ and therefore $0 = ( E - a) F = c ( E - a) K_1 = c
  \cdot K_0$, hence $c = 0$.
\end{proof}

\begin{lemma}
  \label{sol-MO:lem-zeromap}For exact sequence
  \[ 0 \rightarrow \mathcal{S} \tmop{ol}_C ( \mathbbm{R}^n - \{ 0 \} ;
     \lambda, \nu) \rightarrow \mathcal{S} \tmop{ol} ( \mathbbm{R}^n - \{ 0 \}
     ; \lambda, \nu) \xrightarrow{\pi} \mathcal{S} \tmop{ol} ( \mathbbm{R}^n -
     C ; \lambda, \nu) \]
  and when
  \begin{eqnarray}
    & ( \lambda, \nu) \in \left\{ \begin{array}{ll}
      | |, & p > 1\\
      | | \um \backslash\backslash, & p = 1
    \end{array} \right. &  \nonumber
  \end{eqnarray}
  with $\mid \mid$ and $\backslash\backslash$ as in proposition
  \ref{sol-MO:prop-solonnonzero}, we have $\pi = 0$.
\end{lemma}

\begin{proof}
  We assume
  \[ ( \lambda_0, \nu_0) \in \left\{ \begin{array}{ll}
       | |, & p > 1\\
       | | \um \backslash\backslash, & p = 1
     \end{array} \right. \]
  and let
  \[ K_{}^{( \nu)} \assign \frac{| x_p |^{\lambda_0 + \nu - n}}{\Gamma \left(
     \frac{\lambda_0 + \nu - n + 1}{2} \right)} \cdot \frac{| Q |^{-
     \nu}}{\Gamma \left( \frac{1 - \nu}{2} \right)} \in \mathcal{S} \tmop{ol}
     ( \mathbbm{R}^n - \{ 0 \} ; \lambda_0, \nu) . \]
  Now, if we write the Taylor expansion $K_{}^{( \nu)} = K_0 + ( \nu - \nu_0)
  K_1 + ( \nu - \nu_0)^2 K_2 + \ldots$, the hypothesis together with
  proposition \ref{KR-normalization-recur:prop-supp} imply that $K_0 \neq 0$
  is supported within $\{ Q = 0 \}$. Moreover, considering $K^{( \nu)} |_{\{ Q
  \neq 0 \}} / ( \nu - \nu_0)$ which is holomorphic at $\nu_0$, we note that
  $K_1 |_{\{ Q \neq 0 \}} \in \mathcal{S} \tmop{ol} ( \{ Q \neq 0 \} ;
  \lambda_0, \nu_0)$ and is nonzero (the latter follows, since we notice that
  $K_1 |_{\{ Q \neq 0 \}}$ is nonzero multiple of $\frac{| x_p |^{\lambda_0 +
  \nu_0 - n}}{\Gamma \left( \frac{\lambda_0 + \nu_0 - n + 1}{2} \right)} \cdot
  | Q |^{- \nu_0} |_{\{ Q \neq 0 \}}$, which is nonzero).
  
  Now, we assume that $F \in \mathcal{S} \tmop{ol} ( \mathbbm{R}^n \backslash
  \{ 0 \} ; \lambda_0, \nu_0)$ and show that $F |_{\{ Q \neq 0 \}}$=0, this
  will be sufficient to end the proof. As $\mathcal{S} \tmop{ol} ( \{ Q \neq 0
  \} ; \lambda_0, \nu_0)$ is at most-one dimensional (by proposition
  \ref{lem67:prop-67}) and $0 \neq K_1 |_{\{ Q \neq 0 \}} \in \mathcal{S}
  \tmop{ol} ( \{ Q \neq 0 \} ; \lambda_0, \nu_0)$, we have $K_1 |_{\{ Q \neq 0
  \}} = c \cdot F |_{\{ Q \neq 0 \}}$ and it now suffices to show that $c =
  0$.
  
  We now let $U \assign \{ ( x, y) \in \mathbbm{R}^{p, q} | x \neq 0, y \neq 0
  \}$, $k^{( \nu)} \assign K^{( \nu)} |_U \in \mathcal{S} \tmop{ol} ( U ;
  \lambda_0, \nu)$ with corresponding Taylor expansion $k^{( \nu)} = k_0 + (
  \nu - \nu_0) k_1 + \ldots$ with $k_i = K_i |_U$, and $f \assign F |_U \in
  \mathcal{S} \tmop{ol} ( U ; \lambda_0, \nu_0)$. The above implies that $c
  \cdot k_1 |_{\{ Q \neq 0 \}} = f |_{\{ Q \neq 0 \}}$. We also note that as
  $\mathcal{D}' ( \mathbbm{R}^n \backslash \{ 0 \}) \ni K_0 \neq 0$ and
  supported within $\{ Q = 0 \}$, we have $k_0 \neq 0$ (as $\{ Q = 0 \}$ and
  $\mathbbm{R}^p \times \{ 0 \} \cup \{ 0 \} \times \mathbbm{R}^q$ are closed
  and disjoint inside $\mathbbm{R}^n \backslash \{ 0 \}$). Now, lemma
  \ref{supp-Q:lem-operator} implies that (passing to $( \mu, s, \omega_{p -
  1}, \omega_{q - 1})$ coordinates) we have $\left[ \nu_0 - ( \mu - 1)
  \frac{\partial}{\partial \mu} \right] F = 0$ and $\left[ \nu - ( \mu - 1)
  \frac{\partial}{\partial \mu} \right] k^{( \nu)} = 0$. Moreover, the latter
  and {\cite[lem. 11.10]{kobayashi2015symmetry}} imply than that
  \[ \left[ \nu_0 - ( \mu - 1) \frac{\partial}{\partial \mu} \right] k_0 = 0,
     \hspace{1em} \left[ \nu_0 - ( \mu - 1) \frac{\partial}{\partial \mu}
     \right] k_1 + k_0 = 0. \]
  Now, we have $h \assign F_0 - c K_1 \in \mathcal{D}' ( U)$ being supported
  inside $\{ Q = 0 \}$ and moreover that $\left[ \nu_0 - ( \mu - 1)
  \frac{\partial}{\partial \mu} \right]^2 h = 0$. Hence, lemma
  \ref{sol-MO:lem-E2} implies that $\left[ \nu_0 - ( \mu - 1)
  \frac{\partial}{\partial \mu} \right]^{} h = 0$ and thus $c \left[ \nu_0 - (
  \mu - 1) \frac{\partial}{\partial \mu} \right] k_1 = c k_1 = 0$ and as $k_1
  \neq 0$, this implies that $c = 0$ and we are done.
\end{proof}

\subsection{Proofs}

\begin{proof}
  (of prop. \ref{sol-MO:prop-main}) First of all, we recall that we have an
  exact sequence
  \[ 0 \rightarrow \mathcal{S} \tmop{ol}_C ( \mathbbm{R}^n - \{ 0 \})
     \rightarrow \mathcal{S} \tmop{ol} ( \mathbbm{R}^n - \{ 0 \})
     \xrightarrow{\pi} \mathcal{S} \tmop{ol} ( \mathbbm{R}^n - C) \]
  and propositions \ref{supp-Q:prop-onedim} and \ref{lem67:prop-67}
  respectively tell us that
  \begin{eqnarray}
    & \mathcal{S} \tmop{ol}_C ( \mathbbm{R}^{p, q} \backslash \{ 0 \} ;
    \lambda, \nu) =\mathbbm{C} \left\{ \begin{array}{ll}
      \delta^{( \nu - 1)} ( Q) \cdot | x_p |^{\lambda + \nu - n}, & p = 1, \nu
      \in 2\mathbbm{Z}_{\geqslant 0} + 1\\
      \delta^{( \nu - 1)} ( Q) \cdot \frac{| x_p |^{\lambda + \nu - n}}{\Gamma
      ( ( \lambda + \nu - n + 1) / 2)}, & p > 1, \nu \in
      2\mathbbm{Z}_{\geqslant 0} + 1\\
      0, & \nu \nin 2\mathbbm{Z}_{\geqslant 0} + 1
    \end{array} \right. &  \nonumber\\
    & \mathcal{S} \tmop{ol} ( \mathbbm{R}^n - C ; \lambda, \nu) =\mathbbm{C}
    | Q |^{- \nu} \frac{| x_p |^{\lambda + \nu - n}}{\Gamma \left(
    \frac{\lambda + \nu - n + 1}{2} \right)}, &  \nonumber
  \end{eqnarray}
  and so the thing is to know whether $\pi$ is onto or zero map for given $(
  \lambda, \nu) \in \mathbbm{C}^2$. Now, as we have (see prop.
  \ref{KR-normalization-recur:prop-supp})
  \[ \frac{| x_p |^{\lambda + \nu - n}}{\Gamma \left( \frac{\lambda + \nu - n
     + 1}{2} \right)} \cdot \frac{| Q |^{- \nu}}{\Gamma \left( \frac{1 -
     \nu}{2} \right)} \in \mathcal{S} \tmop{ol} ( \mathbbm{R}^n - \{ 0 \} ;
     \lambda, \nu), \]
  it is clear that $\pi$ is onto for $( \lambda, \nu) \nin \mid \mid$ and this
  gives the part of an answer. Moreover, if $p = 1$, we have lemma
  \ref{sol-MO:lem-strangeelement} (more specifically, its second and third
  items) giving us that $\pi$ is onto as well when $( \lambda, \nu) \in \mid
  \mid \cap \backslash\backslash$. On the other hand, for other cases lemma
  \ref{sol-MO:lem-zeromap} tells us that $\pi = 0$ and this ends the proof.
\end{proof}

\begin{proof}
  (of prop. \ref{sol-MO:prop-main}) We employ an exact sequence
  \[ 0 \rightarrow \mathcal{S} \tmop{ol}_{\{ 0 \}} ( \mathbbm{R}^n)
     \rightarrow \mathcal{S} \tmop{ol} ( \mathbbm{R}^n) \xrightarrow{\pi}
     \mathcal{S} \tmop{ol} ( \mathbbm{R}^n \um \{ 0 \}) \]
  and again as $\mathcal{S} \tmop{ol}_{\{ 0 \}} ( \mathbbm{R}^n)$ and
  $\mathcal{S} \tmop{ol} ( \mathbbm{R}^n \um \{ 0 \})$ are explicitly
  determined by propositions \ref{diffSBO:prop-main} and
  \ref{sol-MO:prop-solonnonzero} respectively, the thing is to know what is
  the image of $\pi$ for given $( \lambda, \nu) \in \mathbbm{C}^2$.
  
  We first consider the $p > 1$ case, when $\mathcal{S} \tmop{ol} (
  \mathbbm{R}^n - \{ 0 \} ; \lambda, \nu)$ is spanned by an element $\frac{|
  x_p |^{\lambda + \nu - n}}{\Gamma \left( \frac{\lambda + \nu - n + 1}{2}
  \right)} \cdot \frac{| Q |^{- \nu}}{\Gamma \left( \frac{1 - \nu}{2}
  \right)}$. For $( \lambda, \nu) \nin / /$ we have $\pi$ being onto, as the
  image of $\tilde{K}_{\lambda, \nu}^{\mathbbm{R}^n}$ covers the generating
  elements of $\mathcal{S} \tmop{ol} ( \mathbbm{R}^n - \{ 0 \} ; \lambda,
  \nu)$. Now, for $( \lambda_0, \nu_0) \in / /$ with $\nu_0 \nin L$ we have
  $\tilde{K}_{\lambda_0, \nu_0}^{\mathbbm{R}^n}$ being nonzero and supported
  at $\{ 0 \}$ (hence it is a multiple of $\tilde{K}_{\lambda, \nu}^{\{ 0
  \}}$) and if we let $K_{\lambda} \assign \tilde{K}_{\lambda,
  \nu_0}^{\mathbbm{R}^n}$, we have that it satisfies the hypothesis of lemma
  \ref{sol-MO:lem-zeromap-point} with $S =\mathbbm{R}^n$ (in particular, the
  part about the support of the derivative is granted by proposition
  \ref{KR-normalization-recur:prop-supp}) and hence $\pi = 0$ in this case.
  Finally, if $( \lambda_0, \nu_0) \in / /$ and $\nu_0 \in L$ we have element
  $K_{\lambda} \assign \widetilde{\tilde{K}}_{\lambda, \nu_0}^{\mathbbm{R}^n}$
  satisfying the hypothesis of lemma \ref{sol-MO:lem-zeromap-point} (again
  with $S =\mathbbm{R}^n$), hence $\pi$ is onto in this case by remark
  \ref{KR-normalization-even:rmk-Atilde}.
  
  We now turn to $p = 1$ case. First of all, for $( \lambda, \nu) \in / / \cup
  ( | | \cap \backslash\backslash)$ we have $\tilde{K}_{\lambda,
  \nu}^{\mathbbm{R}^n}$ being nonzero and $\mathcal{S} \tmop{ol} (
  \mathbbm{R}^n - \{ 0 \} ; \lambda \comma \nu)$ is one-dimensional, and we
  see that $\pi$ is onto in this case. Similaly, we have $\pi$ being onto for
  $( \lambda, \nu) \in ( / / \cap L) - ( | | \cap \backslash\backslash)$ (this
  time however, the generator of $\mathcal{S} \tmop{ol} ( \mathbbm{R}^n - \{ 0
  \} ; \lambda \comma \nu)$ is covered not by $\tilde{K}_{\lambda,
  \nu}^{\mathbbm{R}^n}$, but by $\widetilde{\tilde{K}}_{\lambda,
  \nu}^{\mathbbm{R}^n}$). \ Next, for $( \lambda_{}, \nu) \in / / - L - ( \mid
  \mid \cap \backslash\backslash)$ we have $\pi = 0$ by lemma
  \ref{sol-MO:lem-zeromap-point} (applied to $K_{\mu} = \tilde{K}_{\mu,
  \nu}^{\mathbbm{R}^n}$). This gives the first two rows of an answer for $p =
  2$ given in the statement.
  
  We next consider the case $( \lambda, \nu) \in ( | | \cap
  \backslash\backslash) - / /$. Propositions \ref{KP-normalization-2:prop-p=1}
  and \ref{KC-normalization-2:prop-supp} tell us that $\tilde{K}_{\lambda,
  \nu}^P$ and $\tilde{K}_{\lambda, \nu}^C$ are both nonzero and their
  respective images under $\pi$ cover the two generators of $\mathcal{S}
  \tmop{ol} ( \mathbbm{R}^n - \{ 0 \} ; \lambda, \nu)$ listed in proposition
  \ref{sol-MO:prop-solonnonzero}. Hence, $\pi$ is onto in this case. Finally,
  let us assume that $( \lambda_0, \nu_0) \in | | \cap \backslash\backslash
  \cap / /$. As the latter set is nonempty only when $q \in 2\mathbbm{Z}$, we
  may assume it is so in subsequent. In this case $\tilde{K}_{\lambda, \nu}^P$
  and $\tilde{K}_{\lambda, \nu}^C$ are both nonzero and supported at $\{ 0
  \}$. We can introduce linear maps $\lambda, \lambda', \nu, \nu' :
  \mathbbm{C} \rightarrow \mathbbm{C}$ such that for any $\mu \in
  \mathbbm{C}$, we have $( \lambda ( \mu), \nu ( \mu)) \in | |$, $( \lambda' (
  \mu), \nu' ( \mu)) \in \backslash\backslash$ and $( \lambda_{} ( 0), \nu (
  0)) = ( \lambda' ( 0), \nu' ( 0)) = ( \lambda_0, \nu_0)$. We then let
  $K^C_{( \mu)} \assign \tilde{K}_{\lambda ( \mu), \nu ( \mu)}^C$ and $K^P_{(
  \mu)} \assign \tilde{K}^P_{\lambda' ( \mu), \nu' ( \mu)}$ and introduce the
  Taylor series expansions
  \[ K^C_{( \mu)} = K^C_0 + \mu K_1^C + \ldots, \hspace{2em} K^P_{( \mu)} =
     K_0^P + \mu K_1^P + \ldots \]
  It can be seen (inspecting the way we defined $K_{\lambda, \nu}^C$ and
  $K_{\lambda, \nu}^P$ and normalized them) that $K_1^C$ and $K_1^P$ have
  their supports equal to $C$ and $P$ (hence, linear independent) respectively
  and their restrictions to $\mathbbm{R}^n - \{ 0 \}$ are elements of
  $\mathcal{S} \tmop{ol} ( \mathbbm{R}^n - \{ 0 \} ; \lambda_0, \nu_0)$, hence
  they span $\mathcal{S} \tmop{ol} ( \mathbbm{R}^n - \{ 0 \} ; \lambda_0,
  \nu_0)$. We claim that $\pi = 0$. Indeed, suppose $F \in \mathcal{S}
  \tmop{ol} ( \mathbbm{R}^n ; \lambda_0, \nu_0)$. We then have $F
  |_{\mathbbm{R}^n - \{ 0 \}} = a K_1^C |_{\mathbbm{R}^n - \{ 0 \}} + b K_1^P
  |_{\mathbbm{R}^n - \{ 0 \}}$ for some $( a, b) \in \mathbbm{C}^2$ and then
  the proof of lemma \ref{sol-MO:lem-zeromap-point} goes through (with
  $K_{\mu} \assign a K_{( \mu)}^C + b K^P_{( \mu)}$) to show that $F
  |_{\mathbbm{R}^n - \{ 0 \}} = 0$. Since $F$ was arbitrary, we are done.
\end{proof}

\section{Application: Knapp-Stein operator}\label{sec:knappstein}

The material of this section is not used elsewhere in the paper, we rather use
it to show how techniques introduced in {\cite{kobayashi2015symmetry}} and
used in this paper can be applied to find a concrete form of $G$-invariant $I
( \lambda) \rightarrow I ( n - \lambda)$ operator in a relatively elementary
and straightforward way.

\subsection{Main results}

\begin{definition}
  \label{knappstein:def-n+invar}For $F \in \D' (U)$, where $U \subset
  \mathbbm{R}^{p, q}$ is an open set, we say that $F$ is
  \tmtextbf{$N_+$-invariant on $U$} if $\forall b \in \mathbbm{R}^{p, q}$ and
  $x_0 \in U$ such that $\frac{x_0 - Q (x_0) b}{1 - 2 Q (x_0, b) + Q (x_0) Q
  (b)} \in U$ and the expression makes sense (i.e. the denominator is
  non-zero) we have
  \begin{equation}
    \label{knappstein:eq-Nequiv} | 1 - 2 Q (b, x) + Q (x) Q (b) |^{\lambda -
    n} F \left( \frac{x - Q (x) b}{1 - 2 Q (x, b) + Q (x) Q (b)} \right) = F
    (x)
  \end{equation}
  equality holding for $x$ near $x_0$.
\end{definition}

\begin{definition}
  \label{knappstein:def-sol}For $F \in \D' (U)$, where $U \subset
  \mathbbm{R}^{p, q}$ is an open set, we say that $F \in \mathcal{S}
  \tmop{ol}_{( G, G)} ( U ; \lambda, \nu)$ if the following holds:
  \begin{enumerate}
    \item if $x_0 \in U$ and $- x_0 \in U$, then $F (x) = F (- x)$ for $x$
    near $x_0$;
    
    \item if $(m, x_0, m \cdot x_0) \in O ( p, q) \times U \times U$, then $F
    (x) = F (m \cdot x)$ for $x$ near $x_0$;
    
    \item if $(\alpha, x_0, \alpha x_0) \in \mathbbm{R}_{> 0} \times U \times
    U$, then $\alpha^{\lambda - \nu - n} F (x) = F (\alpha x)$ for $x$ near
    $x_0$;
    
    \item $F$ is $N_+$-invariant on $U$.
  \end{enumerate}
\end{definition}

\begin{remark}
  Compare definitions \ref{knappstein:def-n+invar} and
  \ref{knappstein:def-sol} with definitions \ref{def-n-nots:def-n+invar} and
  \ref{sol:def-sol} respectively.
\end{remark}

\begin{proposition}
  \label{knappstein:prop-1}For $\lambda \in \mathbbm{C}$, $I ( \lambda)
  \assign \mathcal{L}_{\lambda} \assign G \times_{\lambda}
  \mathbbm{C}_{\lambda}$ for $G \assign O ( p + 1, q + 1)$, $n \assign p + q$,
  $P = M A N \subset G$ with $A \assign \exp ( t H)$, $H \assign E_{1, p + q +
  2} + E_{p + q + 2, 1}$ maximal parabolic and $P \curvearrowright
  \mathbbm{C}_{\lambda}$ by $( m \exp ( t H) n, x) \mapsto e^{t \lambda} x$ we
  have:
  \begin{enumerate}
    \item $\tmop{Hom}_G ( I ( \lambda), I ( \nu)) \simeq \mathcal{S}
    \tmop{ol}_{( G, G)} ( \mathbbm{R}^n ; \lambda, \nu)$
    
    \item In previous item isomorphism $\tmop{Hom}_G ( I ( \lambda), I ( \nu))
    \tilde{\rightarrow} \mathcal{S} \tmop{ol}_{( G, G)} ( \mathbbm{R}^n ;
    \lambda, \nu)$ is given by composition
    \[ \tmop{Hom}_G ( I ( \lambda), I ( \nu)) \tilde{\rightarrow} \mathcal{D}'
       ( ( G / P)^2, \mathbbm{C}_{\nu} \boxtimes \mathbbm{C}_{n -
       \lambda})^{\Delta ( G)} \tilde{\rightarrow} ( \mathcal{D}' ( G / P,
       \mathbbm{C}_{n - \lambda}) \otimes \mathbbm{C}_{\nu})^{\Delta ( P)}
       \tilde{\rightarrow} \mathcal{S} \tmop{ol}_{( G, G)} ( \mathbbm{R}^n ;
       \lambda, \nu) \]
    where maps from left to right are:
    \begin{enumerate}
      \item kernel, given by Schwartz kernel theorem {\cite[thm.
      5.2.1]{hormander1983analysis}};
      
      \item induced by multiplication map $G^2 \ni ( g, g') \mapsto ( g')^{-
      1} g \in G$;
      
      \item restriction to open Bruhat cell $N_+ \subset G / P$;
    \end{enumerate}
  \end{enumerate}
\end{proposition}

\begin{proposition}
  \label{knappstein:prop-holo}With notation as in \ref{knappstein:prop-1}, we
  have:
  \begin{enumerate}
    \item for $\tmop{Re} ( \lambda) \gg 0$ we have:
    \[ | Q |^{\lambda - n} \in \mathcal{S} \tmop{ol}_{( G, G)} (
       \mathbbm{R}^n, \lambda, n - \lambda) ; \]
    \item For $N_0$ as in proposition \ref{q-norm:prop-1}, \ref{q-norm:prop-2}
    or \ref{q-norm:prop-pqzero} (whichever is applicable) we have $| Q
    |^{\lambda - n} / N_0$ extending to holomorphic in $\lambda \in
    \mathbbm{C}$ nonvanishing element of $\mathcal{S} \tmop{ol}_{( G, G)} (
    \mathbbm{R}^n ; \lambda, n - \lambda)$. We will denote corresponding
    element of $\tmop{Hom}_G ( I ( \lambda), I ( n - \lambda))$ (given by
    prop. \ref{knappstein:prop-1}) by $\tilde{\mathbbm{T}}_{\lambda}$.
  \end{enumerate}
\end{proposition}

\begin{proposition}
  \label{knappstein:prop-kfinite}For $| Q |^{\lambda - n} \in \mathcal{S}
  \tmop{ol}_{( G, G)} ( \mathbbm{R}^n, \lambda, n - \lambda)$ for $\tmop{Re} (
  \lambda) \gg 0$, let $K_{\lambda, \nu}^S \in \mathcal{D}' ( \mathbbm{S}^p
  \times \mathbbm{S}^q)$ be as given by proposition (given by proposition
  \ref{k-finite:prop-holo-to-holo}; note that $\mathcal{S} \tmop{ol}_{( G, G)}
  ( U ; \lambda, \nu) \subseteq \mathcal{S} \tmop{ol} ( U ; \lambda, \nu)$).
  Then, for every $F \in \mathcal{H}^a ( \mathbbm{S}^p) \otimes \mathcal{H}^b
  ( \mathbbm{S}^q) \subset C^{\infty} ( \mathbbm{S}^p \times \mathbbm{S}^q)$
  with $a + b \in 2\mathbbm{Z}$ we have
  \begin{equation}
    \langle K^S_{\lambda, \nu}, F \rangle = \sum'_{( n', m') \in \mathfrak{I}}
    k_{n', m'} \cdot \varphi_{n', m'}
  \end{equation}
  where $\mathfrak{I} \assign \{ ( n', m') \in \mathbbm{Z}_{\geqslant 0}^2 |
  n' + m' \in 2\mathbbm{Z} \}$, $k_{n', m'}$ are some entire nonzero functions
  in $( \lambda, \nu) \in \mathbbm{C}^2$ (depending only on $F$, $p, \; q$ and
  $n', m'$), $\sum'$ denotes finite sum and
  \[ F \simeq \frac{\Gamma \left( \frac{\lambda - n + 1}{2} \right) \Gamma
     \left( \lambda - \frac{n}{2} \right) \Gamma \left( \frac{- \lambda +
     n}{2} + \frac{m' + n'}{2} \right)}{\Gamma \left( \frac{\lambda + n' - m'
     - q + 1}{2} \right) \Gamma \left( \frac{\lambda + m' - n' - p + 1}{2}
     \right) \Gamma \left( \frac{\lambda + n' + m'}{2} \right) \Gamma \left(
     \frac{- \lambda + n}{2} \right)} \]
  Moreover, for every $( n', m') \in \mathfrak{I}$ there exists $F \in
  \sum_{i, a_i + b_i \in 2\mathbbm{Z}}' \mathcal{H}^{a_i} ( \mathbbm{S}^p)
  \otimes \mathcal{H}^{b_i} ( \mathbbm{S}^q)$ such that $\langle K_{\lambda,
  \nu}^S, F \rangle \simeq \varphi_{n', m'}$.
\end{proposition}

\begin{remark}
  The latter result could also be found in {\cite[thm. 3.9.1]{KO1}}, as it is
  shown there that symmetry breaking operator $\mathbbm{C}_{\lambda}
  \rightarrow \mathbbm{C}_{- \lambda}$ has its kernel on $( \mathbbm{S}^p
  \times \mathbbm{S}^q)^2$ being equal (up to normalization) to $( x, y)
  \mapsto | Q_{p + 1, q + 1} ( x, y) |^{\lambda - n / 2}$ (where $Q_{p + 1, q
  + 1}$ denotes indefinite inner product on $\mathbbm{R}^{p + 1, q + 1}$) and
  its eigenvalues on $K$-finite vectors were computed.
  
  With the correspondence of proposition \ref{knappstein:prop-1}, this kernel
  is precisely $| Q |^{\lambda - n / 2} \in \mathcal{D}' ( \mathbbm{R}^{p, q})
  \nosymbol$. We note that in {\cite{KO1}} degenerate principal series
  $\mathbbm{C}_{\lambda}$ are parametrized differently, with the two
  parametrizations being connected by $\mathbbm{C}_{\lambda - n / 2} = I (
  \lambda)$.
\end{remark}

\subsection{Proofs}

\begin{proof}
  (of prop. \ref{knappstein:prop-1}) the statement follows from {\cite[thm.
  3.16]{kobayashi2015symmetry}} similarly to proposition \ref{sol:prop-sol}
  (note that requirement $P N_- P = G$ is satisfied, as we've shown in
  proposition \ref{doublePGP:prop-pnp} that $P' N_- P = G$ for $P' \assign G'
  \cap G \subset P$).
\end{proof}

\begin{proof}
  (of prop. \ref{knappstein:prop-holo}) The first items follows by direct
  check of equations of proposition \ref{knappstein:prop-1}, as we have $| Q
  |^{\lambda - n} \in C^1 ( \mathbbm{R}^n)$ for $\tmop{Re} ( \lambda) \gg 0$.
  The second item follows from propositions \ref{q-norm:prop-1},
  \ref{q-norm:prop-2} and an analogue of proposition \ref{sol:prop-holocont}.
\end{proof}

{\noindent}\tmtextbf{Fact \tmtextup{20}.
}\tmtextit{\label{knappstein:fact-faraut}{\cite[appendix
B]{faraut1979distributions}} With $\simeq$ denoting proportionality up to
finite nonvanishing multiple, $C^{\lambda}_k ( \cdot)$ being Gegenbauer
polynomials and $s \in \mathbbm{C}$, we have
\begin{eqnarray*}
  & \frac{1}{\Gamma ( ( s + 1) / 2)} \int_{\sigma_1, \sigma_n = - 1}^1 |
  \tmop{ch} ( t) \sigma_1 - \tmop{sh} ( t) \sigma_n |^s C_l^{( p - 2) / 2} (
  \sigma_1) C^{( q - 2) / 2}_m ( \sigma_n) ( 1 - \sigma_1^2)^{( p - 3) / 2} (
  1 - \sigma_2^2)^{( p - 3) / 2} d \sigma_1^{} d \sigma_2 \simeq & \\
  & \simeq \frac{s ( s - 2) \ldots ( s - m - l + 2)}{\Gamma ( ( s + l - m +
  p) / 2)} ( \tmop{th} t)^m ( \tmop{ch} t)^s _2 F_1 \left( \frac{- s + m +
  l}{2}, \frac{- s - p + 2 + m - l}{2} ; m + \frac{q}{2} ; \tmop{th}^2 t
  \right) . & 
\end{eqnarray*}}{\hspace*{\fill}}{\medskip}

\begin{proof}
  (of prop. \ref{knappstein:prop-kfinite}) Similarly to proof of proposition
  \ref{k-finite:prop-KR-hook-1}, one arrives at the integral
  \begin{eqnarray*}
    & \int_{r, s = 0}^{\infty} | r^2 - s^2 |^{\lambda - n} R^{- \lambda / 2}
    \left( \frac{r}{\sqrt{R}} \right)^N \left( \frac{s}{\sqrt{R}} \right)^M
    \left( \frac{1 - ( r^2 - s^2)}{\sqrt{R}} \right)^{n'} \left( \frac{1 + (
    r^2 - s^2)}{\sqrt{R}} \right)^{m'} d r d s \times & \\
    & \times \int_{\mathbbm{S}^p} \psi ( \omega_p) d \omega_p \times
    \int_{\mathbbm{S}^q} \psi' ( \omega_q) d \omega_q . & 
  \end{eqnarray*}
  where $R \assign ( 1 - r^2 + s^2)^2 + 4 r^2$. As $(\psi, \psi') \in
  \mathcal{H}^N ( \mathbb{S}^{p - 1}) \times \mathcal{H}^M ( \mathbb{S}^{q -
  1}) \nocomma$, we see that one of the last two multiplicands (hence, the
  whole integral) vanishes if $N > 0$ or $M > 0$.
  
  Consequently, we are left with only the first multiplicand. Now, using the
  variable change $x = ( 1 + r^2 - s^2) / \sqrt{R}$ and $y = ( 1 - r^2 + s^2)
  / \sqrt{R}$, the first multiplicand of the expression above can be rewritten
  as
  \[ \simeq \int_{x, y = - 1}^1 | x - y |^{\lambda - n} x^{m'} y^{n'} ( 1 -
     x^2)^{( q - 2) / 2} ( 1 - y^2)^{( p - 2) / 2} d x d y. \]
  Now, if we let for $m', n' \in \mathbbm{Z}_{\geqslant 0}^2$ with $m' + n'
  \in 2\mathbbm{Z}$,
  \[ C^{\infty} ( \mathbbm{R}^{p, q}) \ni F_{m', n'} \assign R^{- \lambda / 2}
     C_{n'}^{( p - 1) / 2} \left( \frac{1 - ( r^2 - s^2)}{\sqrt{R}} \right)
     C_{m'}^{( q - 1) / 2} \left( \frac{1 + ( r^2 - s^2)}{\sqrt{R}} \right) \]
  with $C^{\lambda}_k ( \cdot)$ denoting Gegenbauer polynomials, the
  computations above imply all of the statements we need except of the formula
  for $\langle | Q |^{\lambda - n}, F_{m', n'} \rangle_{\mathbbm{R}^{p, q}}$
  (recall that $| Q |^{\lambda - n} \in C^1$ for $\tmop{Re} ( \lambda) \gg
  0$). The latter, however, is readily given by fact
  \ref{knappstein:fact-faraut} upon letting $t \rightarrow \infty$, as we have
  $\tmop{th} ( + \infty) = 1$,
  \[ _2 F_1 ( a, b ; c ; 1) = \frac{\Gamma ( c) \Gamma ( c - a - b)}{\Gamma (
     c - a) \Gamma ( c - b)}, \hspace{1em} \tmop{Re} ( c - a - b) > 0 \]
  and $s ( s - 2) \ldots ( s - N + 2) \simeq \Gamma ( - s / 2 + N / 2) /
  \Gamma ( - s / 2)$.
\end{proof}

\begin{thebibliography}{CK{\O}P11}
  \bibitem[CK{\O}P11]{clerc2011generalized}J.-L. Clerc, T.~Kobayashi,
  B.~{\O}rsted, and M.~Pevzner. {\newblock}Generalized bernstein--reznikov
  integrals. {\newblock}\tmtextit{Mathematische Annalen}, 349(2):395--431,
  2011.
  
  \bibitem[CP82]{chazarain2011introduction}J.~Chazarain and A.~Piriou.
  {\newblock}\tmtextit{Introduction to the theory of linear partial
  differential equations}. {\newblock}North-Holland Publishing Company, 1982.
  
  \bibitem[Del98]{delorme1998plancherel}P.~Delorme. {\newblock}Plancherel
  formula for reductive symmetric spaces. {\newblock}147(2):417--452, 1998.
  
  \bibitem[Far79]{faraut1979distributions}J.~Faraut. {\newblock}Distributions
  sph{\'e}riques sur les espaces hyperboliques. {\newblock}\tmtextit{J. Math.
  Pures Appl.}, 58:369--444, 1979.
  
  \bibitem[GGP12]{gan2011symplectic}W.~T. Gan, B.~H. Gross, and D.~Prasad.
  {\newblock}Sur les conjectures de Gross et Prasad.
  {\newblock}\tmtextit{Ast{\'e}risque}, 2012.
  
  \bibitem[GGV66]{gelfand1966generalized}I.~M. Gelfand, M.~I. Graev, and N.~Y.
  Vilenkin. {\newblock}\tmtextit{Generalized functions. Vol. 5, Integral
  geometry and representation theory}. {\newblock}Academic Press, 1966.
  
  \bibitem[GS69]{gelfand1980distribution}I.~M. Gelfand and G.~E. Shilov.
  {\newblock}\tmtextit{Generalized functions. Vol. 1, Properties and
  operations}. {\newblock}Academic Press, 1969.
  
  \bibitem[HC76]{harishchandra1978harmonic}Harish-Chandra. {\newblock}Harmonic
  analysis on real reductive groups III. The Maass-Selberg relations and the
  Plancherel formula. {\newblock}\tmtextit{Annals of Mathematics},
  104(1):117--201, 1976.
  
  \bibitem[H{\"o}r83]{hormander1983analysis}L.~H{\"o}rmander.
  {\newblock}\tmtextit{The Analysis of Linear Partial Differential Operators:
  Vol.: 1.: Distribution Theory and Fourier Analysis}.
  {\newblock}Springer-Verlag, 1983.
  
  \bibitem[HT93]{howe1993homogeneous}R.~E. Howe and E.-C. Tan.
  {\newblock}Homogeneous functions on light cones: the infinitesimal structure
  of some degenerate principal series representations.
  {\newblock}\tmtextit{Bulletin of the American Mathematical Society},
  28(1):1--74, 1993.
  
  \bibitem[Juh09]{juhl2009families}A.~Juhl. {\newblock}\tmtextit{Families of
  conformally covariant differential operators, Q-curvature and holography},
  volume 275. {\newblock}Springer Science \& Business Media, 2009.
  
  \bibitem[KM14]{kobayashi2014classification}T.~Kobayashi and T.~Matsuki.
  {\newblock}Classification of finite-multiplicity symmetric pairs.
  {\newblock}\tmtextit{Transformation Groups}, 19(2):457--493, 2014.
  
  \bibitem[K{\O}03]{KO1}T.~Kobayashi and B.~{\O}rsted. {\newblock}Analysis on
  the minimal representation of \tmtextrm{O}$(p, q)$.\tmtextrm{I}. Realization
  via conformal geometry. {\newblock}\tmtextit{Adv. Math.}, 180:486--512,
  2003.
  
  \bibitem[Kob94]{kobayashi1994discrete1}T.~Kobayashi. {\newblock}Discrete
  decomposability of the restriction of $A_q (\lambda)$ with respect to
  reductive subgroups and its applications. {\newblock}\tmtextit{Inventiones
  mathematicae}, 117(1):181--205, 1994.
  
  \bibitem[Kob98a]{kobayashi1998discrete2}T.~Kobayashi. {\newblock}Discrete
  decomposability of the restriction of $A_q (\lambda)$ with respect to
  reductive subgroups II: Micro-local analysis and asymptotic K-support.
  {\newblock}\tmtextit{Annals of mathematics}, pages 709--729, 1998.
  
  \bibitem[Kob98b]{kobayashi1998discrete3}T.~Kobayashi. {\newblock}Discrete
  decomposability of the restriction of $A_q (\lambda)$ with respect to
  reductive subgroups III. restriction of Harish-Chandra modules and
  associated varieties. {\newblock}\tmtextit{Inventiones mathematicae},
  131(2):229--256, 1998.
  
  \bibitem[Kob15]{kobayashi2015program}T.~Kobayashi. {\newblock}A program for
  branching problems in the representation theory of real reductive groups.
  {\newblock}In \tmtextit{Representations of Reductive Groups}, pages
  277--322. Springer, 2015.
  
  \bibitem[KP15a]{kobayashi2015differential1}Toshiyuki Kobayashi and Michael
  Pevzner. {\newblock}Differential symmetry breaking operators: I. General
  theory and F-method. {\newblock}\tmtextit{Selecta Mathematica}, pages 1--45,
  2015.
  
  \bibitem[KP15b]{kobayashi2015differential2}Toshiyuki Kobayashi and Michael
  Pevzner. {\newblock}Differential symmetry breaking operators: II.
  Rankin--Cohen operators for symmetric pairs. {\newblock}\tmtextit{Selecta
  Mathematica}, pages 1--65, 2015.
  
  \bibitem[Kra82]{krantz1982function}S.~G. Krantz.
  {\newblock}\tmtextit{Function theory of several complex variables}.
  {\newblock}Pure and applied mathematics. Wiley, 1982.
  
  \bibitem[KS15]{kobayashi2015symmetry}T.~Kobayashi and B.~Speh.
  {\newblock}Symmetry breaking for representations of rank one orthogonal
  groups. {\newblock}\tmtextit{Memoirs of the American Mathematical Society},
  238(1126), 2015.
  
  \bibitem[OM84]{oshima1984description}T.~Oshima and T.~Matsuki. {\newblock}A
  description of discrete series for semisimple symmetric spaces.
  {\newblock}\tmtextit{Adv. Stud. Pure Math}, 4:331--390, 1984.
  
  \bibitem[Wal88]{wallach1988real}N.~Wallach. {\newblock}\tmtextit{Real
  Reductive Groups I}, volume 132 of \tmtextit{Pure and Applied Mathematics}.
  {\newblock}Academic press, 1988.
  
  \bibitem[War71]{warner1971foundations}F~Warner. {\newblock}Foundations of
  differential geometry and Lie groups. {\newblock}\tmtextit{Scott Foresman
  and Company, Glenview}, 1971.
\end{thebibliography}

\end{document}
