%japanese
\documentclass[8pt,notes,notheorems]{beamer}
\mode<presentation>{\usetheme[secheader]{Boadilla}}
\usepackage{mystyle}
\usepackage{xcolor}
\usepackage{xeCJK}
\usepackage{amsmath,amssymb,bbm,xypic}
\includecomment{versiona}
\usepackage{tcolorbox}
\usepackage{etoolbox}
\usepackage{pifont}
%%\usepackage{bbding}
\usepackage{scalerel}
%%\usepackage{adjustbox}
%%\usepackage{tcolorbox}
%%\usepackage{lipsum}
%%\usepackage{calc}
%%\usepackage{graphicx}
%%\usepackage{amsmath}
%%\usepackage{amssymb}
%%\usepackage{relsize}
%%\usepackage{multirow}
%%\usepackage{rotating}
%%\usepackage{bm}
%%\usepackage{url}
%%\usepackage{mystyle}
%%\usepackage{enumerate}
%%\usepackage{geometry}
%%\usepackage{setspace}
%%\usepackage{amsmath,amssymb,bbm,xypic}
\usepackage[all,cmtip]{xy}
%%\usepackage{amsmath,amssymb,bbm,float,mystyle}
%%\usepackage[normalem]{ulem}
%%\usepackage{caption}
%%\usepackage{subcaption}
%%\usepackage{setspace}
%%\usepackage{catchfilebetweentags}
%%\usepackage{multirow}
%%\usepackage{bbm}
%%\usepackage{graphicx}
%%\usepackage{multicol}
%%\usepackage{array}
%%\usepackage{stmaryrd}
%%\usepackage{CJKutf8}	

%\usepackage{times}
%\usepackage{helvet}
%\usepackage{bookman}
\usepackage{palatino}

\newcommand{\glue}[2][7.5cm]{\makebox[#1]{#2\hfill}}
\newcommand{\redunderline}[1]{ {\Huge\color{red}{\underline{\color{black}{{\small #1}}}}} }
\newcommand{\red}[1]{{\color[rgb]{0.6,0,0}#1}}
\newcommand{\Hom}{\mbox{\normalfont Hom}}

\makeatletter
\newenvironment<>{proofs}[1][\proofname]{\par\def\insertproofname{#1\@addpunct{.}}\usebeamertemplate{proof begin}#2}
{\usebeamertemplate{proof end}}
\makeatother

\makeatletter
\def\th@mystyle{%
	\normalfont % body font
	\setbeamercolor{block title example}{bg=orange,fg=white}
	\setbeamercolor{block body example}{bg=orange!20,fg=black}
	\def\inserttheoremblockenv{exampleblock}
}
\makeatother

\theoremstyle{mystyle}
\newtheorem{prop}{Proposition}
\newtheorem{example*}{例}
\setCJKmainfont{Hiragino Mincho Pro}
\newcommand{\mybox}[2]{
	\begin{tcolorbox}[colback=green!5,colframe=orange!40,title=#1]#2\end{tcolorbox}
}

\makeatletter
\setbeamertemplate{footline}
{
  \leavevmode%
  \hbox{%
  \begin{beamercolorbox}[wd=.333333\paperwidth,ht=2.25ex,dp=1ex,center]{author in head/foot}%
    \usebeamerfont{author in head/foot} レオンチエフ オレクシィ東京大学、DC 面接資料
  \end{beamercolorbox}%
  \begin{beamercolorbox}[wd=.333333\paperwidth,ht=2.25ex,dp=1ex,center]{title in head/foot}%
    \usebeamerfont{title in head/foot} 研究話題『対称性破れの作用素』
  \end{beamercolorbox}%
  \begin{beamercolorbox}[wd=.333333\paperwidth,ht=2.25ex,dp=1ex,right]{date in head/foot}%
    \usebeamerfont{date in head/foot}\insertshortdate{}\hspace*{2em}
    \insertframenumber{} / \inserttotalframenumber\hspace*{2ex} 
  \end{beamercolorbox}}%
  \vskip0pt%
}
\makeatother

\title{対称性破れの作用素}
\author{レオンチエフ オレクシィ\\東京大学 大学院数理科学研究科\\博士課程1年生\\[2em]特別研究員DC2 受人研究者: 小林俊行教授}
\date{平成28年12月5日}

\begin{document}
\begin{frame}\titlepage\end{frame}
\begin{frame}{DC申請書を提出した後の活動(平成28年5月--11月):}
	\mybox{}{
	(講演)
	\begin{enumerate}
		\item \glue{広島大学幾何セミナー、\smallskip 広島大学}{\underline{7月19日}}
		\item Workshop on ``Actions of Reductive Groups and Global Analysis''、東京大学\\\glue{玉原国際セミナーハウス}\underline{8月11日}
		\item \glue{日本数学会2016年度秋季総合分科会、関西大学} \underline{9月18日}
		\item \glue{広島幾何学研究集会2016、広島大学} \underline{10月7日}
		\item \glue{日本数学会 異分野・異業種研究交流会2016、明治大学} \underline{11月19日}
		\item \glue{Symposium on Representation Theory、沖縄} \underline{11月30日}
	\end{enumerate}
	\vspace{2em}
	(論文)--査読なし
	\begin{enumerate}
		\setcounter{enumi}{6}
		\item Symmetry breaking operators of indefinite orthogonal groups $O(p,q)$、日本数学会2016年度秋季総合、幾何学分科会 講演アブストラクト,2pages,
		\item Symmetry breaking operators for representations of indefinite orthogonal groups $O(p,q)$、Symposium on Representation Theory 2016, pp. 39--52.
		\item Symmetry breaking of conformal transformation group $O(p,q)$、日本数学会2017年度年会 函数解析学分科会 講演アブストラクト,2pages (出版予定).
	\end{enumerate}
}
\end{frame}
\begin{frame}{{\large \mbox{これまでの研究業績A$-$対称性破れ作用素の構成と分類(〜2016年11月)}}}
	\mybox{\color{black}表現論(線形の対称性)の重要問題}{
		\vspace{-0.5em}
		\begin{itemize}
			\item 『最小の単位』を理解する $-$ \hspace{0.1cm}既約表現の分類, \dots
			\item 『最小の単位』 に分解する $-$ \redunderline{\normalsize 分岐則}\hspace{-0.1cm}, \dots
		\vspace{-1.5em}
		\end{itemize}
	}
	\vspace{-0.5em}
	\mybox{\color{black}分規則とは?}{
		ベクトル空間$V$の大きな対称性群($G$の表現)から小さな対称性
		(部分群$G'$の表現)に落としたとき、何が起こるかを理解する。\\[0.5em]
		{\bf 例.}\quad テンソル積の分解の記述。
\begin{itemize}
	\item \glue[4.5cm]{$\;\dim V<\infty$, $G$はコンパクト\dots} 分岐則は古くから良く研究されている
	\item \glue[4.5cm]{\redunderline{{\normalsize$\dim V=\infty$, $G$は非コンパクト\dots}}} 分岐則は難しく、本格的な研究はようやく\\
		\glue[4.5cm]{}\mbox{1990年頃から始まった}
		\vspace{-0.5em}
\end{itemize}
		}\vspace{-0.5em}
		\mybox{\color{black}研究のテーマ:対称性破れの作用素({$\dim V=\infty$ のとき})}{\vspace{-1em}
	\xymatrixrowsep{0.6cm}
	\xymatrixcolsep{1cm}
	\begin{equation*}
\xymatrix{
	{{\begin{array}{c}\mbox{大きな群}\\ G\end{array}}\ar@/_/[d]_{\mbox{既約表現}}}&\subset&{\begin{array}{c}\mbox{小さな群}\\ G'\end{array}}\ar@/^/[d]^{\mbox{既約表現}}\\
	V\ar@{-->}[rr]^{\mbox{\fbox{対称性破れ作用素}}}&&W
}
\vspace{-1em}
	\end{equation*}
		}\vspace{-0.5em}
\end{frame}
\begin{frame}{{\large \mbox{これまでの研究業績B$-$対称性破れ作用素の構成と分類(〜2016年11月)}}}
\mybox{\color{black}主定理}{
	高階の非コンパクトなリー群の組
	\begin{equation*}
		(G, G')=(O(p+1,q), O(p,q))
	\end{equation*}
	の球退化主系列表現の間の\redunderline{{\normalsize 対称性破れ作用素}}\hspace{-0.1cm}を幾何的に構成し、それを完全に分類した。
}
\mybox{\color{black}関連する研究}{
	\begin{itemize}
		\item 先駆的結果(最初の完全な分類結果)\\
			{Kobayashi-Speh, Memoirs of AMS, 112 pages, 2015(q=1の場合に対応する)\hspace{-0.1cm};}
		\item 微分作用素として表される対称性破れ作用素の研究\\
			\mbox{Rankin$-$Cohen, Zagier, Juhl, Fefferman$-$Graham, Kobayashi$-$Pevzner (2016), \dots}
		\item 競合する動き(2016年12月現在、未出版)\\
			Kobayashi$-$Speh (2015)の出版後、Clerc(フラーンス)、Gomez(米国)、{\O}rsted(デンマーク)、M\"ollers(ドイツ)、Zhang(スウェデン) 等が関連研究を始めている。
	\end{itemize}
}
\end{frame}
\begin{frame}{主定理の証明$-$4部構成}
	\mybox{\color{black}『対称性破れ作用素の幾何的構成と分類』の証明ー4部構成}
	{
		\begin{enumerate}[\hspace{-0.5cm}I]
			\item (群論)群$O(p+1,q)$のある両側剰余類とその閉包関係の決定\vspace{0.5em}
			\item (幾何的構成)パートIのsubvarietyに台をもつ超函数を核とする積分変換の構成\vspace{0.5em}
			\item {(解析接続)パートIIで構成された超函数の、表現パラメータに関する解析接続の証明\vspace{0.5em}}
			\item (分類の完成)対称性破れ作用素の関数等式やスペクトラム.留数の決定\vspace{0.5em}
		\end{enumerate}
	}
	\mybox{\color{black}定理の証明ー総計約100〜120頁の論文を準備中}{
		\begin{tabular}{lclclcll}
			I&+&II&+&III&+&IV=&主定理\\
			21頁&+&28頁&+& 25頁&+&33頁.&(現在、加筆・修正中)
		\end{tabular}
	}
\end{frame}
\begin{frame}{今後の研究計画A(2019年3月まで)}
		 \begin{enumerate}[(1)]
			 \item 所属研究室(小林俊行教授)の特徴
				 \begin{itemize}
					 \item[$\ast$]新しい分野を興し、世界をリードする研究を行っている。
					 \item[$\ast$] 世界中からアクティブな研究者が訪れ、文献にない情報が入る。
					 \item[$\ast$]メンバーの専攻分野は代数・幾何・解析と多岐に亘る。
				 \end{itemize}
			 \item 私の研究計画の特徴
				 \begin{itemize}
					 \item[$\ast$]先行研究はランク1の場合に限られている。
					 \item[$\ast$] 一方、私の研究は高階の群に対する最初の『対称性破れ作用素の研究』への挑戦である。
				      \item[$\ast$]扱っている設定は、共形幾何、および、整数論の特別な問題と
					ゆるやかな繋がりがある。
				 \end{itemize}
			 \item 具体的計画
				 \begin{itemize}
					 \item[A](1年目):対称性破れ作用素の\redunderline{特殊値を分析}\hspace{-0.08cm}することによって、
					 部分商として現れるZuckerman導来函手加群の分岐則における重複度を
					 決定する
				 \item[B ] (2年目):をベクトル束値に設定を
					 して\redunderline{一般化}\hspace{-0.1cm}し、特に微分形式の空間における対称性破れ作用素の構成・分類を実行する\vspace{1em}
				\item[] {\huge \dots} 最近の研究を実現しつつ、関連する隣接分野の知識も少しずつ増やす。
				 \end{itemize}
		 \end{enumerate}
\end{frame}
\begin{frame}{今後の研究計画B(2019年3月までの2年間)}
	\begin{figure}[h]
  	\begin{center}
		\includegraphics[scale=0.4]{/Users/nailbiter/Downloads/futur}
	  \vspace{-1em}
	  \end{center}
	\end{figure}
%%				\begin{enumerate}[A]
%%					\item $O(p+1,q)\downarrow O(p,q)$に関する
%%						退化主系列表現の対称性破れ作用素
%%						 (構成と分類-2016年度)
%%					\item (未解決問題)Zuckerman導来函手加群の
%%						分岐則。\\2017年度は、この未解決問題を、
%%						退化主系列表現のsubquotientの場合に解決することを目指す
%%					\item 対称性破れ作用素の``特殊値''の分析、Selberg型積分の分析
%%					\item モジュラー多様体の位相的性質
%%					\item Gross-Prasad予想($O_n\downarrow O_{n-1}$)
%%					\item ホッジ理論の精密化(松島$-$村上$-$Borel$-$Wallach$-$Voan$-$Zuckerman)   小さなフォントを使う
%%					\item ベクトル束への一般化
%%					\item 符号$(p,q)$をもつ擬リーマン多様体とその超曲面における対称性破れ作用素の構成と分類。\\
%%					2018年度は、モデル空間$\Sp^p\times\Sp^q$上の微分形式に作用する対称性破れ作用素の分類を目指す。
%%				\end{enumerate}
\end{frame}
\end{document}


