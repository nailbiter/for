\documentclass[12pt]{article} % use larger type; default would be 10pt

\usepackage{textcomp} %for copyleft symbol
\usepackage{mathtext}                 % підключення кирилиці у математичних формулах
                                          % (mathtext.sty входить в пакет t2).
\usepackage[T1,T2A]{fontenc}         % внутрішнє кодування шрифтів (може бути декілька);
                                          % вказане останнім діє по замовчуванню;
                                          % кириличне має співпадати з заданим в ukrhyph.tex.
\usepackage[utf8]{inputenc}       % кодування документа; замість cp866nav
                                          % може бути cp1251, koi8-u, macukr, iso88595, utf8.
\usepackage[english,ukrainian]{babel} % національна локалізація; може бути декілька
                                          % мов; остання з переліку діє по замовчуванню. 

\usepackage{sectsty}   %in order to make chapter headings and title centered
\chapterfont{\centering}

\usepackage{amsthm}
\usepackage{amsmath}
\usepackage{amssymb}
\usepackage{amsfonts}
\usepackage{graphicx}
\usepackage[pdftex]{hyperref}
\usepackage{caption}
\usepackage{subfig}
\usepackage{fancyhdr}
\usepackage{enumerate}
\usepackage{enumitem}

%custom commands to save typing
\newcommand{\mynorm}[1]{\left|\left|#1\right|\right|}
\newcommand{\myabs}[1]{\left|#1\right|}
\newcommand{\myset}[1]{\left\{#1\right\}}

%put subscript under lim and others
\let\oldlim\lim
\renewcommand{\lim}{\displaystyle\oldlim}
\let\oldmin\min
\renewcommand{\min}{\displaystyle\oldmin}
\let\oldmax\max
\renewcommand{\max}{\displaystyle\oldmax}

\newtheorem{prob}{Завдання}

\title{
Асимптотичні методи в теорії диференційних рівнянь\\
Контрольна робота (8 семестр)\\
No. 4}
\author{Олексій Леонтьєв}
\begin{document}
\maketitle
\begin{prob}Знайти розв’язок, що задовольняє задані початкові умови, у вигляді степеневого ряду. Обчислити декілька перших
коефіцієнтів ряду (до коефіцієнту $x^4$ включно).\end{prob}
По-перше, оскільки рівняння є нелінійним ми не можемо безпосередньо використати Теорему 1 з конспекту щоб гарантувати існування розв’язку у вигляді
ряду в околі точки 0 (до речі, оскільки в завданні не вказано явно, відносно якої точки має бути центрованим степеневий ряд, ми для зручності
вважаємо центром 0, точку, де задані початкові умови). На відміну від цього, ми помітимо, що дане рівняння можна розглядати як диференційне рівняння
на $\mathbb{C}$ і у такому разі його права частина є аналітичною за обома змінними на всьому $\mathbb{C}^2$ (в комплексному аналізі, нагадаємо,
аналітичність є наслідком простої диференційованості, яка очевидно присутня). Оскільки права частина є неперервною (більше того, вона
диференційована, як зауважено вище), вона є обмеженою на довільному компакті $K\subset\mathbb{C}^2$, а отже і на довільній множини, що міститься в
компакті (наприклад, на добутку відкритих дисків в $\mathbb{C}$)
. Це повністю задовольняє гіпотезу теореми 8.1 з \cite{coddington
}, що в свою чергу дозволяє нам застосувати останню. Це і гарантує існування аналітичного
в околі нуля розв’язку рівняння. Аналітичність в $\mathbb{C}$ є сильнішою за аналітичність в $\mathbb{R}$, що виправдовує коректність подальших
маніпуляцій.

Враховуючи початкові умови, запишемо розв’язок у вигляді $y(x)=c_1 x+c_2 x^2+c_3x^3+c_4x^4+o(x^4)$ і підставимо його у 
\[y'=2x+\cos y\]
оскільки ми шукаємо розв’язок з точністю до $o(x^4)$ і $\cos y=1-y^2/2+y^4/4!+\dots$ маємо
\[c_1+2c_2x+3c_3x^2+4c_4x^3+5c_5x^4+o(x^4)=2x+1-\frac{(c_1 x+c_2 x^2+c_3x^3+c_4x^4+o(x^4))^2}{2}+\]
\[+\frac{(c_1 x+c_2 x^2+c_3x^3+c_4x^4+o(x^4))^4}{24}+o(x^4)\]
Розкриваючи дужки і збираючи що можливо в $o(x^4)$ отримуєм
\[c_1+2c_2x+3c_3x^2+4c_4x^3+5c_5x^4=2x+1-\frac{c_1^2}{2}x^2-\frac{c_2^2}{2}x^4-c_1c_2x^3-c_1c_3x^4+\frac{c_1^4}{24}x^4+o(x^4)\]
а отже
\[\begin{cases}
	c_1=1\\
	2c_2=2\\
	3c_3=-\frac{c_1^2}{2}\\
	4c_4=-c_1c_2\\
	5c_5=-\frac{c_2^2}{2}-c_1c_3+\frac{c_1^4}{24}
\end{cases}
\]
Таким чином, отримали відповідь
\[y(x)=x+x^2-\frac{1}{6}x^3-\frac{1}{4}x^4+o(x^4)\]
\begin{prob}Використовуючи інтегрування за допомогою рядів, знайти розв’язок рівняння в околі точки  $x_0=0$. Там, де це можливо,
виразити розв’язок через елементарні функції. Знайти область збіжності одержаних рядів. Знайти загальний розв’язок рівняння.\end{prob}
	Це рівняння повністю підпадає під гіпотезу Теореми 2 з конспекту ($p(x)=1,\;q(x)=x^2-1/4$)
	, і таким чином ми почнемо розв’язок з виписування характеристичного рівняння
	\[\lambda(\lambda-1)+\lambda-\frac{1}{4}=0\iff \lambda=\pm\frac{1}{2}\]
	Оскільки $1/2-(-1/2)\notin\mathbb{Z}$, фундаментальний розв’язок можна записати у вигляді
	\[y(x)=a_0\sum_{k=0}^{\infty}c_k x^{k+\frac{1}{2}}+a_1\sum_{k=0}^{\infty}\tilde{c_k}x^{k-\frac{1}{2}}\]
	де $a_0,\;a_1\in\mathbb{R}$ - довільні сталі. 
	
	Підставимо $y_1(x)=\sum_{k=0}^{\infty}c_kx^{k+1/2}$ у наше рівняння і отримаєм
	\[x^2\sum_{k=0}^{\infty}c_k\left(k+\frac{1}{2}\right)
	\left(k-\frac{1}{2}\right)x^{k+\frac{1}{2}-2}
+x\sum_{k=0}^{\infty}c_k\left(k+\frac{1}{2}\right)x^{k+\frac{1}{2}-1}+\left(x^2-\frac{1}{4}\right)\sum_{k=0}^{\infty}c_kx^{k+\frac{1}{2}}=0\]
	\[\sum_{k=0}^{\infty}c_k\left(k+\frac{1}{2}\right)
	\left(k-\frac{1}{2}\right)x^{k+\frac{1}{2}}+\sum_{k=0}^{\infty}c_k\left(k+\frac{1}{2}\right)
	x^{k+\frac{1}{2}}+\sum_{k=2}^{\infty}c_{k-2}x^{k+\frac{1}{2}}-\sum_{k=0}^{\infty}\frac{c_k}{4}x^{k+\frac{1}{2}}=0\]
	Скоротивши на $x^{\frac{1}{2}}$ і прірівнявши до нуля коефіцієнти ряду зліва маємо
	\[c_0\left(-\frac{1}{4}\right)+c_0\cdot\frac{1}{2}-\frac{c_0}{4}=0\]
	\[c_1\cdot\frac{3}{4}+c_1\cdot\frac{3}{2}-\frac{c_1}{4}=0\implies c_1=0\]
	\[c_k(k^2+k)+c_{k-2}=0\]
	Перше з вищенаписаних рівнянь тривіалізується і не накладає жодних обмежень на $c_0$, тому, оскільки Теорема 2 вимагає $c_0\neq 0$,
	можем покласти $c_0=1$. З другого і третього рівнянь маємо $c_{2k+1}=0$, а $c_{2k}=\frac{c_{2k-2}}{2k(2k+1)}\implies
	c_{2k}=1/(2k+1)!\implies y_1(x)=\sum_{k=0}^{\infty}\frac{x^{2k+\frac{1}{2}}}{(2k+1)!}$

	Підставимо тепер $y_2(x)=\sum_{k=0}^{\infty}c_kx^{k-1/2}$
	\[x^2\sum_{k=0}^{\infty}c_k\left(k-\frac{1}{2}\right)
	\left(k-\frac{3}{2}\right)x^{k-\frac{1}{2}-2}
+x\sum_{k=0}^{\infty}c_k\left(k-\frac{1}{2}\right)x^{k-\frac{1}{2}-1}+\left(x^2-\frac{1}{4}\right)\sum_{k=0}^{\infty}c_kx^{k-\frac{1}{2}}=0\]
	\[\sum_{k=0}^{\infty}c_k\left(k-\frac{1}{2}\right)
	\left(k-\frac{3}{2}\right)x^{k-\frac{1}{2}}+\sum_{k=0}^{\infty}c_k\left(k-\frac{1}{2}\right)
	x^{k-\frac{1}{2}}+\sum_{k=2}^{\infty}c_{k-2}x^{k-\frac{1}{2}}-\sum_{k=0}^{\infty}\frac{c_k}{4}x^{k-\frac{1}{2}}=0\]
	Скоротивши на $x^{-\frac{1}{2}}$ і прірівнявши до нуля коефіцієнти ряду зліва маємо
	\[\frac{3}{4}c_0-\frac{c_0}{2}-\frac{c_0}{4}=0\]
	\[-\frac{c_1}{4}+\frac{c_1}{2}-\frac{c_1}{4}=0\]
	\[c_k(k^2-k)+c_{k-2}=0\]
	Маємо цікаву ситуацію, адже тепер не лише перше, а й друге рівняння тривіально виконуються, отже ми не маєм обмежень на $c_1$, тому
	$y_2$ ніби розпадається на два: $y_1(x)=A\cdot
	\sum_{k=0}^{\infty}\frac{x^{2k-\frac{1}{2}}}{(2k)!}+B\cdot\sum_{k=0}^{\infty}\frac{x^{2k+1-\frac{1}{2}}
	}{(2k+1)!}$. На щастя, ми легко впізнаємо в другому доданку $y_1$, що дозволяє записати загальний розв’язок
	\[y(x)=a_0\sum_{k=0}^{\infty}\frac{x^{2k+\frac{1}{2}}}{(2k+1)!}+a_1\sum_{k=0}^{\infty}\frac{x^{2k-\frac{1}{2}}}{(2k)!}\]
	із областю збіжності $x\in(0,+\infty)$ (що узгоджується з Теоремою 2, адже $p(x)$ та $q(x)$ є поліномами)
	. На думку автора, розв’язки не виражаються в елементарних функціях.
\begin{prob}Знайти три члени розкладу розв’язку за степенями малого параметру $\mu$.\end{prob}
	Оскільки і права частина рівняння і початкові умови є гладкими в околі $(x_0,y_0,\mu_0)=(1,1,0)$, за теоремою
	про диференційованість за початковими даними та параметрами, можем припустити, що до розв’язка $y(x,\mu)$ можна застосувать
	формулу Тейлора $y(x,\mu)=y_0(x)+\mu y_1(x)+\mu^2y_2(x)+o(\mu^2)$. Підставляючи це в рівняння маємо
	\[y_0'(x)+\mu y_1'(x)+\mu^2y_2'(x)+o(\mu^2)=\frac{6\mu}{x}-(y_0(x)+\mu y_1(x)+\mu^2y_2(x)+o(\mu^2))^2\]
	розкриваємо дужки і отримуємо
	\[y_0'(x)+\mu y_1'(x)+\mu^2y_2'(x)=\frac{6\mu}{x}-y_0^2(x)-\mu^2y_1^2(x)-2\mu y_0(x)y_1(x)-2\mu^2y_0(x)y_2(x)+o(\mu^2)\]
	а також
	\[y_0(1)+\mu y_1(1)+\mu^2 y_2(1)+\dots=1+\mu\cdot 3+\mu^2\cdot0+\dots\]
	і отже 
	\[\begin{cases}
		y_0'(x)=-y_0^2(x)\implies y_0(x)=\frac{1}{x}\\
		y_1'(x)=\frac{6}{x}-2y_0(x)y_1(x)\implies y_1(x)=3\\
		y_2'(x)=-y_1^2(x)-2y_0(x)y_2(x)=-9-\frac{2}{x}y_2(x)\implies y_2(x)=-3x
	\end{cases}\]
\begin{thebibliography}{9}
\bibitem{coddington}
	Earl A. Coddington, Norman Levinson
\emph{Theory of Ordinary Differential Equations}. Tata McGraw-Hill Publishing Co. LTD., New Delhi, TMH Edition, 1972 (Ерл. А. Коддінгтон,
Норман Левінсон \textit{Теорія звичайних диференційних рівнянь}. Видавництво ata McGraw-Hill LTD., Нью-Делі, Видання TMH, 1972 р.)
\end{thebibliography}
\end{document}
