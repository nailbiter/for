\documentclass[12pt]{article} % use larger type; default would be 10pt

\usepackage{mathtext}                 % підключення кирилиці у математичних формулах
                                          % (mathtext.sty входить в пакет t2).
\usepackage[T1,T2A]{fontenc}         % внутрішнє кодування шрифтів (може бути декілька);
                                          % вказане останнім діє по замовчуванню;
                                          % кириличне має співпадати з заданим в ukrhyph.tex.
\usepackage[utf8]{inputenc}       % кодування документа; замість cp866nav
                                          % може бути cp1251, koi8-u, macukr, iso88595, utf8.
\usepackage[english,russian,ukrainian]{babel} % національна локалізація; може бути декілька
                                          % мов; остання з переліку діє по замовчуванню. 
\usepackage{empheq}
\usepackage{mystyle}

\newtheorem{prob}{Завдання}
\newcommand{\ds}{\;ds}
\newcommand{\dt}{\;dt}
\newcommand{\dx}{\;dx}
\newcommand{\dy}{\;dy}
\newcommand{\dta}{\;d\tau}
\newcommand{\extr}{\mbox{\normalfont extr}}
\newcommand{\Var}{\mbox{Var}}

\newtheorem{myulem}[mythm]{Лема}

\renewenvironment{myproof}[1][Доведення]{\begin{trivlist}
\item[\hskip \labelsep {\bfseries #1}]}{\myqed\end{trivlist}}

\title{Математична статистика (10 семестр)}
\author{Олексій Леонтьєв}

\begin{document}
\maketitle
\section{Глава 1}
\setcounter{prob}{30}
\begin{prob}\end{prob}
	Вважатимемо $\theta_0$ відомим параметром. Тоді $L(h)>0$ тільки якщо $\forall i,\;h>\myabs{\xi_i-\theta_0}$ і якщо ця
	умова виконується, то $L(h)=1/(2h)^n$ і таким чином
	\[\hat{h}=\max_i\mycbra{\myabs{\xi_i-\theta_0}}\]
	Таким чином так, $\hat{h}$ є оцінкою. Розглядаючи її як випадкову величину, її розподіл визначається наступним чином.
	Кожна з величин $\myabs{\xi_i-\theta_0}$ має розподіл
	\[\bar{f}(x;h)=\begin{cases}
		1/h,&\mbox{якщо }0<x<h\\
		0,&\mbox{в інших випадках}
	\end{cases}\]
	Це рівномірний розподіл, і як розраховано в \cite{turchin}, очікування мінімуму вибірки рівне
	\[M\hat{h}=\frac{n}{n+1}h+\frac{0}{n+1}\to h\]
	Таким чином, оцінка не є незміщеною, але є асимптотично незміщеною.
	Відповідно,
	\[P(\myabs{h-\hat{h}}>\epsilon)=\mybra{1-\frac{\epsilon}{h-0}}^n\to0\]
	і таким чином, оцінка є конзистентною.
\begin{prob}\end{prob}
	Перш за все, 
	\[m_1(\alpha,\theta)=\int_{\mathbb{R}^1} x\frac{1}{\alpha}e^{-(x-\theta)/\alpha}\dx=\int_0^\infty(x+\theta)\frac{1}{\alpha}
	e^{-x/\alpha}\dx=\int_0^\infty(\alpha x+\theta)e^{-x}\dx=\alpha+\theta\]
	і відповідно
	\[m_2(\alpha,\theta)=\int_0^\infty (x+\theta)^2\frac{1}{\alpha}e^{-(x-\theta)/\alpha}\dx=
	\int_0^\infty(\alpha^2x^2+2\alpha x\theta+\theta^2)e^{-x}\dx=2\alpha^2+2\alpha\theta+\theta^2=\alpha^2+(\alpha+\theta)^2\]
	і таким чином
	\[\hat{\alpha}=\sqrt{\frac{1}{n}\sum\xi_i^2-\mybra{\frac{1}{n}\sum\xi_i}^2}\]
	\[\hat{\theta}=\frac{1}{n}\sum\xi_i-\sqrt{\frac{1}{n}\sum\xi_i^2-\mybra{\frac{1}{n}\sum\xi_i}^2}\]
	Це оцінки. Жодна з них не є незміщеною, адже для $n=1,\;M\hat{\alpha}=0,\;M\hat{\theta}=M\xi=\alpha\neq\theta$. Щодо
	конзистентності, перший момент $\frac{1}{n}\sum\xi_i$ та другий момент $\frac{1}{n}\sum\xi_i^2$ є конзистентними
	оцінками відповідно $\mathbb{E}\xi=\alpha+\theta$ та $\mathbb{E}\xi^2=\alpha^2+(\alpha+\theta)^2$. Далі, конзистентність
	$\hat{\alpha}$ та $\hat{\theta}$ випливає з теореми Слуцького.
\begin{prob}\end{prob}
	По-перше, перевіримо, чи є ця оцінка незміщеною.
	\[\mathbb{E}\hat{\theta}=\frac{1}{2}\mathbb{E}\xi^2=\frac{1}{2}\int_0^\infty\frac{x^3}{\theta}\exp\mycbra{-\frac{x^2}{2
	\theta}}\dx=\frac{1}{4}\int_0^\infty\frac{y}{\theta}\exp\mycbra{-\frac{y}{2\theta}}\dy=\frac{4\theta}{4}={\theta}.\]
	Оцінка незміщена. Далі, перевіримо умови нерівності Крамера-Рао, як сформульовано в \cite{wiki_regularity}.
	\begin{enumerate}
		\item \[\frac{\partial}{\partial\theta}\mybra{\frac{x}{\theta}\exp\mycbra{-\frac{x^2}{\theta}}}=
			\mybra{-\frac{x}{\theta^2}+\frac{x^2}{\theta^2}}\exp\mycbra{-\frac{x^2}{\theta}}\] та 
			\[\frac{\partial^2}{\partial\theta^2}\mybra{\frac{x}{\theta}\exp\mycbra{-\frac{x^2}{\theta}}}=
			\mybra{\frac{2x}{\theta^3}-\frac{2x^2}{\theta^3}-\frac{x^3}{\theta^4}+\frac{x^4}{\theta^4}}
			\exp\mycbra{-\frac{x^2}{\theta}}\] існують для
			$\theta>0$.
		\item Покажемо, що для кожного $\theta_0>0$ має місце
			\[\frac{\partial}{\partial\theta}\mysbra{\int T(x)f(x;\theta)\dx}=\int T(x)\mysbra{\frac{\partial}{\partial
			\theta}f(x;\theta)}\dx\]
			де $T(x)=T(x_1,x_2,\hdots,x_n)$ статистика така, що $\forall\theta\;\mathbb{E}_\theta T(x),\;\mathbb{E}_\theta
			T^2(x)<+\infty$.
			Згідно з \cite[Гл. 13, Тм. 17]{dorogovcev} достатньо перевірити рівномірну на околі $\theta_0$
			збіжність інтегралу в правій частині. Маємо
			\[\int_a^\infty T(x)\mysbra{\frac{\partial}{\partial\theta}f(x;\theta)}\dx=
			\int_a^\infty T(x)\mybra{-\frac{x}{\theta^2}+\frac{x^2}{\theta^2}}\exp\mycbra{-\frac{x^2}{\theta}}\dx=\]
			\[=\int_a^\infty T(x)\sqrt{\frac{x}{\theta}}\exp\mycbra{-\frac{x^2}{2\theta}}\frac{(x-1)\sqrt{x}}{{\theta}^{
			\mysfrac{3}{2}}}\exp\mycbra{-\frac{x^2}{2\theta}}\dx\leq\]
			\[\leq\sqrt{\int_a^\infty T^2(x)f(x;\theta)\dx}\sqrt{\int_a^\infty\frac{(x-1)^2x}{\theta^3}
			\exp\mycbra{-\frac{x^2}{\theta}}\dx}\]
			підкореневий вираз з першого множника не може перевищувати $\mathbb{E}_\theta T^2$ і отже логічно припустити,
			що як і досліджувана статистика $\hat{\theta}$ має обмежений другий момент на околі $\theta_0$ (в іншому
			разі, його варіація буде очевидно більшою за варіацію $\hat{\theta}$). Таким чином, залишається
			довести, що підкореневий вираз в другому множнику прямую до нуля рівномірно на околі $\theta_0$. Надалі
			припускатимемо $\theta$ належить $0<b<\theta<c$ малому околі $\theta_0$. Тоді
			\[\int_a^\infty\frac{(x-1)^2x}{\theta^3}\exp\mycbra{-\frac{x^2}{\theta}}\dx=
			\int_{a/\sqrt{\theta}}^\infty\frac{(y\sqrt{\theta}-1)^2y\sqrt{\theta}}{\theta^3}
			\exp\mycbra{-y^2}\sqrt{\theta}\;dy\leq\]
			\[\leq\int_{a/\sqrt{c}}\frac{(y\sqrt{c}-1)^2y}{b^2}\exp\mycbra{-y^2}\;dy\to0,\;\mbox{ при }a\to0\]
	\end{enumerate}
	Таким чином, маємо
	\[D\hat{\theta}\geq\frac{1}{I(\theta)}\]
	де
	\[I(\theta)=M\mybra{\frac{\partial\ln f(\xi;\theta)}{\partial\theta}}^2=
	\int_{\mathbb{R}^n_{>0}}\mycbra{\sum_i\frac{\partial}{\partial\theta}f(x_i;\theta)}
	\Pi_i f(x_i;\theta)\;dx_1\hdots dx_{n-1}dx_n=
	\]\[=
	\int_{\mathbb{R}^n_{>0}}\mycbra{-\frac{n}{\theta}+\frac{1}{2\theta^2}\sum_ix_i^2
	}^2
	\frac{\Pi_i x_i}{\theta^n}\exp\mycbra{-\frac{\sum_ix_i^2}{2\theta}}\;dx_1\hdots dx_{n-1}dx_n=\]
	\[=\int_{\mathbb{R}^n_{>0}}\mysbra{\frac{n^2}{\theta^2}-\frac{n}{\theta^3}\sum_ix_i^2+\frac{1}{4\theta^4}\mysbra{
	\sum_ix_i^2}^2}
	\frac{\Pi_i x_i}{\theta^n}\exp\mycbra{-\frac{\sum_ix_i^2}{2\theta}}\;dx_1\hdots dx_{n-1}dx_n=\]
	\[=\frac{n^2}{\theta^2}-\frac{n}{\theta^3}2n\theta+\frac{1}{4\theta^4}\mysbra{n(n-1)4\theta^2+n\underbrace{\mathbb{E}\xi^4}_
	{8\theta^2}}=n/\theta^2\]
	і $\hat{\theta}$ є ефективною оцінкою, адже
	\[Var(\hat{\theta})=\frac{1}{4n^2}nVar\xi^2=\frac{1}{4n}\mybra{\mathbb{E}\xi^4-\mathbb{E}^2\xi^2}=
	\frac{1}{4n}\mybra{8\theta^2-4\theta^2}=\frac{\theta^2}{n}=\frac{1}{I(\theta)}\]
\section{Глава 2}
\setcounter{prob}{40}
\begin{prob}\end{prob}
	Графіки емпіричної та рівномірної функції розподілу дані нижче
	\mypic{1.0}{../forplots/d1_plot.png}
	Щодо точної вищої границі, то користуючись методом описаним в підручнику, що дозоляє звести розрахунок до знаходження максимуму скінченного
	набору чисел, маємо
	\[\sup_x\myabs{F(x)-\hat{F}(x)}=0.3087\]
\begin{prob}\end{prob}
	Оскільки відрізок $[0,1]$ є компактною множиною, а $g(\xi)=\max^2\mycbra{\xi,1-\xi}$ є неперервною, як і густина ймовірності рівномірного
	на $[0,1]$ розподілу, незміщеною і конзистентною оцінкою шуканої величини буде
	\[\hat{G}_n=\frac{1}{n}\sum_{i=1}^ng(\xi_i)=0.09407\]
\begin{prob}\end{prob}
	Оскільки треба оцінити два параметри, нам знадобляться два перші моменти
	\[m_1(k;p,m)=\sum_{k=0}^mkC^k_mp^k(1-p)^{m-k}=\sum_{k=0}^m\cancel{k}\frac{m}{\cancel{k}}C^{k-1}_{m-1}p^k(1-p)^{m-k}=mp\]
	\[m_2(k;p,m)=\sum_{k=0}^mk^2C^k_mp^k(1-p)^{m-k}=mp(1-p+mp)\]
	і розв’язуючи цю систему рівнянь, маємо
	\[\hat{{m}}=\frac{\hat{m_1}^2}{\hat{m_1}+\hat{m_1}^2-\hat{m_2}}\]
	\[\hat{p}=1+\hat{m_1}-\frac{\hat{m_2}}{\hat{m_1}}\]
	де
	\[\hat{m_1}:=\frac{1}{n}\sum_{i=1}^n\xi_i\]
	\[\hat{m_2}:=\frac{1}{n}\sum_{i=1}^n\xi^2_i\]
\begin{prob}\end{prob}
	При фіксованих $\xi_i$ функція максимальної правдоподібності має вигляд
	\[L(\theta,h)=\frac{1}{(2h)^n}\Pi_{i=1}^nI_{[\theta-h,\theta+h]}(\xi_i)\]
	і таким чином, оскільки ми шукаємо найбільше значення, можна автоматично вважати, що $\forall i,\;\xi_i\in[\theta-h,\theta+h]$ і нам,
	таким чином, потрібно знайти найменше (адже маємо мінімізувати $(2h)^{-n}$) $h_0$, таке що для відповідних $(\theta_0,h_0)$ умова
	$\forall i,\;\xi_i\in[\theta-h,\theta+h]$ виконується. Це нескладно, адже у будь-якому разі $h\geq h_0:=\frac{1}{2}(\xi_{(n)}-\xi_{(1)})$, де
	$\xi_{(n)}:=\max_i\mycbra{\xi_i}$ і $\xi_{(1)}:=\min_i\mycbra{\xi_i}$. З іншого боку, при $\theta_0:=\frac{1}{2}(\xi_{(1)}+\xi_{(n)})$
	маємо якраз $\forall i,\;\xi_i\in[\theta_0-h_0,\theta_0+h_0]$, адже $\xi_{(1)}=\theta_0-h_0,\;\xi_{(n)}=\theta_0+h_0$. Щоб довести
	єдиність максимуму, помітимо, що для іншої пари $(\theta',h')$, такої, що $f(\xi_1,\xi_2,\hdots,\xi_n;\theta',h')=
	f(\xi_1,\xi_2,\hdots,\xi_n;\theta_0,h_0)$,
	має виконуватись $h'=h_0$, як показано вище і до того ж $\xi_{(1)}\in[\theta'-h',\theta'+h']\implies\xi_{(1)}\geq\theta'-h'\implies
	\theta'\leq \xi_{(1)}-h'=\theta_0$. Аналогічно, $\xi_{(n)}\in[\theta'-h',\theta'+h']\implies\theta'\geq\theta_0\implies\theta'=\theta_0$,
	що показує єдиність.

	Таким чином, \[\hat{h}=\frac{1}{2}(\xi_{(n)}-\xi_{(1)})\]
	\[\hat{\theta}=\frac{1}{2}(\xi_{(n)}+\xi_{(1)})\]
	Знаючи формули для математичного сподівання порядкових статистик рівномірного на $[\theta-h,\theta+h]$ розподілу, маємо
	\[\mathbb{E}\hat{h}=\frac{1}{2}\mybra{\mathbb{E}\xi_{(n)}-\mathbb{E}\xi_{(1)}}=h\frac{n-1}{n+1}\]
	\[\mathbb{E}\hat{\theta}=\frac{1}{2}\mybra{\mathbb{E}\xi_{(n)}+\mathbb{E}\xi_{(1)}}=\theta\]
	Таким чином, оцінка $\hat{h}$ є асимптотично незміщеною, але не є незміщеною, в той час як оцінка $\hat{\theta}$ є просто незміщеною. Далі
	ми покажемо конзистентність обох оцінок. Враховуючи, що обидві є асимптотично незміщеними (одна з них навіть просто незміщена) і враховуючи
	нерівність Чебишева, достатньо показити, що $\Var(\hat{h}),\;\Var({\hat{\theta}})\to0$ при $n\to\infty$.
	Враховуючи, що
	\[\Var(X\pm Y)\leq2\Var(X)+2\Var(Y)\]
	а також той факт, що $\xi_{(1)}$ та $\xi_{(n)}$ мають розподіл виду $\theta-h+2hB$, де $B$ -- бета-розподіл із параметрами $(1,n)$ та $(n,1)$
	відповідно. І таким чином, оскільки
	\[\Var(\xi_{(1)})=(2h)^2\frac{n}{(n+1)^2(n+2)}\to0\]
	\[\Var(\xi_{(n)})=(2h)^2\frac{n}{(n+1)^2(n+2)}\to0\]
	бажана конзистентність $\hat{h}$ та $\hat{\theta}$ доведена.
\section{Глава 3}
\setcounter{prob}{20}
\begin{prob}\end{prob}
	\begin{itemize}
		\item (\textit{визначити випадкову величину, яка спостерігається в експерименті (що означають висунуті гіпотези стосовно
			розподілів цієї випадкової величини?);})
			Природно вважати, що при дії препарата на $n$ комах для кожної з комах ураження трапляється з однаковою ймовірністю
			і незалежно від інших. Таким чином, розподіл кількості уражених комах $\xi$ природно вважати біноміальним з $n$ елементів
			із ймовірністю успіху (неураження) $p$. Загалом, гіпотези покупця і виробника стосуються як раз $p$. Виробник стверджує, що
			$p\leq.01$, а покупець хоче перевірити гіпотезу $p=0.1$.
		\item (\textit{сформулювати нульову (основну) й альтернативні гіпотези;}) У світлі сказаного в попередньому пункті, природньо
			сформулювати дві гіпотези: першу ($H_1$), що $p=0.01$ і другу ($H_2$), що $p=0.1$. Як і в прикладі 3.1.1 з \cite{turchin},
			не має значення, яку гіпотезу вибирати, тому ми виберемо з точки зору покупця, для якого відхилення гіпотези $H_2$ при
			її справдженні є більш болючим, тому за основну (нульову) гіпотезу виберемо $H_0:=H_2$. Альтернативною, таким чином, стає 
			$H_1$.
		\item (\textit{з’ясувати, в чому полягають помилки першого та другого родів;})
			Помилка першого роду полягатиме в тому, що гіпотеза $H_0=H_2$ відхилялася тоді, коли вона справджувалась, тобто
			коли препарат насправді має відсоток ураження $0.9$, він класифікується як такий, що має відсоток $0.99$. Ця помилка
			є небажаною для покупця. Помилка другого роду полягатиме в тому, що препарат, який насправді має відсоток ураження $0.9($,
			класифікується як такий, що має відсоток ураження $0.9$, і така помилка є небажаною для виробника.
		\item (\textit{запропонувати критерії для перевірки нульової гіпотези;})
			Як критичну множину природно розглядати множину $S=\mycbra{k:k\leq l}$, що цілком визначається числом $l$. Таким
			чином, критерієм для перевірки $H_0$ стає борелева функція $\phi(x):=I_S(x)$.
		\item (\textit{призначити рівень значущості критерію, обґрунтувати свій вибір;})
			Оскільки за умовою задачі помилка першого роду має бути обмежена зверху числом $1-0.95=0.05$, рівень значущості
			критерію має бути рівним $\alpha:=0.05$, тобто
			\[P(S\big| H_0)=P_0(\xi\leq l)\leq\alpha=0.05.\]
		\item (\textit{обчислити ймовірність помилки першого роду;})
			Настав час вибрати $n$ та $l$. З умови задачі, маємо два співвідношення
			\[\sum_{k=0}^lC_n^k(0.1)^k(0.9)^{n-k}\leq0.05\]
			\[\sum_{k=0}^lC_n^k(0.01)^k(0.99)^{n-k}\geq0.98\]
			Застосувавши теорему Пуассона, це можна переписати як
			\[\sum_{k=0}^l\frac{\lambda_0^k}{k!}e^{-\lambda_0}\leq0.05,\quad\lambda_0=n\cdot0.1\]
			\[\sum_{k=0}^l\frac{\lambda_1^k}{k!}e^{-\lambda_1}\geq0.98,\quad\lambda_1=n\cdot0.01\]
			Далі користуючись таблицею 8.6.1, маємо, що при 
			\begin{empheq}[box=\fbox]{align*}n&=79\\l&=3\end{empheq}
			мають місце обидві нерівності одночасно, і таке $n$ є найменшим. Відповідно, ймовірність помилки першого роду рівна
			\[\sum_{k=0}^l\frac{\lambda_0^k}{k!}e^{-\lambda_0}=0.04847\]
		\item (\textit{дослідити поведінку функції потужності критерію;})
			Потужність критерію
			\[P(S\big|H_1)=1-\sum_{k=0}^l\frac{\lambda_1^k}{k!}e^{-\lambda_1}=0.9916660\]
		\item (\textit{дати частотну інтерпретацію одержаних результатів;})
			Ймовірність помилки першого роду можна інтерпретувати так: при використанні одержаного критерію для препаратів, що
			має відсоток ураження $0.90$ в середньому 5 препаратів із 100 будуть класифікуватися як такі, що мають відсоток
			ураження $0.99$. Ймовірність помилки другого роду, рівної $0.008$ можна трактувати так: при тестуванні препаратів, що
			мають відсоток ураження $0.99$ в середньому 8 з 1000 будуть класифікуватися як такі, що мають ураження 0.90. Значення
			потужності критерію $0.991660$ можна трактувати так: при тестуванні препаратів, що мають відсоток ураження 99\%
			не менше 99 з них буде класифікуватися як такі, що мають відсоток ураження 99\%.
	\end{itemize}
\begin{prob}\end{prob}
	\begin{itemize}
		\item (\textit{визначити випадкову величину, яка спостерігається в експерименті (що означають висунуті гіпотези стосовно
			розподілів цієї випадкової величини?);})
			Як сказано в умовах задачі, трактування кожного знімка здійснюється незалежно і таким чином
			, розподіл кількості знімків, які виявляють хворобу $\xi$ природно вважати біноміальним з $n$ елементів
			із ймовірністю успіху (відсутності ознак хвороби на знімку
			) $p$. Таким чином, з умов задачі, той факт, що у людини є туберкульоз можна переписати як $p=1-p_0=0.1$, а той,
			що у людини немає туберкульозу як $p=p_1=0.01$. 
		\item (\textit{сформулювати нульову (основну) й альтернативні гіпотези;}) У світлі сказаного в попередньому пункті, природньо
			сформулювати дві гіпотези: першу ($H_1$), що $p=0.1$ і другу ($H_2$), що $p=0.01$.
			Як сказано в посібнику, нульову гіпотезу потрібно вибирати з міркувань,так щоб помилка першого роду була небезпечніша
			за помилку другого, і тому за нульову гіпотезу виберемо $H_0:=H_1$, адже тоді помилка першого роду (що полягає у відхиленні
			гіпотези коли вона справджується) буде полягати у класифікації хворої людини як здорової, що є найбільш небезпечним.
			Альтернативною, таким чином, стає $H_2$.
		\item (\textit{з’ясувати, в чому полягають помилки першого та другого родів;})
			Помилка першого роду полягатиме в тому, що гіпотеза $H_0=H_1$ відхилялася тоді, коли вона справджувалась, 
			буде полягати у класифікації хворої людини як здорової, що є найбільш небезпечним.
			Помилка другого роду полягатиме в тому, що здорова людина класифікуватиметься як хвора і може принести лише незручності,
			але не призведе до проблем із здоров’ям людини.
		\item (\textit{запропонувати критерії для перевірки нульової гіпотези;})
			Як критичну множину природно розглядати множину $S=\mycbra{k:k\leq l}$, що цілком визначається числом $l$. Таким
			чином, критерієм для перевірки $H_0$ стає борелева функція $\phi(x):=I_S(x)$. Тобто ми класифікуватимемо гіпотезу
			$H_0$ як вірну тоді і лише тоді коли $\xi(\omega)\in S\iff\xi(\omega)\leq l$.
		\item (\textit{призначити рівень значущості критерію, обґрунтувати свій вибір;})
			Оскільки за умовою задачі помилка першого роду має бути обмежена зверху числом $1-0.99=0.01$, рівень значущості
			критерію має бути рівним $\alpha:=0.01$, тобто
			\[P(S\big| H_0)=P_0(\xi\leq l)\leq\alpha=0.01.\]
		\item (\textit{обчислити ймовірність помилки першого роду;})
			Настав час вибрати $n$ та $l$. З умови задачі, маємо два співвідношення
			\[\sum_{k=0}^lC_n^k(0.1)^k(0.9)^{n-k}\leq0.01\]
			\[\sum_{k=0}^lC_n^k(0.01)^k(0.99)^{n-k}\geq0.95\]
			Застосувавши теорему Пуассона, це можна переписати як
			\[\sum_{k=0}^l\frac{\lambda_0^k}{k!}e^{-\lambda_0}\leq0.01,\quad\lambda_0=n\cdot0.1\]
			\[\sum_{k=0}^l\frac{\lambda_1^k}{k!}e^{-\lambda_1}\geq0.95,\quad\lambda_1=n\cdot0.01\]
			Далі користуючись таблицею 8.6.1, маємо, що при 
			\begin{empheq}[box=\fbox]{align*}n&=101\\l&=3\end{empheq}
			мають місце обидві нерівності одночасно, і таке $n$ є найменшим. Відповідно, ймовірність помилки першого роду рівна
			\[\sum_{k=0}^l\frac{\lambda_0^k}{k!}e^{-\lambda_0}=0.00960529691529771\]
		\item (\textit{дослідити поведінку функції потужності критерію;})
			Потужність критерію
			\[P(S\big|H_1)=1-\sum_{k=0}^l\frac{\lambda_1^k}{k!}e^{-\lambda_1}=0.980392569281137\]
		\item (\textit{дати частотну інтерпретацію одержаних результатів;})
			Ймовірність помилки першого роду можна інтерпретувати так: при використанні одержаного критерію для діагностики для людей,
			що мають туберкольоз в середньому 1 людина із 100 буде класифікуватися як здорова.
			Ймовірність помилки другого роду, в свою чергу можна трактувати так: при діагностиці здорових людей
			2 людини зі 100 будуть класифікуватися як хворі. Значення
			потужності критерію $0.9803925692$ можна трактувати так: при діагностиці здорових людей, в середньому 98 із 100 будуть
			класифіковані як такі, що не мають хвороби.
	\end{itemize}
\section{Глава 4}
\setcounter{prob}{30}
\begin{prob}\end{prob}
	У термінах перевірки статистичних гіпотез задачу можна сформулювати так. Маємо 12 незалежних спостережень для лівого колеса $\xi_i$
	та стільки ж для правого $\eta_j$ з розподілів $N_{a_\xi;\sigma^2}$ та $N_{a_\eta;\sigma^2}$ відповідно. Відносно параметрів
	$a_\xi$ та $a_\eta$ висувається гіпотеза $H_0\;:\;a_\xi=a_\eta$. Це гіпотеза про відсутність істотних відмінностей в температурах.
	Природно визнати двобічну альтернативу. Відхилення гіпотези на користь альтернативи будемо інтерпретувати як наявність систематичної 
	розбіжності.

	Згідно з критерієм Стьюдента для перевірки $H_0\;:\;a_\xi=a_\eta$ при двобічній альтернативі треба порівняти значення 
	\[\myabs{t}=\myabs{\bar{\xi}-\bar{\eta}}\left/s\sqrt{\frac{n+m}{nm}}\right.,\mbox{ де }s:=\frac{1}{n+m-2}\mybra{(n-1)s_\xi^2+(m-1)s_\eta^2}\]
	з $t_{\alpha;(n+m-2)}$ -- верхньою $\alpha$-границею $t_{n+m-2}$-розподілу і відхиляти $H_0$ тоді і лише тоді, коли $\myabs{t}\geq t_{
	\alpha;(n+m-2)}$ (рівень значущості критерію становить $2\alpha$). Розрахунок показує, що
	\[\myabs{t}=0.89852<2.074 =t_{0.025;22}\]
	і таким чином на 5\%-рівні гіпотеза $H_0$ не відхиляється.

	Цей результат можна трактувати так. Припущення (гіпотеза) про відсутність істотних відмінностей в даних показниках температур лівої
	і правої шин під час руху автобуса не суперечить експериментальним даним, тобто іншими словами, такі покази типові для випадку відсутності
	відмінностей. Отже можна вважати, що істотних відмінностей немає.
\begin{prob}\end{prob}
	Задача практично ідентична до попередньою, за виключенням того, що тепер замість $\xi_i$ та $\eta_j$ нам відразу дані значення $m=n=10$,
	$s_\xi^2$, $s_\eta^2$, $\bar{\xi}$ та $\bar{\eta}$. Формули залишаються тими ж самими. Оскільки за умовою задачі точність роботи верстата
	можна вважати незмінною, дисперсій $\xi$ та $\eta$ можна також вважати однаковими, що і дозволяє застосувати ту ж саму техніку, що і вище.

	Розрахунок показує, що
	\[\myabs{t}=0.00350823<2.101 =t_{0.025;18}\]
	і таким чином на 5\%-рівні гіпотеза $H_0$ не відхиляється.

	Цей результат можна трактувати так. Припущення (гіпотеза) про відсутність зміни рівня налагодження верстату
	не суперечить експериментальним даним, тобто іншими словами, такі покази типові для одного й того ж верстату.
	Отже можна вважати, що рівень налагодження не змінився.
\begin{prob}\end{prob}
	У термінах перевірки статистичних гіпотез цю задачу можна сформулювати так. Маємо $\xi_i$ -- реалізацію вибірки з $N_{a;\sigma^2}$.
	Стосовно значення параметра $\sigma$ висувається гіпотеза
	\[H_0:\sigma^2/\sigma_0^2=1.\]
	де $\sigma_0=35.63\mbox{ фунт}^2$. Це гіпотеза про те, що зміна технологічного процесу не призвела до зменшення дисперсії. Оскільки 
	очікується, зменшення, то природно вибрати однобічну альтернативу $\sigma^2/\sigma_0^2<1$,
	тобто відхилення $H_0$ ми будемо трактувати як наявність покращення.

	Згідно із критерієм наведеним в \cite[4.3]{turchin}, $H_0$ будемо відхиляти тоді і лише тоді, коли
	\[\frac{s^2}{\sigma_0^2}<\frac{1}{n-1}\chi^2_{(1-\alpha);(n-1)}\]
	і рівень значущості цього критерію буде рівним $\alpha$. Далі, розрахунок показує, що для $\alpha:=0.025$
	\[\frac{s^2}{\sigma_0^2}=0.02244608<0.40214=\frac{1}{n-1}\chi^2_{(1-\alpha);(n-1)}\]
	Таким чином, можна вважати, що гіпотеза $H_0$ відхиляється на 2.5\%-рівні.

	Цей результат можна трактувати так. Припущення, що зміна технологічного процесу не призвела до істотного зменшення дисперсії, суперечить
	наявним даним. При схожій дисперсії значення, дані в умові задачі, не є можливими (точніше, були б дуже нетиповими). Отже, можна зробити
	висновок, що зміна технологічного процесу призвела до зменшення дисперсії.
\section{Глава 5}
\setcounter{prob}{40}
\begin{prob}\end{prob}
	Покази 1000 годинників можна розглядати як реалізацію вибірки обсягом 1000 з деякого неперервного на відрізку $[0;12)$ розподілу $F$. Щодо
	розподілу $F$ випадкової величини $\xi$ (показів годинників) висувається гіпотеза $H_0:F$ є рівномірним розподілом на відрізку $[0;12)$,
	тобто щільність $f(x)$ розподілу $F$ має вигляд
	\[f(x)=\left\{\begin{array}{cr}1/12,&\mbox{якщо }x\in[0;12);\\0,&\mbox{якщо }x\notin[0;12);\end{array}\right.\]
	Для перевірки гіпотези скористаємось критерієм $\chi^2$. Поділимо множину можливих значень $X=[0;12)$ випадкової величини $\xi$ на
	неперетинні підмножини $X_i=[i,i+1),\;i=0,1,\hdots,11$. Імовірність потрапляння $\xi$ в кожну з цих множин
	\[p_i=P\mycbra{\xi\in[i,i+1)}=\int_i^{i+1}\frac{1}{12}\;dx=\frac{1}{12},\]
	причому
	\[np_i=1000\frac{1}{12}=83.3>10;\quad i=0,1,\hdots,11.\]
	Отже, користуватися критерієм $\chi^2$ можна.

	Обчислимо значення відхилення між емпіричним розподілом $\hat{F}_n$ та гіпотетичним $G$:
	\[D(\hat{F}_n,G)=\sum_{i=1}^r\frac{\mybra{\nu_i-np_i}^2}{np_i}=9.464;\]
	\[D(\hat{F}_n,G)=9.464<19.7=\chi_{0.05;11}^2=\chi^2_{\alpha;(r-1)}.\]
	Тому згідно з критерієм $\chi^2$, гіпотеза про рівномірний на відрізку $[0;12)$ розподіл випадкової величини $\xi$ -- показів механічних
	годинників у вітринах магазинів -- не відхиляється. Дані таблиці характерні для вибірки обсягом 1000 з рівномірного на відрізку $[0;12)$
	розподілу. Припущення про те, що покази годинників розподілені рівномірно на цьому відрізку, не суперечать спостереженням.
\begin{prob}\end{prob}
	Розглянемо кількість пошкоджених виробів в контейнері як випадкову величину $\xi$. Щодо її розподілу висувається гіпотеза
	\[H_0\;:\:P\mycbra{\xi=k}=\frac{\lambda^ke^{-\lambda}}{k!},\;k=0,1,2,\hdots\]
	Треба перевірити цю гіпотезу.

	Параметр $\lambda$ гіпотетичного розподілу невідомий. Його необхідно оцінити за вибіркою. Як відомо, оцінкою максимальної правдоподібності
	параметра $\lambda$ розподілу Пуассона за вибіркою $\xi_1,x_2,\hdots,\xi_n$ є
	\[\hat{\lambda}=\frac{1}{n}\sum_{k=1}^n\xi_k\]
	Зокрема, у розглядуваному прикладі значення оцінки рівне $\hat{\lambda}=1.0$.
\begin{thebibliography}{9}
\bibitem{turchin}
В. М. Турчин \emph{Математична статистика. Навч. посіб.} --
К.: Видавничий центр ``Академія'', 1999. -- 240 с.
\bibitem{wiki_regularity}
{\em Нерівність Рао-Крамера} -- стаття з англійської вікіпедії на
\url{http://en.wikipedia.org/wiki/Cram%C3%A9r%E2%80%93Rao_bound#Regularity_conditions}.
\bibitem{dorogovcev}Дороговцев А. Я. {\em Математический анализ. Краткий курс в современном изложении}. -- Издание второе. --
	К.: Факт, 2004. -- 560 с.
\end{thebibliography}
\end{document}
