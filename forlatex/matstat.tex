\documentclass[12pt]{article} % use larger type; default would be 10pt

\usepackage{mathtext}                 % підключення кирилиці у математичних формулах
                                          % (mathtext.sty входить в пакет t2).
\usepackage[T1,T2A]{fontenc}         % внутрішнє кодування шрифтів (може бути декілька);
                                          % вказане останнім діє по замовчуванню;
                                          % кириличне має співпадати з заданим в ukrhyph.tex.
\usepackage[utf8]{inputenc}       % кодування документа; замість cp866nav
                                          % може бути cp1251, koi8-u, macukr, iso88595, utf8.
\usepackage[english,russian,ukrainian]{babel} % національна локалізація; може бути декілька
                                          % мов; остання з переліку діє по замовчуванню. 
\usepackage{mystyle}

\newtheorem{prob}{Завдання}
\newcommand{\ds}{\;ds}
\newcommand{\dt}{\;dt}
\newcommand{\dx}{\;dx}
\newcommand{\dy}{\;dy}
\newcommand{\dta}{\;d\tau}
\newcommand{\extr}{\mbox{\normalfont extr}}

\newtheorem{myulem}[mythm]{Лема}

\renewenvironment{myproof}[1][Доведення]{\begin{trivlist}
\item[\hskip \labelsep {\bfseries #1}]}{\myqed\end{trivlist}}

\title{Математична статистика (10 семестр)}
\author{Олексій Леонтьєв}

\begin{document}
\maketitle
\section{Глава 1}
\setcounter{prob}{30}
\begin{prob}\end{prob}
	Вважатимемо $\theta_0$ відомим параметром. Тоді $L(h)>0$ тільки якщо $\forall i,\;h>\myabs{\xi_i-\theta_0}$ і якщо ця
	умова виконується, то $L(h)=1/(2h)^n$ і таким чином
	\[\hat{h}=\max_i\mycbra{\myabs{\xi_i-\theta_0}}\]
	Таким чином так, $\hat{h}$ є оцінкою. Розглядаючи її як випадкову величину, її розподіл визначається наступним чином.
	Кожна з величин $\myabs{\xi_i-\theta_0}$ має розподіл
	\[\bar{f}(x;h)=\begin{cases}
		1/h,&\mbox{якщо }0<x<h\\
		0,&\mbox{в інших випадках}
	\end{cases}\]
	Це рівномірний розподіл, і як розраховано в \cite{turchin}, очікування мінімуму вибірки рівне
	\[M\hat{h}=\frac{n}{n+1}h+\frac{0}{n+1}\to h\]
	Таким чином, оцінка не є незміщеною, але є асимптотично незміщеною.
	Відповідно,
	\[P(\myabs{h-\hat{h}}>\epsilon)=\mybra{1-\frac{\epsilon}{h-0}}^n\to0\]
	і таким чином, оцінка є конзистентною.
\begin{prob}\end{prob}
	Перш за все, 
	\[m_1(\alpha,\theta)=\int_{\mathbb{R}^1} x\frac{1}{\alpha}e^{-(x-\theta)/\alpha}\dx=\int_0^\infty(x+\theta)\frac{1}{\alpha}
	e^{-x/\alpha}\dx=\int_0^\infty(\alpha x+\theta)e^{-x}\dx=\alpha+\theta\]
	і відповідно
	\[m_2(\alpha,\theta)=\int_0^\infty (x+\theta)^2\frac{1}{\alpha}e^{-(x-\theta)/\alpha}\dx=
	\int_0^\infty(\alpha^2x^2+2\alpha x\theta+\theta^2)e^{-x}\dx=2\alpha^2+2\alpha\theta+\theta^2=\alpha^2+(\alpha+\theta)^2\]
	і таким чином
	\[\hat{\alpha}=\sqrt{\frac{1}{n}\sum\xi_i^2-\mybra{\frac{1}{n}\sum\xi_i}^2}\]
	\[\hat{\theta}=\frac{1}{n}\sum\xi_i-\sqrt{\frac{1}{n}\sum\xi_i^2-\mybra{\frac{1}{n}\sum\xi_i}^2}\]
	Це оцінки. Жодна з них не є незміщеною, адже для $n=1,\;M\hat{\alpha}=0,\;M\hat{\theta}=M\xi=\alpha\neq\theta$. Щодо
	конзистентності, перший момент $\frac{1}{n}\sum\xi_i$ та другий момент $\frac{1}{n}\sum\xi_i^2$ є конзистентними
	оцінками відповідно $\mathbb{E}\xi=\alpha+\theta$ та $\mathbb{E}\xi^2=\alpha^2+(\alpha+\theta)^2$. Далі, конзистентність
	$\hat{\alpha}$ та $\hat{\theta}$ випливає з теореми Слуцького.
\begin{prob}\end{prob}
	По-перше, перевіримо, чи є ця оцінка незміщеною.
	\[\mathbb{E}\hat{\theta}=\frac{1}{2}\mathbb{E}\xi^2=\frac{1}{2}\int_0^\infty\frac{x^3}{\theta}\exp\mycbra{-\frac{x^2}{2
	\theta}}\dx=\frac{1}{4}\int_0^\infty\frac{y}{\theta}\exp\mycbra{-\frac{y}{2\theta}}\dy=\frac{4\theta}{4}={\theta}.\]
	Оцінка незміщена. Далі, перевіримо умови нерівності Крамера-Рао, як сформульовано в \cite{turchin}. 
	\begin{enumerate}
		\item \[\frac{\partial}{\partial\theta}\mybra{\frac{x}{\theta}\exp\mycbra{-\frac{x^2}{\theta}}}=
			\mybra{-\frac{x}{\theta^2}+\frac{x^2}{\theta^2}}\exp\mycbra{-\frac{x^2}{\theta}}\] та 
			\[\frac{\partial^2}{\partial\theta^2}\mybra{\frac{x}{\theta}\exp\mycbra{-\frac{x^2}{\theta}}}=
			\mybra{\frac{2x}{\theta^3}-\frac{2x^2}{\theta^3}-\frac{x^3}{\theta^4}+\frac{x^4}{\theta^4}}
			\exp\mycbra{-\frac{x^2}{\theta}}\] існують для
			$\theta>0$.
		\item Для кожного фіксованого $x$ маємо 
			\[\frac{\partial}{\partial\phi}\mybra{
			\mybra{-\frac{x}{\theta^2}+\frac{x^2}{\theta^2}}\exp\mycbra{-\frac{x^2}{\theta}}}=0\iff\]
			\[\iff
			{\frac{2x}{\theta^3}-\frac{2x^2}{\theta^3}-\frac{x^3}{\theta^4}+\frac{x^4}{\theta^4}}=0\iff\theta=\frac{x^2}{2}
			\]
			таким чином, ми бачимо, що для кожного фіксованого $x$, якщо $0<x<1$,то
			${\partial f/\partial\theta}<0$ і тому мажорується нулем, а якщо $x>1$, то $\partial f/\partial\theta$
			має єдиний
			максимум на $\mathbb{R}^1$ при $\theta={x^2}/{2}$ і підставляючи маємо
			\[\myabs{\mybra{-\frac{x}{\theta^2}+\frac{x^2}{\theta^2}}\exp\mycbra{-\frac{x^2}{\theta}}}\leq
			\myabs{-\frac{1}{4x^3}+\frac{1}{4x^2}}\exp\mycbra{-2}\]
			таким чином, $\partial f/\partial\theta$ мажорується функцією, яка рівна правій частині попередньою
			нерівності при $x>1$ і нулю при $0<x\leq1$, а така функція інтегровна, відповідна $\partial f/\partial\theta$
			мажорується інтегровною функцією.

			Відповідно, щоб мажорувати другу похідну по $\theta$, ми знайдемо при фіксованому $x$ максимум
			\[f(\theta):=\frac{\partial^2}{\partial\theta^2}\mybra{\frac{x}{\theta}\exp\mycbra{-\frac{x^2}{\theta}}}=
			\mybra{\frac{2x}{\theta^3}-\frac{2x^2}{\theta^3}-\frac{x^3}{\theta^4}+\frac{x^4}{\theta^4}}
			\exp\mycbra{-\frac{x^2}{\theta}}\]
			\[f'(\theta)=\frac{1}{\theta^6}\exp\mycbra{-\frac{x^2}{\theta}}(x-x^2)\mybra{-6\theta^2+6x^2\theta
			-x^4}\]
	\end{enumerate}
\begin{thebibliography}{9}
\bibitem{turchin}
В. М. Турчин \emph{Математична статистика. Навч. посіб.} --
К.: Видавничий центр ``Академія'', 1999. -- 240 с.
\end{thebibliography}
\end{document}
