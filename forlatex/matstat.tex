\documentclass[12pt]{article} % use larger type; default would be 10pt

\usepackage{mathtext}                 % підключення кирилиці у математичних формулах
                                          % (mathtext.sty входить в пакет t2).
\usepackage[T1,T2A]{fontenc}         % внутрішнє кодування шрифтів (може бути декілька);
                                          % вказане останнім діє по замовчуванню;
                                          % кириличне має співпадати з заданим в ukrhyph.tex.
\usepackage[utf8]{inputenc}       % кодування документа; замість cp866nav
                                          % може бути cp1251, koi8-u, macukr, iso88595, utf8.
\usepackage[english,russian,ukrainian]{babel} % національна локалізація; може бути декілька
                                          % мов; остання з переліку діє по замовчуванню. 
\usepackage{mystyle}

\newtheorem{prob}{Завдання}
\newcommand{\ds}{\;ds}
\newcommand{\dt}{\;dt}
\newcommand{\dx}{\;dx}
\newcommand{\dy}{\;dy}
\newcommand{\dta}{\;d\tau}
\newcommand{\extr}{\mbox{\normalfont extr}}

\newtheorem{myulem}[mythm]{Лема}

\renewenvironment{myproof}[1][Доведення]{\begin{trivlist}
\item[\hskip \labelsep {\bfseries #1}]}{\myqed\end{trivlist}}

\title{Математична статистика (10 семестр)}
\author{Олексій Леонтьєв}

\begin{document}
\maketitle
\section{Глава 1}
\setcounter{prob}{30}
\begin{prob}\end{prob}
	Вважатимемо $\theta_0$ відомим параметром. Тоді $L(h)>0$ тільки якщо $\forall i,\;h>\myabs{\xi_i-\theta_0}$ і якщо ця
	умова виконується, то $L(h)=1/(2h)^n$ і таким чином
	\[\hat{h}=\max_i\mycbra{\myabs{\xi_i-\theta_0}}\]
	Таким чином так, $\hat{h}$ є оцінкою. Розглядаючи її як випадкову величину, її розподіл визначається наступним чином.
	Кожна з величин $\myabs{\xi_i-\theta_0}$ має розподіл
	\[\bar{f}(x;h)=\begin{cases}
		1/h,&\mbox{якщо }0<x<h\\
		0,&\mbox{в інших випадках}
	\end{cases}\]
	Це рівномірний розподіл, і як розраховано в \cite{turchin}, очікування мінімуму вибірки рівне
	\[M\hat{h}=\frac{n}{n+1}h+\frac{0}{n+1}\to h\]
	Таким чином, оцінка не є незміщеною, але є асимптотично незміщеною.
	Відповідно,
	\[P(\myabs{h-\hat{h}}>\epsilon)=\mybra{1-\frac{\epsilon}{h-0}}^n\to0\]
	і таким чином, оцінка є конзистентною.
\begin{prob}\end{prob}
	Перш за все, 
	\[m_1(\alpha,\theta)=\int_{\mathbb{R}^1} x\frac{1}{\alpha}e^{-(x-\theta)/\alpha}\dx=\int_0^\infty(x+\theta)\frac{1}{\alpha}
	e^{-x/\alpha}\dx=\int_0^\infty(\alpha x+\theta)e^{-x}\dx=\alpha+\theta\]
	і відповідно
	\[m_2(\alpha,\theta)=\int_0^\infty (x+\theta)^2\frac{1}{\alpha}e^{-(x-\theta)/\alpha}\dx=
	\int_0^\infty(\alpha^2x^2+2\alpha x\theta+\theta^2)e^{-x}\dx=2\alpha^2+2\alpha\theta+\theta^2=\alpha^2+(\alpha+\theta)^2\]
	і таким чином
	\[\hat{\alpha}=\sqrt{\frac{1}{n}\sum\xi_i^2-\mybra{\frac{1}{n}\sum\xi_i}^2}\]
	\[\hat{\theta}=\frac{1}{n}\sum\xi_i-\sqrt{\frac{1}{n}\sum\xi_i^2-\mybra{\frac{1}{n}\sum\xi_i}^2}\]
	Це оцінки. Жодна з них не є незміщеною, адже для $n=1,\;M\hat{\alpha}=0,\;M\hat{\theta}=M\xi=\alpha\neq\theta$. Щодо
	конзистентності, перший момент $\frac{1}{n}\sum\xi_i$ та другий момент $\frac{1}{n}\sum\xi_i^2$ є конзистентними
	оцінками відповідно $\mathbb{E}\xi=\alpha+\theta$ та $\mathbb{E}\xi^2=\alpha^2+(\alpha+\theta)^2$. Далі, конзистентність
	$\hat{\alpha}$ та $\hat{\theta}$ випливає з теореми Слуцького.
\begin{prob}\end{prob}
	По-перше, перевіримо, чи є ця оцінка незміщеною.
	\[\mathbb{E}\hat{\theta}=\frac{1}{2}\mathbb{E}\xi^2=\frac{1}{2}\int_0^\infty\frac{x^3}{\theta}\exp\mycbra{-\frac{x^2}{2
	\theta}}\dx=\frac{1}{4}\int_0^\infty\frac{y}{\theta}\exp\mycbra{-\frac{y}{2\theta}}\dy=\frac{4\theta}{4}={\theta}.\]
	Оцінка незміщена. Далі, перевіримо умови нерівності Крамера-Рао, як сформульовано в \cite{wiki_regularity}.
	\begin{enumerate}
		\item \[\frac{\partial}{\partial\theta}\mybra{\frac{x}{\theta}\exp\mycbra{-\frac{x^2}{\theta}}}=
			\mybra{-\frac{x}{\theta^2}+\frac{x^2}{\theta^2}}\exp\mycbra{-\frac{x^2}{\theta}}\] та 
			\[\frac{\partial^2}{\partial\theta^2}\mybra{\frac{x}{\theta}\exp\mycbra{-\frac{x^2}{\theta}}}=
			\mybra{\frac{2x}{\theta^3}-\frac{2x^2}{\theta^3}-\frac{x^3}{\theta^4}+\frac{x^4}{\theta^4}}
			\exp\mycbra{-\frac{x^2}{\theta}}\] існують для
			$\theta>0$.
		\item Покажемо, що для кожного $\theta_0>0$ має місце
			\[\frac{\partial}{\partial\theta}\mysbra{\int T(x)f(x;\theta)\dx}=\int T(x)\mysbra{\frac{\partial}{\partial
			\theta}f(x;\theta)}\dx\]
			де $T(x)=T(x_1,x_2,\hdots,x_n)$ статистика така, що $\forall\theta\;\mathbb{E}_\theta T(x),\;\mathbb{E}_\theta
			T^2(x)<+\infty$.
			Згідно з \cite[Гл. 13, Тм. 17]{dorogovcev} достатньо перевірити рівномірну на околі $\theta_0$
			збіжність інтегралу в правій частині. Маємо
			\[\int_a^\infty T(x)\mysbra{\frac{\partial}{\partial\theta}f(x;\theta)}\dx=
			\int_a^\infty T(x)\mybra{-\frac{x}{\theta^2}+\frac{x^2}{\theta^2}}\exp\mycbra{-\frac{x^2}{\theta}}\dx=\]
			\[=\int_a^\infty T(x)\sqrt{\frac{x}{\theta}}\exp\mycbra{-\frac{x^2}{2\theta}}\frac{(x-1)\sqrt{x}}{{\theta}^{
			\mysfrac{3}{2}}}\exp\mycbra{-\frac{x^2}{2\theta}}\dx\leq\]
			\[\leq\sqrt{\int_a^\infty T^2(x)f(x;\theta)\dx}\sqrt{\int_a^\infty\frac{(x-1)^2x}{\theta^3}
			\exp\mycbra{-\frac{x^2}{\theta}}\dx}\]
			підкореневий вираз з першого множника не може перевищувати $\mathbb{E}_\theta T^2$ і отже логічно припустити,
			що як і досліджувана статистика $\hat{\theta}$ має обмежений другий момент на околі $\theta_0$ (в іншому
			разі, його варіація буде очевидно більшою за варіацію $\hat{\theta}$). Таким чином, залишається
			довести, що підкореневий вираз в другому множнику прямую до нуля рівномірно на околі $\theta_0$. Надалі
			припускатимемо $\theta$ належить $0<b<\theta<c$ малому околі $\theta_0$. Тоді
			\[\int_a^\infty\frac{(x-1)^2x}{\theta^3}\exp\mycbra{-\frac{x^2}{\theta}}\dx=
			\int_{a/\sqrt{\theta}}^\infty\frac{(y\sqrt{\theta}-1)^2y\sqrt{\theta}}{\theta^3}
			\exp\mycbra{-y^2}\sqrt{\theta}\;dy\leq\]
			\[\leq\int_{a/\sqrt{c}}\frac{(y\sqrt{c}-1)^2y}{b^2}\exp\mycbra{-y^2}\;dy\to0,\;\mbox{ при }a\to0\]
	\end{enumerate}
	Таким чином, маємо
	\[D\hat{\theta}\geq\frac{1}{I(\theta)}\]
	де
	\[I(\theta)=M\mybra{\frac{\partial\ln f(\xi;\theta)}{\partial\theta}}^2=
	\int_{\mathbb{R}^n_{>0}}\mycbra{\sum_i\frac{\partial}{\partial\theta}f(x_i;\theta)}
	\Pi_i f(x_i;\theta)\;dx_1\hdots dx_{n-1}dx_n=
	\]\[=
	\int_{\mathbb{R}^n_{>0}}\mycbra{-\frac{n}{\theta}+\frac{1}{2\theta^2}\sum_ix_i^2
	}^2
	\frac{\Pi_i x_i}{\theta^n}\exp\mycbra{-\frac{\sum_ix_i^2}{2\theta}}\;dx_1\hdots dx_{n-1}dx_n=\]
	\[=\int_{\mathbb{R}^n_{>0}}\mysbra{\frac{n^2}{\theta^2}-\frac{n}{\theta^3}\sum_ix_i^2+\frac{1}{4\theta^4}\mysbra{
	\sum_ix_i^2}^2}
	\frac{\Pi_i x_i}{\theta^n}\exp\mycbra{-\frac{\sum_ix_i^2}{2\theta}}\;dx_1\hdots dx_{n-1}dx_n=\]
	\[=\frac{n^2}{\theta^2}-\frac{n}{\theta^3}2n\theta+\frac{1}{4\theta^4}\mysbra{n(n-1)4\theta^2+n\underbrace{\mathbb{E}\xi^4}_
	{8\theta^2}}=n/\theta^2\]
	і $\hat{\theta}$ є ефективною оцінкою, адже
	\[Var(\hat{\theta})=\frac{1}{4n^2}nVar\xi^2=\frac{1}{4n}\mybra{\mathbb{E}\xi^4-\mathbb{E}^2\xi^2}=
	\frac{1}{4n}\mybra{8\theta^2-4\theta^2}=\frac{\theta^2}{n}=\frac{1}{I(\theta)}\]
\begin{thebibliography}{9}
\bibitem{turchin}
В. М. Турчин \emph{Математична статистика. Навч. посіб.} --
К.: Видавничий центр ``Академія'', 1999. -- 240 с.
\bibitem{wiki_regularity}
{\em Нерівність Рао-Крамера} -- стаття з англійської вікіпедії на
\url{http://en.wikipedia.org/wiki/Cram%C3%A9r%E2%80%93Rao_bound#Regularity_conditions}.
\bibitem{dorogovcev}Дороговцев А. Я. {\em Математический анализ. Краткий курс в современном изложении}. -- Издание второе. --
	К.: Факт, 2004. -- 560 с.
\end{thebibliography}
\end{document}
