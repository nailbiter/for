\documentclass[12pt]{article} % use larger type; default would be 10pt

%\usepackage[utf8]{inputenc} % set input encoding (not needed with XeLaTeX)
\usepackage[T1,T2A]{fontenc}         % внутрішнє кодування шрифтів (може бути декілька);
                                          % вказане останнім діє по замовчуванню;
                                          % кириличне має співпадати з заданим в ukrhyph.tex.
\usepackage[utf8]{inputenc}       % кодування документа; замість cp866nav
                                          % може бути cp1251, koi8-u, macukr, iso88595, utf8.
\usepackage[english,russian,ukrainian]{babel} % національна локалізація; може бути декілька
                                          % мов; остання з переліку діє по замовчуванню. 
\usepackage[10pt]{type1ec}          % use only 10pt fonts
\usepackage[T1]{fontenc}
\usepackage{graphicx}
\usepackage{float}
\usepackage{subfig}
\usepackage{amsmath}
\usepackage{amsfonts}
\usepackage{hyperref}
\usepackage{enumerate}
\usepackage{enumitem}
\usepackage{mystyle}

\newtheorem*{fact}{Fact}

\begin{document}
Давай попробуем разобраться, чего от тебя хотят и как мы можем им это дать. Начнем с таблицы.
В первой строчке тебя просят посчитать $\alpha^t$ для заданного $t$. Во второй строчке тебя просят прибавить
еще и единицу. В последней, просят логарифм. Логарифм считается следующим образом: допустим, у тебя есть какое-то
$\alpha^t+1$. Поскольку $\alpha$ -- генератор, то в первой строчке обязательно встретится что-то, равное $\alpha^t+1$. Таким
образом, для какого-то $t'$ будет $\alpha^{t'}=\alpha^t+1$. Так вот тогда, $t'=\log_\alpha(\alpha^t+1)$.

Еще насчет первой строчки таблицы. Тебя просят доказать, что $\alpha$ -- генератор. По определению, это значит,
 что в первой строчке таблицы будут встречаться \textit{все} элементы поля $\F_3[x]/\myabra{x^2+1}$. Напомню, что эти элементы:
\[\F_3[x]/\myabra{x^2+1}=\mycbra{0,1,2,x,x+1,x+2,2x,2x+1,2x+2}\]

Теперь насчет $A$. Допущу, что ты \textit{умеешь} умножать и прибавлять в поле $\F_3[x]/\myabra{x^2+1}$ (если нет, смотри ниже). Тогда единственное, что
может тебя напугать -- большие числа в 11, 19 и 30. Они не страшные, так как возводить в степень можно следующим способом: к примеру
\[(x+2)^2=x^2+4x+4=4x+3=x\]
\[(x+2)^4=((x+2)^2)^2=(x)^2=-1=2\]
\[(x+2)^8=((x+2)^4)^2=(2)^2=4=1\]
\[(x+2)^{11}=(x+2)^8(x+2)^2(x+2)=1\cdot x(x+2)=2x-1=2x+2\]\newpage
Что касается прибавление и умножения $a$ к $b$ в $\F_3[x]/\myabra{x^2+1}$, то алгоритм такой 
\begin{enumerate}
\item Просто прибавляешь (или множишь) $a$ к (на) $b$ как два многочлена, к примеру если $a=x+2$, $b=x$ у тебя получится
\[ab=x^2+2x,\;a+b=2x+2\]
\item Делишь на $x^2+1$ и пишешь остаток, к примеру получается
\[ab=x^2+2x=2x-1,\;a+b=2x+2\]
\item Для каждого коэффициента пишешь остаток при делении на 3, к примеру получается
\[ab=2x-1=2x+2,\;a+b=2x+2\]
\end{enumerate}
\end{document}
