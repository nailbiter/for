\documentclass[11pt,a4paper,twoside]{jarticle}
%==== 科研費LaTeX =============================================
%	2017(H29)年度 DC
%============================================================
% 2008-03-08: Taku Yamanaka (JSPS Research Center for Science Systems / Osaka Univ.)
% 2009-03-03: Yoko Yanagida (Assistant)
% 2010-03-03: Taku: Imported new features introduced in 2009 fall.
% 2011-02-25: Taku: Revised for JFY2013.
% 2013-03-14: Taku: Revised for JFY2014.
% 2014-02-22: Taku: Revised for JFY2015.
% 2016-02-26: Taku: Revised for JFY2017.
%============================================================
\input{forms/form00_header}
% user01_header
%=== 様式のファイルの形式の指定 =================
%   PDFではなく、eps の様式を読み込む場合は、次の行の頭に「%」をつけてください。
\setboolean{usePDFform}{true}
%===================================
\input{forms/form01_header}

% user02_header
%=== 予算の表の印刷 =====================
% 予算の集計の表を出すためには、次の行の頭の%を消してください。
%\setboolean{BudgetSummary}{true}
%=================================

%=== For English, uncomment the next line to left-justify inside table columns.
%\renewcommand{\KLCLLang}{\KLCL}

% === 一部のページだけタイプセット ==============
% New in 2009 fall version!
% 選んだページだけタイプセットするには、次の例の頭の%を消し、並べてください。
% 複数のページを選ぶこともできます。
% 提出前には、必ず全てコメントアウト(頭に%をつける)してください。
%ーーーーーーーーーーーーーーーーーーーーーーーーーーーーーーーーー
%\KLTypesetPage{1}			% p.1 (or p.1を含む連続したページ),
%\KLTypesetPage{3}			% p.3 (or p.3を含む連続したページ),
%\KLTypesetPagesInRange{5}{6}	% p.5 ~ p.6,
%\KLTypesetPagesInRange{8}{10}	% and p.8 ~ p.10
%=================================

% ===== my favorite packages ====================================
% ここに、自分の使いたいパッケージを宣言して下さい。
\usepackage{wrapfig}
\bibliographystyle{alpha}
\usepackage{amssymb}
%\usepackage{mb}
% \usepackage{color} % でも科研費の書類はグレースケールで印刷されます
%\DeclareGraphicsRule{.tif}{png}{.png}{`convert #1 `dirname #1`/`basename #1 .tif`.png}
%==========================================================

\newcommand{\KLShouKeiLine}[1]{\cline{#1}}
%もし、小計の上の線を取れと事務に言われたら、
%「そのようなことは、記入要項に書かれていないし、学振はそのようなことは気にしていない。」と
% 突っぱねる。
% それでもなお消せと理不尽なことを言われたら、次の行の 最初の「%」を消す。	
%\renewcommand{\KLShouKeiLine}[1]{}

\newcommand{\KLBudgetTableFontSize}{small}	% 予算の表のフォントの大きさ: small, footnotesize
\newcommand{\KLFundsTableFontSize}{normalsize}	%応募中、受入れ予定の研究費のフォントの大きさ:normalsize, small, footnotesize

% ===== my personal definitions ==================================
% ここに、自分のよく使う記号などを定義して下さい。
\newcommand{\klpionn}{K_L \to \pi^0 \nu \overline{\nu}}
\newcommand{\kppipnn}{K^+ \to \pi^+ \nu \overline{\nu}}


\input{forms/hook3} % for future maintenance
% ===== Global definitions for the PD form ======================
% 基本情報
%
%------ 研究課題名  -------------------------------------------
\newcommand{\研究課題名}{Symmetry breaking operators}

%----- 研究機関名と研究代表者の氏名-----------------------
\newcommand{\研究機関名}{東京大学}
\newcommand{\申請者氏名}{レオンチエフ\,オレクシィ}
\newcommand{\研究代表者氏名}{\申請者氏名}

%---- 研究期間の最終年度 ----------------
\newcommand{\研究期間の最終元号年度}{31}	%平成で、半角数字のみ
%=========================================================
\input{forms/form02_2017_header}	%<<<
\input{forms/form03_header}
\input{forms/form05_dc_header}
\input{forms/form07_header}
%============================================================
%endPrelude

\begin{document}
\input{forms/hook5} % for future maintenance
%============================================================
%     User Inputs
%============================================================

%form: dc_form_03-04.tex ; user: dc_03-04_preparation_etc.tex
%========== DC =========
%===== p. 03-04 現在までの研究状況 =============
\section{現在までの研究状況}
%watermark: w03_past_dc
\newcommand{\myquotes}[1]{``#1''}
\newcommand{\研究の背景}{%
%begin  研究の背景===================

	\begin{thebibliography}{99}
	\bibitem{HT93} R. E. Howe and E.-C. Tan, \myquotes{Homogeneous functions on light cones: the infinitesimal structure of some degenerate principal series representations},
		Bulletin of the A. M. S., A. M. S., 28(1), pp. 1--74, 1993.
	\bibitem{Juh09}  A. Juhl, \myquotes{Families of conformally covariant differential operators, Q-curvature and holography}, 275, Birkh\"auser Basel, 2009.
	\bibitem{CKOP11}J. L. Clerc, T. Kobayashi, B. {{\O}}rsted, M. Pevzner, \myquotes{Generalized Bernstein-Reznikov integrals}, Mathematische Annalen, Springer, 349(2), pp. 395--431, 2011.
	\bibitem{GGP12}  W. T. Gan, B. H. Gross, and D. Prasad, \myquotes{Sur les conjectures de Gross et Prasad}, Ast\'erisque, {Soci{\'e}t{\'e} Math{\'e}matique de France}, 346, 2012.
	\bibitem{KS15}  T. Kobayashi and B. Speh. \myquotes{Symmetry breaking for representations of rank one orthogonal groups},
		Memoirs of the A. M. S., A. M. S., 238(1126), 2015. 
	\bibitem{KOSS15}T. Kobayashi, B. {{\O}}rsted, P. Somberg, V. Sou{{\v c}}ek, \myquotes{Branching laws for Verma modules and applications in parabolic geometry. I}, Advances in Mathematics, Elsevier
		, 285, pp. 1796--1852, 2015.
%%	\bibitem{K15}T. Kobayashi,
%%		\myquotes{A program for branching problems in the representation theory of real reductive groups}, Representations of Lie Groups: In Honor of David A. Vogan, Jr. on his 60th Birthday,
%%		 Birkh\"auser, 312, pp. 277--322., 2015.
	\bibitem{KP16b}T. Kobayashi and M. Pevzner, \myquotes{Differential symmetry breaking operators. II. Rankin-Cohen operators for symmetric pairs}, Selecta Mathematica (N.S.), Springer, 22(2), 
		pp. 847--911, 2015.
	\end{thebibliography}
%end  研究の背景 ====================
}

\newcommand{\現在までの研究状況}{%
%begin  現在までの研究状況===================
	The main subject of my studies are the symmetry breaking operators (SBO), which in the most generality can be defined as 
	$G'$-intertwining operators between the irreducible representation of Lie group $G$ and the irreducible representation of its closed subgroup $G'$.

	Up to this point my work concerned the classification of $G'$-intertwining operators between induced representations $I(\lambda)$ and $J(\nu)$ of $G=O(p+1,q)$ and $G'=O(p,q)$ respectively
	corresponding to induction from maximal parabolic subgroup,
	parametrized each by one complex parameter. Already the special case $q=1$, which was extensively treated in \cite{KS15} by Toshiyuki Kobayashi and Birgit Speh,
	was shown to be interacting with numerous mathematical objects and theories (e.g. conformal geometry, 
	Knapp-Stein operators, de-Sitter spaces, harmonic analysis, branching laws of complementary series, etc.).
	For higher $q$ one can immediately point out the special case $(p,q)=(1,2)$ as being equivalent to finding invariant trilinear forms for representations of $SL_2$, since one might see the 
	latter as SBOs of $O(2,2)\supset O(2,1)$.

	Of particular importance are so-called differential SBOs. In the special case $q=1$
	these were shown to coincide with Juhl's conformal equivariant operators (see \cite{Juh09}).
	Moreover, based on the viewpoint introduced in \cite{KP16b}, one may see the Rankin-Cohen brackets 
	(which in turn were originally introduced for the study of modular forms in number theory, but since then were seen to have connections with covariant quantization and ring structures
	on representations spaces) as differential symmetry breaking operators of $O(2,2)\supset O(2,1)$ branching.
	In the course of my work, I have been trying to study both differential SBO and the integral ones. 
	The work can be divided into the following:\\
	\textbf{Project A.} Geometry of flag manifold

	I gave complete classification of $P'\backslash G/P$ orbit decomposition of flag variety $G/P$ of $G$ under the action of $P'$ (maximal parabolic subgroup of $G'$).
	That is, I was able to describe all orbits for every p,q and derive the closure relations.
	This information is extensively used in subsequent computations, as closed subsets of $G/P$ invariant under $P'$ define a coarse invariant
	of a symmetry breaking operator (since support of SBO is closed $P'$-invariant subset of $G/P$).
	One should also mention that these orbit decompositions can be seen as generalizations of Iwasawa decomposition
	(if one takes $G'=K$, being maximal compact) and Bruhat decomposition (if one takes $G'=G$).\\
	\textbf{Project B.} Construction of symmetry breaking operators

	As shown in \cite{KS15},
	Schwartz kernel theorem tells us that SBO are characterized by their distribution kernels,
	which can then be associated to $P'$-covariant generalized functions on $G/P$ flag manifold.
	Understanding the geometry of flag variety and using relative invariace (this being possible by results of Project A above),
	I was able to construct explicitly these kernels with support in every possible set predicted by orbit decomposition for values of parameters regular enough.\\
	\textbf{Project C.} Spectrum of symmetry breaking operators

	As spherical vectors are mapped to spherical vectors under SBOs and spaces of spherical vectors are one-dimensional in degenerated principal series,
	one naturally becomes interested in multiplicity factors. Using integral formulae,
	I was able to find closed expressions for these multiplicity factors. The particular case of $(p,q)=(1,2)$ was earlier obtained by Bernstein-Reznikov integrals \cite{CKOP11}.\\
	\textbf{Project D.} Meromorphic continuation of symmetry breaking operators and normalization

	As representations we are dealing with depend on a pair of complex parameters, so do the SBOs. Having constructed them for regular values of parameters,
	one then has to determine whether the extension to the whole parameter space is possible. Moreover, as for parameters regular enough the dependence turns out to be holomorphic,
	one finds himself using theory of generalized functions depending on holomorphic (meromorphic) parameter in order to carry out the extension. In doing so, poles usually occur and one is 
	then interested in determining their precise location and multiplicity, so to carry out the normalization, after which dependence on parameters becomes holomorphic. 
	I was able to normalize all of the SBOs. Moreover, I investigated residues at poles and determined their support.\\
	\textbf{Project E.} Classification of symmetry breaking operators for every value of parameters

	Having constructed enough holomorphic families of SBOs as a result of Project D, I was able to show that they span space of SBOs for every value of parameters. Hence, I was able to
	give the explicit basis for $\mbox{Hom}_{G'}(I\left( \lambda \right),J\left( \nu \right))$ for every $\left( \lambda,\nu \right)\in\mathbb{C}^2$.\\
	\textbf{Project F.} Functional relations and residue formulae

	Similar to the \cite{KS15}, it turns out that generically all SBOs are (proportional to the) residues of a regular family $\tilde{K}_{\lambda,\nu}^{\mathbb{R}^n}$
	and I was able to precisely prove this fact and compute the proportionality coefficients.
	Similarly, for Knapp-Stein intertwining operators $\tilde{\mathbb{T}}_\nu:J(\nu)\to J(n-1-\nu)$ and $\tilde{\mathbb{T}}_\lambda:I(\lambda)\to I(n-\lambda)$
	we have that operators $\tilde{\mathbb{T}}_{n-1-\nu}\circ \tilde{K}_{\lambda,n-1-\nu}^{\mathbb{R}^n}$ and $\tilde{K}_{\lambda,\nu}^{\mathbb{R}^n}$ are proportional
	and $\tilde{K}_{n-\lambda,\nu}^{\mathbb{R}^n}\circ \tilde{\mathbb{T}}_\lambda$ and $\tilde{K}_{\lambda,\nu}^{\mathbb{R}^n}$ are proportional. I was able to determine precisely
	proportionality coefficients and derive similar relations for other families of SBO. The exact form of these relations contains vast amount of representation-theoretic information.
	Interestingly enough, while in \cite{KS15} the functional relations were used to carry out the normalization (that is, Project D), in our case a different approach was taken which allowed
	to do the normalization without this additional information, by purely analytical means.
%end  現在までの研究状況 ====================
}

%form: dc_form_05.tex ; user: dc_05_purpose.tex
%========== DC =========
%===== p. 05 研究の目的・内容 =============
\subsection{研究の目的・内容}
\newcommand{\研究目的}{%
%begin  研究目的と内容 (figureやtable使用可)===================
In the field of representation theory one can roughly be said to studying objects (representations) and morphisms between them (symmetry breaking operators).
Now, while the former has been well studied for more than 70 years with some fairly good classification results available at this point,
the latter has attracted attention only relatively recently with general theory introduced no earlier than around 1990.
The most fundamental case is that of symmetry breaking between the pair of reductive groups; however, the symmetry breaking in this case is known to be difficult.
The first example of complete classification was accomplished only recently in \cite{KS15} by T. Kobayashi and B. Speh.
Their work was immediately followed by Clerk (France), {\O}rsted (Denmark), M\"ullers (Germany) and Gomez (USA) among others.

The general problem of symmetry breaking between reductive groups, as led by professor Kobayashi, besides
being fundamental in representation theory, is also known to interact with geometry, PDE, real analysis and number theory.

My primary research goal is to apply techniques of \cite{KS15} to other settings beyond $O(p+1,q)\supset O(p,q)$ (e.g. $U(p+1,q)\supset U(p,q)$ and $Sp(p+1,q)\supset Sp(p,q)$)
so to obtain a complete classification of symmetry breaking operators for these settings. As mentioned before, techniques employed are mostly analytic in nature: essentially, the problem
reduces to classifying all distributions over Euclidean space that satisfy certain system of PDEs together with some group-invariance conditions. Basically, this problem is solved in the 
following way:\begin{enumerate}
	\item One first classifies $P'\backslash G/P$ cosets so to understand what supports can ($P'$-invariant distributions on $G/P$ corresponding to) the SBOs have;
	\item For values of parameters regular enough one constructs several families of SBOs holomorphically dependent on parameters having support in each set as predicted in the previous step;
	\item One then determines location of poles for every family and normalizes them (in general, this is the most technical part: there are several ways to do this and in my work
		for $O(p+1,q)\supset O(p,q)$ in fact I made a point of normalizing every familiy with a different technique; nevertheless the most general technique involves $K$-finite vectors,
		but one is then faced with the nontrivial problem of finding a closed form expression for the integral); after normalization is done one has to determine the support of residues;
	\item One proves that constructed (normalized) families exhaust the space of SBOs (this part is also highly nontrivial, but heuristically, if normalization was done in the ``optimal'' way, it
		turns out to work out), thus obtaining basis for SBO space for every value of parameters;
	\item One can then study the obtained families: derive the functional relations and residue formulae between them, study their kernel and images and how the latter match up with the composition
		series of the representations.
\end{enumerate}
As mentioned above, it should be the case that this scheme will apply to the symmetry breaking between degenerate principal series of $U(p+1,q)\supset U(p,q)$ and $Sp(p+1,q)\supset Sp(p,q)$
module some minor technical issues which I (hopefully) will be able to resolve,
and (with less degree of confidence) to the non-degenerate principal series of $O(p+1,q)\supset O(p,q)$ (that is, the induction done with respect to \textit{minimal}, not \textit{maximal} parabolic).
%end  研究目的と内容 (figureやtable使用可) ====================
}

%form: dc_form_06.tex ; user: dc_06_plan.tex
%========== DC =========
%===== p. 06 研究の特色・独創的な点、年次計画 =============
\newcommand{\研究の特色と独創的な点}{%
%begin  研究の特色と独創的な点===================FIXME
	To the best of my knowledge, \cite{KS15} is the first paper to tackle the question of symmetry breaking
	between the \textit{infinite-dimensional} representations of the \textit{non-compact} groups. Moreover, it is the first
	paper to give a complete classification result. My work (which is a direct generalization of \cite{KS15}) can be seen as a demonstration of the 
	robustness of the techniques introduced there, which allow one to tackle the (highly nontrivial) problem of symmetry breaking in a uniform and direct way.
	Moreover, in the course of my work I met some difficulties which were non-existent in
	$O(n+1,1)\supset O(n,1)$ case (most notably, difficult to compute integrals) and consquently was able to resolve them. 

	In my opinion, these techniques provide new, completely analytic approach to the problem of symmetry breaking and are thus worth studying. 
	Obtained formulae generalize and relate to several well-known results (e.g. \cite{CKOP11}). Moreover, they allow to see these results from the
	different unified viewpoint. 
%end  研究の特色と独創的な点 ====================
}

\newcommand{\年次計画1年目}{%
%begin  年次計画1年目===================
	First, there's still something to be done in $O(p+1,q)\supset O(p,q)$ case. After
	one completely understands the structure of SBO space and has functional identities,
	the latter can be used in order to get further information (e.g. composition series) about representations corresponding to induction from maximal parabolic.
	One may then investigate how images/kernels of various SBOs match up with the composition series structure.

 	Next, I'm planning to extend the results obtained for $O(p+1,q)\supset O(p,q)$ above
	to $U(p+1,q)\supset U(p,q)$ and $Sp(p+1,q)\supset Sp(p,q)$ settings. It appears to be possible to achieve results in the scope of
	projects A, B and D (geometry of flag manifold; construction of SBO; meromorphic continuation and normalization) at least. The reason to this
	being that we have a very concrete picture for degenerate principal series representation in this case: the space of homogeneous functions on the light cone
	(cf. \cite{HT93}). 
%end  年次計画1年目 ====================
}

\newcommand{\年次計画2年目}{%
%begin  年次計画2年目===================
	For the second year I'm planning to investigate the symmetry breaking between principal series representations of $O(p+1,q)$ and $O(p,q)$ corresponding to induction
	from the \textit{minimal} parabolic, as opposed to the \textit{maximal} parabolic. This setting is expected to be much more complicated, as the number of $P'\backslash G/P$ cosets
	grows with $(p,q)$ and the closure relations consequently also become depending on a $(p,q)$ in a very nontrivial way. Similarly, principal series are given no more by two complex
	parameters, but the number of complex parameters grows linearly with $\left( p,q \right)$.
%end  年次計画2年目 ====================
}

\newcommand{\年次計画3年目}{%
%begin  年次計画3年目===================
%end  年次計画3年目 ====================
}

%form: dc_form_07.tex ; user: dc_07_rights.tex
%========== DC =========
%===== p. 07 人権の保護及び法令等の遵守への対応 =============
\subsection{人権の保護及び法令等の遵守への対応}
\newcommand{\人権の保護及び法令等の遵守への対応}{%
%begin  人権の保護及び法令等の遵守への対応 ===================
	該当なし。
%end  人権の保護及び法令等の遵守への対応 ====================
}

%form: dc_form_08.tex ; user: dc_08_publications.tex
%========== DC =========
%===== p. 08 研究業績 =============
\section{研究業績}
\subsection{学術雑誌(紀要・論文集等も含む)に発表した論文及び著書}
\newcommand{\学術雑誌等に発表した論文または著書}{%
%begin  学術雑誌等に発表した論文または著書===================
	
	\begin{enumerate}
		\item[](査読有り)%===========================
		\item \underline{O. Leontiev}, P. Feketa,
			``A new criterion for the roughness of exponential dichotomy on $\mathbb{R}$'',
				Miskolc Mathematical Notes, 16(2), pp. 987--994, 2015.

		\item[](査読なし)%=============================
%%		\item Kobo Abe$^3$, \underline{H. Yukawa}$^1$, 
%%				``仔象は死んだ'', 
%%				安部公房全集, {\bf 26}, 100-200, (2004).
			
			なし
	\end{enumerate}
%end  学術雑誌等に発表した論文または著書 ====================
}

\subsection{学術雑誌等又は商業誌における解説・総説}
\newcommand{\学術雑誌等または商業誌における解説や総説}{%
%begin  学術雑誌等または商業誌における解説や総説===================

	なし
%%	\begin{enumerate}
%%		\item R.~Kipling, \underline{H. Yukawa},
%%				``The Elephant's Child (象の鼻はなぜ長い)'', 
%%				Nature, {\bf 999}, 777-779, (2003).
%%	\end{enumerate}
%end  学術雑誌等または商業誌における解説や総説 ====================
}

\subsection{国際会議における発表}
\newcommand{\国際会議における発表}{%
%begin  国際会議における発表===================

	なし
%%	\begin{enumerate}
%%		\item 
%%			$\circ$ 湯川秀樹、
%%			``Theory of Elephant Eggs'', 
%%			原始殻物理国際会議、
%%			カラチ、2006年2月
%%
%%		\item $\circ$ 湯川秀樹、Jacques-Yves Cousteau,
%%			``How to search for whale eggs'',
%%			国際海洋探索会議、ハワイ、2003年4月
%%	\end{enumerate}
%end  国際会議における発表 ====================
}

\subsection{国内学会・シンポジウムにおける発表}
\newcommand{\国内学会やシンポジウムにおける発表}{%
%begin  国内学会やシンポジウムにおける発表===================

	なし
%%	\begin{enumerate}
%%		\item $\circ$ 湯川秀樹、朝永振一郎、
%%			「ほ乳類の真の意味」、
%%			ほ乳類学会、
%%			東京、2003年6月
%%	\end{enumerate}
%%	他3件
%end  国内学会やシンポジウムにおける発表 ====================
}

\subsection{特許等}
\newcommand{\特許等}{%
%begin  特許等===================

	なし
%%	\begin{enumerate}
%%		\item[](公開中)
%%		\item 800800号、「クジラの卵を用いた深海潜水艇」\underline{湯川秀樹}、2003年4月
%%%		\item[] (申請中)
%%%		\item 8000000号、「象の卵を用いた(ひ・み・つ)」、\underline{湯川秀樹}、2007年4月
%%	\end{enumerate}		
%end  特許等 ====================
}

\subsection{その他の業績}
\newcommand{\その他の業績}{%
%begin  その他の業績===================
		\begin{enumerate}
			\item Monbukagakusho Scholarship recipient April 2014--March 2016.
			\item The University of Tokyo Graduate School of Mathematical Sciences Director's Prize of the academic year 2015.
		\end{enumerate}
%end  その他の業績 ====================
}

\subsection{発表前}
\newcommand{\発表前の業績}{%
%begin  発表前の業績===================
%%	【発表(印刷)前】\\
%%	{\bf(1)学術雑誌等(紀要・論文集等も含む)に採録決定されたもの}\\
%%	 (査読有り)\\
%%	 1) 島崎藤村$^2$ 「夜明け前」『中央公論』中央公論社、800号、pp1-800 (1935). 
%%	  (証明書\textcircled{1}添付)\\
%%	  \noindent
%%	  {\bf(4)国際学会・シンポジウム等における発表の申し込みが受理されたもの}\\
%%	 2) ◯\underline{湯川秀樹}「象の卵、ついに発見!」『オリンピック』ロンドン, 2012年
%%	  (証明書\textcircled{2}添付)
%end  発表前の業績 ====================
}

%form: dc_form_09.tex ; user: dc_09_myself.tex
%========== DC =========
%===== p. 09 自己評価 =============
\section{自己評価}
\newcommand{\自己評価}{%
%begin  自己評価===================
{\bf 1. 研究職を志望する動機、目指す研究者像、自己の長所等}

My primary motivation for research work in representation theory is the beauty of the subject. Representation theory enjoys being at the junction of analysis, algebra and geometry,
hence the techniques from all three areas (and their mixes) can and are extensively employed. Feeling myself more confident with analysis and geometry, I am very enthusiastic about
applying analytic techniques to questions of representation theory. Besides, the method works by reducing the problem to solving system of PDEs and this is additional inspiration for me,
as I am interested in problem of finding the exact symbolic solutions to PDEs -- something that one does not see happening very often. Similarly, I find it being additional motivation that
this work required symbolic computation of integrals and particular values of special functions.

Though I'm working on a very technical and narrow problem now, my future vision of myself as a researcher is a bit different. Similarly to my academic supervisor, I would like to become
well-versed in many areas of representation theory, so to be able to see my work and work of others in much more broad context.

My strong points can be summarized as follows:\begin{enumerate}
	\item I've completely solved the case of $O(p+1,q)\supset O(p,q)$ branching, so I guess I have reasons to expect success in other settings as well;
	\item I'm deeply interested in the subject
\end{enumerate}

\vspace{5mm}
{\bf 2. 自己評価をする上で、特に重要と思われる事項}

The points I would like to emphasize in the evaluation above are as follows:\begin{enumerate}
	\item I was the recipient of the prestigious Monbukagakusho Scholarship during
		my Master studies;
	\item My Master thesis was awarded with graduate School of Mathematical Sciences Director's Prize of the academic year 2015.
\end{enumerate}
%end  自己評価 ====================
}

% hook9 : right before \end{document} ============

%endUserFiles
\input{forms/hook7} % for future maintenance

\input{forms/dc_forms}
%endFormatFile

\input{forms/hook9} % for future maintenance
\end{document}
%TODO:
%write english version:
	%		current situation 
	%		research target
%		research specialty
%		year plan <--
	%		research publish <--
%		self evaluation
%show to K
%(translate)
%show to other ppl:
%	Orita
%	
