
\documentclass[10pt]{article} % use larger type; default would be 10pt

%%\usepackage[T1,T2A]{fontenc}
%%\usepackage[utf8]{inputenc}
%%\usepackage[english,ukrainian]{babel} % може бути декілька мов; остання з переліку діє по замовчуванню. 
\usepackage{enumerate}
\usepackage{CJKutf8}
\usepackage{mystyle}

%%\usepackage{fancyhdr}
%%\pagestyle{fancy}
%%\fancyfoot[C]{text me at \href{mailto:leontiev@ms.u-tokyo.ac.jp}{leontiev@ms.u-tokyo.ac.jp} if there are mistakes/obscurities}
%%\fancyhead{}

\begin{document}
\begin{CJK}{UTF8}{min}
質問というのはこれです。そういう超関数は:
\[F_{ \lambda,\nu }(x,y):=\myabs{ x_p }^{ \lambda+\nu-n }\mybra{ \myabs{ x }^2_p-\myabs{ y }^2_q }^{ -\nu }_+,\quad(x,y)\in\R^{ p,q }\]
symmetry breaking operatorの kernelの方程式に満たされるのようですが、symmetry breaking operatorのkernelではないと思います。

私は正しく分かると、前回先生はおっしゃってたのは、$m_-=\mysbra{\begin{smallmatrix}-1&0&0\\
	0&I_{ p+q-2 }&0\\0&0&-1\end{smallmatrix}}$のinvarianceの方程式に満たされていません。しかし,$m_-$のinvarianc
eの方程式はそんな感じになります:
\[F(-x,-y)=F(x,y)\]
上の$F_{ \lambda,\nu }$はもちろんこういう方程式に満たされると思います。
%%\begin{thebibliography}{9}
%%\bibitem{gelbaum}Gelbaum, B.R. and Olmsted, J.M.H.. Counterexamples in Analysis. Dover Publications. 2003
%%\end{thebibliography}
\end{CJK}
\end{document}
