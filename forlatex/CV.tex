\documentclass[10pt]{article}
\usepackage{fontspec}
\usepackage{array, xcolor, lipsum, bibentry}
\usepackage[margin=3cm]{geometry}
\usepackage{sectsty} % Allows changing the font options for sections in a document
 
\title{\bfseries\Huge Oleksii Leontiev}
\author{inp9822058@cs.nctu.edu.tw}
\date{}
 
\definecolor{lightgray}{gray}{0.8}
\newcolumntype{L}{>{\raggedleft}p{0.2\textwidth}}
\newcolumntype{R}{p{0.8\textwidth}}
\newcommand\VRule{\color{lightgray}\vrule width 0.5pt}
 
%font configuration
\defaultfontfeatures{Mapping=tex-text}
\setromanfont[Ligatures={Common}, Numbers={OldStyle}, Variant=01]{Linux Libertine O} % Main text font
\sectionfont{\mdseries\upshape\Large} % Set font options for sections
\subsectionfont{\mdseries\scshape\normalsize} % Set font options for subsections
\subsubsectionfont{\mdseries\upshape\large} % Set font options for subsubsections
\chardef\&="E050 % Custom ampersand character
 
\begin{document}
\maketitle
\vspace{1em}
\begin{minipage}[ht]{0.48\textwidth}
The Institute of Mathematical Sciences\\
The Chinese University of Hong Kong\\
Hong Kong, Shatin, N.T.
\end{minipage}
\begin{minipage}[ht]{0.48\textwidth}
Ukrainian\\
December 24, 1991\\
+886 9 573 12 46
\end{minipage}
\vspace{20pt}
 
\section*{Education}
\begin{tabular}{L!{\VRule}R}
2013--2014&{MPhil in Mathematics (differential topology; advisor: Yi-Jen Lee)}, Chinese University of Hong Kong, Hong Kong\vspace{6pt}\\
2009--2013&{BSc in Applied Math and Computer Science (double degree) \textbf{(average: 94.87)}}, National Chiao Tung University, Taiwan.\vspace{5pt}\\
2006--2009&High School Graduate, Nature Science Lyceum 145, Ukraine.\\
\end{tabular}

\section*{Key Courses taken}
\begin{tabular}{L!{\VRule}R}
4th year, 2nd sem. & Introduction to Differentiable Manifolds\\
4th year, 1st sem. & Functional Analysis\\
4th year, 1st sem. &Cryptography\\
3rd year, 2nd sem. &Computer Vision\\
3rd year, 2nd sem. &Video Compression\\
3rd year, 1st sem. &Computer Graphics\\
3rd year, 1st sem. &Topology\\
2nd year, 2nd sem. &Stochastic Processes\\
2nd year, 1st sem. &Probability\\
\end{tabular}

\section*{Areas of interest}
Geometry and Differential Topology, Stochastic Processes, PDE, Computer Science (Cryptography and Pseudo-Randomness), Computer Vision and Geometry Processing.

\section*{Working Experience \& Projects}
\begin{tabular}{L!{\VRule}R}
2013--2013&{\bf Google Summer of Code 2013}\\& 
I was a successful participant of Google Summer of Code 2013, the initiative, where students can have an opportunity to work on some
project for open-source company of their choice. I've submitted a proposal for Open Source Computer Vision Library (openCV) and it has passed
the competition. During the course of a project, I've implemented the generic numerical optimization module for openCV.\\
2013--2013&{\bf Pipeline for 3D models reconstruction}\\& 
Starting from my junior year I was working for Prof. Jong-Hong Chuang (Professor,
Department of Computer Science, NCTU). I was helping to implement the
framework for the reconstruction of 3D models from the partial scans of a given body.
Pipeline include registration (that is, grouping partial scans together to get the set of
points describing the whole object), filtering and normal estimation (that is, clearing the
result from previous step and estimating normal for each point) and mesh reconstruction. I was doing mostly the first part (registration). My most notable achievement is the
naive implementation of a Chavdar Papazov’s ”Stochastic Optimization for Rigid Point
Set Registration” algorithm.
 \\
2010--2013&{\bf Trying to solve Schrödinger equation of helium}\\& 
This project is lead by Henryk Witek (Professor, Department of Applied Chemistry, NCTU). Professor Witek is interested in analytical (as opposed
to numerical) solution to the Schrödinger equation of the helium. Under the guidance of Professor Witek I have been helping to
investigate the problem using asymptotic expansion. Due to the fair amount of algebra involved, we have used Maple computer algebra system. Besides
the asymptotic expansion, it was attempted to attack the problem using the Lie theory.
\\
2012--2012&{\bf Privacy-Preserving Smart Meter system}\\&
This is the implementation of a paper by Dr. Hsiao-Ying Lin. The project is a
simulation in software of so-called privacy-preserving smart meter, a hypothetic hardware. This hardware, being installed in houses
can bill the electricity (network traffic, gas consumption etc.) in privacy-preserving manner, that is, without revealing the sensitive data about
per-month consumption to provider. Thus, it somehow allows seeing sums without seeing addends. This is achieved by using techniques from secure
aggregation, subfield of cryptography. I've implemented the project using openSSL library on C language in one semester.
. The code, documentation and all the rights belong to Hsiao-Ying Lin.
\\
2011--2012&{\bf "Privacy-Preserving Keyword Search" project.}\\
&
This project was done in partial fulfilment of the graduation requirements for the Computer Science department in NCTU. Thee project was done together with Sean Lin in
a team of two. Our advisors were Prof. Bao-Shuh Paul Lin (Chair Professor, Department of Computer Science, NCTU) and Dr.Hsiao-Ying Lin (Research Fellow, Information
and Communication Technology Laboratories, NCTU). Thee project was financed with
the Diamond Project Scholarship. Within one year we have delivered the prototype of
a system that was able to do the keyword search in privacy-preserving way. During
the project we have learned about important issues, such as cryptographic security and
pseudo-randomness.
\\
2010--today&{\bf Calculus Tutor.}\\
&Starting from my sophomore year I am working as a Calculus Tutor in NCTU.\\
\end{tabular}

\section*{Honors \& Awards}
\begin{tabular}{L!{\VRule}R}
2013--2013&Successful participant in Google Summer of Code 2013\\
2013--2013&Nominated as an Exchange Student to Hong Kong University\\
2009--2013&Golden Bamboo Scholarship Award, NCTU\\
2011--2012&Diamond Project Scholarship, NCTU\\
2010--2012&Best Student in Class Award, NCTU\\
\end{tabular}
 
 
\section*{Languages}
\begin{tabular}{L!{\VRule}R}
Russian&Mother tongue\\
Ukrainian&Fluent\\
{\bf English}&{\bf Fluent (TOEFL iBT score 112/120 in 2012)}\\
Chinese&Intermediate (TOCFL 2012, level 3/5, score 96/100)\\
Japanese&Basic\\
\end{tabular}

\section*{Hobbies}
Skiing, swimming, weightlifting, Android programming.

{\vspace{20pt}
\vspace{20pt}
\scriptsize\hfill Based on a template from http://texblog.org}

\end{document}
