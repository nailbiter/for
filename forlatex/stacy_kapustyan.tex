\documentclass[10pt]{article}

\usepackage{mathtext}                 % підключення кирилиці у математичних формулах
                                          % (mathtext.sty входить в пакет t2).
\usepackage[T1,T2A]{fontenc}         % внутрішнє кодування шрифтів (може бути декілька);
                                          % вказане останнім діє по замовчуванню;
                                          % кириличне має співпадати з заданим в ukrhyph.tex.
\usepackage[utf8]{inputenc}       % кодування документа; замість cp866nav
                                          % може бути cp1251, koi8-u, macukr, iso88595, utf8.
\usepackage[english,russian,ukrainian]{babel} % національна локалізація; може бути декілька
                                          % мов; остання з переліку діє по замовчуванню. 
\usepackage{amsthm}
\usepackage{amsmath}
\usepackage{amsfonts}
\usepackage{graphicx}
\usepackage[pdftex]{hyperref}
\usepackage{caption}
\usepackage{subfig}
\usepackage{fancyhdr}
\usepackage{cancel}
\usepackage{ulem}

\newtheorem{prob}{Завдання}
\newcommand{\ds}{\;ds}
\newcommand{\dt}{\;dt}
\newcommand{\dx}{\;dx}
\newcommand{\dta}{\;d\tau}

\usepackage{mystyle}

\newtheorem{myulem}[mythm]{Лема}

\renewenvironment{myproof}[1][Доведення]{\begin{trivlist}
\item[\hskip \labelsep {\bfseries #1}]}{\myqed\end{trivlist}}
\title{}
\author{}
\begin{document}
\maketitle
\begin{prob}Довести, що
\[\begin{cases}\dot{x}=-x+ty\\\dot{y}=-tx-2y+te^{-t}\end{cases}\]
має два розв’язки, що прямують до 0 при $t\to\infty$.
\end{prob}
Помітимо, по-перше, що ця система є лінійною неоднорідною, а отже може бути приведена до однорідної, якщо ми знайдемо частковий розв’язок. З огляду
на вільний член $te^{-t}$, розв’язок шукатимемо у вигляді $x(t)=p(t)e^{-t},\;y(t)=q(t)e^{-t}$. Після підстановки і скорочення отримаємо
\[\begin{cases}p'=tq\\q'=-pt-q+t\end{cases}\]
Бачимо, що $p=1,q=1\implies x(t)=e^{-t},\;y(t)=0$ задовольняють систему. Таким чином, заміною $x(t)=a(t)-e^{-t},\;y(t)=b(t)$ отримуємо рівносильну
систему
\[\begin{cases}a'=-a+tb\\b'=-ta-2b\end{cases}\iff \begin{bmatrix}x'\\y'\end{bmatrix}=A\begin{bmatrix}x\\y\end{bmatrix},\;A(t):=\begin{bmatrix}
-1&t\\-t&-2\end{bmatrix}\]
Оскільки $x(t)=a(t)-e^{-t},\;y(t)=b(t)$, бачимо, що $(x,y)\to0\iff (a,b)\to0$ і таким чином достатньо показати, що система вище має два
розв’язки, що прямують до нуля. Одним, очевидно, є тривіальний розв’язок. Існування другого (нетривіального) дає нам теорема про асимптотичну
поведінкy системи із знаковизначеною матрицею. Її можна застосувати, адже матриця
\[\frac{A(t)+A^T(t)}{2}=\begin{bmatrix}-1&0\\0&-2\end{bmatrix}\]
є знаковизначеною і $\int_0^t\operatorname{tr}A(s)\;ds=\int_0^t-3\;ds=-3t\to-\infty$ при $t\to+\infty$.
\begin{prob}Довести, що
\[\begin{cases}\dot{x}=-x+(\cos t)y\\\dot{y}=(\cos t)x-y\end{cases}\] є асимптотично стійкою
\end{prob}
Ми застосуємо Теорему Лаппо-Данілєвського. Щоб її застосувати, необхідно спочатку перевірити відповідну умову. Позначимо
\[A(t):=\begin{bmatrix}-1&\cos t\\\cos t&-1\end{bmatrix},\;B(a,t):=\int_a^tA(s)\;ds=\begin{bmatrix}-t&\sin t\\\sin t&-t\end{bmatrix}\]
Нам потрібно показати, що для $\forall a,t$ маємо
\[A(t)(B(t)-B(a))=(B(t)-B(a))A(t)\]
Очевидно, що достатньо перевірити $\forall a,t,\;A(t)B(a)=B(a)A(t)$. Після безпосередніх розрахунків маємо,
\[A(t)B(a)=B(a)A(t)=\begin{bmatrix}a+\sin a\cos t&-\sin a-a\cos t\\-a\cos t-\sin a&\sin a\cos t+a\end{bmatrix}\]
Залишається перевірити, що існує границя
\[A:=\lim_{t\to\infty}\frac{1}{t}\int_a^tA(s)\;ds\]
вона існує, адже
\[\lim_{t\to\infty}\frac{1}{t}\int_a^tA(s)\;ds=\begin{bmatrix}-1&0\\0&-1\end{bmatrix}\]
всі власні числа $A$, тобто $-1$ та $-1$, мають від’ємну дійсну частину, що і показує асимптотичну стійкість за теоремою Лаппо-Данілєвського.
\end{document}
