\documentclass[12pt]{article} % use larger type; default would be 10pt

\usepackage{mathrsfs}
\usepackage{hyperref}
\usepackage[normalem]{ulem}
\usepackage{enumerate}
\usepackage{geometry}
\usepackage{setspace}
\usepackage{graphicx}
\usepackage{amsmath,amssymb,xypic}
\usepackage[all,cmtip]{xy}
\usepackage{float}
\usepackage{geometry,graphicx,bbm,latexsym}
\geometry{letterpaper}
\usepackage{xr}
\usepackage[normalem]{ulem}
\usepackage{caption}
\usepackage{setspace}
\usepackage{multirow}
\usepackage[table]{xcolor}
\usepackage{minibox}
\usepackage{subcaption}
\captionsetup{compatibility=false}
\usepackage{float}
\usepackage{tikz}
\usepackage{ulem}
\usepackage{footnote}
\usepackage{lpic}
\usepackage{array}
\usepackage{mystyle}

\theoremstyle{remark}
\newtheorem{remark}{Remark}

\newcommand{\longhookrightarrow}{{\lhook\joinrel\relbar\joinrel\rightarrow}}
\newcommand{\tmverbatim}[1]{ {\ttfamily #1} }
\newcommand{\tmscript}[1]{ {\scriptsize #1} }
\newcommand{\assign}{:=}
\renewcommand{\iff}{\Leftrightarrow}
\newcommand{\myInd}[1]{$\mbox{Ind}_{P_{\scriptsize max}}^G\left( \varepsilon\otimes \C_{#1} \right)$}

\begin{document}
\section{Notation}
We will use the following shortcuts in this note:\begin{equation*}
	\begin{array}[]{c}
		\mbox{for $b\in\Z$ we let }\myabra{m-n\le b}:=\displaystyle\bigoplus_{\scriptsize \begin{array}[]{c}
			m-n\le b\\ m-n\equiv b\mbox{ mod 2}
		\end{array}}\mathcal{H}^m(\C^{p})\otimes \mathcal{H}^n(\C^q)\\
		\mbox{similarly for $\myabra{n-m\le b}$, $\myabra{m-n\ge b}$ etc.}
	\end{array}
\end{equation*}
\begin{remark}
	I admit these shortcuts are bad. Any comments are appreciated.
\end{remark}
We also recall the definitions of {\ttfamily master\_announcement2.pdf}:
\begin{eqnarray}
  & b \equiv b (\lambda, p, q) \assign \lambda - \frac{p}{2} + \frac{q}{2} +
  1 \in \mathbb{Z},  \label{eqn:20-beplus} & \\
  & \lambda \in B (p, q) \assign \left\{ \begin{array}{ll}
    \frac{2 - p - q}{2} +\mathbb{N}_+, & p \in 2\mathbb{N}_+, q \in
    2\mathbb{N}_+ + 1,\\
    \frac{p - q}{2} +\mathbb{Z}, & p \in 2\mathbb{N}_+ + 1, q \in
    2\mathbb{N}_+,\\
    \left\{ \lambda \in \mathbb{Z}+ \frac{p + q}{2} \mid \lambda > - 1
    \right\} = A_0 (p, q), & \mbox{otherwise}
  \end{array} \right. &  \nonumber\\
  & 0 \rightarrow Y_{+, \lambda}^{p, q} \longhookrightarrow
  \mbox{Ind}_{P_{\max}}^G (\varepsilon \otimes \mathbb{C}_{\lambda}), & 
  \nonumber\\
  & \mbox{Ind}_{P_{\max}}^G (\varepsilon \otimes \mathbb{C}_{- \lambda})
  \twoheadrightarrow Y_{+, \lambda}^{p, q} \rightarrow 0, &  \nonumber\\
  & Y_{+, \lambda}^{p, q} \mid_K = \displaystyle\bigoplus_{\tmscript{\begin{array}{c}
    m - n \geqslant b (\lambda, p, q)\\
    m - n \equiv b (\lambda, p, q) \mbox{mod} 2
  \end{array}}} \mathcal{H}^m (\mathbb{S}^{p - 1}) \otimes \mathcal{H}^n
  (\mathbb{S}^{q - 1}), &  \nonumber\\
  & \varepsilon \equiv \varepsilon (\lambda, p, q) = (- 1)^{b (\lambda, p,
  q)}, &  \nonumber\\
  & \lambda \in B (q, p) \assign \left\{ \begin{array}{ll}
    \frac{2 - p - q}{2} +\mathbb{N}_+, & p \in 2\mathbb{N}_+ + 1, q \in
    2\mathbb{N}_+,\\
    \frac{q - p}{2} +\mathbb{Z}, & p \in 2\mathbb{N}_+, q \in 2\mathbb{N}_+
    + 1,\\
    \left\{ \lambda \in \mathbb{Z}+ \frac{p + q}{2} \mid \lambda > - 1
    \right\} = A_0 (q, p), & \mbox{otherwise}
  \end{array} \right. &  \nonumber\\
  & 0 \rightarrow Y_{-, \lambda}^{p, q} \longhookrightarrow
  \mbox{Ind}_{P_{\max}}^G (\varepsilon \otimes \mathbb{C}_{\lambda}), & 
  \nonumber\\
  & \mbox{Ind}_{P_{\max}}^G (\varepsilon \otimes \mathbb{C}_{- \lambda})
  \twoheadrightarrow Y_{-, \lambda}^{p, q} \rightarrow 0, &  \nonumber\\
  & Y_{-, \lambda}^{p, q} \mid_K = \displaystyle\bigoplus_{\tmscript{\begin{array}{c}
    n - m \geqslant b (\lambda, q, p)\\
    n - m \equiv b (\lambda, q, p) \mbox{mod} 2
  \end{array}}} \mathcal{H}^m (\mathbb{S}^{p - 1}) \otimes \mathcal{H}^n
  (\mathbb{S}^{q - 1}),  \label{eqn:20-beminus} & \\
  & \varepsilon = (- 1)^{b (\lambda, q, p)} . &  \nonumber
\end{eqnarray}
\section{$p$ odd, $q$ even}
\begin{center}
\begin{tabular}[]{c|c|c}
	$\begin{array}[]{c}
		\lambda=\\-a-\frac{p+q}{2}+1
	\end{array}$
	&affected by $A^{+\pm}$ (i.e. $\varepsilon=(-1)^a$)&affected by $A^{-\pm}$ (i.e. $\varepsilon=(-1)^{a+1}$)\\
	\hline\\
	 $\begin{array}[]{c}
	a\in\N\iff\\
	\lambda\in 1-\frac{p+q}{2}-\N
\end{array}$&
{\begin{lpic}[]{howetan2d30(0.5)}
		\lbl[bl]{70,40;\scriptsize $\bullet$}
		\lbl[bl]{57,30;\scriptsize $ Y_{+,-\lambda}^{p,q}=\myabra{m-n\ge a+q}$}
		\lbl[bl]{25,70; \myInd{\lambda}}
	\end{lpic}}&
	{\begin{lpic}[]{howetan2d31(0.5)}
		\lbl[bl]{25,85; \myInd{\lambda}}
		\lbl[bl]{36,65;\scriptsize $ \bullet Y_{-,-\lambda}^{p,q}=\myabra{m-n\le -a-p}$}
		\lbl[bl]{32,30;\scriptsize $ \bullet Y_{+,\lambda}^{p,q}=\myabra{m-n\ge -a-p+2}$}
	\end{lpic}}\\\hline\\
	&{\begin{tabular}[]{c}
		$m-n=a+q-2$ intersects\\
		$m=0$ (``$y$-axis'')
\end{tabular}}&{\begin{tabular}[]{c}
	a\\b
\end{tabular}}
\end{tabular}
\end{center}
\end{document}
