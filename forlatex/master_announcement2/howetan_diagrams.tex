\documentclass[12pt]{article} % use larger type; default would be 10pt

\usepackage{mathrsfs}
\usepackage{hyperref}
\usepackage[normalem]{ulem}
\usepackage{enumerate}
\usepackage{geometry}
\usepackage{setspace}
\usepackage{graphicx}
\usepackage{amsmath,amssymb,xypic}
\usepackage[all,cmtip]{xy}
\usepackage{float}
\usepackage{geometry,graphicx,bbm,latexsym,theorem}
\geometry{letterpaper}
\usepackage{xr}
%\usepackage{mystyle}
\usepackage[normalem]{ulem}
\usepackage{caption}
\usepackage{setspace}
\usepackage{multirow}
\usepackage[table]{xcolor}
\usepackage{minibox}
\usepackage{subcaption}
\captionsetup{compatibility=false}
\usepackage{float}
\usepackage{tikz}
\usepackage{ulem}
\usepackage{footnote}

\theoremstyle{remark}
\newtheorem{remark}{Remark}

\newcommand{\longhookrightarrow}{{\lhook\joinrel\relbar\joinrel\rightarrow}}
\newcommand{\tmverbatim}[1]{ {\ttfamily #1} }
\newcommand{\tmscript}[1]{ {\scriptsize #1} }
\newcommand{\assign}{:=}

\begin{document}
%%\begin{remark}
%%  More detailed (and hopefully, correct) versions of Figures
%%  \ref{fig:howetan2d31}, \ref{fig:howetan2d32} is attached as
%%  \tmverbatim{Ypm.pdf}.
%%\end{remark}

In subsequent let us assume $p \geqslant 1, q \geqslant 2$ for simplicity. Let
us recall the definitions we have made in
\tmverbatim{master\_announcement2.pdf}:

We define as in \cite{KO2} \footnote{I will study well-definedness of $Y_{\pm, \lambda}^{p, q}$ in the next version}
\begin{eqnarray}
  & b \equiv b (\lambda, p, q) \assign \lambda - \frac{p}{2} + \frac{q}{2} +
  1 \in \mathbb{Z},  \label{eqn:20-beplus} & \\
  & \lambda \in B (p, q) \assign \left\{ \begin{array}{ll}
    \frac{2 - p - q}{2} +\mathbb{N}_+, & p \in 2\mathbb{N}_+, q \in
    2\mathbb{N}_+ + 1,\\
    \frac{p - q}{2} +\mathbb{Z}, & p \in 2\mathbb{N}_+ + 1, q \in
    2\mathbb{N}_+,\\
    \left\{ \lambda \in \mathbb{Z}+ \frac{p + q}{2} \mid \lambda > - 1
    \right\} = A_0 (p, q), & \mbox{otherwise}
  \end{array} \right. &  \nonumber\\
  & 0 \rightarrow Y_{+, \lambda}^{p, q} \longhookrightarrow
  \mbox{Ind}_{P_{\max}}^G (\varepsilon \otimes \mathbb{C}_{\lambda}), & 
  \nonumber\\
  & \mbox{Ind}_{P_{\max}}^G (\varepsilon \otimes \mathbb{C}_{- \lambda})
  \twoheadrightarrow Y_{+, \lambda}^{p, q} \rightarrow 0, &  \nonumber\\
  & Y_{+, \lambda}^{p, q} \mid_K = \bigoplus_{\tmscript{\begin{array}{c}
    m - n \geqslant b (\lambda, p, q)\\
    m - n \equiv b (\lambda, p, q) \mbox{mod} 2
  \end{array}}} \mathcal{H}^m (\mathbb{S}^{p - 1}) \otimes \mathcal{H}^n
  (\mathbb{S}^{q - 1}), &  \nonumber\\
  & \varepsilon \equiv \varepsilon (\lambda, p, q) = (- 1)^{b (\lambda, p,
  q)}, &  \nonumber\\
  & \lambda \in B (q, p) \assign \left\{ \begin{array}{ll}
    \frac{2 - p - q}{2} +\mathbb{N}_+, & p \in 2\mathbb{N}_+ + 1, q \in
    2\mathbb{N}_+,\\
    \frac{q - p}{2} +\mathbb{Z}, & p \in 2\mathbb{N}_+, q \in 2\mathbb{N}_+
    + 1,\\
    \left\{ \lambda \in \mathbb{Z}+ \frac{p + q}{2} \mid \lambda > - 1
    \right\} = A_0 (q, p), & \mbox{otherwise}
  \end{array} \right. &  \nonumber\\
  & 0 \rightarrow Y_{-, \lambda}^{p, q} \longhookrightarrow
  \mbox{Ind}_{P_{\max}}^G (\varepsilon \otimes \mathbb{C}_{\lambda}), & 
  \nonumber\\
  & \mbox{Ind}_{P_{\max}}^G (\varepsilon \otimes \mathbb{C}_{- \lambda})
  \twoheadrightarrow Y_{-, \lambda}^{p, q} \rightarrow 0, &  \nonumber\\
  & Y_{-, \lambda}^{p, q} \mid_K = \bigoplus_{\tmscript{\begin{array}{c}
    n - m \geqslant b (\lambda, q, p)\\
    n - m \equiv b (\lambda, q, p) \mbox{mod} 2
  \end{array}}} \mathcal{H}^m (\mathbb{S}^{p - 1}) \otimes \mathcal{H}^n
  (\mathbb{S}^{q - 1}),  \label{eqn:20-beminus} & \\
  & \varepsilon = (- 1)^{b (\lambda, q, p)} . &  \nonumber
\end{eqnarray}
Now, let us first consider Figure \ref{fig:howetan2d31}. We recall that this
diagram describes the socle filtration of $S^{a \varepsilon} (X^0)$ for
\[ \varepsilon = (- 1)^{a + 1}, \quad p \in 2\mathbb{N}+ 1, \quad q \in
   2\mathbb{N}, \quad \mathbb{Z} \ni a > 4 - (p + q) . \]
We note that the notation of {\cite{howe1993homogeneous}} and {\cite{KO2}} are
connected via
\begin{eqnarray}
  & \mbox{Ind}_{P_{\max}}^G \left( \varepsilon \otimes \mathbb{C}_{- a -
  \frac{p + q}{2} + 1} \right) = S^{a \varepsilon} (X^0), &  \nonumber
\end{eqnarray}
hence Figure \ref{fig:howetan2d31} describes the socle filtration of
$\mbox{Ind}_{P_{\max}}^G \left( (- 1)^{a + 1} \otimes \mathbb{C}_{- a -
\frac{p + q}{2} + 1} \right)$. Now, substituting this in
$(\ref{eqn:20-beplus})$, we get
\begin{eqnarray}
  & b \equiv b (\lambda, p, q) = \left( - a - \frac{p + q}{2} + 1 \right) -
  \frac{p - q}{2} + 1 = - a - p + 2, &  \nonumber\\
  & \varepsilon = (- 1)^{- a - p + 2} = (- 1)^{a + 1}, &  \nonumber
\end{eqnarray}
hence the boundary $m - n \leqslant - a - p + 2$ crosses the axis $m = 0$ and
$n = a + p - 2$, which is precisely as Figure \ref{fig:howetan2d31}.

Next, consider the Figure \ref{fig:howetan2d32}. It describes the socle
filtration $S^{a \varepsilon}, \varepsilon = (- 1)^a$ for $p \in 2\mathbb{N}+
1, q \in 2\mathbb{N}, \mathbb{Z} \ni a > 0$. Substituting this in
$(\ref{eqn:20-beminus})$, we get
\begin{eqnarray}
  & m - n \leqslant - \left( - a - \frac{p + q}{2} + 1 \right) - \frac{p -
  q}{2} - 1 = a + q - 2, &  \nonumber\\
  & \varepsilon = (- 1)^{a + p - 2} = (- 1)^a,^{} &  \nonumber
\end{eqnarray}
hence the boundary $m - n \leqslant a + q - 2$ crosses the axis $n = 0$ and $m
= a + q - 2$, which is precisely as Figure \ref{fig:howetan2d32}.

\begin{thebibliography}{99}
\expandafter\ifx\csname urlstyle\endcsname\relax
  \providecommand{\doi}[1]{doi:\discretionary{}{}{}#1}\else
  \providecommand{\doi}{doi:\discretionary{}{}{}\begingroup
  \urlstyle{rm}\Url}\fi

\bibitem{bernstein2004estimates}
J.~Bernstein and A.~Reznikov.
\newblock Estimates of automorphic functions.
\newblock \emph{{\normalfont Mosc. Math. J}}, \textbf{\textbf{4}}, (2004),
  pp.~19--37.

  \bibitem{clerc2011generalized}
J.-L. Clerc, T.~Kobayashi, B.~{\O}rsted and M.~Pevzner.
\newblock Generalized {B}ernstein--{R}eznikov integrals.
\newblock \emph{{\normalfont Math.~Ann.}}, \textbf{349}, (2011),.
\href{http://dx.doi.org/10.1007/s00208-010-0516-4}{pp.~395--431}.

\bibitem{howe1993homogeneous}
R.~E. Howe and E.-C. Tan.
\newblock Homogeneous functions on light cones: the infinitesimal structure of
  some degenerate principal series representations.
\newblock \emph{{\normalfont Bull.~Amer.~Math.~Soc.}}, \textbf{28},
  (1993), pp.~1--74.

\bibitem{juhl2009families}
A.~Juhl.
\newblock \emph{Families of {C}onformally {C}ovariant {D}ifferential
  {O}perators, {Q}-curvature and {H}olography}, \emph{{\normalfont Progr.~ Math.},} \textbf{275},
\newblock Birkh{\"a}user (2009).
\newblock ISBN 978-3-7643-9900-9.

\bibitem{kobyashi1992singular}
T.~Kobayashi.
\newblock \emph{Singular unitary representations and discrete series for indefinite Stiefel manifolds $U (p, q; \mathbb{F})/U (p-m, q; \mathbb{F})$},
Mem. Amer. Soc., \textbf{\href{http://www.ams.org/bookstore-getitem/item=MEMO-95-462}{462}}, Amer. Math. Soc., (1992).

\bibitem{kobayashi93restriction}
\newblock
T.~Kobayashi. \emph{The restriction of $A_q \left( \lambda \right)$ to reductive subgroups},
Proc. Japan Acad. Ser. A Math. Sci. 69 (1993), no. 7, 262--267.

\bibitem{kobayashi1998discrete2}
T.~Kobayashi.
\newblock Discrete decomposability of the restriction of {$A_q(\lambda)$} with
  respect to reductive subgroups {II}: Micro-local analysis and asymptotic
  {K}-support.
  \newblock \emph{{\normalfont Ann. Math. (2)}}, \textbf{147}, (1998),
\href{http://dx.doi.org/10.2307/120963}{pp.~709--729}.

\bibitem{kobayashi1998discrete3}
T.~Kobayashi.
\newblock Discrete decomposability of the restriction of {$A_q(\lambda)$} with
  respect to reductive subgroups {III}. {R}estriction of {H}arish-{C}handra
  modules and associated varieties.
\newblock \emph{{\normalfont Invent. Math.}}, \textbf{131}, (1998), 
\href{http://dx.doi.org/10.1007/s002220050203}{pp.~229--256}.

\bibitem{Kobayashi2014}
T.~Kobayashi.
\newblock {S}hintani functions, real spherical manifolds, and
  symmetry breaking operators.
  \newblock \emph{{\normalfont Dev.~Math.}}, \textbf{37}, (2014),
 \href{http://dx.doi.org/10.4171/OWR/2014/3}{pp.~127--159}.

\bibitem{kobayashi2015program}
T.~Kobayashi.
\newblock A program for branching problems in the representation theory of real
  reductive groups.
\newblock \emph{{\normalfont Progr.~Math.}}, \textbf{312}, (2015), 
\href{http://dx.doi.org/10.1007/978-3-319-23443-4_10}{pp.~277--322}.
\newblock In: \emph{{\normalfont Special issue in honor of Vogan's 60th years
  birthday}}.

\bibitem{kokupe2016forms}
T.~Kobayashi, T.~Kubo, and M.~Pevzner,
\newblock 
Conformal symmetry breaking operators for anti-de Sitter spaces.
preprint, 
\href{https://arxiv.org/abs/1610.09475}{arXiv:1610.09475}.

\bibitem{kobayashi2014classification}
T.~Kobayashi and T.~Matsuki.
\newblock Classification of finite-multiplicity symmetric pairs.
\newblock \emph{{\normalfont Transformation Groups}}, \textbf{19}, (2014),
\href{http://dx.doi.org/10.1007/s00031-014-9265-x}{pp.~457--493}.
\newblock In: \emph{{\normalfont Special Issue in honour of Dynkin
  for his 90th birthday}}.


  \bibitem{KO1}
T.~Kobayashi and B.~{\O}rsted.
\newblock Analysis on the minimal representation of\/ {${\rm
  O}(p,q)$}.{\;}{{\rm{I}}. Realization via conformal geometry}.
\newblock \emph{\normalfont Adv.~Math.}, \textbf{180}, (2003),
\href{http://dx.doi.org/10.1016/S0001-8708(03)00012-4}{pp.~486--512}.

  \bibitem{KO2}
T.~Kobayashi and B.~{\O}rsted.
\newblock Analysis on the minimal representation of\/ {${\rm O}(p,q)$}.{\;}{{\rm{II}}}. {B}ranching laws.
\newblock \emph{\normalfont Adv.~Math.}, \textbf{180}, (2003),
\href{http://dx.doi.org/10.1016/S0001-8708(03)00013-6}{pp.~513--550}.

\bibitem{kobayashi2015branching}
T.~Kobayashi, B.~{\O}rsted, P.~Somberg and V.~Sou{\v{c}}ek.
\newblock Branching laws for verma modules and applications in parabolic
  geometry. {I}.
\newblock \emph{{\normalfont Adv.~Math.}}, \textbf{285}, (2015),
\href{http://dx.doi.org/10.1016/j.aim.2015.08.020}{pp.~1796--1852}.

\bibitem{kobayashi2013finite}
T.~Kobayashi and T.~Oshima.
\newblock Finite multiplicity theorems for induction and restriction.
\newblock \emph{{\normalfont Adv.~Math.}}, \textbf{248}, (2013), 
 \href{http://dx.doi.org/10.1016/j.aim.2013.07.015}{pp.~921--944}.

\bibitem{kobayashi2016differential1}
T.~Kobayashi and M.~Pevzner.
\newblock Differential symmetry breaking operators: I. {G}eneral theory and
  {F}-method.
\newblock \emph{{\normalfont Selecta Math.}}, \textbf{22}, (2016),
\href{http://dx.doi.org/10.1007/s00029-015-0207-9}{pp.~801--845}.

\bibitem{kobayashi2015symmetry}
T.~Kobayashi and B.~Speh.
\newblock \emph{Symmetry {B}reaking for {R}epresentations of {R}ank {O}ne
  {O}rthogonal {G}roups}, \emph{{\normalfont Memoirs of the Amer.~Math.~Soc},}
  \textbf{\href{http://dx.doi.org/10.1090/memo/1126}{238}}, (2015).
\newblock ISBN 978-1-4704-1922-6.

\bibitem{vogan1984unitarizability}
D.~Vogan. 
\newblock Unitarizability of Certain Series of Representations.
\newblock Ann. of Math., \textbf{120}, (1984), \href{www.jstor.org/stable/2007074}{pp.~141–-187}.


\bibitem{wallach1988real2}
N.~Wallach.
\newblock \emph{Real Reductive Groups II}, \emph{{\normalfont Pure and Applied
  Mathematics},} \textbf{132},
\newblock Academic {P}ress (1992).
\newblock ISBN 978-0127329611.

\end{thebibliography}
\end{document}
