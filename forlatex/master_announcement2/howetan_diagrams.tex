\documentclass[12pt]{article} % use larger type; default would be 10pt

\usepackage{dashrule}
\usepackage{mathrsfs}
\usepackage{hyperref}
\usepackage[normalem]{ulem}
\usepackage{enumerate}
\usepackage{geometry}
\usepackage{setspace}
\usepackage{graphicx}
\usepackage{amsmath,amssymb,xypic}
\usepackage[all,cmtip]{xy}
\usepackage{float}
\usepackage{geometry,graphicx,bbm,latexsym}
\geometry{letterpaper}
\usepackage{xr}
\usepackage[normalem]{ulem}
\usepackage{caption}
\usepackage{setspace}
\usepackage{multirow}
\usepackage[table]{xcolor}
\usepackage{minibox}
\usepackage{subcaption}
\captionsetup{compatibility=false}
\usepackage{float}
\usepackage{tikz}
\usepackage{ulem}
\usepackage{footnote}
\usepackage{lpic}
\usepackage{array}
\usepackage{mystyle}

\theoremstyle{plain}
\newtheorem{definition}{Definition}
\theoremstyle{remark}
\newtheorem{remark}{Remark}

\newcommand{\longhookrightarrow}{{\lhook\joinrel\relbar\joinrel\rightarrow}}
\newcommand{\tmverbatim}[1]{ {\ttfamily #1} }
\newcommand{\tmscript}[1]{ {\scriptsize #1} }
\newcommand{\assign}{:=}
\renewcommand{\iff}{\Leftrightarrow}
\newcommand{\myInd}[1]{\mbox{Ind}_{P_{\mbox{\scriptsize\normalfont max}}}^G\left( \varepsilon\otimes \C_{#1} \right)}

\begin{document}
\section{Notation}
We will use the following shortcuts in this note:\begin{equation*}
	\begin{array}[]{c}
		\mbox{for $b\in\Z$ we let }\myabra{m-n\le b}:=\displaystyle\bigoplus_{\scriptsize \begin{array}[]{c}
			m-n\le b\\ m-n\equiv b\mbox{ mod 2}
		\end{array}}\mathcal{H}^m(\Sp^{p-1})\otimes \mathcal{H}^n(\Sp^{q-1})\\
		\mbox{similarly for $\myabra{n-m\le b}$, $\myabra{m-n\ge b}$ etc.}
	\end{array}
\end{equation*}
\begin{remark}
	I admit these shortcuts are bad. Any comments are appreciated.
\end{remark}
We also recall the definitions of {\ttfamily master\_announcement2.pdf}:
\begin{definition}[$Y_{+,\lambda}^{p,q}$]
	For $\lambda\in B(p,q)$ with\begin{equation*}
		B (p, q) \assign \left\{ \begin{array}{ll}
    \frac{2 - p - q}{2} +\mathbb{N}_+, & p \in 2\mathbb{N}_+, q \in
    2\mathbb{N}_+ + 1,\\
    \frac{p - q}{2} +\mathbb{Z}, & p \in 2\mathbb{N}_+ + 1, q \in
    2\mathbb{N}_+,\\
    \left\{ \lambda \in \mathbb{Z}+ \frac{p + q}{2} \mid \lambda > - 1
    \right\} = A_0 (p, q), & \mbox{otherwise}
  \end{array} \right. 
	\end{equation*}
	we let
	\begin{equation}
	b \equiv b (\lambda, p, q) \assign \lambda - \frac{p}{2} + \frac{q}{2} +
  1 \in \mathbb{Z},  \label{eqn:20-beplus} 	
	\end{equation}
	and define $Y_{+,\lambda}^{p,q}$ as submodule of $\myInd{\lambda},\varepsilon=(-1)^{b(\lambda,p,q)}$ with the $K$-type
	\begin{equation*}
		\begin{array}[]{c}
   Y_{+, \lambda}^{p, q} \mid_K = \displaystyle\bigoplus_{\tmscript{\begin{array}{c}
    m - n \geqslant b (\lambda, p, q)\\
    m - n \equiv b (\lambda, p, q) \mbox{\normalfont\;mod } 2
  \end{array}}} \mathcal{H}^m (\mathbb{S}^{p - 1}) \otimes \mathcal{H}^n
  (\mathbb{S}^{q - 1}),\\
%%  \lambda \in B (q, p) \assign \left\{ \begin{array}{ll}
%%    \frac{2 - p - q}{2} +\mathbb{N}_+, & p \in 2\mathbb{N}_+ + 1, q \in
%%    2\mathbb{N}_+,\\
%%    \frac{q - p}{2} +\mathbb{Z}, & p \in 2\mathbb{N}_+, q \in 2\mathbb{N}_+
%%    + 1,\\
%%    \left\{ \lambda \in \mathbb{Z}+ \frac{p + q}{2} \mid \lambda > - 1
%%    \right\} = A_0 (q, p), & \mbox{otherwise}
%%  \end{array} \right.\\
		\end{array}
	\end{equation*}
\end{definition}
\begin{definition}[$Y_{-,\lambda}^{p,q}$]
	For $\lambda\in B(q,p)$	we
	 define $Y_{-,\lambda}^{p,q}$ as submodule of $\myInd{\lambda},\varepsilon=(-1)^{b(\lambda,q,p)}$ with the $K$-type
	\begin{equation*}
		\begin{array}[]{c}
   Y_{-, \lambda}^{p, q} \mid_K = \displaystyle\bigoplus_{\tmscript{\begin{array}{c}
    n-m \geqslant b (\lambda, q,p)\\
    n-m \equiv b (\lambda, q, p) \mbox{\normalfont\; mod\;} 2
  \end{array}}} \mathcal{H}^m (\mathbb{S}^{p - 1}) \otimes \mathcal{H}^n
  (\mathbb{S}^{q - 1}),\\
%%  \lambda \in B (q, p) \assign \left\{ \begin{array}{ll}
%%    \frac{2 - p - q}{2} +\mathbb{N}_+, & p \in 2\mathbb{N}_+ + 1, q \in
%%    2\mathbb{N}_+,\\
%%    \frac{q - p}{2} +\mathbb{Z}, & p \in 2\mathbb{N}_+, q \in 2\mathbb{N}_+
%%    + 1,\\
%%    \left\{ \lambda \in \mathbb{Z}+ \frac{p + q}{2} \mid \lambda > - 1
%%    \right\} = A_0 (q, p), & \mbox{otherwise}
%%  \end{array} \right.\\
		\end{array}
	\end{equation*}
\end{definition}
%%\begin{eqnarray}
%%  0 \rightarrow Y_{-, \lambda}^{p, q} \longhookrightarrow
%%  \myInd{\lambda}, &
%%  \nonumber\\
%%  %& \myInd{-\lambda}\twoheadrightarrow Y_{-, \lambda}^{p, q} \rightarrow 0, &  \nonumber\\
%%  & Y_{-, \lambda}^{p, q} \mid_K = \displaystyle\bigoplus_{\tmscript{\begin{array}{c}
%%    n - m \geqslant b (\lambda, q, p)\\
%%    n - m \equiv b (\lambda, q, p) \mbox{mod} 2
%%  \end{array}}} \mathcal{H}^m (\mathbb{S}^{p - 1}) \otimes \mathcal{H}^n
%%  (\mathbb{S}^{q - 1}),  \label{eqn:20-beminus} & \\
%%  & \varepsilon = (- 1)^{b (\lambda, q, p)} . &  \nonumber
%%\end{eqnarray}
\section{$p$ odd, $q$ even}
\hspace*{-2cm}
\begin{tabular}[]{c|c|c}
	$\begin{array}[]{c}
		\lambda=\\-a-\frac{p+q}{2}+1
	\end{array}$
	&$\begin{array}[]{c}\mbox{affected by }A^{+\pm}\\ \mbox{(i.e. $\varepsilon=(-1)^a=(-1)^{b(\lambda,q,p)}$)}\end{array}$&$\begin{array}[]{@{}c@{}}\mbox{affected by $A^{-\pm}$}\\\mbox{(i.e. $\varepsilon=(-1)^{a+1}=(-1)^{b(\lambda,p,q)}$)}\end{array}$ \\
	\hline\\
	 $\begin{array}[]{c}
	a\in\N\iff\\
	\lambda\in 1-\frac{p+q}{2}-\N
\end{array}$&
{\begin{lpic}[]{howetan2d30(0.5)}
		\lbl[bl]{70,40;\scriptsize $\bullet$}
		\lbl[bl]{57,30;\scriptsize $ Y_{+,-\lambda}^{p,q}=\myabra{m-n\ge a+q}$}
		\lbl[bl]{25,70; $\myInd{\lambda}$}

		\lbl[bl]{10,-11,45; \makebox[3cm]{\dotfill}}
		\lbl[bl]{9.4,-10,90; \makebox[3cm]{\hrulefill}}
		\lbl[bl]{9.0,-10; $\bullet$}
		\lbl[bl]{13.8,-11; \scriptsize$ 2-a-q =\lambda+\frac{p-q}{2}+1= b\left( \lambda,q,p \right)$}
		\lbl[bl]{71,19; \scriptsize$ =-\lambda-\frac{p-q}{2}-1=-b\left( \lambda,q,p \right)$}
		
%%		\lbl[bl]{25,70; \makebox[3cm]{\dotfill}} %dot 
%%		\lbl[bl]{25,65; \makebox[3cm]{\hrulefill}} %solid
%%		\lbl[bl]{25,65; \hdashrule[0.5ex]{3cm}{1pt}{3mm}} %dash
	\end{lpic}}\vspace{2em}
	&
	{\begin{lpic}[]{howetan2d31(0.5)}
		\lbl[bl]{25,85; $\myInd{\lambda}$}
		\lbl[bl]{36,65;\scriptsize $ \bullet Y_{-,-\lambda}^{p,q}=\myabra{m-n\le -a-p}$}
		\lbl[bl]{32,30;\scriptsize $ \bullet Y_{+,\lambda}^{p,q}=\myabra{m-n\ge -a-p+2}$}
		\lbl[bl]{3,21; \makebox[3cm]{\hrulefill}} %solid
		\lbl[bl]{10,21,45; \makebox[3cm]{\dotfill}}
		\lbl[bl]{10,20; $\bullet$}
		\lbl[bl]{10,13; \scriptsize $ 2-a-p = \lambda-\frac{p-q}{2}+1=b\left( \lambda,p,q \right)$}
		\lbl[bl]{32,37; \scriptsize $=-\lambda+\frac{p-q}{2}-1=-b\left( \lambda,p,q \right)$}
	\end{lpic}}\\\hline\\
structure&
{
\begin{tabular}[]{l}
	$Y_{+,-\lambda}^{p,q}$ appears as \textbf{quotient}\\
	note that $\lambda\in B(p,q)=\frac{p-q}{2}+\Z=\frac{1}{2}+\Z$
\end{tabular}
}
&{
	\begin{tabular}[]{l}
$Y_{+,\lambda}^{p,q}$ appears as \textbf{submodule}\\
(note that $\lambda\in B(p,q)=\frac{p-q}{2}+\Z=\frac{1}{2}+\Z$)\\
$Y_{-,-\lambda}^{p,q}$ appears as \textbf{quotient}\\
%%note that $\lambda\in B(p,q)\cap B(q,p)$\\
%%$B(q,p)=$
	\end{tabular}
}
\end{tabular}
\newpage
\hspace*{-4cm}
\begin{tabular}[]{@{}c@{}|c|c}
	$\begin{array}[]{c}
		\lambda=\\-a-\frac{p+q}{2}+1
	\end{array}$
	&$\begin{array}[]{c}\mbox{affected by }A^{+\pm}\\ \mbox{(i.e. $\varepsilon=(-1)^a=(-1)^{b(\lambda,q,p)}$)}\end{array}$&$\begin{array}[]{@{}c@{}}\mbox{affected by $A^{-\pm}$}\\\mbox{(i.e. $\varepsilon=(-1)^{a+1}=(-1)^{b(\lambda,p,q)}$)}\end{array}$ \\
	\hline\\
	 $\begin{array}[]{c}
	-\N_+\ni a\ge3-(p+q)\iff\\
	1-\frac{p+q}{2}<\lambda\le\frac{p+q}{2}-2\mid \lambda\in\frac{1}{2}+\Z
\end{array}$&
{\begin{lpic}[]{howetan2d32(0.5)}%draft
		\lbl[bl]{70,40;\scriptsize $\bullet$}
		\lbl[bl]{47,30;\scriptsize $ Y_{+,-\lambda}^{p,q}=\myabra{m-n\ge a+q}$}
		\lbl[bl]{15,70; $\myInd{\lambda}$}
		\lbl[bl]{10,60;\scriptsize $\bullet Y_{-,\lambda}^{p,q}=\myabra{m-n\le a+q-2}$}

		\lbl[bl]{4.4,8,45; \makebox[3cm]{\dotfill}}
		\lbl[bl]{4.4,-10,90; \makebox[3cm]{\hrulefill}}
		\lbl[bl]{4,6; $\bullet$}
		\lbl[bl]{10,8; \scriptsize$2-a-q =\lambda+\frac{p-q}{2}+1= b\left( \lambda,q,p \right)$}
			
		\lbl[bl]{38,19; \scriptsize$ =-\lambda-\frac{p-q}{2}-1=-b\left( \lambda,q,p \right)$}
	\end{lpic}}&
	{\begin{lpic}[]{howetan2d31(0.5)}
		\lbl[bl]{25,85; $\myInd{\lambda}$}
		\lbl[bl]{36,65;\scriptsize $ \bullet Y_{-,-\lambda}^{p,q}=\myabra{m-n\le -a-p}$}
		\lbl[bl]{32,30;\scriptsize $ \bullet Y_{+,\lambda}^{p,q}=\myabra{m-n\ge -a-p+2}$}
		\lbl[bl]{3,21; \makebox[3cm]{\hrulefill}} %solid
		\lbl[bl]{10,21,45; \makebox[3cm]{\dotfill}}
		\lbl[bl]{10,20; $\bullet$}
		\lbl[bl]{10,13; \scriptsize $ 2-a-p = \lambda-\frac{p-q}{2}+1=b\left( \lambda,p,q \right)$}
		\lbl[bl]{32,37; \scriptsize $=-\lambda+\frac{p-q}{2}-1=-b\left( \lambda,p,q \right)$}
	\end{lpic}}
\\\hline structure&
{
\begin{tabular}[]{l}
	$Y_{+,-\lambda}^{p,q}$ appears as \textbf{quotient}\\
	(note that $-\lambda\in B(p,q)$)\\
	$Y_{-,\lambda}^{p,q}$ appears as \textbf{submodule}\\
	(note that $\lambda\in B(q,p)=1-\frac{p+q}{2}+\N_+$)\\
%%	note that $\lambda\in B(p,q)=\frac{p-q}{2}+\Z=\frac{1}{2}+\Z$
\end{tabular}
}
&{
	\begin{tabular}[]{l}
$Y_{+,\lambda}^{p,q}$ appears as \textbf{submodule}\\
(note that $\lambda\in B(p,q)$)\\
$Y_{-,-\lambda}^{p,q}$ appears as \textbf{quotient}\\
note that $-\lambda\in B(p,q)\cap B(q,p)$\\
%%$B(q,p)=$
	\end{tabular}
}
\end{tabular}
\newpage
\hspace*{-4cm}
\begin{tabular}[]{@{}c@{}|c|c}
	$\begin{array}[]{c}
		\lambda=\\-a-\frac{p+q}{2}+1
	\end{array}$
	&$\begin{array}[]{c}\mbox{affected by }A^{+\pm}\\ \mbox{(i.e. $\varepsilon=(-1)^a=(-1)^{b(\lambda,q,p)}$)}\end{array}$&$\begin{array}[]{@{}c@{}}\mbox{affected by $A^{-\pm}$}\\\mbox{(i.e. $\varepsilon=(-1)^{a+1}=(-1)^{b(\lambda,p,q)}$)}\end{array}$ \\
	\hline\\
	 $\begin{array}[]{c}
	-\N_+\ni a<3-\left( p+q \right)\iff\\
	\frac{p+q}{2}-2<\lambda\mid \lambda\in\frac{1}{2}+\Z
\end{array}$&
{\begin{lpic}[]{howetan2d32(0.5)}%draft
		\lbl[bl]{70,40;\scriptsize $\bullet$}
		\lbl[bl]{47,30;\scriptsize $ Y_{+,-\lambda}^{p,q}=\myabra{m-n\ge a+q}$}
		\lbl[bl]{15,70; $\myInd{\lambda}$}
		\lbl[bl]{10,60;\scriptsize $\bullet Y_{-,\lambda}^{p,q}=\myabra{m-n\le a+q-2}$}

		\lbl[bl]{4.4,8,45; \makebox[3cm]{\dotfill}}
		\lbl[bl]{4.4,-10,90; \makebox[3cm]{\hrulefill}}
		\lbl[bl]{4,6; $\bullet$}
		\lbl[bl]{10,8; \scriptsize$2-a-q =\lambda+\frac{p-q}{2}+1= b\left( \lambda,q,p \right)$}
			
		\lbl[bl]{38,19; \scriptsize$ =-\lambda-\frac{p-q}{2}-1=-b\left( \lambda,q,p \right)$}
	\end{lpic}}&
	{\begin{lpic}[]{howetan2d33(0.5)}
		\lbl[bl]{25,85; $\myInd{\lambda}$}
		%\lbl[bl]{36,65;\scriptsize $ \bullet Y_{-,-\lambda}^{p,q}=\myabra{m-n\le -a-p}$}
		\lbl[bl]{110,26;\scriptsize $  Y_{+,\lambda}^{p,q}=\myabra{m-n\ge -a-p+2}$}
		\lbl[bl]{150,36;\scriptsize $\bullet$}

		\lbl[bl]{46.3,-30,45; \makebox[4cm]{\dotfill}}
		\lbl[bl]{46.3,-40,90; \makebox[4cm]{\hrulefill}}
		\lbl[bl]{46.3,-30; $\bullet$ \scriptsize $ a+p-2=-\lambda+\frac{p-q}{2}-1=-b\left( \lambda,p,q \right)$}

		\lbl[bl]{125,18; \scriptsize $ =\lambda+\frac{p-q}{2}+1= b\left( \lambda,q,p \right)$}
	\end{lpic}}
	\vspace{2cm}
\\\hline structure&
{
\begin{tabular}[]{l}
	$Y_{+,-\lambda}^{p,q}$ appears as \textbf{quotient}\\
	(note that $-\lambda\in -B(p,q)$)\\
	$Y_{-,\lambda}^{p,q}$ appears as \textbf{submodule}\\
	(note that $\lambda\in B(q,p)$)\\
%%	note that $\lambda\in B(p,q)=\frac{p-q}{2}+\Z=\frac{1}{2}+\Z$
\end{tabular}
}
&{
	\begin{tabular}[]{l}
$Y_{+,\lambda}^{p,q}$ appears as \textbf{submodule}\\
(note that $\lambda\in B(p,q)$)\\
%%note that $\lambda\in B(p,q)\cap B(q,p)$\\
%%$B(q,p)=$
	\end{tabular}
}
\end{tabular}
\end{document}
