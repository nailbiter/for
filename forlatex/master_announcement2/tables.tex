%%%%%%%%%%%%%%%%%%%%%%%%%%%%%%%%
\renewcommand{\mystack}[2]{\begin{array}{c}#1,\\#2\end{array}}
\newcommand{\mytable}[9]{
$\begin{array}{|@{}c@{}|@{}c@{}|@{}c@{}|}
  \hline
	#1& #2&#3\\
  \hline
	#4& #5&#6\\
  \hline
	#7& #8&#9\\
  \hline
\end{array} \newline$
}
\newcommand{\mytableFourTwo}[8]{
$\begin{array}{|@{}c@{}|@{}c@{}|}
  \hline
	#1& #2\\
  \hline
	#3& #4\\
  \hline
	#5& #6\\
  \hline
	#7& #8\\
  \hline
\end{array} \newline$
}
\newcommand{\mytableThreeTwo}[6]{
$\begin{array}{|@{}c@{}|@{}c@{}|}
  \hline
	#1& #2\\
  \hline
	#3& #4\\
  \hline
	#5& #6\\
  \hline
\end{array} \newline$
}
%%%%%%%%%%%%%%%%%%%%%%%%%%%%%%%%%%%%
\begin{enumerate}[(1)]
	\item $p=1,q\in2\Z$\;
		\myfootnote{the requirement $y\ge q/2$ in this and the next item is not necessary.
			Note that without it, $\pi_{-,y}^{1,q+1}$ will span the whole $J(\nu)$. I remember that I made computations for this case,
		}\\
\hspace*{0cm}\mytableThreeTwo	%#1
{\mystack{p=1}{q\in2\Z}}	{\pimyStack[\mid y\geq q/2]}
{\pipxStack}			{0}
{\pimxStack}			{h\left( \frac{y-x-{1}/{2}}{2} \right)}
	\item $p=1,q\in2\Z+1$\\
\end{enumerate}
\hspace*{0cm}\mytableThreeTwo	%#2
{\mystack{p=1}{q\in\tzo}}	{\pimyStack[\mid y\geq q/2]}
{\pipxStack}			{0}
{\pimxStack}			{h\left( \frac{y-x-{1}/{2}}{2} \right)}
\hspace*{-0cm}\mytable	%3
{p,q\d{\in}\tz}	{\pipyStack}				{\pimyStack}
{\pipxStack}	{h\left(\frac{x-y-1/2}{2}\right)} 	{0}
{\pipxStack}	{0} 					{h\left( \frac{y-x}{2}-\frac{1}{4} \right)}
\mytable	%4
{\mystack{p\in2\Z}{q\in2\Z+1}}{\pipyStack}{\pimyStack}
{\pipxStack} {0}		{h\left( \frac{-1/2-x-y}{2} \right)}
{\pimxStack} {0} {h\left( \frac{y-x-1/2}{2} \right)}
\mytable	%5
{\mystack{p\in2\Z+1}{q\in2\Z}}	{\pipyStack}		{\pimyStack}
{\pipxStack}			{0} 			{h\left( \frac{-1/2-x-y}{2} \right)}	
{\pimxStack} 			{0} 			{h\left( \frac{y-x-1/2}{2} \right)}
\mytable	%6
{p,q\d{\in}2\Z\d{+}1}	{\pipyStack}	{\pimyStack}
{\mystack{\pipx}{x\d{\in} \Azeven(p\d{+}1,q\d{+}1)}}		{h\left( \frac{x-y-1/2}{2} \right)}			{0}
{\mystack{\pimx}{x\d{\in} \Azeven(q\d{+}1,p\d{+}1)}}		{0}	{h\left( \frac{y-x-1/2}{2} \right)}	
