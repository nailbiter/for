%platex
\documentclass[reqno,12pt]{pja00} % use larger type; default would be 10pt

\usepackage{mathrsfs}
\usepackage{hyperref}
\usepackage[normalem]{ulem}
\usepackage{enumerate}
\usepackage{geometry}
\usepackage{setspace}
\usepackage{graphicx}
\usepackage{amsmath,amssymb,xypic}
\usepackage[all,cmtip]{xy}
\usepackage{amsmath,amssymb,float,mystyle}
\usepackage[normalem]{ulem}
\usepackage{caption}
\usepackage{setspace}
\usepackage{multirow}
\usepackage[table]{xcolor}
\usepackage{minibox}
\usepackage{subcaption}
\captionsetup{compatibility=false}
\usepackage{float}
\usepackage{tikz}
\usepackage{ulem}
\usetikzlibrary{patterns}

\catcode`\<=\active \def<{
\fontencoding{T1}\selectfont\symbol{60}\fontencoding{\encodingdefault}}
\catcode`\>=\active \def>{
\fontencoding{T1}\selectfont\symbol{62}\fontencoding{\encodingdefault}}
\newcommand{\assign}{:=}
\newcommand{\comma}{{,}}
\newcommand{\nin}{\not\in}
\newcommand{\tmop}[1]{\ensuremath{\operatorname{#1}}}
\newcommand{\tmtextit}[1]{{\itshape{#1}}}
\newcommand{\um}{-}
\newcommand{\mystack}[2]{$\begin{array}{c}#1\\#2\end{array}$}
\renewcommand{\iff}{\Leftrightarrow}
\newcommand{\gk}{\left( \mathfrak{g},K \right)}

\newcommand{\yipx}{Y^{p+1, q + 1}_{+, x}}
\newcommand{\yipy}{Y^{p, q + 1}_{+, y}}
\newcommand{\yimx}{Y^{p+1, q + 1}_{-, x}}
\newcommand{\yimy}{Y^{p, q + 1}_{-, y}}

\newcommand{\pipx}{\pi^{p+1, q + 1}_{+, x}}
\newcommand{\pipy}{\pi^{p, q + 1}_{+, y}}
\newcommand{\pimx}{\pi^{p+1, q + 1}_{-, x}}
\newcommand{\pimy}{\pi^{p, q + 1}_{-, y}}

%\newcommand{\pipxStack}[1][]{\mystack{\pipx}{x\d{\in} \Azeven(p\d{+}1,q\d{+}1)#1}}
%\newcommand{\pimxStack}[1][]{\mystack{\pimx}{x\d{\in} \Azeven(q\d{+}1,p\d{+}1)#1}}
%\newcommand{\pipyStack}[1][]{\mystack{\pipy}{y\d{\in} \Azeven(p,q\d{+}1)#1}}
%\newcommand{\pimyStack}[1][]{\mystack{\pimy}{y\d{\in} \Azeven(q\d{+}1,p)#1}}
\newcommand{\pipxStack}[1][]{\mystack{\pipx}{x\d{\in} \Azeven(p\d{+}1,q\d{+}1)#1}}
\newcommand{\pimxStack}[1][]{\mystack{\pimx}{x\d{\in} \Azeven(q\d{+}1,p\d{+}1)#1}}
\newcommand{\pipyStack}[1][]{\mystack{\pipy}{y\d{\in} \Azeven(p,q\d{+}1)#1}}
\newcommand{\pimyStack}[1][]{\mystack{\pimy}{y\d{\in} \Azeven(q\d{+}1,p)#1}}

%\newcommand{\yipxStack}[1][]{\mystack{\yipx}{x\d{\in} \Bzeven(p\d{+}1,q\d{+}1)#1}}
%\newcommand{\yimxStack}[1][]{\mystack{\yimx}{x\d{\in} \Bzeven(q\d{+}1,p\d{+}1)#1}}
%\newcommand{\yipyStack}[1][]{\mystack{\yipy}{y\d{\in} \Bzeven(p,q\d{+}1)#1}}
%\newcommand{\yimyStack}[1][]{\mystack{\yimy}{y\d{\in} \Bzeven(q\d{+}1,p)#1}}
\newcommand{\yipxStack}[1][]{\mystack{\yipx}{x\d{\in} \Bzeven(p\d{+}1,q\d{+}1)#1}}
\newcommand{\yimxStack}[1][]{\mystack{\yimx}{x\d{\in} \Bzeven(q\d{+}1,p\d{+}1)#1}}
\newcommand{\yipyStack}[1][]{\mystack{\yipy}{y\d{\in} \Bzeven(p,q\d{+}1)#1}}
\newcommand{\yimyStack}[1][]{\mystack{\yimy}{y\d{\in} \Bzeven(q\d{+}1,p)#1}}

\newcommand{\tzo}{2\Z+1}
\newcommand{\tz}{2\Z}
\newcommand{\tno}{2\N+1}

\renewcommand{\d}[1]{\kern-0.1cm #1\kern-0.1cm}

\newcommand{\ppqpl}{\pi_{+,\lambda}^{p,q}}
\newtheorem{theorem}{Theorem}[section]
\newcommand{\sol}{\mathcal{S}\!{\it ol}(\R^{p,q};\lambda,\nu)}
\newcommand{\Hom}{\mbox{\normalfont Hom}}
\newcommand{\Sol}{\mathcal{S}\!{\it ol}}
\newcommand{\Ind}{\mbox{\normalfont Ind}}
\newcommand{\Supp}{\mathcal{S}\!{\it upp}}
\newtheorem{remark}[theorem]{Remark}
\newtheorem{fact}[theorem]{Fact}
\newtheorem{proposition}[theorem]{Proposition}
\theoremstyle{definition}
\newtheorem{definition}[theorem]{Definition}

\theoremstyle{exampstyle} \newtheorem{examp}[theorem]{Theorem}

\catcode`\<=\active \def<{
\fontencoding{T1}\selectfont\symbol{60}\fontencoding{\encodingdefault}}
\catcode`\>=\active \def>{
\fontencoding{T1}\selectfont\symbol{62}\fontencoding{\encodingdefault}}
\newcommand{\dueto}[1]{\textup{\textbf{(#1) }}}
\newcommand{\tmrsub}[1]{\ensuremath{_{\textrm{#1}}}}
\newcommand{\tmrsup}[1]{\textsuperscript{#1}}
\newcommand{\tmtextbf}[1]{{\bfseries{#1}}}
\newcommand{\Op}{\mbox{\normalfont Op}}
\newcommand{\Res}{\operatorname{Res}\displaylimits}
\newcommand{\OpR}{\mbox{\it R}}
\renewcommand{\Q}{Q_{p,q}}
\newcommand{\IlambdaGprime}{I(\lambda)\kern-0.3em\mid_{G'}}
\newcommand{\SBO}{\Hom_{G'}\left(\IlambdaGprime,J(\nu) \right)}

\let\oldsetminus\setminus
\let\setminus-

\setlength{\parskip}{0.4em}
\setlength{\parindent}{2em}

\newcommand{\even}{2\Z}
\newcommand{\odd}{2\Z+1}
\newcommand{\teven}{\mbox{\textrm{: even}}}
\newcommand{\todd}{\mbox{\textrm{: odd}}}
\newcommand{\tevenText}[1]{\vspace{-3cm}$\begin{array}{l}\nu\teven\\\nu#1\end{array}$}
\newcommand{\toddText}[1]{\vspace{-3cm}$\begin{array}{l}\nu\todd\\\nu#1\end{array}$}
\newcommand{\mm}{\mid\mid}
\newcommand{\bb}{\backslash\backslash}
\newcommand{\Azeven}{A_0^{\rm\footnotesize even}}
\newcommand{\Bzeven}{B^{\rm\footnotesize even}}
\renewcommand{\ss}{//}

%custom footnote mechanism
\newif\ifGrammarNoteFtnt
\GrammarNoteFtntfalse %\GrammarNoteFtntfalse
\newcommand{\myGrammarNoteFootnote}[1]{
\ifGrammarNoteFtnt
\stepcounter{myfootnotecounter}$^\themyfootnotecounter$\footnotetext{\themyfootnotecounter: #1}
\else
\fi
}
\newcounter{myfootnotecounter}
\newcommand{\myfootnote}[1]{\stepcounter{myfootnotecounter}$^\themyfootnotecounter$\footnotetext{\themyfootnotecounter: #1}}

\begin{document}

\title{Symmetry breaking operators for the restriction of representations of indefinite orthogonal groups $O(p,q)$ (part 2)}
\Author{1}{Toshiyuki}{Kobayashi}
\Author{2}{Alex}{Leontiev}
\affiliation{1}{Kavli IPMU and Graduate School of Mathematical Sciences, The University of Tokyo}
\affiliation{2}{Graduate School of Mathematical Sciences, The University of Tokyo}
\KeyWords{ {Representation theory}{reductive group}{branching law}{broken symmetry}{conformal geometry}{symmetry breaking operator}{Zuckerman modules}}
\Subject[2010]{22E46; 33C45, 53C35}

  \maketitle
\begin{abstract}
This is the second part of results on the symmetry breaking operators for the pair $(G, G') =(O(p+1, q+1), O(p,q+1))$.
In the first announcement we have outlined the construction and classification of SBOs together with some finer results
(e.g. residue formul\ae, functional equations). In this second announcement we outline the results on complete determination of
images of SBOs for all parameters. Together with the (partial) knowledge about their kernels, this allowed us to give the complete classification
of the $G'$-intertwining maps between the certain families of Zuckerman derived modules that occur as submodules/quotients of degenerate principal series.
%The results contribute to Program C of branching problems suggested by the first author [Progr. Math. 2015].
\end{abstract}
%	Since the representation $J(\nu)$ of $G'=O(p,q+1)$ is multiplicity-free as a $K'$-module, we can describe its $(\mathfrak{g}',K')$-submodules by means of subsets of $\N_{+}^2$
%	for $p>1$, which parametrize the $K'$-structure of $J(\nu)$ by the spherical harmonics $\mathcal{H}^a(\Sp^{p-1})\boxtimes\mathcal{H}^b(\Sp^q)$.
%	As in \cite{howe1993homogeneous}, we also indicate the Jordan--H\"older series (socle filtrations) of $J(\nu)$ by using arrows.
%	We then have:
%\begin{theorem}[images of SBOs]
%	The regular SBO $R_{\lambda,\nu}^X:I(\lambda)\to J(\nu)$ is surjective,
%	unless $\nu\in\Z$. In the latter case, the images of the underlying $(\mathfrak{g},K)$-module $I(\lambda)_K$ under
%	$R_{\lambda,\nu}^X$ are given as follows (here we set $l:=\frac{1}{2}\left( \nu-\lambda \right)\in\N$ for $(\lambda,\nu)\in//$ and $k:=\frac{1}{2}\left( n-1-\lambda-\nu \right)
%	\in\N$ for
%	$(\lambda,\nu)\in\backslash\backslash$; the barriers $A^{\pm\pm}$ are defined as in \cite{howe1993homogeneous}): \\
%	for $p>1$:
%\end{theorem}
%\begin{enumerate}[(1)]
%	\item Suppose $p\in2\N_++1$ and $q\in2\Z$. Then, if $\nu\in2\Z,0<\nu<n-1$, $R_{\lambda,\nu}^X$ is surjective. Otherwise,\newpage
%		\hspace*{-1cm}\begin{figure}[h]
%			\noindent\begin{tabular}{m{1.6cm}rrr}
%	      $(\lambda,\nu)\in$&$\mybra{//\cup\backslash\backslash}^c$ & $\backslash\backslash-//$  & $//\cap\backslash\backslash,k> l$\\[0pt]
%	      {\vspace{-3cm} $ \begin{array}{l}
%	      \nu\teven\\ \nu\le0
%      \end{array}$}&{\begin{tikzpicture}[scale=0.5]
\draw [<->,thick] (0.0,3.0) node (yaxis) [above] {}
	|- (3.0,0.0) node (xaxis) [below] {};
\draw (0,1.5)  -- (1.5,0) ;
\draw [->] (0.375,1.125) -- (0.125,0.875);
\draw [->] (0.75,0.75) -- (0.5,0.5);
\draw [->] (1.125,0.375) -- (0.875,0.125);
\node at (1.4,1.4) {\tiny $A^{++}$};
\node at (0.3999999999999999,-0.3) {\tiny $-\nu$};
\node at (-0.6,1.2) {\tiny $-\nu$};
\draw [fill=gray,opacity=0.2,gray] (1.5,0.0) -- (0.0,1.5) -- (0.0,0.0) ;
\end{tikzpicture}}
&\hspace{-0.5cm}{\begin{tikzpicture}[scale=0.5]
\draw [<->,thick] (0.0,3.0) node (yaxis) [above] {}
	|- (3.0,0.0) node (xaxis) [below] {};
\draw (0,1.5)  -- (1.5,0) ;
\draw [->] (0.375,1.125) -- (0.125,0.875);
\draw [->] (0.75,0.75) -- (0.5,0.5);
\draw [->] (1.125,0.375) -- (0.875,0.125);
\node at (1.4,1.4) {\tiny $A^{++}$};
\node at (0.3999999999999999,-0.3) {\tiny $-\nu$};
\node at (-0.6,1.2) {\tiny $-\nu$};
\draw [fill=gray,opacity=0.2,gray] (1.5,0.0) -- (0.0,1.5) -- (0.0,0.0) ;
\end{tikzpicture}}
&\hspace{-0.5cm}{\begin{tikzpicture}[scale=0.5]
\draw [<->,thick] (0.0,3.0) node (yaxis) [above] {}
	|- (3.0,0.0) node (xaxis) [below] {};
\draw (0,1.5)  -- (1.5,0) ;
\draw [->] (0.375,1.125) -- (0.125,0.875);
\draw [->] (0.75,0.75) -- (0.5,0.5);
\draw [->] (1.125,0.375) -- (0.875,0.125);
\node at (1.4,1.4) {\tiny $A^{++}$};
\node at (0.3999999999999999,-0.3) {\tiny $-\nu$};
\node at (-0.6,1.2) {\tiny $-\nu$};
\draw [pattern=north east lines, pattern color=green!50!black] (3.0,0.0) -- (3.0,3.0) -- (0.0,3.0) -- (0.0,0.0) ;
\draw (0,1.5)  -- (1.5,0) ;
\draw [->] (0.375,1.125) -- (0.125,0.875);
\draw [->] (0.75,0.75) -- (0.5,0.5);
\draw [->] (1.125,0.375) -- (0.875,0.125);
\node at (1.4,1.4) {\tiny $A^{++}$};
\node at (0.3999999999999999,-0.3) {\tiny $-\nu$};
\node at (-0.6,1.2) {\tiny $-\nu$};
\draw [pattern=north west lines, pattern color=purple] (1.5,0.0) -- (0.0,1.5) -- (0.0,0.0) ;
\end{tikzpicture}}
\\[0pt]
%      \vspace{-3cm}$\begin{array}{l}
%	      \nu\todd\\ \nu\le\frac{n-3}{2}
%      \end{array}$&{\begin{tikzpicture}[scale=0.5]
\draw [<->,thick] (0.0,3.0) node (yaxis) [above] {}
	|- (3.0,0.0) node (xaxis) [below] {};
\draw (1.8,0.0) -- (3.0,1.2) ;
\draw [->] (2.1,0.3) -- (1.9100000000000001,0.49);
\draw [->] (2.4,0.6) -- (2.21,0.79);
\draw [->] (2.7,0.8999999999999999) -- (2.5100000000000002,1.0899999999999999);
\node at (3.7,1.5) {\tiny $A^{+-}$};
\node at (1.8,-0.9) {\tiny $$};
\draw [fill=gray,opacity=0.2,gray] (1.8,0.0) -- (3.0,1.2) -- (3.0,3.0) -- (1.2,3.0) -- (0.0,1.8) -- (0.0,0.0) ;
\draw (0.0,1.8) -- (1.2,3.0) ;
\draw [->] (0.3,2.1) -- (0.49,1.9100000000000001);
\draw [->] (0.6,2.4) -- (0.79,2.21);
\draw [->] (0.8999999999999999,2.7) -- (1.0899999999999999,2.5100000000000002);
\node at (1.6,3.4) {\tiny $A^{-+}$};
\node at (-1.7,1.9000000000000001) {\tiny $$};
\draw [fill=gray,opacity=0.2,gray] (0.0,1.8) -- (1.2,3.0) -- (0.0,3.0) ;
\end{tikzpicture}}
&\hspace{-0.5cm}{\begin{tikzpicture}[scale=0.5]
\draw [<->,thick] (0.0,3.0) node (yaxis) [above] {}
	|- (3.0,0.0) node (xaxis) [below] {};
\draw (1.8,0.0) -- (3.0,1.2) ;
\draw [->] (2.1,0.3) -- (1.9100000000000001,0.49);
\draw [->] (2.4,0.6) -- (2.21,0.79);
\draw [->] (2.7,0.8999999999999999) -- (2.5100000000000002,1.0899999999999999);
\node at (3.7,1.5) {\tiny $A^{+-}$};
\node at (1.8,-0.9) {\tiny $$};
\draw [fill=gray,opacity=0.2,gray] (1.8,0.0) -- (3.0,1.2) -- (3.0,3.0) -- (1.2,3.0) -- (0.0,1.8) -- (0.0,0.0) ;
\draw (0.0,1.8) -- (1.2,3.0) ;
\draw [->] (0.3,2.1) -- (0.49,1.9100000000000001);
\draw [->] (0.6,2.4) -- (0.79,2.21);
\draw [->] (0.8999999999999999,2.7) -- (1.0899999999999999,2.5100000000000002);
\node at (1.6,3.4) {\tiny $A^{-+}$};
\node at (-1.7,1.9000000000000001) {\tiny $$};
\end{tikzpicture}}
&\hspace{-0.5cm}{\begin{tikzpicture}[scale=0.5]
\draw [<->,thick] (0.0,3.0) node (yaxis) [above] {}
	|- (3.0,0.0) node (xaxis) [below] {};
\draw (1.8,0.0) -- (3.0,1.2) ;
\draw [->] (2.1,0.3) -- (1.9100000000000001,0.49);
\draw [->] (2.4,0.6) -- (2.21,0.79);
\draw [->] (2.7,0.8999999999999999) -- (2.5100000000000002,1.0899999999999999);
\node at (3.7,1.5) {\tiny $A^{+-}$};
\node at (1.8,-0.9) {\tiny $$};
\draw [pattern=north west lines, pattern color=purple] (1.8,0.0) -- (3.0,1.2) -- (3.0,3.0) -- (0.0,3.0) -- (0.0,0.0) ;
\draw (0.0,1.8) -- (1.2,3.0) ;
\draw [->] (0.3,2.1) -- (0.49,1.9100000000000001);
\draw [->] (0.6,2.4) -- (0.79,2.21);
\draw [->] (0.8999999999999999,2.7) -- (1.0899999999999999,2.5100000000000002);
\node at (1.6,3.4) {\tiny $A^{-+}$};
\node at (-1.7,1.9000000000000001) {\tiny $$};
\draw [pattern=north east lines, pattern color=green!50!black] (3.0,0.0) -- (3.0,3.0) -- (0.0,3.0) -- (0.0,0.0) ;
\end{tikzpicture}}
\\[0pt]
%	      $(\lambda,\nu)\in$&$\mybra{//\cup\backslash\backslash}^c$ && $//\cap\backslash\backslash,k=l$\\[0pt]
%	      \vspace{-3cm}$\begin{array}{l}\nu\todd\\\nu=\frac{n-1}{2}
%	      \end{array}$&{\begin{tikzpicture}[scale=0.5]
\draw [<->,thick] (0.0,3.0) node (yaxis) [above] {}
	|- (3.0,0.0) node (xaxis) [below] {};
\draw (1.8,0.0) -- (3.0,1.2) ;
\draw [->] (2.1,0.3) -- (1.9100000000000001,0.49);
\draw [->] (2.4,0.6) -- (2.21,0.79);
\draw [->] (2.7,0.8999999999999999) -- (2.5100000000000002,1.0899999999999999);
\node at (3.7,1.5) {\tiny $A^{+-}$};
\node at (1.8,-0.9) {\tiny $$};
\draw [fill=gray,opacity=0.2,gray] (1.8,0.0) -- (3.0,1.2) -- (3.0,3.0) -- (0.0,3.0) -- (0.0,0.0) ;
\draw (1.8,0.0) -- (3.0,1.2) ;
\draw [->] (2.1,0.3) -- (2.29,0.10999999999999999);
\draw [->] (2.4,0.6) -- (2.59,0.41);
\draw [->] (2.7,0.8999999999999999) -- (2.89,0.71);
\node at (3.7,1.5) {\tiny $A^{+-}$};
\node at (1.8,-0.9) {\tiny $$};
\end{tikzpicture}}
&\hspace{-0.5cm}&\hspace{-0.5cm}{\begin{tikzpicture}[scale=0.5]
\draw [<->,thick] (0.0,3.0) node (yaxis) [above] {}
	|- (3.0,0.0) node (xaxis) [below] {};
\draw (1.8,0.0) -- (3.0,1.2) ;
\draw [->] (2.1,0.3) -- (1.9100000000000001,0.49);
\draw [->] (2.4,0.6) -- (2.21,0.79);
\draw [->] (2.7,0.8999999999999999) -- (2.5100000000000002,1.0899999999999999);
\node at (3.7,1.5) {\tiny $A^{+-}$};
\node at (1.8,-0.9) {\tiny $$};
\draw [pattern=north east lines, pattern color=green!50!black] (3.0,0.0) -- (3.0,3.0) -- (0.0,3.0) -- (0.0,0.0) ;
\draw (1.8,0.0) -- (3.0,1.2) ;
\draw [->] (2.1,0.3) -- (2.29,0.10999999999999999);
\draw [->] (2.4,0.6) -- (2.59,0.41);
\draw [->] (2.7,0.8999999999999999) -- (2.89,0.71);
\node at (3.7,1.5) {\tiny $A^{+-}$};
\node at (1.8,-0.9) {\tiny $$};
\draw [pattern=north west lines, pattern color=purple] (1.8,0.0) -- (3.0,1.2) -- (3.0,3.0) -- (0.0,3.0) -- (0.0,0.0) ;
\end{tikzpicture}}
\\[0pt]
%	      $(\lambda,\nu)\in$&$\mybra{//\cup\backslash\backslash}^c$ & $//-\backslash\backslash$  & $//\cap\backslash\backslash,k< l$\\[0pt]
%	      \vspace{-3cm}$\begin{array}{l}\nu\teven\\\nu\ge{n-1}\end{array}$&{\begin{tikzpicture}[scale=0.5]
\draw [<->,thick] (0.0,3.0) node (yaxis) [above] {}
	|- (3.0,0.0) node (xaxis) [below] {};
\draw (0,1.5)  -- (1.5,0) ;
\draw [->] (0.375,1.125) -- (0.625,1.375);
\draw [->] (0.75,0.75) -- (1.0,1.0);
\draw [->] (1.125,0.375) -- (1.375,0.625);
\node at (-1.4,0.6) {\tiny $\nu-n+1$};
\node at (-0.30000000000000004,-0.3) {\tiny $\nu-n+1$};
\node at (1.4,1.4) {\tiny $A^{--}$};
\draw [fill=gray,opacity=0.2,gray] (3.0,0.0) -- (3.0,3.0) -- (0.0,3.0) -- (0.0,0.0) ;
\end{tikzpicture}}
&\hspace{-0.5cm}{\begin{tikzpicture}[scale=0.5]
\draw [<->,thick] (0.0,3.0) node (yaxis) [above] {}
	|- (3.0,0.0) node (xaxis) [below] {};
\draw (0,1.5)  -- (1.5,0) ;
\draw [->] (0.375,1.125) -- (0.625,1.375);
\draw [->] (0.75,0.75) -- (1.0,1.0);
\draw [->] (1.125,0.375) -- (1.375,0.625);
\node at (-1.4,0.6) {\tiny $\nu-n+1$};
\node at (-0.30000000000000004,-0.3) {\tiny $\nu-n+1$};
\node at (1.4,1.4) {\tiny $A^{--}$};
\draw [fill=gray,opacity=0.2,gray] (3.0,0.0) -- (3.0,3.0) -- (0.0,3.0) -- (0.0,0.0) ;
\end{tikzpicture}}
&\hspace{-0.5cm}{\begin{tikzpicture}[scale=0.5]
\draw [<->,thick] (0.0,3.0) node (yaxis) [above] {}
	|- (3.0,0.0) node (xaxis) [below] {};
\draw (0,1.5)  -- (1.5,0) ;
\draw [->] (0.375,1.125) -- (0.625,1.375);
\draw [->] (0.75,0.75) -- (1.0,1.0);
\draw [->] (1.125,0.375) -- (1.375,0.625);
\node at (-1.4,0.6) {\tiny $\nu-n+1$};
\node at (-0.30000000000000004,-0.3) {\tiny $\nu-n+1$};
\node at (1.4,1.4) {\tiny $A^{--}$};
\draw [fill=gray,opacity=0.2,gray] (0.0,1.5) -- (1.5,0.0) -- (3.0,0.0) -- (3.0,3.0) -- (0.0,3.0) ;
\end{tikzpicture}}
\\[0pt]
%	    \end{tabular}
%	  \end{figure}
%		\begin{figure}[h]
%			\noindent\begin{tabular}{m{1.3cm}rrr}
%	      $(\lambda,\nu)\in$&$\mybra{//\cup\backslash\backslash}^c$ & $//-\backslash\backslash$  & $//\cap\backslash\backslash,k< l$\\[0pt]
%	      \vspace{-3cm}$\begin{array}{l}\nu\todd\\\nu\ge\frac{n+1}{2}\end{array}$&{\begin{tikzpicture}[scale=0.5]
\draw [<->,thick] (0.0,3.0) node (yaxis) [above] {}
	|- (3.0,0.0) node (xaxis) [below] {};
\draw (1.8,0.0) -- (3.0,1.2) ;
\draw [->] (2.1,0.3) -- (2.29,0.10999999999999999);
\draw [->] (2.4,0.6) -- (2.59,0.41);
\draw [->] (2.7,0.8999999999999999) -- (2.89,0.71);
\node at (3.7,1.5) {\tiny $A^{-+}$};
\node at (2.8,-0.7) {\tiny $\nu-p+2$};
\draw (0.0,1.8) -- (1.2,3.0) ;
\draw [->] (0.3,2.1) -- (0.10999999999999999,2.29);
\draw [->] (0.6,2.4) -- (0.41,2.59);
\draw [->] (0.8999999999999999,2.7) -- (0.71,2.89);
\node at (1.6,3.4) {\tiny $A^{+-}$};
\node at (-1.4,2.7) {\tiny $\nu-q+1$};
\draw [fill=gray,opacity=0.2,gray] (0.0,1.8) -- (1.2,3.0) -- (0.0,3.0) ;
\end{tikzpicture}}
&\hspace{-0.5cm}{\begin{tikzpicture}[scale=0.5]
\draw [<->,thick] (0.0,3.0) node (yaxis) [above] {}
	|- (3.0,0.0) node (xaxis) [below] {};
\draw (1.8,0.0) -- (3.0,1.2) ;
\draw [->] (2.1,0.3) -- (2.29,0.10999999999999999);
\draw [->] (2.4,0.6) -- (2.59,0.41);
\draw [->] (2.7,0.8999999999999999) -- (2.89,0.71);
\node at (3.7,1.5) {\tiny $A^{-+}$};
\node at (2.8,-0.7) {\tiny $\nu-p+2$};
\draw [pattern=north east lines, pattern color=green!50!black] (3.0,0.0) -- (3.0,3.0) -- (0.0,3.0) -- (0.0,0.0) ;
\draw (0.0,1.8) -- (1.2,3.0) ;
\draw [->] (0.3,2.1) -- (0.10999999999999999,2.29);
\draw [->] (0.6,2.4) -- (0.41,2.59);
\draw [->] (0.8999999999999999,2.7) -- (0.71,2.89);
\node at (1.6,3.4) {\tiny $A^{+-}$};
\node at (-1.4,2.7) {\tiny $\nu-q+1$};
\draw [pattern=north west lines, pattern color=purple] (0.0,1.8) -- (1.2,3.0) -- (0.0,3.0) ;
\end{tikzpicture}}
&\hspace{-0.5cm}{\begin{tikzpicture}[scale=0.5]
\draw [<->,thick] (0.0,3.0) node (yaxis) [above] {}
	|- (3.0,0.0) node (xaxis) [below] {};
\draw (1.8,0.0) -- (3.0,1.2) ;
\draw [->] (2.1,0.3) -- (2.29,0.10999999999999999);
\draw [->] (2.4,0.6) -- (2.59,0.41);
\draw [->] (2.7,0.8999999999999999) -- (2.89,0.71);
\node at (3.7,1.5) {\tiny $A^{-+}$};
\node at (2.8,-0.7) {\tiny $\nu-p+2$};
\draw [pattern=north east lines, pattern color=green!50!black] (1.8,0.0) -- (3.0,1.2) -- (3.0,0.0) ;
\draw (0.0,1.8) -- (1.2,3.0) ;
\draw [->] (0.3,2.1) -- (0.10999999999999999,2.29);
\draw [->] (0.6,2.4) -- (0.41,2.59);
\draw [->] (0.8999999999999999,2.7) -- (0.71,2.89);
\node at (1.6,3.4) {\tiny $A^{+-}$};
\node at (-1.4,2.7) {\tiny $\nu-q+1$};
\draw [pattern=north east lines, pattern color=green!50!black] (0.0,1.8) -- (1.2,3.0) -- (0.0,3.0) ;
\draw (0.0,1.8) -- (1.2,3.0) ;
\draw [->] (0.3,2.1) -- (0.10999999999999999,2.29);
\draw [->] (0.6,2.4) -- (0.41,2.59);
\draw [->] (0.8999999999999999,2.7) -- (0.71,2.89);
\node at (1.6,3.4) {\tiny $A^{+-}$};
\node at (-1.4,2.7) {\tiny $\nu-q+1$};
\draw [pattern=north west lines, pattern color=purple] (0.0,1.8) -- (1.2,3.0) -- (0.0,3.0) ;
\end{tikzpicture}}
\\[25pt]
%	    \end{tabular}
%	  \end{figure}
%	\item Suppose $p,q\in\odd$ and $p>1$. Then,\clearpage
%		\begin{figure}[h]
%			\noindent\begin{tabular}{m{1.3cm}rrr}
%			$(\lambda,\nu)\in$&$\mybra{\ss\cup\bb}^c$ & $\bb-\ss$  & $\ss-\bb$\\[0pt]
%			\tevenText{\le0}&{\begin{tikzpicture}[scale=0.5]
\draw [<->,thick] (0.0,3.0) node (yaxis) [above] {}
	|- (3.0,0.0) node (xaxis) [below] {};
\draw (0,1.5)  -- (1.5,0) ;
\draw [->] (0.375,1.125) -- (0.125,0.875);
\draw [->] (0.75,0.75) -- (0.5,0.5);
\draw [->] (1.125,0.375) -- (0.875,0.125);
\node at (1.4,1.4) {\tiny $A^{++}$};
\node at (0.3999999999999999,-0.3) {\tiny $-\nu$};
\node at (-0.6,1.2) {\tiny $-\nu$};
\draw [fill=gray,opacity=0.2,gray] (1.5,0.0) -- (0.0,1.5) -- (0.0,0.0) ;
\draw (1.8,0.0) -- (3.0,1.2) ;
\draw [->] (2.1,0.3) -- (1.9100000000000001,0.49);
\draw [->] (2.4,0.6) -- (2.21,0.79);
\draw [->] (2.7,0.8999999999999999) -- (2.5100000000000002,1.0899999999999999);
\node at (3.7,1.5) {\tiny $A^{+-}$};
\node at (1.8,-0.9) {\tiny $$};
\end{tikzpicture}}
&\hspace{-0.5cm}{\begin{tikzpicture}[scale=0.5]
\draw [<->,thick] (0.0,3.0) node (yaxis) [above] {}
	|- (3.0,0.0) node (xaxis) [below] {};
\draw (0,1.5)  -- (1.5,0) ;
\draw [->] (0.375,1.125) -- (0.125,0.875);
\draw [->] (0.75,0.75) -- (0.5,0.5);
\draw [->] (1.125,0.375) -- (0.875,0.125);
\node at (1.4,1.4) {\tiny $A^{++}$};
\node at (0.3999999999999999,-0.3) {\tiny $-\nu$};
\node at (-0.6,1.2) {\tiny $-\nu$};
\draw [fill=gray,opacity=0.2,gray] (1.5,0.0) -- (0.0,1.5) -- (0.0,0.0) ;
\draw (1.8,0.0) -- (3.0,1.2) ;
\draw [->] (2.1,0.3) -- (1.9100000000000001,0.49);
\draw [->] (2.4,0.6) -- (2.21,0.79);
\draw [->] (2.7,0.8999999999999999) -- (2.5100000000000002,1.0899999999999999);
\node at (3.7,1.5) {\tiny $A^{+-}$};
\node at (1.8,-0.9) {\tiny $$};
\end{tikzpicture}}
&\hspace{-0.5cm}{\begin{tikzpicture}[scale=0.5]
\draw [<->,thick] (0.0,3.0) node (yaxis) [above] {}
	|- (3.0,0.0) node (xaxis) [below] {};
\draw (0,1.5)  -- (1.5,0) ;
\draw [->] (0.375,1.125) -- (0.125,0.875);
\draw [->] (0.75,0.75) -- (0.5,0.5);
\draw [->] (1.125,0.375) -- (0.875,0.125);
\node at (1.4,1.4) {\tiny $A^{++}$};
\node at (0.3999999999999999,-0.3) {\tiny $-\nu$};
\node at (-0.6,1.2) {\tiny $-\nu$};
\draw [pattern=north west lines, pattern color=purple] (1.5,0.0) -- (0.0,1.5) -- (0.0,0.0) ;
\draw (1.8,0.0) -- (3.0,1.2) ;
\draw [->] (2.1,0.3) -- (1.9100000000000001,0.49);
\draw [->] (2.4,0.6) -- (2.21,0.79);
\draw [->] (2.7,0.8999999999999999) -- (2.5100000000000002,1.0899999999999999);
\node at (3.7,1.5) {\tiny $A^{+-}$};
\node at (1.8,-0.9) {\tiny $$};
\draw [pattern=north east lines, pattern color=green!50!black] (3.0,0.0) -- (3.0,3.0) -- (0.0,3.0) -- (0.0,0.0) ;
\end{tikzpicture}}
\\[0pt]
%			\toddText{\le n-3}&{\begin{tikzpicture}[scale=0.5]
\draw [<->,thick] (0.0,3.0) node (yaxis) [above] {}
	|- (3.0,0.0) node (xaxis) [below] {};
\draw (0.0,1.8) -- (1.2,3.0) ;
\draw [->] (0.3,2.1) -- (0.49,1.9100000000000001);
\draw [->] (0.6,2.4) -- (0.79,2.21);
\draw [->] (0.8999999999999999,2.7) -- (1.0899999999999999,2.5100000000000002);
\node at (1.6,3.4) {\tiny $A^{-+}$};
\node at (-1.7,1.9000000000000001) {\tiny $$};
\draw [fill=gray,opacity=0.2,gray] (3.0,0.0) -- (3.0,3.0) -- (0.0,3.0) -- (0.0,0.0) ;
\end{tikzpicture}}
&\hspace{-0.5cm}{\begin{tikzpicture}[scale=0.5]
\draw [<->,thick] (0.0,3.0) node (yaxis) [above] {}
	|- (3.0,0.0) node (xaxis) [below] {};
\draw (0.0,1.8) -- (1.2,3.0) ;
\draw [->] (0.3,2.1) -- (0.49,1.9100000000000001);
\draw [->] (0.6,2.4) -- (0.79,2.21);
\draw [->] (0.8999999999999999,2.7) -- (1.0899999999999999,2.5100000000000002);
\node at (1.6,3.4) {\tiny $A^{-+}$};
\node at (-1.7,1.9000000000000001) {\tiny $$};
\draw [fill=gray,opacity=0.2,gray] (0.0,1.8) -- (1.2,3.0) -- (3.0,3.0) -- (3.0,0.0) -- (0.0,0.0) ;
\end{tikzpicture}}
&\hspace{-0.5cm}{\begin{tikzpicture}[scale=0.5]
\draw [<->,thick] (0.0,3.0) node (yaxis) [above] {}
	|- (3.0,0.0) node (xaxis) [below] {};
\draw (0.0,1.8) -- (1.2,3.0) ;
\draw [->] (0.3,2.1) -- (0.49,1.9100000000000001);
\draw [->] (0.6,2.4) -- (0.79,2.21);
\draw [->] (0.8999999999999999,2.7) -- (1.0899999999999999,2.5100000000000002);
\node at (1.6,3.4) {\tiny $A^{-+}$};
\node at (-1.7,1.9000000000000001) {\tiny $$};
\draw [fill=gray,opacity=0.2,gray] (3.0,0.0) -- (3.0,3.0) -- (0.0,3.0) -- (0.0,0.0) ;
\end{tikzpicture}}
\\[0pt]
%			\tevenText{>0}&{\begin{tikzpicture}[scale=0.5]
\draw [<->,thick] (0.0,3.0) node (yaxis) [above] {}
	|- (3.0,0.0) node (xaxis) [below] {};
\draw (1.8,0.0) -- (3.0,1.2) ;
\draw [->] (2.1,0.3) -- (1.9100000000000001,0.49);
\draw [->] (2.4,0.6) -- (2.21,0.79);
\draw [->] (2.7,0.8999999999999999) -- (2.5100000000000002,1.0899999999999999);
\node at (3.7,1.5) {\tiny $A^{+-}$};
\node at (1.8,-0.9) {\tiny $$};
\draw [fill=gray,opacity=0.2,gray] (1.8,0.0) -- (3.0,1.2) -- (3.0,3.0) -- (0.0,3.0) -- (0.0,0.0) ;
\end{tikzpicture}}
&\hspace{-0.5cm}{\begin{tikzpicture}[scale=0.5]
\draw [<->,thick] (0.0,3.0) node (yaxis) [above] {}
	|- (3.0,0.0) node (xaxis) [below] {};
\draw (1.8,0.0) -- (3.0,1.2) ;
\draw [->] (2.1,0.3) -- (1.9100000000000001,0.49);
\draw [->] (2.4,0.6) -- (2.21,0.79);
\draw [->] (2.7,0.8999999999999999) -- (2.5100000000000002,1.0899999999999999);
\node at (3.7,1.5) {\tiny $A^{+-}$};
\node at (1.8,-0.9) {\tiny $$};
\draw [fill=gray,opacity=0.2,gray] (1.8,0.0) -- (3.0,1.2) -- (3.0,3.0) -- (0.0,3.0) -- (0.0,0.0) ;
\end{tikzpicture}}
&\hspace{-0.5cm}{\begin{tikzpicture}[scale=0.5]
\draw [<->,thick] (0.0,3.0) node (yaxis) [above] {}
	|- (3.0,0.0) node (xaxis) [below] {};
\draw (1.8,0.0) -- (3.0,1.2) ;
\draw [->] (2.1,0.3) -- (1.9100000000000001,0.49);
\draw [->] (2.4,0.6) -- (2.21,0.79);
\draw [->] (2.7,0.8999999999999999) -- (2.5100000000000002,1.0899999999999999);
\node at (3.7,1.5) {\tiny $A^{+-}$};
\node at (1.8,-0.9) {\tiny $$};
\draw [pattern=north west lines, pattern color=purple] (1.8,0.0) -- (3.0,1.2) -- (3.0,3.0) -- (0.0,3.0) -- (0.0,0.0) ;
\draw (1.8,0.0) -- (3.0,1.2) ;
\draw [->] (2.1,0.3) -- (1.9100000000000001,0.49);
\draw [->] (2.4,0.6) -- (2.21,0.79);
\draw [->] (2.7,0.8999999999999999) -- (2.5100000000000002,1.0899999999999999);
\node at (3.7,1.5) {\tiny $A^{+-}$};
\node at (1.8,-0.9) {\tiny $$};
\draw [pattern=north east lines, pattern color=green!50!black] (3.0,0.0) -- (3.0,3.0) -- (0.0,3.0) -- (0.0,0.0) ;
\end{tikzpicture}}
\\[0pt]
%			\toddText{>n-3}&{\begin{tikzpicture}[scale=0.5]
\draw [<->,thick] (0.0,3.0) node (yaxis) [above] {}
	|- (3.0,0.0) node (xaxis) [below] {};
\draw (0,1.5)  -- (1.5,0) ;
\draw [->] (0.375,1.125) -- (0.625,1.375);
\draw [->] (0.75,0.75) -- (1.0,1.0);
\draw [->] (1.125,0.375) -- (1.375,0.625);
\node at (-1.4,0.6) {\tiny $\nu-n+1$};
\node at (-0.30000000000000004,-0.3) {\tiny $\nu-n+1$};
\node at (1.4,1.4) {\tiny $A^{--}$};
\draw [fill=gray,opacity=0.2,gray] (3.0,0.0) -- (3.0,3.0) -- (0.0,3.0) -- (0.0,0.0) ;
\draw (1.8,0.0) -- (3.0,1.2) ;
\draw [->] (2.1,0.3) -- (2.29,0.10999999999999999);
\draw [->] (2.4,0.6) -- (2.59,0.41);
\draw [->] (2.7,0.8999999999999999) -- (2.89,0.71);
\node at (3.7,1.5) {\tiny $A^{-+}$};
\node at (2.8,-0.7) {\tiny $\nu-p+2$};
\end{tikzpicture}}
&\hspace{-0.5cm}{\begin{tikzpicture}[scale=0.5]
\draw [<->,thick] (0.0,3.0) node (yaxis) [above] {}
	|- (3.0,0.0) node (xaxis) [below] {};
\draw (0,1.5)  -- (1.5,0) ;
\draw [->] (0.375,1.125) -- (0.625,1.375);
\draw [->] (0.75,0.75) -- (1.0,1.0);
\draw [->] (1.125,0.375) -- (1.375,0.625);
\node at (-1.4,0.6) {\tiny $\nu-n+1$};
\node at (-0.30000000000000004,-0.3) {\tiny $\nu-n+1$};
\node at (1.4,1.4) {\tiny $A^{--}$};
\draw (1.8,0.0) -- (3.0,1.2) ;
\draw [->] (2.1,0.3) -- (2.29,0.10999999999999999);
\draw [->] (2.4,0.6) -- (2.59,0.41);
\draw [->] (2.7,0.8999999999999999) -- (2.89,0.71);
\node at (3.7,1.5) {\tiny $A^{-+}$};
\node at (2.8,-0.7) {\tiny $\nu-p+2$};
\draw [fill=gray,opacity=0.2,gray] (1.8,0.0) -- (3.0,1.2) -- (3.0,0.0) ;
\end{tikzpicture}}
&\hspace{-0.5cm}{\begin{tikzpicture}[scale=0.5]
\draw [<->,thick] (0.0,3.0) node (yaxis) [above] {}
	|- (3.0,0.0) node (xaxis) [below] {};
\draw (0,1.5)  -- (1.5,0) ;
\draw [->] (0.375,1.125) -- (0.625,1.375);
\draw [->] (0.75,0.75) -- (1.0,1.0);
\draw [->] (1.125,0.375) -- (1.375,0.625);
\node at (-1.4,0.6) {\tiny $\nu-n+1$};
\node at (-0.30000000000000004,-0.3) {\tiny $\nu-n+1$};
\node at (1.4,1.4) {\tiny $A^{--}$};
\draw [fill=gray,opacity=0.2,gray] (3.0,0.0) -- (3.0,3.0) -- (0.0,3.0) -- (0.0,0.0) ;
\draw (1.8,0.0) -- (3.0,1.2) ;
\draw [->] (2.1,0.3) -- (2.29,0.10999999999999999);
\draw [->] (2.4,0.6) -- (2.59,0.41);
\draw [->] (2.7,0.8999999999999999) -- (2.89,0.71);
\node at (3.7,1.5) {\tiny $A^{-+}$};
\node at (2.8,-0.7) {\tiny $\nu-p+2$};
\end{tikzpicture}}
\\[0pt]
%			  
%		\end{tabular}
%		\end{figure}
%	\item Suppose $p,q\in\even$. Then,\clearpage
%		\begin{figure}[h]
%			\noindent\begin{tabular}{m{1.3cm}rrr}
%			$(\lambda,\nu)\in$&$\mybra{\ss\cup\bb}^c$ & $\bb-\ss$  & $\ss-\bb$\\[0pt]
%			\tevenText{\le0}&{\begin{tikzpicture}[scale=0.5]
\draw [<->,thick] (0.0,3.0) node (yaxis) [above] {}
	|- (3.0,0.0) node (xaxis) [below] {};
\draw (0,1.5)  -- (1.5,0) ;
\draw [->] (0.375,1.125) -- (0.125,0.875);
\draw [->] (0.75,0.75) -- (0.5,0.5);
\draw [->] (1.125,0.375) -- (0.875,0.125);
\node at (1.4,1.4) {\tiny $A^{++}$};
\node at (0.3999999999999999,-0.3) {\tiny $-\nu$};
\node at (-0.6,1.2) {\tiny $-\nu$};
\draw [fill=gray,opacity=0.2,gray] (1.5,0.0) -- (0.0,1.5) -- (0.0,0.0) ;
\draw (0.0,1.8) -- (1.2,3.0) ;
\draw [->] (0.3,2.1) -- (0.49,1.9100000000000001);
\draw [->] (0.6,2.4) -- (0.79,2.21);
\draw [->] (0.8999999999999999,2.7) -- (1.0899999999999999,2.5100000000000002);
\node at (1.6,3.4) {\tiny $A^{-+}$};
\node at (-1.7,1.9000000000000001) {\tiny $$};
\end{tikzpicture}}
&\hspace{-0.5cm}{\begin{tikzpicture}[scale=0.5]
\draw [<->,thick] (0.0,3.0) node (yaxis) [above] {}
	|- (3.0,0.0) node (xaxis) [below] {};
\draw (0,1.5)  -- (1.5,0) ;
\draw [->] (0.375,1.125) -- (0.125,0.875);
\draw [->] (0.75,0.75) -- (0.5,0.5);
\draw [->] (1.125,0.375) -- (0.875,0.125);
\node at (1.4,1.4) {\tiny $A^{++}$};
\node at (0.3999999999999999,-0.3) {\tiny $-\nu$};
\node at (-0.6,1.2) {\tiny $-\nu$};
\draw [fill=gray,opacity=0.2,gray] (1.5,0.0) -- (0.0,1.5) -- (0.0,0.0) ;
\draw (0.0,1.8) -- (1.2,3.0) ;
\draw [->] (0.3,2.1) -- (0.49,1.9100000000000001);
\draw [->] (0.6,2.4) -- (0.79,2.21);
\draw [->] (0.8999999999999999,2.7) -- (1.0899999999999999,2.5100000000000002);
\node at (1.6,3.4) {\tiny $A^{-+}$};
\node at (-1.7,1.9000000000000001) {\tiny $$};
\end{tikzpicture}}
&\hspace{-0.5cm}{\begin{tikzpicture}[scale=0.5]
\draw [<->,thick] (0.0,3.0) node (yaxis) [above] {}
	|- (3.0,0.0) node (xaxis) [below] {};
\draw (0,1.5)  -- (1.5,0) ;
\draw [->] (0.375,1.125) -- (0.125,0.875);
\draw [->] (0.75,0.75) -- (0.5,0.5);
\draw [->] (1.125,0.375) -- (0.875,0.125);
\node at (1.4,1.4) {\tiny $A^{++}$};
\node at (0.3999999999999999,-0.3) {\tiny $-\nu$};
\node at (-0.6,1.2) {\tiny $-\nu$};
\draw [pattern=north west lines, pattern color=purple] (1.5,0.0) -- (0.0,1.5) -- (0.0,0.0) ;
\draw (0.0,1.8) -- (1.2,3.0) ;
\draw [->] (0.3,2.1) -- (0.49,1.9100000000000001);
\draw [->] (0.6,2.4) -- (0.79,2.21);
\draw [->] (0.8999999999999999,2.7) -- (1.0899999999999999,2.5100000000000002);
\node at (1.6,3.4) {\tiny $A^{-+}$};
\node at (-1.7,1.9000000000000001) {\tiny $$};
\draw [pattern=north east lines, pattern color=green!50!black] (3.0,0.0) -- (3.0,3.0) -- (0.0,3.0) -- (0.0,0.0) ;
\end{tikzpicture}}
\\[0pt]
%			\toddText{\le n-3}&{\begin{tikzpicture}[scale=0.5]
\draw [<->,thick] (0.0,3.0) node (yaxis) [above] {}
	|- (3.0,0.0) node (xaxis) [below] {};
\draw (1.8,0.0) -- (3.0,1.2) ;
\draw [->] (2.1,0.3) -- (1.9100000000000001,0.49);
\draw [->] (2.4,0.6) -- (2.21,0.79);
\draw [->] (2.7,0.8999999999999999) -- (2.5100000000000002,1.0899999999999999);
\node at (3.7,1.5) {\tiny $A^{+-}$};
\node at (1.8,-0.9) {\tiny $$};
\draw [fill=gray,opacity=0.2,gray] (3.0,0.0) -- (3.0,3.0) -- (0.0,3.0) -- (0.0,0.0) ;
\end{tikzpicture}}
&\hspace{-0.5cm}{\begin{tikzpicture}[scale=0.5]
\draw [<->,thick] (0.0,3.0) node (yaxis) [above] {}
	|- (3.0,0.0) node (xaxis) [below] {};
\draw (1.8,0.0) -- (3.0,1.2) ;
\draw [->] (2.1,0.3) -- (1.9100000000000001,0.49);
\draw [->] (2.4,0.6) -- (2.21,0.79);
\draw [->] (2.7,0.8999999999999999) -- (2.5100000000000002,1.0899999999999999);
\node at (3.7,1.5) {\tiny $A^{+-}$};
\node at (1.8,-0.9) {\tiny $$};
\draw [fill=gray,opacity=0.2,gray] (1.8,0.0) -- (3.0,1.2) -- (3.0,3.0) -- (0.0,3.0) -- (0.0,0.0) ;
\end{tikzpicture}}
&\hspace{-0.5cm}{\begin{tikzpicture}[scale=0.5]
\draw [<->,thick] (0.0,3.0) node (yaxis) [above] {}
	|- (3.0,0.0) node (xaxis) [below] {};
\draw (1.8,0.0) -- (3.0,1.2) ;
\draw [->] (2.1,0.3) -- (1.9100000000000001,0.49);
\draw [->] (2.4,0.6) -- (2.21,0.79);
\draw [->] (2.7,0.8999999999999999) -- (2.5100000000000002,1.0899999999999999);
\node at (3.7,1.5) {\tiny $A^{+-}$};
\node at (1.8,-0.9) {\tiny $$};
\draw [pattern=north east lines, pattern color=green!50!black] (3.0,0.0) -- (3.0,3.0) -- (0.0,3.0) -- (0.0,0.0) ;
\draw (1.8,0.0) -- (3.0,1.2) ;
\draw [->] (2.1,0.3) -- (1.9100000000000001,0.49);
\draw [->] (2.4,0.6) -- (2.21,0.79);
\draw [->] (2.7,0.8999999999999999) -- (2.5100000000000002,1.0899999999999999);
\node at (3.7,1.5) {\tiny $A^{+-}$};
\node at (1.8,-0.9) {\tiny $$};
\draw [pattern=north west lines, pattern color=purple] (1.8,0.0) -- (3.0,1.2) -- (3.0,3.0) -- (0.0,3.0) -- (0.0,0.0) ;
\end{tikzpicture}}
\\[0pt]
%			\tevenText{>0}&{\begin{tikzpicture}[scale=0.5]
\draw [<->,thick] (0.0,3.0) node (yaxis) [above] {}
	|- (3.0,0.0) node (xaxis) [below] {};
\draw (0.0,1.8) -- (1.2,3.0) ;
\draw [->] (0.3,2.1) -- (0.49,1.9100000000000001);
\draw [->] (0.6,2.4) -- (0.79,2.21);
\draw [->] (0.8999999999999999,2.7) -- (1.0899999999999999,2.5100000000000002);
\node at (1.6,3.4) {\tiny $A^{-+}$};
\node at (-1.7,1.9000000000000001) {\tiny $$};
\draw [fill=gray,opacity=0.2,gray] (0.0,1.8) -- (1.2,3.0) -- (3.0,3.0) -- (3.0,0.0) -- (0.0,0.0) ;
\end{tikzpicture}}
&\hspace{-0.5cm}{\begin{tikzpicture}[scale=0.5]
\draw [<->,thick] (0.0,3.0) node (yaxis) [above] {}
	|- (3.0,0.0) node (xaxis) [below] {};
\draw (0.0,1.8) -- (1.2,3.0) ;
\draw [->] (0.3,2.1) -- (0.49,1.9100000000000001);
\draw [->] (0.6,2.4) -- (0.79,2.21);
\draw [->] (0.8999999999999999,2.7) -- (1.0899999999999999,2.5100000000000002);
\node at (1.6,3.4) {\tiny $A^{-+}$};
\node at (-1.7,1.9000000000000001) {\tiny $$};
\draw [fill=gray,opacity=0.2,gray] (0.0,1.8) -- (1.2,3.0) -- (3.0,3.0) -- (3.0,0.0) -- (0.0,0.0) ;
\end{tikzpicture}}
&\hspace{-0.5cm}{\begin{tikzpicture}[scale=0.5]
\draw [<->,thick] (0.0,3.0) node (yaxis) [above] {}
	|- (3.0,0.0) node (xaxis) [below] {};
\draw (0.0,1.8) -- (1.2,3.0) ;
\draw [->] (0.3,2.1) -- (0.49,1.9100000000000001);
\draw [->] (0.6,2.4) -- (0.79,2.21);
\draw [->] (0.8999999999999999,2.7) -- (1.0899999999999999,2.5100000000000002);
\node at (1.6,3.4) {\tiny $A^{-+}$};
\node at (-1.7,1.9000000000000001) {\tiny $$};
\draw [fill=gray,opacity=0.2,gray] (3.0,0.0) -- (3.0,3.0) -- (0.0,3.0) -- (0.0,0.0) ;
\end{tikzpicture}}
\\[0pt]
%			\toddText{>n-3}&{\begin{tikzpicture}[scale=0.5]
\draw [<->,thick] (0.0,3.0) node (yaxis) [above] {}
	|- (3.0,0.0) node (xaxis) [below] {};
\draw (0,1.5)  -- (1.5,0) ;
\draw [->] (0.375,1.125) -- (0.625,1.375);
\draw [->] (0.75,0.75) -- (1.0,1.0);
\draw [->] (1.125,0.375) -- (1.375,0.625);
\node at (-1.4,0.6) {\tiny $\nu-n+1$};
\node at (-0.30000000000000004,-0.3) {\tiny $\nu-n+1$};
\node at (1.4,1.4) {\tiny $A^{--}$};
\draw (0.0,1.8) -- (1.2,3.0) ;
\draw [->] (0.3,2.1) -- (0.10999999999999999,2.29);
\draw [->] (0.6,2.4) -- (0.41,2.59);
\draw [->] (0.8999999999999999,2.7) -- (0.71,2.89);
\node at (1.6,3.4) {\tiny $A^{+-}$};
\node at (-1.4,2.7) {\tiny $\nu-q+1$};
\draw [fill=gray,opacity=0.2,gray] (0.0,1.8) -- (1.2,3.0) -- (0.0,3.0) ;
\end{tikzpicture}}
&\hspace{-0.5cm}{\begin{tikzpicture}[scale=0.5]
\draw [<->,thick] (0.0,3.0) node (yaxis) [above] {}
	|- (3.0,0.0) node (xaxis) [below] {};
\draw (0,1.5)  -- (1.5,0) ;
\draw [->] (0.375,1.125) -- (0.625,1.375);
\draw [->] (0.75,0.75) -- (1.0,1.0);
\draw [->] (1.125,0.375) -- (1.375,0.625);
\node at (-1.4,0.6) {\tiny $\nu-n+1$};
\node at (-0.30000000000000004,-0.3) {\tiny $\nu-n+1$};
\node at (1.4,1.4) {\tiny $A^{--}$};
\draw (0.0,1.8) -- (1.2,3.0) ;
\draw [->] (0.3,2.1) -- (0.10999999999999999,2.29);
\draw [->] (0.6,2.4) -- (0.41,2.59);
\draw [->] (0.8999999999999999,2.7) -- (0.71,2.89);
\node at (1.6,3.4) {\tiny $A^{+-}$};
\node at (-1.4,2.7) {\tiny $\nu-q+1$};
\draw [fill=gray,opacity=0.2,gray] (0.0,1.8) -- (1.2,3.0) -- (0.0,3.0) ;
\end{tikzpicture}}
&\hspace{-0.5cm}{\begin{tikzpicture}[scale=0.5]
\draw [<->,thick] (0.0,3.0) node (yaxis) [above] {}
	|- (3.0,0.0) node (xaxis) [below] {};
\draw (0,1.5)  -- (1.5,0) ;
\draw [->] (0.375,1.125) -- (0.625,1.375);
\draw [->] (0.75,0.75) -- (1.0,1.0);
\draw [->] (1.125,0.375) -- (1.375,0.625);
\node at (-1.4,0.6) {\tiny $\nu-n+1$};
\node at (-0.30000000000000004,-0.3) {\tiny $\nu-n+1$};
\node at (1.4,1.4) {\tiny $A^{--}$};
\draw [pattern=north east lines, pattern color=green!50!black] (3.0,0.0) -- (3.0,3.0) -- (0.0,3.0) -- (0.0,0.0) ;
\draw (0.0,1.8) -- (1.2,3.0) ;
\draw [->] (0.3,2.1) -- (0.10999999999999999,2.29);
\draw [->] (0.6,2.4) -- (0.41,2.59);
\draw [->] (0.8999999999999999,2.7) -- (0.71,2.89);
\node at (1.6,3.4) {\tiny $A^{+-}$};
\node at (-1.4,2.7) {\tiny $\nu-q+1$};
\draw [pattern=north west lines, pattern color=purple] (0.0,1.8) -- (1.2,3.0) -- (0.0,3.0) ;
\end{tikzpicture}}
\\[0pt]
%		\end{tabular}
%		\end{figure}
%	\item Suppose $p\in\even,q\in\odd$. Then for $\nu\in\odd$, $R_{\lambda,\nu}^X$ is surjective. Otherwise (for $\nu\in\even$) we have,\clearpage
%	  \begin{figure}[h]
%		  \noindent\begin{tabular}{@{}m{1.6cm}@{}ccc}
%	      $(\lambda,\nu)\in$&$\mybra{//\cup\backslash\backslash}^c$ & $\backslash\backslash-//$  & $//\cap\backslash\backslash,k> l$\\[0pt]
%	      \vspace{-3cm}$\nu\leq0$&{\begin{tikzpicture}[scale=0.5]
\draw [<->,thick] (0.0,3.0) node (yaxis) [above] {}
	|- (3.0,0.0) node (xaxis) [below] {};
\draw (0,1.5)  -- (1.5,0) ;
\draw [->] (0.375,1.125) -- (0.125,0.875);
\draw [->] (0.75,0.75) -- (0.5,0.5);
\draw [->] (1.125,0.375) -- (0.875,0.125);
\node at (1.4,1.4) {\tiny $A^{++}$};
\node at (0.3999999999999999,-0.3) {\tiny $-\nu$};
\node at (-0.6,1.2) {\tiny $-\nu$};
\draw [fill=gray,opacity=0.2,gray] (1.5,0.0) -- (0.0,1.5) -- (0.0,0.0) ;
\draw (1.8,0.0) -- (3.0,1.2) ;
\draw [->] (2.1,0.3) -- (1.9100000000000001,0.49);
\draw [->] (2.4,0.6) -- (2.21,0.79);
\draw [->] (2.7,0.8999999999999999) -- (2.5100000000000002,1.0899999999999999);
\node at (3.7,1.5) {\tiny $A^{+-}$};
\node at (1.8,-0.9) {\tiny $$};
\draw (0.0,1.8) -- (1.2,3.0) ;
\draw [->] (0.3,2.1) -- (0.49,1.9100000000000001);
\draw [->] (0.6,2.4) -- (0.79,2.21);
\draw [->] (0.8999999999999999,2.7) -- (1.0899999999999999,2.5100000000000002);
\node at (1.6,3.4) {\tiny $A^{-+}$};
\node at (-1.7,1.9000000000000001) {\tiny $$};
\end{tikzpicture}}
&\hspace{-0.5cm}{\begin{tikzpicture}[scale=0.5]
\draw [<->,thick] (0.0,3.0) node (yaxis) [above] {}
	|- (3.0,0.0) node (xaxis) [below] {};
\draw (0,1.5)  -- (1.5,0) ;
\draw [->] (0.375,1.125) -- (0.125,0.875);
\draw [->] (0.75,0.75) -- (0.5,0.5);
\draw [->] (1.125,0.375) -- (0.875,0.125);
\node at (1.4,1.4) {\tiny $A^{++}$};
\node at (0.3999999999999999,-0.3) {\tiny $-\nu$};
\node at (-0.6,1.2) {\tiny $-\nu$};
\draw [fill=gray,opacity=0.2,gray] (1.5,0.0) -- (0.0,1.5) -- (0.0,0.0) ;
\draw (1.8,0.0) -- (3.0,1.2) ;
\draw [->] (2.1,0.3) -- (1.9100000000000001,0.49);
\draw [->] (2.4,0.6) -- (2.21,0.79);
\draw [->] (2.7,0.8999999999999999) -- (2.5100000000000002,1.0899999999999999);
\node at (3.7,1.5) {\tiny $A^{+-}$};
\node at (1.8,-0.9) {\tiny $$};
\draw (0.0,1.8) -- (1.2,3.0) ;
\draw [->] (0.3,2.1) -- (0.49,1.9100000000000001);
\draw [->] (0.6,2.4) -- (0.79,2.21);
\draw [->] (0.8999999999999999,2.7) -- (1.0899999999999999,2.5100000000000002);
\node at (1.6,3.4) {\tiny $A^{-+}$};
\node at (-1.7,1.9000000000000001) {\tiny $$};
\end{tikzpicture}}
&\hspace{-0.5cm}{\begin{tikzpicture}[scale=0.5]
\draw [<->,thick] (0.0,3.0) node (yaxis) [above] {}
	|- (3.0,0.0) node (xaxis) [below] {};
\draw (0,1.5)  -- (1.5,0) ;
\draw [->] (0.375,1.125) -- (0.125,0.875);
\draw [->] (0.75,0.75) -- (0.5,0.5);
\draw [->] (1.125,0.375) -- (0.875,0.125);
\node at (1.4,1.4) {\tiny $A^{++}$};
\node at (0.3999999999999999,-0.3) {\tiny $-\nu$};
\node at (-0.6,1.2) {\tiny $-\nu$};
\draw [pattern=north west lines, pattern color=purple] (1.5,0.0) -- (0.0,1.5) -- (0.0,0.0) ;
\draw (1.8,0.0) -- (3.0,1.2) ;
\draw [->] (2.1,0.3) -- (1.9100000000000001,0.49);
\draw [->] (2.4,0.6) -- (2.21,0.79);
\draw [->] (2.7,0.8999999999999999) -- (2.5100000000000002,1.0899999999999999);
\node at (3.7,1.5) {\tiny $A^{+-}$};
\node at (1.8,-0.9) {\tiny $$};
\draw [pattern=north east lines, pattern color=green!50!black] (3.0,0.0) -- (3.0,3.0) -- (0.0,3.0) -- (0.0,0.0) ;
\draw (0.0,1.8) -- (1.2,3.0) ;
\draw [->] (0.3,2.1) -- (0.49,1.9100000000000001);
\draw [->] (0.6,2.4) -- (0.79,2.21);
\draw [->] (0.8999999999999999,2.7) -- (1.0899999999999999,2.5100000000000002);
\node at (1.6,3.4) {\tiny $A^{-+}$};
\node at (-1.7,1.9000000000000001) {\tiny $$};
\end{tikzpicture}}
\\[0pt]
%	      \vspace{-3cm}$
%	      \begin{array}{l}
%		      \nu>0\\\nu\le\frac{n-3}{2}
%	      \end{array}
%	      $&{\begin{tikzpicture}[scale=0.5]
\draw [<->,thick] (0.0,3.0) node (yaxis) [above] {}
	|- (3.0,0.0) node (xaxis) [below] {};
\draw (1.8,0.0) -- (3.0,1.2) ;
\draw [->] (2.1,0.3) -- (1.9100000000000001,0.49);
\draw [->] (2.4,0.6) -- (2.21,0.79);
\draw [->] (2.7,0.8999999999999999) -- (2.5100000000000002,1.0899999999999999);
\node at (3.7,1.5) {\tiny $A^{+-}$};
\node at (1.8,-0.9) {\tiny $$};
\draw [fill=gray,opacity=0.2,gray] (1.8,0.0) -- (3.0,1.2) -- (3.0,3.0) -- (1.2,3.0) -- (0.0,1.8) -- (0.0,0.0) ;
\draw (0.0,1.8) -- (1.2,3.0) ;
\draw [->] (0.3,2.1) -- (0.49,1.9100000000000001);
\draw [->] (0.6,2.4) -- (0.79,2.21);
\draw [->] (0.8999999999999999,2.7) -- (1.0899999999999999,2.5100000000000002);
\node at (1.6,3.4) {\tiny $A^{-+}$};
\node at (-1.7,1.9000000000000001) {\tiny $$};
\draw [fill=gray,opacity=0.2,gray] (0.0,1.8) -- (1.2,3.0) -- (0.0,3.0) ;
\end{tikzpicture}}
&\hspace{-0.5cm}{\begin{tikzpicture}[scale=0.5]
\draw [<->,thick] (0.0,3.0) node (yaxis) [above] {}
	|- (3.0,0.0) node (xaxis) [below] {};
\draw (1.8,0.0) -- (3.0,1.2) ;
\draw [->] (2.1,0.3) -- (1.9100000000000001,0.49);
\draw [->] (2.4,0.6) -- (2.21,0.79);
\draw [->] (2.7,0.8999999999999999) -- (2.5100000000000002,1.0899999999999999);
\node at (3.7,1.5) {\tiny $A^{+-}$};
\node at (1.8,-0.9) {\tiny $$};
\draw [fill=gray,opacity=0.2,gray] (1.8,0.0) -- (3.0,1.2) -- (3.0,3.0) -- (1.2,3.0) -- (0.0,1.8) -- (0.0,0.0) ;
\draw (0.0,1.8) -- (1.2,3.0) ;
\draw [->] (0.3,2.1) -- (0.49,1.9100000000000001);
\draw [->] (0.6,2.4) -- (0.79,2.21);
\draw [->] (0.8999999999999999,2.7) -- (1.0899999999999999,2.5100000000000002);
\node at (1.6,3.4) {\tiny $A^{-+}$};
\node at (-1.7,1.9000000000000001) {\tiny $$};
\end{tikzpicture}}
&\hspace{-0.5cm}{\begin{tikzpicture}[scale=0.5]
\draw [<->,thick] (0.0,3.0) node (yaxis) [above] {}
	|- (3.0,0.0) node (xaxis) [below] {};
\draw (1.8,0.0) -- (3.0,1.2) ;
\draw [->] (2.1,0.3) -- (1.9100000000000001,0.49);
\draw [->] (2.4,0.6) -- (2.21,0.79);
\draw [->] (2.7,0.8999999999999999) -- (2.5100000000000002,1.0899999999999999);
\node at (3.7,1.5) {\tiny $A^{+-}$};
\node at (1.8,-0.9) {\tiny $$};
\draw [pattern=north west lines, pattern color=purple] (1.8,0.0) -- (3.0,1.2) -- (3.0,3.0) -- (0.0,3.0) -- (0.0,0.0) ;
\draw (0.0,1.8) -- (1.2,3.0) ;
\draw [->] (0.3,2.1) -- (0.49,1.9100000000000001);
\draw [->] (0.6,2.4) -- (0.79,2.21);
\draw [->] (0.8999999999999999,2.7) -- (1.0899999999999999,2.5100000000000002);
\node at (1.6,3.4) {\tiny $A^{-+}$};
\node at (-1.7,1.9000000000000001) {\tiny $$};
\draw [pattern=north east lines, pattern color=green!50!black] (3.0,0.0) -- (3.0,3.0) -- (0.0,3.0) -- (0.0,0.0) ;
\end{tikzpicture}}
\\[0pt]
%              $(\lambda,\nu)\in$&$\mybra{//\cup\backslash\backslash}^c$ && $//\cap\backslash\backslash,k=l$\\[0pt]
%	      \vspace{-3cm}$
%	      \begin{array}{l}
%		      \nu\todd\\\nu=\frac{n-1}{2}
%	      \end{array}
%	      $&{\begin{tikzpicture}[scale=0.5]
\draw [<->,thick] (0.0,3.0) node (yaxis) [above] {}
	|- (3.0,0.0) node (xaxis) [below] {};
\draw (1.8,0.0) -- (3.0,1.2) ;
\draw [->] (2.1,0.3) -- (1.9100000000000001,0.49);
\draw [->] (2.4,0.6) -- (2.21,0.79);
\draw [->] (2.7,0.8999999999999999) -- (2.5100000000000002,1.0899999999999999);
\node at (3.7,1.5) {\tiny $A^{+-}$};
\node at (1.8,-0.9) {\tiny $$};
\draw [fill=gray,opacity=0.2,gray] (1.8,0.0) -- (3.0,1.2) -- (3.0,3.0) -- (0.0,3.0) -- (0.0,0.0) ;
\draw (1.8,0.0) -- (3.0,1.2) ;
\draw [->] (2.1,0.3) -- (2.29,0.10999999999999999);
\draw [->] (2.4,0.6) -- (2.59,0.41);
\draw [->] (2.7,0.8999999999999999) -- (2.89,0.71);
\node at (3.7,1.5) {\tiny $A^{+-}$};
\node at (1.8,-0.9) {\tiny $$};
\end{tikzpicture}}
&\hspace{-0.5cm}&\hspace{-0.5cm}{\begin{tikzpicture}[scale=0.5]
\draw [<->,thick] (0.0,3.0) node (yaxis) [above] {}
	|- (3.0,0.0) node (xaxis) [below] {};
\draw (1.8,0.0) -- (3.0,1.2) ;
\draw [->] (2.1,0.3) -- (1.9100000000000001,0.49);
\draw [->] (2.4,0.6) -- (2.21,0.79);
\draw [->] (2.7,0.8999999999999999) -- (2.5100000000000002,1.0899999999999999);
\node at (3.7,1.5) {\tiny $A^{+-}$};
\node at (1.8,-0.9) {\tiny $$};
\draw [pattern=north west lines, pattern color=purple] (1.8,0.0) -- (3.0,1.2) -- (3.0,3.0) -- (0.0,3.0) -- (0.0,0.0) ;
\draw (1.8,0.0) -- (3.0,1.2) ;
\draw [->] (2.1,0.3) -- (2.29,0.10999999999999999);
\draw [->] (2.4,0.6) -- (2.59,0.41);
\draw [->] (2.7,0.8999999999999999) -- (2.89,0.71);
\node at (3.7,1.5) {\tiny $A^{+-}$};
\node at (1.8,-0.9) {\tiny $$};
\draw [pattern=north east lines, pattern color=green!50!black] (3.0,0.0) -- (3.0,3.0) -- (0.0,3.0) -- (0.0,0.0) ;
\end{tikzpicture}}
\\[0pt]
%	      $(\lambda,\nu)\in$&$\mybra{//\cup\backslash\backslash}^c$ & $//-\backslash\backslash$  & $//\cap\backslash\backslash,k< l$\\[0pt]
%	      \vspace{-3cm}
%	      $
%	      \begin{array}{l}
%		      \nu\ge\frac{n+1}{2}\\\nu\le n-3
%	      \end{array}
%	      $
%	      &{\begin{tikzpicture}[scale=0.5]
\draw [<->,thick] (0.0,3.0) node (yaxis) [above] {}
	|- (3.0,0.0) node (xaxis) [below] {};
\draw (1.8,0.0) -- (3.0,1.2) ;
\draw [->] (2.1,0.3) -- (2.29,0.10999999999999999);
\draw [->] (2.4,0.6) -- (2.59,0.41);
\draw [->] (2.7,0.8999999999999999) -- (2.89,0.71);
\node at (3.7,1.5) {\tiny $A^{-+}$};
\node at (2.8,-0.7) {\tiny $\nu-p+2$};
\draw (0.0,1.8) -- (1.2,3.0) ;
\draw [->] (0.3,2.1) -- (0.10999999999999999,2.29);
\draw [->] (0.6,2.4) -- (0.41,2.59);
\draw [->] (0.8999999999999999,2.7) -- (0.71,2.89);
\node at (1.6,3.4) {\tiny $A^{+-}$};
\node at (-1.4,2.7) {\tiny $\nu-q+1$};
\draw [fill=gray,opacity=0.2,gray] (0.0,1.8) -- (1.2,3.0) -- (0.0,3.0) ;
\end{tikzpicture}}
&\hspace{-0.5cm}{\begin{tikzpicture}[scale=0.5]
\draw [<->,thick] (0.0,3.0) node (yaxis) [above] {}
	|- (3.0,0.0) node (xaxis) [below] {};
\draw (1.8,0.0) -- (3.0,1.2) ;
\draw [->] (2.1,0.3) -- (2.29,0.10999999999999999);
\draw [->] (2.4,0.6) -- (2.59,0.41);
\draw [->] (2.7,0.8999999999999999) -- (2.89,0.71);
\node at (3.7,1.5) {\tiny $A^{-+}$};
\node at (2.8,-0.7) {\tiny $\nu-p+2$};
\draw [pattern=north east lines, pattern color=green!50!black] (3.0,0.0) -- (3.0,3.0) -- (0.0,3.0) -- (0.0,0.0) ;
\draw (0.0,1.8) -- (1.2,3.0) ;
\draw [->] (0.3,2.1) -- (0.10999999999999999,2.29);
\draw [->] (0.6,2.4) -- (0.41,2.59);
\draw [->] (0.8999999999999999,2.7) -- (0.71,2.89);
\node at (1.6,3.4) {\tiny $A^{+-}$};
\node at (-1.4,2.7) {\tiny $\nu-q+1$};
\draw [pattern=north west lines, pattern color=purple] (0.0,1.8) -- (1.2,3.0) -- (0.0,3.0) ;
\end{tikzpicture}}
&\hspace{-0.5cm}{\begin{tikzpicture}[scale=0.5]
\draw [<->,thick] (0.0,3.0) node (yaxis) [above] {}
	|- (3.0,0.0) node (xaxis) [below] {};
\draw (1.8,0.0) -- (3.0,1.2) ;
\draw [->] (2.1,0.3) -- (2.29,0.10999999999999999);
\draw [->] (2.4,0.6) -- (2.59,0.41);
\draw [->] (2.7,0.8999999999999999) -- (2.89,0.71);
\node at (3.7,1.5) {\tiny $A^{-+}$};
\node at (2.8,-0.7) {\tiny $\nu-p+2$};
\draw [pattern=north east lines, pattern color=green!50!black] (1.8,0.0) -- (3.0,1.2) -- (3.0,0.0) ;
\draw (0.0,1.8) -- (1.2,3.0) ;
\draw [->] (0.3,2.1) -- (0.10999999999999999,2.29);
\draw [->] (0.6,2.4) -- (0.41,2.59);
\draw [->] (0.8999999999999999,2.7) -- (0.71,2.89);
\node at (1.6,3.4) {\tiny $A^{+-}$};
\node at (-1.4,2.7) {\tiny $\nu-q+1$};
\draw [pattern=north west lines, pattern color=purple] (0.0,1.8) -- (1.2,3.0) -- (0.0,3.0) ;
\draw (0.0,1.8) -- (1.2,3.0) ;
\draw [->] (0.3,2.1) -- (0.10999999999999999,2.29);
\draw [->] (0.6,2.4) -- (0.41,2.59);
\draw [->] (0.8999999999999999,2.7) -- (0.71,2.89);
\node at (1.6,3.4) {\tiny $A^{+-}$};
\node at (-1.4,2.7) {\tiny $\nu-q+1$};
\draw [pattern=north east lines, pattern color=green!50!black] (0.0,1.8) -- (1.2,3.0) -- (0.0,3.0) ;
\end{tikzpicture}}
\\[0pt]
%	      \vspace{-3cm}$
%	      \nu>n-3$&{\begin{tikzpicture}[scale=0.5]
\draw [<->,thick] (0.0,3.0) node (yaxis) [above] {}
	|- (3.0,0.0) node (xaxis) [below] {};
\draw (0,1.5)  -- (1.5,0) ;
\draw [->] (0.375,1.125) -- (0.625,1.375);
\draw [->] (0.75,0.75) -- (1.0,1.0);
\draw [->] (1.125,0.375) -- (1.375,0.625);
\node at (-1.4,0.6) {\tiny $\nu-n+1$};
\node at (-0.30000000000000004,-0.3) {\tiny $\nu-n+1$};
\node at (1.4,1.4) {\tiny $A^{--}$};
\draw (1.8,0.0) -- (3.0,1.2) ;
\draw [->] (2.1,0.3) -- (2.29,0.10999999999999999);
\draw [->] (2.4,0.6) -- (2.59,0.41);
\draw [->] (2.7,0.8999999999999999) -- (2.89,0.71);
\node at (3.7,1.5) {\tiny $A^{-+}$};
\node at (2.8,-0.7) {\tiny $\nu-p+2$};
\draw (0.0,1.8) -- (1.2,3.0) ;
\draw [->] (0.3,2.1) -- (0.10999999999999999,2.29);
\draw [->] (0.6,2.4) -- (0.41,2.59);
\draw [->] (0.8999999999999999,2.7) -- (0.71,2.89);
\node at (1.6,3.4) {\tiny $A^{+-}$};
\node at (-1.4,2.7) {\tiny $\nu-q+1$};
\draw [fill=gray,opacity=0.2,gray] (0.0,1.8) -- (1.2,3.0) -- (0.0,3.0) ;
\end{tikzpicture}}
&\hspace{-0.5cm}{\begin{tikzpicture}[scale=0.5]
\draw [<->,thick] (0.0,3.0) node (yaxis) [above] {}
	|- (3.0,0.0) node (xaxis) [below] {};
\draw (0,1.5)  -- (1.5,0) ;
\draw [->] (0.375,1.125) -- (0.625,1.375);
\draw [->] (0.75,0.75) -- (1.0,1.0);
\draw [->] (1.125,0.375) -- (1.375,0.625);
\node at (-1.4,0.6) {\tiny $\nu-n+1$};
\node at (-0.30000000000000004,-0.3) {\tiny $\nu-n+1$};
\node at (1.4,1.4) {\tiny $A^{--}$};
\draw (1.8,0.0) -- (3.0,1.2) ;
\draw [->] (2.1,0.3) -- (2.29,0.10999999999999999);
\draw [->] (2.4,0.6) -- (2.59,0.41);
\draw [->] (2.7,0.8999999999999999) -- (2.89,0.71);
\node at (3.7,1.5) {\tiny $A^{-+}$};
\node at (2.8,-0.7) {\tiny $\nu-p+2$};
\draw [pattern=north east lines, pattern color=green!50!black] (3.0,0.0) -- (3.0,3.0) -- (0.0,3.0) -- (0.0,0.0) ;
\draw (0.0,1.8) -- (1.2,3.0) ;
\draw [->] (0.3,2.1) -- (0.10999999999999999,2.29);
\draw [->] (0.6,2.4) -- (0.41,2.59);
\draw [->] (0.8999999999999999,2.7) -- (0.71,2.89);
\node at (1.6,3.4) {\tiny $A^{+-}$};
\node at (-1.4,2.7) {\tiny $\nu-q+1$};
\draw [pattern=north west lines, pattern color=purple] (0.0,1.8) -- (1.2,3.0) -- (0.0,3.0) ;
\end{tikzpicture}}
&\hspace{-0.5cm}{\begin{tikzpicture}[scale=0.5]
\draw [<->,thick] (0.0,3.0) node (yaxis) [above] {}
	|- (3.0,0.0) node (xaxis) [below] {};
\draw (0,1.5)  -- (1.5,0) ;
\draw [->] (0.375,1.125) -- (0.625,1.375);
\draw [->] (0.75,0.75) -- (1.0,1.0);
\draw [->] (1.125,0.375) -- (1.375,0.625);
\node at (-1.4,0.6) {\tiny $\nu-n+1$};
\node at (-0.30000000000000004,-0.3) {\tiny $\nu-n+1$};
\node at (1.4,1.4) {\tiny $A^{--}$};
\draw (1.8,0.0) -- (3.0,1.2) ;
\draw [->] (2.1,0.3) -- (2.29,0.10999999999999999);
\draw [->] (2.4,0.6) -- (2.59,0.41);
\draw [->] (2.7,0.8999999999999999) -- (2.89,0.71);
\node at (3.7,1.5) {\tiny $A^{-+}$};
\node at (2.8,-0.7) {\tiny $\nu-p+2$};
\draw [pattern=north east lines, pattern color=green!50!black] (1.8,0.0) -- (3.0,1.2) -- (3.0,0.0) ;
\draw (0.0,1.8) -- (1.2,3.0) ;
\draw [->] (0.3,2.1) -- (0.10999999999999999,2.29);
\draw [->] (0.6,2.4) -- (0.41,2.59);
\draw [->] (0.8999999999999999,2.7) -- (0.71,2.89);
\node at (1.6,3.4) {\tiny $A^{+-}$};
\node at (-1.4,2.7) {\tiny $\nu-q+1$};
\draw [pattern=north west lines, pattern color=purple] (0.0,1.8) -- (1.2,3.0) -- (0.0,3.0) ;
\draw (0.0,1.8) -- (1.2,3.0) ;
\draw [->] (0.3,2.1) -- (0.10999999999999999,2.29);
\draw [->] (0.6,2.4) -- (0.41,2.59);
\draw [->] (0.8999999999999999,2.7) -- (0.71,2.89);
\node at (1.6,3.4) {\tiny $A^{+-}$};
\node at (-1.4,2.7) {\tiny $\nu-q+1$};
\draw [pattern=north east lines, pattern color=green!50!black] (0.0,1.8) -- (1.2,3.0) -- (0.0,3.0) ;
\end{tikzpicture}}
\\[0pt]
%	    \end{tabular}
%	  \end{figure}
%	\end{enumerate}
%	\vspace{-0.9cm}
%	In the diagrams above some of them are filled not with gray, but with colored diagonal lines. This means that the image of the regular
%	SBO $R_{\lambda,\nu}^X$ is zero and the (green/purple)
%	ascending/descending diagonal lines show the images of its residues $R_{\lambda,\nu}^{ \left\{ o \right\}}$ and $\tilde{R}_{\lambda,\nu}^X$ respectively.
%
%	For $p=1$ we have:\clearpage
%
%	\begin{figure}[h]
%		\begin{tabular}{p{4.1cm}p{2.0cm}p{2.0cm}}
%		$(\lambda,\nu)\in$ & $\kern-1.2cm\mybra{\ss\cup\bb}^c$ & $\kern-1.2cm\ss-\bb$ \\
%		\mystack{\nu\teven}{\nu\le0}&{\hspace{-1.5cm}\begin{tikzpicture}[scale=0.7]
\draw (0.0,0.0)  -- (3.0,0.0) ;
\draw[fill=black] (0.0,0.0) circle (2pt);
\draw[fill=black] (1.5,0.0) circle (2pt);
\node at (1.5,0) {\huge ]};
\node[align=center, above] at (1.5,0.5) {\tiny $-\nu$}
;\draw [fill=gray,opacity=0.2,gray] (-0.05,-0.36) -- (-0.05,0.36) -- (1.55,0.36) -- (1.55,-0.36) 
;\end{tikzpicture}}
&{\hspace{-1.5cm}\begin{tikzpicture}[scale=0.7]
\draw (0.0,0.0)  -- (3.0,0.0) ;
\draw[fill=black] (0.0,0.0) circle (2pt);
\draw[fill=black] (1.5,0.0) circle (2pt);
\node at (1.5,0) {\huge ]};
\node[align=center, above] at (1.5,0.5) {\tiny $-\nu$}
;\draw [pattern=north west lines, pattern color=purple] (-0.05,-0.36) -- (-0.05,0.36) -- (1.55,0.36) -- (1.55,-0.36) 
;\draw[fill=black] (1.5,0.0) circle (2pt);
\node at (1.5,0) {\huge ]};
\node[align=center, above] at (1.5,0.5) {\tiny $-\nu$}
;\draw [pattern=north east lines, pattern color=green!50!black] (-0.05,-0.36) -- (-0.05,0.36) -- (3.05,0.36) -- (3.05,-0.36) 
;\end{tikzpicture}}
\\
%		\vspace{-0.5cm}\mystack{\nu,q\teven}{0<\nu<q}&{\hspace{-1.5cm}\begin{tikzpicture}[scale=0.7]
\draw (0.0,0.0)  -- (3.0,0.0) ;
\draw[fill=black] (0.0,0.0) circle (2pt);
\draw [fill=gray,opacity=0.2,gray] (-0.05,-0.36) -- (-0.05,0.36) -- (3.05,0.36) -- (3.05,-0.36) 
;\end{tikzpicture}}
&{\hspace{-1.5cm}\begin{tikzpicture}[scale=0.7]
\draw (0.0,0.0)  -- (3.0,0.0) ;
\draw[fill=black] (0.0,0.0) circle (2pt);
\draw [fill=gray,opacity=0.2,gray] (-0.05,-0.36) -- (-0.05,0.36) -- (3.05,0.36) -- (3.05,-0.36) 
;\end{tikzpicture}}
\\
%		\vspace{-0.5cm}\mystack{\nu\teven,q\todd}{0<\nu<q}&{\hspace{-1.5cm}\begin{tikzpicture}[scale=0.7]
\draw (0.0,0.0)  -- (3.0,0.0) ;
\draw[fill=black] (0.0,0.0) circle (2pt);
\draw [fill=gray,opacity=0.2,gray] (-0.05,-0.36) -- (-0.05,0.36) -- (3.05,0.36) -- (3.05,-0.36) 
;\end{tikzpicture}}
&{\hspace{-1.5cm}\begin{tikzpicture}[scale=0.7]
\draw (0.0,0.0)  -- (3.0,0.0) ;
\draw[fill=black] (0.0,0.0) circle (2pt);
\draw [pattern=north west lines, pattern color=purple] (-0.05,-0.36) -- (-0.05,0.36) -- (3.05,0.36) -- (3.05,-0.36) 
;\draw [pattern=north east lines, pattern color=green!50!black] (-0.05,-0.36) -- (-0.05,0.36) -- (3.05,0.36) -- (3.05,-0.36) 
;\end{tikzpicture}}
\\
%		\vspace{-0.7cm}\mystack{\nu,q\teven}{\nu\ge q}&{\hspace{-1.5cm}\begin{tikzpicture}[scale=0.7]
\draw (0.0,0.0)  -- (3.0,0.0) ;
\draw[fill=black] (0.0,0.0) circle (2pt);
\draw[fill=black] (1.5,0.0) circle (2pt);
\node at (1.5,0) {\huge [};
\node[align=center, above] at (1.5,0.5) {\tiny $\nu-q$}
;\draw [fill=gray,opacity=0.2,gray] (-0.05,-0.36) -- (-0.05,0.36) -- (3.05,0.36) -- (3.05,-0.36) 
;\end{tikzpicture}}
&{\hspace{-1.5cm}\begin{tikzpicture}[scale=0.7]
\draw (0.0,0.0)  -- (3.0,0.0) ;
\draw[fill=black] (0.0,0.0) circle (2pt);
\draw[fill=black] (1.5,0.0) circle (2pt);
\node at (1.5,0) {\huge [};
\node[align=center, above] at (1.5,0.5) {\tiny $\nu-q$}
;\draw [fill=gray,opacity=0.2,gray] (-0.05,-0.36) -- (-0.05,0.36) -- (3.05,0.36) -- (3.05,-0.36) 
;\end{tikzpicture}}
\\
%		\vspace{-0.7cm}\mystack{\nu\teven,q\todd}{\nu\ge q}&{\hspace{-1.5cm}\begin{tikzpicture}[scale=0.7]
\draw (0.0,0.0)  -- (3.0,0.0) ;
\draw[fill=black] (0.0,0.0) circle (2pt);
\draw[fill=black] (1.5,0.0) circle (2pt);
\node at (1.5,0) {\huge [};
\node[align=center, above] at (1.5,0.5) {\tiny $\nu-q$}
;\draw [fill=gray,opacity=0.2,gray] (1.45,-0.36) -- (1.45,0.36) -- (3.05,0.36) -- (3.05,-0.36) 
;\end{tikzpicture}}
&{\hspace{-1.5cm}\begin{tikzpicture}[scale=0.7]
\draw (0.0,0.0)  -- (3.0,0.0) ;
\draw[fill=black] (0.0,0.0) circle (2pt);
\draw[fill=black] (1.5,0.0) circle (2pt);
\node at (1.5,0) {\huge [};
\node[align=center, above] at (1.5,0.5) {\tiny $\nu-q$}
;\draw [pattern=north west lines, pattern color=purple] (1.45,-0.36) -- (1.45,0.36) -- (3.05,0.36) -- (3.05,-0.36) 
;\draw[fill=black] (1.5,0.0) circle (2pt);
\node at (1.5,0) {\huge [};
\node[align=center, above] at (1.5,0.5) {\tiny $\nu-q$}
;\draw [pattern=north east lines, pattern color=green!50!black] (-0.05,-0.36) -- (-0.05,0.36) -- (3.05,0.36) -- (3.05,-0.36) 
;\end{tikzpicture}}
\\
%		\vspace{-0.7cm}\mystack{\nu\todd,q\teven}{\nu\le0}&{\hspace{-1.5cm}\begin{tikzpicture}[scale=0.7]
\draw (0.0,0.0)  -- (3.0,0.0) ;
\draw[fill=black] (0.0,0.0) circle (2pt);
\draw[fill=black] (1.5,0.0) circle (2pt);
\node at (1.5,0) {\huge ]};
\node[align=center, above] at (1.5,0.5) {\tiny $-\nu$}
;\draw [fill=gray,opacity=0.2,gray] (-0.05,-0.36) -- (-0.05,0.36) -- (3.05,0.36) -- (3.05,-0.36) 
;\end{tikzpicture}}
&{\hspace{-1.5cm}\begin{tikzpicture}[scale=0.7]
\draw (0.0,0.0)  -- (3.0,0.0) ;
\draw[fill=black] (0.0,0.0) circle (2pt);
\draw[fill=black] (1.5,0.0) circle (2pt);
\node at (1.5,0) {\huge ]};
\node[align=center, above] at (1.5,0.5) {\tiny $-\nu$}
;\draw [pattern=north west lines, pattern color=purple] (-0.05,-0.36) -- (-0.05,0.36) -- (3.05,0.36) -- (3.05,-0.36) 
;\draw[fill=black] (1.5,0.0) circle (2pt);
\node at (1.5,0) {\huge ]};
\node[align=center, above] at (1.5,0.5) {\tiny $-\nu$}
;\draw [pattern=north east lines, pattern color=green!50!black] (-0.05,-0.36) -- (-0.05,0.36) -- (3.05,0.36) -- (3.05,-0.36) 
;\end{tikzpicture}}
\\
%		\vspace{-0.7cm}\mystack{\nu,q\todd}{\nu\le0}&{\hspace{-1.5cm}\begin{tikzpicture}[scale=0.7]
\draw (0.0,0.0)  -- (3.0,0.0) ;
\draw[fill=black] (0.0,0.0) circle (2pt);
\draw[fill=black] (1.5,0.0) circle (2pt);
\node at (1.5,0) {\huge ]};
\node[align=center, above] at (1.5,0.5) {\tiny $-\nu$}
;\draw [fill=gray,opacity=0.2,gray] (-0.05,-0.36) -- (-0.05,0.36) -- (3.05,0.36) -- (3.05,-0.36) 
;\end{tikzpicture}}
&{\hspace{-1.5cm}\begin{tikzpicture}[scale=0.7]
\draw (0.0,0.0)  -- (3.0,0.0) ;
\draw[fill=black] (0.0,0.0) circle (2pt);
\draw[fill=black] (1.5,0.0) circle (2pt);
\node at (1.5,0) {\huge ]};
\node[align=center, above] at (1.5,0.5) {\tiny $-\nu$}
;\draw [fill=gray,opacity=0.2,gray] (-0.05,-0.36) -- (-0.05,0.36) -- (3.05,0.36) -- (3.05,-0.36) 
;\end{tikzpicture}}
\\
%		\vspace{-0.5cm}\mystack{\nu\todd,q\teven}{0<\nu<q}&{\hspace{-1.5cm}\begin{tikzpicture}[scale=0.7]
\draw (0.0,0.0)  -- (3.0,0.0) ;
\draw[fill=black] (0.0,0.0) circle (2pt);
\draw [fill=gray,opacity=0.2,gray] (-0.05,-0.36) -- (-0.05,0.36) -- (3.05,0.36) -- (3.05,-0.36) 
;\end{tikzpicture}}
&{\hspace{-1.5cm}\begin{tikzpicture}[scale=0.7]
\draw (0.0,0.0)  -- (3.0,0.0) ;
\draw[fill=black] (0.0,0.0) circle (2pt);
\draw [pattern=north west lines, pattern color=purple] (-0.05,-0.36) -- (-0.05,0.36) -- (3.05,0.36) -- (3.05,-0.36) 
;\draw [pattern=north east lines, pattern color=green!50!black] (-0.05,-0.36) -- (-0.05,0.36) -- (3.05,0.36) -- (3.05,-0.36) 
;\end{tikzpicture}}
\\
%		\vspace{-0.5cm}\mystack{\nu,q\todd}{0<\nu<q}&{\hspace{-1.5cm}\begin{tikzpicture}[scale=0.7]
\draw (0.0,0.0)  -- (3.0,0.0) ;
\draw[fill=black] (0.0,0.0) circle (2pt);
\draw [fill=gray,opacity=0.2,gray] (-0.05,-0.36) -- (-0.05,0.36) -- (3.05,0.36) -- (3.05,-0.36) 
;\end{tikzpicture}}
&{\hspace{-1.5cm}\begin{tikzpicture}[scale=0.7]
\draw (0.0,0.0)  -- (3.0,0.0) ;
\draw[fill=black] (0.0,0.0) circle (2pt);
\draw [fill=gray,opacity=0.2,gray] (-0.05,-0.36) -- (-0.05,0.36) -- (3.05,0.36) -- (3.05,-0.36) 
;\end{tikzpicture}}
\\
%		\vspace{-0.7cm}\mystack{\nu\todd,q\teven}{\nu\ge q}&{\hspace{-1.5cm}\begin{tikzpicture}[scale=0.7]
\draw (0.0,0.0)  -- (3.0,0.0) ;
\draw[fill=black] (0.0,0.0) circle (2pt);
\draw[fill=black] (1.5,0.0) circle (2pt);
\node at (1.5,0) {\huge [};
\node[align=center, above] at (1.5,0.5) {\tiny $\nu-q$}
;\draw [fill=gray,opacity=0.2,gray] (1.45,-0.36) -- (1.45,0.36) -- (3.05,0.36) -- (3.05,-0.36) 
;\end{tikzpicture}}
&{\hspace{-1.5cm}\begin{tikzpicture}[scale=0.7]
\draw (0.0,0.0)  -- (3.0,0.0) ;
\draw[fill=black] (0.0,0.0) circle (2pt);
\draw[fill=black] (1.5,0.0) circle (2pt);
\node at (1.5,0) {\huge [};
\node[align=center, above] at (1.5,0.5) {\tiny $\nu-q$}
;\draw [pattern=north west lines, pattern color=purple] (1.45,-0.36) -- (1.45,0.36) -- (3.05,0.36) -- (3.05,-0.36) 
;\draw[fill=black] (1.5,0.0) circle (2pt);
\node at (1.5,0) {\huge [};
\node[align=center, above] at (1.5,0.5) {\tiny $\nu-q$}
;\draw [pattern=north east lines, pattern color=green!50!black] (-0.05,-0.36) -- (-0.05,0.36) -- (3.05,0.36) -- (3.05,-0.36) 
;\end{tikzpicture}}
\\
%		\vspace{-0.7cm}\mystack{\nu,q\todd}{\nu\ge q}&{\hspace{-1.5cm}\begin{tikzpicture}[scale=0.7]
\draw (0.0,0.0)  -- (3.0,0.0) ;
\draw[fill=black] (0.0,0.0) circle (2pt);
\draw[fill=black] (1.5,0.0) circle (2pt);
\node at (1.5,0) {\huge [};
\node[align=center, above] at (1.5,0.5) {\tiny $\nu-q$}
;\draw [fill=gray,opacity=0.2,gray] (-0.05,-0.36) -- (-0.05,0.36) -- (3.05,0.36) -- (3.05,-0.36) 
;\end{tikzpicture}}
&{\hspace{-1.5cm}\begin{tikzpicture}[scale=0.7]
\draw (0.0,0.0)  -- (3.0,0.0) ;
\draw[fill=black] (0.0,0.0) circle (2pt);
\draw[fill=black] (1.5,0.0) circle (2pt);
\node at (1.5,0) {\huge [};
\node[align=center, above] at (1.5,0.5) {\tiny $\nu-q$}
;\draw [fill=gray,opacity=0.2,gray] (-0.05,-0.36) -- (-0.05,0.36) -- (3.05,0.36) -- (3.05,-0.36) 
;\end{tikzpicture}}
\\
%	\end{tabular}\end{figure}
%
%	\begin{figure}[h]
%		\hspace{2cm}
%		\begin{tabular}{p{2.0cm}p{2.3cm}p{2.3cm}}
%		$\kern-1.3cm\bb-\ss$ & $\kern-1.5cm\ss\cap\bb,k<l$ & $\kern-1.3cm\ss\cap\bb,k\geq l$\\
%		{\hspace{-1.5cm}\begin{tikzpicture}[scale=0.7]
\draw (0.0,0.0)  -- (3.0,0.0) ;
\draw[fill=black] (0.0,0.0) circle (2pt);
\draw[fill=black] (1.5,0.0) circle (2pt);
\node at (1.5,0) {\huge ]};
\node[align=center, above] at (1.5,0.5) {\tiny $-\nu$}
;\draw [fill=gray,opacity=0.2,gray] (-0.05,-0.36) -- (-0.05,0.36) -- (1.55,0.36) -- (1.55,-0.36) 
;\end{tikzpicture}}
&{\vspace{-0.9cm}$\kern-0.6cm\times$}&{\hspace{-1.5cm}\begin{tikzpicture}[scale=0.7]
\draw (0.0,0.0)  -- (3.0,0.0) ;
\draw[fill=black] (0.0,0.0) circle (2pt);
\draw[fill=black] (1.5,0.0) circle (2pt);
\node at (1.5,0) {\huge ]};
\node[align=center, above] at (1.5,0.5) {\tiny $-\nu$}
;\draw [pattern=north east lines, pattern color=green!50!black] (-0.05,-0.36) -- (-0.05,0.36) -- (3.05,0.36) -- (3.05,-0.36) 
;\draw[fill=black] (1.5,0.0) circle (2pt);
\node at (1.5,0) {\huge ]};
\node[align=center, above] at (1.5,0.5) {\tiny $-\nu$}
;\draw [pattern=north west lines, pattern color=purple] (-0.05,-0.36) -- (-0.05,0.36) -- (1.55,0.36) -- (1.55,-0.36) 
;\end{tikzpicture}}
\\[0.75em]
%		{\hspace{-1.5cm}\begin{tikzpicture}[scale=0.7]
\draw (0.0,0.0)  -- (3.0,0.0) ;
\draw[fill=black] (0.0,0.0) circle (2pt);
\draw [fill=gray,opacity=0.2,gray] (-0.05,-0.36) -- (-0.05,0.36) -- (3.05,0.36) -- (3.05,-0.36) 
;\end{tikzpicture}}
&{\hspace{-1.5cm}\begin{tikzpicture}[scale=0.7]
\draw (0.0,0.0)  -- (3.0,0.0) ;
\draw[fill=black] (0.0,0.0) circle (2pt);
\draw [fill=gray,opacity=0.2,gray] (-0.05,-0.36) -- (-0.05,0.36) -- (3.05,0.36) -- (3.05,-0.36) 
;\end{tikzpicture}}
&{\hspace{-1.5cm}\begin{tikzpicture}[scale=0.7]
\draw (0.0,0.0)  -- (3.0,0.0) ;
\draw[fill=black] (0.0,0.0) circle (2pt);
\draw [fill=gray,opacity=0.2,gray] (-0.05,-0.36) -- (-0.05,0.36) -- (3.05,0.36) -- (3.05,-0.36) 
;\end{tikzpicture}}
\\[0.75em]
%		{\hspace{-1.5cm}\begin{tikzpicture}[scale=0.7]
\draw (0.0,0.0)  -- (3.0,0.0) ;
\draw[fill=black] (0.0,0.0) circle (2pt);
\draw [fill=gray,opacity=0.2,gray] (-0.05,-0.36) -- (-0.05,0.36) -- (3.05,0.36) -- (3.05,-0.36) 
;\end{tikzpicture}}
&{\hspace{-1.5cm}\begin{tikzpicture}[scale=0.7]
\draw (0.0,0.0)  -- (3.0,0.0) ;
\draw[fill=black] (0.0,0.0) circle (2pt);
\draw [fill=gray,opacity=0.2,gray] (-0.05,-0.36) -- (-0.05,0.36) -- (3.05,0.36) -- (3.05,-0.36) 
;\end{tikzpicture}}
&{\hspace{-1.5cm}\begin{tikzpicture}[scale=0.7]
\draw (0.0,0.0)  -- (3.0,0.0) ;
\draw[fill=black] (0.0,0.0) circle (2pt);
\draw [fill=gray,opacity=0.2,gray] (-0.05,-0.36) -- (-0.05,0.36) -- (3.05,0.36) -- (3.05,-0.36) 
;\end{tikzpicture}}
\\[0.75em]
%		{\hspace{-1.5cm}\begin{tikzpicture}[scale=0.7]
\draw (0.0,0.0)  -- (3.0,0.0) ;
\draw[fill=black] (0.0,0.0) circle (2pt);
\draw[fill=black] (1.5,0.0) circle (2pt);
\node at (1.5,0) {\huge [};
\node[align=center, above] at (1.5,0.5) {\tiny $\nu-q$}
;\draw [fill=gray,opacity=0.2,gray] (1.45,-0.36) -- (1.45,0.36) -- (3.05,0.36) -- (3.05,-0.36) 
;\end{tikzpicture}}
&{\hspace{-1.5cm}\begin{tikzpicture}[scale=0.7]
\draw (0.0,0.0)  -- (3.0,0.0) ;
\draw[fill=black] (0.0,0.0) circle (2pt);
\draw[fill=black] (1.5,0.0) circle (2pt);
\node at (1.5,0) {\huge [};
\node[align=center, above] at (1.5,0.5) {\tiny $\nu-q$}
;\draw [fill=gray,opacity=0.2,gray] (1.45,-0.36) -- (1.45,0.36) -- (3.05,0.36) -- (3.05,-0.36) 
;\end{tikzpicture}}
&{\vspace{-0.9cm}$\kern-0.6cm\times$}\\[1.6em]
%		{\hspace{-1.5cm}\begin{tikzpicture}[scale=0.7]
\draw (0.0,0.0)  -- (3.0,0.0) ;
\draw[fill=black] (0.0,0.0) circle (2pt);
\draw[fill=black] (1.5,0.0) circle (2pt);
\node at (1.5,0) {\huge [};
\node[align=center, above] at (1.5,0.5) {\tiny $\nu-q$}
;\draw [fill=gray,opacity=0.2,gray] (1.45,-0.36) -- (1.45,0.36) -- (3.05,0.36) -- (3.05,-0.36) 
;\end{tikzpicture}}
&{\hspace{-1.5cm}\begin{tikzpicture}[scale=0.7]
\draw (0.0,0.0)  -- (3.0,0.0) ;
\draw[fill=black] (0.0,0.0) circle (2pt);
\draw[fill=black] (1.5,0.0) circle (2pt);
\node at (1.5,0) {\huge [};
\node[align=center, above] at (1.5,0.5) {\tiny $\nu-q$}
;\draw [fill=gray,opacity=0.2,gray] (1.45,-0.36) -- (1.45,0.36) -- (3.05,0.36) -- (3.05,-0.36) 
;\end{tikzpicture}}
&{\vspace{-0.9cm}$\kern-0.6cm\times$}\\[1.8em]
%		{\hspace{-1.5cm}\begin{tikzpicture}[scale=0.7]
\draw (0.0,0.0)  -- (3.0,0.0) ;
\draw[fill=black] (0.0,0.0) circle (2pt);
\draw[fill=black] (1.5,0.0) circle (2pt);
\node at (1.5,0) {\huge ]};
\node[align=center, above] at (1.5,0.5) {\tiny $-\nu$}
;\draw [fill=gray,opacity=0.2,gray] (-0.05,-0.36) -- (-0.05,0.36) -- (3.05,0.36) -- (3.05,-0.36) 
;\end{tikzpicture}}
&{\vspace{-0.9cm}$\kern-0.6cm\times$}&{\hspace{-1.5cm}\begin{tikzpicture}[scale=0.7]
\draw (0.0,0.0)  -- (3.0,0.0) ;
\draw[fill=black] (0.0,0.0) circle (2pt);
\draw[fill=black] (1.5,0.0) circle (2pt);
\node at (1.5,0) {\huge ]};
\node[align=center, above] at (1.5,0.5) {\tiny $-\nu$}
;\draw [pattern=north east lines, pattern color=green!50!black] (-0.05,-0.36) -- (-0.05,0.36) -- (3.05,0.36) -- (3.05,-0.36) 
;\draw[fill=black] (1.5,0.0) circle (2pt);
\node at (1.5,0) {\huge ]};
\node[align=center, above] at (1.5,0.5) {\tiny $-\nu$}
;\draw [pattern=north west lines, pattern color=purple] (-0.05,-0.36) -- (-0.05,0.36) -- (3.05,0.36) -- (3.05,-0.36) 
;\end{tikzpicture}}
\\[0.75em]
%		{\hspace{-1.5cm}\begin{tikzpicture}[scale=0.7]
\draw (0.0,0.0)  -- (3.0,0.0) ;
\draw[fill=black] (0.0,0.0) circle (2pt);
\draw[fill=black] (1.5,0.0) circle (2pt);
\node at (1.5,0) {\huge ]};
\node[align=center, above] at (1.5,0.5) {\tiny $-\nu$}
;\draw [fill=gray,opacity=0.2,gray] (-0.05,-0.36) -- (-0.05,0.36) -- (3.05,0.36) -- (3.05,-0.36) 
;\end{tikzpicture}}
&{\vspace{-0.9cm}$\kern-0.6cm\times$}&{\hspace{-1.5cm}\begin{tikzpicture}[scale=0.7]
\draw (0.0,0.0)  -- (3.0,0.0) ;
\draw[fill=black] (0.0,0.0) circle (2pt);
\draw[fill=black] (1.5,0.0) circle (2pt);
\node at (1.5,0) {\huge ]};
\node[align=center, above] at (1.5,0.5) {\tiny $-\nu$}
;\draw [fill=gray,opacity=0.2,gray] (-0.05,-0.36) -- (-0.05,0.36) -- (3.05,0.36) -- (3.05,-0.36) 
;\end{tikzpicture}}
\\[0.3em]
%		{\hspace{-1.5cm}\begin{tikzpicture}[scale=0.7]
\draw (0.0,0.0)  -- (3.0,0.0) ;
\draw[fill=black] (0.0,0.0) circle (2pt);
\draw [pattern=north west lines, pattern color=red] (-0.05,-0.36) -- (-0.05,0.36) -- (3.05,0.36) -- (3.05,-0.36) 
;\draw [pattern=north east lines, pattern color=blue] (-0.05,-0.36) -- (-0.05,0.36) -- (3.05,0.36) -- (3.05,-0.36) 
;\end{tikzpicture}}
&{\hspace{-1.5cm}\begin{tikzpicture}[scale=0.7]
\draw (0.0,0.0)  -- (3.0,0.0) ;
\draw[fill=black] (0.0,0.0) circle (2pt);
\draw [pattern=north west lines, pattern color=purple] (-0.05,-0.36) -- (-0.05,0.36) -- (3.05,0.36) -- (3.05,-0.36) 
;\draw [pattern=north east lines, pattern color=green!50!black] (-0.05,-0.36) -- (-0.05,0.36) -- (3.05,0.36) -- (3.05,-0.36) 
;\end{tikzpicture}}
&{\hspace{-1.5cm}\begin{tikzpicture}[scale=0.7]
\draw (0.0,0.0)  -- (3.0,0.0) ;
\draw[fill=black] (0.0,0.0) circle (2pt);
\draw [fill=gray,opacity=0.2,gray] (-0.05,-0.36) -- (-0.05,0.36) -- (3.05,0.36) -- (3.05,-0.36) 
;\end{tikzpicture}}
\\[0.75em]
%		{\hspace{-1.5cm}\begin{tikzpicture}[scale=0.7]
\draw (0.0,0.0)  -- (3.0,0.0) ;
\draw[fill=black] (0.0,0.0) circle (2pt);
\draw [pattern=north west lines, pattern color=red] (-0.05,-0.36) -- (-0.05,0.36) -- (3.05,0.36) -- (3.05,-0.36) 
;\draw [pattern=north east lines, pattern color=blue] (-0.05,-0.36) -- (-0.05,0.36) -- (3.05,0.36) -- (3.05,-0.36) 
;\end{tikzpicture}}
&{\hspace{-1.5cm}\begin{tikzpicture}[scale=0.7]
\draw (0.0,0.0)  -- (3.0,0.0) ;
\draw[fill=black] (0.0,0.0) circle (2pt);
\draw [pattern=north west lines, pattern color=purple] (-0.05,-0.36) -- (-0.05,0.36) -- (3.05,0.36) -- (3.05,-0.36) 
;\draw [pattern=north east lines, pattern color=green!50!black] (-0.05,-0.36) -- (-0.05,0.36) -- (3.05,0.36) -- (3.05,-0.36) 
;\end{tikzpicture}}
&{\hspace{-1.5cm}\begin{tikzpicture}[scale=0.7]
\draw (0.0,0.0)  -- (3.0,0.0) ;
\draw[fill=black] (0.0,0.0) circle (2pt);
\draw [fill=gray,opacity=0.2,gray] (-0.05,-0.36) -- (-0.05,0.36) -- (3.05,0.36) -- (3.05,-0.36) 
;\end{tikzpicture}}
\\[0.85em]
%		{\hspace{-1.5cm}\begin{tikzpicture}[scale=0.7]
\draw (0.0,0.0)  -- (3.0,0.0) ;
\draw[fill=black] (0.0,0.0) circle (2pt);
\draw[fill=black] (1.5,0.0) circle (2pt);
\node at (1.5,0) {\huge [};
\node[align=center, above] at (1.5,0.5) {\tiny $\nu-q$}
;\draw [pattern=north east lines, pattern color=blue] (1.45,-0.36) -- (1.45,0.36) -- (3.05,0.36) -- (3.05,-0.36) 
;\draw[fill=black] (1.5,0.0) circle (2pt);
\node at (1.5,0) {\huge [};
\node[align=center, above] at (1.5,0.5) {\tiny $\nu-q$}
;\draw [pattern=north west lines, pattern color=red] (1.45,-0.36) -- (1.45,0.36) -- (3.05,0.36) -- (3.05,-0.36) 
;\end{tikzpicture}}
&{\hspace{-1.5cm}\begin{tikzpicture}[scale=0.7]
\draw (0.0,0.0)  -- (3.0,0.0) ;
\draw[fill=black] (0.0,0.0) circle (2pt);
\draw[fill=black] (1.5,0.0) circle (2pt);
\node at (1.5,0) {\huge [};
\node[align=center, above] at (1.5,0.5) {\tiny $\nu-q$}
;\draw [pattern=north west lines, pattern color=purple] (1.45,-0.36) -- (1.45,0.36) -- (3.05,0.36) -- (3.05,-0.36) 
;\draw[fill=black] (1.5,0.0) circle (2pt);
\node at (1.5,0) {\huge [};
\node[align=center, above] at (1.5,0.5) {\tiny $\nu-q$}
;\draw [pattern=north east lines, pattern color=green!50!black] (1.45,-0.36) -- (1.45,0.36) -- (3.05,0.36) -- (3.05,-0.36) 
;\end{tikzpicture}}
&{\vspace{-0.9cm}$\kern-0.6cm\times$}\\[1.8em]
%		{\hspace{-1.5cm}\begin{tikzpicture}[scale=0.7]
\draw (0.0,0.0)  -- (3.0,0.0) ;
\draw[fill=black] (0.0,0.0) circle (2pt);
\draw[fill=black] (1.5,0.0) circle (2pt);
\node at (1.5,0) {\huge [};
\node[align=center, above] at (1.5,0.5) {\tiny $\nu-q$}
;\draw [pattern=north east lines, pattern color=blue] (1.45,-0.36) -- (1.45,0.36) -- (3.05,0.36) -- (3.05,-0.36) 
;\draw[fill=black] (1.5,0.0) circle (2pt);
\node at (1.5,0) {\huge [};
\node[align=center, above] at (1.5,0.5) {\tiny $\nu-q$}
;\draw [pattern=north west lines, pattern color=red] (-0.05,-0.36) -- (-0.05,0.36) -- (3.05,0.36) -- (3.05,-0.36) 
;\end{tikzpicture}}
&{\hspace{-1.5cm}\begin{tikzpicture}[scale=0.7]
\draw (0.0,0.0)  -- (3.0,0.0) ;
\draw[fill=black] (0.0,0.0) circle (2pt);
\draw[fill=black] (1.5,0.0) circle (2pt);
\node at (1.5,0) {\huge [};
\node[align=center, above] at (1.5,0.5) {\tiny $\nu-q$}
;\draw [pattern=north east lines, pattern color=green!50!black] (-0.05,-0.36) -- (-0.05,0.36) -- (3.05,0.36) -- (3.05,-0.36) 
;\draw[fill=black] (1.5,0.0) circle (2pt);
\node at (1.5,0) {\huge [};
\node[align=center, above] at (1.5,0.5) {\tiny $\nu-q$}
;\draw [pattern=north west lines, pattern color=purple] (-0.05,-0.36) -- (-0.05,0.36) -- (3.05,0.36) -- (3.05,-0.36) 
;\end{tikzpicture}}
&{\vspace{-0.9cm}$\kern-0.6cm\times$}\\
%	\end{tabular}\end{figure}
%	In the diagrams above some of them are filled not with gray, but with colored diagonal lines. This means that the image of the regular SBO $R_{\lambda,\nu}^X$ is zero and:
%	\begin{itemize}
%		\item For $(\lambda,\nu)\in\ss$ the (green/purple)
%			ascending/descending diagonal lines show the images of its residues $R_{\lambda,\nu}^{ \left\{ o \right\}}$ and $\tilde{R}_{\lambda,\nu}^X$ 
%			respectively.
%		\item For $(\lambda,\nu)\notin\ss$ the (blue/red) ascending/descending diagonal lines show the images of its residues $R_{\lambda,\nu}^{Y}$ and ${R}_{\lambda,\nu}^C$ 
%			respectively.
%	\end{itemize}
%\begin{remark}
%	We can also find the images of the other SBOs constructed in previous announcement$^1$\footnotetext{1: should I give the explicit Theorem number? How should I cite the prev. announcement then?} as well.
%	Note that
%	the proof of this theorem is performed \textit{independent of} of \cite{howe1993homogeneous}.
%\end{remark}

Let $G=O(p,q)$ be the automorphism group of the quadratic form
on $\R^{p+q}$ of signature $(p,q)$ defined by
\begin{equation*}
	Q_{p,q}(x)
		=x_1^2+\cdots+x_{p}^2-x_{p+1}^2-\cdots-x_{p+q}^2.
\end{equation*}

Let $p,q\in\N$ with $p\ge1$.
Following the notation \cite[(5.1.1)]{KO2},
we set
%%\myfootnote{the results in the table below concern $\lambda\in \Azeven(p,q)$ instead of the more general $A_0(p,q)$. The classification can (I guess) by generalized to the latter, but
%%	this will require classification of $\Hom_{G'}\left( I(\lambda\otimes\varepsilon),J(\nu\otimes\varepsilon') \right)$ for $\varepsilon,\varepsilon'\in\left\{ \pm1 \right\}$.
%%Should I mention all this?}
\begin{equation*}
	\begin{array}[]{c}
	A_0(p,q):=\left\{
		\begin{array}[]{ll}
			\left\{ \lambda\in\Z+\frac{p+q}{2}\mid \lambda>-1 \right\}&(p>1,q\neq0),\\
			\left\{ \lambda\in\Z+\frac{p+q}{2}\mid\lambda\ge\frac{p}{2}-1 \right\}&(p>1,q=0),\\
			\emptyset&(p=1,q\neq0)\\
			&\kern0.55cm\mbox{or }(p=0),\\
			\left\{ -\frac{1}{2},\frac{1}{2} \right\}&(p=1,q=0).
		\end{array}
		\right.\\
	\end{array}
\end{equation*}
Let $p>1$.
We recall from \cite{KO2} that
for any $\lambda\in A_0(p,q)$ each of the following 
5 conditions defines uniquely $(\mathfrak{g},K)$-modules, which are mutually isomorphic.
We shall denote
it\myGrammarNoteFootnote{``it''->``them''?} by $\left( \pi_{+,\lambda}^{p,q} \right)_K$. The $(\mathfrak{g},K)$-module $\left( \pi_{+,\lambda}^{p,q} \right)_K$ is non-zero and irreducible.
\begin{enumerate}
	\item[(i)] A subrepresentation of the degenerate principal series representation $\Ind_{P^{\scalebox{0.5}{$\max$}}}^G\left( \varepsilon\otimes\mathbb{C}_{\lambda} \right)$
		with $K$-type\begin{equation*}
			\begin{array}[]{c}
				\Xi(K:b)\equiv \Xi\left( O(p)\times O(q):b \right)\\:=\displaystyle
				\bigoplus_{\scalebox{0.5}{$\begin{array}[]{c}
				m,n\in\N\\m-n\ge b\\m-n\equiv b\mbox{ mod }2
			\end{array}$}}\mathcal{H}^m\left( \R^p \right)\boxtimes \mathcal{H}^n\left( \R^q \right),\\
			b\equiv b(\lambda,p,q):=\lambda-\frac{p}{2}+\frac{q}{2}+1\in\Z,\\
			\varepsilon\equiv\varepsilon(\lambda,p,q):=(-1)^b.
			\end{array}
		\end{equation*}
	\item[(i)$'$] A quotient of $\Ind_{P^{\scalebox{0.5}{$\max$}}}^G\left( \varepsilon\otimes \C_{-\lambda} \right)$ with $K$-type $\Xi(K:b)$.
	\item[(ii)] A subrepresentation of $C^\infty_\lambda\left( X(p,q) \right)_K$ with $K$-type $\Xi(K:b)$,
		where we define a hyperboloid $X(p,q)$ by\begin{equation*}
			X(p,q):=\left\{ (x,y)\in\R^{p,q}\mid \myabs{x}^2-\myabs{y}^2=1 \right\}
		\end{equation*}
		together with the natural action of $O(p,q)$.
	\item[(iii)]
		The underlying $\left( \mathfrak{g},K \right)$-module of the Dolbeault cohomology group $H_{\overline{\partial}}^{p-2}\left( G/L,\mathscr{L}_{\left( \lambda+\frac{p+q-2}{2} \right)} 
		\right)_K$.
	\item[(iii)$'$] The Zuckerman derived functor module $\mathscr{R}_{\mathfrak{q}}^{p-2}\left( \C_{\lambda f_1} \right)$.
\end{enumerate}
Zuckerman's derived functor modules $\mathscr{R}_{\mathfrak{q}}^S\left(\mathbb C_\lambda\right)$
are defined as an algebraic analogue of the geometric
quantization of elliptic coadjoint orbits,
which gives a vast generalization of Borel-Weil-Bott theorem
for compact Lie groups.
The general theory \cite{vogan1984unitarizability}
assures their nonvanishing and irreducibility in the
``good range'' of parameters $\lambda$, and their unitarizability
in the ``weakly fair range'' of parameters.
These sufficient conditions are not always necessary.
In \cite{kobyashi1992singular}, a precise condition
for $\lambda$ for nonvanishing, irreducibility, and unitarizability
was determined in certain family of Zuckerman's derived functor modules
of $O(p,q)$ and some other classical groups, which include $\gk$-modules 
corresponding to
minimal elliptic orbits as above.

{Each of realizations}
 (i), (i)$'$, (ii), and (iii) also gives a globalization of $\ppqpl$,
namely, a continuous representation of $G$ on a topological vector space. Because all of $(\ppqpl)_K\;\left( \lambda\in A_0(p,q) \right)$ are unitarizable we may and do take the globalization 
$\ppqpl$ to be the unitary representation of $G$.

In turn, for $(p,q)=(1,0)$ it is convenient to define
\begin{equation*}
	\pi_{+,\lambda}^{1,0}=\left\{
		\begin{array}[]{ll}
			\mbox{\textbf{1}}&\left(\lambda=-\frac{1}{2}  \right)\\
			\mbox{\normalfont sgn}&\left( \lambda=\frac{1}{2} \right)\\
			0&\mbox{(otherwise)}
		\end{array}
		\right.
\end{equation*}
Together with the definition for $p>1$ above, this defines the irreducible unitary representations $\pi_{+,\lambda}^{p,q}$ for all $p,q\in\N$ and $\lambda\in A_0(p,q)$.

We define a subset of $A_0(p,q)$ by
\begin{equation*}
\Azeven(p,q):=\left\{ \lambda\in A_0(p,q)\mid \lambda-\frac{p-q}{2}+1\in2\Z\right\}
\end{equation*}
We note that for $\lambda\in A_0(p,q)$, $\varepsilon(\lambda,p,q)=1$ if and only if $\lambda\in \Azeven(p,q).$\myfootnote{this is true -- I checked this before,
this was my reason for defining $\Azeven(p,q)$ in the first place (this footnote will disappear in next version).}

Similarly, we can also define an
irreducible unitary representations $\pi_{-,\lambda}^{p,q}$ of $O(p,q)$
for $\lambda\in A_0(q,p)$, such that the underlying $(\mathfrak{g},K)$-module has the following $K$-type:
%%as a 
%%	subrepresentation of the degenerate principal series representation $\Ind_{P^{\scalebox{0.5}{$\max$}}}^G\left( \varepsilon\otimes\mathbb{C}_{\lambda} \right)$
%%		with $K$-type
		\begin{equation*}
			\begin{array}[]{c}
				\displaystyle
				\bigoplus_{\scalebox{0.5}{$\begin{array}[]{c}
					m,n\in\N\\m-n\le -\lambda+\frac{q}{2}-\frac{p}{2}-1\\m-n\equiv -\lambda+\frac{q}{2}-\frac{p}{2}-1\mbox{ mod }2
			\end{array}$}}\mathcal{H}^m\left( \R^p \right)\boxtimes \mathcal{H}^n\left( \R^q \right).
			\end{array}
		\end{equation*}
%%A degenerate spherical principal series representation $I(\lambda):=\Ind_P^G(\C_{\lambda})$ with parameter $\lambda\in\C$ of $G$ is induced from
%%a character $\C_\lambda$ of a maximal parabolic subgroup $P=MAN_+$
%%with Levi part
%%$M A \simeq O (p, q) \times \{ \pm 1 \} \times \mathbb{R}$.
%%We realize $I(\lambda)$ on the space of $C^\infty$ sections
%%of the $G$-equivariant line bundle\[
%%	\mathcal{L}_\lambda=G\times_{P}\C_\lambda\to G/P
%%\]
%%so that $I(\lambda)$ itself is the smooth Fr\'echet globalization of moderate growth.
%%Our parametrization is chosen in a way that
%%$I(\lambda)$ contains a finite-dimensional submodule if $-\lambda\in2\N$ and a finite-dimensional quotient if $\lambda-\left( p+q\right)\in2\N$ ({\it cf.} \cite{howe1993homogeneous}).

%%Similarly to $I(\lambda)$,
%%we denote by $J(\nu):=\Ind_{P'}^{G'}\left( \C_\nu \right)$ the representation of $G'$
%%induced from a character $\C_\nu$ of a
%%maximal parabolic
%%subgroup $P'$ of $G'$ with Levi part $O(p-1,q)\times\left\{ \pm1 \right\}\times\R$.
From now, we set $\left( G,G' \right)=\left( O(p+1,q+1),O(p,q+1) \right)$.
\begin{theorem}[$G'$-invariant maps between $\pi_{\pm,x}^{p+1,q+1}$ and $\pi_{\pm,y}^{p,q+1}$]\label{thm:Aq}
	The dimensions of $\Hom_{G'}\left(\pi_{\pm,x}^{p+1,q+1}\kern-0.3em\mid_{G'} ,\pi_{\pm,y}^{p,q+1} \right)$
	are then either 0 or 1. 
	In what follows, we assume \begin{equation*}
		\begin{array}[]{c}
			x\in\left\{\begin{array}[]{ll}
				\Azeven(p+1,q+1),&\mbox{for }\delta=+,\\
				\Azeven(q+1,p+1),&\mbox{for }\delta=-,\\
			\end{array}\right.\\
			y\in\left\{\begin{array}[]{ll}
				\Azeven(p,q+1),&\mbox{for }\varepsilon=+,\\
				\Azeven(q+1,p),&\mbox{for }\varepsilon=-,\\
			\end{array}\right.
		\end{array}
	\end{equation*}
	to describe $\gk$-modules $\pi_{\delta,x}^{p+1,q+1}$
	and $\left( \mathfrak{g}',K' \right)$-modules $\pi_{\varepsilon,y}^{p,q+1}$.
	\newline
	%%%%%%%%%%%%%%%%%%%%%%%%%%%%%%%%
\renewcommand{\mystack}[2]{\begin{array}{c}#1,\\#2\end{array}}
\newcommand{\mytable}[9]{{\centering
$\begin{array}{|@{}c@{}|@{}c@{}|@{}c@{}|}
  \hline
	#1& #2&#3\\
  \hline
	#4& #5&#6\\
  \hline
	#7& #8&#9\\
  \hline
\end{array} \newline$
}}
\newcommand{\mytableThreeTwo}[6]{\begin{center}
$\begin{array}{|@{}c@{}|@{}c@{}|}
  \hline
	#1& #2\\
  \hline
	#3& #4\\
  \hline
	#5& #6\\
  \hline
\end{array} \newline$
\end{center}}
\newcommand{\commonShift}{\hspace*{-0.0cm}}
%%%%%%%%%%%%%%%%%%%%%%%%%%%%%%%%%%%%
\begin{enumerate}[(1)]
	\item $p=1,q\in2\Z$
		\\
\hspace*{0cm}\commonShift\mytableThreeTwo	%#1
{}		{\pimyStack[\mid y\geq q/2]}
{\pipx}		{0}
{\pimx}		{g\kern-0.05cm\left( {y-x} \right)}
	\item $p=1,q\in2\Z+1$\\
\hspace*{0cm}\commonShift\mytableThreeTwo	%#2
{}		{\pimyStack[\mid y\geq q/2]}
{\pipx}		{0}
{\pimx}		{g\kern-0.05cm\left( {y-x} \right)}
	\item $p,q\in2\Z$\\
\hspace*{-0cm}\commonShift\mytable	%3
{}	{\pipy}				{\pimy}
{\pipx}	{g\kern-0.05cm\left({x-y}\right)} 	{0}
{\pipx}	{0} 				{g\kern-0.05cm\left( {y-x} \right)}
\item $p\in2\Z,q\in2\Z+1$\\
\commonShift\mytable	%4
{}	{\pipy}	{\pimy}
{\pipx} {0}	{g\kern-0.05cm\left( {-x-y} \right)}
{\pimx} {0} 	{g\kern-0.05cm\left( {y-x} \right)}
\item $p\in2\Z+1,q\in2\Z$\\
\commonShift\mytable	%5
{}			{\pipy}		{\pimy}
{\pipx}			{0} 		{g\kern-0.05cm\left( {-x-y} \right)}	
{\pimx} 		{0} 		{g\kern-0.05cm\left( {y-x} \right)}
\item $p,q\in2\Z+1$\\
\commonShift\mytable	%6
{}		{\pipy}				{\pimy}
{\pipx}		{g\kern-0.05cm\left( {x-y} \right)}	{0}
{\pimx}		{0}				{g\kern-0.05cm\left( {y-x} \right)}	
\end{enumerate}

\end{theorem}
Here $h(x)$ is defined as $h(x)=1$ for $x\in\N$ and $=0$ otherwise.
\begin{remark}
%%Theorem \ref{thm:Aq} generalizes \cite[Thms. 1.2 and 1.3]{kobayashi2015symmetry}
%%(the latter corresponds to the $q=0$ case of the former.
	In the book \cite{kobayashi2015symmetry}, where the SBOs for the $(G,G')=\left( O(n+1,1),O(n,1) \right)$ setting were
	thoroughly studied, two families of irreductible $G$-modules were introduced for $i\in\N$: $T(i)$ and $F(i)$. Among these, the latter is finitely-dimensional, while the former
	is infinitely dimensional. They are characterized by the following non-splitting exact sequences:\begin{equation*}
		\begin{array}[]{c}
			0\to F(i)\to I(-i)\to J(i)\to0,\\
			0\to T(i)\to I(n+i)\to F(i)\to0,\quad i\in\N.
		\end{array}
	\end{equation*}
	In Prop. 16.1 of the same book, the following isomorphism of $\left( \mathfrak{o}(n+1,1),O(n+1)\times O(1) \right)$-modules was shown for $\Z\ni i\ge-n/2$:\begin{equation*}
		\pi^{n+1,1}_{+,\frac{n}{2}+i}\simeq \left\{\begin{array}[]{ll}
			T(i)_K,&i\in\N,\\
			I(n+i)_K,&\mbox{otherwise.}
		\end{array}\right.
	\end{equation*}
	Similarly, for finite dimensional modules $F(i)$ ($i\in\N$) introduced in the same book, the following isomorphism of
	$\left( \mathfrak{o}(n+1,1),O(n+1)\times O(1) \right)$-modules is easy to observe for $i\in\N$:\myfootnote{should I elaborate here?}\begin{equation*}
		\pi^{n+1,1}_{-,\frac{n}{2}+i}\simeq \left\{\begin{array}[]{ll}
			T(i)_K,&i\in\N,\\
			I(n+i)_K,&\mbox{otherwise.}
		\end{array}\right.
	\end{equation*}
	Using these identifications, the Theorem 1.2 of \cite{kobayashi2015symmetry} can be rewritten as follows:
	\begin{enumerate}[(1)]
		\item Suppose that $i\ge j$.\begin{enumerate}
				\item [(1-a)] Assume $i\equiv j\mbox{ mod 2}$. Then
					\begin{equation*}
						m\left( \pi^{n+1,1}_{+,\frac{n}{2}+i},\pi^{n,1}_{+,\frac{n-1}{2}+j} \right)=1,\quad m\left(  \right)
					\end{equation*}<++>
				\item [(1-b)] tesime
			\end{enumerate}
	\end{enumerate}
\end{remark}
\begin{remark}
	Note that the bottom-rightmost entry of every table jointly generalize 
	\cite[Thm. 3.3]{kobayashi93restriction}.
\end{remark}

\begin{definition}[$Y_{\pm,\lambda}^{p,q}$]\label{def:Y}
	Let $p,q\in\N_+$. We define the parameter set $B(p,q)$ as
\begin{equation*}
	\kern-2cm\begin{array}[]{c} 
	B(p,q):=\left\{
		\begin{array}[]{@{}ll@{}} 
			%\emptyset,			&p=1,q>1,\\
			%A_0(p,q),			&p\ge q=1,\\
			\frac{2-p-q}{2}+\N_+,		&p\in2\N_+,q\in2\N_++1,\\
			\frac{p-q}{2}+\Z,		&p\in2\N_++1,q\in2\N_+,\\
			A_0(p,q),&\mbox{otherwise,}
		\end{array}
		\right.\\
	\end{array}
\end{equation*}
Now, direct check (cf. \cite{howe1993homogeneous}) implies that 
for any $\lambda\in B(p,q)$ each of the following conditions defines uniquely
$(\mathfrak{g},K)$-modules,\myfootnote{We talk about $\mathfrak{g}$ here, assuming implicitly that $G=O(p,q)$. Nevertheless, we've set
$(G,G')=(O(p+1,q+1),O(p,q+1))$. I remember that You said that we will fix this in subsequent -- I just keep this for the record.} which are mutually isomorphic. We shall denote
it\myGrammarNoteFootnote{``it''->``them''?} by $\left( Y_{+,\lambda}^{p,q} \right)_K$. The $(\mathfrak{g},K)$-module $\left( Y_{+,\lambda}^{p,q} \right)_K$ is non-zero and irreducible.
\begin{enumerate}
	\item[(i)] A subrepresentation of
		the degenerate principal series representation
		$\Ind_{P^{\scalebox{0.5}{$\max$}}}^G\left( \varepsilon\otimes\mathbb{C}_{\lambda} \right)$
		with $K$-type\begin{equation*}
			\begin{array}[]{c}
				\Sigma(K:b)\equiv \Sigma\left( O(p)\times O(q):b \right)\\:=\displaystyle
				\bigoplus_{\scalebox{0.5}{$\begin{array}[]{c}
				m,n\in\N\\m-n\ge b\\m-n\equiv b\mbox{ mod }2
			\end{array}$}}\mathcal{H}^m\left( \R^p \right)\boxtimes \mathcal{H}^n\left( \R^q \right),\\
			b\equiv b(\lambda,p,q):=\lambda-\frac{p}{2}+\frac{q}{2}+1\in\Z.\\
			\varepsilon\equiv\varepsilon(\lambda,p,q):=(-1)^b.
			\end{array}
		\end{equation*}
	\item[(i)$'$] A\myfootnote{In previous revision, You said that the previous item was ``Wrong''. I did not quite get the reason\ldots Is it still wrong?} quotient of $\Ind_{P^{\scalebox{0.5}{$\max$}}}^G\left( \varepsilon\otimes\C_{-\lambda} \right)$ with $K$-type $\Sigma(K:b)$.
\end{enumerate}
Each of realizations (i) and (i)$'$ also gives a globalization,
namely, a continuous representation of $G$ on a topological vector space. 
%Therefore, we may and to take the globalization
%$Y_{+,\lambda}^{p,q}$ to be the representation of $G$. The case $(p,q)=(1,0)$ is special and we define $Y_{+,\lambda}^{1,0}:=\pi_{+,\lambda}^{1,0}$.

Similarly, we can also define an
irreducible representations $Y_{-,\lambda}^{p,q}$ of $O(p,q)$
for $\lambda\in B(q,p)$, such that the underlying $(\mathfrak{g},K)$-module has the following $K$-type:
%%as a 
%%	subrepresentation of the degenerate principal series representation $\Ind_{P^{\scalebox{0.5}{$\max$}}}^G\left( \varepsilon\otimes\mathbb{C}_{\lambda} \right)$
%%		with $K$-type
		\begin{equation*}
			\begin{array}[]{c}
				\displaystyle
				\bigoplus_{\scalebox{0.5}{$\begin{array}[]{c}
					m,n\in\N\\m-n\le -\lambda+\frac{q}{2}-\frac{p}{2}-1\\m-n\equiv -\lambda+\frac{q}{2}-\frac{p}{2}-1\mbox{ mod }2
			\end{array}$}}\mathcal{H}^m\left( \R^p \right)\boxtimes \mathcal{H}^n\left( \R^q \right).
			\end{array}
		\end{equation*}
\end{definition}

%and still remain equivalent. Using these, we can define $(\mathfrak{g},K)$-modules $\left( Y_{\pm,\lambda}^{p,q} \right)_K$ similarly to $\pi_{\pm,\lambda}^{p,q}$,
%defined for $\lambda\in A_0'(p,q)$ (or $\lambda\in A_0'(q,p)$ for $Y_{-,\lambda}^{p,q}$).
%Again, conditions (i) and (i)$'$ in fact allow us to consider the globalization: representation $Y_{\pm,\lambda}^{p,q}$ of $G$. 
\begin{remark}
We note that $Y_{+,\lambda}^{p,q}$ (or $Y_{-,\lambda}^{p,q}$) is isomorphic to $\pi_{+,\lambda}^{p,q}$ (or $\pi_{-,\lambda}^{p,q}$) for $\lambda\in A_0(p,q)$ (or $A_0(q,p)$).
Also note that $Y_{\pm,\lambda}^{p,q}$ is not unitarizable in general.
\end{remark}


By analogy with $\Azeven(p,q)$ defined above, we define its superset\begin{equation*}
	\Bzeven(p,q):=\left\{ \lambda\in B(p,q)\left| \lambda-\frac{p-q}{2}+1\in2\Z \right.\right\}.
\end{equation*}
Now, the Definition \ref{def:Y} allows us to give the following generalization of Theorem \ref{thm:Aq}:
\begin{theorem}[$G'$-invariant maps between $Y_{\pm,x}^{p+1,q+1}$ and $Y_{\pm,y}^{p,q+1}$]\label{thm:Y}
	The dimensions of $\Hom_{G'}\left(Y_{\pm,x}^{p+1,q+1}\kern-0.3em\mid_{G'} ,Y_{\pm,y}^{p,q+1} \right)$
	are then either 0 or 1. 
	In what follows, we assume \begin{equation*}
		\begin{array}[]{c}
			x\in\left\{\begin{array}[]{ll}
				\Bzeven(p+1,q+1),&\mbox{for }\delta=+,\\
				\Bzeven(q+1,p+1),&\mbox{for }\delta=-,\\
			\end{array}\right.\\
			y\in\left\{\begin{array}[]{ll}
				\Bzeven(p,q+1),&\mbox{for }\varepsilon=+,\\
				\Bzeven(q+1,p),&\mbox{for }\varepsilon=-,\\
			\end{array}\right.
		\end{array}
	\end{equation*}
	to describe $\gk$-modules $Y_{\delta,x}^{p+1,q+1}$
	and $\left( \mathfrak{g}',K' \right)$-modules $Y_{\varepsilon,y}^{p,q+1}$.
	\newline
	%%%%%%%%%%%%%%%%%%%%%%%%%%%%%%%%
\renewcommand{\mystack}[2]{\begin{array}{c}#1,\\#2\end{array}}
\newcommand{\mystackiii}[3]{\begin{array}{c}#1,\\#2,\\#3\end{array}}
\newcommand{\mrv}[5]{#1&#2&#3&#4&#5\\\hline}	
\newcommand{\mriii}[3]{#1&#2&#3\\\hline}	
\newcommand{\mytable}[9]{\begin{center}
$\begin{array}{|@{}c@{}|@{}c@{}|@{}c@{}|}
  \hline
	#1& #2&#3\\
  \hline
	#4& #5&#6\\
  \hline
	#7& #8&#9\\
  \hline
\end{array} \newline$
\end{center}}
\newcommand{\mytableThreeTwo}[6]{\begin{center}
$\begin{array}{|@{}c@{}|@{}c@{}|}
  \hline
	#1& #2\\
  \hline
	#3& #4\\
  \hline
	#5& #6\\
  \hline
\end{array} \newline$
\end{center}}
\newcommand{\commonShift}{\hspace*{-0.0cm}}
%%%%%%%%%%%%%%%%%%%%%%%%%%%%%%%%%%%%
\begin{enumerate}[(1)]
	\item $p=1,q\in2\Z$\\
\hspace*{0cm}\commonShift\mytableThreeTwo	%#1
{}		{\yimyStack[\mid y\geq q/2]}
{\yipx}			{0}
{\yimx}			{g\kern-0.05cm\left( {y-x} \right)}
	\item $p=1,q\in2\Z+1$\\
\hspace*{0cm}\commonShift\mytableThreeTwo	%#2
{}		{\yimyStack[\mid y\geq q/2]}
{\yipx}			{0}
{\yimx}			{g\kern-0.05cm\left( {y-x} \right)}
	\item $p,q\in2\Z$\\
\hspace*{-0cm}\commonShift\mytable	%3
{}	{\yipy}					{\yimy}
{\yipx}	{g\kern-0.05cm\left({x-y}\right)} 	{g\kern-0.05cm\left( {-x-y} \right)}
{\yipx}	{0} 					{g\kern-0.05cm\left( {y-x}\right)}
\item $p\in2\Z,q\in2\Z+1$\\
\commonShift\mytable	%4
{}	{\yipy}	{\yimy}
{\yipx} {0}	{g\kern-0.05cm\left( {-x-y} \right)}
{\yimx} {0} 	{g\kern-0.05cm\left( {y-x} \right)}
\item $p\in2\Z+1,q\in2\Z$\\
\commonShift\mytable	%5
{}		{\yipy}		{\yimy}
{\yipx}		{0} 		{g\kern-0.05cm\left( {-x-y} \right)}	
{\yimx} 	{0} 		{g\kern-0.05cm\left( {y-x} \right)}
\item $p,q\in2\Z+1$\\
\commonShift\mytable	%6
{}		{\yipy}							{\yimy}
{{\yipx}}	{g\kern-0.05cm\left( {x-y} \right)}			{0}
{{\yimx}}	{0}							{g\kern-0.05cm\left( {y-x} \right)}	
\end{enumerate}

\end{theorem}

	A detailed proof will appear elsewhere.

	{\bf Acknowledgement.} The first author was partially supported by the Grant-in-Aid for Scientific Research (A) 25247006.
\nocite{kobayashi1998discrete2}
\nocite{kobayashi2015program}
\small
\begin{thebibliography}{99}
\expandafter\ifx\csname urlstyle\endcsname\relax
  \providecommand{\doi}[1]{doi:\discretionary{}{}{}#1}\else
  \providecommand{\doi}{doi:\discretionary{}{}{}\begingroup
  \urlstyle{rm}\Url}\fi

\bibitem{bernstein2004estimates}
J.~Bernstein and A.~Reznikov.
\newblock Estimates of automorphic functions.
\newblock \emph{{\normalfont Mosc. Math. J}}, \textbf{\textbf{4}}, (2004),
  pp.~19--37.

  \bibitem{clerc2011generalized}
J.-L. Clerc, T.~Kobayashi, B.~{\O}rsted and M.~Pevzner.
\newblock Generalized {B}ernstein--{R}eznikov integrals.
\newblock \emph{{\normalfont Math.~Ann.}}, \textbf{349}, (2011),.
\href{http://dx.doi.org/10.1007/s00208-010-0516-4}{pp.~395--431}.

\bibitem{howe1993homogeneous}
R.~E. Howe and E.-C. Tan.
\newblock Homogeneous functions on light cones: the infinitesimal structure of
  some degenerate principal series representations.
\newblock \emph{{\normalfont Bull.~Amer.~Math.~Soc.}}, \textbf{28},
  (1993), pp.~1--74.

\bibitem{juhl2009families}
A.~Juhl.
\newblock \emph{Families of {C}onformally {C}ovariant {D}ifferential
  {O}perators, {Q}-curvature and {H}olography}, \emph{{\normalfont Progr.~ Math.},} \textbf{275},
\newblock Birkh{\"a}user (2009).
\newblock ISBN 978-3-7643-9900-9.

\bibitem{kobyashi1992singular}
T.~Kobayashi.
\newblock \emph{Singular unitary representations and discrete series for indefinite Stiefel manifolds $U (p, q; \mathbb{F})/U (p-m, q; \mathbb{F})$},
Mem. Amer. Soc., \textbf{\href{http://www.ams.org/bookstore-getitem/item=MEMO-95-462}{462}}, Amer. Math. Soc., (1992).

\bibitem{kobayashi93restriction}
\newblock
T.~Kobayashi. \emph{The restriction of $A_q \left( \lambda \right)$ to reductive subgroups},
Proc. Japan Acad. Ser. A Math. Sci. 69 (1993), no. 7, 262--267.

\bibitem{kobayashi1998discrete2}
T.~Kobayashi.
\newblock Discrete decomposability of the restriction of {$A_q(\lambda)$} with
  respect to reductive subgroups {II}: Micro-local analysis and asymptotic
  {K}-support.
  \newblock \emph{{\normalfont Ann. Math. (2)}}, \textbf{147}, (1998),
\href{http://dx.doi.org/10.2307/120963}{pp.~709--729}.

\bibitem{kobayashi1998discrete3}
T.~Kobayashi.
\newblock Discrete decomposability of the restriction of {$A_q(\lambda)$} with
  respect to reductive subgroups {III}. {R}estriction of {H}arish-{C}handra
  modules and associated varieties.
\newblock \emph{{\normalfont Invent. Math.}}, \textbf{131}, (1998), 
\href{http://dx.doi.org/10.1007/s002220050203}{pp.~229--256}.

\bibitem{Kobayashi2014}
T.~Kobayashi.
\newblock {S}hintani functions, real spherical manifolds, and
  symmetry breaking operators.
  \newblock \emph{{\normalfont Dev.~Math.}}, \textbf{37}, (2014),
 \href{http://dx.doi.org/10.4171/OWR/2014/3}{pp.~127--159}.

\bibitem{kobayashi2015program}
T.~Kobayashi.
\newblock A program for branching problems in the representation theory of real
  reductive groups.
\newblock \emph{{\normalfont Progr.~Math.}}, \textbf{312}, (2015), 
\href{http://dx.doi.org/10.1007/978-3-319-23443-4_10}{pp.~277--322}.
\newblock In: \emph{{\normalfont Special issue in honor of Vogan's 60th years
  birthday}}.

\bibitem{kokupe2016forms}
T.~Kobayashi, T.~Kubo, and M.~Pevzner,
\newblock 
Conformal symmetry breaking operators for anti-de Sitter spaces.
preprint, 
\href{https://arxiv.org/abs/1610.09475}{arXiv:1610.09475}.

\bibitem{kobayashi2014classification}
T.~Kobayashi and T.~Matsuki.
\newblock Classification of finite-multiplicity symmetric pairs.
\newblock \emph{{\normalfont Transformation Groups}}, \textbf{19}, (2014),
\href{http://dx.doi.org/10.1007/s00031-014-9265-x}{pp.~457--493}.
\newblock In: \emph{{\normalfont Special Issue in honour of Dynkin
  for his 90th birthday}}.


  \bibitem{KO1}
T.~Kobayashi and B.~{\O}rsted.
\newblock Analysis on the minimal representation of\/ {${\rm
  O}(p,q)$}.{\;}{{\rm{I}}. Realization via conformal geometry}.
\newblock \emph{\normalfont Adv.~Math.}, \textbf{180}, (2003),
\href{http://dx.doi.org/10.1016/S0001-8708(03)00012-4}{pp.~486--512}.

  \bibitem{KO2}
T.~Kobayashi and B.~{\O}rsted.
\newblock Analysis on the minimal representation of\/ {${\rm O}(p,q)$}.{\;}{{\rm{II}}}. {B}ranching laws.
\newblock \emph{\normalfont Adv.~Math.}, \textbf{180}, (2003),
\href{http://dx.doi.org/10.1016/S0001-8708(03)00013-6}{pp.~513--550}.

\bibitem{kobayashi2015branching}
T.~Kobayashi, B.~{\O}rsted, P.~Somberg and V.~Sou{\v{c}}ek.
\newblock Branching laws for verma modules and applications in parabolic
  geometry. {I}.
\newblock \emph{{\normalfont Adv.~Math.}}, \textbf{285}, (2015),
\href{http://dx.doi.org/10.1016/j.aim.2015.08.020}{pp.~1796--1852}.

\bibitem{kobayashi2013finite}
T.~Kobayashi and T.~Oshima.
\newblock Finite multiplicity theorems for induction and restriction.
\newblock \emph{{\normalfont Adv.~Math.}}, \textbf{248}, (2013), 
 \href{http://dx.doi.org/10.1016/j.aim.2013.07.015}{pp.~921--944}.

\bibitem{kobayashi2016differential1}
T.~Kobayashi and M.~Pevzner.
\newblock Differential symmetry breaking operators: I. {G}eneral theory and
  {F}-method.
\newblock \emph{{\normalfont Selecta Math.}}, \textbf{22}, (2016),
\href{http://dx.doi.org/10.1007/s00029-015-0207-9}{pp.~801--845}.

\bibitem{kobayashi2015symmetry}
T.~Kobayashi and B.~Speh.
\newblock \emph{Symmetry {B}reaking for {R}epresentations of {R}ank {O}ne
  {O}rthogonal {G}roups}, \emph{{\normalfont Memoirs of the Amer.~Math.~Soc},}
  \textbf{\href{http://dx.doi.org/10.1090/memo/1126}{238}}, (2015).
\newblock ISBN 978-1-4704-1922-6.

\bibitem{vogan1984unitarizability}
D.~Vogan. 
\newblock Unitarizability of Certain Series of Representations.
\newblock Ann. of Math., \textbf{120}, (1984), \href{www.jstor.org/stable/2007074}{pp.~141–-187}.


\bibitem{wallach1988real2}
N.~Wallach.
\newblock \emph{Real Reductive Groups II}, \emph{{\normalfont Pure and Applied
  Mathematics},} \textbf{132},
\newblock Academic {P}ress (1992).
\newblock ISBN 978-0127329611.

\end{thebibliography}
\end{document}
%%y=\nu-n'2 <=> \nu=(n-1)/2+y
%%x=n/2-\lambda <=> \lambda=n/2-x
%%(lambda,nu)\in \\ <=> y-x+1/2\in -2N
%%(lambda,nu)\in //<=> 1/2-x-y\in -2N
%%(n-lambda,nu)\in //<=> 1/2+x-y\in -2N

