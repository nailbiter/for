\documentclass[12pt]{article} % use larger type; default would be 10pt

%%\usepackage[T1,T2A]{fontenc}
%%\usepackage[utf8]{inputenc}
%%\usepackage[english,ukrainian]{babel} % може бути декілька мов; остання з переліку діє по замовчуванню. 
\usepackage{amsthm}
\usepackage{changepage}
\usepackage{enumerate}
\usepackage{mystyle}
\usepackage{tabu}
\renewcommand{\setminus}{\ensuremath{{}-{}}}
\newcommand{\Le}{L_{\mbox{even}}}
\newcommand{\on}{\mathbf{1}_\nu}
\newtheorem{myfac}{Fact}
\begin{document}
\section{Definitions}
\begin{center}\begin{equation*} //:=\mycbra{\lambda-\nu=-2\Z_{\geq0}} \end{equation*}\end{center}
\begin{center}\begin{equation*} \backslash\backslash:=\mycbra{\lambda+\nu=n-1-2\Z_{\geq0}} \end{equation*}\end{center}
\begin{center}\begin{equation*} \mathbb{X}:=//\cap\backslash\backslash \end{equation*}\end{center}
\begin{center}\begin{equation*}\Le:=\mycbra{\lambda\leq\nu;\;\lambda,\nu\in-\Z_{\geq0};\;\lambda-\nu\in2\Z}\end{equation*}\end{center}
\begin{center}\begin{equation*}(\lambda,\nu)\in//\longrightarrow 2l=\nu-\lambda\end{equation*}\end{center}
\begin{center}\begin{equation*}(\lambda,\nu)\in\backslash\backslash\longrightarrow 2k=n-1-\nu-\lambda\end{equation*}\end{center}
\begin{center}\begin{equation*} m=n-1\end{equation*}\end{center}
$$\tilde{\mathbb{T}}_\nu\to J(-\nu+m)$$
$$(\tilde{\mathbb{T}}_\nu f)(y):=\frac{1}{\Gamma(\nu-\frac{m}{2})}\int_{\R^m}\myabs{x-y}^{2(\nu-m)}f(x)dx.$$
\section{Prerequisites}
\begin{adjustwidth}{-4cm}{}
{\scriptsize
\def\arraystretch{3.0}
\begin{tabu}{ |c | c | c | c |c |c |c|c|}
	\hline
		operator name &
		domain &
		$\supp=\R^n$ &
		$\supp=\R^{n-1}$ &
		$\supp=\mycbra{0}$ &
		$\supp=\emptyset$ &
		$\mathbf{1}_\lambda\mapsto$ &
		$K\in\mathcal{D}'(\R^n)$
	\\\hline
		$\tilde{\mathbb{A}}=\frac{\mathbb{A}}{\Gamma(\frac{\lambda+\nu-n+1}{2})\Gamma(\frac{\lambda-\nu}{2})}$ &
		$\C^2$ &
		otherwise\footnote[1]{Proposition 8.6, p.74} &
		$\backslash\backslash\setminus\mathbb{X}$\footnotemark[1]&
		$//\setminus L_{\mbox{even}}$ \footnotemark[1]& 
		$L_{\mbox{even}}$\footnotemark[1] &
		$\frac{\pi^{\frac{n-1}{2}}}{\Gamma(\lambda)}\on$ \footnote[9]{Proposition 7.4, p.63}&
		$\frac{\myabs{x_n}^{\lambda+\nu-n}(\mynorm{x}^2+x_n^2)^{-\nu}}
			{\Gamma(\frac{\lambda+\nu-n+1}{2})\Gamma(\frac{\lambda-\nu}{2})}$\footnote[10]{Expression (7.1), 
			p. 60}
	\\\hline
		$\tilde{\mathbb{B}}$ &
		$\backslash\backslash$ &
		$\emptyset$\footnote[2]{Proposition 9.8, p.81} &
		$\backslash\backslash\setminus\mathbb{X}$\footnotemark[2] &
		$\mathbb{X}\setminus L_{\mbox{even}}$\footnotemark[2]&
		$L_{\mbox{even}}$\footnote[7]{Theorem 9.1, p.77}&
		$\frac{(-1)^k2^k\pi^{\frac{n-1}{2}}(2k-1)!!}{\Gamma(\lambda)}\on$  \footnote[5]{Proposition 9.6, p.80}&
	\\\hline
		$\tilde{\mathbb{C}}$ &
		$//$ &
		$\emptyset$ &
		$\emptyset$ &
		$//$ &
		$\emptyset$\footnote[3]{Fact
			10.4 p.85 together with the observation that $\tilde{C}$ is the only differential operator on $\Le$}&
		$\frac{(-1)^l2^{2l}(\lambda)_{2l}}{l!}\on$ \footnote[8]{Proposition 10.7, p.87}&

	\\\hline
		$\tilde{\tilde{\mathbb{B}}}$ &
		$n$: odd, $\Le$ &
		$\emptyset$\footnote[4]{Proposition 9.9, p. 82} &
		$n$: odd, $\Le$ \footnotemark[4] &
		$\emptyset$\footnotemark[4] &
		$\emptyset$\footnotemark[4]&
		&
		$(\mynorm{x}^2+x_n^2)^{-\nu}\delta^{(2k)}(x_n)$\footnote[11]{Expression (9.10), p. 81}
	\\\hline
		$\tilde{\tilde{\mathbb{A}}}={\Gamma(\frac{\lambda-\nu}{2})}{\tilde{\mathbb{A}}}$&
		$(\C,-\N)$ &
		otherwise \footnote[6]{Proposition 8.9, p. 76} &
		$\backslash\backslash$\footnotemark[6]&
		$\emptyset$\footnotemark[6] &
		$\emptyset$\footnotemark[6]&
		$\frac{\pi^{\frac{n-1}{2}}\Gamma(\frac{\lambda-\nu}{2})}{\Gamma(\lambda)}\on$&
\\
		\hline
\end{tabu}
}
\end{adjustwidth}
\begin{myfac}[Theorem 8.5] The following equality holds on $(\lambda,\nu)\in\C^2$
	$$ \tilde{\mathbb{T}}_\nu\circ\tilde{\mathbb{A}}_{\lambda,\nu}=\frac{\pi^{m/2}}{\Gamma(\nu)}\tilde{\mathbb{A}}_{\lambda,
	m-\nu}$$
\end{myfac}
\begin{proof}(outline) Ininitially the identity is proven on some particular smaller parameter domain (once this is done, 
	we apply holomorphic continuation). On that domain it is known that dimension of symmetry breaking operator
	space is no more than one, and then equality is sufficient to be proven with both left- and right-hand sides applied
	to spherical constant vector. The remaining is computation (done with the 
	help of known formula for Fourier transform of $r^\nu$ and its
	analytic continuation), given that $\tilde{\mathbb{T}}_\nu(\mathbf{1}_\nu)=
	\pi^{m/2}/\Gamma(\nu)\on$ (Proposition 4.6).
\end{proof}
\begin{myfac}[Expression 4.27] The following equality holds
	$$ \tilde{\mathbb{T}}_\nu\circ\tilde{\mathbb{T}}_{m-\nu}=\frac{\pi^m}{\Gamma(m-\nu)\Gamma(\nu)}\mbox{id}. $$
\end{myfac}
\begin{proof}(outline) follows from the direct computation via the Fourier transform (see previous proof) which turns
	convolutions of kernels (resulting from composition) to products.
\end{proof}
\begin{myfac}[(4.22)] The following residue formula holds
	$$\frac{r^\nu}{\Gamma(\frac{\nu+m}{2})}\bigg|_{\nu=-m-2k}=\frac{(-1)^k\pi^{m/2}}{2^{2k}\Gamma(\frac{m}{2}+k)}\Delta^k_{\R^m}
	\delta(x)$$
\end{myfac}
As a corollary of a latter formula we have 
\begin{mycor}
for $j\in\Z_{\geq0}$
$$\tilde{\mathbb{T}}_{\frac{m}{2}-j}=\frac{(-1)^j\pi^{m/2}}{2^{2j}\Gamma(\frac{m}{2}+j)}\Delta^j_{\R^m}.$$
\end{mycor}
By taking $m=1$ and substituting $\lambda+\nu-n$ in place of $\nu$ we get
\begin{mycor}
The following equality holds for $\backslash\backslash$: 
		$$\frac{\myabs{t}^{\lambda+\nu-n}}{\Gamma(\frac{\lambda+\nu-n+1}{2})}=\frac{(-1)^k}{2^k(2k-1)!!}\delta^{(2k)}(t)$$
\end{mycor}
\end{document}
% (report seq, proof_schemes) --> <<do_prereq>> --> topdf
%prereq statements
	%Theorem 8.5
%problem equalities
	%t/Gamma = delta -- written
	% residue forumla 4.22 -- TODO
	%fourier of r^nu
%latex2html kspeh_04_16.tex -dir ~/public_html/kspeh --info 0 --no_footnode --numbered_footnotes --no_antialias_text -split 0 --no_navigation
