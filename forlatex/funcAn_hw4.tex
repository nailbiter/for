\documentclass[12pt]{article} % use larger type; default would be 10pt

\usepackage{mathtext}                 % підключення кирилиці у математичних формулах
                                          % (mathtext.sty входить в пакет t2).
\usepackage[T1,T2A]{fontenc}         % внутрішнє кодування шрифтів (може бути декілька);
                                          % вказане останнім діє по замовчуванню;
                                          % кириличне має співпадати з заданим в ukrhyph.tex.
\usepackage[utf8]{inputenc}       % кодування документа; замість cp866nav
                                          % може бути cp1251, koi8-u, macukr, iso88595, utf8.
\usepackage[english,russian,ukrainian]{babel} % національна локалізація; може бути декілька
                                          % мов; остання з переліку діє по замовчуванню. 

\usepackage{mystyle}
\newtheorem{prob}{Завдання}
\newcommand{\ds}{\;ds}
\newcommand{\dt}{\;dt}
\newcommand{\dx}{\;dx}
\newcommand{\dta}{\;d\tau}


\newtheorem{myulem}[mythm]{Лема}

\renewenvironment{myproof}[1][Доведення]{\begin{trivlist}
\item[\hskip \labelsep {\bfseries #1}]}{\myqed\end{trivlist}}

\title{Контрольна робота з функціонального аналізу (9 семестр)\\Вар. 1}
\author{Олексій Леонтьєв}

\begin{document}
\maketitle
\begin{prob}{\bf I3}{\normalfont .3)} Побудувати резольвенту інтегрального рівняння
	\[x(t)=\lambda\int_0^tK(t,s)x(s)\ds+y(t),\;t\in[0,T],\]
	та за її допомогою розв’язати рівняння у випадку
	\[K(t,s)=3^{t-s},\;\lambda=-1,\;y(t)=t3^t.\]
\end{prob}
Це рівняння Вольтерра другого роду ($K\in L_2$, адже воно неперервне, а відрізок компактний), і для нього,
як вказано в \cite[с. 37]{kukush} резольвента визначена відношеннями $R_{\lambda}(t,s)=\sum_{n=1}^\infty\lambda^{n-1}K_n(t,s)$, де
\[K_1(t,s)=K(t,s)=3^{t-s}\]
\[K_n(t,s)=\int_s^tK(t,\tau)K_{n-1}(\tau,s)\dta\]
Покажемо методом математичної індукцій, що $K_n(t,s)=3^{t-s}(t-s)^{n-1}/(n-1)!$. Дійсно, це так для $n=1$, а для більших $n$ маємо
\[K_n(t,s)=\int_s^tK(t,\tau)K_{n-1}(\tau,s)\dta=\int_s^t3^{t-\tau}\frac{(\tau-s)^{n-2}}{(n-2)!}3^{\tau-s}\dta=3^{t-s}\int_s^t
\frac{(\tau-s)^{n-2}}{(n-2)!}\dta=\]\[=3^{t-s}\int_0^{t-s}\frac{x^{n-2}}{(n-2)!}=3^{t-s}\frac{(t-s)^{n-1}}{(n-1)!}\]
Таким чином, резольвента рівна
\[R_{\lambda}(t,s)=\sum_{n=1}^\infty\lambda^{n-1}K_n(t,s)=3^{t-s}\sum_{n=0}^\infty \frac{\lambda^n(t-s)^n}{n!}=3^{t-s}e^{\lambda(t-s)}\]
а розв’язок із даними умовами записується як
\[x(t)=y(t)+\lambda\int_0^tR_{\lambda}(t,s)y(s)\ds=t3^t-\int_0^t3^{t-s}e^{s-t}s3^s\ds=t3^t-3^te^{-t}\int_0^tse^{s}\ds=\]\[=3^te^{-t}(e^t-1).\]

\begin{prob}{\bf I4}.{\normalfont 11)} Побудувати резольвенту інтегрального рівняння
	\[x(t)=\lambda\int_a^bK(t,s)x(s)\ds+y(t),\;t\in[a,b],\]
	та за її допомогою розв’язати рівняння у випадку
	\[K(t,s)=e^t\cos s,\;a=0,\;b=\pi,\;\lambda=\frac{1}{e^\pi+1},\;y(t)=\frac{t}{4}.\]
\end{prob}
Це інтегральне рівняння Фредгольма другого роду, і у відповідності із \cite[с. 37]{kukush},
резольвента визначена співвідношеннями $R_{\lambda}(t,s)=\sum_{n=1}^\infty\lambda^{n-1}K_n(t,s)$, де
\[K_1(t,s)=K(t,s)=e^t\cos s\]
\[K_n(t,s)=\int_0^\pi K(t,\tau)K_{n-1}(\tau,s)\dta\]
Покажемо методом математичної індукцій, що $K_n(t,s)=e^t\cos s((-e^\pi-1)/2)^{n-1}$. Дійсно, це так для $n=1$, а для більших $n$ маємо
\[K_n(t,s)=\int_0^\pi K(t,\tau)K_{n-1}(\tau,s)\dta=\int_0^\pi e^t\cos\tau e^\tau\cos s\mybra{-\frac{e^\pi+1}{2}}^{n-1}\dta=\]
\[=e^t\cos s\mybra{-\frac{e^\pi+1}{2}}^{n-1}\int_0^\pi \cos\tau e^\tau\dta=e^t\cos s\mybra{-\frac{e^\pi+1}{2}}^n\]
адже
\[\int_0^\pi \cos\tau e^\tau\dta
={\underbrace{\cos\tau e^\tau\bigg|_0^\pi}_{=-e^\pi-1}+\int_0^\pi e^\tau\sin\tau\dta}=
{-e^\pi-1+\underbrace{\sin\tau e^\tau\bigg|_0^\pi}_{=0}-\int_0^\pi e^\tau\cos\tau\dta}\implies\]
\[\implies 2\int_0^\pi \cos\tau e^\tau\dta=-e^\pi-1\implies\int_0^\pi \cos\tau e^\tau\dta=-\frac{e^\pi+1}{2}\]
і таким чином
\[R_{\lambda}(t,s)=\sum_{n=1}^\infty\lambda^{n-1}K_n(t,s)=e^t\cos s\sum_{n=0}^\infty\lambda^n\mybra{-\frac{e^\pi+1}{2}}^n=e^t\cos s
\mybra{1+\lambda\frac{e^\pi+1}{2}}^{-1}\]
а розв’язок із даними умовами записується як
\[x(t)=y(t)+\lambda\int_a^bR_{\lambda}(t,s)y(s)\ds=\frac{t}{4}+\frac{1}{e^\pi+1}\int_0^\pi \frac{s}{4}e^t\cos s \frac{2}{3}\ds=\]
\[=\frac{t}{4}+e^t\frac{1}{6(e^\pi+1)}\underbrace{\int_0^\pi s\cos s\ds}_{=-2}=
\frac{t}{4}-e^t\frac{1}{3(e^\pi+1)}\]
адже
\[\int_0^\pi s\cos s\ds=\underbrace{s\sin s\bigg|_0^\pi}_{=0}-\int_0^\pi\sin s\ds=\cos s\bigg|_0^\pi=-2\]

\begin{prob}{\bf I2}.{\normalfont 4)} Знайти всі значення параметрів $p,q$, при яких інтегральне рівняння 
	$x(t)=\lambda\int_a^bK(t,s)x(s)\ds+y(t),\;t\in[a,b]$,
	має розв’язок у просторі $L_2([a;b])$ для будь-яких $\lambda\in\mathbb{C}$:
	\[K(t,s)=\frac{ts}{2}+t^2s^2,\;y(t)=pt+q,t\in[a;b]=[-1;1];\]
\end{prob}
Ми скористаємося однією з теорем Фредгольма, \cite[Теорема 2.1, \S2, Глава IX]{tb}, яка стверджує, що рівняння матиме розв’язок для
даного $y(t)$ тоді і лише тоді, коли $y$ буде ортогональним до кожного розв’язку рівняння
\[x(t)=\lambda\int_{-1}^1K(s,t)x(s)\ds=\lambda\int_{-1}^1\mybra{\frac{st}{2}+s^2t^2}x(s)\ds\]
Оскільки ядро цього рівняння (як і оригінального) є виродженим, можемо підставити $x(t)=at+bt^2$, адже всі можливі розв’язки мають цю форму. Маємо
\[at+bt^2=\lambda\mybra{\frac{t}{2}\int_{-1}^1s(as+bs^2)\ds+t^2\int_{-1}^1s^2(as+bs^2)\ds}\]
\[at+bt^2=\lambda\mybra{\frac{t}{2}\cdot\frac{2a}{3}+t^2\frac{2b}{5}}\]
Єдині можливі нетривіальні розв’язки мають місце при $\lambda=3$ ($x(t)=at$) і $\lambda=5/2$ ($x(t)=bt^2$), і оскільки $y(t)$ має гарантувати
розв’язок для \textit{кожного} $\lambda\in\mathbb{C}$, необхідною і достатньою умовою є перпендикулярність до цих двох розв’язків, тобто
\[\begin{cases}
\int_{-1}^1(pt+q)t\dt=0\iff 2p/3=0\\
\int_{-1}^1(pt+q)t^2\dt=0\iff 2q/3=0\\
\end{cases}\]
і, таким чином, єдине можливе значення параметрів це $p=q=0$.

\begin{prob}{\bf I3}.{\normalfont 3)} Знайти всі $\lambda$, при яких інтегральне рівняння
	\[x(t)=\lambda\int_a^bK(t,s)x(s)\ds+y(t),\;t\in[a,b],\]
	має єдиний розв’язок для будь-якого $y\in C([a;b])$ у випадку:
	\[K(t,s)=\cos(2t-s),\;[a;b]=[0;2\pi]\]
\end{prob}
Ми скористаємося Теоремою 2.3 з \cite[\S2, Глава IX]{tb}, яка стверджує, що рівняння матиме єдиний розв’язок для кожного $y$ тоді і лише тоді,
коли відповідне однорідне рівняння матиме лише тривіальні розв’язки. Однорідне рівняння записується як
\[x(t)=\lambda\int_0^{2\pi}\cos(2t-s)x(s)\ds\]
Як і в попередньому прикладі, оскільки ядро однорідне, можемо одразу підставити $x(t)=a\cos2t+b\sin2t$
\[a\cos2t+b\sin2t=\lambda(a\cos2t\int_0^{2\pi}\cos s(a\cos2s+b\sin2s)\ds+b\sin2t\int_0^{2\pi}\sin s(a\cos2s+b\sin2s)\ds)=0\]
а тому для довільного $\lambda\in\mathbb{C}$ єдині розв’язки -- тривіальні, і рівняння має єдиний розв’язок для довільного $y\in C([a;b])$
і довільного $\lambda\in\mathbb{C}$.
\begin{thebibliography}{9}
\bibitem{kukush}
	О. Ю. Константiнов,
	О. Г. Кукуш, Ю. С. Мiшура, О. Н. Нестеренко, А. В. Чайковський. \emph{Збiрник задач з функцiонального аналiзу. Компактнi оператори. 
	Iнтегральнi рiвняння. Узагальненi функцiї.}
	К.:ВПЦ "Київський унiверситет", 2005. -- 126 с.
\bibitem{tb}
	Березанський Ю. М., Ус Г. Ф., Шефтель З. Г.
	Митропольський Ю. А., Самойленко А. М., Кулик В. Л.
	\emph{Функціональний аналіз}.
	Київ, "Вища школа"{}, 1990, російською мовою, 600 с.
\end{thebibliography}
\end{document}
