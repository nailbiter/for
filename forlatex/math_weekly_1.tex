\documentclass[12pt]{article} % use larger type; default would be 10pt

%\usepackage[utf8]{inputenc} % set input encoding (not needed with XeLaTeX)
\usepackage[10pt]{type1ec}          % use only 10pt fonts
\usepackage[T1]{fontenc}
%\usepackage{CJK}
\usepackage{graphicx}
\usepackage{float}
\usepackage{CJKutf8}
\usepackage{subfig}
\usepackage{amsmath}
\usepackage{amsfonts}
\usepackage{hyperref}
\usepackage{enumerate}
\usepackage{enumitem}
\usepackage{mystyle}

\begin{document}
I've figured that it might be a good idea to send You report about my weekly progress. In subsequent, I'll send them on Friday.
 Please, feel free to ignore them if You are busy, this is just an indicator and motivation for me. Otherwise, any of Your comments would be 
highly appreciated.

I've resumed my reading of Kobayashi-Speh paper. I've decided that it might be a good exercise to try to adapt the considerations of Your paper
to some other case. As of know, I'm thinking of $(G,G')=(SO(n,1),SO(n-1,1))$. I have the following plan 
\begin{enumerate}
\item Try to clearly formulate all the assumptions that You had in Your paper about the pair $(G,G')$;
\begin{itemize}
\item $G$ and $G'$ are both semi-simple
\item $G=P'N_-P$
\item $M'=M\cap G',\;A'=A\cap G',\;N'=N\cap G',\;P'=P\cap G'$ %should be ok once we are fixed by involution \theta
\end{itemize}
\item Check whether the pair $(G,G')=(SO(n,1),SO(n-1,1))$ satisfies them;
\item Find parabolic subgroup and $MAN$ decomposition for $G$ and $G'$;
\item Find real and complex flag varieties associated to $SO(n,1)$ (\textit{answer: $S^n$});
\item find the orbits of $P'\subset G'$ (parabolic subgroup of $G'$) on the flag variety of $G$ (we'll need this to classify
supports of kernel of operators, as in Kobayashi-Speh);
\item Find what these orbits correspond to under the parametrization of Bruhat cell by $\R^n$;
\item Set up the functional equations for kernels of symmetry breaking operators.
\end{enumerate}
Related to this plan, I have the following questions, which I'll try to answer myself:
\begin{enumerate}
\item Is that right that the decomposition $MAN$ above and parabolic subgroup can all be computed purely on Lie algebra level (as, say
 $\mbox{Lie}(P)=\mathfrak{p}$)? (\textit{answer: no, as, say $P\neq\exp(\mathfrak{p})$});
\item How were equations for operator kernels found? To what extent do they depend on the underlying $(G,G')$ pair?;
\item What method was used to solve the equations (\textit{NB: has to do with Kobayashi-Pevzner paper})?;
\item How do we use knowledge about possible operator kernels?;
\item How answer to step 6 above was found in Kobayashi-Speh?
\end{enumerate}
Finally, while thinking about this I've came up across a question which puzzles me. Suppose $\mathfrak{g}$ is some $n$-dimensional Lie algebra,
with basis $\mycbra{a_i}_{i=1}^n$ as a vector space and $n^3$ numbers $\mycbra{c_{ij}^k}_{i,j,k=1}^n$ are known, such that
\[\mysbra{a_i,a_j}=\sum_{k=1}^nc_{ij}^ka_k\]
Having them how can we:
\begin{enumerate}
\item Find out the center $\mathfrak{z(g)}$ (\textit{A: Yes. In $O(n^3)$});
\item Find whether $\mathfrak{g}$ is semi-simple $\mathfrak{z(g)}$ (\textit{A: Yes. Check the degeneracy of Killing form. In $O(n^4)$});
\item Find Cartan subalgebra (\textit{Yes: see \cite{graaf}});
\item Find $\mathfrak{m},\mathfrak{a},\mathfrak{n}$ of $MAN$ decomposition;
\item Find the root system of $\mathfrak{g}$?
\end{enumerate}
And as a practical question, how can one compute $\mycbra{c_{ij}^k}_{i,j,k}$ for $\mathfrak{so(n,1)}$ (\textit{done: $X=I_{m,n}X^TI_{m,n}$})?
\begin{thebibliography}{9}
\bibitem{graaf}
De Graaf, Willem and Ivanyos, Gábor and Rónyai, Lajos. Computing Cartan subalgebras of Lie algebras. Springer-Verlag. p. 339-349 1996
\url{ http://dx.doi.org/10.1007/BF01293593}
\end{thebibliography}
\end{document}
