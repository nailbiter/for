%        File: may10.tex
%     Created: 水  5 10 06:00 PM 2017 J
% Last Change: 水  5 10 06:00 PM 2017 J
%
\documentclass[a4paper,12pt]{article}
\usepackage{enumerate}
\usepackage{xeCJK}
\usepackage{ruby}
\usepackage[]{amsmath,amsthm,amssymb,mystyle}

\setCJKmainfont[AutoFakeBold=true]{Hiragino Mincho Pro} %my Mac

\newtheorem{theorem}{Theorem}
\newtheorem{remark}[theorem]{Remark}
\newtheorem{fact}[theorem]{Fact}
\newtheorem{proposition}[theorem]{Proposition}
\theoremstyle{definition}
\newtheorem{definition}[theorem]{Definition}

\title{Report on the Four Year Seminar}
\date{10 May 2017}
\begin{document}
\maketitle
\begin{definition}[\cite{bergeron2016spectrum}[2.2]]
	Let $F$ be a field and $a,b\in F$. We define\begin{equation*}
		D_{a,b}(F)=\left\{ x_0+x_1i+x_2j+x_3k\mid x_i\in F,\begin{array}[]{c}
		i^2=a,\\ j^2=b,\\ ij=k=-ji
	\end{array}\right\}
	\end{equation*}
	For $x_0+x_1i+x_2j+x_3k\in D_{a,b}(F)$ we let\begin{equation*}
		N_{\mbox{\scriptsize red}}\left(  x_0+x_1i+x_2j+x_3k\right):=x_0^2-ax_1^2-bx_2^2+abx_3^2\quad\mbox{:reduced norm}
	\end{equation*}
\end{definition}
\begin{remark}
	When $F=\R$ we have the following:\begin{itemize}
		\item if at least one of $a,b$ is positive, $D_{a,b}(\R)$ is isomorphic to $M_2(\R)$;
		\item if $a,b<0$ $D_{a,b}(\R)$ is isomorphic to $\mathbb{H}$: Hamiltonian field.
	\end{itemize}
\end{remark}
\begin{definition}

\end{definition}
\begin{theorem}[main, \cite{bergeron2016spectrum}[Theorem 2.3, 3.]]
	
\end{theorem}
\bibliographystyle{alpha}
\bibliography{bibliography}
\end{document}
