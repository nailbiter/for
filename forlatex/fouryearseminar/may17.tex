%        File: may17.tex
%     Created: 水  5 17 06:00 PM 2017 J
% Last Change: 水  5 17 06:00 PM 2017 J
%
\documentclass[a4paper,12pt]{article}
\usepackage{enumerate}
\usepackage{xeCJK}
\usepackage{ruby}
\usepackage[]{amsmath,amsthm,amssymb,mystyle}

\setCJKmainfont[AutoFakeBold=true]{Hiragino Mincho Pro} %my Mac

\newtheorem{theorem}{Theorem}
\newtheorem{remark}{Remark}
\newtheorem{fact}{Fact}
\newtheorem{proposition}{Proposition}
\theoremstyle{definition}
\newtheorem{definition}{Definition}

\renewcommand{\implies}{\Rightarrow}

\title{Report on the Four Year Seminar}
\date{17 May 2017}
\begin{document}
\maketitle
\begin{definition}[{\cite[2.2]{bergeron2016spectrum}}]
	Let $F$ be a field and $a,b\in F$. We define\begin{equation*}
		D_{a,b}(F)=\left\{ x_0+x_1i+x_2j+x_3k\mid x_i\in F,\begin{array}[]{c}
		i^2=a,\\ j^2=b,\\ ij=k=-ji
	\end{array}\right\}
	\end{equation*}
	For $x_0+x_1i+x_2j+x_3k\in D_{a,b}(F)$ we let\begin{equation*}
		N_{\mbox{\scriptsize red}}\left(  x_0+x_1i+x_2j+x_3k\right):=x_0^2-ax_1^2-bx_2^2+abx_3^2\quad\mbox{:reduced norm}
	\end{equation*}
\end{definition}
\begin{remark}
	When $F=\R$ we have the following:\begin{itemize}
		\item if at least one of $a,b$ is positive, $D_{a,b}(\R)$ is isomorphic to $M_2(\R)$;
		\item if $a,b<0$ $D_{a,b}(\R)$ is isomorphic to $\mathbb{H}$: Hamiltonian field.
	\end{itemize}
\end{remark}
\begin{theorem}[main, {\cite[Thm. 2.3, 3.]{bergeron2016spectrum}}]
	The following are equivalent:
	\begin{enumerate}
		\item $\Gamma_{a,b}\subset G:=\mbox{\normalfont SL}_2(\R)$ is cocompact;\label{item1}
		\item $D_{a,b}(\Q)$ is divisible;\label{item2}
		\item one (or, equivalently, all) of the following holds:\label{item3}\begin{enumerate}
				\item $(0,0,0)$ is the only real root of $x^2-ay^2-bz^2=0$;\label{itema}
				\item $(0,0,0,0)$ is the only real root of $x^2-ay^2-bz^2+abw^2=0$; \label{itemb}
				\item $(0,0,0)$ is the only real root of $-ax^2-by^2+abz^2=0$.\label{itemc}
			\end{enumerate}
%%		\item There exists a group isomorphism from\begin{equation*}
%%				D_{a,b}(\R)^1:=\mbox{SL}(1,D_{a,b}(\R)):=\left\{ g\in D_{a,b}(\R)\mid N_{\mbox{\scriptsize reg}}(g)=1 \right\}
%%			\end{equation*}
%%			to $G:=\mbox{SL}(2,\R)$;
%%		\item Let $\Gamma_{a,b}$ be the image of $\mathcal{O}^1:=\mbox{SL}(1,D_{a,b}(\Z))$ under this isomorphism. Then $\Gamma_{a,b}$ is cocompact in $G$
%%			if any only if $D_{a,b}(\R)$ is a division algebra if and only if $(0,0,0)$ is the unique solution of $x^2-ay^2-bz^2=0$ over $\Z$.
	\end{enumerate}
\end{theorem}
\begin{remark}
	Note the following:\begin{itemize}
		\item the implications \ref{item2} $\implies$ \ref{item1}, \ref{item2} $\implies$ \ref{item3} are easy to see;
		\item implication \ref{item3} $\implies$ \ref{item2} follows by an explicit construction.
	\end{itemize}
\end{remark}

\underline{Last time} we have shown that the item \ref{itemc} implies the item \ref{itema}.

\underline{Today} we show that not \ref{itemc} implies not \ref{itema}.

\begin{definition}[{\cite[p.36]{bergeron2016spectrum}}]
	Two subgroups $\Gamma,\Gamma'$ of $G$ are said to be commensurable in $G$ if
\end{definition}<++>

\underline{Plan}:\begin{enumerate}
	\item assume the contrary of item \ref{itemc};
	\item show that $\Gamma_{a,b}$ and $\Gamma_{1,1}$ are commesurable;
\end{enumerate}<++>

\bibliographystyle{alpha}
\bibliography{bibliography}
\end{document}
