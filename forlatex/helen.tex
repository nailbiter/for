\documentclass[10pt]{article}

\usepackage{mathtext}                 % підключення кирилиці у математичних формулах
                                          % (mathtext.sty входить в пакет t2).
\usepackage[T1,T2A]{fontenc}         % внутрішнє кодування шрифтів (може бути декілька);
                                          % вказане останнім діє по замовчуванню;
                                          % кириличне має співпадати з заданим в ukrhyph.tex.
\usepackage[utf8]{inputenc}       % кодування документа; замість cp866nav
                                          % може бути cp1251, koi8-u, macukr, iso88595, utf8.
\usepackage[english,russian,ukrainian]{babel} % національна локалізація; може бути декілька
                                          % мов; остання з переліку діє по замовчуванню. 
\usepackage{amsthm}
\usepackage{amsmath}
\usepackage{amsfonts}
\usepackage{graphicx}
\usepackage[pdftex]{hyperref}
\usepackage{caption}
\usepackage{subfig}
\usepackage{fancyhdr}
\usepackage{cancel}
\usepackage{ulem}

\newtheorem{prob}{Завдання}
\newcommand{\ds}{\;ds}
\newcommand{\dt}{\;dt}
\newcommand{\dx}{\;dx}
\newcommand{\dta}{\;d\tau}

\usepackage{mystyle}

\newtheorem{myulem}[mythm]{Лема}

\renewenvironment{myproof}[1][Доведення]{\begin{trivlist}
\item[\hskip \labelsep {\bfseries #1}]}{\myqed\end{trivlist}}
\title{Контрольна робота з функціонального аналізу (9 семестр)\\Вар. 2}
\author{Олексій Леонтьєв}
\begin{document}
\maketitle
\setcounter{prob}{8}
\begin{prob}
	Знайти характеристичні числа, відповідні власні функції та розв’язки інтегрального рівняння
	\[x(t)=\lambda\int_{0}^{\pi}\sin(t+s) x(s)\ds+\sin t,\;t\in[0;\pi]\]
\end{prob}
	Почнемо з того, що знайдемо розв’язки інтегрального рівняння. Це рівняння з виродженим ядром, адже $\sin(t+s)=\sin t\cos s+
	\sin s\cos t$, а отже розв’язки вичерпуються функціями вигляду
	$x(t)=\lambda x_1\sin t+\lambda x_2\cos t+\sin t$, де $x_1$ та $x_2$ задовольняють
	\[\begin{bmatrix}
		\lambda\int_0^\pi \sin t\cos t\dt-1 & \lambda\int_0^\pi\cos t\cos t\dt\\
		\lambda\int_0^\pi \sin t\sin t\dt & \lambda\int_0^\pi\cos t\sin t\dt-1
	\end{bmatrix}
	\begin{bmatrix}x_1\\x_2\end{bmatrix}=\begin{bmatrix}-\int_0^\pi\sin t\cos t\dt\\
		-\int_0^\pi\sin t\sin t\dt\end{bmatrix}\]
	\[\begin{bmatrix}
		-1&\lambda\pi/2\\
		\lambda\pi/2&-1
	\end{bmatrix}
	\begin{bmatrix}x_1\\x_2\end{bmatrix}=\begin{bmatrix}0\\-\pi/2\end{bmatrix}\]
	Ця система не має розв’язків при $\lambda=\myfrac{2}{\pi}$ і єдиний розв’язок в інших випадках, що записується як
	\[x_1=\frac{\lambda\pi^2/4}{1-\lambda^2\pi^2/4},\;x_2=\frac{\pi/2}{1-\lambda^2\pi^2/4}\]
	Відповідно, інтегральне рівняння не має розв’язок при $\lambda=\myfrac{2}{\pi}$ і єдиний розв’язок в іншому випадку, що
	записується як
	\[x(t)=\frac{\lambda^2\pi^2/4}{1-\lambda^2\pi^2/4}\sin t+\frac{\lambda\pi/2}{1-\lambda^2\pi^2/4}\cos t+\sin t\]

	Відповідно розв’язки однорідного рівняння вичерпуються функціями вигляду
	$x(t)=\lambda x_1\sin t+\lambda x_2\cos t$, де $x_1$ та $x_2$ задовольняють
	\[\begin{bmatrix}
		-1&\lambda\pi/2\\
		\lambda\pi/2&-1
	\end{bmatrix}
	\begin{bmatrix}x_1\\x_2\end{bmatrix}=\begin{bmatrix}0\\0\end{bmatrix}\]
	Ця система (а отже і відповідне однорідне інтегральне рівняння) не має нетривіальних розв’язків при 
	$\lambda\neq\myfrac{2}{\pi}$ і розв’язки $a\begin{bmatrix}1&1\end{bmatrix}^T$ в іншому випадку.

	Таким чином, для інтегрального рівняння єдиним власним значенням є $\lambda=\myfrac{2}{\pi}$ і єдиною відповідною нормованою
	власною функцією є $x_0(t)=(\sin t+\cos t)/\sqrt{\pi}$.
\begin{prob}
	Довести, що функціонал є узагальненою функцією	\[f(\phi)=\int_{\mathbb{R}}e^{-2x}\phi'(x)\dx,\;\phi\in\mathcal{D}(\mathbb{R})\]
	Чи буде вона регулярною?
\end{prob}
\newcommand{\supp}{\mbox{supp }}
За означенням, нам треба показати, що $\phi\mapsto f(\phi)$ є лінійним неперервним функціоналом на $\mathcal{D}$. Лінійність
випливає з лінійності інтеграла (який завжди збіжний, адже $\supp \phi$ обмежена за означенням). Покажемо неперервність.
Нехай $\mathcal{D}(\mathbb{R})\ni\phi_n\to\phi$. За означенням
збіжності в $\mathcal{D}$, існує $r>0$, таке що для довільного
$n$ маємо $\supp\phi_n\subset\widetilde{B_r}(0)$ і таким чином, оскільки $\supp\phi'_n\subset\supp\phi_n$, маємо
\[f(\phi_n)=\int_{B_r(0)}e^{-2x}\phi'_n(x)\dx\to\int_{B_r(0)}e^{-2x}\phi'(x)\dx=\int_{\mathbb{R}}e^{-2x}\phi'(x)\dx\]
адже збіжність $\phi_n\to\phi$ є рівномірною на $B_r(0)$ за припущенням.

Щодо регулярності, якщо $\phi\in\mathcal{D}(\mathbb{R})$ і $\supp\phi'\subset\supp\phi\subset[-A,A]$, причому $\phi(\pm A)=0$, використовуючи
інтегрування частинами, маємо
\[f(\phi)=\int_{-A}^Ae^{-2x}\phi'(x)\dx=e^{-2x}\phi(x)\bigg|_{-A}^A-\int_{-A}^A(-2e^{-2x})\phi(x)\dx=2\int_{-A}^Ae^{-2x}\phi(x)\dx=
\int_\mathbb{R}2e^{-2x}\phi(x)\dx\]
і $f$ є регулярною загальною функцією.
\begin{prob}
	\[f(x)=\left\{\begin{array}{ll}\sin x,\;x>0\\-2,\;x<0\end{array}\right.,\;\mbox{$f',\;f''$--?{ в }$\mathcal{D}'(\mathbb{R})$}\]
\end{prob}
\begin{prob}
	Довести
	\[F[\frac{1}{2}(\delta_h+\delta_{-h})]=\frac{1}{\sqrt{2\pi}}\cos(hy),\;h\in\mathbb{R},\;\mbox{де}\]
	$F$ -- перетворення Фур’є
\end{prob}
Нехай $\psi\in\mathcal{S}(\mathbb{R})$. Оскільки, $\myabra{F[\frac{1}{2}(\delta_h+\delta_{-h})],\psi}$ визначено як
\[\myabra{F[\frac{1}{2}(\delta_h+\delta_{-h})],\psi}=\myabra{\frac{1}{2}(\delta_h+\delta_{-h}),F(\psi)}\]
нам треба показати, що
\[\myabra{\frac{1}{2}(\delta_h+\delta_{-h}),F(\psi)}=\myabra{\frac{1}{\sqrt{2\pi}}\cos(hy),\psi}\]
Оскільки $e^{iht}+e^{-iht}=2\cos(ht)$, маємо
\[\frac{1}{2}\cdot\frac{1}{\sqrt{2\pi}}{\int_{\mathbb{R}}\psi(t)(e^{iht}+e^{-iht})\dt}=\frac{1}{\sqrt{2\pi}}\int_{\mathbb{R}}\psi(t)
\cos(ht)\dt\]
Згадаємо, що $F(\psi)$ визначено як
\[F(\psi)(s)=\frac{1}{\sqrt{2\pi}}\int_{-\infty}^\infty \psi(t)e^{its}\dt\]
а отже
\[\myabra{\delta_h,F(\psi)}=F(\psi)(h)=\frac{1}{\sqrt{2\pi}}\int_{-\infty}^\infty \psi(t)e^{ith}\dt\]
і таким чином
\[\myabra{\frac{1}{2}(\delta_h+\delta_{-h}),F(\psi)}=\frac{1}{2}\cdot\frac{1}{\sqrt{2\pi}}{\int_{\mathbb{R}}\psi(t)(e^{iht}+e^{-iht})\dt
}=
\myabra{\frac{1}{\sqrt{2\pi}}\cos(hy),\psi}\]
\begin{thebibliography}{9}
\bibitem{tb}
Березанський Ю. М., Ус Г. Ф., Шефтель З. Г.
Митропольський Ю. А., Самойленко А. М., Кулик В. Л.
\emph{Функціональний аналіз}.
Київ, "Вища школа"{}, 1990, російською мовою, 600 с.
\end{thebibliography}
\end{document}
