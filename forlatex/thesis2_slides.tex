\documentclass{beamer}

\usepackage[utf8]{inputenc}
\usepackage[english, ukrainian]{babel}
\usepackage[T2A]{fontenc}
\usepackage{amssymb}
\usepackage{verbatim}

\beamertemplatetransparentcovereddynamic \beamertemplateballitem
\beamertemplatesolidbuttons
\mode<presentation>{\usetheme{Warsaw}}
\usepackage{mystyle}
\newcommand{\red}[1]{{\color[rgb]{0.6,0,0}#1}}

\newtheorem{thm}{Теорема}
\newtheorem{cor}{Наслідок}
\newtheorem{pro}{Твердження}
\newtheorem{lmm}{Лема}
\newtheorem{prop}{Твердження}
\newtheorem{defi}{Означення}
\newtheorem{rem}{Зауваження}
\newtheorem{ex}{Приклад}

\sloppy
\begin{document}
\setbeamertemplate{background canvas}{\includegraphics[width=\paperwidth,height=\paperheight]{clouds.jpg}}
\frame{
\mbox{}\\[5pt]
\begin{center}
{\LARGE \bf Назва роботи}\\[8pt]

{\bf Прізвище Ім'я} \\[7.5mm]


Київський Національний Університет імені Тараса Шевченка\\
механіко-математичний факультет\\[15mm]
\end{center}

\begin{flushright}
Науковий керівник -- доктор \\
фізико-математичних наук, професор \\
Прізвище, Ім'я, по-Батькові \vfill
\end{flushright}
}

\bigskip


\frame{ \frametitle{Заголовок слайду}

Далі можемо писати текст чи формули
$$x+y=y+z.$$

}

\frame{

Заголовок слайду потрібен не завжди. Іноді достатньо скористатись
властивостями характеристичної функції, та знайти моменти
випадкового процесу:
$$m_k=(-i)^kf^{(k)}(0).$$
Можливо, формула нам ще десь знадобиться, тоді можна її
пронумерувати:
\begin{equation}\label{equation_no_1}
m_k=(-i)^kf^{(k)}(0).
\end{equation} }

\frame{ \frametitle{Основний результат}

Ним може бути теорема:
\begin{thm}
Тут слід навести формулювання теореми, при необхідності
використати посилання на рівняння (\ref{equation_no_1}).
\end{thm}

Після успішного компілювання необхідно натиснути ``PDF TeXify'', в
результаті чого одержимо pdf-файл презентації. В Adobe Reader
повноекранний режим зручно вмикати сполученням клавіш Ctrl-L. Файл
презентації бажано назвати англійською у форматі Ваше Прізвище,
Ім'я.

}

\newpage
\mbox{}\\[55pt]
\begin{center}
\Huge {\bf Дякую за увагу! ;)}
\mypic{0.5}{everybody_sleeping.jpg}
\end{center}

\end{document}
