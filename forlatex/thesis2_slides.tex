\documentclass{beamer}

\usepackage[utf8]{inputenc}
\usepackage[english, ukrainian]{babel}
\usepackage[T2A]{fontenc}
\usepackage{amssymb}
\usepackage{verbatim}
\usepackage{color}
\usepackage{xcolor}
\usepackage{colortbl}
\usepackage{centernot}

\beamertemplatetransparentcovereddynamic \beamertemplateballitem
\beamertemplatesolidbuttons
\mode<presentation>{\usetheme{Warsaw}}
\usepackage{mystyle}
\newcommand{\red}[1]{{\color[rgb]{0.6,0,0}#1}}

\newtheorem{thm}{Теорема}
\newtheorem{cor}{Наслідок}
\newtheorem{pro}{Твердження}
\newtheorem{lmm}{Лема}
\newtheorem{prop}{Твердження}
\newtheorem{defi}{Означення}
\newtheorem{rem}{Зауваження}
\newtheorem{ex}{Приклад}

\sloppy
\begin{document}
\setbeamertemplate{background canvas}{\includegraphics[width=\paperwidth,height=\paperheight]{clouds.jpg}}
\frame{
\mbox{}\\[18pt]
\begin{center}
{\LARGE \bf Нові критерії грубості експоненціальної дихотомії в лінійних системах диференціальних рівнянь}\\[8pt]

{\bf Леонтьєв Олексій} \\[3.0mm]

Київський Національний Університет імені Тараса Шевченка\\
механіко-математичний факультет\\[8mm]
\end{center}

\begin{flushright}
Науковий керівник -- кандидат \\
фізико-математичних наук, асистент \\
Фекета Петро Володимирович\vfill
\end{flushright}
}

\bigskip

\frame{ \frametitle{Експоненціальна дихотомія. Означення}
Рівняння \begin{equation}\label{Homog}
	\dot{x}(t)=A(t)x(t)\end{equation} називається {\it рівнянням з (експоненціальною) дихотомією}, якщо $\mathbb{R}^n=E^+\oplus E^-$ і
\[\forall x^+\in E^+,\;\mynorm{\Omega_0^t(A)x^+}\leq K \mynorm{\Omega_0^\tau(A)x^+}\exp\mycbra{-\gamma(t-\tau)},\;\tau\leq t\]
\[\forall x^-\in E^-,\;\mynorm{\Omega_0^t(A)x^-}\leq K \mynorm{\Omega_0^\tau(A)x^-}\exp\mycbra{\gamma(t-\tau)},\;t\leq\tau\]
}
\frame{\frametitle{Поняття про грубість}
\center{\bf Ключові факти}
\begin{enumerate}
	\item Дихотомія на $\mathbb{R}^+$ ($\mathbb{R}^-$) {\it значно} грубша за дихотомію на $\mathbb{R}$.
	%\item Дихотомія на $\mathbb{R}^+$ та $\mathbb{R}^-\centernot\implies$ дихотомію на $\mathbb{R}$.
\end{enumerate}
}
\newcommand{\mygraycenteredcell}[1]{\multicolumn{1}{c|}{#1}}
\newcommand{\mygraycenteredcello}{}
\frame{\small\frametitle{Поняття про грубість}
\begin{center}
\begin{tabular}{ |l| p{0.30\textwidth} | p{0.30\textwidth} | p{0.30\textwidth}| }\hline
	\mygraycenteredcello&\multicolumn{2}{c|}{Критерії дихотомії}&\mygraycenteredcello\\\hline
&\mygraycenteredcell{Необхідні}&\mygraycenteredcell{Достатні}&\mygraycenteredcell{Критерії грубості}\\\hline\hline

$\mathbb{R}^+$&
&
\textbf{iff} $BC^1(\mathbb{R})\ni x\mapsto \dot{x}-Ax\in BC(\mathbb{R})$ є оператором Фредгольма. \cite{palmer88}&
$\mynorm{B}_\infty<\delta$ \cite{coppel} або $\mynorm{B}_1<\infty$\cite{coppel}\\\hline

&
&
\textbf{iff}$x'=Ax+f$ має принаймні один обмежений розв’язок для кожної обмеженої  $f$. \cite{coppel}&
$\lim_{t\to+\infty}\mynorm{B(t)}=0$\cite{coppel}\\\hline
\end{tabular}
\end{center}
}
\frame{\small\frametitle{Поняття про грубість}
\begin{center}
\begin{tabular}{ |l| p{0.30\textwidth} | p{0.30\textwidth} | p{0.30\textwidth}| }\hline
	\mygraycenteredcello&\multicolumn{2}{c|}{Критерії дихотомії}&\mygraycenteredcello\\\hline
$\mathbb{R}$&
$x'=Ax+f$ має обмежений розв’язок для кожної $f$ неперервної та обмеженої на $\mathbb{R}$.&
\textbf{iff}\qquad Дихотомія на $\mathbb{R}^+$ та $\mathbb{R}^-$, а також $\mathbb{R}^n$ є сумою стійкого і нестійкого підпросторів.
	\cite[Proposition 2.1]{palmer84} &
	$\mynorm{B}_\infty<\delta$\cite{coppel}\\\hline
&
&
\textbf{iff}існує симетрична форма $S(t)$, така що похідна $V(t,x):=\mysca{S(t)x}{x}$ в силу системи
$\mysca{\mysbra{\dot{S}+SA+A^*S}x}{x}\leq-\beta\mynorm{x}^2,\;\beta>0$, \cite[Твердження 1.2]{mitrop}&
\\\hline
\end{tabular}
\end{center}
}
\frame{\frametitle{Поняття про грубість}
\center{\bf Ключові факти}
\begin{enumerate}
	\item Дихотомія на $\mathbb{R}^+$ ($\mathbb{R}^-$) {\it значно} грубша за дихотомію на $\mathbb{R}$.
\pause\begin{center}Чому це відбувається??\end{center}
	\pause\item Дихотомія на $\mathbb{R}^+$ та $\mathbb{R}^-\centernot\implies$ дихотомію на $\mathbb{R}$.
	\item Дихотомія на $[a,+\infty)\implies$ дихотомію на $\mathbb{R}^+$.
\end{enumerate}
}
\frame{\frametitle{Дихотомія на $\mathbb{R}^+$ та $\mathbb{R}^-\centernot\implies$ дихотомію на $\mathbb{R}$}
\mypic{1.0}{thesis2_slides_decomp.png}
\[\text{Дихотомія на }\mathbb{R}\iff\mathbb{R}^n=E_{\text{о}}^+\oplus E_{\text{о}}^-\]
\pause Таким чином, два типи перешкод:
\pause\begin{enumerate}
	\item $E_{\text{о}}^+\oplus E_{\text{о}}^-\neq\mycbra{0}\iff$ обмежені на $\mathbb{R}$ нетривіальні розв’язки рівняння (\ref{Homog}).
		\pause\item $E_{\text{о}}^++ E_{\text{о}}^-\subsetneq\mathbb{R}^n\iff$ необмежені на $\mathbb{R}^{+/-}$ розв’язки рівняння
	(\ref{Homog}).
\end{enumerate}
\pause{\bf Зауваження.} Якщо відомі розмірності $E_{\text{о}}^{+/-}$, однієї перешкоди достатньо.
}
\frame{\frametitle{Питання}
\begin{center}{\it Чи можна сформулювати більш широкі критерії грубості (можливо для спеціальних класів $A(t)$)?}\end{center}
}
\frame{\frametitle{Мотивуючий приклад}
\begin{ex}\[\dot{x}(t)=\mybra{\begin{bmatrix}-1&0\\0&-1\end{bmatrix}+\underbrace{B(t)}_{\lim_{\myabs{t}\to\infty} B(t)=0}}x(t)\]
Асимптотична стійкість не змінюється при збуреннях, що прямують до 0.
\end{ex}
}
\frame{\frametitle{Напрямок 1 (функціональний аналіз)}
\begin{thm}Дихотомія на $\mathbb{R}\iff\forall f(t)\in BC,\;\exists! x(t)\in BC$, розв’язок \[\dot{x}(t)=A(t)x(t)+f(t)\]
\end{thm}
}
\frame{\frametitle{Напрямок 1 (функціональний аналіз)}
\begin{cor}Дихотомія на $\mathbb{R}\iff$ оператор
	\[BC^1\ni x\mapsto \dot{x}-Ax\in BC\]
	є ізоморфізмом Банахових просторів.
\end{cor}
}
\frame{\frametitle{Напрямок 2}
\begin{thm}[Prop. 2, p. 35 з \cite{coppel}]
	Експоненціальна дихотомія на $\mathbb{R}^+$ зберігається у випадку
	\[\int_0^\infty\mynorm{B(t)}\;dt<+\infty\]
	причому розмірність простору $E^+$ не змінюється.
\end{thm}
}
\frame{\frametitle{Напрямок 2}
\begin{thm}
	Дихотомія на $\mathbb{R}$ зберігається у випадку
	\[\int_{-\infty}^\infty\mynorm{B(t)}\;dt<+\infty\]
	якщо $\forall t\in\mathbb{R}$, $\int_0^tA(s)\;ds$ та $\int_0^tB(s)\;ds$ комутують.
\end{thm}
}

\frame{
\mbox{}\\[12pt]
\begin{center}
\Huge {\bf Дякую за увагу! ;)}\\[5pt]
\mypic{0.5}{everybody_sleeping.jpg}
\end{center}
}
\frame{\frametitle{Література}\scriptsize\mbox{}
\begin{thebibliography}{9}
\bibitem{krein}
Ю. Л. Далецкий, М. Г. Крейн
\emph{Устойчивость решений дифференциальных уравнений в банаховом пространстве}.
Видавництво "Наука"{}, Головна редакція фізико-математичної літератури. Москва, 1970.
\bibitem{mitrop}
Митропольський Ю. А., Самойленко А. М., Кулик В. Л.
\emph{Исследование дихотомии линейных систем дифференциальных уравнений с помощью функций Ляпунова}.
АН УССР. Ін-т математики. Київ, "Наукова думка", 1990.
\bibitem{palmer84}
	Kenneth J. Palmer, {\em Exponential Dichotomies and Transversal Homoclinic Points} (1983).
\bibitem{palmer88}
	Kenneth J. Palmer, {\em Exponential Dichotomies and Fredholm Operators} (1988).
\bibitem{naulin}
	Ra\'ul Naulin and Manuel Pinto, {\em Admissible perturbations of Exponential Dichotomy Roughness} (1997).
\bibitem{coppel}
	W. A. Coppel, {\em Dichotomies in Stability Theory} (1978).
\end{thebibliography}
}

\end{document}
