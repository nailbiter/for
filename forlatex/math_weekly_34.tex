\documentclass[12pt]{article} % use larger type; default would be 10pt

%\usepackage[utf8]{inputenc} % set input encoding (not needed with XeLaTeX)
\usepackage[10pt]{type1ec}          % use only 10pt fonts
\usepackage[T1]{fontenc}
\usepackage{graphicx}
\usepackage{float}
\usepackage{subfig}
\usepackage{amsmath}
\usepackage{amsfonts}
\usepackage{hyperref}
\usepackage{enumerate}
\usepackage{enumitem}
\usepackage{mystyle}

\newtheorem*{fact}{Fact}

\begin{document}
Our seminar with Taito and Naoya will commence from this Friday, I guess. We are hoping to start with Chapter II of Knapp.

As You remember, I've decided that it might be a good exercise to try to adapt the considerations of Your paper
to some other case. As of know, I'm thinking of $(G,G')=(SO(n,1),SO(n-1,1))$. I had the following plan, which I've
outlined in my previous email. Now, Tanaka-san kindly told me some fact (please, correct me, if I restate it wrongly), which I have yet
to fully understand. Nevertheless, it resolves some of questions I had.
\begin{fact}
	If $G'\subset $ are two linear Lie groups, both fixed by involution $\theta: g\mapsto {^tg^{-1}}$, then for $M,A,N$ and $M',A',N'$
	decompositions of $G$ and $G'$ respectively we have $M'=G'\cap M$, $A'=A\cap G'$, $M'=G'\cap M$ and the same for $P'=P\cap G'$.
\end{fact}
\begin{enumerate}
\item Try to clearly formulate all the assumptions that You had in Your paper about the pair $(G,G')$;
\begin{itemize}
	\item $G$ and $G'$ are both semi-simple,{\it true, as mentioned in Knapp.}
\item $G=P'N_-P$
\item $M'=M\cap G',\;A'=A\cap G',\;N'=N\cap G',\;P'=P\cap G'$ {\it true, by the fact}
\end{itemize}
\item Check whether the pair $(G,G')=(SO(n,1),SO(n-1,1))$ satisfies them;
\item Find parabolic subgroup and $MAN$ decomposition for $G$ and $G'$; {\it done by the fact}
\item Find real and complex flag varieties associated to $SO(n,1)$ (\textit{answer: $S^n$});
\item find the orbits of $P'\subset G'$ (parabolic subgroup of $G'$) on the flag variety of $G$ (we'll need this to classify
supports of kernel of operators, as in Kobayashi-Speh); {\it decomposition is the same, applying the techniques of Chapter 5 of
Kobayashi-Speh }
\item Find what these orbits correspond to under the parametrization of Bruhat cell by $\R^n$;
\item Set up the functional equations for kernels of symmetry breaking operators.
\end{enumerate}
Related to this plan, I have the following questions, which I'll try to answer myself:
\begin{enumerate}
\item How were equations for operator kernels found? To what extent do they depend on the underlying $(G,G')$ pair?;
\item What method was used to solve the equations (\textit{NB: has to do with Kobayashi-Pevzner paper})?;
\item How $P'-$invariant sets of flag variety $G/P$ project to the Bruhat cell $\simeq\R^n$?;
\item How do we use knowledge about possible operator kernels?;
\end{enumerate}
\begin{thebibliography}{9}
\bibitem{graaf}
De Graaf, Willem and Ivanyos, Gábor and Rónyai, Lajos. Computing Cartan subalgebras of Lie algebras. Springer-Verlag. p. 339-349 1996
\url{ http://dx.doi.org/10.1007/BF01293593}
\end{thebibliography}
\end{document}
