\documentclass[10pt]{article} % use larger type; default would be 10pt

\newcommand{\norm}[1]{\left|\left|#1\right|\right|}
\usepackage{mystyle}
\usepackage{enumerate}
\usepackage{CJKutf8}

%custom theorem environments
\newtheorem{definition}{Definition}[section]
\renewcommand{\thedefinition}{\arabic{definition}}
\newtheorem{example}{\indent Example}[section]
\renewcommand{\theexample}{\arabic{example}}
\newtheorem{exercise}{Exercise}
\newtheorem{theorem}{Theorem}
\newtheorem{lemma}{Lemma}
\newtheorem{observation}{Observation}
\newtheorem*{fact}{Fact}
\newtheorem{proposition}{Proposition}
\newtheorem{corollary}[proposition]{Corollary}
\theoremstyle{remark}
\newtheorem{remark}{Remark}

\title{45901-112, 数物先端科学IV\\Final Report}
%45901-111 数物先端科学Ⅲ
\author{Alex Leontiev, 45-146044}
\begin{document}
\begin{CJK}{UTF8}{bsmi}
\maketitle
\end{CJK}
\tableofcontents
\section{Jones Polynomial of Torus Knot}
In the first part of the report we will be dealing with closed formula for Jones polynomial of $(p,q)$ torus knot. The proof below
closely follows Chapters 13 and 14 of \cite{Lickorish}. Many attempts to streamline the proof were made. Therefore, I've tried to omit the statements
(and proofs) of all the results, which were not used.

My main goal was to 
clearly explain to myself the method of proof, so I naturally start from where I feel comfortable with. Naturally, some proofs are omitted
if I felt myself comfortable with these results, otherwise exposition would become much, much bigger.
\subsection{Preliminary results}
\subsubsection{Knot and planar diagram}
\begin{definition}A link $L$ of $m$ components is a subset of $S^3$ or $\mathbb{R}^3$, that consist of $m$ disjoint, piecewise linear, simple
closed curves. A link of one component is called a knot.\end{definition}
\mypicwtitle{0.3}{3fd_pics/link_ex.png}{Link with 3 components.}
Here and in subsequent we consider outer space $\mathbb{R}^3$ or $S^3$ as simplicial complex, so to work, following \cite{lickorish},
in piecewise linear category, rather than in $C^1$.
\begin{definition} Links $L_1$ and $L_2$ are equivalent if there is an orientation-preserving piecewise linear homeomorphism $h:S^3\to S^3$
such that $h(L_1)=L_2$.
\end{definition}
We shall assume, that any link can be brought to an equivalent one, which is in general position w.r.t. standard projection $p:\mathbb{R}^3\mapsto
\mathbb{R}^2$. This means that projection of any two segments intersect in at most one point which for disjoint segments is not an end point
and that no point of plane belongs to projections of three segments.
\begin{definition}
For a link, which is in general position w.r.t. standard projection $p:\mathbb{R}^3\mapsto\mathbb{R}^2$, its projection on $\mathbb{R}^2$
together with information about under- and overcrossings, will be called the link diagram.
\end{definition}
\mypicwtitle{0.3}{3fd_pics/planar.png}{Planar diagram of a trefoil knot.}
It is apparent that any of the following moves, called Reidemeister moves, change a diagram of link to a diagram of an equivalent link.
\mypicwtitle{0.5}{3fd_pics/rmoves.png}{Reidemeister moves}
It takes on a proof, however, to see that \textit{any} two link diagrams of two equivalent knots can be connected via a sequence of 
orientation-preserving plane homomorphisms and Reidemeister moves.
\subsubsection{Torus knot}

\subsubsection{Jones Polnomial}
%TODO
\begin{definition}uoeoeu\end{definition}
\subsubsection{Algebras}
%TODO
\subsection{Main Theorem}
%TODO
\subsubsection{Statement}
%TODO
\subsubsection{Proof}
%TODO
\section{Irreducible modules of $\mathfrak{sl}(2,\mathbb{R})$}
\begin{thebibliography}{9}
\bibitem{lickorish}
Lickorish, W.B.R.. An Introduction to Knot Theory. Springer New York. 1997
\end{thebibliography}
\end{document}
% 6--> 3 --> 4
