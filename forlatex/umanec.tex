\documentclass[12pt]{article} % use larger type; default would be 10pt

\usepackage{mathtext}                 % підключення кирилиці у математичних формулах
                                          % (mathtext.sty входить в пакет t2).
\usepackage[T1,T2A]{fontenc}         % внутрішнє кодування шрифтів (може бути декілька);
                                          % вказане останнім діє по замовчуванню;
                                          % кириличне має співпадати з заданим в ukrhyph.tex.
\usepackage[utf8]{inputenc}       % кодування документа; замість cp866nav
                                          % може бути cp1251, koi8-u, macukr, iso88595, utf8.
\usepackage[english,russian,ukrainian]{babel} % національна локалізація; може бути декілька
                                          % мов; остання з переліку діє по замовчуванню. 
\usepackage{amsthm}
\usepackage{amsmath}
\usepackage{amsfonts}
\usepackage{graphicx}
\usepackage[pdftex]{hyperref}
\usepackage{caption}
\usepackage{subfig}
\usepackage{fancyhdr}
\usepackage{cancel}

\newtheorem{prob}{Завдання}
\newcommand{\ds}{\;ds}
\newcommand{\dt}{\;dt}
\newcommand{\dx}{\;dx}
\newcommand{\dta}{\;d\tau}
\let\oldint\int
\renewcommand{\int}{\oldint\limits}
\let\phi\varphi

\usepackage{mystyle}

\newtheorem{myulem}[mythm]{Лема}

\renewenvironment{myproof}[1][Доведення]{\begin{trivlist}
\item[\hskip \labelsep {\bfseries #1}]}{\myqed\end{trivlist}}

\title{Контрольна робота з функціонального аналізу (9 семестр)\\Вар. 1}
\author{Олексій Леонтьєв}

\begin{document}
\maketitle
\begin{prob}Знайти оператор, спряжений до $A:l_2\mapsto l_2$\[Ax=(\frac{1}{2}x_2,x_3,x_4,\dots)\]\end{prob}
	Нагадаємо, що нам потрібно знайти (єдиний, за теоремою Ріса) оператор, що для всіх $x,y\in l_2$ задовольняв би рівності
	\[\mysca{Ax}{y}=\mysca{x}{A^*y}\]
	Ми стверджуємо, що $A^*:l_2\mapsto l_2$ заданий як
	\[A^*y=(0,\frac{1}{2}y_1,y_2,\dots)\]
	задовольняє цій умові. Це легко перевірити, адже $A^*$ є неперервним оператором на $l_2$ і до того ж ми маємо (всі суми нижче збіжні)
	\[\mysca{Ax}{y}=
	\frac{1}{2}x_2\overline{y_1}+x_3\overline{y_2}+x_4\overline{y_3}+\dots=
	x_1\cdot 0+x_2\cdot\overline{\frac{1}{2}y_1}+x_3\overline{y_2}+x_4\overline{y_3}+\dots=\]
	\[\mysca{(x_1,x_2,x_3,\dots)}{(0,\frac{1}{2}y_1,y_2,y_3,\dots)}=\mysca{x}{A^*y}\]
\begin{prob}Довести, що оператор $A:B\mapsto B$ є скінченновимірним, де $B=C[0,\pi]$, $(Ax)(t)=\int_0^{\pi}\sin(t+\tau)x(\tau)d\tau,\;t\in[0,\pi]$.
	Чи буде $A$ компактним оператором?\end{prob}
	Помітимо, що
	\[\forall x\in B,\;(Ax)(t)=\int_0^{\pi}\sin(t+\tau)x(\tau)d\tau=\int_0^{\pi}\left(\sin t\cos\tau+\cos t\sin\tau\right)x(\tau)d\tau=\]
	\[\sin t\int_0^{\pi}\cos\tau x(\tau)d\tau+\cos t\int_0^{\pi}\sin\tau x(\tau)d\tau\in \left<\left\{\sin t,\cos t\right\}\right>\]
	Оскільки $\left<\left\{\sin t,\cos t\right\}\right>:=\mysetn{\alpha\sin t+\beta\cos t}{\alpha,\beta\in\mathbb{C}}$ є скінченновимірним
	підпростором $B$ (розмірності 2, адже $\sin t$ і $\cos t$ лінійно незалежні в $B$)
	, скінченновимірним є і множина значень $A$, таким чином останній є скінченновимірним.\\
	Більше того, $A$ є компактним оператором. Дійсно, якщо $X\subset B$ - обмежена множина, $A(X)\subset
	\left<\left\{\sin t,\cos t\right\}\right>$ також буде обмеженою в $\left<\left\{\sin t,\cos t\right\}\right>\simeq \mathbb{C}^2$
	, через обмеженість оператора $A$, а отже замикання $A(X)$ в $B$ буде міститися в $\mathbb{C}^2\simeq
	\left<\left\{\sin t,\cos t\right\}\right>$
	(адже останній є скінченновимірним підпростором $B$ і тому замкненою підмножиною $B$) і буде замкненою обмеженою множиною. Проте, 
	кожна замкнена обмежена множина в $\mathbb{C}^2\simeq
	\left<\left\{\sin t,\cos t\right\}\right>$ є компактною, а отже $A(X)$ буде прекомпактною в $B$ і тому $A$ - компактний оператор.
\begin{prob}Знайти спектр, власні числа, норму і спектральний радіус оператора $A:l_2\mapsto l_2$, $Ax=(x_1,x_1-x_2,x_3,x_4,\dots)$\end{prob}
\begin{prob}Знайти спектр, власні числа і власні функції оператора $A\in L(H)$, $H=L_2[0,2\pi]$, $(Ax)(t)=\int_0^{2\pi}\cos^2(t-\tau)x(\tau)
	d\tau,\;t\in[0,2\pi]$
\end{prob}
\begin{prob}За допомогою повторних ядер побудувати резольвенту інтегрального рівняння $x(t)=\lambda
	\displaystyle\int_{-1}^{1}e^{\tau-t}x(\tau)\;d\tau+
	y(t),\;t\in[-1;1]$, і знайти його розв’язок при $\lambda=1+e,\;y(t)=\sin\pi t$.\end{prob}
\begin{prob}Звести до системи алгебраїчних рівнянь і розв’язати інтегральне рівняння
	\[x(t)=\lambda\int_{-1}^1(t^2-t\tau)x(\tau)\;d\tau+y(t),\;t\in[-1;1],\;\mbox{при }y(t)=t^2+t,\;\lambda\in\mathbb{C}\]
\end{prob}
\begin{prob}За допомогою альтернативи Фредгольма знайти всі $\lambda\in\mathbb{C}$, при яких наступне інтегральне рівняння
	має єдиний розв’язок при всіх $y\in C[0,2\pi]$:
	\[x(t)=\lambda\int_0^{2\pi}\cos(2t-\tau)x(\tau)\;d\tau+y(t),\;t\in[0,2\pi]\]
\end{prob}
\begin{prob}
	Знайти характеристичні числа, відповідні власні функції та розв’язки інтегрального рівняння
	\[x(t)=\lambda\int_{-\pi}^{\pi}\sin t\sin\tau x(\tau)\;d\tau-\sin t+\cos t,\;t\in[-\pi,\pi]\]
\end{prob}
\begin{prob}
	Довести, що функціонал є узагальненою функцією	\[f(\phi)=\int_{\mathbb{R}}e^{-x}\phi'(x)\dx,\;\phi\in\mathcal{D}(\mathbb{R})\]
	Чи буде вона регулярною?
\end{prob}
\newcommand{\supp}{\mbox{supp }}
Лінійність очевидна, залишається лише довести неперервність. Нехай $\mathcal{D}(\mathbb{R})\ni\phi_n\to\phi$. За означенням
збіжності в $\mathcal{D}(\mathbb{R})$, $\exists r>0,\;
\forall n\;\widetilde{B_r}(0)\supset\supp\phi_n$ і таким чином, оскільки $\supp\phi'_n\subset\supp\phi_n$, маємо
\[f(\phi_n)=\int_{B_r(0)}e^{-x}\phi'_n(x)\dx\to\int_{B_r(0)}=\int_{B_r(0)}e^{-x}\phi'(x)\dx=\int_{\mathbb{R}}e^{-x}\phi'(x)\dx\]
за теоремою Лебега про граничний перехід під знаком інтеграла,
оскільки збіжність $\phi_n\to\phi$ є рівномірною на $B_r(0)$ за означенням збіжності в $\mathcal{D}(\mathbb{R})$.

Щодо регулярності, якщо $\phi\in\mathcal{D}(\mathbb{R})$ і $\supp\phi'\subset\supp\phi\subset[-A,A]$, причому $\phi(\pm A)=0$, використовуючи
інтегрування частинами, маємо
\[f(\phi)=\int_{-A}^Ae^{-x}\phi'(x)\dx=e^{-x}\phi(x)\bigg|_{-A}^A-\int_{-A}^A(-e^{-x})\phi(x)\dx=\int_{-A}^Ae^{-x}\phi(x)\dx=\int_\mathbb{R}
e^{-x}\phi(x)\dx\]
і $f$ є регулярною загальною функцією.
\begin{prob}\[\myabra{\frac{1}{2}(\delta_h+\delta_{-h}),F(\psi)}\]
	\[f(x)=\left\{\begin{array}{ll}\cos x,\;x\leq0\\1,\;x>0\end{array}\right.,\;\mbox{$f',\;f''$--?{ в }$\mathcal{D}'(\mathbb{R})$}\]
\end{prob}
\begin{prob}
Довести
\[F[\frac{1}{2i}(\delta_h-\delta_{-h})]=\frac{1}{\sqrt{2\pi}}\sin(hy),\;h\in\mathbb{R},\;\mbox{де}\]
$F$ -- перетворення Фур’є
\end{prob}
Нехай $\psi\in\mathcal{S}(\mathbb{R})$. Нам потрібно показати, що
\[\myabra{\frac{1}{2i}(\delta_h-\delta_{-h}),F(\psi)}=\myabra{\frac{1}{\sqrt{2\pi}}\sin(hy),\psi}\]
переписуючи ліву і праву частину, маємо
\[\frac{1}{2i}\cdot\frac{1}{\sqrt{2\pi}}{\int_{\mathbb{R}}\psi(t)(e^{iht}-e^{-iht})\dt}=\frac{1}{\sqrt{2\pi}}\int_{\mathbb{R}}\psi(t)
\sin(ht)\dt\]
оскільки $e^{iht}-e^{-iht}=2i\sin(ht)$, рівність доведено.
\begin{thebibliography}{9}
\bibitem{tb}
Березанський Ю. М., Ус Г. Ф., Шефтель З. Г.
Митропольський Ю. А., Самойленко А. М., Кулик В. Л.
\emph{Функціональний аналіз}.
Київ, "Вища школа"{}, 1990, російською мовою, 600 с.
\end{thebibliography}
\end{document}
