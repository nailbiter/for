\documentclass[8pt]{article} % use larger type; default would be 10pt

\usepackage[margin=1in]{geometry}
\usepackage{graphicx}
\usepackage{float}
\usepackage{subfig}
\usepackage{amsmath}
\usepackage{amsfonts}
\usepackage{hyperref}
\usepackage{enumitem}
\usepackage[neverdecrease]{paralist}
\usepackage{cancel}

\usepackage{mystyle}

\title{Math 1540\\University Mathematics for Financial Studies\\2013-14 Term 1\\Suggested solutions for\\
Sec. 14.4-14.6}
\begin{document}
\maketitle
\section{Section 14.4}
\begin{description}
	\item[\# 5.]{\begin{inparaenum}[\bfseries(a)]\item {\it express $dw/dt$ as a function of $t$, both by using
			the Chain Rule and by expressing $w$ in terms of $t$ and differentiating
			directly with respect to $t$. Then }\item{\it evaluate $dw/dt$ at the given value of $t$.} \end{inparaenum}
			\[\begin{array}{l}w=2ye^x-\ln z,\quad x=\ln(t^2+1),\quad y=\tan^{-1}t,\quad z=e^t;\\ t=1\end{array}\]
		\begin{enumerate}[\bfseries(a)]
			\item Chain Rule gives us
			\[\frac{dw}{dt}=\frac{\partial w}{\partial x}\frac{dx}{dt}+
			\frac{\partial w}{\partial z}\frac{dz}{dt}+\frac{\partial w}{\partial z}\frac{dz}{dt}=2ye^x\cdot
			\frac{2t}{t^2+1}+2e^x\frac{1}{1+t^2}-\frac{1}{z}\cdot e^t=\]
			\[=2\tan^{-1}(t)\cdot (t^2+1)\frac{2t}{t^2+1}+2({t^2+1})\frac{1}{{t^2+1}
			}-\frac{1}{e^t}\cdot e^t=4t\cdot\tan^{-1}(t)+1\]
			while the direct differentiation gives
			\[w(t)=2\tan^{-1}(t)\cdot
			(t^2+1)-t\implies\frac{dw}{dt}=2\tan^{-1}(t)\cdot 2t+2\cdot\frac{1}{t^2+1}\cdot(t^2+1)-1=
			4t\cdot\tan^{-1}(t)+1\]
			which is exactly the same.
			\item \[4t\cdot\tan^{-1}(t)+1\bigg|_{t=1}=4\cdot\frac{\pi}{4}+1=\pi+1\]
			as $\tan^{-1}(1)=\pi/4$.
		\end{enumerate}
		}
	\item[\# 7.]{\begin{inparaenum}[\bfseries(a)]\item {\it express $\partial z/\partial u$ and $\partial z/\partial v$
		both by using the Chain Rule and by expressing $z$ directly in terms of $u$ and $v$ before
		differentiating. Then }\item {\it evaluate $\partial z/\partial u$ and $\partial z/\partial v$ at the given point}
		$(u,v)$.\end{inparaenum}\[\begin{array}{l}z=4e^x\ln y,\quad x=\ln(u\cos v),\quad y=u\sin v;\\
		\;(u,v)=(2,\pi/4)\end{array}\]
	\begin{enumerate}[\bfseries(a)]
		\item Chain Rule gives us 
		\[\frac{\partial z}{\partial u}=\frac{\partial z}{\partial x}\frac{\partial x}{\partial u}+\frac{\partial z}{\partial
		y}\frac{\partial y}{\partial u}=4e^x\ln(y)\cdot\frac{1}{u\cos v}\cos v+\frac{4e^x}{y}\sin v=4u\cos(v)\cdot\ln(u\sin v)
		\cdot\frac{1}{u\cos v}+\frac{4u\cos v}{u\sin v}\sin v=\]\[=4\ln(u\sin v)+4\cos v\]
		\[\frac{\partial z}{\partial v}=\frac{\partial z}{\partial x}\frac{\partial x}{\partial v}+\frac{\partial z}{\partial
		y}\frac{\partial y}{\partial v}=4e^x\ln(y)\cdot\frac{1}{u\cos v}u(-\sin v)
		+\frac{4e^x}{y}u\cos v=\]\[=-4u\cos(v)\cdot\ln(u\sin v)
		\frac{1}{u\cos v}u\sin v+\frac{4u\cos v}{u\sin v}u\cos v=\]\[=-4u\sin(v)\ln(u\sin v)+4\frac{u\cos^2v}{\sin v}\]
		while direct differentiation gives
		\[z(u,v)=4u\cos(v)\ln(u\sin v)\]
		\[\frac{\partial z}{\partial u}=4\cos(v)\ln(u\sin v)+4\frac{u\cos v}{u\sin v}\sin v=4\cos(v)\ln(u\sin v)+4\cos v\]
		\[\frac{\partial z}{\partial v}=-4u\sin(v)\ln(u\sin v)+4\frac{u\cos v}{u\sin v}u\cos v=-4u\sin(v)\ln(u\sin v)+4
		\frac{u\cos^2 v}{\sin v}\]
		which is exactly the same
	\item \[\frac{\partial z}{\partial u}(2,\pi/4)=4\cos\mybra{\frac{\pi}{4}}\ln\mybra{2\sin\mybra{\frac{\pi}{4}}}
		+4\cos\mybra{\frac{\pi
		}{4}}=2\sqrt{2}\ln\mybra{\sqrt{2}}+2\sqrt{2}\approx 3.808685 \]
		\[\frac{\partial z}{\partial v}(2,\pi/4)=-4\cdot2\cdot\ln\mybra{2\sin\mybra{\frac{\pi}{4}}}+4\frac{2\cdot\cos\mybra{
		\frac{\pi}{4}}}{\sin\mybra{\frac{\pi}{4}}}=-8\ln\mybra{\sqrt{2}}+8\approx 5.227411\]
	\end{enumerate}
		}
	\item[\# 29.]{{\it Find the values of $\partial z/\partial x$ and $\partial z/\partial y$ at the point given}
		\[z^3-xy+yz+y^3-2=0,\quad (1,1,1)\]
		Following the Example 6 in Chapter 14.3 of a textbook, let $F(x,y,z)=z^3-xy+yz+y^3-2$ then
		\[F_x=-y,\quad F_y=-x+z+3y^2,\quad F_z=3z^2+y\]
		and as all partial derivative are continuous and $F_z(1,1,1)=4\neq 0$, Implicit Function Theorem gives us
		\[\frac{\partial z}{\partial x}=-\frac{F_x}{F_z}=-\frac{-y}{3z^2+y}=\frac{1}{4}\]
		\[\frac{\partial z}{\partial y}=-\frac{F_y}{F_z}=-\frac{-x+z+3y^2}{3z^2+y}=-\frac{3}{4}\]
		}
	\item[\# 42.]{{\it The lengths $a,$ $b,$ and $c$ of the edges of a rectangular box are changing with time. At the instant
		in question}, $a=1\mbox{ m},$ $b=1\mbox{ m},$ $c=3\mbox{ m},$ $da/dt=db/dt=1\mbox{ m/sec},$ and $dc/dt=-3\mbox{ m/sec
		}.${\it At what rates are the box's volume $V$ and surface area $S$ changing at that instant? Are the box's interior
		diagonals increasing in length or decreasing?\\}
		As \[V=abc,\; S=bc+ac+ab\] we have according to the Chain Rule
		\[\frac{dV}{dt}=
\frac{\partial V}{\partial a}\frac{da}{dt}+\frac{\partial V}{\partial b}\frac{db}{dt}+\frac{\partial V}{\partial c}\frac{dc}{dt}=
bc\cdot\frac{da}{dt}+ac\cdot\frac{db}{dt}+ab\cdot\frac{dc}{dt}=3+3-3=3\mbox{ m$^3$/sec}\]
	and
		\[\frac{dS}{dt}=
\frac{\partial S}{\partial a}\frac{da}{dt}+\frac{\partial S}{\partial b}\frac{db}{dt}+\frac{\partial S}{\partial c}\frac{dc}{dt}=
	(b+c)\cdot\frac{da}{dt}+(a+c)\cdot\frac{db}{dt}+(a+b)\cdot\frac{dc}{dt}=4+4-6=2\mbox{ m$^2$/sec}\]
	The length of the interior diagonal (in fact, there are 4 of them, but as box is rectangular, all 4 have the same length, so
	we can talk about \textit{"the"} diagonal) $D$ is given by \[D=\sqrt{a^2+b^2+c^2}\] and by Chain Rule
	\[\frac{dD}{dt}=
\frac{\partial D}{\partial a}\frac{da}{dt}+\frac{\partial D}{\partial b}\frac{db}{dt}+\frac{\partial D}{\partial c}\frac{dc}{dt}=
\frac{a}{\sqrt{a^2+b^2+c^2}}\frac{da}{dt}+
\frac{b}{\sqrt{a^2+b^2+c^2}}\frac{db}{dt}+
\frac{c}{\sqrt{a^2+b^2+c^2}}\frac{dc}{dt}=\]
\[=\frac{1}{\sqrt{11}}+\frac{1}{\sqrt{11}}+\frac{-9}{\sqrt{11}}=\frac{-7}{\sqrt{11}}<0\]
and hence interior diagonals decrease in length.
		}
\section{Section 14.5}
	\item[\# 17.]{{\it Find the derivative of the function at $P_0$ in the direction of $\mathbf{u}$.
		\[g(x,y,z)=3e^x\cos yz,\quad P_0(0,0,0),\quad \mathbf{u}=2\mathbf{i}+\mathbf{j}-2\mathbf{k}\]
		}
		As $g$ is differentiable on the whole $\mathbf{R}^3$, we have that the derivative of $g$ in the direction of $\mathbf{
		\tilde{u}}$ obeys the relation
		\[D_\mathbf{u}g(P_0)=(\nabla g)_{P_0}\cdot\mathbf{u}\]
		However, it should be born in mind that it holds only when $\mathbf{\tilde{u}}$ is unit vector. Hence, we should
		find $\mathbf{\tilde{u}}$ that has the same direction as $\mathbf{u}$, but has unit length. This is done via the 
		normalization \[\mathbf{\tilde{u}}=\frac{\mathbf{u}}{\myabs{\mathbf{u}}}\]
		
		\[D_\mathbf{\tilde{u}}g(P_0)=(\nabla g)_{P_0}\cdot\mathbf{\tilde{u}}
		\mybra{3\mathbf{i}}\cdot\frac{{2\mathbf{i}+\mathbf{j}-2\mathbf{k}}}{\myabs{2\mathbf{i}+\mathbf{j}-2\mathbf{k}}}
		=\frac{6}{3}=2\]
		}
	\item[\# 29.]{{\it Let $f(x,y)=x^2-xy+y^2-y$. Find the directions of $\mathbf{u}$ and the values of $D_{\mathbf{u}}f(1,-1)$
		for which}
		\begin{inparaenum}[\bfseries a.]
			\setlength{\tabcolsep}{15pt}
			\begin{tabular}{ll}
		\item {\it $D_{\mathbf{u}}f(1,-1)$ is largest}
		&\item {\it $D_{\mathbf{u}}f(1,-1)$ is smallest}\\
		\item {\it $D_{\mathbf{u}}f(1,-1)=0$}
		&\item {\it $D_{\mathbf{u}}f(1,-1)=4$}\\
		\item {\it $D_{\mathbf{u}}f(1,-1)=-3$}\\
		\end{tabular}
		\end{inparaenum}\\
		}
		Note, that $f(x,y)$ is differentiable on the whole $\mathbb{R}^2$ and hence formula
		$D_{\mathbf{u}}g(P_0)=(\nabla g)_{P_0}\cdot\mathbf{u}$ applies. As it holds only
		when $\myabs{\mathbf{u}}=1$ and 
		$D_{\mathbf{u}}f$ is independent of the length of $\mathbf{u}$ we will assume in subsequent that $\myabs{\mathbf{u}}
		=1$.
		\begin{enumerate}[\bfseries a.]
			\item From the properties of dot product,
				\[D_{\mathbf{u}}f\leq\myabs{D_{\mathbf{u}}f}=\myabs{\nabla f\cdot\mathbf{u}}\leq
				\myabs{\nabla f}\cdot\myabs{\mathbf{u}}=\myabs{\nabla f}\]
				with equality attained if and only if $\nabla f$ and $\mathbf{u}$ have the same direction. As
				\[\nabla f(1,-1)=\myabra{2x-y,-x+2y-1}\bigg|_{(1,-1)}=\myabra{3,-4}\]
				we may set \[\mathbf{u}=\myabra{\frac{3}{5},-\frac{4}{5}}\]
				and in this case
				\[D_{\mathbf{u}}f(1,-1)=\myabs{\nabla f(1,-1)}=5\]
			\item Similarly,
				\[D_{\mathbf{u}}f\geq-\myabs{D_{\mathbf{u}}f}=-\myabs{\nabla f\cdot\mathbf{u}}\geq-\myabs{\nabla f}
				\cdot\myabs{\mathbf{u}}=-\myabs{\nabla{f}}\]
				with equality attained if and only if $\nabla f$ and $\mathbf{u}$ have the opposite direction. As
				\[\nabla f(1,-1)=\myabra{2x-y,-x+2y-1}\bigg|_{(1,-1)}=\myabra{3,-4}\]
				we may set \[\mathbf{u}=\myabra{-\frac{3}{5},\frac{4}{5}}\]
				and in this case
				\[D_{\mathbf{u}}f(1,-1)=-\myabs{\nabla f(1,-1)}=-5\]
			\item Writing $\mathbf{u}=\myabra{u_x,u_y}$ we may write
				\[0=D_{\mathbf{u}}f=\nabla f\cdot\mathbf{u}=\myabra{3,-4}\cdot\myabra{u_x,u_y}=3u_x-4u_y\implies u_x=\frac{4}{3}u_y\]
				if we also require $u_x^2+u_y^2=1$ we have only two solutions possible
				\[\mathbf{u}=\myabra{\frac{4}{5},\frac{3}{5}}\]
				and
				\[\mathbf{u}=\myabra{-\frac{4}{5},-\frac{3}{5}}\]
				In both cases,
				\[D_{\mathbf{u}}f(1,-1)=0\]
			\item Similarly to previous subproblem, we have
				\[4=D_{\mathbf{u}}f=\nabla f\cdot\mathbf{u}=\myabra{3,-4}\cdot\myabra{u_x,u_y}=3u_x-4u_y
				\implies u_x=\frac{4}{3}+\frac{4}{3}u_y\]
				and as we also require $u_x^2+u_y^2=1$ we have
				\[1=\frac{16}{9}(1+u_y)^2+u_y^2=\frac{16}{9}+\frac{32}{9}u_y+\frac{25}{9}u_y^2\implies \frac{25}{9}u_y^2+\frac{32}
				{9}u_y+\frac{7}{9}=0\]
				Solving the quadratic equation above gives us two solutions: $u_y=-1$ and $u_y=-7/25$ and three possible values
				for $\mathbf{u}$ altogether
				\[\mathbf{u}=\myabra{0,-1},\quad \mathbf{u}=\myabra{-\frac{24}{25},-\frac{7}{25}},\mbox{ and }
				\mathbf{u}=\myabra{\frac{24}{25},-\frac{7}{25}}\]
				In all cases,
				\[D_{\mathbf{u}}f(1,-1)=4\]
			\item Similarly to above,
				\[-3=D_{\mathbf{u}}f=\nabla f\cdot\mathbf{u}=\myabra{3,-4}\cdot\myabra{u_x,u_y}=3u_x-4u_y\implies u_x=-1+\frac{4}{3}
				u_y\]
				and as we also require $u_x^2+u_y^2=1$ we have
				\[1=1-\frac{8}{3}u_y+\frac{16}{9}u_y^2\implies u_y\mybra{\frac{2}{3}u_y-1}=0\]
				thus we have two solutions: $u_y=0$ and $u_y=\frac{3}{2}$ and only 2 possible values for $\mathbf{u}$ (as
				$\myabs{3/2}>1$, it cannot be a component of a \textit{unit} vector)
				\[\mathbf{u}=\myabra{1,0},\mbox{ and }\mathbf{u}=\myabra{-1,0}\]
				In both cases,
				\[D_{\mathbf{u}}f(1,-1)=-3\]
		\end{enumerate}
	\item[\# 35.]{{\it The derivative of $f(x,y)$ at $P_0(1,2)$ in the direction of $\mathbf{i}+\mathbf{j}$ is $2\sqrt{2}$ and in the direction
		of $-2\mathbf{j}$ is $-3$. What is the derivative of $f$ in the direction of $-\mathbf{i}-2\mathbf{j}$? Give reasons for your answer.
		}\\
		\textbf{Note.} As posed, this problem clearly has {\it infinitely many solutions}, as the function
		\[f(1+x,2+y)=\left\{\begin{array}{ll}
			0, &(x,y)=(0,0)\\
			2\sqrt{2}\lambda, &(x,y)=\lambda\myabra{\cos\theta,\sin\theta},\;\lambda> 0,\;\theta=\frac{\pi}{4}\\
			-3\lambda, &(x,y)=\lambda\myabra{\cos\theta,\sin\theta},\;\lambda>0,\;\theta=\frac{3\pi}{2}\\
			c\lambda, &(x,y)=\lambda\myabra{\cos\theta,\sin\theta},\;\lambda>0,\;\theta=\theta_0:=
			\pi+\cos^{-1}\mybra{\frac{1}{\sqrt{5}}}\\
			\mybra{2\sqrt{2}t+c(1-t)}\cdot\lambda, &(x,y)=\lambda\myabra{\cos\theta,\sin\theta},\;\lambda> 0,\;
			\theta=\frac{\pi}{4}\cdot t+\theta_0(1-t),\;0<t<1\\
			\mybra{ct+(-3)(1-t)}\cdot\lambda, &(x,y)=\lambda\myabra{\cos\theta,\sin\theta},\;\lambda> 0,\;
			\theta=\theta_0\cdot t+\frac{3\pi}{2}(1-t),\;0<t<1\\
			\mybra{-3t+2\sqrt{2}(1-t)}\cdot\lambda, &(x,y)=\lambda\myabra{\cos\theta,\sin\theta},\;\lambda> 0,\;
			\theta=-\frac{\pi}{2}t+\frac{\pi}{4}(1-t),\;0<t<1
		\end{array}\right.\]
		This function is continuous on $\mathbb{R}^2$, satisfies statement of the problem, and its derivative at $P_0$ in the direction
		of $-\mathbf{i}-2\mathbf{j}$ is $c$, which can be taken to be arbitrary real number.\\
		Nevertheless, some firm statements can be made if we impose an additional assumption on $f$: the \textit{differentiability} on
		the open set containing $P_0(1,2)$. In this case, formula
		$D_{\mathbf{u}}g(P_0)=(\nabla g)_{P_0}\cdot\mathbf{u}$ applies if $\myabs{\mathbf{u}}=1$. Then,
		denoting $\mathbf{u_1}=\mathbf{i}+\mathbf{j}$ and $\mathbf{u_2}=-2\mathbf{j}$ we have $\myabs{\mathbf{u_1}}=\sqrt{2}$ and
		$\myabs{\mathbf{u_2}}=2$ and
		\[-\mathbf{i}-2\mathbf{j}=-\mathbf{u_1}+\frac{\mathbf{u_2}}{2}=-\sqrt{2}\frac{\mathbf{u_1}}{\sqrt{2}}+\frac{\mathbf{u_2}}{2}\]
		hence
		\[D_{\frac{-\mathbf{i}-2\mathbf{j}}{\sqrt{5}}} f(P_0)=(\nabla f)_{P_0}\cdot
		\frac{-\mathbf{i}-2\mathbf{j}}{\sqrt{5}}=\frac{1}{\sqrt{5}}\cdot(\nabla f)_{P_0}\cdot\mybra{-\sqrt{2}\frac{\mathbf{u_1}}
		{\sqrt{2}}+\frac{\mathbf{u_2}}{2}}=\]
		\[=\frac{1}{\sqrt{5}}\cdot\mybra{(\nabla f)_{P_0}\cdot\mybra{-\sqrt{2}\frac{\mathbf{u_1}}
		{\sqrt{2}}}
		+(\nabla f)_{P_0}\cdot\frac{\mathbf{u_2}}{2}}=\frac{1}{\sqrt{5}}\cdot\mybra{\mybra{-\sqrt{2}}\cdot
		D_{\frac{\mathbf{u_1}}{\sqrt{2}}}f(P_0)+
		D_{\frac{\mathbf{u_2}}{2}}f(P_0)
		}=\]
		\[=\frac{1}{\sqrt{5}}\cdot\mybra{(-\sqrt{2})\cdot2\sqrt{2}-3}=-\frac{7}{\sqrt{5}}\]
		}
\section{Section 14.6}
	\item[\# 5.]{{\it Find equations for the}
		\begin{enumerate}[\bfseries(a)]
			\item {\it tangent plane and}
			\item {\it normal line at the point $P_0$ on the given surface}
		\end{enumerate}
		\[\cos\pi x-x^2y+e^{xz}+yz=4,\quad P_0(0,1,2)\]
		\begin{enumerate}[\bfseries(a)]
			\item As surface is given as a level set $f(x,y,z)=4$ of a differentiable function $f(x,y,z)=
				\cos\pi x-x^2y+e^{xz}+yz=4$, the equation of tangent plane to $f(x,y,z)$ at $P_0$ can be written as
				\[f_x(P_0)(x-0)+f_y(P_0)(y-1)+f_z(P_0)(z-2)=0\]
				and evaluating partial derivatives, we get
				\[\mybra{-\pi\sin\pi x-2xy+ze^{xz}}\bigg|_{(0,1,2)}(x-0)+
				\mybra{-x^2+z}\bigg|_{(0,1,2)}(y-1)+
				\mybra{xe^{xz}+y}\bigg|_{(0,1,2)}(z-2)=0
				\]
				\[2x+2y+z=4\]
			\item Normal line is given as
				\[x=0+2t,\quad y=1+2t,\quad z=2+t\]
		\end{enumerate}
		}
	\item[\# 8.]{{\it Find equations for the}
		\begin{enumerate}[\bfseries(a)]
			\item {\it tangent plane and}
			\item {\it normal line at the point $P_0$ on the given surface}
		\end{enumerate}
		\[x^2+y^2-2xy-x+3y-z=-4,\quad P_0(2,-3,18)\]
		\begin{enumerate}[\bfseries(a)]
			\item Similarly to previous problem, since surface is given as a level set of differentiable function
				\[f_x(P_0)(x-0)+f_y(P_0)(y-1)+f_z(P_0)(z-2)=0\]
				and evaluating partial derivatives, we get
				\[\mybra{2x-2y-1}\bigg|_{(2,-3,18)}(x-2)+
				\mybra{2y-2x+3}\bigg|_{(2,-3,18)}(y+3)+
				\mybra{-1}\bigg|_{(2,-3,18)}(z-18)=0
				\]
				\[9x-7y-z=21\]
			\item Normal line is given as
				\[x=2+9t,\quad y=-3-7t,\quad z=18-t\]
		\end{enumerate}
		}
	\item[\# 15.]{{\it Find parametric equation for the line tangent to the curve of intersection of the surfaces at the given point.}
		\[\begin{array}{ll}
			\mbox{Surfaces:} &x^2+2y+2z=4,\quad y=1\\
			\mbox{Point:} &(1,1,1/2)
		\end{array}\]
		Following the textbook, the tangent line should be orthogonal to the normal lines of both surfaces at point $(1,1,1/2)$ and
		hence if we denote $f(x,y,z)=x^2+2y+2z$ and $g(x,y,z)=y$, tangent line should be parallel to
		\[\nabla f\bigg|_{(1,1,1/2)}\times \nabla g\bigg|_{(1,1,1/2)}=\mybra{2\mathbf{i}+2\mathbf{j}+2\mathbf{k}}\times
		\mathbf{j}=2\mathbf{k}-2\mathbf{i}\]
		Hence tangent line is given by equation
		\[x=1-t,\quad y=1,\quad z=1/2+t\]
		}
	\item[\# 17.]{{\it Find parametric equation for the line tangent to the curve of intersection of the surfaces at the given point.}
		\[\begin{array}{ll}
			\mbox{Surfaces:} &x^3+3x^2y^2+y^3+4xy-z^2=0,\\
			&x^2+y^2+z^2=11\\
			\mbox{Point:} &(1,1,3)
		\end{array}\]
	Exactly as in the previous problem, tangent line should be parallel to
		\[\nabla (x^3+3x^2y^2+y^3+4xy-z^2
		)\bigg|_{(1,1,3)}\times \nabla(x^2+y^2+z^2)\bigg|_{(1,1,3)}=\mybra{13\mathbf{i}+13\mathbf{j}-6\mathbf{k}}\times
		\mybra{2\mathbf{i}+2\mathbf{j}+6\mathbf{k}}=90\mathbf{i}-90\mathbf{j}\]
		and hence is given by equation
		\[x=1+t,\quad y=1-t,\quad z=3\]
		}
	\item[\# 23.]{{\it Suppose that the Celsius temperature at the point $(x,y)$ in the $xy$-plane is $T(x,y)=x\sin 2y$ and that distance in
		the $xy$-plane is measured in meters. A particle is moving {\textit clockwise} around the circle of radius 1 m centered at the origin
		at the constant rate of 2 m/sec.}
		\begin{enumerate}[\bfseries(a)]
			\item {\it How fast is the temperature experienced by the particle changing in degrees Celsius per meter at the point
				$P(1/2,\sqrt{3/2})$?}
			\item {\it How fast is the temperature experienced by the particle changing in degrees per Celsius per second at $P$?}
		\end{enumerate}
		First of all, we can describe the position of the particle at time $t$ as \[x(t)=\cos(-2t),\quad y(t)=\sin(-2t)\]
		(minus signs are to account for the fact that particle is moving clockwise)
		\begin{enumerate}[\bfseries(a)]
			\item The distance $s$ traveled by particle from time $0$ to time $t$ is given by $2t$. Now, temperature experienced by
				the particle can be expressed as the function $T_s=T_s(s)$ of distance travelled and subproblem
				essentially asks us to find $T_s'(s)$ for $s$ being distance travelled till point $P$. Now, by Chain Rule
				\[\frac{dT_s(s(t))}{dt}=T_s'(s(t))\frac{ds}{dt}=2T_s'(s(t))\implies \frac{1}{2}\frac{dT_s(s(t))}{dt}=\frac{1}{2}
				\frac{T(x(t),y(t))}{dt}=T_s'(s(t))\]
				by Chain Rule (this time, for function of two variables)
				\[\frac{T(x(t),y(t))}{dt}=\frac{\partial T}{\partial x}\frac{dx}{dt}+\frac{\partial T}{\partial y}\frac{dy}{dt}\]
				It remains essentially to find the right $t_0$, corresponding to the particle being in point $P$ (in fact, there
				are infinitely many $t_0$ that would do, but arbitrary can be chosen). In particular, $t_0:=-\pi/6$ would do.
				Then, \[T_s'(s(t_0))=\frac{1}{2}\mybra{
				\frac{\partial T}{\partial x}\bigg|_{(1/2,\sqrt{3/2})}\cdot\frac{dx}{dt}\bigg|_{t_0}+
				\frac{\partial T}{\partial y}\bigg|_{(1/2,\sqrt{3/2})}\cdot\frac{dy}{dt}\bigg|_{t_0}}=
				\frac{1}{2}\mybra{\sin\sqrt{6}\cdot\sqrt{6}-\cos\sqrt{6}}\]in degrees Celsius per meter.
			\item Similarly to previous subproblem, 
				\[\frac{dT_s}{dt}\bigg|_{t=t_0}=
				\frac{\partial T}{\partial x}\bigg|_{(1/2,\sqrt{3/2})}\cdot\frac{dx}{dt}\bigg|_{t_0}+
				\frac{\partial T}{\partial y}\bigg|_{(1/2,\sqrt{3/2})}\cdot\frac{dy}{dt}\bigg|_{t_0}=
				{\sin\sqrt{6}\cdot\sqrt{6}-\cos\sqrt{6}}\]in degrees Celsius per second.
		\end{enumerate}
		}
	\item[\# 29.]{{\it Find the linearization of the function at each point}\\
	\begin{inparaenum}[\bfseries a.]\begin{tabular}{lll}
		$f(x,y)=e^x\cos y$ at &\item $(0,0),$ &\item $(0,\pi/2)$
	\end{tabular}\end{inparaenum}\\\\
		Linearization, as defined in textbook, is computed as
		\[L(x,y)=f(x_0,y_0)+f_x(x_0,y_0)(x-x_0)+f_y(x_0,y_0)(y-y_0)\]
		Now \[f_x(x_0,y_0)=e^{x_0}\cos y_0\] \[f_y(x_0,y_0)=-e^{x_0}\sin y_0\]
		and hence the answers are
		\begin{enumerate}[\bfseries a.]
			\item \[L(x,y)=1+1\cdot(x-0)+0\cdot(y-0)=1+x\]
			\item \[L(x,y)=0+0\cdot(x-0)-1\cdot(y-\pi/2)=\pi/2-y\]
		\end{enumerate}
		}
\end{description}
\end{document}
