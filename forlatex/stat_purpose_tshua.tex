\documentclass[10pt]{article}
\usepackage{fontspec}
\usepackage{array, xcolor, lipsum, bibentry}
\usepackage[margin=4cm]{geometry}
\usepackage{sectsty} % Allows changing the font options for sections in a document
\usepackage{hyperref}
 
\title{\bfseries\Huge Oleksii Leontiev}
\author{inp9822058@cs.nctu.edu.tw}
\date{}
 
\definecolor{lightgray}{gray}{0.8}
\newcolumntype{L}{>{\raggedleft}p{0.2\textwidth}}
\newcolumntype{R}{p{0.8\textwidth}}
\newcommand\VRule{\color{lightgray}\vrule width 0.5pt}
 
%font configuration
\defaultfontfeatures{Mapping=tex-text}
\setromanfont[Ligatures={Common}, Numbers={OldStyle}, Variant=01]{Linux Libertine O} % Main text font
\sectionfont{\mdseries\upshape\Large} % Set font options for sections
\subsectionfont{\mdseries\scshape\normalsize} % Set font options for subsections
\subsubsectionfont{\mdseries\upshape\large} % Set font options for subsubsections
\chardef\&="E050 % Custom ampersand character

\title{Personal Statement}
\author{Applicant's Name: Oleksii Leontiev}
 
\begin{document}
\maketitle
I was interested in Science and Technology since high school. Computer Science, Physics, Chemistry and Mathematics have been
attracting me with the modest clearance of concepts and strict reasoning. Their structure seemed verifiable and their yield seemed useful.
My parents (two generations of my family are engineers) encouraged me on this path, and my middle and high school both were technically inclined.
However, my decision to pursue a degree in Mathematics was a bit sudden, even for me. It did not happen in a moment
, I hated and hardly borne through the calculations in middle school, but I was wholly rewarded in the last years of high school. I believe
my obsession with math started with my first Analysis textbook when I was 15.\\
Mathematics has somewhat a special place among other sciences as both a queen and a servant. On the one hand, it is as versatile as human imagination.
On the other hand, it lives in its own realm and therefore one does not have to make compromises. Thus said, I am not a proponent
of "pure" mathematics - I prefer to find mathematics in the world around me. This may be the reason why I came to Taiwan (National Chiao Tung
University) for my bachelor studies - University with the best Applied Math department in Taiwan.
\\Even before I came to the university, I knew that I still need to learn a lot.
And that's what I did in my first two years:
\begin{itemize}
	\item{I took classes in all of the fields of Mathematics I knew at that time.}
	\item{I wanted to know more, so I have applied for a distant study program in the university in my hometown (
\href{http://mechmat.univ.kiev.ua/e/}{National Taras Schevchenko University of Kyiv}
, Ukraine), which is especially
strong in pure mathematics and probability. It was not always easy to manage studying in two universities at the same time, but it was
worthwhile.}
	\item{
I also took a
double major in Computer Science, in order to see how Mathematics can be applied there.
		}
\end{itemize}
% talk more about my background
My devotion towards Math payed off. Despite the relatively high course-load due to the double-degree and distant study I was the first in my class
during my sophomore and junior years in NCTU. I also have been participating in various projects in order to understand better how Math skills
can be applied to problems from Computer Science and Physics (see my CV).
%start to talk about Geometric Imaging
My exploration of Computer Science was mostly in one direction: find the field where I can exploit my Mathematical skills to the highest extent,
while still doing something useful. 
%about their program
%talk about me and once more about background
Therefore, I am applying here with the hope that I will be considered good enough for admission.
\end{document}
