\documentclass[8pt]{article} % use larger type; default would be 10pt

\usepackage{mystyle}
\usepackage{enumerate}
\usepackage{CJKutf8}
\usepackage{mathtools}
\usepackage{amssymb}
\usepackage{upgreek}
\usepackage{ulem}

\newcommand{\sgn}{\mbox{\normalfont{sgn}}}
\newcommand{\Aut}{\mbox{\normalfont{Aut}}}
\newcommand{\Hom}{\mbox{\normalfont{Hom}}}
\newcommand{\End}{\mbox{\normalfont{End}}}
\newcommand{\Ad}{\mbox{\normalfont{Ad}}}
\newcommand{\Lie}{\mbox{\normalfont{Lie}}}
\newcommand{\tr}{\mbox{\normalfont{tr}}}
\newcommand{\diag}{\mbox{\normalfont{diag}}}
\providecommand{\F}{\mathcal{F}}
\newcommand{\myprobshort}[2]{\begin{myprob}[#1]#2\end{myprob}}
%\newcommand{\myprobshort}[2]{}

\title{45901-109 数物先端科学V\\Final Report}
\author{Alex Leontiev, 45-146044
\\Tokyo University, Graduate School of Mathematical Sciences, M1}
\begin{document}
\begin{CJK}{UTF8}{bsmi}
\maketitle
\end{CJK}
We will proceed straightly with solving exercises given during the course.
\myprobshort{April 7, Example  1.13}{ Let $G:=O(n)$ be acting on $V:=\C[x_1,\hdots,x_n]$ by $(g\cdot f)(x):=
f(g^{ -1 }x),\;\forall (g,f)\in G\times V$. Show that $V^G=\C[x_1^2+x_2^2+\hdots+x_n^2]$.}
We shall prove this via induction on $n$ -- number of variables. Base case ($n=1,\;G=\mycbra{ \pm1 }$) is clear. Indeed, $G$
does not change the degree when acts on monomial, so WLOG any invariant polynomial can be seen as homogeneous, that is as monomial.
Odd monomials cannot be invariant, as they're odd functions and thus are not invariant under $f(x)\mapsto f(-x)$, which is one
of $G$'s actions. Hence, any invariant polynomial can contain only even monomials, hence $V^G\subset\C[x_1^2]$. As inverse inclusion
is obvious, this ends the proof in the base case. Now, let's proceed with the case of general $n$, assuming statement holds for any
smaller number of variables.\par
Indeed, the $\supset$ part is obvious, so we just need to prove the inverse inclusion. Assume $f\in V^G$ and we want to show
that $f\in\C[x_1^2+x_2^2+\hdots+x_n^2]$. As $G$ does not change the degree when acts on monomial, we may assume $f$ is homogeneous.
Similarly to base case,
there can be no homogeneous invariant polynomials of odd degree, as for these we will have $f(-I\cdot x)=-f(x)$,
which, as $-I\in G$, makes $f$ non-invariant under $G$. Hence, we may assume degree of $f$ is $2k$.\par
Now, we can write $f=\sum_{i=0}^{ 2k }x_1^{ i }c_ip_i(x_2,x_3,\hdots,x_n)$. As $f$ has to be invariant under $x_1\mapsto -x_1$
substitution, we see (as such expansion of $f$ is unique) in the summation above all $c_i$ with odd indexes vanish and we can write
$f=\sum_{i=0}^{ k }x_1^{ 2i }c_ip_i(x_2,x_3,\hdots,x_n)$. Moreover, as $O(n-1)$ acting on variables $x_2,x_3,\hdots,x_n$
is subgroup of $O(n)$ we see that (as expansion is unique) $p_i$ has to be invariant under that subgroup and hence
by inductive assumption $f=\sum_{i=0}^{ k }x_1^{ 2i }c_i(x_2^2+x_3^2+\hdots+x_n^2)^{ k-i }$ which can also be written as
\begin{equation}\label{Prob1}
	f=\sum_{i=0}^{ k }x_1^{ 2i }c_i\sum_{ j=0 }^{ k-i }x_2^{ 2j }\binom{k-i}{j}(x_3^2+x_4^2+\hdots+x_n^2)^{ k-i }
\end{equation}. Now, as permutation
$x_1\leftrightarrow x_2$ is one of actions of $O(n)$ we should have
\[f=\sum_{ i=0 }^kc_ix_2^{ 2i }(x_1^2+x_3^2+x_4^2+\hdots+x_n^2)^{ k-i }=\sum_{ i=0 }^kc_ix_2^{ 2i }\sum_{ j=0 }^{ i-j }
\binom{ k-i }{ j }x_1^{ 2j }(x_3^2+x_4^2+\dots+x_n^2)^{ k-i-j }=\]
\[=\sum_{ j=0 }^kx_1^{ 2j }\sum_{ i=0 }^{ k-j }c_i\binom{ k-i }{ j }x_2^{ 2i }(x_3^2+x_4^2+\dots+x_n^2)^{ k-i-j }\]
and again as expansion is unique, comparing this with \ref{Prob1}
we see that we should have $c_j\binom{ k-j }{ i }=c_i\binom{ k-i }{j}$ or setting $i:=0$, $c_j=c_0\binom{ k }{ j }$. Hence we have
\[f=\sum_{i=0}^kc_0x_1^{ 2i }\binom{ k }{ i }(x_2^2+x_3^2+\hdots+x_n^2)^{ k-i }=c_0(x_1^2+x_2^2+\dots+x_n^2)^k\]
\myprobshort{April 7}{
	Suppose $\pi:G\to GL(V)$ is finitely-dimensional representation over the field $\F$. Let $V^\vee:=\Hom_\F(V,\F)$ and
	$\pi^\vee:G\ni g\mapsto\pi^\vee(g)\in V^\vee$ defined as $(\pi^\vee(g)v^\vee)(v):=v^\vee(\pi^{ -1 }(g)v)$. Show
that it defines the representation.}
Indeed, we just need to show that $\pi^\vee$ respects group structure (it will then imply that the image is in $GL(V^\vee)\subset
\End(V^\vee)$). Now, identity is mapped to identity, as $\forall(v,v^\vee)\in V\times V^\vee,\;
\pi^\vee(1)v^\vee(v):=v^\vee(\pi(1)^{ -1 }v)=v^\vee(v)$. Similarly, inverse is mapped to inverse, as $\forall(g,v,v^\vee)\in G\times
V\times V^\vee,\;\pi^\vee(g^{ -1 })\pi^\vee(g)v^\vee(v):=\pi^\vee(g)v^\vee(\pi(g)v)=v^\vee(\pi(g^{ -1 })\pi(g)v)=v^\vee(v)$.
Finally, multiplication is respected, as $\forall(g_1,g_2,v,v^\vee)\in G\times G\times V\times V^\vee,\;\pi^\vee(g_1)\pi^\vee(g_2)
v^\vee(v)=v^\vee(\pi(g_2^{ -1 })\pi(g_1^{ -1 })v)=v^\vee(\pi((g_1g_2)^{ -1 })v)=\pi(g_1g_2)v^\vee(v)$.
\myprobshort{April 7}{Show that $\phi:\R\to GL_n(\C)$: continuous homomorphism, then $\phi\in C^\omega$.}
Indeed, we shall show that $\phi(t)=\exp(tA)$ for some $A\in M_n(\C)$ and statement desired will follow from this, as matrix
exponent $\exp$ is analytic function. Before doing this, let us note that continuity here is necessary, as if it would
not be assumed, we could take ${ r_\alpha }$ be any basis of $\R$ over $\Q$ (existing by axiom of choice) and then as
any $r\in\R$ can be uniquely written as finite sum $r=\sum_\alpha r_\alpha q_\alpha,\;q_\alpha\in\Q$ and hence
$\phi(\sum_\alpha r_\alpha q_\alpha):=\sum_\alpha r_\alpha I\in GL_n(\C)$ will yield well defined homomorphism,
which is not even continuous (hence not $C^\omega)$, as its image is subset of $GL_n(\Q)\subset GL_n(\C)$, which is not connected.

Now, map $\exp:M_n(\C)\to GL_n(\C)$ is local diffeomorphism near $0\in M_n(\C)$, say $U\xrightarrow{ \widetilde{} }V$
and hence (as $\phi$ is continuous) we have that
there exists $t_0>0$ such that $\forall t\in\R:\myabs{t}\leq t_0,\;\phi(t)\in U$, hence we have well-defined function
$\psi:[-t_0,t_0]\to U\subset M_n(\C)$, such that $\exp(\psi(t))=\phi(t)$. Now, uniqueness tells us that
$\forall n\in\Z_{>0}$ we have $\exp(\psi(t_0)/n)=\phi(t_0/n)$, hence $\forall (m,n)\in \Z\times\Z_{>0}:\myabs{m}\leq n,\;
\psi((m/n)t_0)=(m/n)\psi(t_0)$, hence by continuity $\psi(t)=t\cdot\psi(t_0)\;\forall t\in[-1,1]$. Hence, $\phi(t)=\exp(t\cdot
\psi(t_0))$ for $\forall t\in[-t_0,t_0]$. Now, additivity of $\phi$ implies that equality should hold for all $t\in\R$ and this
ends the proof.
\myprobshort{April 7, Exercise 1.4.5}{Let $G$ be the group of upper-triangular $3\times3$ matrices, with 1 on diagonal divided
by subgroup of matrices of form $\left[\begin{smallmatrix}1&0&n\\0&1&0\\0&0&1\end{smallmatrix}\right]$
with all $n\in\Z$. Show that $G$ cannot be realized as a matrix group.}
We shall use $H$ to denote the group of upper-triangular $3\times3$ matrices, with 1 on diagonal and $N$ to denote the
subgroup of matrices of form $\left[\begin{smallmatrix}1&0&n\\0&1&0\\0&0&1\end{smallmatrix}\right]$.
We note that $G$ has non-trivial center, as say equivalence class of element
$\left[\begin{smallmatrix}1&0&1/2\\0&1&0\\0&0&1\end{smallmatrix}\right]$ is nontrivial and central in $G$. Hence, as we
are seeking to show that there is no monomorphism $G\to GL_n(\C)$, it's sufficient to prove the following claim: if 
$H\to GL_n(\C)$ is homomorphism with $N$ in kernel, then (it factors to $G\to GL_n(\C)$ homomorphism) $Z(G)$ is in kernel as well.
Indeed, assuming this is proven and $G\to GL_n(\C)$ is arbitrary
homomorphism, it induces $H\to GL_n(\C)$ homomorphism, where $N$ is in
kernel by construction, hence in original $G\to GL_n(\C)$ homomorphism, $Z(G)$ was in kernel, hence it could not be monomorphism. 
Hence, it's sufficient to prove the claim.\par
So let us assume that $\Sigma: H\to GL_n(\C)$ is a homomorphism with $N$ in kernel. Note that as $H$ is simply-connected,
$\exp:\Lie(H)\to H$ is a diffeomorphism.
Now, if $A\in H$ maps to central element
of $H$ this means that $\forall (X,t)\in \Lie(H)\times\R,\;\exp(t\cdot\Ad(A)X)=\exp(tX)N$ and as $N$ is discrete, it means that
$\exp(t\cdot\Ad(A)X)=\exp(tX)$ for small $t$ and taking derivative we see that $Ad(A)(X)=X$ for arbitrary $X\in\Lie(H)$,
hence (as $\exp$ is onto $H$) $A\in Z(H)$. Therefore, it's sufficient to show that $Z(H)$ (every element of which is of form
$\left[\begin{smallmatrix}1&0&z\\0&1&0\\0&0&1\end{smallmatrix}\right]$ with $z\in\R$) gets mapped to zero.\par
Every element of the center is exponent of the element of form $\left[\begin{smallmatrix}0&0&z\\0&1&0\\0&0&0\end{smallmatrix}\right]$.
Let's take arbitrary element $B$, corresponding to $z=z_0\neq0$ (if $z=0$, $\exp(B)=I$ is obviously in kernel)
of this latter form. We claim that it's sufficient to show that $\sigma(B)\in\mathfrak{gl}_n(\C)$
is nilpotent (here $\sigma:\Lie(H)\to\mathfrak{gl}_n(\C)$ is differential of $\Sigma$). Indeed, assuming this was shown
we should have $\Sigma(\exp(tB))-I=\exp(\sigma(tB))-I$ being polynomial in $t$. Moreover, as $\exp(nt_0B)\in N$ for $t_0:=1/z$,
this polynomial should vanish at $nt_0$, and hence should vanish for all $t\in\R$ and hence $\Sigma(\exp(B))=1$, which is what we
want to prove. Hence, it suffices to show that $\sigma(B)\in\mathfrak{gl}_n(\C)$ is nilpotent.\par
As $\C^n$ is direct sum of generalized eigenspaces of $\sigma(B)$, it suffices to show that $\sigma(B)$ is nilpotent when restricted
to any generalized eigenspace. 
So let $V_\lambda$ be the generalized eigenspace of $\sigma(B)$ corresponding to $\lambda\in\C$. We
note that $B-\lambda I$ is nilpotent on $V_\lambda$. Furthermore, if we let 
$A:=\left[\begin{smallmatrix}0&z&0\\0&0&0\\0&0&0\end{smallmatrix}\right]$ and 
$C:=\left[\begin{smallmatrix}0&0&0\\0&0&1\\0&0&0\end{smallmatrix}\right]$ be two elements of $\Lie(H)$ we have that
$[A,C]=B$. Hence, as $A,\;C$ both commute with $B$ ($B$ is central) so do $\sigma(A),\;\sigma(C)$ commute with $\sigma(B)$
and henceforth $\sigma(A),\;\sigma(C)$ both stabilize $V_\lambda$. However,
\[\lambda\cdot\dim V_\lambda=\tr(\sigma(B)\bigg|_{ V_\lambda })+\underbrace{\tr(\lambda I-\sigma(B)\bigg|_{ V_\lambda })}_{
	=0,\text{ as it is nilpotent}}=\tr\mybra{ \mysbra{ \sigma(A),\sigma(C) } \bigg|_{ V_\lambda }}=0\]
	as trace of commutator. Hence, all generalized eigenvalues are zero, $\sigma(B)$ is nilpotent and we are done.
\myprobshort{April 14}{Let $G\subset GL_n(\R)$ be matrix Lie group and $\Lie(G):=\mysetn{X\in M_n(\R)}{e^{ sX }\in G,\;\forall
s\in\R}$. Show that $\Lie(G)$ is a Lie algebra.}
Indeed, let's start by showing that $\Lie(G)$ is a vector space. The fact that $\forall(t,v)\in\R\times\Lie(G),\;t\cdot v\in
\Lie(G)$ is clear from definition and we'll show the additivity now. To show the latter it is sufficient to show
that \begin{equation}\label{Prob5}\forall A,B\in M_n(\C)\forall t\in\R
	,\quad\lim_{n\to\infty}\mybra{ \exp\frac{t}{n}A\cdot\exp\frac{ t }{ n }B
}^n=\exp\mybra{ t(A+B) }\end{equation}.
As then if $A,B\in\Lie(G)$ we have left-hand side belonging to $G$ for every fixed $n$ and (due to the assumed closedness
of $G\subset GL_n(\R)$) we have that right-hand side should also belong to $G$ for every $t\in\R$. In turn, in order
to prove \ref{Prob5} it suffices to show that $\exp(tX)\exp(tY)=\exp(t(X+Y)+O(t^2))$, as then $\lim_{n\to\infty}
(\exp((t/n)X)\cdot\exp((t/n)Y))^n=\lim_{ n\to\infty }\exp(t(X+Y)+nO(t^2/n^2))=\exp(t(X+Y))$. To see the latter,
note that $\exp(tX)\exp(tY)$ is $C^\infty$ in $t$ and hence (as $\exp$ is local diffeomorphism near 0),
for small $t$ we have $\exp(tX)\exp(tY)=\exp(Z(t))$ for well-defined smooth $Z:I\to M_n(\R)$. Now, taking derivative 
of left-hand side at $t=0$ we get (via the product rule and knowing that $(d/dt)\big|_{ t=0 }\exp(tX)=X$)
that it's equal to $X+Y$ and hence $Z'(0)=X+Y$, thus Taylor expansion of $Z(t)$ is of form $Z(t)=0+t(X+Y)+O(t^2)$ and
this gives the required statement.\par
Now, let's show that $[X,Y]\in \Lie(G)$ if $X,Y\in\Lie(G)$. Note, that $\forall t\in\R$ we have $\exp(tX)Y\exp(-tX)\in
\Lie(G)$, as $\forall s\in\R,\;\exp\mybra{ s\exp(tX)Y\exp(-tX) }=\exp(tX)\exp(sY)\exp(-tX)\in G$. Then,
as $\Lie(G)$ is vector space as was shown above, $\Lie(G)\ni\frac{d}{dt}\bigg|_{ t=0 }\mybra{ \exp(tX)Y\exp(-TX) }=
[X,Y]$
\myprobshort{ April 14 }{ Find Lie algebras and dimension of the following matrix Lie groups: $G=SL_n(\R),\;O(n),\;O(p,q)$. }
We shall use the definition of Lie algebra from the previous problem. Let's start from $G=SL_n(\R)$. Indeed, for
$X\in M_n(\R)$ we have $X\in\Lie(G)\iff\forall t\in\R ,\;\det(\exp(tX))=1$. As can be seen by
putting matrix $X$ in Jordan form over $\C$, $\det(\exp(X))=\exp(\tr(X))$ and hence $X\in\Lie(G)\iff\forall t\in\R,\;\exp(t\cdot
\tr X)=1$ and the latter happens iff $\tr(X)=0$ (in particular, "only if" follows by taking derivative of $\exp(t\cdot\tr X)$ at
$t=0$). Hence, $\Lie(SL_n(\R))=\mysetn{X\in M_n(\R)}{\tr X=0}$ and the latter has dimension $n^2-1$, elements $\mycbra{ E_{i,j} }_{
i,j=1,i\neq j}^{n,n}$ and $\mycbra{E_{ i,i }-E_{ i+1,i+1 } }_{i=1 }^{n-1}$ clearly forming a basis.\par
Now, let $G=O(n)$. Similarly, we have for $X\in M_n(\R)$ that $X\in\Lie(G)\iff \forall t\in\R,\;\exp(tX)\exp(tX)^T=I$. And
as $\exp(X)=\exp(X^T)$ the latter holds iff $\forall  t\in\R,\;\exp(tX)\exp(tX^T)=I$. Taking derivative at $t=0$ we see
that the necessary condition for latter to hold is that $X+X^T=0$. Conversely, if $X+X^T=0$, we have $\forall t\in\R,\;
\exp(tX)\exp(tX^T)=\exp(tX)\exp(-tX)=\exp(0)=I$. Hence, 
$\Lie(SL_n(\R))=\mysetn{X\in M_n(\R)}{X+X^T=0}$ and
the latter has dimension $n(n-1)/2$, elements $\mycbra{ E_{ i,j }-E_{ j,i } }_{ i,j=1,i\neq j }^{ n,n }$ forming a basis.\par
Finally, let $G=O(p,q)$. Similarly, we have for $X\in M_n(\R)$ that $X\in\Lie(G)\iff \forall t\in\R,\;\exp(tX)I_{ p,q }
\exp(tX)^T=I\iff\exp(tX)I_{ p,q }\exp(tX^T)=I$. Again, we see (taking derivative at $t=0$) that the necessary condition for this is
$XI_{ p,q }+I_{ p,q }X^T=0$. Conversely, if $XI_{ p,q }+I_{ p,q }X^T=0$ then $X_{ p,q }XI_{ p,q }=-X^T$ and so
$\forall t\in\R,\;\exp{ tX }I_{ p,q }\exp(tX^T)=
\exp(tX)I_{ p,q }\exp(-tI_{ p,q }XI_{ p,q })=\exp(tX)I_{ p,q }I_{ p,q }\exp(-tX)I_{ p,q }=\exp(tX)\exp(-tX)I_{ p,q }=I_{ p,q }$
and thus 
$\Lie(O(p,q))=\mysetn{X\in M_n(\R)}{XI_{ p,q }+I_{ p,q }X^T=0}$ and
the latter has dimension $pq+p(p-1)/2+q(q-1)/2$, elements 
$\mycbra{ E_{ i,j }-E_{ j,i } }_{ i,j=1,i\neq j }^{ p,p }$,  
$\mycbra{ E_{ i,j }-E_{ j,i } }_{ i,j=p+1,i\neq j }^{ p+q,p+q }$ and
$\mycbra{ E_{ i,j }+E_{ j,i } }_{ i=1,j=p+1 }^{ p,p+q }$
forming a basis.
\myprobshort{ April 14, Example 2.3.2 }{ $G=O(1,n+1)$ acts on $\Xi:=\mysetn{ x\in\R^{ 1,n+1 } }{ x_0^2=x_1^2+x_2^2+\hdots+x_n^2 }$
and the diffeomorphism $\Xi/\R^{ \times }\simeq \mys{n}$ induces action of $G$ on $\mys{ n }$. Show that this action
preserves angles and find isotropy of $\mysbra{ 1,1,0,0,\hdots,0 }$.}
Indeed, let us start with showing conformality. As we have $KAK$ decomposition for $G$ (the latter is reductive of non-inner type,
but $KAK$ holds for these) and maximal compact subgroup $K\simeq O(n+1)\times O(1)$ of $G$ acts as $O(n)$ on $\mys{n}$, which
is isometry, hence conformal map, it suffices to show the statement for the elements of $A=\mycbra{\exp(t(E_{ 1,n+2 }+E_{ n+2,1 }))
}_{ t\in\R }$. \par
Now, it is well-known and direct computations show that for $p_+:=(1,0,0,\hdots,0,1)\in\Xi$ and map $\iota_N:\mathfrak{n}_-\ni X
\mapsto\exp(X)\cdot p_+\in\Xi$ composed with projection $\Xi\to\Xi/\R^\times\simeq\mys{ n }$ yields compactification
of $\R^n\simeq\mathfrak{ n }_-\ni x\mapsto \mybra{ \frac{ 1-r^2 }{ 1+r^2 },\frac{ 2x }{ 1+r^2 }}\in\mys{ n }$
(where $r:=\myabs{ x }$).
What is more, for $H:=E_{ 1,n+2 }+E_{ n+2,1 }$, $t\in\R$ and $X\in\mathfrak{ n }_+$ we have
$\mys{ n }\ni e^{ tH }\exp{ X }p_+\R^\times=e^{ tH }\exp{ X }e^{ -tH }p_+\R^\times=\exp(e^{ -t }X)p_+\R^\times$
and hence when pulled back via compactification mentioned above, $e^{ tH }$ acts simply as dilation by ${ e^{ -t } }$,
which is clearly conformal. Thus, if we'd be able to show that the aforementioned compactification is a conformal map,
it would follows that $e^{ tH }$ is also so
(conformality at unique point not mapped by conformal compactification -- the $(1,0,0,\hdots,0)\in\mys{ n 
}$ follows by continuity). Hence, it suffices to show that compactification 
$F:\R^n\ni x\mapsto \mybra{ \frac{ 1-r^2 }{ 1+r^2 },\frac{ 2x }{ 1+r^2 }}\in\mys{ n }$ is conformal.\par
This is done by direct computation as follows. We just need to show that $F^*(dx_1^2+dx_2^2+\hdots+dx_{ n+1 }^2)$ is proportional
to a standard metric on $\R^n$. But indeed,
\[F^*(dx_1^2+dx_2^2+\hdots+dx_{ n+1 }^2)=d\mybra{ \frac{ 1-r^2 }{ 1+r^2 } }^2+\sum_{ i=1 }^{ n }d\mybra{ \frac{ 2x_i }{ 1+r^2 } }^2=\]
\[=\mybra{\frac{ -\sum_i4x_idx_i }{(1+r^2)^2}}^2+\sum_i\mybra{\frac{ 2dx_i(1+r^2-2x_i^2)-2x_i\sum_{ j\neq i }2x_jdx_j }{(1+r^2)^2}}^2
\]
As we only want to show the proportionality to standard metric, it suffices to consider the enumerators only and
divide everything by 4. Furthermore,
symmetry shows that coefficients for $dx_idx_i$ will all be the same, so it suffices to prove that coefficient for $dx_idx_j$
for $i<j$ will all vanish. Again, the apparent symmetry reduces this to considering only $i=1,\;j=2$ case. First addend
in latter expression gives coefficient $8x_1x_2$ for $dx_1dx_2$. Going to the sum, the terms for summation index $i=1$
and $i=2$ will give $2(1+r^2-2x_1^2)(-2x_1x_2)$ and $2(1+r^2-2x_2^2)(-2x_1x_2)$ respectively, while
terms for $i>2$ will give (when summed up) the coefficient $8x_1x_2\sum_{ j=3}^nx_j^2$. Summing these coefficients, we have
\[8x_1x_2-4x_1x_2(2+2r^2-2x_1^2-2x_2^2-2\sum_{ j=3}^nx_j^2)=8x_1x_2-4x_1x_2\cdot2=0.\]
\myprobshort{ April 21, Example 3.2.3 }{Suppose $M$ is a manifold and $TM=\bigsqcup_{ x\in M }T_xM$:tangent bundle. Show
that $TM$ is a vector bundle and cocycles are given by Jacobian of transition functions.}%TODO
Let $\mycbra{\mybra{ U_\alpha,\phi_\alpha:U_\alpha\to V_\alpha\subset\R^n }}_\alpha$ be a chart covering for $M$.
Recall that by definition, the manifold structure on $TM$ is compatible with the chart covering 
$\mycbra{ \mybra{\pi^{ -1 }(U_\alpha),\widetilde{ \phi_\alpha }  } }_\alpha$, where $\pi:TM\to M$ is
the union of $T_xM\to\mycbra{ x }$ maps and $\widetilde{\phi_\alpha}:\pi^{ -1 }(U_\alpha)\ni
\sum_iv_i\frac{ \partial }
{ \partial \phi_\alpha^i }\bigg|_p\mapsto
(\phi(p),v)\in V_\alpha\times\R^n$. We claim that 
$\mycbra{ \mybra{\pi^{ -1 }(U_\alpha),(\phi_\alpha^{ -1 }\times id_{ \R^n })\circ\widetilde{ \phi_\alpha }  } }_\alpha$ provide the trivialization for $TM\to M$, thus making it a vector bundle.\par
Indeed, as for every $p\in U_\alpha$ fixed $\mycbra{ \frac{\partial}{ \partial \phi_\alpha^i }\bigg|_p }_i$ provide a basis 
for $T_pM$ we have $T_pM\ni\sum_iv_i\frac{ \partial }
{ \partial \phi_\alpha^i }\bigg|_p\mapsto (\phi(p),v)\xrightarrow{ proj_2 }v\in\R^n$ is clearly an isomorphism. Furthermore,
$\pi^{ -1 }(U_\alpha)\xrightarrow{ (\phi_\alpha^{ -1 }\times id_{ \R^n })\circ\widetilde{\phi_\alpha }}U_\alpha\times\R^n\xrightarrow
{proj_1}U_\alpha$ coincides with $\pi$ and hence 
$\mycbra{ \mybra{\pi^{ -1 }(U_\alpha),(\phi_\alpha^{ -1 }\times id_{ \R^n })\circ\widetilde{ \phi_\alpha }  } }_\alpha$ do form
a trivialization.\par
Finally, let's compute the cocycles. Indeed, let $p\in U_{ \alpha\beta }:=U_\alpha\cap U_\beta$ so
that
$\sum_iv_i\frac{ \partial }{ \partial \phi_\alpha^i }\bigg|_p
\mapsto(\phi_\alpha(p),v)\in V_{ \alpha }\cap V_{ \beta }\times\R^n$ and
$\sum_iw_i\frac{ \partial }
{ \partial \phi_\beta^i }\bigg|_p\mapsto(\phi_\beta(p),w)\in V_{ \alpha }\cap V_{ \beta }\times\R^n$.
Suppose now that 
$\sum_iv_i\frac{ \partial }{ \partial \phi_\alpha^i }\bigg|_p=
\sum_jw_j\frac{ \partial }{ \partial \phi_\beta^j }\bigg|_p$ and we want to express $w$ in terms of $v$.
This is easy, as $\sum_iv_i\frac{ \partial }{ \partial \phi_\alpha^i }\bigg|_p=\sum_iv_i\sum_j\frac{ \partial\phi_\beta^j }{ \partial
\phi_\alpha^i}\frac{ \partial }{ \partial \phi_\beta^j }\bigg|_{ \phi_\alpha(p) }$, hence as 
$\mycbra{ \frac{\partial}{ \partial \phi_\beta^i }\bigg|_p }_i$ form a basis we should have
\[w_j=\sum_iv_i\frac{ \partial\phi_\beta^j }{ \partial\phi_\alpha^i }\bigg|_{ \phi_\alpha(p) }\]
and as matrix $\mycbra{ \frac{ \partial\phi_\beta^i }{ \partial\phi_\alpha^j }\bigg|_{ \phi_\alpha(p) } }_{ i,j }$
is precisely the Jacobian of $V_\alpha\to V_\beta$ transition function evaluated at $\phi_\alpha(p)\in V_\alpha$, we are done.
\myprobshort{ April 21, Exercise after Def 3.2.4 }{Suppose $\upupsilon\to M$ is a $G$-equivariant fiber bundle (that is,
	$G$ acts on $\upupsilon$ and $M$ and projection $\pi:\upupsilon\to M$ intertwines these actions)
	then for $g\in G$ and $s\in\Gamma(M,\upupsilon)$ $(\pi(g)\cdot s)(x):=g\cdot s(g^{ -1 }x)$ gives an action $G\curvearrowright\Gamma
(M,\upupsilon)$.}
First, we need to verify that $\pi(g)s$ so defined is indeed a section. This is easy, however, as (note that
we denote the projection $\upupsilon\to M$ and action $\pi(g)$ with the same letter) $\pi((\pi(g)s)(x))=
\pi(g\cdot s(g^{ -1 }x))=g\cdot\pi(s(g^{ -1 }x))=g\cdot g^{ -1 }\cdot x=x$, the second equality using $G$-equivariance
of $\pi$ and the third using the fact that $s$ is a section.\par
Next, we need to show that $\pi(\cdot)$ respects group structure. Definition makes it obvious that $\pi(1)s=s$ and
the $\pi(g^{ -1 }s)=\pi^{ -1 }(g\cdot s)$ will follow if we'll be able to show that $\pi(g_1g_2)s=\pi(g_1)\pi(g_2)s$ and this
will finish the proof. Latter is easy, however, as $(\pi(g_1)\pi(g_2)s)(x)=g_1\cdot(\pi(g_2)s)(g_1^{ -1 }x)=
g_1g_2s(g_2^{ -1 }g_1^{ -1 }x)=(\pi(g_1g_2)s)(x)$.
\myprobshort{April 28, Example 4.1.6}{Let $\Xi$ denote the space of all 2-by-4 complex matrices of rank 2 (with a structure of
	a complex manifold), $H:=GL_2(\C)\times(\C^\times)^4/\Delta(\C^\times)$, where $\Delta(\C^\times):=
\mysetn{\mybra{\mybra{\begin{smallmatrix}a^{ -1 }&0\\0&a^{ -1 }\end{smallmatrix}},(a,a,a,a)}}{a\in\C^\times}$ ($\Delta(C^\times)$
is normal in $\widetilde{H}:=GL_2(\C)\times(\C^\times)^4$) and $\widetilde{H}\curvearrowright\Xi$ via $(A,(t_1,t_2,t_3,t_4))\cdot
Z=A\cdot Z\cdot\diag\mybra{ t_1,t_2,t_3,t_4 }$ and this factors through $\Delta(\C^\times)$ to give $H\curvearrowright \Xi$. Show that
this action is \dashuline{free}\textsuperscript{ (1) }. Furthermore, 
$\Xi/H\simeq\mathbb{ PC }^1\setminus\mycbra{ 0,1,\infty }$\textsuperscript{ (2) }}
\newcommand{\pcone}{\mathbb{PC}^1}
\def\expr{\frac{(z_4-z_1)(z_3-z_2)}{(z_2-z_1)(z_3-z_4)}}
First, we will show that the action $H\curvearrowright \Xi$ is free.
	 Assume that for $(A,t_{ 1,2,3,4 })\in \widetilde{ H }$ and $Z\in \Xi$ we have $AZ\cdot\diag\mybra{ t_1,t_2,t_3,t_4 }=Z$.
		Inspecting this equality we see that every column of $Z$ becomes the eigenvector of $A$ with
		eigenvalues $t_{ 1,2,3,4 }$. Suppose that among these eigenvalues there are two different,
		say $t_1\neq t_2$, we have that first two columns (they should form a basis for $\C^2$)
		form eigenbasis of $\C^2$ and hence as other two columns are also eigenvectors, each of them
		should be proportional to first or second column (as eigenvalues are different), which contradicts
		the definition of $\Xi$. Hence, $t_{ 1,2,3,4 }$ are all the same and thus by adjusting with element
		of $\Delta({ \C^\times })$ we can assume $AZ=Z$, which implies $A=I$. Hence, action is free.\par
Let the map $\Xi\to\mathbb{ PC }^1$ be given by considering 4 columns of element of $\Xi$
as 4 elements $z_{ 1,2,3,4 }\in\mathbb{ PC }^1$ and mapping to $\frac{(z_4-z_1)(z_3-z_2)}{(z_2-z_1)(z_3-z_4)}\in\mathbb{PC}^1$.
To finish the proof of the theorem, we shall prove the following things:
\begin{enumerate}
	\item This map $\Xi\to\mathbb{ PC }^1$ is well-defined.
	\item image does not include $0,1$ or $\infty\in\pcone$.
	\item the whole $\pcone\setminus\mycbra{ 0,1,\infty}$ gets covered.
	\item elements conjugate via $H$ map to the same element of $\pcone\setminus\mycbra{ 0,1,\infty}$.
	\item if two elements map to the same in $\pcone$, they are conjugate via $H$.
	\item map $\Xi\to\mathbb{ PC }^1\setminus\mycbra{ 0,1,\infty}$ is local diffeomorphism.
\end{enumerate}
And this is enough to finish the proof. So let's proceed:
\begin{enumerate}
	\item By considering 4 columns of $Z$ as members $z_{ 1,2,3,4 }$
		of $\pcone$, the requirement on the elements of $\Xi$ to have rank
		2 is the same as requiring $z_i$ to be pairwise different. 
		As all $z_i$ are pairwise different, their pairwise differences are nonzero,
		hence both enumerator and denominator of $\frac{(z_4-z_1)(z_3-z_2)}{(z_2-z_1)(z_3-z_4)}$ is nonzero.
		If one (hence, exactly one) of $z_i$
		is $\infty$, we define the $\frac{(z_4-z_1)(z_3-z_2)}{(z_2-z_1)(z_3-z_4)}$ by (well-defined, as can
		be seen by direct calculation) $\to\infty$ limit. 
	\item First suppose that one of $z_i$ is $\infty$. For definiteness we suppose that $z_1=\infty$ (all other cases
		are treated analogously). According to definition, in this case $\frac{(z_4-z_1)(z_3-z_2)}{(z_2-z_1)(z_3-z_4)}=
		\frac{ z_3-z_2 }{ z_3-z_4 }$ with $z_{ 2,3,4 }\in\C$ and this fraction cannot be 0 (as $z_3\neq z_2$),
		1 (as $z_3-z_2=z_3-z_4\implies z_2=z_4$, but $z_2\neq z_4$) or $\infty$ (as $z_3\neq z_4$).\par
		Second, assume that all of $z_i$ are in $\C$. As they are pairwise different, ratio cannot become
		$0$ or $\infty$. Suppose thus that $\frac{(z_4-z_1)(z_3-z_2)}{(z_2-z_1)(z_3-z_4)}=1$ and this gives
		$z_4z_3+z_1z_2=z_2z_3+z_1z_4$, equivalently $z_4(z_3-z_1)=z_2(z_3-z_1)$ and as $z_3-z_1\neq0$ we can
		cancel it out to get $z_4=z_2$ -- contradiction.
	\item Indeed, let $z\in\C$ be given, such that $z\neq0,1$. We take $x\in\C$ such that $x^2=z$ (such choice can
		always be made). Note that as $z\neq1$ we may assume $x\neq\pm1$. Take further $z_{ 1,2,3,4 }\in\C\subset\pcone$
		such that $z_3-z_2=x,\;z_2-z_1=1,\;z_3-z_4=1$ -- such choice can always be made. Then $z_4-z_1=
		-(z_3-z_4)+z_3-z_2+z_2-z_1=x$ and hence $\expr=x^2=z$.\par
		To finish the proof of this part we observe that that all $z_{ 1,2,3,4 }$ are pairwise different, hence their
		quadruple can be identified with an element of $\Xi$. Indeed, we see that $z_2-z_1=1\neq0$, $z_3-z_1=
		z_3-z_2+z_2-z_1=x+1\neq0$ (as $x\neq-1$ by above), $z_4-z_1=x\neq0$, $z_3-z_2=x\neq0$, $z_4-z_2=x-1\neq0$
		and finally $z_4-z_3=-1\neq0$.
	\item Note first that if 4 columns of element $M$ of $\Xi$ are realized as projective complex numbers, the action
		of $(A,t_{ 1,2,3,4})$ on $M$ is as follows. First, $\diag(t_1,t_2,t_3,t_4)$ does not affect projective
	complex numbers. Second, as matrix $A:=\left[\begin{smallmatrix}a&b\\c&d\end{smallmatrix}\right]$
		acts on column $[m_1;m_2]^T$ as $\frac{am_1+bm_2}{ cm_1+dm_2 }$, when 
		$[m_1;m_2]^T$ is seen as projective complex number $z\in\pcone$ this action can be written as
		$\frac{ az+b }{ cz+d }$ (understood as limit if $z=\infty$).\par
		Now, if we replace $z_{ 1,2 }$ by the results of $A=
		\left[\begin{smallmatrix}a&b\\c&d\end{smallmatrix}\right]$ action
		and recompute $z_1-z_2$ we get
		\[\frac{ az_1+b }{ cz_1+d }-\frac{ az_2+b }{ cz_2+d }=\frac{ (az_1+b)(cz_2+d)-(az_2+b)(cz_1+d) }{ (cz_1+d)(cz_2+d) }
		=\frac{ (ad-bc)(z_2-z_1) }{ (cz_1+d)(cz_2+d) }\]
		hence if we act upon each $z_{ 1,2,3,4 }$ with $A$ the expression $\expr$ does not change.
	\item In the light of the remarks of the previous item, we can assume from the outset that we have two quadruples
		$z_{ 1,2,3,4 }$ and $z'_{ 1,2,3,4 }$ of projective complex numbers such that $\expr=
		\frac{(z'_4-z'_1)(z'_3-z'_2)}{(z'_2-z'_1)(z'_3-z'_4)}$ and we want to show that there exists
		$A\in GL_2(\C)$ such that $Az_i=z_i'$.\par
		First, we note that $SL_2(\C)$ acts transitively on triples of complex numbers, as for $z_{ 1,2,3 }\in\C$
		pairwise distinct
	we have matrix $\left[\begin{smallmatrix}z_3(z_2-z_1)&z_1(z_3-z_2)\\z_2-z_1&z_3-z_2\end{smallmatrix}\right]\in GL_2(\C)$
		mapping $0,1,\infty$ to $z_1$, $z_2$ and $z_3$ respectively. Adjusting it by complex factor brings the matrix to
		$SL_2(\C)$.\par
		Now, returning to our two quadruples 
$z_{ 1,2,3,4 }$ and $z'_{ 1,2,3,4 }$, by adjusting both of them with the matrix $\left[\begin{smallmatrix}1+\epsilon&0\\\epsilon&1
\end{smallmatrix}\right]
		$ for $\epsilon\in\C$ small that brings $\infty$ to $(1+\epsilon)/\epsilon\in\C$ we may assume from the
		outset that none of quadruples contain $\infty$.\par
		Next, we adjust $z_{ 1,2,3,4 }$ with element of $SL_2(\C)$ (it will not change the $\expr$) so that
		the first three elements of quadruples coincide. Then, as 
		$\expr=\frac{(z'_4-z'_1)(z'_3-z'_2)}{(z'_2-z'_1)(z'_3-z'_4)}$ we have $z_4-z_1=z_4'-z_1'$ and hence
		$z_4=z_4'$ and the quadruples coincide.
	\item Finally, following \cite[Chapter 9]{lee} we shall show first that $H\curvearrowright\Xi$ is proper
		action to apply \cite[Theorem 9.16]{lee} that will grant us that $\Xi/H$ is manifold and
		$\pi:\Xi\to \Xi/H$ is a submersion. To prove the properness, assume that $\Xi\ni Z_n\to Z_0\in\Xi$
		and $A_nZ_n\cdot\diag(t_1^{ (n) },t_2^{ (n) },t_3^{ (n) },t_4^{ (n) })\to Z'_0\in\Xi$. And we want
		to show that some subsequence of $(A_n,(t_1^{ (n) },t_2^{ (n) },t_3^{ (n) },t_4^{ (n) }))$ converges (
		enough to prove convergence in $\widetilde{H}$ as convergence in $H$ will then follow by continuity of
		$\widetilde{ H }\to H$ epimorphism map).
		Indeed, by adjusting with member of $\Delta(\C^\times)$ (for every $n$ separately) we may assume that
		$(t_1^{ (n) },t_2^{ (n) },t_3^{ (n) },t_4^{ (n) })$ has some of its absolute values equal to one, hence
		every $t_i^{ (n) }$ is bounded and (passing to subsequence) we may assume for every $1\leq i\leq 4$ 
		we have $t_i^{ (n) }$ converges. We wish to show that limit cannot be zero. Indeed, assume $t_1^{ (n) }\to0$.
		As at least one $t_i^{ (n) }$ should converge to non-zero, we may assume $t_2^{ (n) }$ does so.
		Then, we should have (calling columns of $Z_n$ by $z_{ 1,2,3,4 }^{ (n) }$) that $A_nz_1^{ (n) }\to\infty$.
		Similarly, $A_nz_2^{ (n) }$ should converge in $\C^2$. Now, as $Z_n$ converges in $\Xi$ we should have
		$z_{ 3,4 }^{ (n) }=a_{ 3,4 }^{ (n) }z_1^{ (n) }+b_{ 3,4 }^{ (n) }z_2^{ (n) }$
		with $a_{ 3,4 }^{ (n) }$
		convergent to 4 nonzero complex numbers that form invertible 2-by-2 matrix.
		. Hence, we have $A_nz_{ 3,4 }^{ (n) }\to\infty$. However, as
		$z_{ 3,4 }^{ (n) }=a_{ 3,4 }^{ (n) }z_1^{ (n) }+b_{ 3,4 }^{ (n) }z_2^{ (n) }$ 
		with $a_{ 3,4 }^{ (n) }$
		convergent to 4 nonzero complex numbers can be inverted, to get $z_2^{ (n) }$ expressed as linear (converging to)
		nontrivial combination of $z_3^{ (n) }$ and $z_4^{ (n) }$, hence we have $A_nz_2^{ (n) }\to\infty$, contrary to our
		assumption. Thus we have $t_i^{ (n)}$ converging to elements of $\C^\times$. Now, as we have
		$A_nZ_n\cdot\diag(t_1^{ (n) },t_2^{ (n) },t_3^{ (n) },t_4^{ (n) })\to Z'_0\in\Xi$ and $Z_n$ convergent
		, we $Z_n\cdot\diag(t_1^{ (n) },t_2^{ (n) },t_3^{ (n) },t_4^{ (n) })$ converges in $\Xi$  and hence
		$A_n$ convergent as well (in $GL_2(\C)$). Hence, 
		Thus properness of $H\curvearrowright\Xi$ action is proven.\par
		Now, as we have that $\pi:H\to\Xi/H$ is	smooth submersion, using appropriate coordinate chart
		(given via constant rank theorem) that renders $\pi$ as simply a (restriction of) an $\R^{ 16 }\to\R^2$
		projection, to see that $\Xi/H\to\pcone\setminus\mycbra{ 0,1,\infty }$ is local diffeomorphism,
		it suffices to show that $\Xi\to\pcone\setminus\mycbra{ 0,1,\infty }$ is constant map of rank 2 (of full rank).
		Latter, however, is true as fractional-linear map is always so (it has nonzero complex derivative).
\end{enumerate}
\myprobshort{ April 28, remark after Def. 4.1.7}
{ With $\Xi$ and $H\curvearrowright\Xi$ as in previous problem, consider further the right action $\mathfrak{ S }_4
\curvearrowright\Xi$ acting by column permutations
show that $\Xi\to\mathbb{ PC }^1\setminus\mycbra{ 0,1,\infty }$ is $\mathfrak{ S }_4$
-equivariant bundle (that is, projection is $\mathfrak{ S }_4$-intertwiner)
and $H$-principal (that is equivariant for $H\curvearrowright\mathbb{ PC }^1\setminus\mycbra{
0,1,\infty }$ trivial action, and $H$ acts on fibers
transitively freely). Show further that $H$-action and $\mathfrak{ S }_4$ do not commute.}
$H$-principality was shown in previous problem (to be precise, we {\it define} action of $H$ on $\pcone\setminus\mycbra{ 0,1,\infty }$
in such a way that $\Xi/G$ becomes $G$-diffeomorphism and as we've seen $H$-orbits are precisely fibers
). $\mathfrak{ S }_4$ acts equivariantly. In fact, we can {\it define} the action $\mathfrak{ S }_4\curvearrowright\pcone\setminus
\mycbra{ 0,1,\infty }$ to make $\pi:\Xi\to\Xi/H$ intertwiner, once we show that elements of the same $\pi$-fiber go to the same fiber.
As $\mathfrak{ S }_4$ is generated by three cycles $(1,2)$, $(2,3)$ and $(3,4)$ and it's enough to show fiber-preserving for these.
But letting $A_{ 1,2,3,4 }$ denote $\expr$ with $id,(1,2),(2,3),(3,4)\mathfrak{ S }_4$ acted upon it respectively,
we see by direct computation that $A_2=1-A_1$, $A_3=A_1/(A_1-1)$ and $A_4=1-A_1$ and this shows the desired.\par
Finally, let us show that the actions do not commute. This is obvious, as, say $\diag(1,2,1,1)\in H$ and $(1,2)\in\mathfrak{S}_4$
acting on an element of $\Xi$ do not commute.
\myprobshort{ April 28, Prop. 4.2.1}{Suppose $\varpi:\Xi\to M$ is $H$-principal bundle and we have finite dimensional
	representation $\tau:H\to\Aut(V)$. Define equivalence on $\Xi\times V$ via $(\xi,v)\sim(\xi h,\tau(h)^{ -1 }v)$ for $h\in H$
	and write $\upupsilon:=\Xi\times_HV:=(\Xi\times V)/\sim$. Then,
	\begin{enumerate}
		\item $\upupsilon\ni[\xi,v]\to\varpi(\xi)\in M$ is vector bundle with fiber $V$;
		\item If further $\Xi\to M$ is $G$-equivariant $H$-principal bundle, then $\upupsilon\to M$ is 
			$G$-equivariant as well. 
\end{enumerate}}
\begin{enumerate}
	\item Let $\mycbra{ \mybra{ U_\alpha\subset M,\phi_\alpha:\varpi^{ -1 }(U_\alpha)\to U_\alpha\times H } }_\alpha$
		be the local trivialization of $\Xi\to M$.
			We define $H_\alpha$ to be the part of the $\upupsilon$ bundle
			above $U_\alpha\subset M$, that is the set $\mysetn{ [\xi,v]\in\Xi\times_HV }{ \varpi(\xi)\in U_\alpha }$
			and write two components of map $\phi_\alpha:\varpi^{ -1 }(U_\alpha)\to U_\alpha\times H$ as 
			$\phi_\alpha=(\varpi,h_\alpha)\in U_\alpha\times H$.
			and define a map $F_\alpha:H_\alpha\to U_\alpha\times V$ via $[\xi,v]\mapsto(\varpi(\xi),\tau(h_\alpha
			(\xi))v)\in U_\alpha\times V$. This map is well defined, as 
			$F_\alpha([\xi h,\tau(h)^{ -1 }v])=(\varpi(\xi\cdot h),\tau(h_\alpha(\xi h))\tau(h)^{ -1 }v)=
			(\varpi(\xi),\tau(h_\alpha(\xi))v)$ ($h_\alpha$ intertwines $H$-action, as $\phi_\alpha$ has to do so,
			$\Xi$ being the principal $H$-bundle) and $F_\alpha$ is well-defined.\par
			Furthermore, $F_\alpha$ descends to a diffeomorphism $H_\alpha\to U_\alpha\times V$, as 
			can be seen using local
			coordinates given by Constant Rank theorem for $\Xi\times V\to \Xi\times_HV$ submersion
			(submersion, as $\Xi\times_HV$ may be seen as factor of $\Xi\times V$ via the right action of group $H$
			by $h\cdot(\xi,v):=(\xi\cdot h,\tau(h)^{ -1 }v)$ and that action is smooth, free 
			(because $H\curvearrowright\Xi$ is free) 
			and proper 
			(as $H\curvearrowright\Xi$ is free, which follows from equivariance of $h_\alpha$)
			and then applying \cite[Theorem 9.16]{lee}) and noting that
			$\varpi^{ -1 }(U_\alpha)\times V\ni(\xi,v)\to(\psi_\alpha(\xi),\tau(h_\alpha
			(\xi))v)\in U_\alpha\times V$ was a submersion itself.\par
			Finally, the fact that $\forall x\in M,\;F_\alpha\big|_{ \upupsilon_x }:\upupsilon_x\to\mycbra{ x }\times V$
			is an isomorphism of linear spaces is apparent, as for every fixed $\xi\in\Xi$, $\tau(h_\alpha(\xi))$
			is an isomorphism.
		\item We let $G\curvearrowright\upupsilon$ via $g[\xi,v]:=[g\xi,v]$ (well-defined as left-multiplication
			and right-multiplication commute; and factors through to a smooth action on $\upupsilon$, again as
			$\Xi\times V\to\upupsilon=\Xi\times_HV$ is a submersion)
			. Then it is clear that $\varpi([g\xi,v])=\varpi(g\xi)=g\varpi(\xi)$, the $\Xi\to M$ being the 
			$G$-equivariant bundle.
		$\varpi:\upupsilon\to M$ intertwines $G$-action, as 
\end{enumerate}
\myprobshort{April 28, Prop. 4.2.1}{In the setting of the previous problem, show that there's a natural bijection
	$\Gamma(M,\upupsilon)\simeq C^\infty(\Xi,V)^H:=\mysetn{ F\in C^\infty(\Xi,V) }{\forall h\in H,\;F(\xi\cdot h)=\tau(h)^{ -1 }
	F(\xi)}$. Show further that if $\Xi$ is $G$-equivariant principal $H$-bundle, then 
	$\Gamma(M,\upupsilon)\simeq C^\infty(\Xi,V)^H$ is $G$-isomorphism, (with $G\curvearrowright C^\infty(\Xi,V)^H$
	via $g\cdot f:=f(g^{ -1 }\cdot)$)}
	First we construct the map $P:C^\infty(\Xi,V)^H\to\Gamma(M,\upupsilon)$ as follows.
	Given $F\in C^\infty(\Xi,V)^H$ we define the section $s$ of $\upupsilon\to M$ as follows. For $\xi\in\Xi$
	such that $\varpi(\xi)=x\in M$ we let $s(x):=[\xi,F(\xi)]$. It is independent of choice of $\xi$, as
	$[\xi\cdot h,F(\xi h)]=[\xi,\tau(h)\tau(h)^{ -1 }F(\xi)]=[\xi,F(\xi)]$, hence $s$ is a well-defined $M\to\upupsilon$
	map. It is smooth, as $\Xi\ni\xi\mapsto(\xi,F(\xi))\in\Xi\times V$ is smooth, hence (as $\Xi\times V\Xi\times_HV$ is
	a smooth) the latter factors to smooth map $\Xi\to\upupsilon$ and finally (as $\Xi\to M$ is a surjection) map
	$\Xi\to\upupsilon$ factors (as was shown above) to a {\it smooth} map $s:M\to\upupsilon$. Finally, $s$ is a section,
	as $\varpi(s(x))=\varpi([\xi,F(\xi)])=\varpi(\xi)=x$ according to the way we've chosen $\xi$.\par
	Next, we seek to show that $P:C^\infty(\Xi,V)^H\to\Gamma(M,\upupsilon)$ defined in previous paragraph is onto and injective.
	The latter is clear, as if we'd have two maps $F,F'\in C^\infty(\Xi,V)^H$ giving rise to the same section of $\upupsilon$,
	then this implies that for every $\xi\in\Xi$ we have $[\xi,F(\xi)]=[\xi,F'(\xi)]$, which implies (as $H$-action is
	free) that $F(\xi)=F'(\xi)$ and hence $F=F'$. To show the surjectivity of $P$, let $s\in\Gamma(M,\upupsilon)$ be given.
	We construct $F\in C^\infty(\Xi,V)^H$ as follows. Let $\xi\in \Xi$ be given and take arbitrary
	$(\xi',v')\in s(\varpi(\xi))\in\upupsilon_{ \varpi
	(\xi) }$. As $\varpi(\xi)=\varpi(\xi')$, we should have $\xi'=\xi\cdot h$ and we let $F(\xi):=\tau(h)v'$.
	This definition is independent of choice of $\xi'$ as can be directly seen. Moreover, $F(\xi\cdot h)=\tau(h)^{ -1 }F(\xi)$
	as can be seen by construction. Finally, $F(\cdot):\Xi\to V$ is smooth, which can be seen as follows. Let
	$\mycbra{(U_\alpha\subset M,(\varpi,h_\alpha):\Xi\supset\varpi^{ -1 }(U_\alpha)\to U_\alpha\times H)}_\alpha$
	and $\mycbra{ \mybra{ U_\alpha,\upupsilon\supset\varpi^{ -1 }(U_\alpha)\ni[\xi,v]\mapsto(\varpi(\xi),\tau(h_\alpha(\xi))^
	{ -1 }v) } }$ be local trivializations of $\Xi\to M$ and $\upupsilon\to M$ respectively as discussed in a previous problem.
	Then, above any $U_\alpha$ $s$ gives rise to a smooth function $s':U_\alpha\to V$ such that $\forall \xi\in \varpi^{ -1 }
	(U_\alpha)$,
	$[\xi,\tau(h_\alpha(\xi))^{ -1 }s'(x)]=s(\varpi(\xi))$
	. Then, for any $\xi\in\varpi^{ -1 }(U_\alpha)$ we should have
	$F(\xi)=\tau(h_\alpha(\xi))^{ -1 } 
	s'(\varpi(\xi))$ according to the construction of $F$ above and hence $F$ is smooth on $U_\alpha$.
	Now it is clear that $P(F)=s$, as above $U_\alpha$ we have (taking any $\xi$ such that $x=\varpi(\xi)$)
	$P(F)(x)=[\xi,F(\xi)]=[\xi,\tau(h_\alpha(\xi))^{ -1 }s'(x)]=
	s(\varpi(\xi))=s(x)$.\par
 	It is easy to note that under the additional hypothesis of $G$-equivariance of $\Xi$ the $P$ constructed
	above intertwines $G$-action, as if $P(F)=s$ (recall that this means that $\forall \xi\in\Xi,\;s(\varpi(\xi))=[\xi,F(\xi)]
	$)
	and $\varpi(\xi)=x$, we have $P(g\cdot F)(x)=P(F(g^{ -1 }\cdot))(x)=
	g[g^{ -1 }\xi,F(g^{ -1 }\xi)]=gs(g^{ -1 }x)$ (the last equality follows from the fact that $\varpi(g^{ -1 }\xi)=g^{ -1 }x$)
	hence $P(g\cdot F)=gs(g^{ -1 }\cdot)=g\cdot s$.
	\myprobshort{ May 12, Exercise 5.13}{
		Show that for $k=-l\in-\Z_{ \geq0 }$ we have $\mysetn{ F\in\mathcal{ O }(\C^2\setminus\mycbra{ 0 } }
	{\forall\lambda\in\C^\times,\;F(\lambda z_1,\lambda z_2)=\lambda^{ k }F(z_1,z_2)}$ consists of homogeneous
polynomials of degree $l$ and is a vector space of degree $l+1$.}
We start with showing that 
$H:=\mysetn{ F\in\mathcal{ O }(\C^2\setminus\mycbra{ 0 } }
	{\forall\lambda\in\C^\times,\;F(\lambda z_1,\lambda z_2)=\lambda^{ k }F(z_1,z_2)}$ consists of homogeneous polynomials of
	degree $l$.
	As it is clear that all homogeneous polynomials are indeed belong to $H$, it remains to prove the reverse inclusion.
	So let $F\in H$ be given. Its homogeneity implies that it is bounded near $0$, hence has a removable singularity there,
	hence $F$ extends to a holomorphic function on $\C^2$. Next, we write $F$ as power series $\sum_{ i,j=1 }^{ \infty }
	a_{ i,j }z_1^iz_2^j$
	centered around $0$ and
	homogenity tells us that we should have for $E:=z_1\frac{ \partial }{ \partial z_1 }+z_2\frac{ \partial }{ \partial z_2 }$
	$EF=lF$ holding on $C^2\setminus\mycbra{ 0 }$, hence at $0$ as well (by continuity)
	and hence uniqueness of expansion tells us that
	for $i+j\neq l$ we should have $a_{ i,j }=0$. Hence $F$ is a homogeneous polynomial.\par
	Finally, the space homogeneous polynomials of degree $l$ clearly has dimension $l+1$, the $\mycbra{ z_1^iz_2^{ l-i } }_{ i=0 }
	^l$ forming a basis.
\myprobshort{ April 28, Theorem 5.2.2 }{ Considering $G\to G/H$ as $G$-equivariant $H$-principal bundle,
	construct the associated bundle $\upupsilon\to G/H$
	as above (with assumed $H\curvearrowright V$ finitely-dimensional representation).
	Show that there is a natural bijection $\Gamma(G/H,\upupsilon)^G\simeq V^H$.}
	As was shown above $\Gamma(G/H,\upupsilon)\simeq C^\infty(G,V)^H$ and (recalling that $G$
	acts on $C^\infty(G,V)^H$ with $g\cdot F=F(g\cdot)$)
	this isomorphism identifies $\Gamma(G/H,\upupsilon)^G$ with constant functions on $G$
	and we get the desired morphism by mapping these functions to their value, say to $F(1)$. 
	We indeed have $F\in V^H$, as $\forall h\in H,\;\tau(h)^{ -1 }F(1)=F(h)=F(1)$, the first
	equality using the fact that $F\in C^\infty(G,V)^H$ and the second is the fact that $F$ is constant.
	Conversely, given $v\in V^H$ constant function with value $v$ $G$ will be the member of $C^\infty(G,V)^H$
	by the same computation as above and these constructions are clearly inverse to each other.
%%\begin{myprob}[April 28, Proposition 4.2.1] Let $\varpi:\Xi\to M$ is $H$-principal bundle with right action,
%%	$\tau:H\to\Diff(V)$ is group homomorphism (assume $V$ is a manifold)
%%	. Define equivalence $\sim$ on $\Xi\times V$, equivalence class of
%%	$(\xi,v)$ being all the elements $(\xi h,\tau(h)^{-1}v)$ for all $h\in H$.
%%	Show that the following holds:
%%	\begin{enumerate}
%%		\item Map $\nu:=\Xi\times V/\sim\to M$, $[\xi,v]\mapsto\varpi(\xi)$ is a well-defined map,
%%			which defines a fiber bundle with a fiber $V$.
%%		\item If $\varpi:\Xi\to M$ was a $G$-equivariant $H$-principal bundle, then the bundle 
%%			$\nu:=\Xi\times V/\sim\to M$ is $G$-equivariant.
%%	\end{enumerate}
%%\end{myprob}
%%	\begin{enumerate}
%%		\item First of all, the map $\nu$ is well-defined. Indeed, for any $h\in H$ we have
%%			$\varpi[\xi h,\tau(h)^{-1}v]:=\varpi(\xi h)=\varpi(\xi)$, the last equality holds
%%			true, since $\varpi:\Xi\to M$ is $H$-principal by hypothesis, hence
%%			$\varpi[\xi h,\tau(h)^{-1}v]:=\varpi(\xi h)=\varpi(\xi)=:\varpi[\xi,v]$.
%%
%%			eu%TODO
%%	\end{enumerate}
%%\begin{myprob}[April 28, Thm. 4.2.3] There is natural bijection for associated bundle $\nu$ $\Gamma(M,\nu)\simeq
%%	C^{\infty}(\Xi,V)^H:=\mysetn{F\in  C^{\infty}(M,V)}{F(gh)=\tau(h)^{-1}F(g)\forall h\in H\forall g\in G}$
%%	and the action $G$ on $C^{\infty}(\Xi,V)$ leaves $C^{\infty}(\Xi,V)^H$ invariant.
%%\end{myprob}%TODO
\begin{thebibliography}{9}
	\bibitem{lee}Lee, J.M. (2003). {\it Introduction to Smooth Manifolds}. Springer
\end{thebibliography}
\end{document}
%NB: 10 pages
