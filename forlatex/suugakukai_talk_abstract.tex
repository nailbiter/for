%japanese
\documentclass[a4paper,10pt]{article} % use larger type; default would be 10pt

\usepackage{enumerate}
\usepackage{setspace}
\usepackage{geometry,amsmath,amssymb,bbm,xypic}
\usepackage[all,cmtip]{xy}
\usepackage{xeCJK}
\setCJKmainfont{IPAMincho}
\usepackage{amsmath,amssymb,bbm,ulem,float,mystyle}
\usepackage{xeCJK}
\setCJKmainfont{IPAMincho}

%%%%%%%%%% Start TeXmacs macros
\catcode`\<=\active \def<{
\fontencoding{T1}\selectfont\symbol{60}\fontencoding{\encodingdefault}}
\catcode`\>=\active \def>{
\fontencoding{T1}\selectfont\symbol{62}\fontencoding{\encodingdefault}}
\newcommand{\assign}{:=}
\newcommand{\comma}{{,}}
\newcommand{\nin}{\not\in}
\newcommand{\tmop}[1]{\ensuremath{\operatorname{#1}}}
\newcommand{\tmtextit}[1]{{\itshape{#1}}}
\newcommand{\um}{-}
\newtheorem{theorem}{定理}
\newcommand{\sol}{\mathcal{S}ol(\R^{p,q};\lambda,\nu)}
\newcommand{\Hom}{\mbox{Hom}}
\newcommand{\Op}{\mbox{\normalfont Op}}
%%%%%%%%%% End TeXmacs macros

\begin{document}

\uline{以下の}は小林氏とSpeh氏の$O(n,1)$の対称性破れ作用素の性質についての
論文\cite{kobayashi2015symmetry}を一般化する
目指し、小林俊行氏と共同作業。
%%This is joint work with Toshiyuki Kobayashi aimed to extend the results of his
%%earlier paper with Birgit Speh studying properties of symmetry breaking
%%operators of $O (n, 1)$. 
対称性破れ作用素というのは、$G=O(p+1,q+1)$の退化主系列表現
$I(\lambda)$から、$G$の閉部分群$G':=\left\{ g\in G\mid g\cdot e_{p+1}=e_{p+1} \right\}\simeq O(p,q+1)$の退化主系列表現$J(\nu)$への
$G'$絡み作用素である(ここで$\left\{ e_i \right\}_{i=1}^{p+q}$が$\R^{p+q}$の標準基底、$g\cdot e_{p+1}$が$O(p+1,q+1)\curvearrowright\R^{p+q}$作用である)。
%%$G'$退会主系列
%%Symmetry breaking operators (SBOs) are
%%$G'$-intertwining operators between the degenerate principal series
%%representations $I (\lambda)$ and $J (\nu)$ of $G \assign O (p + 1, q + 1)$
%%and its closed subgroup $G' \assign O (p, q + 1)$ respectively, parametrized
%%by $(\lambda, \nu) \in \mathbbm{C}^2$.

得られた結果が以下の通りになる:

\begin{theorem}
$p, q \geqslant 1$\uline{で}、 $P,P':=P\cap G'$を$G,G'$の最大故物型部分群とする。
  $P'\backslash G/P$の閉集合と\uline{閉包関係}が以下の通りになり:\\
  $p>1$の時
  \begin{figure}[H]\centerline{\xymatrix{&G/P\ar@{-}[ld]\ar@{-}[rd]&\\H
	  :=\mysetn{[\xi,\eta]\in G/P\simeq \Sp^p\times\Sp^q/\left\{ \pm\right\}}{\xi_{p}=0}
	  \ar@{-}[rd]&&C
	  :=\mysetn{[\xi,\eta]\in G/P\simeq \Sp^p\times\Sp^q/\left\{ \pm\right\}}{\xi_{0}=\eta_q}
	  \ar@{-}[ld]\\&H\cap C\ar@{-}[d]&\\&\{[0]\}:=\left\{ [1,0_{p+q},1] \right\}&}}\end{figure}
  $p=1$の時
  \begin{figure}[H]\centerline{\xymatrix{&G/P\ar@{-}[ld]\ar@{-}[rd]&\\H\ar@{-}[rd]&&C\ar@{-}[ld]\\&\{[0]\}&}}\end{figure}
\end{theorem}

\begin{theorem}
	以下のように$\Op$と$\mathcal{S}$写像を定義する:
	\begin{figure}[H]\centerline{\xymatrixcolsep{7pc}\xymatrix{\Hom_{G'}(I(\lambda),J(\nu))\ar[r]^{\simeq} \ar@/^2pc/[rr]^{\mathcal{S}}
		&\left( \mathcal{D}'(G/P,\mathbb{C}_{n-\lambda})\otimes\mathbb{C}_\nu \right)^{P'}
		\ar[r]_{F\mapsto \supp(F)}\ar[d]^{\simeq}_{\mbox{rest}}
		&P'\backslash G/P\\
		&\sol\ar[lu]^{\mbox{Op}}_{\simeq}&
		}}\end{figure}
	そうすると、次のような結果が成り立つ
	\begin{description}
		\item[G/P場合]ある$(\lambda,\nu)\in \mathbb{C}^2$に正則に依存する対称性破れ作用素$\Op^{\R^n}_{\lambda,\nu}$。
			更に、$\Re(\lambda+\nu-n),\Re(-\nu)>0$を満たす$(\lambda,\nu)\in \mathbb{C}^2$に対して$\mathcal{S}(\Op^{\R^n}_{\lambda,\nu})=G/P$が成り立つ。
			さらに、$\Re(\lambda+\nu-n),\Re(-\nu)>0$を満たす$(\lambda,\nu)\in \mathbb{C}^2$に対して
			\[\Op^{-1}(\Op^{\R^n}_{\lambda,\nu})=\frac{\myabs{x_p}^{\lambda+\nu-n}\myabs{Q}^{-\nu}}{N},\quad N:=
				{\Gamma\left( \frac{\lambda-\nu}{2} \right)
				\Gamma\left( \frac{1-\nu}{2} \right)\Gamma\left( \frac{\lambda+\nu-n+1}{2} \right)}\]
				更に、$\mysetn{(\lambda,\nu)\in\mathbb{C}^2}{\Op^{\R^n}_{\lambda,\nu}=0}\subset\mathbb{C}^2$が\uline{知られている}離散集合。
			\item[H場合]
				ある$(\lambda,\nu)\in \mathbb{C}^2:\lambda+\nu-n=-1-2k\in-1-2\Z_{\ge0}$に正則に依存する対称性破れ作用素$\Op^{H}_{\lambda,\nu}\neq0$。
				$Re(\nu)<-2k$を満たす$\nu\in\mathbb{C}$に対して、$\mathcal{S}(\Op^{H}_{\lambda,\nu})=H$。さらに、
				\[\Op^{-1}(\Op^{H}_{\lambda,\nu})=\frac{\rho\left( \delta^{(2k)}(x_p)\times\myabs{Q}^{-\nu} \right)}{N_H}\]
				ここで $a \nin - n -\mathbbm{Z}_{\geqslant 0}$-次の斉次性を持つ超関数$F \in \mathcal{D}' (\mathbbm{R}^n \backslash \{ 0 \})$
				に対して$\rho(F)\in\mathcal{D}' (\mathbbm{R}^n)$ が唯一$a$-次の斉次性を持つ$\rho(F)\big|_{\R^n\setminus\left\{ 0 \right\}}=F$を
				満たす超関数(\uline{cf}. {\cite[thm. 3.2.3]{hormander1983analysis}}) 。$\times$が
				$\mathbb{R}^n\setminus\left\{ 0 \right\}$上の超関数積(cf. {\cite[thm. 8.2.10]{hormander1983analysis}})。
				$N_H$が上の$N$のような$(\lambda,\nu)$の関数。
			\item[C場合]
				ある$(\lambda,\nu)\in \mathbb{C}\times\left( -1-2\Z_{\ge0} \right)$に正則に依存する対称性破れ作用素$\Op^{C}_{\lambda,\nu}\neq0$。
				$Re(\lambda+\nu-n+1)>0$を満たす$\lambda\in\mathbb{C}$に対して、$\mathcal{S}(\Op^{C}_{\lambda,\nu})=C$。さらに、
				\[\Op^{-1}(\Op^{H}_{\lambda,\nu})=\frac{\rho\left(\myabs{x_p}^{\lambda+\nu-n+1}\times\delta^{(-1-\nu}(Q) \right)}{N_C}\]
				ここで$N_C$が上の$N$のような$(\lambda,\nu)$の関数。
			\item[$\left\{ 0 \right\}$場合] ある
				ある$(\lambda,\nu)\in \mathbb{C}^2:\lambda-\nu\in-2\Z_{\ge0}$に正則に依存する対称性破れ作用素$\Op^{ \left\{ 0 \right\} }_{\lambda,\nu}
				\neq0$。
				更に、$\mathcal{S}\left(\Op^{ \left\{ 0 \right\} }_{\lambda,\nu}  \right)=\left\{ [0] \right\}$。更に、
				\[\Op^{-1}(\Op^{ \left\{0\right\}}_{\lambda,\nu})=\tilde{C}_{\nu-\lambda}^{\lambda-(n-1)/2}(-\tilde{\Delta}\tilde{\delta},\delta(x_p))\]
				ここで$\tilde{C}(s,t)$の二変数多項式が\cite[(16.3)]{kobayashi2015symmetry}のように、
				$\tilde{\Delta}$が $(p - 1, q)$-ラプラス作用素, $\tilde{\delta}$ が$\left\{ x_1,\dots,x_{p-1},y_1,\dots,y_q \right\}$
				変数のデルタ関数。
				更に、
				$H\in\Hom_{G'}(I(\lambda),J(\nu))$が$\mathcal{S}(H)\subset\left\{ [0] \right\}$を満たすと、
				$H\in\mathbb{C}\cdot\Op^{ \left\{0\right\}}_{\lambda,\nu}$が成り立つ。
	\end{description}

\end{theorem}

\begin{theorem}
  $p > 1$に対して
  \begin{eqnarray}
    & \mathcal{S} \tmop{ol} (\mathbbm{R}^n ; \lambda \comma \nu) = \left\{
    \begin{array}{ll}
      \mathbbm{C} \tilde{K}_{\lambda, \nu}^{\mathbbm{R}^n} \oplus \mathbbm{C}
      \tilde{K}^{\{ 0 \}}_{\lambda, \nu}, & (\lambda, \nu) \in / /, \nu \in
      L\\
      \mathbbm{C} \tilde{K}^{\mathbbm{R}^n}_{\lambda, \nu}, &
      \tmop{otherwise},
    \end{array} \right. &  \nonumber
  \end{eqnarray}
  ここで $\mid \mid, / /$と $L$ が
  $\mathbbm{C}^2$の知られている部分集合.$p=1$に対して似ている命題も成り立つ。
\end{theorem}

\begin{theorem}
	$1_{\lambda} \in I (\lambda), 1_{\nu} \in I (\nu)$\uline{球面ベクトル}に対して
  \[ \tmop{Op} (\tilde{K}_{\lambda, \nu}^{\mathbbm{R}^n}) 1_{\lambda} = 2^{1 -
     \lambda} \frac{\pi^{n / 2}}{\Gamma \left( \frac{\lambda}{2} \right)
     \Gamma \left( - \frac{q}{2} + \frac{\lambda + 1}{2} \right) \Gamma \left(
     \frac{q - \nu + 1}{2} \right)} 1_{\nu} \quad\mbox{が成り立つ。}\]
  上に記載された$\tilde{K}_{\lambda, \nu}^C, \tilde{K}_{\lambda, \nu}^P$対称性破れ作用素に対して似ている命題も成り立つ。
\end{theorem}

\begin{theorem}
  $(\lambda, \nu) \in / / \assign \{ (\lambda, \nu) \in \mathbbm{C}^2 |
  \lambda - \nu = - 2 k \in - 2\mathbbm{Z}_{\geqslant 0} \}$に対して
  $\tilde{K}^{\mathbbm{R}^n}_{\lambda, \nu} = q_{\{ 0 \}}^{\mathbbm{R}^n}
  (\lambda, \nu) \tilde{K}^{\{ 0 \}}_{\lambda, \nu}$が成り立つ。ここで
  $q^{\mathbbm{R}^n}_{\{ 0 \}} (\lambda, \nu)$ と $q^{\mathbbm{R}^n
  T}_{\mathbbm{R}^n} (\lambda, \nu)$が知られている$(\lambda,\nu)$の関数。
  上に記載された$\tilde{K}_{\lambda, \nu}^C, \tilde{K}_{\lambda, \nu}^P$対称性破れ作用素に対して似ている命題も成り立つ。
\end{theorem}

\begin{theorem}
  $\tilde{\mathbbm{T}}_{\lambda} : I (\lambda) \rightarrow I (n -
  \lambda)$ がKnapp-Stein作用素で、$(\lambda, \nu) \in \mathbbm{C}^2$とする。
  そうすると、以下のような結果が成り立つ:
  \begin{eqnarray}
    & \tilde{\mathbbm{T}}_{n - 1 - \nu} \circ \tmop{Op} (\tilde{K}_{\lambda,
    n - 1 - \nu}^{\mathbbm{R}^n}) = q^{T\mathbbm{R}^n}_{\mathbbm{R}^n}
    (\lambda, \nu) \tmop{Op} (\tilde{K}_{\lambda, \nu}^{\mathbbm{R}^n}) & 
    \nonumber\\
    & \tmop{Op} (\tilde{K}_{n - \lambda, \nu}^{\mathbbm{R}^n}) \circ
    \tilde{\mathbbm{T}}_{\lambda} = q^{\mathbbm{R}^n T}_{\mathbbm{R}^n}
    (\lambda, \nu) \tmop{Op} (\tilde{K}_{\lambda, \nu}^{\mathbbm{R}^n}), & 
    \nonumber
  \end{eqnarray}
  ここで $q^{T\mathbbm{R}^n}_{\mathbbm{R}^n} (\lambda, \nu)$ と
  $q^{\mathbbm{R}^n T}_{\mathbbm{R}^n} (\lambda, \nu)$が知られている関数。
  上に記載された$\tilde{K}_{\lambda, \nu}^C, \tilde{K}_{\lambda, \nu}^P$対称性破れ作用素に対して似ている命題も成り立つ。
\end{theorem}

\begin{thebibliography}{Hör83}
  \bibitem[H{\"o}r83]{hormander1983analysis}L.~H{\"o}rmander.
  {\newblock}\tmtextit{The Analysis of Linear Partial Differential Operators:
  Vol.: 1.: Distribution Theory and Fourier Analysis}.
  {\newblock}Springer-Verlag, 1983.
  
  \bibitem[KS15]{kobayashi2015symmetry}T.~Kobayashi and B.~Speh.
  {\newblock}Symmetry breaking for representations of rank one orthogonal
  groups. {\newblock}\tmtextit{Memoirs of the American Mathematical Society},
  238(1126), 2015.
\end{thebibliography}

\end{document}
