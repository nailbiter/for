%japanese
\documentclass[a4paper,10pt]{article} % use larger type; default would be 10pt

\usepackage{enumerate}
\usepackage{setspace}
\usepackage{geometry,amsmath,amssymb,bbm,xypic}
\usepackage[all,cmtip]{xy}
\usepackage{xeCJK}
\setCJKmainfont{IPAMincho}
\usepackage{amsmath,amssymb,bbm,ulem,float,mystyle}
\usepackage{xeCJK}
\setCJKmainfont{IPAMincho}
\usepackage{caption}
\usepackage{subcaption}

%%%%%%%%%% Start TeXmacs macros
\catcode`\<=\active \def<{
\fontencoding{T1}\selectfont\symbol{60}\fontencoding{\encodingdefault}}
\catcode`\>=\active \def>{
\fontencoding{T1}\selectfont\symbol{62}\fontencoding{\encodingdefault}}
\newcommand{\assign}{:=}
\newcommand{\comma}{{,}}
\newcommand{\nin}{\not\in}
\newcommand{\tmop}[1]{\ensuremath{\operatorname{#1}}}
\newcommand{\tmtextit}[1]{{\itshape{#1}}}
\newcommand{\um}{-}
\newtheorem{theorem}{定理}
\newcommand{\sol}{\mathcal{S}ol(\R^{p,q};\lambda,\nu)}
\newcommand{\Hom}{\mbox{Hom}}
\newcommand{\Op}{\mbox{\normalfont Op}}
\newcommand{\OpR}{\mbox{\normalfont R}}
%%%%%%%%%% End TeXmacs macros

\begin{document}

\uline{以下の}は小林氏とSpeh氏の$O(n+1,1)\downarrow O(n,1)$の対称性破れ作用素の性質についての
論文\cite{kobayashi2015symmetry}を一般化する
目指し、小林俊行氏と共同作業。
%%This is joint work with Toshiyuki Kobayashi aimed to extend the results of his
%%earlier paper with Birgit Speh studying properties of symmetry breaking
%%operators of $O (n, 1)$. 
対称性破れ作用素というのは、$G=O(p+1,q+1)$の極大放物型$P$から誘導して得られ球退化主系列表現
$I(\lambda)$から、$G$の閉部分群$G':=\left\{ g\in G\mid g\cdot e_{p+1}=e_{p+1} \right\}\simeq O(p,q+1)$の極大放物型$P':=G'\cap P$から誘導して得られ球退化主系列表現$J(\nu)
$への$G'$絡み作用素である\sout{(ここで$\left\{ e_i \right\}_{i=1}^{p+q}$が$\R^{p+q}$の標準基底、$g\cdot e_{p+1}$が$O(p+1,q+1)\curvearrowright\R^{p+q}$作用である)}。
%%$G'$退会主系列
%%Symmetry breaking operators (SBOs) are
%%$G'$-intertwining operators between the degenerate principal series
%%representations $I (\lambda)$ and $J (\nu)$ of $G \assign O (p + 1, q + 1)$
%%and its closed subgroup $G' \assign O (p, q + 1)$ respectively, parametrized
%%by $(\lambda, \nu) \in \mathbbm{C}^2$.

得られた結果が以下の通りになる:

\begin{theorem}
$p, q \geqslant 1$\uline{で}、
  $\backslash G/P$の$P'$不変閉集合と\uline{閉包関係}が以下の通りになる:\\
  \[G/P=:X\simeq X^{p,q}:=\Sp^p\times\Sp^q/\left\{ \pm \right\}\]
  \[Y:=\mysetn{[\xi,\eta]\in G/P\simeq \Sp^p\times\Sp^q/\left\{ \pm\right\}}{\xi_{p}=0}=:X^{p-1,q}\]
  \[C:=\mysetn{[\xi,\eta]\in G/P\simeq \Sp^p\times\Sp^q/\left\{ \pm\right\}}{\xi_{0}=\eta_q}\simeq X^{p-1,q-1}\cup\Xi^{p,q}\]
  \[\left\{ [0] \right\}:=\left\{ [1,0_{p+q},1] \right\}\]
  \begin{figure}[H]
    \centering
    \begin{subfigure}{0.3\textwidth}
	\xymatrix{&X\ar@{-}[ld]\ar@{-}[rd]&\\Y\ar@{-}[rd]&&C\ar@{-}[ld]\\&Y\cap C\ar@{-}[d]&\\&\{[0]\}&}
	\caption{$p>1$の時}
    \end{subfigure}
    ~ %add desired spacing between images, e. g. ~, \quad, \qquad, \hfill etc. 
      %(or a blank line to force the subfigure onto a new line)
    \begin{subfigure}{0.3\textwidth}
	\raisebox{40mm}
	{\xymatrix{&X\ar@{-}[ld]\ar@{-}[rd]&\\Y\ar@{-}[rd]&&C\ar@{-}[ld]\\&\{[0]\}&}}
	\caption{$p=1$の時}
    \end{subfigure}
\end{figure}
\end{theorem}

\begin{theorem}
	以下のように$\Op$と$\mathcal{S}$写像を定義する:
	\begin{figure}[H]\centerline{\xymatrixcolsep{7pc}\xymatrix{\Hom_{G'}(I(\lambda),J(\nu))\ar[r]^{\simeq} \ar@/^2pc/[rr]^{\mathcal{S}}
		&\left( \mathcal{D}'(G/P,\mathcal{L}_{n-\lambda})\otimes\mathbb{C}_\nu \right)^{P'}
		\ar[r]_{F\mapsto \supp(F)}\ar[d]^{\simeq}_{\mbox{rest}}
		&P'\backslash G/P\\
		&\sol\subset\mathcal{D}'(\R^{p,q})\ar[lu]^{\mbox{Op}}_{\simeq}&
		}}\end{figure}
	そうすると、次のような結果が成り立つ
	\begin{description}
		\item[case $X$:]$(\lambda,\nu)\in \mathbb{C}^2$に正則に依存する対称性破れ作用素$\OpR^{X}_{\lambda,\nu}:I(\lambda)\to J(\nu)$
			が存在し、次の性質を持つ:
			\begin{enumerate}
			\item 
				$\Re(\lambda+\nu-n),\Re(-\nu)>0$を満たす$(\lambda,\nu)\in \mathbb{C}^2$に対して$\mathcal{S}(\OpR^{X}_{\lambda,\nu})=X$をみたし、
				更に
				\[\Op^{-1}(\OpR^{X}_{\lambda,\nu})=\frac{\myabs{x_p}^{\lambda+\nu-n}\myabs{Q}^{-\nu}}{N(\lambda,\nu)},\quad N(\lambda,\nu):=
					{\Gamma\left( \frac{\lambda-\nu}{2} \right)
					\Gamma\left( \frac{1-\nu}{2} \right)\Gamma\left( \frac{\lambda+\nu-n+1}{2} \right)};\]
			\item
				$\mysetn{(\lambda,\nu)\in\mathbb{C}^2}{\OpR^{X}_{\lambda,\nu}=0}$は$\mathbb{C}^2$にはおける可算無限集合があって、
				具体的に決定できる。
			\end{enumerate}
			\item[case $Y$:]
				$(\lambda,\nu)\in \mathbb{C}^2:\lambda+\nu-n=-1-2k\in-1-2\Z_{\ge0}$に正則に依存する対称性破れ作用素$\OpR^{Y}_{\lambda,\nu}\neq0$
				が存在し、次の性質を持つ:
				\begin{enumerate}
					\item $Re(\nu)<-2k$を満たす$\nu\in\mathbb{C}$に対して、$\mathcal{S}(\OpR^{Y}_{\lambda,\nu})=H$;
					\item 
					\[\Op^{-1}(\OpR^{Y}_{\lambda,\nu})=\frac{\rho\left( \delta^{(2k)}(x_p)\times\myabs{Q}^{-\nu} \right)}{N_Y(\lambda,\nu)},\quad
					Q:=\sum_{i=1}^px_i^2-\sum_{j=p+1}^{p+q}x_j^2\mbox{。}\]
				\end{enumerate}
				ここで $a \nin - n -\mathbbm{Z}_{\geqslant 0}$-次の斉次性を持つ超関数$F \in \mathcal{D}' (\mathbbm{R}^n \backslash \{ 0 \})$
				に対して$\rho(F)\in\mathcal{D}' (\mathbbm{R}^n)$ が唯一$a$-次の斉次性を持つ$\rho(F)\big|_{\R^n\setminus\left\{ 0 \right\}}=F$を
				満たす超関数(\uline{cf}. {\cite[thm. 3.2.3]{hormander1983analysis}}) 。$\times$が
				$\mathbb{R}^n\setminus\left\{ 0 \right\}$上の超関数積(cf. {\cite[thm. 8.2.10]{hormander1983analysis}})。
				$N_Y(\lambda,\nu)$が上の$N(\lambda,\nu)$のような$(\lambda,\nu)$の関数。$\delta^{(l)}(x)$が一変数デルタ関数の$l$階微分である。
			\item[case $C$:]
				$(\lambda,\nu)\in \mathbb{C}\times\left( -1-2\Z_{\ge0} \right)$に正則に依存する対称性破れ作用素$\OpR^{C}_{\lambda,\nu}\neq0$
				が存在し、次の性質を持つ:
				\begin{enumerate}
					\item $Re(\lambda+\nu-n+1)>0$を満たす$\lambda\in\mathbb{C}$に対して、$\mathcal{S}(\OpR^{C}_{\lambda,\nu})=C$;
					\item \[\Op^{-1}(\OpR^{C}_{\lambda,\nu})=\frac{\rho\left(\myabs{x_p}^{\lambda+\nu-n+1}\times\delta^{(-1-\nu}(Q) \right)}{N_C
						(\lambda,\nu)}
					\mbox{。}\]
				\end{enumerate}
				ここで$N_C(\lambda,\nu)$が上の$N(\lambda,\nu)$のような$(\lambda,\nu)$の関数。
			\item[case $\left\{ 0 \right\}$:] 
				$(\lambda,\nu)\in \mathbb{C}^2:\lambda-\nu\in-2\Z_{\ge0}$に正則に依存する対称性破れ作用素$\OpR^{ \left\{ 0 \right\} }_{\lambda,\nu}\neq0$
				が存在し、次の性質を持つ:
				\begin{enumerate}
					\item $\mathcal{S}\left(\OpR^{ \left\{ 0 \right\} }_{\lambda,\nu}  \right)=\left\{ [0] \right\}$;
					\item 
				\[\Op^{-1}(\OpR^{ \left\{0\right\}}_{\lambda,\nu})=\tilde{C}_{\nu-\lambda}^{\lambda-(n-1)/2}(-\tilde{\Delta}\tilde{\delta},\delta(x_p))
				\mbox{。}\]
				ここで$\tilde{C}(s,t)$の二変数多項式が\cite[(16.3)]{kobayashi2015symmetry}のように、
				$\tilde{\Delta}$が $(p - 1, q)$-ラプラシアン, $\tilde{\delta}$ が$\left\{ x_1,\dots,x_{p-1},y_1,\dots,y_q \right\}$
				変数のデルタ関数;
					\item
				$R\in\Hom_{G'}(I(\lambda),J(\nu))$が$\mathcal{S}(R)\subset\left\{ [0] \right\}$を満たすと、
				$R\in\mathbb{C}\cdot\OpR^{ \left\{0\right\}}_{\lambda,\nu}$が成り立つ。
				\end{enumerate}
	\end{description}
\end{theorem}

\begin{theorem}[対称性破れ作用素の分類]
  $p > 1$に対して
  \begin{eqnarray}
	  & \Hom_{G'}(I(\lambda),J(\nu))= \left\{
    \begin{array}{ll}
      \mathbbm{C} {\OpR}_{\lambda, \nu}^{X} \oplus \mathbbm{C}
      {\OpR}^{\{ 0 \}}_{\lambda, \nu}, & (\lambda, \nu) \in / /, \nu \in
      L\\
      \mathbbm{C} \OpR^X_{\lambda, \nu}, &
      \tmop{otherwise},
    \end{array} \right. &  \nonumber
  \end{eqnarray}
  得に、$\dim\Hom_{G'}(I(\lambda),J(\nu))=1\iff(\lambda,\nu)\in//,\nu\in L$。
  ここで $\mid \mid, / /$と $L\subset\mathbb{C}^2$ 
  を具体的に決定できる。
  $p=1$に対して同様の命題が成り立つ。
\end{theorem}

\begin{theorem}[$K$不変ベクトルにおける``固有値'']
	$1_{\lambda} \in I (\lambda), 1_{\nu} \in I (\nu)${$K$(と$K'$)不変ベクトル}に対して
  \[ \OpR^X_{\lambda, \nu} 1_{\lambda} = 2^{1 -
     \lambda} \frac{\pi^{n / 2}}{\Gamma \left( \frac{\lambda}{2} \right)
     \Gamma \left( - \frac{q}{2} + \frac{\lambda + 1}{2} \right) \Gamma \left(
     \frac{q - \nu + 1}{2} \right)} 1_{\nu} \quad\mbox{が成り立つ。}\]
     上に記載された$\OpR_{\lambda,\nu}^C, \OpR_{\lambda,\nu}^Y$対称性破れ作用素に対して同様な命題も成り立つ。
\end{theorem}
\begin{theorem}[留数定理]
  $(\lambda, \nu) \in / / \assign \{ (\lambda, \nu) \in \mathbbm{C}^2 |
  \lambda - \nu = - 2 k \in - 2\mathbbm{Z}_{\geqslant 0} \}$に対して
  $\OpR^{X}_{\lambda, \nu} = q_{\{ 0 \}}^{X}
  (\lambda, \nu) \OpR^{\{ 0 \}}_{\lambda, \nu}$が成り立つ。ここで
  $q^{X}_{\{ 0 \}} (\lambda, \nu)$ 
  が具体的に決定できる$(\lambda,\nu)$の関数。
  上に記載された$\OpR_{\lambda, \nu}^C, \OpR_{\lambda, \nu}^Y$対称性破れ作用素に対して同様な命題も成り立つ。
\end{theorem}
\begin{theorem}[factorization identities]
  $\tilde{\mathbbm{T}}_{\lambda} : I (\lambda) \rightarrow I (n -
  \lambda)$ がKnapp-Stein作用素で、$(\lambda, \nu) \in \mathbbm{C}^2$とする。
  そうすると、以下のような結果が成り立つ:
  \begin{eqnarray}
    & \tilde{\mathbbm{T}}_{n - 1 - \nu} \circ \OpR_{\lambda,
    n - 1 - \nu}^{X} = q^{T X}_{X}
    (\lambda, \nu) \OpR_{\lambda, \nu}^{X} & 
    \nonumber\\
    & \OpR_{n - \lambda, \nu}^X \circ
    \tilde{\mathbbm{T}}_{\lambda} = q^{X T}_{X}
    (\lambda, \nu) \OpR_{\lambda, \nu}^{X}, & 
    \nonumber
  \end{eqnarray}
  ここで $q^{T{X}}_{{X}} (\lambda, \nu)$ と
  $q^{{X} T}_{{X}} (\lambda, \nu)$が具体的に決定できる関数。
  上に記載された$\OpR_{\lambda, \nu}^C, \OpR_{\lambda, \nu}^Y$対称性破れ作用素に対して同様な命題も成り立つ。
\end{theorem}

\begin{thebibliography}{Hör83}
  \bibitem[H{\"o}r83]{hormander1983analysis}L.~H{\"o}rmander.
  {\newblock}\tmtextit{The Analysis of Linear Partial Differential Operators:
  Vol.: 1.: Distribution Theory and Fourier Analysis}.
  {\newblock}Springer-Verlag, 1983.
  
  \bibitem[KS15]{kobayashi2015symmetry}T.~Kobayashi and B.~Speh.
  {\newblock}Symmetry breaking for representations of rank one orthogonal
  groups. {\newblock}\tmtextit{Memoirs of the American Mathematical Society},
  238(1126), 2015.
\end{thebibliography}

\end{document}
