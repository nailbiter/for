%platex
\documentclass[12pt]{msjproc} % use larger type; default would be 10pt

\usepackage{enumerate}
\usepackage{setspace}
\usepackage{amsmath,amssymb,bbm,xypic}
\usepackage[all,cmtip]{xy}
\usepackage{amsmath,amssymb,bbm,ulem,float,mystyle}
\usepackage{caption}
\usepackage{subcaption}
\usepackage{setspace}

%%%%%%%%%% Start TeXmacs macros
\catcode`\<=\active \def<{
\fontencoding{T1}\selectfont\symbol{60}\fontencoding{\encodingdefault}}
\catcode`\>=\active \def>{
\fontencoding{T1}\selectfont\symbol{62}\fontencoding{\encodingdefault}}
\newcommand{\assign}{:=}
\newcommand{\comma}{{,}}
\newcommand{\nin}{\not\in}
\newcommand{\tmop}[1]{\ensuremath{\operatorname{#1}}}
\newcommand{\tmtextit}[1]{{\itshape{#1}}}
\newcommand{\um}{-}
\newtheorem{theorem}{定理}
\newcommand{\sol}{\mathcal{S}ol(\R^{p,q};\lambda,\nu)}
\newcommand{\Hom}{\mbox{\normalfont Hom}}
\newcommand{\Op}{\mbox{\normalfont Op}}
\newcommand{\OpR}{\mbox{\it R}}
\newtheorem{remark}{注意}

\setlength{\parskip}{0.4em}
\setlength{\parindent}{2em}
%%%%%%%%%% End TeXmacs macros

\begin{document}

\title{不定値直交群$O(p,q)$の対称性破れ作用素}

  %%%% 講演者1
  \author{小林俊行}{東京大学}
  \author{レオンチエフ アレックス}{東京大学}

  %%%% 講演者2

  %%%% 日付
%  \date{2012年3月26日}

  %%%% 謝辞、キーワード、MSCコード  

  \maketitle

\begin{spacing}{0.8}
{以下で}は小林氏とSpeh氏の$O(n+1,1)\downarrow O(n,1)$の対称性破れ作用素についての
論文\cite{kobayashi2015symmetry}の一般化を
目指し、小林俊行氏と共同研究。
対称性破れ作用素としてここで扱うのは、$G=O(p+1,q+1)$の最大次元の放物型部分群$P$から誘導して得られる球退化主系列表現
$I(\lambda)$から、$G$の閉部分群$G'\simeq O(p,q+1)$の球退化主系列表現$J(\nu)$への$G'$絡み作用素である。
%%$G'$退会主系列
%%Symmetry breaking operators (SBOs) are
%%$G'$-intertwining operators between the degenerate principal series
%%representations $I (\lambda)$ and $J (\nu)$ of $G \assign O (p + 1, q + 1)$
%%and its closed subgroup $G' \assign O (p, q + 1)$ respectively, parametrized
%%by $(\lambda, \nu) \in \mathbbm{C}^2$.
\end{spacing}

{以下では}$\R^{p+1,q+1}$の符号$(p+1,q+1)$の標準二次形式を$Q_{p+1,q+1}$と表し、
$G$を$Q_{p+1,q+1}$を保つ群、$\Xi^{p+1,q+1}:=\mysetn{x\in\R^{p+1,q+1}\setminus\left\{ 0 \right\}}{Q_{p+1,q+1}(x)=0}$、$P$を
$[e_0+e_{p+q+1}]\in X^{p,q}:=\Xi^{p+1,q+1}/\R^{\times}$の固定部分群、$G'$を$e_p$の固定部分群{として}実現し、$P':=P\cap G'$とおく。

\[
	X:=G/P\simeq X^{p,q},\quad Y:=\mysetn{[\xi,\eta]\in G/P\simeq X^{p,q}}{\xi_{p}=0}\simeq X^{p-1,q}\]
	\[C:=\mysetn{[\xi,\eta]\in G/P\simeq X^{p,q}}{\xi_{0}=\eta_q}\simeq X^{p-1,q-1}\cup\Xi^{p,q},\quad\left\{ [0] \right\}:=\left\{ [1,0_{p+q},1] \right\}
  \mbox{とおく。}\]

\begin{theorem}
$p, q \geqslant 1$に対して、
  $G/P$の$P'$不変閉集合と{包含関係}が以下の通りになる:\\
  \begin{figure}[H]
	  
    \centering
    \begin{subfigure}{0.3\textwidth}
	\xymatrix{&X\ar@{-}[ld]\ar@{-}[rd]&\\Y\ar@{-}[rd]&&C\ar@{-}[ld]\\&Y\cap C\ar@{-}[d]&\\&\{[0]\}&}
	\caption{$p>1$の時}
    \end{subfigure}
    ~ %add desired spacing between images, e. g. ~, \quad, \qquad, \hfill etc. 
      %(or a blank line to force the subfigure onto a new line)
    \begin{subfigure}{0.3\textwidth}
	\raisebox{40mm}
	{\xymatrix{&X\ar@{-}[ld]\ar@{-}[rd]&\\Y\ar@{-}[rd]&&C\ar@{-}[ld]\\&\{[0]\}&}}
	\caption{$p=1$の時}
    \end{subfigure}
\end{figure}
\end{theorem}

以下では
の2つの写像$\Op$と$\mathcal{S}$を定義する:
\begin{figure}[H]
	  
	\centerline{\xymatrixcolsep{7pc}\xymatrix{\Hom_{G'}(I(\lambda),J(\nu))\ar[r]^{\simeq} \ar@/^2pc/[rr]^{\mathcal{S}}
	&\left( \mathcal{D}'(G/P,\mathcal{L}_{n-\lambda})\otimes\mathbb{C}_\nu \right)^{P'}
	\ar[r]_{F\mapsto \supp(F)}\ar[d]^{\simeq}_{\mbox{rest}}
	&2^{P'\backslash G/P}\\
	&\sol\subset\mathcal{D}'(\R^{p,q})\ar[lu]^{\mbox{Op}}_{\simeq}&
	}}
	  
\end{figure}
\begin{theorem}[対称性破れ作用素の構成]
次のような結果が成り立つ:
	\begin{description}
		\item[regular case:]$(\lambda,\nu)\in \mathbb{C}^2$に正則に依存する対称性破れ作用素$\OpR^{X}_{\lambda,\nu}:I(\lambda)\to J(\nu)$
			が存在し、次の性質を持つ:
			\begin{enumerate}
			\item 
				$\Re(\lambda+\nu-n),\Re(-\nu)>0$ならば$\mathcal{S}(\OpR^{X}_{\lambda,\nu})=X$をみたし、
				更に
				\[\Op^{-1}(\OpR^{X}_{\lambda,\nu})={N(\lambda,\nu)}{\myabs{x_p}^{\lambda+\nu-n}\myabs{Q}^{-\nu}},\quad N^{-1}(\lambda,\nu):=
					{\Gamma\left( \scalebox{0.8}{$\frac{\lambda-\nu}{2}$} \right)
					\Gamma\left( \scalebox{0.8}{$\frac{1-\nu}{2}$} \right)
				\Gamma\left( \scalebox{0.8}{$\frac{\lambda+\nu-n+1}{2}$} \right)};\]
			\item
				$\mysetn{(\lambda,\nu)\in\mathbb{C}^2}{\OpR^{X}_{\lambda,\nu}=0}$は$\mathbb{C}^2$における可算無限集合であって、
				具体的に決定できる。
			\end{enumerate}
			\item[特異積分$Y$:]
				{$k\in\Z_{\ge0},\nu\in\mathbb{C},\lambda:=n-2k-1-\nu$とおく。$\nu\in\mathbb{C}$に}
				正則に依存する対称性破れ作用素$\OpR^{Y}_{\lambda,\nu}\neq0$
				が存在し、次の性質を持つ。
					$\Re(\nu)<-2k$を満たす$\nu\in\mathbb{C}$に対して、$\mathcal{S}(\OpR^{Y}_{\lambda,\nu})=Y$、更に
						\[\Op^{-1}(\OpR^{Y}_{\lambda,\nu})={N_Y(\lambda,\nu)}{\delta^{(2k)}(x_p)\times\myabs{Q_{p,q}}
							^{-\nu}}.\]
				ここで $a \nin - n -\mathbbm{Z}_{\geqslant 0}$-次の斉次性を持つ超関数$F \in \mathcal{D}' (\mathbbm{R}^n \backslash \{ 0 \})$
				は一意的に拡張できるので、同じ記号で表記した。
				$\times$は
				$\mathbb{R}^n\setminus\left\{ 0 \right\}$上の超関数積。
				$\delta^{(l)}(x)$は一変数デルタ関数の$l$階微分である。
			\item[特異積分$C$:]
				$\nu\in-1-2\Z_{\ge0}$とする。
				$\lambda\in \mathbb{C}$に正則に依存する対称性破れ作用素$\OpR^{C}_{\lambda,\nu}\neq0$
				が存在し、次の性質を持つ。
				 $\Re(\lambda+\nu-n+1)>0$を満たす$\lambda\in\mathbb{C}$に対して、$\mathcal{S}(\OpR^{C}_{\lambda,\nu})=C$、更に
					 \[\Op^{-1}(\OpR^{C}_{\lambda,\nu})={N_C
						(\lambda,\nu)}{\myabs{x_p}^{\lambda+\nu-n+1}\times\delta^{(-1-\nu)}(Q_{p,q}) }
					.\]
			\item[微分対称性破れ作用素:] 
				{$k\in\Z_{\ge0},\nu\in\mathbb{C},\nu:=\lambda+2k$とおく。$\lambda\in\mathbb{C}$に}
				に正則に依存する対称性破れ作用素$\OpR^{ \left\{ 0 \right\} }_{\lambda,\nu}\neq0$
				が存在し、次の性質を持つ:
				\begin{enumerate}
					\item $\mathcal{S}\left(\OpR^{ \left\{ 0 \right\} }_{\lambda,\nu}  \right)=\left\{ [0] \right\}$、更に
		\[\Op^{-1}(\OpR^{ \left\{0\right\}}_{\lambda,\nu})=\tilde{C}_{\nu-\lambda}^{\lambda-(n-1)/2}(-\tilde{\Delta}\tilde{\delta},\delta(x_p)).
				\]
				ここで$\tilde{C}(s,t)$は \cite[(16.3)]{kobayashi2015symmetry}で与えられた二変数多項式、
				$\tilde{\Delta}$は$(p - 1, q)$-ラプラシアン, $\tilde{\delta}$ が$\left\{ x_1,\dots,x_{p-1},y_1,\dots,y_q \right\}$を変数とするデルタ関数;
					\item
				$R\in\Hom_{G'}(I(\lambda),J(\nu))$が$\mathcal{S}(R)\subset\left\{ [0] \right\}$を満たせば、
				$R\in\mathbb{C}\cdot\OpR^{ \left\{0\right\}}_{\lambda,\nu}$が成り立つ。
				\end{enumerate}
	\end{description}
\end{theorem}
\begin{theorem}[対称性破れ作用素の分類]
  $p > 1$に対して
  \begin{eqnarray}
	  & \Hom_{G'}(I(\lambda),J(\nu))= \left\{
    \begin{array}{ll}
      \mathbbm{C} {\OpR}_{\lambda, \nu}^{X} \oplus \mathbbm{C}
      {\OpR}^{\{ 0 \}}_{\lambda, \nu}, & (\lambda, \nu) \in / /\cap 
      L\\
      \mathbbm{C} \OpR^X_{\lambda, \nu}, &
      \mbox{\normalfont otherwise.}
    \end{array} \right. &  \nonumber
  \end{eqnarray}
  特に、$\dim\Hom_{G'}(I(\lambda),J(\nu))=1\iff(\lambda,\nu)\in//\cap L$。
  ここで、$//\assign \{ (\lambda, \nu) \in \mathbbm{C}^2 |
  \lambda - \nu = - 2 k \in - 2\mathbbm{Z}_{\geqslant 0} \}$。$L\subset\mathbb{C}^2$は
  複数余次元$1$の部分集合(具体形は省略する)。
  $p=1$に対して同様の命題が成り立つ。
\end{theorem}

\begin{theorem}[$K$不変ベクトルにおける``固有値'']
	正規化された$K$不変ベクトル$1_{\lambda} \in I (\lambda)$と$K'$不変ベクトル$1_{\nu} \in I (\nu)$に対して
	\[ \OpR^X_{\lambda, \nu} 1_{\lambda} = 2^{1 -
     \lambda} \frac{\pi^{n / 2}}{\Gamma \left( \frac{\lambda}{2} \right)
     \Gamma \left( - \frac{q}{2} + \frac{\lambda + 1}{2} \right) \Gamma \left(
     \frac{q - \nu + 1}{2} \right)} 1_{\nu} \quad\mbox{が成り立つ。}\]
\end{theorem}

\begin{theorem}[留数定理]
  $(\lambda, \nu) \in / / $に対して
  $\OpR^{X}_{\lambda, \nu} = q_{\{ 0 \}}^{X}
  (\lambda, \nu) \OpR^{\{ 0 \}}_{\lambda, \nu}$が成り立つ。
\end{theorem}
\begin{theorem}[factorization identities]
  $\tilde{\mathbbm{T}}_{\lambda} : I (\lambda) \rightarrow I (n -
  \lambda)$ をKnapp-Stein作用素とする。$(\lambda, \nu) \in \mathbbm{C}^2$に対して
  $\tilde{\mathbbm{T}}_{n - 1 - \nu} \circ \OpR_{\lambda,
    n - 1 - \nu}^{X} = q^{T X}_{X}
    (\lambda, \nu) \OpR_{\lambda, \nu}^{X}$と$ \OpR_{n - \lambda, \nu}^X \circ
    \tilde{\mathbbm{T}}_{\lambda} = q^{X T}_{X}
    (\lambda, \nu) \OpR_{\lambda, \nu}^{X}$が成り立つ。
\end{theorem}
\begin{remark}
	上に記述された $N_C(\lambda,\nu)$、$N_Y(\lambda,\nu)$、$q_{ \left\{ 0 \right\}}^{X}(\lambda,\nu)$、$q_X^{TX}(\lambda,\nu)$、$q_X^{XT}(\lambda,\nu)$
	は$\Gamma$関数を用いて明示的に記述される。
\end{remark}
\begin{theorem}[対称性破れ作用素の像]
%%	$\nu\notin\Z$ならば、$R_{\lambda,\nu}^X$が全射になる。与えられた$\nu\in\Z$に対した、$R_{\lambda,\nu}^X$の像
%%	を具体的に決定できる。
	$\nu$が整数でなければ、$J(\nu)$は既約である。
	この場合は$R_{\lambda,\nu}^X$は$(\mathfrak{g},K)$-加群のレベルで全射となる。
	$\nu$が既約でない場合は全射とは限らないが、その像が、
	どのような$G'$の表現としてどのような表現であるか $(\mathfrak{g},K)$-加群のレベルで
	完全に記述することができる。
\end{theorem}
\begin{remark}
  	上に記載された対称性破れ作用素$\OpR_{\lambda, \nu}^C, \OpR_{\lambda, \nu}^Y$に対して定理4、5、6、7と同様な命題も成り立つ。
\end{remark}
\bibliographystyle{alpha}
\bibliography{todai_master}
\end{document}
