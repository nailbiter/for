%texmacs ~/for/fortexmacs/master_extract.tm
\documentclass[10pt]{article} % use larger type; default would be 10pt

\usepackage{enumerate}
\usepackage[left=0.6in,top=1in,bottom=1in]{geometry}
\usepackage{setspace}
\usepackage{amsmath,amssymb,bbm,xypic}
\usepackage[all,cmtip]{xy}
\usepackage{amsmath,amssymb,bbm,ulem,float,mystyle}
\usepackage{caption}
\usepackage{subcaption}
\usepackage{setspace}
\usepackage{comment}
\usepackage{catchfilebetweentags}
\usepackage{multirow}
\usepackage[table]{xcolor}
\includecomment{versiona}

%%%%%%%%%% Start TeXmacs macrosoeuoeu
\catcode`\<=\active \def<{
\fontencoding{T1}\selectfont\symbol{60}\fontencoding{\encodingdefault}}
\catcode`\>=\active \def>{
\fontencoding{T1}\selectfont\symbol{62}\fontencoding{\encodingdefault}}
\newcommand{\assign}{:=}
\newcommand{\comma}{{,}}
\newcommand{\nin}{\not\in}
\newcommand{\tmop}[1]{\ensuremath{\operatorname{#1}}}
\newcommand{\tmtextit}[1]{{\itshape{#1}}}
\newcommand{\um}{-}
\newtheorem{theorem}{Theorem}
\newcommand{\sol}{\mathcal{S}ol(\R^{p,q};\lambda,\nu)}
\newcommand{\Hom}{\mbox{\normalfont Hom}}
\newcommand{\Sol}{\mathcal{S}ol}
\newtheorem{remark}{Remark}
\newtheorem{fact}{Fact}
%\newtheorem{definition}{Definition}
\theoremstyle{definition}
\newtheorem{definition}{Definition}

\catcode`\<=\active \def<{
\fontencoding{T1}\selectfont\symbol{60}\fontencoding{\encodingdefault}}
\catcode`\>=\active \def>{
\fontencoding{T1}\selectfont\symbol{62}\fontencoding{\encodingdefault}}
\newcommand{\dueto}[1]{\textup{\textbf{(#1) }}}
\newcommand{\tmrsub}[1]{\ensuremath{_{\textrm{#1}}}}
\newcommand{\tmrsup}[1]{\textsuperscript{#1}}
\newcommand{\tmtextbf}[1]{{\bfseries{#1}}}
\newtheorem{proposition}{Proposition}
\newcommand{\Op}{\mbox{\normalfont Op}}
\newcommand{\Res}{\operatorname{Res}\displaylimits}
\newcommand{\OpR}{\mbox{\it R}}
%%%%%%%%%% End TeXmacs macros

\setlength{\parskip}{0.4em}
\setlength{\parindent}{2em}

%commands for socles
\newcommand{\mycolo}{\cellcolor{blue!25}}
\newcommand{\socle}[1]{{
\begin{tabular}{|c|}
	\hline
	$#1$\\
	\hline
\end{tabular}
}}
\newcommand{\socleA}[2]{
{\begin{tabular}{|c|}
\hline
$#2$\\
\hline
$#1$\\
\hline
\end{tabular}}}
\newcommand{\socleAP}[3]{{
\begin{tabular}{|c|c|c|}
\hline
$#2$ & $\oplus$  & $#3$ \\
\hline
\multicolumn{3}{|c|}{$#1$}\\
\hline
\end{tabular}
}}
\newcommand{\socleP}[2]{{
\begin{tabular}{|c|c|c|}
\hline
$#1$ & $\oplus$  & $#2$ \\
\hline
\end{tabular}
}}
\newcommand{\soclePA}[3]{{
\begin{tabular}{|c|c|c|}
\hline
\multicolumn{3}{|c|}{$#3$}\\
\hline
$#1$ & $\oplus$  & $#2$ \\
\hline
\end{tabular}
}}
\newcommand{\socleAA}[3]{{
\begin{tabular}{|c|}
\hline
$#3$\\
\hline
$#2$\\
\hline
$#1$\\
\hline
\end{tabular}
}}
\newcommand{\socleAAP}[4]{{
\begin{tabular}{|c|c|c|}
\hline
$#3$ & $\oplus$  & $#4$ \\
\hline
\multicolumn{3}{|c|}{$#2$}\\
\hline
\multicolumn{3}{|c|}{$#1$}\\
\hline
\end{tabular}
}}
\newcommand{\soclePAA}[4]{{
\begin{tabular}{|c|c|c|}
\hline
\multicolumn{3}{|c|}{$#4$}\\
\hline
\multicolumn{3}{|c|}{$#3$}\\
\hline
$#1$ & $\oplus$  & $#2$ \\
\hline
\end{tabular}
}}

\newcommand{\even}{2\Z}
\newcommand{\odd}{2\Z+1}
\newcommand{\bb}{\backslash\backslash}
\renewcommand{\ss}{//}
%%%%%%%%%% End TeXmacs macros

\begin{document}

\title{Symmetry breaking operators of indefinite orthogonal groups $O(p,q)$ (60 minutes talk)}

  %%%% 講演者1
  \author{Toshiyuki Kobayashi (The University of Tokyo, Kavli IPMU)\\
  Alex Leontiev (The University of Tokyo)}

  %%%% 講演者2

  %%%% 日付
%  \date{2012年3月26日}

  %%%% 謝辞、キーワード、MSCコード  

  \maketitle
{\tableofcontents}
\section{Setting and Background (What are SBOs?)}

\newdir{:=}{{}}
  \begin{figure}[H]\centering
		%\xymatrixcolsep{5pc}
		\xymatrix{
			& \mathcal{L}_\lambda\mbox{ :conformally equivariant line bundle},\lambda\in\mathbb{C}
			\ar[d]\\
  		G=O(p+1,q+1)
		\ar@/^2pc/[r] &G/P\simeq (\Sp^p\times\Sp^q)/\left\{ \pm I \right\}\\
		P=MAN\ar@{:=}[u]_{\hspace{-0.25cm}\bigcup}
		\ar@/^2pc/[rd]^{{\begin{array}{c}\; \\\mbox{conformal transformations}\end{array}}}
		%\mbox\newline oeueou}\vspace{0.8cm}}
		&\\
	M_+N=O(p,q)\ltimes \mathbb{R}^{p,q}
	\ar@{:=}[u]_{\hspace{-0.25cm}\bigcup}
	\ar@/^2pc/[r]^{\mbox{isometries}}&
	\mathbb{R}^{p,q}=\left( \mathbb{R}^{p+q},ds^2=dx_1^2+\ldots+dx_p^2-dx_{p+1}^2-\ldots-dx_{p+q}^2 \right)\ar@{^{(}->}[uu]
	_{\mbox{conformal 
	compactification}}
	\vspace{2cm}
		}
  \end{figure}
Similarly, for $O(p,q+1)\simeq G'\subset G$ and $\nu\in\C$ we defined conformally equivariant line bundle $J(\nu)$. Similarly to above, it is isomorphic to degenerate spherical principal series
as $G'$-module.
\[\begin{array}{ccc}

G:=O(p+1,q+1)&\curvearrowright &I(\lambda):=C^\infty(G/P,\mathcal{L}_\lambda)\\
\bigcup&&\\
G':=O(p,q+1)&\curvearrowright &J(\nu):=C^\infty(G'/P',\mathcal{L}_\nu)
\end{array}\]
Main Questions:
\begin{enumerate}
	\item For any given $(\lambda,\nu)\in\C^2$ describe all elements of $\Hom_{G'}(I(\lambda),J(\nu)$ -- the space of \textit{symmetry breaking operators} (aka. SBOs). Equivalently,
		find a basis of $\Hom_{G'}(I(\lambda),J(\nu))$ for every $(\lambda,\nu)\in\C^2$.
	\item Study special properties of SBOs (ie. functional identities, images etc.)
\end{enumerate}
\begin{remark}
	Work of \cite{kobayashi2015classification,kobayashi2014classification} implies that $\dim\Hom_{G'}(I(\lambda),J(\nu))$ is uniformly bounded in $(\lambda,\nu)\in\C^2$.
\end{remark}
\section{Motivation (Why study SBOs?)}
Two main reasons:\begin{enumerate}
	\item This is \uline{simple} question with \uline{simple} answer (cf. Poincare hypothesis, Riemann Hypothesis)
	\item SBO interact with many areas:
\end{enumerate}
\begin{description}
	\item[Representation theory:] Recall branching for compact groups: when $\pi\in \hat{G}$: finitely-dimensional, irreducible, we have
		\begin{equation*}
		\pi\big|_{G'}=\sum_{\tau\in\hat{G'}}m(\pi,\tau)\tau
		\end{equation*}
		However, when $\pi$: infinitely-dimensional or $G'$:non-compact, the problem becomes much more involved. In particular, one has several
		\textit{inequivalent} definitions of $m(\pi,\tau)$. In particular, one shown to be particularly good is $m(\pi,\tau):=\Hom_{G'}(\pi,\tau)$;
	\item[Conformal Geometry:] Juhl's equivariant differential operators \cite{juhl2009families} are differential SBOs;
	\item[Number Theory:] Rankin-Cohen brackets are differential SBOs \cite{kobayashi2015differential1};
	\item[Geometry of generalized flag variety:] The $P'$-invariant closed subsets $G/P$ and their closure relations determine the structure of SBO space;
	%\item[Generalized functions:]
\end{description}
\section{Main Results (How much can we say about SBOs of $O(p,q)$?)}
  \begin{versiona}
\end{versiona}

\begin{fact}[Kobayashi-Speh]\label{fact1}
Let $n:=p+q$. The following diagram commutes:
\begin{figure}[H]
\centerline{
	\xymatrixcolsep{5pc}
	\xymatrix{\Hom_{G'}(I(\lambda),J(\nu))\ar[r]^{\simeq} \ar@/^2pc/[rr]^{\mathcal{S}}
	&\left( \mathcal{D}'(G/P,\mathcal{L}_{n-\lambda}) \otimes\mathbb{C}_\nu \right)^{P'}
\ar[r]_-{F\mapsto \supp(F)}\ar[d]^{\simeq}_{\mbox{rest}}
&2^{P'\backslash G/P}\\
&\sol\subset\mathcal{D}'(\R^{p,q})\ar[lu]^{\mbox{Op}}_{\simeq}&
}
}
\end{figure}
\end{fact}

Note that $G$ acts on $\Xi^{p+1,q+1}:=\mysetn{(x,y)\in\R^{p+1,q+1}\setminus\left\{ 0 \right\}}{\myabs{x}^2=\myabs{y}^2}$ and on its quotient space
$X^{p,q}:=\Xi^{p+1,q+1}/\R^{\times}\simeq G/P$. Let
\[
	X:=G/P\simeq X^{p,q},\quad Y:=\mysetn{[\xi:\eta]\in G/P\simeq X^{p,q}}{\xi_{p}=0}\simeq X^{p-1,q}\]
	\[C:=\mysetn{[\xi:\eta]\in G/P\simeq X^{p,q}}{\xi_{0}=\eta_q}\simeq X^{p-1,q-1}\cup\Xi^{p,q},\quad\left\{ [0] \right\}:=\left\{ [1,0_{p+q},1] \right\}\]
\begin{theorem}[classification of closed $P'$-invariant subsets of $G/P$]
	The left $P'$-invariant closed subspaces of $G/P$ are as follows (numbers indicate codimension):\\
  \begin{figure}[H]
    \centering
    \begin{subfigure}[t]{0.3\textwidth}
	    \xymatrixrowsep{0.5pc}
	    \xymatrix{&X\ar@{-}[ld]_1\ar@{-}[rd]^1&\\Y\ar@{-}[rd]_1&&C\ar@{-}[ld]^1\\&Y\cap C\ar@{-}[dd]^{p+q-2}&\\&&\\&\{[0]\}&}
	\caption{when $p>1$}
    \end{subfigure}
    ~ %add desired spacing between images, e. g. ~, \quad, \qquad, \hfill etc. 
      %(or a blank line to force the subfigure onto a new line)
    \begin{subfigure}[t]{0.3\textwidth}
	    \xymatrixrowsep{0.5pc}
	    {\xymatrix{&X\ar@{-}[ld]_1\ar@{-}[rd]^1&\\Y\ar@{-}[rddd]_{p+q-2}&&C\ar@{-}[lddd]^{p+q-2}\\&&\\&&\\&\{[0]\}&}}
	\caption{when $p=1$}
    \end{subfigure}
\end{figure}
\end{theorem}
\begin{theorem}[construction of SBOs]\label{thm:construction}
We can construct the following families of SBOs which holomorphically depend on parameters:\\
\begin{tabular}{|l|l|l|l|}
  \hline
  & $\tmop{Op} : \mathcal{S} \tmop{ol} (\mathbbm{R}^{p, q} ; \lambda, \nu)
  \rightarrow \tmop{Hom}_{G'} (I (\lambda), J (\nu))$ & defined for &
  $\mathcal{S} (\cdot) =$\\
  \hline
  $R_{\lambda, \nu}^X =$ & $\tmop{Op} \left( \frac{| x_p |^{\lambda + \nu - n}
  | Q |^{- \nu}}{\Gamma \left( \frac{\lambda - \nu}{2} \right) \Gamma \left(
  \frac{\lambda + \nu - n + 1}{2} \right) \Gamma \left( \frac{1 - \nu}{2}
  \right)} \right)$ & $(\lambda, \nu) \in \mathbbm{C}^2$ & generically $=X$\\
  \hline
  $\tilde{R}^X_{\lambda, \nu} =$ & $\tmop{Op} \left( \frac{| x_p |^{\lambda +
  \nu - n} | Q |^{- \nu}}{\Gamma \left( \frac{\lambda + \nu - n + 1}{2}
  \right) \Gamma \left( \frac{1 - \nu}{2} \right)} \right)$ & $(\lambda, \nu)
  \in \mid \mid \mid$ & generically $=X$\\
  \hline
  $R_{\lambda, \nu}^Y =$ & $\tmop{Op} \left( \frac{| x_p |^{\lambda + \nu - n}
  | Q |^{- \nu} q_Y^X (\lambda, \nu)}{\Gamma \left( \frac{\lambda - \nu}{2}
  \right) \Gamma \left( \frac{\lambda + \nu - n + 1}{2} \right) \Gamma \left(
  \frac{1 - \nu}{2} \right)} \right)$ & $(\lambda, \nu) \in
  \backslash\backslash$ & generically $=Y$ and never $=\emptyset$\\
  \hline
  $R_{\lambda, \nu}^C =$ & $\tmop{Op} \left( \frac{| x_p |^{\lambda + \nu - n}
  | Q |^{- \nu} q_C^X (\lambda, \nu)}{\Gamma \left( \frac{\lambda - \nu}{2}
  \right) \Gamma \left( \frac{\lambda + \nu - n + 1}{2} \right) \Gamma \left(
  \frac{1 - \nu}{2} \right)} \right)$ & $(\lambda, \nu) \in \mid \mid$ &
  generically $=C$ and never $=\emptyset$\\
  \hline
  $R_{\lambda, \nu}^{\{ 0 \}} =$ & $\tmop{Op} \left( \tilde{C}_{\nu -
  \lambda}^{\lambda - \frac{n - 1}{2}} (\Delta_{\mathbbm{R}^{p - 1, q}}
  \delta_{\mathbbm{R}^{p + q - 1}}, \delta (x_p)) \right)$ & $(\lambda, \nu)
  \in / /$ & $\{ [0] \}$\\
  \hline
\end{tabular}

Here:
\begin{itemize}
	\item $\mid \mid \mid \assign \{ (\lambda, \nu) \in \mathbbm{C}^2 \mid \nu \in
	- 2\mathbbm{N} \cup (q + 1 + 2\mathbbm{Z}) \},\quad \backslash\backslash:=\mysetn{(\lambda,\nu)\in\C^2}{\lambda+\nu-n+1\in-2\N}$;
\item $/ / \assign
\{ (\lambda, \nu) \in \mathbbm{C}^2 \mid \lambda - \nu \in
-2\N \},\quad \mid\mid:=\mysetn{(\lambda,\nu)\in\C^2}{\nu\in1+2\N}$;
\item $\tilde{C}(s,t)$ is a two-variable inflation of renormalized Gegenbauer polynomial, defined as in \cite{kobayashi2015symmetry};
\item $Q:x\mapsto \sum_{i=1}^px_i^2-\sum_{i=p+1}^{p+q}x_i^2$
\end{itemize}
Moreover,\\
\[ q_C^X (\lambda, \nu) : = \left\{ \begin{array}{ll}
     1, & q \in 2\mathbbm{Z}+ 1, p > 1\\
     \Gamma^{} \left( \frac{\lambda + \nu - n + 1}{2} \right), & q \in
     2\mathbbm{Z}, p = 1 \comma \nu \leqslant q - \nu\\
     \Gamma^{} \left( \frac{\lambda - \nu}{2} \right), & q \in 2\mathbbm{Z}, p
     = 1 \comma \nu > q - \nu\\
     \Gamma \left( \frac{\lambda - \nu}{2} \right), & q \in 2\mathbbm{Z}, p >
     1\\
     \Gamma \left( \frac{\lambda + \nu - n + 1}{2} \right), & q \in
     2\mathbbm{Z}+ 1, p = 1
   \end{array} \right. \]
   \[ q_Y^X (\lambda, \nu) : = \dots\]
\end{theorem}
\begin{remark}
	Note that information in the rightmost column implies that for every $(\lambda,\nu)\in\C^2$ we have $R_{\lambda,\nu}^{ \left\{ 0 \right\}},R_{\lambda,\nu}^Y,R_{\lambda,\nu}^C\neq0$, while
	$R^X_{\lambda,\nu}=0$ iff $(\lambda,\nu)$ belongs to a discrete set
	\[\begin{cases}
			//\cap\mid\mid\mid,&p>1\\
			\mybra{//\cap\mid\mid\mid} \cup \mybra{\backslash\backslash\cap\mid\mid},&p=1,
		\end{cases}
	\]
	and $\tilde{R}_{\lambda,\nu}^X=0$ iff $p=1$ and $(\lambda,\nu)$ is in a discrete set $\backslash\backslash\cap \mid\mid$.
\end{remark}
\begin{theorem}[classification of SBOs]
		We can find basis for $\Hom_{G'}(I(\lambda),J(\nu))$ for every $(\lambda,\nu)\in \mathbb{C}^2$. More concretely,
		\ExecuteMetaData[.master_extract.tex]{classification}
\end{theorem}
\begin{remark}
	Note that the theorem implies that for arbitrary $(\lambda,\nu)\in\C^2$ we have $\Hom_{G'}(I(\lambda),J(\nu))\neq0$.
\end{remark}
\begin{theorem}[spherical ``multiple'']
	Let $1_\lambda\in I(\lambda)^K,1_\nu\in J(\nu)^{K'}$ be the spherical vectors. We then have
\[ \OpR^X_{\lambda, \nu} 1_{\lambda} = 2^{1 -
\lambda} \frac{\pi^{n / 2}}{\Gamma \left( \frac{\lambda}{2} \right)
\Gamma \left(  \frac{\lambda + 1-q}{2} \right) \Gamma \left(
\frac{q - \nu + 1}{2} \right)} 1_{\nu}. \]
\end{theorem}
\begin{theorem}[residue formula]
		The distribution
		\[K_{\lambda,\nu}^X:=\frac{\myabs{x_p}^{\lambda+\nu-n}}{\Gamma\left( \frac{\lambda+\nu-n+1}{2} \right)}\times
		\frac{\myabs{Q}^{-\nu}}{\Gamma\left( \frac{1-\nu}{2} \right)}\]
		has the pole at $(\lambda,\nu)\in//$ with the residue given by
		\[\Res_{(\lambda,\nu)\in//}K_{\lambda,\nu}^{\mathbb{R}^{p,q}}=\frac{K_{\lambda,\nu}^{\mathbb{R}^{p,q}}}{\Gamma\left( \frac{\lambda-\nu}{2} \right)}
			=\frac{ (- 1)^k k!\pi^{(n - 2) / 2} 
		}{2^{ \nu + 2 k-1}}\cdot  \frac{\sin\left( \frac{1+q-\nu}{2}\pi \right)}{\Gamma\left( \frac{\nu}{2} \right)}
	\tilde{C}_{\nu - \lambda}^{\lambda - \frac{n
  	- 1}{2}} ({\Delta}_{\mathbb{R}^{p-1,q}} {\delta}_{\mathbb{R}^{p+q-1}}, \delta (x_p))
		\]
		where $k:=\frac{\nu-\lambda}{2}$.
		Hence taking $\Op(\cdot)$ on both sides we get
  \[\OpR_{\lambda,\nu}^X  = \frac{ (- 1)^k k!\pi^{(n - 2) / 2} 
		}{2^{ \nu + 2 k-1}}\cdot  \frac{\sin\left( \frac{1+q-\nu}{2}\pi \right)}{\Gamma\left( \frac{\nu}{2} \right)}
     \OpR_{\lambda,\nu}^{ \left\{ 0 \right\} },\quad(\lambda,\nu)\in// . \]
	\end{theorem}
	\begin{definition}
		Similarly to the construction of Fact \ref{fact1}, for $G:=O(p+1,q+1)$ we have $\Hom_G(I(\lambda),I(\nu))\simeq\Sol_G(\R^{p,q};\lambda,\nu)$
		where the $\Sol_G(\R^{p,q};\lambda,\nu)\subset\mathcal{D}(\R^{p,q})$ is defined to be the space of generalized functions on $\R^{p,q}$ that satisfy
		the four items in definition \ref{def2}, except that in second item $O(p,q)_{e_p}$ is replaced by $O(p,q)$ and the fourth item is replaced by $N_+$-invariance
		on $\R^{p,q}$, which in turn is defined as in definition \ref{def1}, with the only difference that we do not assume $b_p=0$ anymore.

		Now, the generalized function defined as
		\begin{equation*}
			\myabs{Q}^{\lambda-n}\times\begin{cases}
				\Gamma^{-1}\left( \lambda-n/2 \right),&\min\left\{ p,q \right\}=0\\
				\Gamma^{-1}\left( \frac{\lambda-n+1}{2} \right)\Gamma^{-1}\left( \lambda-n/2 \right),&\min\left\{ p,q \right\}>0,n\in2\Z+1\\
  \Gamma^{-1} \left( \frac{\nu + 1}{2} \right) \Gamma ^{-1}\left( \frac{\nu + n / 2 +
  1}{2} \right), &\min\left\{ p,q \right\}>0, n / 2 + p \in 2\mathbbm{Z}+ 1\\
  \Gamma^{-1} \left( \frac{\nu + 1}{2} \right) \Gamma ^{-1}\left( \frac{\nu + n / 2}{2}
  \right), & \min\left\{ p,q \right\}>0,n / 2 + p \in 2\mathbbm{Z}
			\end{cases}
		\end{equation*}
		is the member of $\Sol_G(\R^{p,q},\lambda,n-\lambda)$ and we shall denote the corresponding member of $\Hom_{G}(I(\lambda),I(n-\lambda))$ by $\tilde{\mathbb{T}}_{\lambda}$
		and call it \textit{Knapp-Stein operator}.
		The result of this construction repeted with $G'$ in placed of $G$ will be denoted by $\tilde{\mathbb{T}}_\nu$ and called Knapp-Stein operator as well.
	\end{definition}
	\begin{theorem}[functional identities]
		Let $n':=n-1$. We then have:
\ExecuteMetaData[.master_extract.tex]{functional}
	\end{theorem}
\begin{theorem}[images of SBOs]
\end{theorem}
We can compute the image of every SBO defined in theorem \ref{thm:construction} for every $(\lambda,\nu)\in\C^2$ (note that for some values $\nu\in\Z$ $J(\nu)$ does become reducible).
\begin{remark}
	Note that the proof of this theorem was done \textit{without} using the information of \cite{howe1993homogeneous}.
\end{remark}
Now, we recall the defined in \cite{KO2} irreducible representations $\pi_{\pm,\lambda}^{p,q}$ of $O(p,q)$. Comparing the definitions, one sees that as 
representations of $O(p+1,q+1)$ we have for $p>0$:
\begin{equation*}
	\lambda+q-1\in2\Z\implies\pi_{+,\lambda+\frac{n}{2}}^{p+1,q+1}\simeq A^1\left(\lambda  \right),\quad\lambda-p+1\in2\Z\implies\pi_{-,\lambda+\frac{n}{2} }^{p+1,q+1}\simeq A^2\left(\lambda\right)
\end{equation*}
and for $q>0$ (which is our constant assumption in this work) we have $T(\nu-q)=\pi_{-,\nu-q/2}^{1,q+1}$.

\begin{theorem}[$G'$-invariant maps between Zuckerman modules $\pi_{\pm,\lambda}^{p,q}$]
	The dimensions of $\Hom_{G'}\left(\pi_{\pm,{n}/{2}-\lambda}^{p+1,q+1} ,\pi_{\pm,\nu-{n'}/{2}}^{p,q+1} \right)$
	are as follows:\newline
\ExecuteMetaData[.master_extract.tex]{Aq}
\end{theorem}

\newpage
\noindent\textbf{\large Auxiliary info}\\
We fix $p, q \in \mathbbm{N}_+$, $n \assign p + q$, $G \assign O (p +
1, q + 1)$ and $G' \assign \mysetn{g \in
G}{g \cdot e_{p + 1} = e_{p + 1}} \simeq O (p, q + 1)$. We shall be interested
in the following
\ExecuteMetaData[.master_extract.tex]{tagsetting}
Then $P:=MAN_{+}$ is a Langlands decomposition of a maximal parabolic subgroup of $G$
and $P':=P\cap G'$ is maximal parabolic subgroup of $G'$
\begin{definition}
	\label{def1}For $F \in \mathcal{D}' (\R^{p,q})$
  we say that $F$ is
  \tmtextbf{$N_+'$-invariant} if $\forall b \in \mathbbm{R}^{p, q}$
  with $b_p = 0$ and $x_0 \in \R^{p,q}$ such that $\frac{x_0 - Q (x_0) b}{1 - 2 Q
  (x_0, b) + Q (x_0) Q (b)} \in \R^{p,q}$ and the expression makes sense (i.e. the
  denominator is non-zero) we have
  \begin{equation*}
    \label{eq-Nequiv} | 1 - 2 Q (b, x) + Q (x) Q (b) |^{\lambda - n} F \left(
    \frac{x - Q (x) b}{1 - 2 Q (x, b) + Q (x) Q (b)} \right) = F (x)
  \end{equation*}
  equality holding for $x$ near $x_0$.
\end{definition}

\begin{definition}
	\label{def2}For $F \in \mathcal{D}' (\R^{p,q})$
	we say that $F \in \sol$ if the
  following holds:
  \begin{enumerate}
    \item if $x_0 \in \R^{p,q}$ and $- x_0 \in \R^{p,q}$, then $F (x) = F (- x)$ for $x$
    near $x_0$;
    
    \newcommand{\Stab}{O(p,q)_{e_p}}
    \item if $(m, x_0, m \cdot x_0) \in \Stab \times \R^{p,q} \times \R^{p,q}$, then $F (x)
    = F (m \cdot x)$ for $x$ near $x_0$, where $\Stab \assign \{g \in O (p, q)
    |g \cdot e_p = e_p \}$;
    
    \item if $(\alpha, x_0, \alpha x_0) \in \mathbbm{R}_{> 0} \times \R^{p,q} \times
    \R^{p,q}$, then $\alpha^{\lambda - \nu - n} F (x) = F (\alpha x)$ for $x$ near
    $x_0$;
    
    \item $F$ is $N_+'$-invariant on $\R^{p,q}$.{
    
    }
  \end{enumerate}
\end{definition}

\bibliography{todai_master}
\bibliographystyle{apalike}
\end{document}
%TODO: 
%explain G/P
%Knapp normalization
