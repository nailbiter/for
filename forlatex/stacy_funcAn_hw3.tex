\documentclass[10pt]{article}

\usepackage{mathtext}                 % підключення кирилиці у математичних формулах
                                          % (mathtext.sty входить в пакет t2).
\usepackage[T1,T2A]{fontenc}         % внутрішнє кодування шрифтів (може бути декілька);
                                          % вказане останнім діє по замовчуванню;
                                          % кириличне має співпадати з заданим в ukrhyph.tex.
\usepackage[utf8]{inputenc}       % кодування документа; замість cp866nav
                                          % може бути cp1251, koi8-u, macukr, iso88595, utf8.
\usepackage[english,russian,ukrainian]{babel} % національна локалізація; може бути декілька
                                          % мов; остання з переліку діє по замовчуванню. 
\usepackage{amsthm}
\usepackage{amsmath}
\usepackage{amsfonts}
\usepackage{graphicx}
\usepackage[pdftex]{hyperref}
\usepackage{caption}
\usepackage{subfig}
\usepackage{fancyhdr}
\usepackage{cancel}
\usepackage{ulem}

\newtheorem{prob}{Завдання}
\newcommand{\ds}{\;ds}
\newcommand{\dt}{\;dt}
\newcommand{\dx}{\;dx}
\newcommand{\dta}{\;d\tau}

\usepackage{mystyle}

\newtheorem{myulem}[mythm]{Лема}

\renewenvironment{myproof}[1][Доведення]{\begin{trivlist}
\item[\hskip \labelsep {\bfseries #1}]}{\myqed\end{trivlist}}
\title{Контрольна робота з функціонального аналізу (9 семестр)\\Вар. 2}
\author{Олексій Леонтьєв}
\begin{document}
\maketitle
\begin{prob}Знайти оператор, спряжений до $A:l_2\mapsto l_2$\[Ax=({2}x_2,x_3,x_4,\dots)\]\end{prob}
	Спряженим є оператор $A^*:l_2\mapsto l_2$ заданий як
	\[A^*y=(0,2y_1,y_2,\dots)\]
	адже $A^*$ є неперервним оператором на $l_2$ і
	\[\mysca{Ax}{y}=2x_2{y_1}+x_3{y_2}+\hdots=
	x_1\cdot 0+x_2\cdot{{2}y_1}+x_3{y_2}+\hdots=\]
	\[\mysca{(x_1,x_2,x_3,\dots)}{(0,{2}y_1,y_2,y_3,\dots)}=\mysca{x}{A^*y}\]
\begin{prob}Довести, що оператор $A:C[0,1]\mapsto C[0,1]$ є скінченновимірним, $(Ax)(t)=\int_0^{1}\cos(t+\tau)x(\tau)d\tau,\;t\in[0,1]$.
	Чи буде $A$ компактним оператором?\end{prob}
	Для довільного $x\in C[0,1]$, маємо
	\[(Ax)(t)=\int_0^{1}\cos(t+\tau)x(\tau)d\tau=\int_0^{1}\left(\cos t\cos\tau-\sin t\sin\tau\right)x(\tau)d\tau=\]
	\[\cos t\int_0^{1}\cos\tau x(\tau)d\tau-\sin t\int_0^{1}\sin\tau x(\tau)d\tau\]
	Таким чином, $Ax$ є лінійною комбінацією $\sin t$ і $\cos t$, а оскільки $x$ було довільним, вся множина значень $A$ лежить в просторі
	лінійних комбінацій цих двох функцій, а тому $A$ є скінченновимірним.\\
	Оскільки кожний скінченновимірний оператор є компактним, $A$ компактний.
	\begin{prob}Знайти спектр, власні числа, норму і спектральний радіус оператора $A:l_2\mapsto l_2$, $Ax=(0,x_2,0,x_4,0,\hdots,x_{2k},0,\hdots)
		$\end{prob}
	Візьмемо довільне $\lambda\notin\mycbra{0,1}$ і покажемо, що воно \uline{не є} власним числом, тобто що $\lambda I-A$ є неперервно
	оборотнім. Для $x=(x_1,x_2,x_3,\hdots)\in l_2$ маємо
	\[(\lambda I-A)(x)=(\lambda x_1,(\lambda-1)x_2,\lambda x_3,\hdots)\]
	і оскільки $\lambda\neq0,\;\lambda-1\neq0$,
	бачимо, що оператор \[B(y_1,y_2,y_3,\hdots)=(\frac{1}{\lambda}y_1,\frac{1}{\lambda-1}y_2,\hdots,\frac{1}{\lambda}y_{2k-1},\frac{1}{\lambda-1
	}y_{2k},\hdots)\]
	є оберненим до $\lambda I-A$ ($B$ є неперервним оператором, адже 
	$\mynorm{By}\leq\max\mycbra{1/\myabs{\lambda},1/\myabs{\lambda-1}}\mynorm{y}$).

	Далі, $\lambda=0$ є власним числом $A$ із відповідним власним вектором $(1,0,0,\hdots)$, а $\lambda=1$ -- власним числом із власним вектором
	$(0,1,0,0,\hdots)$, тому обидва числа належать спектру, і використовуючи результат попереднього параграфа бачимо, що спектром $A$ є
	множина $\mycbra{0,-1}$, і вона ж співпадає із множиною власних значень. Спектральний радіус, таким чином, рівний $\rho=1$.

	Залишилось довести, що $\mynorm{A}=1$. Дійсно, з одного боку, якщо $x\in l_2$ маємо
	\[\mynorm{Ax}^2=\myabs{x_2}^2+\myabs{x_4}^2+\hdots\leq1\cdot\mynorm{x}^2\]
	а з другого боку для $x_0=(1,0,0,\hdots)\in l_2$ маємо
	$\mynorm{Ax_0}=1=\mynorm{x_0}$
	що і завершує доведення бажаної рівності $\mynorm{A}=1$.
\begin{prob}Знайти спектр, власні числа і власні функції оператора $A\in L(H)$, $H=L_2[0,2\pi]$, $(Ax)(t)=\int_0^{2\pi}\cos^2(t+\tau)x(\tau)
	d\tau,\;t\in[0,2\pi]$
\end{prob}
\begin{prob}Чи можуть наступні множини бути спектром деякого компактного оператора в $l_2$?\end{prob}
\begin{enumerate}
	\renewcommand{\labelenumi}{\myralph{enumi})}
\item $[0;2]$ \uline{не може} бути спектром, адже ця множина незліченна.
\item $\mycbra{0;2}$ є спектром, наприклад, оператора $A(x_1,x_2,\hdots)=(2x_1,0,0,\hdots)$. Дійсно, i 2, і 0 є власними значеннями (із відповідними
	власними векторами $(1,0,0,\hdots)$ та $(0,1,0,0,\hdots)$ відповідно), тому належать спектру. З іншого боку, для $\lambda\notin\mycbra{0,2}$,
	маємо $(\lambda I-A)(x_1,x_2,\hdots)=((\lambda-2)x_1,\lambda x_2,\hdots)$, оберненим до якого є 
	$B(y_1,y_2,y_3,\hdots)=(1/(\lambda-2)\cdot y_1,1/\lambda\cdot y_2,1/\lambda\cdot y_3,\hdots)$.
\item $\mycbra{\frac{1}{n},\;n\geq1}$ \uline{не може} бути спектром, адже ця множина не містить 0.
\item $\mycbra{0}\cup\mycbra{\frac{1}{n},\;n\geq1}$ є спектром оператора $A(x_1,x_2,x_3,\hdots)=(x_1/1,x_2/2,x_3/3,\hdots)$. По-перше, це дійсно
	неперервний оператор $l_2\to l_2$, адже $\forall x\in l_2,\;\mynorm{Ax}\leq\mynorm{x}$. По-друге, цей оператор компактний, адже він
	є границею послідовності скінченновимірних (а отже, компактних) операторів $A_n(x_1,x_2,x_3,\hdots)=(x_1/1,x_2/2,x_3/3,\hdots,x_n/n,0,0,
	\hdots)$. Дійсно, для довільного $x\in l_2$, маємо $\mynorm{(A-A_n)(x)}=\sqrt{\sum_{k=n+1}^{\infty}\myabs{x_k}^2/k
	^2}\leq\sqrt{\sum_{k=n+1}^{\infty}\myabs{x_k}^2/n^2}\leq\mynorm{x}/n\implies\mynorm{A-A_n}\leq 1/n$, тому $\mynorm{A-A_n}\to0$.

	Наостанок, кожне з чисел $1/n$ є власним числом оператора $A$ (із відповідним власним вектором, що має 1 на $n$-й позиції та 0 в інших
	позиціях), а отже кожне з них належить спектру. Також, нуль належить спектру, адже оператор $A$ будучи компактним, є необоротним. Залишається
	довести, таким чином, що довільне комплексне
	$\lambda\notin\mycbra{0}\cup\mycbra{\frac{1}{n},\;n\geq1}$ \uline{не} належить спектру $A$. Дійсно, для $x=(x_1,x_2,x_3,\hdots)\in l_2$
	маємо
	\[(\lambda I-A)(x)=((\lambda-1)x_1,(\lambda-1/2)x_2,(\lambda-1/3)x_3,\hdots)\]
	і ми стверджуємо, що оператор 
	\[B(x)=\mybra{{1}/\mybra{\lambda-1}x_1,1/(\lambda-1/2)x_2,{1}/(\lambda-1/3)x_3,\hdots}\]
	є оберненим до $\lambda I-A$. $B$ дійсно лінійний оператор, адже послідовність $\mycbra{\lambda-1/n}_{n\in\mathbb{N}}$, збіжна до $\lambda$,
	що за припущенням не рівне нулю, а отже послідовність комплексних чисел $\mycbra{1/(\lambda-1/n)}_{n\in\mathbb{N}}$ збіжна, а тому обмежена
	Таким чином, $\mynorm{B(x)}\leq\max\mycbra{1/(\lambda-1/n)}_{n\in\mathbb{N}}\mynorm{x}$, тому $B$ є неперервним оператором, і він є оберненим
	до $\lambda I-A$ за побудовою.
\item $\mycbra{1-\frac{1}{n},\;n\geq1}$ \uline{не може} бути спектром компактного оператора, адже вона має граничну точку $1\neq0$.
\end{enumerate}
\begin{prob}За допомогою повторних ядер побудувати резольвенту інтегрального рівняння $x(t)=\lambda\int_{0}^{1}\frac{1+t^2}{1+s^2}x(s)\ds+
	y(t),\;t\in[0;1]$, і знайти його розв’язок при $\lambda=1,\;y(t)=1+t^2$.\end{prob}
	\newcommand{\dm}{\;dm}
	Позначимо $K(t,s)=(1+t^2)/(1+s^2)\in C([0;1]\times[0;1])$. Ми позначимо $K^{(1)}(t,
	s)=K(t,s)$ і далі \[K^{(n+1)}(t,s)=\int_{0}^1 K(t,m)K^{(n)}(m,s)\;dm\]
	За ММІ, $K^{(n)}(t,s)=K(t,s)$. Рівність виконується для
	$n=1$, а далі
	\[K^{(n+1)}(t, s)=\int
	_{0}^1 K(t,m)K^{(n)}(m,s)\dm=\int_{0}^1\frac{1+t^2}{1+m^2}\frac{1+m^2}{1+s^2}\dm=\int_0^1K(t,s)\dm=K(t,s)\]
	Резольвента, відповідно, являється інтегральним оператором із ядром
	\[\mathcal{R}(t, s;\lambda)=\sum_{n=1}^\infty \lambda^{n-1}K^{(n)}(t, s)=\frac{1+t^2}{(1+s^2)(1-\lambda)}\]

	Таким чином, для даних $\lambda$ і $y(t)$ розв’язок записується як
	\[x(t)=y(t)+\lambda\int_{0}^1\mathcal{R}(t, s;\lambda)y( s)\;d s=1+t^2+\int_{0}^1\frac{1+t^2}{(1+s^2)}\frac{1}{1-1}\ds\]
	Отриманий нуль в знаменнику показує, що розв’язок, принаймні в такій формі, записати неможливо. Дійсно, можна показати, що для даних
	$\lambda$ та $y(t)$, розв’язку рівняння не має. Припустимо, що $x(t)$ задовольняє
	\[x(t)=\int_0^1\frac{1+t^2}{1+s^2}x(s)\ds+1+t^2\]
	Поділивши обидві частини на $1+t^2$ маємо
	\[\frac{x(t)}{1+t^2}=\int_0^1\frac{x(s)}{1+s^2}\ds+1\]
	це і дає протиріччя, адже права частина незалежна від $t$, тому $y(t)\equiv\alpha\in\mathbb{C}$, а рівність вище переписується як
	$\alpha=\alpha+1$.
\begin{prob}Звести до системи алгебраїчних рівнянь і розв’язати інтегральне рівняння
	\[x(t)=\frac{3}{8}\int_{-1}^1(1+t^2+s^3)x(s)\ds+5t-7t^3,\;t\in[-1;1]\].
\end{prob}
Введемо $y(t)=x(t)-(5t-7t^3)$. Тоді рівняння переписується як
	\[y(t)=\frac{3}{8}\int_{-1}^1(1+t^2+s^3)(x(s)+5s-7s^3)\ds\]
	\[y(t)=\frac{3}{8}\int_{-1}^1(x(s)+5s-7s^3)\ds\cdot(1+t^2)+\frac{3}{8}\int_{-1}^1s^3(x(s)+5s-7s^3)\ds\cdot1\]
	Таким чином, $y(t)=a(1+t^2)+b$ для певних $a,\;b\in\mathbb{C}$. Підставляючи, маємо
	\[a(1+t^2)+b=\frac{3}{8}\int_{-1}^1(a(1+s^2)+b+5s-7s^3)\ds\cdot(1+t^2)+\frac{3}{8}\int_{-1}^1s^3(a(1+s^2)+b+5s-7s^3)\ds\]
	\[a(1+t^2)+b=\frac{3}{8}(1+t^2)(2a+2b+\frac{2}{3}a)+\frac{3}{8}(2-2)\]
	\[a(1+t^2)+b=\frac{3}{8}(1+t^2)(2a+2b+\frac{2}{3}a)\]
	\[a(1+t^2)+b=\frac{3}{8}(1+t^2)(2b)+a(1+t^2)\]
	Оскільки $1+t^2$ та $1$ незалежні в $C[-1,1]$, маємо $b=0,\;a\in\mathbb{C}$ може бути довільним. Таким чином, остаточна відповідь
	\[x(t)=5t-7t^3+a(1+t^2),\;\forall a\in\mathbb{C}\]
\begin{prob}За допомогою альтернативи Фредгольма знайти всі $\lambda\in\mathbb{C}$, при яких наступне інтегральне рівняння має єдиний
	розв’язок при всіх $y\in C[0,2\pi]$:
	\[x(t)=\lambda\int_0^{2\pi}\cos(t+s)x(s)\ds+y(t),\;t\in[0,2\pi]\]
\end{prob}
Альтернатива Фредгольма (для інтегральних рівнянь) стверджує, що рівняння 
	\[x(t)=\lambda\int_0^{2\pi}\cos(t+s)x(s)\ds+y(t),\;t\in[0,2\pi]\]
	має єдиний розв’язок для кожного $y\in C[0,2\pi]$ тоді і лише тоді, коли
	\[x(t)=\lambda\int_0^{2\pi}\cos(t+s)x(s)\ds\]
має лише тривіальний розв’язок. Друге рівняння, будучи рівнянням з виродженим ядром,
має нетривіальні розв’язки тоді і лише тоді, коли їх має система
\[\begin{bmatrix}\lambda\int_0^{2\pi}\cos(t)\cos(t)\;dt-1&0\\0&-\lambda\int_0^{2\pi}\sin(t)\sin(t)\;dt-1\end{bmatrix}
	\begin{bmatrix}x_1\\x_2\end{bmatrix}=\begin{bmatrix}0\\0\end{bmatrix}\]
		що в свою чергу відбувається коли і тільки коли $1/\lambda\in\mycbra{
		\int_0^{2\pi}\cos^2t\dt,-\int_0^{2\pi}\sin^2t\dt}=\mycbra{\pm\pi}$, а отже
		\[x(t)=\lambda\int_0^{2\pi}\cos(t+s)x(s)\ds+y(t)\]
		має
		єдиний розв’язок для довільного $y\in C[0,2\pi]$ тоді і лише тоді, коли $\lambda\notin\mycbra{\pm\myfrac{1}{\pi}}$.
\begin{prob}
	Знайти характеристичні числа, відповідні власні функції та розв’язки інтегрального рівняння
	\[x(t)=\lambda\int_{0}^{\pi}\sin(t+s) x(s)\ds+\sin t,\;t\in[0;\pi]\]
\end{prob}
	Почнемо з того, що знайдемо розв’язки інтегрального рівняння. Це рівняння з виродженим ядром, адже $\sin(t+s)=\sin t\cos s+
	\sin s\cos t$, а отже розв’язки вичерпуються функціями вигляду
	$x(t)=\lambda x_1\sin t+\lambda x_2\cos t+\sin t$, де $x_1$ та $x_2$ задовольняють
	\[\begin{bmatrix}
		\lambda\int_0^\pi \sin t\cos t\dt-1 & \lambda\int_0^\pi\cos t\cos t\dt\\
		\lambda\int_0^\pi \sin t\sin t\dt & \lambda\int_0^\pi\cos t\sin t\dt-1
	\end{bmatrix}
	\begin{bmatrix}x_1\\x_2\end{bmatrix}=\begin{bmatrix}-\int_0^\pi\sin t\cos t\dt\\
		-\int_0^\pi\sin t\sin t\dt\end{bmatrix}\]
	\[\begin{bmatrix}
		-1&\lambda\pi/2\\
		\lambda\pi/2&-1
	\end{bmatrix}
	\begin{bmatrix}x_1\\x_2\end{bmatrix}=\begin{bmatrix}0\\-\pi/2\end{bmatrix}\]
	Ця система не має розв’язків при $\lambda=\myfrac{2}{\pi}$ і єдиний розв’язок в інших випадках, що записується як
	\[x_1=\frac{\lambda\pi^2/4}{1-\lambda^2\pi^2/4},\;x_2=\frac{\pi/2}{1-\lambda^2\pi^2/4}\]
	Відповідно, інтегральне рівняння не має розв’язок при $\lambda=\myfrac{2}{\pi}$ і єдиний розв’язок в іншому випадку, що
	записується як
	\[x(t)=\frac{\lambda^2\pi^2/4}{1-\lambda^2\pi^2/4}\sin t+\frac{\lambda\pi/2}{1-\lambda^2\pi^2/4}\cos t+\sin t\]

	Відповідно розв’язки однорідного рівняння вичерпуються функціями вигляду
	$x(t)=\lambda x_1\sin t+\lambda x_2\cos t$, де $x_1$ та $x_2$ задовольняють
	\[\begin{bmatrix}
		-1&\lambda\pi/2\\
		\lambda\pi/2&-1
	\end{bmatrix}
	\begin{bmatrix}x_1\\x_2\end{bmatrix}=\begin{bmatrix}0\\0\end{bmatrix}\]
	Ця система (а отже і відповідне однорідне інтегральне рівняння) не має нетривіальних розв’язків при 
	$\lambda\neq\myfrac{2}{\pi}$ і розв’язки $a\begin{bmatrix}1&1\end{bmatrix}^T$ в іншому випадку.

	Таким чином, для інтегрального рівняння єдиним власним значенням є $\lambda=\myfrac{2}{\pi}$ і єдиною відповідною нормованою
	власною функцією є $x_0(t)=(\sin t+\cos t)/\sqrt{\pi}$.
\begin{prob}
	Довести, що функціонал є узагальненою функцією	\[f(\phi)=\int_{\mathbb{R}}e^{-x}\phi'(x)\dx,\;\phi\in\mathcal{D}(\mathbb{R})\]
	Чи буде вона регулярною?
\end{prob}
\newcommand{\supp}{\mbox{supp }}
За означенням, нам треба показати, що $\phi\mapsto f(\phi)$ є лінійним неперервним функціоналом на $\mathcal{D}$. Лінійність
випливає з лінійності інтеграла (який завжди збіжний, адже $\supp \phi$ обмежена за означенням). Покажемо неперервність.
Нехай $\mathcal{D}(\mathbb{R})\ni\phi_n\to\phi$. За означенням
збіжності в $\mathcal{D}$, існує $r>0$, таке що для довільного
$n$ маємо $\supp\phi_n\subset\widetilde{B_r}(0)$ і таким чином, оскільки $\supp\phi'_n\subset\supp\phi_n$, маємо
\[f(\phi_n)=\int_{B_r(0)}e^{-2x}\phi'_n(x)\dx\to\int_{B_r(0)}e^{-2x}\phi'(x)\dx=\int_{\mathbb{R}}e^{-2x}\phi'(x)\dx\]
адже збіжність $\phi_n\to\phi$ є рівномірною на $B_r(0)$ за припущенням.

Щодо регулярності, якщо $\phi\in\mathcal{D}(\mathbb{R})$ і $\supp\phi'\subset\supp\phi\subset[-A,A]$, причому $\phi(\pm A)=0$, використовуючи
інтегрування частинами, маємо
\[f(\phi)=\int_{-A}^Ae^{-2x}\phi'(x)\dx=e^{-2x}\phi(x)\bigg|_{-A}^A-\int_{-A}^A(-2e^{-2x})\phi(x)\dx=2\int_{-A}^Ae^{-2x}\phi(x)\dx=
\int_\mathbb{R}2e^{-2x}\phi(x)\dx\]
і $f$ є регулярною загальною функцією.
\begin{prob}
	\[f(x)=\left\{\begin{array}{ll}\cos x,\;x\leq0\\1,\;x>0\end{array}\right.,\;\mbox{$f',\;f''$--?{ в }$\mathcal{D}'(\mathbb{R})$}\]
\end{prob}
\begin{prob}
	Довести
	\[F[\frac{1}{2}(\delta_h+\delta_{-h})]=\frac{1}{\sqrt{2\pi}}\cos(hy),\;h\in\mathbb{R},\;\mbox{де}\]
	$F$ -- перетворення Фур’є
\end{prob}
Нехай $\psi\in\mathcal{S}(\mathbb{R})$. Оскільки, $\myabra{F[\frac{1}{2}(\delta_h+\delta_{-h})],\psi}$ визначено як
\[\myabra{F[\frac{1}{2}(\delta_h+\delta_{-h})],\psi}=\myabra{\frac{1}{2}(\delta_h+\delta_{-h}),F(\psi)}\]
нам треба показати, що
\[\myabra{\frac{1}{2}(\delta_h+\delta_{-h}),F(\psi)}=\myabra{\frac{1}{\sqrt{2\pi}}\cos(hy),\psi}\]
Оскільки $e^{iht}+e^{-iht}=2\cos(ht)$, маємо
\[\frac{1}{2}\cdot\frac{1}{\sqrt{2\pi}}{\int_{\mathbb{R}}\psi(t)(e^{iht}+e^{-iht})\dt}=\frac{1}{\sqrt{2\pi}}\int_{\mathbb{R}}\psi(t)
\cos(ht)\dt\]
Згадаємо, що $F(\psi)$ визначено як
\[F(\psi)(s)=\frac{1}{\sqrt{2\pi}}\int_{-\infty}^\infty \psi(t)e^{its}\dt\]
а отже
\[\myabra{\delta_h,F(\psi)}=F(\psi)(h)=\frac{1}{\sqrt{2\pi}}\int_{-\infty}^\infty \psi(t)e^{ith}\dt\]
і таким чином
\[\myabra{\frac{1}{2}(\delta_h+\delta_{-h}),F(\psi)}=\frac{1}{2}\cdot\frac{1}{\sqrt{2\pi}}{\int_{\mathbb{R}}\psi(t)(e^{iht}+e^{-iht})\dt
}=
\myabra{\frac{1}{\sqrt{2\pi}}\cos(hy),\psi}\]
\begin{thebibliography}{9}
\bibitem{tb}
Березанський Ю. М., Ус Г. Ф., Шефтель З. Г.
Митропольський Ю. А., Самойленко А. М., Кулик В. Л.
\emph{Функціональний аналіз}.
Київ, "Вища школа"{}, 1990, російською мовою, 600 с.
\end{thebibliography}
\end{document}
