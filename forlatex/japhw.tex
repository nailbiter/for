\documentclass[10pt]{book}
\usepackage{fontspec}
\usepackage{array, xcolor, lipsum, bibentry}
\usepackage[margin=3cm]{geometry}
\usepackage{hyperref}
\usepackage{xeCJK}
\usepackage{graphicx}
\usepackage{caption}
\usepackage{subcaption}
%%\usepackage{float,subfig}
\setlength{\parskip}{10pt plus 1pt minus 1pt}

\setCJKmainfont{VL ゴシック}

\begin{document}
\begin{figure}
        \centering
        \begin{subfigure}[b]{0.3\textwidth}
                \includegraphics[width=\textwidth]{/home/u406/デスクトップ/1.jpg}
        \end{subfigure}%
        ~ %add desired spacing between images, e. g. ~, \quad, \qquad, \hfill etc.
          %(or a blank line to force the subfigure onto a new line)
        \begin{subfigure}[b]{0.075\textwidth}
                \includegraphics[width=\textwidth]{/home/u406/デスクトップ/2.jpg}
        \end{subfigure}
\end{figure}
%%\begin{figure}[H]
%%        \centering
%%        \begin{subfloat}{}
%%                {\includegraphics[width=0.3\textwidth]{/home/u406/デスクトップ/1.jpg}}
%%        \end{subfloat}~
%%        \begin{subfloat}{}
%%                {\includegraphics[width=0.08\textwidth]{/home/u406/デスクトップ/2.jpg}}
%%        \end{subfloat}
%%\end{figure}
これは【スマホ用自分撮りスティック】という製品である。細長いスティックの形をしていて、端に特別な取り付け部がついている:
スティックの長さを変えることもできる。\\

旅行の時にも、友達と遊ぶ時にもスマホで自分を撮影したいことがある。でも一般的にはパスポート写真みたいな写真ばかりにな
なってしまうことが多くて、おもしろくない。
しかし、自分撮りスティックにスマホを取り付ければ、好みのアングルで自分撮りができるようになっている。スティックの長さは撮りたい風景に合
わせて調節できるから、自分の撮影ばかりでなく、運動会など人垣越しのハイアングル撮影もできる!\\

だから、次回旅行時にこのスティックを是非使って見てよ!
\end{document}
