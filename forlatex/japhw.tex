\documentclass[10pt]{book}
\usepackage{fontspec}
\usepackage{array, xcolor, lipsum, bibentry}
\usepackage[margin=3cm]{geometry}
\usepackage{hyperref}
\usepackage{xeCJK}
\usepackage{float,subfig}
\setlength{\parskip}{10pt plus 1pt minus 1pt}

\usepackage{mystyle}
 
%font configuration
\setCJKmainfont{Ume P Mincho}

\begin{document}
\begin{figure}[H]
        \centering
        \begin{subfloat}{}
                {\includegraphics[width=0.3\textwidth]{~/public_html/1.jpg}}
        \end{subfloat}~
        \begin{subfloat}{}
                {\includegraphics[width=0.3\textwidth]{~/public_html/2.jpg}}
        \end{subfloat}
\end{figure}\vspace{1in}
これは【スマホ用自分撮りスティック】という製品です。\\\newline

細長いステーキの形をしていて、端に特別な取り付け部がついている。{\it あのステーキは伸縮もされられます。}\\\newline

旅行の時にも、友達と遊ぶの時にもスマホで自分を撮影したいことがある。でも一般的にはパスポート写真みたいな写真ばかりになっています。でも、自分撮りスティックにスマホを取り付ければ、好みのアングルで自分撮りが出来ます。
\end{document}
%intro
%description
%background and purpose
%http://www.thanko.jp/product/4602.html
