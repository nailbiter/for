\documentclass[8pt]{article} % use larger type; default would be 10pt

%\usepackage[utf8]{inputenc} % set input encoding (not needed with XeLaTeX)
%\usepackage{CJK}
\usepackage{graphicx}
\usepackage{float}
\usepackage{subfig}
\usepackage{amsmath}
\usepackage{amsfonts}
\usepackage{hyperref}
\usepackage{enumerate}
\usepackage{enumitem}

\usepackage{mystyle}

\title{ ENGG 5501: Foundations of Optimization\\Homework 1}
\author{Alex Leontiev, 1155040702, CUHK}
\begin{document}
\maketitle
\begin{enumerate}[label=\bfseries Problem \arabic*]
	\item{Following the article in Charnes-Cooper transformation, let us introduce 
		\[y=\frac{1}{f^Tx+g}x,\;t=\frac{1}{f^Tx+g}\]
		Then, the minimization of $\frac{c^Tx+d}{f^Tx+g}$ can be translated to minimization of $c^Ty+dt$ clearly. Constraint
		$f^Tx+g\geq 0$ can be replaced with equivalent $t\geq 0$ or $-t\leq 0$
		. $Ax\leq b$ can be translated to $Ay\leq bt$ by division of both sides
		on non-negative $t$, or to $Ay-bt\leq 0$. Finally, we need an additional constraint $f^Ty+gt=1$ to enforce the relation
		between $y$ and $t$. Indeed, together $f^Ty+gt=1$ and $y=\frac{1}{f^Tx+g}$ imply that $t=\frac{1}{f^Tx+g}$.

		Thus, we end up with linear problem
		\[\mbox{minimize }c^Ty+dt\]
		\[\mbox{subject to} Ay\leq bt\]
		\[f^Ty+gt\leq 1\]
		\[-f^Ty-gt\leq -1\]
		\[-t\leq 0\]
		}
	\item{\begin{enumerate}[label=(\alph*)]
			\item{To prove the desired statement, assume that it is not true, hence $\exists a_1\in \frac{2}{t+1}(\mathbb{R}^n\setminus
				(tA))+\frac{t-1}{t+1}A$ such that $a_1\notin\mathbb{R}^n\setminus A\implies a_1\in A$. Thus,
				$a_1=\frac{2}{t+1}b+\frac{t-1}{t+1}a$ for $a\in A$, $b\in\mathbb{R}^n\setminus A$. Multiplying both
				sides of equality for $a_1$ by $\frac{t+1}{2t}$ we get
				\[b+\frac{t-1}{2t}a=\frac{t+1}{2t}a_1\]
				thus
				\[b=\frac{t+1}{2t}a_1+\frac{t-1}{2t}(-a)\]
				Now as $A$ is symmetric, $-a\in A$ and thus $b\in A$ as convex combination of $-a$ and $a_1$, contradicting
				the assumption.
				}
			\item{}
			\item{Using the equality derived in (a) and definition of logconcave measures we get
				\[1-\theta=\mu(\mathbb{R}^n\setminus A)\geq \mu(\frac{2}{t+1}(\mathbb{R}^n\setminus
				(tA))+\frac{t-1}{t+1}A)\geq \mu(\mathbb{R}^n\setminus (tA))^{\frac{2}{t+1}}\mu(A)^{\frac{t-1}{t+1}}\]
				thus
				\[1-\theta>\mu(\mathbb{R}^n\setminus (tA))^{\frac{2}{t+1}}\theta^{\frac{t-1}{t+1}}\]
				hence
				\[\mu(\mathbb{R}^n\setminus (tA))<(1-\theta)^{\frac{t+1}{2}}\theta^{\frac{t-1}{2}}\]
				and to establish the desired bound $\theta\left(\frac{1-\theta}{\theta}\right)^{(t+1)/2}$ we just need to
				show that $\theta^{\frac{1-t}{2}}\geq \theta^{\frac{t-1}{2}}$, but this is equivalent to statement $\theta^t\leq\theta$
				, however the latter is obvious, as $\theta\leq 1$, $t> 1$.
				}
		\end{enumerate}}
	\item{First, let us show that $z^*=\Pi_K(x)$ implies that $z^*\in K$, $x-z^*\in K^\circ$, and $(x-z^*)^Tz^*=0$. Of these
		first follows directly from the definition of $\Pi_K(x)$ as $\Pi_K(x):=\arg\min_{z\in K}\mynorm{x-z}_2$. 
		
		Now, let us prove
		$(x-z^*)^Tz^*=0$. It is obvious if $x\in K$, for in this case $z^*=x$ and $(x-z^*)^Tz^*=0^Tz^*=0$, so we shall assume
		$x\notin K$. Then, in particular $x=\neq 0$, because $0\in K$ (as cone is nonempty by assumption, there is some $x_0\in K$ and
		as $K$ is cone $\forall n\in\mathbb{Z},\;n>0\implies \frac{1}{n}>0\implies\frac{1}{n}\cdot x_0\in K$, and as
		$\lim_{n\to\infty}\frac{1}{n}\cdot x_0=0$ and $K$ is closed by assumption, $0\in K$ -- this fact will be used in subsequent).
		Note, that the statement is obviously true if $z^*=0$, so we may freely assume it is not so.
		Consider the line $L:=\mysetn{\lambda z^*}{\lambda\in\mathbb{R}}$. As this is convex closed set, we have 
		$\lambda'z^*:=z':=\Pi_L(x)$. It is well known that $z'$ is characterized by $x-z'\perp z'$. We may also define ray
		$R:=\mysetn{\lambda z^*}{\lambda\geq 0}$. As this is also closed convex set, $\Pi_R(x)$ is well-defined and we should have
		$\Pi_R(x)=z^*$, as $R\subset K$, as $K$ is the closed cone.
		Now, there are two cases: $\lambda'<0$ or $\lambda'\geq 0$. In the first case, $\mynorm{\lambda z^*-x}=
		\mynorm{(\lambda-\lambda')z^*+\lambda'z^*-x}=\mynorm{(\lambda-\lambda')z^*}+\mynorm{\lambda'z^*-x}$ (as
		$(\lambda-\lambda')z^*\perp(\lambda'z^*-x)=(z'-x)$). Therefore, $\min_{\lambda\geq 0}\mynorm{\lambda z^*-x}=
		\mynorm{\lambda'z^*-x}+
		\min_{\lambda\geq 0}\myabs{\lambda-\lambda'}\mynorm{z^*}$ Now, as $\myabs{\lambda-(-\lambda')}$ is increasing for
		$\lambda>0$, because $\lambda'<0$, we have $\min_{\lambda\geq 0}\myabs{\lambda-\lambda'}=\myabs{0-\lambda'}$ and
		this implies that $\Pi_R(x)=0\cdot z^*=z^*\implies z^*=0$, but we have previously assumed it is not so. Then, if we assume
		$\lambda'\geq 0$, this means that $z'\in R$ and thus $z'=:\arg\min_{z\in L}\mynorm{z-x}=\arg\min_{z\in R}\mynorm{z-x}=z^*$
		and thus $x-z'\perp z'\implies x-z^*\perp z^*\implies (x-z^*)^Tz^*=0$.

		It remains to show $x-z^*\in K^\circ$. It means (according to the definition of $K^\circ$ that $\forall z\in K,\;-(x-z^*)^Tz\geq 0
		\iff (x-z^*)^Tz\leq 0$. Note, that this is trivial if $x\in K$, for $x\in K\implies x=z^*\implies \forall z\in K,\;
		(x-z^*)^Tz=0^Tz=0\leq 0$, thus we assume that $x\notin K\implies x-z^*\neq 0$ in subsequent.
		We shall prove this by contradiction and for this we assume that $z\in K$ and
		$(x-z^*)^Tz=>0$. 
		%As $x-z^*\neq 0$, $z$ can be uniquely decomposed as $z=\alpha(x-z^*)+h,\;(x-z^*)^Th=0(\iff
		%h\perp(x-z^*))$ and note also that $0<\epsilon=(x-z^*)^Tz=\alpha\mynorm{x-z^*}^2\implies \alpha>0$. As we have shown above,
		%$z^*\perp(x-z^*)$, $z-z^*=\alpha(x-z^*)+h-z^*=\alpha(x-z^*)+h',\;(x-z^*)\perp h'$. 
		Now, let us define $f:=\mysetn{t}{t\geq 0}\ni
		t\mapsto f(t):=\mynorm{z^*+t(z-z^*)-x}^2\in\mathbb{R}$. Note, that $f(0)=\mynorm{z^*-x}$, whereas
		$f'(0)=\frac{d}{dt}(z^*+t(z-z^*)-x)^T(z^*+t(z-z^*)-x)\mid_{t=0}=2(z-z^*)^T(z^*+t(z-z^*)-x)\mid_{t=0}=2(z-z^*)^T(z^*-x)=
		2(z-z^*+z^*)^T(z^*-x)$ (as $(z^*-x)^Tz^*=0$) and thus $f'(0)=2(z-z^*+z^*)^T(z^*-x)=z^T(z^*-x)<0$ Therefore, for some $\epsilon>0$
		small enough $z^*+\epsilon(z-z^*)\in K$ and $\mynorm{z^*+\epsilon(z-z^*)-x}=f(\epsilon)<f(0)=\mynorm{z^*-x}$, which contradicts
		the definition of $z^*$ as $z^*:=\Pi_K(x)$.

		Finally, we need to prove that $z^*\in K$, $x-z^*\in K^\circ$, and $(x-z^*)^Tz^*=0$ implies that $z^*=\Pi_K(x)$. Again, we may assume
		$x\notin K$, as case $x\in K$ is trivial. Indeed, $x\in K$ means that if some $z'\in K$ satisfies $(x-z')^Tz'=0$
		$x-z'\in K^\circ$, then
		$\forall z\in K,\;-(x-z')^Tz\geq 0\iff (x-z')^Tz\leq 0$ and thus in particular $(x-z')^Tx\leq 0$. Therefore,
		$0\geq (x-z')^Tx-(x-z')^Tz'=(x-z')^T(x-z')=\mynorm{x-z'}^2\implies x-z'=0\implies z'=x=\Pi_K(x)$. Thus, as was said,
		case $x\in K$ is trivial and in subsequent we shall assume it is not so.
		As was shown above,
		$z^*:=\Pi_K(x)$ satisfies all three conditions, so there is at least one point in $K$ that does. It remains to show that there is
		exactly one such point. Thus, assume that $z'\neq z^*$, $z'\in K$, $x-z'\in K^\circ$, and $(x-z')^Tz^*=0$, seeking contradiction.
		Note, that as $x-z^*\in K^\circ$, by definition of $K^\circ$ we have $\forall z\in K,\;-(x-z^*)^Tz\geq 0\iff (x-z^*)^Tz\leq 0$
		and thus in particular $(x-z^*)^Tz'\leq 0$. As $(x-z^*)^T z^*=0$, we have $0\geq(x-z^*)^Tz'=(x-z^*)^Tz'-(x-z^*)^Tz^*=(x-z^*)(z'-z^*)$
		As $x-z^*\neq 0$ (because $x\notin K$ by assumption) and $z'-z^*\neq 0$, we have that $\triangle xz^*z'$ has
		$\angle xz^*z'>\frac{\pi}{2}$. Repeating all of the above for $z'$ and $z^*$ exchanged (this can be done, as they have symmetric
		properties, by assumption) we have $\angle xz'z^*>\frac{\pi}{2}$ and thus $\triangle xz^*z'$ has two angles obtuse. Contradiction
		yielded ends the proof.
		}
	\item{\begin{enumerate}[label=(\alph*)]
			\renewcommand{\myset}{\mathcal{C}^n}
			\newcommand{\conv}{\mbox{conv}\left(\left\{vv^T\;:\;v\geq\mathbf{0}\right\}\right)}
			\newcommand{\mydset}{\left(\myset\right)^*}
		\item{$\mathcal{C}^n$ is cone, for if $X\in\myset$, $\lambda>0$, then $\lambda X\in\mathcal{S}^n$ and
			$\forall v\geq \mathbf{0}$ we have $v^T\lambda Xv=\lambda v^T Xv\geq 0$ and thus $\lambda X\in\myset$.
			$\myset$ is also convex, as if $X,\;Y\in\myset$, $\alpha\in[0,1]$ we have $\forall v\geq \mathbf{0}$
			$v^T(\alpha X+(1-\alpha)Y)v=\alpha v^TXv+(1-\alpha)v^TYv\geq 0$, as
			$v^TXv,\;v^TYv,\;alpha,\;1-\alpha\geq 0$. $\myset$ is also clearly nonempty, as $0\in\myset$ by direct check.
			Finally, $\myset$ is closed, for assume $\left\{X_n\right\}_{n=1}^{\infty},\;X_n\in\myset$ and
			$X_n\to X_0\in\mathcal{S}^n$. Let us take arbitrary $v\geq\mathbf{0}$. Then, by continuity of matrix
			product, $v^TX_nv\to v^TX_0 v$, but as $X_n\in\myset$ we have $v^TX_nv\geq 0$ and thus limit
			$v^TX_0v$ is also nonnegative. As $v$ was arbitrary, this shows that $X_0\in\myset$ and latter is thus closed.
				}
			\item{To begin with, let us show that $\conv$ is non-empty closed convex cone. It is nonempty, as
				$0^T0=0\in\conv$. It is cone, as if $\alpha>0$, $\sum\lambda_iv_iv_i^T\in\conv$, then
				$\alpha\sum\lambda_iv_iv_i^T=\sum\lambda_i(\sqrt{\alpha}v_i)(\sqrt{\alpha}v_i)^T\in\conv$. Finally,
				Let us show it is closed. First of all, note that if we will consider $n\times n$ matrix as a vector
				and apply to it standard 2-norm, we will have $\mynorm{vv^T}^2=\sum_{i,j=1}^n\myabs{v_iv_j}^2=
				\left(\sum_{i=1}^n\myabs{v_i}^2\right)^2=\mynorm{v}^4$. 
				Now, let us denote $N:=\left\{
				vv^T,\;v\geq 0,\;\mynorm{v}=1\right\}$. $N$ is definitely closed set, as if $v_nv_n^T\to V$, we have
				$\mynorm{v_n}=1$ and as set of vectors with norm $1$ is compact, by passing to subsequence,
				we may assume that $v_n\to v_0,\;\mynorm{v_0}=1\implies v_nv_n^T\to v_0v_0^T\in N$. As $N$ is clearly
				also bounded (all matrices in it have norm 1) it is compact. Now, let us fix some number $m$
				consider set $M$ consisting
				of all sums $\sum_{i=1}^m \alpha_i n_i$ where $\alpha_i\geq 0,\;n_i\in N$. This set is definitely
				closed. This is because for normalizing function $f(v)=v/\mynorm{v}$ that maps $\mathcal{S}^n$ to
				$\mysetn{X\in\mathcal{S}^n}{\mynorm{X}=1}$ we have that $M=f^{-1}(N)$ as $f$ is continuous
				and $N$ is closed in $\mysetn{X\in\mathcal{S}^n}{\mynorm{X}=1}$, $M$ is closed in $\mathcal{S}^n$.
				Now it just remains to show that $M=\conv$. Now, recall that $\conv$ can be seen as set of convex
				combinations of $m$ elements in $\left\{vv^T\;:\;v\geq\mathbf{0}\right\}$ for some $m$, depending on
				$n$. We will fix this $m$ as $m$ in definition of $M$. First, $M\subset\conv$, as if $
				M\ni v=\sum_{i=1}^m\alpha_iv_iv_i^T$ we can renormalize $v_i$ so that coefficient in front of
				$v_iv_i^T$ will become $1/n$ and thus $v\in\conv$. Conversely, elements of $\conv$ - convex
				combinations of elements in $\left\{vv^T\;:\;v\geq\mathbf{0}\right\}$ can have their elements
				renormalized to have norm of 1. Thus $\conv=M$ and is closed.

				Having $\conv$ and $\mydset$ both to be non-empty closed convex cones and knowing the fact
				that for non-empty closed cone $C^{**}=C$, it is just enough to show that
				\[\myset=\left(\conv\right)^*\]
				First, let us show that $\myset\subset\left(\conv\right)^*$. Let $X\in\myset$. Then for
				arbitrary $v\geq\mathbf{0}$ and for $Y:=vv^T$ we have $Y_{ij}=v_iv_j$ and thus $X\bullet Y=
				\sum_{i,\;j=1}^{\infty}X_{ij}Y_{ij}=\sum_{i,\;j=1}^{\infty}X_{ij}v_iv_j=v^TXv\geq 0$. Now,
				if $Z\in\conv$ is a convex combinations of $Y_i=v_iv_i^T,\;v_i\geq 0$, as $X\bullet Y_i\geq 0$,
				$Z\bullet Y_i$ is also nonnegative, thus $X\in\left(\conv\right)^*$.

				Conversely, let us demonstrate that $\myset\supset\left(\conv\right)^*$. If $X\in
				\left(\conv\right)^*$, then for any $v\geq\mathbf{0}$ $X\bullet vv^T\geq 0$, as $X\in
				\left(\conv\right)^*$ and thus $X\in\myset$ by definition of $\myset$.


				}
		\end{enumerate}}
\end{enumerate}
\end{document}
