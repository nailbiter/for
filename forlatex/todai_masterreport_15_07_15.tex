\documentclass[8pt,pdf,notes]{beamer}
\mode<presentation>{\usetheme[secheader]{Boadilla}}
\usepackage{mystyle}
\includecomment{versiona}

\newcommand{\red}[1]{{\color[rgb]{0.6,0,0}#1}}
\newcommand{\Sol}{\mbox{Sol}}
\newcommand{\D}{\mathcal{D}}
\newcommand{\A}{\mathcal{A}}

%%\makeatletter
%%\newenvironment<>{proofs}[1][\proofname]{\par\def\insertproofname{#1\@addpunct{.}}\usebeamertemplate{proof begin}#2}
%%{\usebeamertemplate{proof end}}
%%\makeatother
%%
%%\makeatletter
%%\def\th@mystyle{%
%%	\normalfont % body font
%%	\setbeamercolor{block title example}{bg=orange,fg=white}
%%	\setbeamercolor{block body example}{bg=orange!20,fg=black}
%%	\def\inserttheoremblockenv{exampleblock}
%%}
%%\makeatother
%%
%%\theoremstyle{mystyle}
%%\newtheorem{prop}{Proposition}
\newenvironment{prop}{\begin{exampleblock}{Proposition}\it}{\end{exampleblock}}
\makeatletter
\def\th@mystyle{%
    \normalfont % body font
    \setbeamercolor{block title example}{bg=orange,fg=white}
    \setbeamercolor{block body example}{bg=orange!20,fg=black}
    \def\inserttheoremblockenv{exampleblock}
  }
\makeatother
\theoremstyle{mystyle}
\newtheorem*{remark}{Remark}

\title{Presentation on Master Thesis preparation}
\author{Alex Leontiev}

\begin{document}
\begin{frame}\titlepage\end{frame}
\begin{frame}{Outline}
	\tableofcontents
\end{frame}

\section{$P'/G\backslash P$}
\begin{frame}<beamer:0>%this frame won't be seen
	\begin{theorem}[Kobayashi-Speh]\[G/P=P'[q_+]\sqcup P'[p_+]\sqcup P'[p_+]=
		(S^n-S^{n-1})\sqcup (S^{n-1}-{p_+})\sqcup p_+\]
		and these are pulled back to $\R^n-\R^{n-1}$, $\R^{n-1}-0$ and $0$ respectively via Bruhat parametrization and
		only the first one is open dense in $\R^n$.
\end{theorem}
\begin{theorem}[$O(p,q),\;n:=(p-1)+(q-1)$]
	\[G/P=P'[0_{p-1},1,0_{q-1},1]\sqcup P'[q+]\sqcup P'[0_{p-2},1,0,1,0_{q-1}]\sqcup P'[p_+]\sqcup P'[p_-]\]
	and these are pulled back to $\R^n-A-B,\;B-A,\;A\cap B-{0},\;{0}$ and $A-B$ respectively via Bruhat parametrization and
		only the first one is open dense in $\R^n$.
Here $A:=\mycbra{x_{p-1}=0}$ and $B:=\mysetn{(x,y)\in\R^{p-1,q-1}}{\myabs{x}_{p-1}=\myabs{y}_{q-1}}$ closed subsets of $\R^n$.
\end{theorem}
\end{frame}
\section{Equations for kernels}
\begin{frame}<beamer:0>
\begin{theorem}[Kobayashi-Speh]Kernel of symmetry breaker is distribution $F$ on $\R^n$ satisfies
	\[E-(\lambda-\nu-n))F=0\quad\mbox{($A'$-action)}\]
\[(2\nu x_j+(\myabs{x}^2+x_n^2)\frac{\partial}{\partial x_j})F=0,\;1\leq j\leq n-1\quad\mbox{($N_+'$-action)}\]
\[F(mx,x_n)=F(x,x_n),\;\forall m\in O(n-1)\quad\mbox{($M'$-action)}\]
\[F(-\tilde{x})=F(\tilde{x})\quad\mbox{($M'$-action)}\]
\end{theorem}
\begin{theorem}[O(p,q), $Q(\cdot)$ denotes indefinite $(p-1,q-1)$-norm on $\R^{ p,q }$]Kernel of symmetry breaker is distribution $F$ on $\R^{(p-1)+(q-1)}=:\R^n$ satisfies
\[E-(\lambda-\nu-n))F=0\quad\mbox{($A$-action)}\]
\[(2\epsilon_j\nu x_j+Q(x)\frac{\partial}{\partial x_j})F=0,\;1\leq j\leq n,\;j\neq p-1\quad\mbox{
($N'_+$-action)}\]
\[F(x,x_{p-1},y)=F(x',x_{p-1},y'),\;(x',y'):=m(x,y)\;\forall m\in O(p-2,q-1)\quad\mbox{($M'$-action)}\]
\[F(-\tilde{x})=F(\tilde{x})\quad\mbox{($M'$-action)}\]
\end{theorem}
\end{frame}
\section{Differential symmetry breaking operators}
\begin{frame}<beamer:0>
\begin{theorem}[Kobayashi-Speh]
	Differential symmetry breakers exist only when $\lambda-\nu=0,-2,-4,\hdots$ and in that case space of
	differential symmetry breaking operators is one-dimensional (given by Gegenbauer polynomial)
\end{theorem}
\begin{theorem}[$O(p,q),\;n:=(p-1)+(q-1)$]
	literally the same
\end{theorem}
\end{frame}
\section{Lemma 6.7}
\begin{frame}
	The lemma deals with the exact sequence
	\[0\rightarrow \Sol_{\mycbra{0}}(\R^n)\rightarrow \Sol(\R^n)\rightarrow\Sol(\R^n-\mycbra{0})\]
	more precisely, it explicitly constructs basis for the rightmost term.
In my opinion, it is important to the whole argument. It's purpose is twofold:
\begin{enumerate}
	\item it gives good upper estimates on dimension of symmetry breakers for all parameters (unlike, given by say [Sun-Zhu]),
		which can later be refined by showing that the rightmost morphism is zero map.
	\item it provides clue about how kernels of symmetry breaking operators should look like, by showing how solutions should
		look like when problem is restricted to open dense subset.
\end{enumerate}
\end{frame}
\begin{frame}
\begin{theorem}[Kobayashi-Speh]
	\[\Sol(\R^n-\mycbra{0})=\C g(x_n)(\myabs{x}^2+x_n^2)^{-\nu}\], where
	\[\mathcal{D}'(\R)\ni g(x):=\left\{\begin{array}{ll}
				x_+^{\lambda+\nu-n}+x_-^{\lambda+\nu-n}&\mbox{if }\lambda+\nu-n\neq-1,-3,-5,\hdots\\
				\delta^{-\lambda-\nu+n-1}(x)&\mbox{if }\lambda+\nu-n\in\mycbra{-1,-3,-5,\hdots}\\
		\end{array}\right.\]
\end{theorem}
Let $Q(\cdot)$ be indefinite $(p-1,q-1)$ norm on $\R^{p-1,q-1}$.  
Trying to mimic the logic of [Kobayashi-Speh] we are led to consider the exact sequence
\[0\rightarrow \Sol_{\mycbra{Q=0}}(\R^n)\rightarrow \Sol(\R^n)\rightarrow\Sol(
	\mycbra{Q\neq0})\]
\begin{theorem}[$O(p,q),\;n:=(p-1)+(q-1)$]
	\[\Sol(\mycbra{Q\neq0})=\Sol(\mycbra{Q<0})\oplus \Sol(\mycbra{Q>0})\]
	\[\Sol(\mycbra{Q>0})=\C g(x_p)\myabs{Q(x)}^{-\nu}\big|_{Q>0},\mbox{ if }p-1>1\]
	\begin{multline*}\Sol(\mycbra{Q>0})=\C \myabs{x_p}^{\lambda+\nu-n}\myabs{Q(x)}^{-\nu}\big|_{Q>0},\mbox{ if }p-1=1\\
	\mbox{(here $\myabs{x_p}^{\lambda+\nu-n}$ is the regular functional on $\R-\mycbra{0}$)}\end{multline*}
	\[\Sol(\mycbra{Q<0})=\C\myabs{Q(x)}^{-\nu}g(x_p)\big|_{Q<0}\]
\end{theorem}
\end{frame}
\begin{frame}
	Trying to understand $Sol_{\mycbra{Q=0}}(\R^n)$ we are led to another exact sequence
	\[0\rightarrow\Sol_{\mycbra{0}}(\R^n)\rightarrow\Sol_{\mycbra{Q=0}}(\R^n)\rightarrow
	\Sol_{\mycbra{Q=0}-\mycbra{0}}(\R^n-\mycbra{0})\]

\begin{theorem}
	If $\nu\notin\Z_{ >0 }$ we have $\Sol_{\mycbra{Q=0}-\mycbra{0}}(\R^n-\mycbra{0})=0$. Otherwise,
	in $(\mu,s,\omega_{ p-2 },\omega_{ q-2 })$ coordinates\footnotemark
	$\Sol_{\mycbra{Q=0}-\mycbra{0}}(\R^n-\mycbra{0})=\C\delta^{ (\nu-1) }(\mu-1)
	\otimes s^{-\nu }u$, where $u\in\mathcal{ D }'(\R_{ >0 }\times\mathbb{ S }^{ p-2 })$
	is a pullback of $g(x_{ p-1 })$ via the polar coordinate change\footnotemark
\end{theorem}
\begin{prop}
	For fixed $\nu\in\Z_{ >0 }$
	the non-trivial element of $\Sol_{\mycbra{Q=0}-\mycbra{0}}(\R^n-\mycbra{0})$ for $\lambda\in\C$ such that 
	$\Re(\lambda+\nu)>n-1$ and $\Re(\lambda-\nu)>0$ can be extended to the generalized function 
	$K_{\lambda,\nu}^C\in\D'(\R^{p,q})$ with support in $\left\{ Q=0 \right\}$ which also satisfies the equations.
\end{prop}
\begin{remark}
	Such kernel does not appear in case of [Kobayashi-Speh].
\end{remark}
\footnotetext{ $\R^{ p-1}\times\R^{ q-1 }\ni(x,y)=(\sqrt{ s }\omega_{ p-2 },\sqrt{ \mu s }\omega_{ q-1 })$}
\footnotetext{ $\R^{ p-1}\setminus\mycbra{ 0 }\ni x=\sqrt{ s }\omega_{ p-1 }$}
\end{frame}
\section{$K$-finite vectors in non-compact picture}
\begin{frame}
\begin{theorem}[Kobayashi-Speh; Proposition 4.3]
	The $K$-finite vectors in non-compact picture are equal to
	\[\C\myabra{ F_\lambda[\psi,h]:N\in\Z_{ \geq0 },\;\psi\in\mathcal{ H }^N(\mathbb{ S }^{ n-1 }),\;h\in\C[s] }.\]
	where
	\[F_\lambda[\psi,h](r\omega):=(1+r^2)^{ -\lambda }\mybra{ \frac{ 2r }{ 1+r^2 } }^N\psi(\omega)h\mybra{\frac{ 1-r^2 }{1+r^2}}\]
\end{theorem}
\begin{theorem}[$O(p,q),\;n:=(p-1)+(q-1)$]
	The $K$-finite vectors in non-compact picture are contained in
	\[\C\myabra{ F_\lambda[\psi,\psi',N,M,n',m']:N,M,n',m'\in\Z_{ \geq0 },
	\;\psi\in\mathcal{ H }^N(\mathbb{ S }^{ p-2 }),\;\psi'\in\mathcal{ H }^M(\mathbb{ S }^{ q-2 }) }.\]
	where
	\[F_\lambda[\psi,\psi',N,M,n',m'](r\omega_{ p-1 },r\sqrt{ \mu }\omega_{ q-1 }):=\]
	\[:=R(r,\mu)^\Theta\psi(\omega_{ p-1 })\psi'(\omega_{ q-1 })(2r)^{ M+N }\mu^{ M/2 }(1-r^2(1-\mu))^{ n' }(1+r^2(1-\mu))^
	{m'}\]
	where
	\[\Theta:={ -\lambda/2-N/2-n'/2-M/2-m'/2 }\]
	\[R(r,\mu)=1+2(1+\mu)r^2+(\mu-1)^2r^4\]
\end{theorem}
\end{frame}
\section{Normalization of singular ${K}^C_{ \lambda,\nu }$ supported on $\left\{ Q=0 \right\}$}
\begin{frame}
\begin{prop}
	For fixed $\nu\in\Z_{ >0 }$
	the non-trivial element of $\Sol_{\mycbra{Q=0}-\mycbra{0}}(\R^n-\mycbra{0})$ for $\lambda\in\C$ such that 
	$\Re(\lambda+\nu)>n-1$ and $\Re(\lambda-\nu)>0$ can be extended to the generalized function 
	$K_{\lambda,\nu}^C\in\D'(\R^{p,q})$ with support in $\left\{ Q=0 \right\}$ which also satisfies the equations.
\end{prop}
%%\begin{theorem}[Kobayashi-Speh; Proposition 7.3]
%%	Set $\tilde{K}^{\mathbb{ A }}_{ \lambda,\nu }:={{K}^{\mathbb{ A }}_{ \lambda,\nu }}/\Gamma(\frac{ \lambda+\nu-n+1 }{2})/
%%	\Gamma(\frac{ \lambda-\nu }{ 2 })$. Then $\forall f\in C^{ \infty }(S^n)_K$ we have $
%%	\myabra{\tilde{K}^{\mathbb{ A }}_{ \lambda,\nu },f}$ being holo on $\mathbb{ C }^2$. Moreover, $\mysbra{\forall f\in C^{ \infty }(S^n)
%%	_K,\;\myabra{\tilde{K}^{\mathbb{ A }}_{ \lambda,\nu },f}=0}\iff (\lambda,\nu)\in L_{ \mbox{even} }$
%%\end{theorem}
\begin{theorem}[$O(p+1,q+1),\;n:=p+q$]
	Normalize $K^C_{ \lambda,\nu }$ as $\tilde{K}^C_{\lambda,\nu}:=K^C_{\lambda,\nu}/N$ with\[
		N:=\begin{cases}
		\Gamma\left( \frac{\lambda+\nu-n+1}{2} \right)\Gamma\left( \frac{\lambda-\nu}{2} \right),&q\in2\Z+1\\
		\Gamma\left( \frac{\lambda+\nu-n+1}{2} \right),&q\in2\Z,q-1\ge\nu\\
		%\Gamma\left( \frac{\lambda+\nu-n+1}{2} \right),&q\in2\Z,q-2\nu\ge0\\
		%\Gamma\left( \frac{\lambda+\nu-n+1}{2} \right),&q\in2\Z,q-2\nu<0,q-1\ge\nu\\
		\Gamma\left( \frac{\lambda+\nu-n+1}{2} \right)\Gamma\left( \frac{\lambda-\nu+q-2}{2} \right),&q\in2\Z,q-1<\nu\\
		\end{cases}
	\]
	Then this normalization is {\bf perfect on $\C^1$}
	, in sense that $\tilde{K}_{\lambda,\nu}^C$ is holomorphic in $\lambda\in\C$ and
	never vanishes.
\end{theorem}
\begin{remark}
	I do not know the support of residues of $K^C_{\lambda,\nu}$ at this point.
\end{remark}
\end{frame}
\section{Normalization of regular $K_{\lambda,\nu}^{\R^{n}}$}
\begin{frame}
	\begin{theorem}[from sec. 7.1 of \cite{kobayashi2015symmetry}]
		For $\Re(\lambda-\nu)>0$ and $\Re(\lambda+\nu)>n-1$, $K_{\lambda,\nu}^{\mathbb{A}}(x,x_n):=\myabs{x_n}^{\lambda+\nu-n}
		\left( \myabs{x}^2+x_n^2 \right)^{-\nu}\in L_{loc}^1(\R^n)$ defines kernel of regular SBO.
		\label{}
	\end{theorem}
	\begin{theorem}[$G:=O(p+1,q+1),\;n:=p+q$]
		For $\Re(\lambda+\nu-n)>0$ and $\Re(-\nu)>0$\footnote{perhaps, one can do better than that},
		$K^{\R^n}_{\lambda,\nu}(x):=\myabs{x_p}^{\lambda+\nu-n}\myabs{Q(x)}^{-\nu
		}\in C(\R^{p,q})$ defines kernel of regular SBO. 
		\label{}
	\end{theorem}
\end{frame}
\begin{frame}
	\begin{theorem}[prop. 7.3 of \cite{kobayashi2015symmetry}]
		If we let \[\tilde{K}_{\lambda,\nu}^{\A}:=\frac{
			K_{\lambda,\nu}^{\A}}{\Gamma\left( \frac{\lambda+\nu-n+1}{2} \right)\Gamma
		\left( \frac{\lambda-\nu}{2} \right)}\] then this normalization is {\bf perfect on $\C^2$} in the sense
		that $\tilde{K}_{\lambda,\nu}^{\A}$ is holomorphic in $(\lambda,\nu)\in\C^2$ and vanishes only on discrete
		set \[L_{even}:=\mysetn{(\lambda,\nu)\in\Z_{\le0}^2}{\lambda-\nu\in2\Z_{\le0}}\]
		\label{}
	\end{theorem}
	\begin{theorem}[$G:=O(p+1,q+1),\;n:=p+q$]
		Let \[\tilde{K}_{\lambda,\nu}^{\R^n}:=\frac{K_{\lambda,\nu}^{\R^n}}{\Gamma\left( \frac{\lambda-\nu-q+2}{2} \right)
		\Gamma\left( \frac{\lambda+\nu-n+1}{2} \right)\Gamma\left( \frac{-\nu+1}{2} \right)}\]. Then for $q=2$
		this is {\bf perfect on $\C^2$} normalization and it vanishes on \[L_{even}\sqcup\mysetn{(\lambda,\nu)\in\Z\times
		\left( 2\Z_{\ge0}+1 \right)}{\lambda-\nu\in2\Z_{\le0}}\].
		\label{}
	\end{theorem}
\end{frame}
\section{References}
\begin{frame}
	{\footnotesize\bibliographystyle{alpha}
\bibliography{todai_master}}
\end{frame}
%%\section{meromorphic extension of ${K}_{ \lambda,\nu }$}
%%\begin{frame}
%%\begin{theorem}[Kobayashi-Speh; Lemma 8.2]
%%	\[\frac{ \partial }{ \partial x_n }K^+_{ \lambda+1,\nu }=(\lambda+\nu-n+1)K^+_{ \lambda,\nu }-2\nu K^+_{ \lambda,\nu +1}\]
%%	\[\Delta_{ n-1 }K^+_{ \lambda+1,\nu-1 }=2(\nu-1)(2\nu-n+1)K^+_{ \lambda,\nu }-4(\nu-1)K^+_{\lambda+1,\nu+1  }\]
%%\end{theorem}
%%\begin{theorem}[$O(p+1,q+1),\;n:=(p)+(q)$]
%%	\[\frac{ \partial }{ \partial x_p }K^+_{ \lambda,\nu }=-2K_{ \lambda,\nu+1 }+(\lambda+\nu-n)K_{ \lambda-1,\nu }\]
%%	\[\Delta_{ p-1,q }K^+_{ \lambda,\nu }=-2(-3+p+q-2\nu) K^+_{ \lambda-1,\nu+1 }-4K^+_{ \lambda,\nu+2 }\]
%%\end{theorem}
%%\end{frame}
%%\section{Further work}
%%\begin{frame}
%%	Possible approaches:
%%	\begin{itemize}
%%		\item Continue to mimic the argument in [Kobayashi-Speh]
%%			\begin{enumerate}
%%				\item verify that the regularization of non-trivial element of $\Sol_S(\R^n\setminus\mycbra{ 0 })$
%%					is indeed a kernel of an intertwiner
%%				\item see how it acts on $K$-finite vectors and where the poles are
%%				\item Renormalize it and analytically continuate it to the kernel of 
%%					regular symmetry breaking operator with support in $S$
%%				\item use logic similar to Prop 11.12 and Prop 11.13 in [Kobayashi-Speh] to 
%%					the question of when the last morphism
%%	$0\rightarrow\Sol_{\mycbra{0}}(\R^n)\rightarrow\Sol_{\mycbra{Q=0}}(\R^n)\rightarrow
%%	\Sol_{\mycbra{Q=0}-\mycbra{0}}(\R^n-\mycbra{0})$
%%	is onto
%%				\item go on to consider 
%%				$0\rightarrow\Sol_{\mycbra{ Q=0 }}(\R^n)\rightarrow\Sol(\R^n)\rightarrow
%%				\Sol(\mycbra{Q\neq0})$
%%			\end{enumerate}
%%%%		\item F-method
%%%%			\begin{enumerate}
%%%%				\item {[done]} perform algebraic Fourier transform on the defining equations for kernel of symmetry
%%%%					breaker. The only equation which will change substantially is the $N_+'$-action. It will
%%%%					become second order PDE.
%%%%				\item Read books [Slavyanov] and [Smirnov] suggested by Prof. Pevzner to find (maybe) and solve
%%%%					equations for $N_+'$ action.
%%%%				\item filter out among solutions to $N_+'$ action equations those that also satisfy $M'$ and $A'$
%%%%					invariance.
%%%%				\item make inverse algebraic Fourier transform to get explicit form of kernels
%%%%			\end{enumerate}
%%%%		\item Apply [Sun-Zhu]
%%%%			\begin{enumerate}
%%%%				\item Use data in [Howe-Tan] to find for fixed $(p,q)$ for which values of $(\lambda,\nu)$ both
%%%%					$I(\lambda)$ is irreducible for $O(p,q)$ and $J(\nu)$ for $O(p-1,q)$.
%%%%				\item Use the result [Sun-Zhu] to conclude that for those cases dimension of symmetry breakers
%%%%					is no more than one.
%%%%				\item Improve the upper bound to precise and construct explicit symmetry breakers in those cases.
%%%%				\item see what we are left with
%%%%			\end{enumerate}
%%	\end{itemize}
%%\end{frame}
%%\begin{frame}{Questions (which I'll answer myself)}
%%	\begin{enumerate}
%%		\item Understand the proof of Prop 11.12 of [Kobayashi-Speh]
%%		\item Understand the proof of Prop 11.7 of [Kobayashi-Speh]
%%	\end{enumerate}
%%\end{frame}
%%\begin{frame}{References}
%%	\begin{itemize}
%%		\item {[Sun-Zhu]} Sun, Binyong, and Chen-Bo Zhu. "Multiplicity one theorems: the Archimedean case." 
%%			{\it arXiv preprint arXiv:0903.1413 (2009)}.
%%		\item {[Kobayashi-Speh]} Kobayashi, Toshiyuki, and Birgit Speh. "Symmetry breaking for representations of rank one orthogonal groups." (2015).
%%		\item {[Slavyanov]} Slavianov, S.I.U. \&
%%			Lay, W. (2000). {\it Special Functions: A Unified Theory Based on Singularities}. Oxford University Press
%%		\item {[Smirnov]} Smirnov, V.I. (1964). {\it A Course of Higher Mathematics}.
%%			Pergamon Press; [U.S.A. ed. distributed by Addison-Wesley Publishing Company, Reading, Mass.
%%			\item {[Howe-Tan]} Howe, Roger E., and Eng-Chye Tan. "Homogeneous functions on light cones: the infinitesimal structure of some degenerate principal series representations." Bulletin of the American Mathematical Society 28.1 (1993): 1-74.
%%		\item {[Gelfand-Shilov I]} Gelfand, I.M. \& Shilov, G.E. (1969).
%%				{\it Generalized Functions. Vol. 1: Properties and Operations}. New York
%%	\end{itemize}
%%\end{frame}

\end{document}
