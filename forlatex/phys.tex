\documentclass[12pt]{article} % use larger type; default would be 10pt

\usepackage{mathtext}                 % підключення кирилиці у математичних формулах
                                          % (mathtext.sty входить в пакет t2).
\usepackage[T1,T2A]{fontenc}         % внутрішнє кодування шрифтів (може бути декілька);
                                          % вказане останнім діє по замовчуванню;
                                          % кириличне має співпадати з заданим в ukrhyph.tex.
\usepackage[utf8]{inputenc}       % кодування документа; замість cp866nav
                                          % може бути cp1251, koi8-u, macukr, iso88595, utf8.
\usepackage[english,russian,ukrainian]{babel} % національна локалізація; може бути декілька
                                          % мов; остання з переліку діє по замовчуванню. 
\usepackage{amsthm}
\usepackage{amsmath}
\usepackage{amsfonts}
\usepackage{graphicx}
\usepackage[pdftex]{hyperref}
\usepackage{caption}
\usepackage{subfig}
\usepackage{fancyhdr}
\usepackage{cancel}
\usepackage{enumerate}

\newtheorem{prob}{Завдання}
\newcommand{\ds}{\;ds}
\newcommand{\dt}{\;dt}
\newcommand{\dx}{\;dx}
\newcommand{\dv}{\;dv}
\newcommand{\dta}{\;d\tau}
\let\oldint\int
\renewcommand{\int}{\oldint\limits}
\let\phi\varphi
\newcommand{\extr}{\mbox{\normalfont extr}}

\usepackage{mystyle}

\newtheorem{myulem}[mythm]{Лема}

\renewenvironment{myproof}[1][Доведення]{\begin{trivlist}
\item[\hskip \labelsep {\bfseries #1}]}{\myqed\end{trivlist}}

\title{Методи мат. фізики (9 семестр)}
\author{Олексій Леонтьєв}

\begin{document}
\maketitle
\begin{prob}\end{prob}
	По-перше, ми покажемо, що \[C_V=\frac{nR}{\gamma-1}\]
	де $n$ -- кількість газу в молях. Для цього, в свою чергу, достатньо показати, що $C_P-C_V=nR$ (і застосувати потім дане $C_P/C_V=\gamma$
	як друге рівняння). Дійсно, різниця між нагріванням із сталим об’ємом і сталим тиском в тому, що в останньому буде здійснюватися робота по
	підйому поршня, рівна $PdV=nRdT$. Таким чином, $C_PdT-C_VdT=A=PdV=nRdT\implies C_P-c_V=nR$.

	Далі, при зупинці кінетична енергія газу спадає на $\frac{nMv^2}{2}$, а оскільки система замкнена, внутрішня енергія має зрости на цю
	величину, таким чином
	\[dT=\frac{dQ}{C_V}=\frac{Mv^2(\gamma-1)}{2}\]
\begin{prob}\end{prob}
	Ми припустимо, що на початку система знаходилась в рівновазі, поршень не рухався, а отже тиск навколишнього середовища рівний початковому
	тиску газу $P_0$. В процесі зміни об’єму, тиск газу буде змінюватися як функція від поточного об’єму $P=P(V)$ і таким чином, ми
	будемо виконовути роботу проти тиску $P_0-P(V)$ 
	Оскільки  взаємозв’язок трьох параметрів $P$, $T$ та $V$ описується рівнянням
	\[PV=\nu RT=1\cdot RT\]
	робота, таким чином, рівна 
	\[A=\int_{V_0}^{nV_0}(P_0-P(V))dV=(n-1)P_0V_0-\int_{V_0}^{nV_0}\frac{RTdV}{V}=(n-1)RT_0-RT\ln n=RT_0(n-1-\ln n)\]
\begin{prob}\end{prob}
	Перший закон термодинаміки записується як
	\[dQ=dU+dA\]
	і проінтерпретувавши умову задачі як $dQ=-dU$ маємо
	\[C=\frac{dQ}{dT}=-\frac{dU}{dT}=-C_V=-\frac{\nu R}{\gamma-1}-\frac{1\cdot R}{\gamma-1}\]
	що дає відповідь на перше питання задачі, а підставляючи $dQ=-dA$ в перший закон маємо
	\[dA=-2dU\]
	\[PdV=-2C_VdT\]
	\[\frac{ \cancel{\nu}\cancel{ R}T}{V}dV=-\frac{2\cancel{\nu}\cancel{ R}}{\gamma-1}dT\]
	\[\frac{\gamma-1}{2}\frac{dV}{V}+\frac{dT}{T}=0\]
	інтегруючи, отримуємо
	\[\frac{\gamma-1}{2}\ln V+\ln T=const\]
	\[V^{\frac{\gamma-1}{2}}T=const\]
\begin{prob}\end{prob}
	Якщо $C=C_V+\alpha T$, записуючи перший закон термодинаміки маємо
	\[\nu (\cancel{C_V}+\alpha T)dT=PdV+\cancel{\nu C_VdT}\]
	\[\cancel{\nu}\alpha \cancel{T}dT=\frac{\cancel{\nu} R\cancel{T}dV}{V}\]
	Інтегруючи, маємо
	\[-\alpha T+R\ln V=const\]
	\[e^{-\alpha T}V^R=const\]
	\[e^{-\alpha T/R}V=const\]

	Якщо $C=C_V+\beta V$, записуючи перший закон термодинаміки маємо
	\[\nu (\cancel{C_V}+\beta V)dT=PdV+\cancel{\nu C_VdT}\]
	\[\cancel{\nu}\beta VdT=\frac{\cancel{\nu}RTdV}{V}\]
	Інтегруючи, маємо
	\[\beta\ln T+\frac{R}{V}=const\]
	\[Te^{R/\beta V}=const\]

	Якщо $C=C_V+\gamma P$, записуючи перший закон термодинаміки маємо
	\[\nu (\cancel{C_V}+\gamma P)dT=PdV+\cancel{\nu C_VdT}\]
	\[\nu\gamma \cancel{P}dT=\cancel{P}dV\]
	Інтегруючи, маємо
	\[V-\nu\gamma T=const\]
	\setcounter{prob}{0}
\begin{prob}\end{prob}%5
	\begin{enumerate}[a)]
		\item Записуючи перший закон термодинаміки і враховуючи, що $dV=0$, маємо
			\[TdS=dU+\cancel{PdV}\]
			\[dS=C_V\frac{dT}{T}=\frac{1\cdot R}{\gamma-1}\frac{dT}{T}\]
			\[\Delta S=\int_T^{nT}dS=\frac{R}{\gamma-1}\ln n\]
		\item Записуючи перший закон термодинаміки і враховуючи, що $P=const$, маємо
			\[TdS=dU+\cancel{PdV}=C_VdT+PdV=C_PV\]
			В першому завданні ми показали, як можна розрахувати $C_V$. Той же метод дозволяє порахувати $C_P$ і ми отримуємо
			\[dS=C_P\frac{dT}{T}=\frac{1\cdot\gamma R}{\gamma-1}\frac{dT}{T}\]
			\[\Delta S=\int_T^{nT}dS=\frac{\gamma R}{\gamma-1}\ln n\]
	\end{enumerate}
\begin{prob}\end{prob}%6
	\[CdT=TdS\implies C=T\mybra{\frac{dT}{dS}}^{-1}=\frac{aS^n}{anS^{n-1}}=S/n\]
	і відповідно $S<0\iff n<0$.
\begin{prob}\end{prob}%7
	Оскільки ентропія системи є сумою ентропій підсистем, те ж саме вірне і для змін ентропії. Таким чином, треба лише обчислити
	зміну ентропії міді і води окремо, а потім додати їх. За першим законом термодинаміки,
	\[TdS=CdT\implies \Delta S=C\int_{T_0}^{T_1}\frac{dT}{T}=C\ln\frac{T_1}{T_0}\]
	і таким чином, якщо ми позначимо температуру після вирівнювання за $T$ (її ми обчислимо нижче), матимемо
	\[\Delta S=m_1C_1\ln\frac{T}{T_1}+m_2C_2\ln\frac{T}{T_2}\]
	температура ж після вирівнювання розраховується за допомогою того факту, що загальна енергія системи не змінилася, а отже
	\[m_1C_1T+m_2C_2T=m_1C_1T_1+m_2C_2T_2\implies T=\frac{m_1C_1T_1+m_2C_2T_2}{m_1C_1+m_2C_2}\]
\begin{prob}\end{prob}%8
	Ми можемо вважати, що вирівнювання температур проходить ізохорично, наприклад, після відкриття крану, гази розділені тонкою теплопровідною
	плівкою. Таким чином, теплоємність незмінна і рівна $C_V$, а задача дуже схожа із попередньою. Фінальна температура після вирівнювання
	рівна півсуми $T_1$ і $T_2$, а зміна ентропії
	\[\Delta S=C_V\ln\frac{T}{T_1}+C_V\ln\frac{T}{T_2}=C_V\ln\frac{T^2}{T_1T_2}=C_V=\ln\frac{(T_1+T_2)^2}{4T_1T_2}\]
\setcounter{prob}{0}
\begin{prob}\end{prob}%9
	Розподіл Максвела для імпульсу записується як
	\[F(p)=\sqrt{\frac{2}{\pi}}\mybra{mkT}^{-\myfrac{3}{2}}p^2\exp\mybra{-\frac{p^2}{2mkT}}\]
	де $m$ позначає молярну масу. Для початку, знайдемо найбільш ймовірну швидкість $v_0$. Оскільки вона має максимізувати $F(p)$, за
	теоремою Ферма маємо (пам’ятаймо, що $p=mv$)
	\[F'(v_0)=0\iff \frac{d}{dv}\mybra{v^2\exp\mybra{-\frac{mv^2}{2kT}}}\bigg|_{v_0}=0\]
	\[2v\cancel{\exp(\hdots)}-v^2\frac{2mv}{2kT}\cancel{\exp(\hdots)}=0\]
	\[v=\sqrt{\frac{2kT}{m}}=\sqrt{\frac{2P_0V}{\nu m}}=\sqrt{\frac{2P_0}{\rho}}\]
	Середня швидкість розраховується як
	\[\myabra{v}=\int_0^{\infty}vF(mv)\dv=\sqrt{\frac{2}{\pi}}\mybra{mkT}^{-\myfrac{3}{2}}m^2\int_0^\infty v^3\exp\mybra{-\frac{mv^2}{2kT}}\dv=\]
	\[=\sqrt{\frac{2}{\pi}}\mybra{mkT}^{-\myfrac{3}{2}}{m^2}\mybra{\frac{kT}{{m}}}^2\int_0^\infty x^3\exp(-x^2/2)\dx=\]
	\[=\sqrt{\frac{2}{\pi}}\mybra{m^2\frac{P_0}{\rho}}^{-\frac{3}{2}}m^2\mybra{\frac{P_0}{\rho}}^2\int_0^\infty x^3\exp(-x^2/2)\dx\]
\begin{prob}\end{prob}%10
\end{document}
