\documentclass[12pt]{article} % use larger type; default would be 10pt
\usepackage{mystyle}

\newtheorem{prob}{Завдання}
\newcommand{\ds}{\;ds}
\newcommand{\dt}{\;dt}
\newcommand{\dx}{\;dx}
\newcommand{\dv}{\;dv}
\newcommand{\dpp}{\;dp}
\newcommand{\dta}{\;d\tau}

\newtheorem{myulem}[mythm]{Лема}

\renewenvironment{myproof}[1][Доведення]{\begin{trivlist}
\item[\hskip \labelsep {\bfseries #1}]}{\myqed\end{trivlist}}

\title{Методи мат. фізики (9 семестр)}
\author{Олексій Леонтьєв}

\begin{document}
\maketitle
\begin{prob}\end{prob}
	По-перше, ми покажемо, що \[C_V=\frac{nR}{\gamma-1}\]
	де $n$ -- кількість газу в молях. Для цього, в свою чергу, достатньо показати, що $C_P-C_V=nR$ (і застосувати потім дане $C_P/C_V=\gamma$
	як друге рівняння). Дійсно, різниця між нагріванням із сталим об’ємом і сталим тиском в тому, що в останньому буде здійснюватися робота по
	підйому поршня, рівна $PdV=nRdT$. Таким чином, $C_PdT-C_VdT=A=PdV=nRdT\implies C_P-c_V=nR$.

	Далі, при зупинці кінетична енергія газу спадає на $\frac{nMv^2}{2}$, а оскільки система замкнена, внутрішня енергія має зрости на цю
	величину, таким чином
	\[dT=\frac{dQ}{C_V}=\frac{Mv^2(\gamma-1)}{2}\]
\begin{prob}\end{prob}
	Ми припустимо, що на початку система знаходилась в рівновазі, поршень не рухався, а отже тиск навколишнього середовища рівний початковому
	тиску газу $P_0$. В процесі зміни об’єму, тиск газу буде змінюватися як функція від поточного об’єму $P=P(V)$ і таким чином, ми
	будемо виконовути роботу проти тиску $P_0-P(V)$ 
	Оскільки  взаємозв’язок трьох параметрів $P$, $T$ та $V$ описується рівнянням
	\[PV=\nu RT=1\cdot RT\]
	робота, таким чином, рівна 
	\[A=\int_{V_0}^{nV_0}(P_0-P(V))dV=(n-1)P_0V_0-\int_{V_0}^{nV_0}\frac{RTdV}{V}=(n-1)RT_0-RT\ln n=RT_0(n-1-\ln n)\]
\begin{prob}\end{prob}
	Перший закон термодинаміки записується як
	\[dQ=dU+dA\]
	і проінтерпретувавши умову задачі як $dQ=-dU$ маємо
	\[C=\frac{dQ}{dT}=-\frac{dU}{dT}=-C_V=-\frac{\nu R}{\gamma-1}-\frac{1\cdot R}{\gamma-1}\]
	що дає відповідь на перше питання задачі, а підставляючи $dQ=-dA$ в перший закон маємо
	\[dA=-2dU\]
	\[PdV=-2C_VdT\]
	\[\frac{ \cancel{\nu}\cancel{ R}T}{V}dV=-\frac{2\cancel{\nu}\cancel{ R}}{\gamma-1}dT\]
	\[\frac{\gamma-1}{2}\frac{dV}{V}+\frac{dT}{T}=0\]
	інтегруючи, отримуємо
	\[\frac{\gamma-1}{2}\ln V+\ln T=const\]
	\[V^{\frac{\gamma-1}{2}}T=const\]
\begin{prob}\end{prob}
	Якщо $C=C_V+\alpha T$, записуючи перший закон термодинаміки маємо
	\[\nu (\cancel{C_V}+\alpha T)dT=PdV+\cancel{\nu C_VdT}\]
	\[\cancel{\nu}\alpha \cancel{T}dT=\frac{\cancel{\nu} R\cancel{T}dV}{V}\]
	Інтегруючи, маємо
	\[-\alpha T+R\ln V=const\]
	\[e^{-\alpha T}V^R=const\]
	\[e^{-\alpha T/R}V=const\]

	Якщо $C=C_V+\beta V$, записуючи перший закон термодинаміки маємо
	\[\nu (\cancel{C_V}+\beta V)dT=PdV+\cancel{\nu C_VdT}\]
	\[\cancel{\nu}\beta VdT=\frac{\cancel{\nu}RTdV}{V}\]
	Інтегруючи, маємо
	\[\beta\ln T+\frac{R}{V}=const\]
	\[Te^{R/\beta V}=const\]

	Якщо $C=C_V+\gamma P$, записуючи перший закон термодинаміки маємо
	\[\nu (\cancel{C_V}+\gamma P)dT=PdV+\cancel{\nu C_VdT}\]
	\[\nu\gamma \cancel{P}dT=\cancel{P}dV\]
	Інтегруючи, маємо
	\[V-\nu\gamma T=const\]
	\setcounter{prob}{0}
\begin{prob}\end{prob}%5
	\begin{enumerate}[a)]
		\item Записуючи перший закон термодинаміки і враховуючи, що $dV=0$, маємо
			\[TdS=dU+\cancel{PdV}\]
			\[dS=C_V\frac{dT}{T}=\frac{1\cdot R}{\gamma-1}\frac{dT}{T}\]
			\[\Delta S=\int_T^{nT}dS=\frac{R}{\gamma-1}\ln n\]
		\item Записуючи перший закон термодинаміки і враховуючи, що $P=const$, маємо
			\[TdS=dU+\cancel{PdV}=C_VdT+PdV=C_PV\]
			В першому завданні ми показали, як можна розрахувати $C_V$. Той же метод дозволяє порахувати $C_P$ і ми отримуємо
			\[dS=C_P\frac{dT}{T}=\frac{1\cdot\gamma R}{\gamma-1}\frac{dT}{T}\]
			\[\Delta S=\int_T^{nT}dS=\frac{\gamma R}{\gamma-1}\ln n\]
	\end{enumerate}
\begin{prob}\end{prob}%6
	\[CdT=TdS\implies C=T\mybra{\frac{dT}{dS}}^{-1}=\frac{aS^n}{anS^{n-1}}=S/n\]
	і відповідно $S<0\iff n<0$.
\begin{prob}\end{prob}%7
	Оскільки ентропія системи є сумою ентропій підсистем, те ж саме вірне і для змін ентропії. Таким чином, треба лише обчислити
	зміну ентропії міді і води окремо, а потім додати їх. За першим законом термодинаміки,
	\[TdS=CdT\implies \Delta S=C\int_{T_0}^{T_1}\frac{dT}{T}=C\ln\frac{T_1}{T_0}\]
	і таким чином, якщо ми позначимо температуру після вирівнювання за $T$ (її ми обчислимо нижче), матимемо
	\[\Delta S=m_1C_1\ln\frac{T}{T_1}+m_2C_2\ln\frac{T}{T_2}\]
	температура ж після вирівнювання розраховується за допомогою того факту, що загальна енергія системи не змінилася, а отже
	\[m_1C_1T+m_2C_2T=m_1C_1T_1+m_2C_2T_2\implies T=\frac{m_1C_1T_1+m_2C_2T_2}{m_1C_1+m_2C_2}\]
\begin{prob}\end{prob}%8
	Ми можемо вважати, що вирівнювання температур проходить ізохорично, наприклад, після відкриття крану, гази розділені тонкою теплопровідною
	плівкою. Таким чином, теплоємність незмінна і рівна $C_V$, а задача дуже схожа із попередньою. Фінальна температура після вирівнювання
	рівна півсуми $T_1$ і $T_2$, а зміна ентропії
	\[\Delta S=C_V\ln\frac{T}{T_1}+C_V\ln\frac{T}{T_2}=C_V\ln\frac{T^2}{T_1T_2}=C_V=\ln\frac{(T_1+T_2)^2}{4T_1T_2}\]
\setcounter{prob}{0}
\begin{prob}\end{prob}%9
	Розподіл Максвела для імпульсу записується як
	\[F(p)=\sqrt{\frac{2}{\pi}}\mybra{mkT}^{-\myfrac{3}{2}}p^2\exp\mybra{-\frac{p^2}{2mkT}}\]
	де $m$ позначає молярну масу. Для початку, знайдемо найбільш ймовірну швидкість $v_0$. Оскільки вона має максимізувати $F(p)$, за
	теоремою Ферма маємо (пам’ятаймо, що $p=mv$)
	\[F'(v_0)=0\iff \frac{d}{dv}\mybra{v^2\exp\mybra{-\frac{mv^2}{2kT}}}\bigg|_{v_0}=0\]
	\[2v\cancel{\exp(\hdots)}-v^2\frac{2mv}{2kT}\cancel{\exp(\hdots)}=0\]
	\[v=\sqrt{\frac{2kT}{m}}=\sqrt{\frac{2P_0V}{\nu m}}=\sqrt{\frac{2P_0}{\rho}}\]
	Щоб знайти середню швидкість ми почнемо з того, що знайдемо середній імпульс
	\[\myabra{p}=\int_0^{\infty}pF(p)\dpp=\sqrt{\frac{2}{\pi}}\mybra{mkT}^{-\myfrac{3}{2}}
	\int_0^\infty p^3\exp\mybra{-\frac{p^2}{2mkT}}\dpp=\begin{vmatrix}x:=p/\sqrt{mkT}\\dx=dp/\sqrt{mkT}\end{vmatrix}=\]
	\[=\sqrt{\frac{2}{\pi}}(mkT)^{-\frac{3}{2}}(mkT)^2\int_0^\infty x^3e^{-x^2}\dx=
	\begin{vmatrix}t:=x^2/2\\dt=x\;dx\end{vmatrix}=\]
	\[=2\sqrt{\frac{2}{\pi}}\sqrt{mkT}\underbrace{\int_0^\infty te^{-t}\dt}_{=\Gamma(2)=1}=\]
	\[=2m\sqrt{\frac{2P_0}{\rho}}\]
	і таким чином, оскільки масу кожної частинки ми вважаємо однаковою
	\[\myabra{v}=\frac{\myabra{p}}{m}=2\sqrt{\frac{2P_0}{\rho}}\]
	І нарешті, визначимо середній квадратичний імпульс
	\[\sqrt{\myabra{p^2}}=\sqrt{\int_0^\infty p^2F(p)\dpp}=\sqrt{\sqrt{\frac{2}{\pi}}(mkT)^{-\frac{3}{2}}(mkT)^{\frac{5}{2}}
	\int_0^\infty x^4e^{-x^2/2}\dx}=\]
	\[=\sqrt{\sqrt{\frac{\cancel{2}}{\bcancel{\pi}}}mkT\cancel{2}\cancel{\sqrt{2}}\underbrace{\int_0^\infty t^{\frac{3}{2}}e^{-t}\dt}_{=\Gamma(5/2)=\frac{3}{\cancel{4}}
	\sqrt{\bcancel{\pi}}}}=\sqrt{3mkT}=m\sqrt{\frac{3P_0}{\rho}}\]
	і таким чином, оскільки масу кожної частинки вважаємо однаковою,
	\[\sqrt{\myabra{v^2}}=\frac{\sqrt{\myabra{p^2}}}{m}=\sqrt{\frac{3P_0}{\rho}}\]
\begin{prob}\end{prob}%10
	Це неважко -- ми позначимо за $m_1$ та $m_2$ масу частинки гелію та водню відповідно
	\[F^{(v)}_1(v)=F^{(v)}_2(m_2v)\]
	Оскільки Максвеллівська функція розподілу швидкостей $F^{(v)}(v)$ записується як
	\[F^{(v)}(v)=\sqrt{\mybra{\frac{m}{2\pi kT}}^3}4\pi v^2\exp\mybra{-\frac{mv^2}{2kT}}\]
	маємо
	\[\sqrt{\mybra{\frac{m_1}{\cancel{2\pi kT}}}^3}\bcancel{4\pi v^2}\exp\mybra{-\frac{m_1v^2}{2kT}}
	=\sqrt{\mybra{\frac{m_2}{\cancel{2\pi kT}}}^3}\bcancel{4\pi v^2}\exp\mybra{-\frac{m_2v^2}{2kT}}\]
	\[\exp\mybra{-\frac{(m_1-m_2)v^2}{2kT}}=\mybra{\frac{m_2}{m_1}}^{\frac{3}{2}}\]
	\[{-\frac{(m_1-m_2)v^2}{2kT}}=\frac{3}{2}\ln{\frac{m_2}{m_1}}\]
	\[v=\sqrt{\frac{3kT\ln(m_2/m_1)}{m_2-m_1}}\]
\begin{prob}\end{prob}%11
	Оскільки гравітаційне поле однорідне, то потенціальна енергія залежить лише від висоти $h$ і рівна $U(h)=gh$, де $g$ -- довільна
	константа. Таким чином, за розподілом Больцмана, густина ймовірності записується як
	\[\rho(\myvec{r})=C\exp(-\frac{gh}{kT})\]
	і оскільки геометрія циліндра не змінюється з висотою, ми можемо вважати (змінивши $C$) вираз вище відносною концентрацією по висоті,
	тобто $\rho(h)\;dh$ -- це процент частинок, що знаходяться на висоті $[h,h+dh]$, і $C$, таким чином, визначається з умови
	\[\int_0^\infty C\exp(-\frac{gh}{kT})\;dh=1\]
	Таким чином,
	\[\myabra{U}=\int_0^\infty U(h)\rho(h)\;dh=\int_0^\infty ghC\exp(-\frac{gh}{kT})\;dh=\underbrace{
	-ghC\frac{kT}{g}\exp(-\frac{gh}{kT})\bigg|_0^\infty}_{=0-0=0}
	+\int_0^\infty \cancel{g}C\frac{kT}{\cancel{g}}\exp(-\frac{gh}{kT})\;dh=\]
	\[=kT\int_0^\infty C\exp(-\frac{gh}{kT})\;dh=kT\]
\begin{prob}\end{prob}%12
	Помітимо, що концентрація, за законом Больцмана, записується як
	\[\rho(r)=C\exp(-\frac{\alpha r^2}{kT})\]
	Але кількість молекул, що знаходяться на відстані $[r,r+dr]$ записується трохи по-іншому -- нам треба густину помножити на 
	об’єм простору, що знаходиться на відстані $[r,r+dr]$, тобто різницю об’ємів шарів розмірів $r$ та $r+dr$, тобто $d(\frac{4}{3}\pi r^3)=
	4\pi r^2\;dr$, тобто число записується як (константа $C$ вже інша)
	\[\nu(r)=4C\pi r^2\exp(-\frac{\alpha r^2}{kT})\]
	Щоб знайти найбільше значення $\nu(r)$, нам треба прирівняти похідну до нуля (відкидаючи константи)
	\[\frac{d}{dr}\mybra{r^2\exp(-\frac{\alpha r^2}{kT})}=0\]
	\[2 r\exp(\hdots)-\frac{2 \alpha r^3}{kT}\exp(\hdots)=0\]
	\[r=\sqrt{\frac{kT}{\alpha}}\]
\section{Електродинаміка}
\subsection{Електростатичне поле в вакуумі}
\setcounter{prob}{0}
\begin{prob}\end{prob}%13
	За законом Кулона,
	\[\myvec{E}=\int_0^{2\pi}\underbrace{\frac{\lambda_0\cos\oldphi\myvec{(-R\cos\oldphi,-R\sin\oldphi)}}{4R^3\pi\epsilon_0}}
	_{=1\cdot q(\phi)\myvec{r}/4\pi r^3\epsilon_0}\;\underbrace{d(R\oldphi)}_{dl}=\]
	\[=-\frac{\lambda_0}{4\pi\epsilon_0R^2}\myvec{\mybra{\underbrace{\int_0^{2\pi}R\cos^2\oldphi\;d\oldphi}_{=\pi},
	\underbrace{\int_0^{2\pi}R\cos\oldphi\sin\oldphi\;d\oldphi}_{=0}}}=\]
	\[=\frac{\lambda_0}{4\pi\epsilon_0R^2}(R\pi,0)\implies E=\mynorm{\myvec{E}}=\frac{\lambda_0}{4R\epsilon_0}\]
\begin{prob}\end{prob}%14
	Ми будемо інтегрувати в сферичних координатах $(\phi,\theta),\;0\leq\phi<2\pi,\;0\leq\theta<\pi$, в цих координатах декартові записуються
	як $(x,y,z)=(r\sin\theta\cos\phi,r\sin\theta\sin\phi,r\cos\theta)$, а площа нескінченно малого шматочка сфери біля точки $(\phi,\theta)$
	як \[dS(\phi,\theta)=\myabs{(\frac{\partial}{\partial\phi}x,\frac{\partial}{\partial\phi}y,\frac{\partial}{\partial\phi}z)\times
	(\frac{\partial}{\partial\theta}x,\frac{\partial}{\partial\theta}y,\frac{\partial}{\partial\theta}z)}\;d\phi\;d\theta=r^2\sin\theta
	\;d\phi\;d\theta\]записуючи $\myvec{a}=:(a,b,c)$, маємо
	\[\myvec{E}=\int_0^{2\pi}\int_0^\pi\frac{(\myvec{a},\myvec{r}(\phi,\theta))\cdot(-\myvec{r}(\phi,\theta))}
	{4r^3\pi\epsilon_0}\;dS(\phi,\theta)=\]
	\[=-\frac{r}{4\pi\epsilon_0}\myvec{\mybra{
	a{\int_0^{2\pi}\int_0^\pi(aA+bB+cC)\sin\theta A \;d\theta\;d\phi},
	b{\int_0^{2\pi}(aA+bB+cC)\sin\theta B\;d\theta\;d\phi},
	c{\int_0^{2\pi}(aA+bB+cC)\sin\theta C \;d\theta\;d\phi}
	}}\]
	де
	\[A:=\sin\theta\cos\phi,\;B:=\sin\theta\sin\phi,\;C:=\cos\theta\]
	і оскільки
	\newcommand{\myint}[1]{\int_0^{2\pi}\int_0^\pi#1\;d\theta\;d\phi}
	\[\myint{A^2\sin\theta}=\myint{\sin^3\theta\cos^2\phi}=\frac{4}{3}\pi\]
	\[\myint{AB\sin\theta}=\myint{\sin^3\theta\cos\phi\sin\phi}=0\]
	\[\myint{AC\sin\theta}=\myint{\sin^2\theta\cos\theta\cos\phi}=0\]
	\[\myint{B^2\sin\theta}=\myint{\sin^3\theta\sin^2\phi}=\frac{4}{3}\pi\]
	\[\myint{BC\sin\theta}=\myint{\sin^2\theta\cos\theta\sin\phi}=0\]
	\[\myint{C^2\sin\theta}=\myint{\cos^2\theta\sin\theta}=\frac{4}{3}\pi\]
	маємо
	\[\myvec{E}=-\frac{r}{4\pi\epsilon_0}\frac{4\pi}{3}\myvec{\mybra{a,b,c}}=-\frac{r\cdot\myvec{a}}{3\epsilon_0}\]
\begin{prob}\end{prob}%15
	Ми знову будемо інтегрувати в сферичних координатах, як і в попередній задачі, додавши $0\leq r\leq R$, об’єм нескінченно малого 
	шматка кулі має вираз
	\[dV(\phi,\theta,r)=|\begin{vmatrix}
		\frac{\partial x}{\partial\theta}&\frac{\partial y}{\partial\theta}&\frac{\partial z}{\partial\theta}\\
		\frac{\partial x}{\partial\phi}&\frac{\partial y}{\partial\phi}&\frac{\partial z}{\partial\phi}\\
		\frac{\partial x}{\partial r}&\frac{\partial y}{\partial r}&\frac{\partial z}{\partial r}\\
	\end{vmatrix}|=r^2\sin\theta\;dr\;d\theta\;d\phi\]
	\renewcommand{\myint}[1]{\int_0^{2\pi}\int_0^\pi\int_0^R#1\;dr\;d\theta\;d\phi}
	і таким чином,
	\[\myvec{E}=\myint{\frac{(\myvec{a},\myvec{r}(\phi,\theta))\cdot(-\myvec{r}(\phi,\theta))}{4\pi\epsilon_0r^3}r^2\sin\theta}\]
	\renewcommand{\myint}[1]{\int_0^{2\pi}\int_0^\pi#1\;d\theta\;d\phi}
	\[=-\frac{\int_0^Rr\;dr}{4\pi\epsilon_0}\myvec{\mybra{
	a{\int_0^{2\pi}\int_0^\pi(aA+bB+cC)\sin\theta A \;d\theta\;d\phi},
	b{\int_0^{2\pi}(aA+bB+cC)\sin\theta B\;d\theta\;d\phi},
	c{\int_0^{2\pi}(aA+bB+cC)\sin\theta C \;d\theta\;d\phi}
	}}\]
	де
	\[A:=\sin\theta\cos\phi,\;B:=\sin\theta\sin\phi,\;C:=\cos\theta\]
	і, як в попередній задачі, отримуємо
	\[\myvec{E}=-\frac{R^2}{8\pi\epsilon_0}\frac{4\myvec{a}\pi}{3}=-\frac{\myvec{a}R^2}{6\epsilon_0}\]
\begin{prob}\end{prob}%16
	Перш за все, знайдемо вектор поля, що діє на частинку на відстані $r_0<R$ від центра сфери. За симетрією, вектор поля має мати лише
	радиальну складову, абсолютне значення якої, за теоремою Гауса (застосованої до сфери з тим же центром, але радіусу $r$)
	\[E=\frac{Q}{\epsilon_0 S}=\frac{4\pi r^3}{3\epsilon_0 4\pi r^2}=\frac{r\rho}{3\epsilon_0}\]
	і таким чином,
	\[\Phi=\iint_S\mybra{\myvec{E},\myvec{n}}\;dS=\int_0^{2\pi}\int_0^{\sqrt{R^2-r_0^2}}\frac{\rho\sqrt{r_0^2+r^2}}{3\epsilon_0}\cos\alpha
	r\;dr\;d\phi=2\pi\int_0^{\sqrt{R^2-r_0^2}}\frac{\rho\cancel{\sqrt{r_0^2+r^2}}}{3\epsilon_0}\frac{r_0}{\cancel{\sqrt{r_0^2+r^2}}}r\;dr=\]
	\[=\frac{2\pi\rho r_0}{3\epsilon_0}\frac{R^2-r_0^2}{2}=\frac{\pi\rho r_0(R^2-r_0^2)}{3\epsilon_0}=\]
\begin{prob}\end{prob}%17
	З міркувань симетрії, вектор поля має бути напрямлено в напрямі перпендикулярному до нитки, а отже якщо ми оточимо її циліндром
	з висотою 1, радіусом $r$ і віссю, співпадаючою з ниткою, потік через "низ" і "верх" циліндра буде рівним нулю, а в кожній точці "стінки"
	напрям поля буде співпадати з напрямом нормалі і ми матимемо за теоремою Гауса
	\[\frac{1}{\epsilon_0}\lambda\cdot1=2\pi r\cdot 1\cdot E\implies E=\frac{\lambda}{2\pi\epsilon_0 r}\]
\begin{prob}\end{prob}%18
	Скористаємось теоремою Гауса-Остроградського. Дивергенція рівна
	\[(\myvec{\nabla},\myvec{E})=\frac{\partial}{\partial x}\mybra{\frac{ax}{x^2+y^2}}+\frac{\partial}{\partial y}\mybra{\frac{ay}{x^2+y^2}}=\]
	\[a\frac{y^2-x^2}{(x^2+y^2)^2}+a\frac{x^2-y^2}{(x^2+y^2)^2}=0\]
	а тому потік рівний нулю.
\begin{prob}\end{prob}%19
	Ми знайдемо лише поле в \textit{центрі} порожнини. Ситуацію можна представити як суперпозицію двох шарів: оригінального (без порожнини)
	і шара з центром на кінці вектора $\myvec{a}$ (з радіусом, скажімо, $r<R$, де $R$ -- радіус великого шару) і від’ємним зарядом. Оскільки
	в одній з попередніх задач ми розрахували поле, що діє на частинку всередені зарядженого шару, маємо
	\[\myvec{E}=\frac{\myvec{a}\rho}{3\epsilon_0}+\frac{\myvec{0}\cdot(-\rho)}{3\epsilon_0}=\frac{\myvec{a}\rho}{3\epsilon_0}\]
\end{document}
