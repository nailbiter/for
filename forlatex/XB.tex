\documentclass[10pt]{article} % use larger type; default would be 10pt

\newcommand{\norm}[1]{\left|\left|#1\right|\right|}
\usepackage{mystyle}
\usepackage{enumerate}
\usepackage{CJKutf8}

\title{45901-18, 基礎解析学概論\\Final Report}
\author{Alex Leontiev, 45-146044}
\begin{document}
\begin{CJK}{UTF8}{bsmi}
\maketitle
\end{CJK}
\begin{enumerate}[\bf{[}1{]}]
	\item Indeed, let $0=\alpha_0\leq\alpha_1\leq\hdots\leq\alpha_n=1$ be any partition of $[0,1]$. We shall show that
		\[\sum_{i=0}^{n-1}\myabs{f(\alpha_i)-f(\alpha_{i+1})}\leq M.\]
		where $N$ is some constant, independent of $\mycbra{\alpha_i}$. It is a well-known fact, that for $C^1$ function
		on a closed interval $[a,b]$
		\[TV(f;a,b)=\int_a^b\myabs{f'(x)}\;dx\]
		Keeping this in mind, we write
		\[\sum_{i=0}^{n-1}\myabs{f(\alpha_i)-f(\alpha_{i+1})}=\]
		\[=\myabs{f(\alpha_0)-f(\alpha_1)}+\sum_{i=1}^{n-1}\myabs{f(\alpha_i)-f(\alpha_{i+1})}=\]
		\[\leq\myabs{f(\alpha_0)-f(\alpha_1)}+TV(f;\alpha_1,1)
		\leq\myabs{f(\alpha_0)-f(\alpha_1)}+\int_{\alpha_1}^1\myabs{f'(x)}\;dx\leq\]
		\[\leq\myabs{f(\alpha_0)-f(\alpha_1)}+\int_0^1\myabs{f'(x)}\;dx.\]
		Now, as the function $f$ is continuous on $[0,1]$, the first term is bounded independently of partition (by $2\sup
		\myabs{f(x)}$), so it just remains to show that the second (improper integral) is also bounded. This is simple, since
		\[\int_0^1\myabs{f'(x)}\;dx=\int_0^1\myabs{2x\sin\frac{1}{x}-\cos\frac{1}{x}}\;dx\leq\]
		\[\leq\int_0^1\myabs{2x\sin\frac{1}{x}}\;dx+\int_0^1\myabs{\cos\frac{1}{x}}\;dx\]
		The first integral converges, as it is the integral of a continuous function on $[0,1]$. Regarding the second,
		via the variable change we have
		\[\int_0^1\myabs{\cos\frac{1}{x}}\;dx=\int_1^{\infty}\myabs{\cos x}\frac{dx}{x^2}\leq\int_1^{\infty}\frac{1}{x^2}
		<\infty\]
	\item Indeed, in particular the well-known Cantor distribution is known to possess these properties. In what follow,
		we shall outline its construction and explain, why it indeed has the properties required.

		Let $c:[0,1]\to[0,1]$ be the Cantor function. We recall the properties of it, we shall need in subsequent
		\begin{enumerate}[1)]
			\item $c$ is continuous on $[0,1]$;
			\item $c$ is non-decreasing no $[0,1]$;
		\end{enumerate}
		Having these, we can define a measure $\mu$ on $[0,1]$, related to $c$ and characterized by $\mu((0,x])=c(x)$.
		Now, if for some $x_0\in[0,1]$ we would have $\mu(\mycbra{x_0})>0$, this would imply the discontinuity in $c$, hence
		$\mu$ is atomless. It is also \textit{not} absolutely continuous w.r.t. Lebesgue measure, as if it would be,
		there would be a Lebesgue integrable function $g$, such that $\mu(A)=\int_Ag\;dx$. That $g$, in turn, would have
		to be equal to zero almost everywhere, as $c'=0$ is equal to zero outside the Cantor set, hence almost everywhere,
		which is impossible, since $\mu$ is non-trivial by construction ($\mu([0,1])=1>0$).
		\setcounter{enumi}{5}
	\item By symmetry, it is enough to show that
		\[\mbox{dist}(u_1,A)-\mbox{dist}(u_2,A)\leq \mynorm{u_1-u_2}\]
		Let $\mycbra{x_n}_{n=1}^\infty$ and
		$\mycbra{y_n}_{n=1}^\infty$
		be sequences in $A$, so that $\mynorm{u_1-x_n}\to\mbox{dist}(u_1,A)+$ and $\mynorm{u_2-y_n}\to\mbox{dist}(u_2,A)+$ 
		respectively. Let $\epsilon>0$ be arbitrary. By dropping (if necessary) finitely many terms from the beginning
		of every sequence, we may assume $\forall n\geq1,\;\mynorm{u_1-x_n}\leq\mbox{dist}(u_1,A)+\epsilon$ and
		$\forall m\geq1,\;\mynorm{u_2-y_m}\leq\mbox{dist}(u_2,A)+\epsilon$.

		Now, by triangle inequality we have
		\[\mynorm{u_1-u_2}\geq\mynorm{u_1-y_1}-\mynorm{u_2-y_1}\]
		By taking $n$ big enough, we may also have $\mynorm{u_1-x_n}\leq\mynorm{u_1-y_1}$, thus we will have
		\[\mynorm{u_1-u_2}\geq\mynorm{u_1-x_n}-\mynorm{u_2-y_1}\geq\mbox{dist}(u_1,A)-\mbox{dist}(u_2,A)-\epsilon\]
		and since $\epsilon$ was arbitrary, we are done.
\end{enumerate}
\end{document}
% 6--> 3 --> 4
