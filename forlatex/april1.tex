\documentclass[8pt]{article} % use larger type; default would be 10pt

%\usepackage[utf8]{inputenc} % set input encoding (not needed with XeLaTeX)
\usepackage[T1]{fontenc}
%\usepackage{CJK}
\usepackage{graphicx}
\usepackage{float}
\usepackage{CJKutf8}
\usepackage{subfig}
\usepackage{amsmath}
\usepackage{amsfonts}
\usepackage{hyperref}
\usepackage{enumerate}
\usepackage{enumitem}

\newcommand{\norm}[1]{\left|\left|#1\right|\right|}
\usepackage{mystyle}

\title{}
\author{Igor Tereshkov}
\begin{document}
\maketitle
\begin{enumerate}[label=\bfseries Problem \arabic*.]
	\item{In subsequent, we shall denote the identity element of $G$ by $e$.
		\begin{enumerate}[label=(\arabic*)]
			\item{Indeed, we just need to verify three standard properties:
				\begin{description}
					\item[(Reflexivity)]{As $A$ and $B$ both are subgroups of $G$, $e\in A,\;e\in B\implies
						\forall g\in G,\; g=ege,\;e\in A,\;e\in B\implies \forall g\in G,\; g\sim g$,
						which is exactly reflexivity.}
					\item[(Symmetry)]{Assume $h,g\in G$ and $h\sim g\implies h=agb,\;a\in A,b\in B$. 
						Then, as $A$ and $B$ are groups, they are
						closed under taking the inverse and hence, $g=a^{-1}hb^{-1},\;a^{-1}\in A,\;b^{-1}\in
						B\implies g\sim h$, which is exactly symmetry.}
					\item[(Transitivity)]{Finally, assume $h,g,k\in G,\; h\sim g,g\sim k\implies
						h=agb,\;g=a'kb',\;a,a'\in A,\;b,b'\in B$. Substituting
						the expression for $g$ in right-hand
						side of an expression for $h$ and noting
						that $A$ and $B$ are both closed under multiplication (as they are groups)
						, we get $h=aa'kb'b,\;aa'\in A,\;bb'\in B\implies
						h\sim k$, which is exactly transitivity.}
				\end{description}
				}
			\item{Indeed, let $g\in G$ be arbitrary and let's denote it's equivalence class under the relation above
				as $[g]$. What we need to prove is then written simply as
				\[[g]=\mysetn{agb}{a\in A,\;b\in B}\]
				First, let us prove $[g]\subset \mysetn{agb}{a\in A,\;b\in B}$. Let $h\in[g]$ be arbitrary,
				then $h\sim g\implies h=agb,\;a\in A,\;b\in B\implies h\in\mysetn{agb}{a\in A,\;b\in B}$, which
				shows the desired inclusion.

				Second, we aim at showing $\mysetn{agb}{a\in A,\;b\in B}\subset [g]$. For this let
				$h\in\mysetn{agb}{a\in A,\;b\in B}$ be arbitrary. It then can be written as $h=agb,\;a\in A,\;b\in B$
				and thus by definition $h\sim g\implies h\in [g]$, which shows the second desired inclusion.
				}
			\item{Let $g\in G$ be arbitrary. Let us further define the mapping $F:A\times B\ni(a,b)\mapsto agb\in AgB$. In subsequent we shall define the 
				equivalence relation on $A\times B$ as $(a,b)\sim (a',b')\iff agb=a'gb'$ and show that $\forall (a,b)\in A\times B,\;\myabs{[(a,b)]}=
				\myabs{A\cap gBg^{-1}}$. This will give us the desired
				\[\myabs{AgB}=\frac{\myabs{A}\myabs{B}}{\myabs{A\cap gBg^{-1}}}\]

				Let us fix arbitrary $(a,b)\in A\times B$. To show 
				$\forall (a,b)\in A\times B,\;\myabs{[(a,b)]}=\myabs{A\cap gBg^{-1}}$, we shall exhibit the explicit bijection.
				On the one hand, given
				$(\alpha,\beta)\in[(a,b)]$ this means that $(\alpha,\beta)\sim(a,b)\implies agb=\alpha g\beta\implies
				\alpha^{-1}a=g\beta b^{-1}g^{-1}\in A\cap gBg^{-1}$. On the other hand, given $\alpha_0=g\beta_0 g^{-1}\in A\cap
				gBg^{-1}$ we may see that for $\alpha:=a\alpha_0^{-1}\in A$ and $\beta:=\beta_0 b$ we see that
				$\alpha^{-1}a=g\beta b^{-1}g^{-1}\in A\cap gBg^{-1}\implies (\alpha,\beta)\in[a,b]$. As 
				$[a,b]\ni(\alpha,\beta)\leftrightarrow \alpha_0=\alpha^{-1}a\in A\cap gBg^{-1}$ is bijective as shown above
				, we are done.

				No, in general not all the cosets contain equal number of elements. For example, if $G=S_3$, $A=B=\left\{
				e,(1,2)\right\}$, we have
				\[\myabs{AeB}=\myabs{A}=2\]
				whereas for $g=(1,2,3)$ $gAg^{-1}=\{e,(2,3)\}$ and consequently
				\[\myabs{AgB}=\frac{2\cdot 2}{\myabs{A\cap gAg^{-1}}}=\frac{4}{\myabs{\{e\}}}=4\]
				}
			\end{enumerate}
			}
		
	\item{Indeed, let $f:\mathbb{R}\to\mathbb{C}^*$ be and arbitrary group homomorphism between $\mathbb{R}$ and $\mathbb{C}^*$
		and let us further assume that it is continuous.
		Recall that for $H:=\mysetn{z\in\mathbb{C}}{\Im z>0}$ we can define an analytic branch of $\log z$, so that $\log 1=0$ and besides
		if $z_1,z_2,z_1z_2\in H\implies \log(z_1z_2)=\log z_1+\log z_2$. Now as $f$ is continuous and $f(0)=1\in H$
		(since $f$ is homomorphism) there exists $N\in
		\mathbb{Z},\;N>0
		$, such that $f((-\frac{2}{N},+\frac{2}{N}))\subset H$. Now, let us denote $c:=N\log f(\frac{1}{N})$. Then, in particular,
		$f(\frac{1}{N})=e^{c\cdot\frac{1}{N} }$.
		
		Let us show that for any $m\in\mathbb{Z},\;m>0\implies f(\frac{1}{m})=e^{c\cdot\frac{1}{m}}$. Indeed, by the property of homomorphism
		\[c\cdot\frac{1}{N}=\ln f(\frac{1}{N})=\ln(f(\frac{1}{Nm})^m)=m\ln f(\frac{1}{Nm})\]
		Therefore, $f(\frac{1}{Nm}=e^{c\cdot\frac{1}{Nm}}\implies f(\frac{1}{m})=e^{c\cdot\frac{1}{m}}$. In particular, $f(1)=e^c$.


		Now, as $f(x)$ is group homomorphism, we have $\forall n\in\mathbb{Z},\;n>0\implies
		f(n)=f\left(\underbrace{1+1+\cdots+1}_{n}\right)=\underbrace{f(1)\cdot f(1)\cdot\dotsm\cdot f(1)}_{n}=e^{cn}$.
		Therefore $\forall x\in\mathbb{Z},\;x>0\implies f(x)=e^{cx}$. As $f$ is homeomorphism, identity
		maps to identity, hence $f(0)=1=e^{c\cdot 0}$.

		Moreover, as $1=f(0)=f((-1)+1)=f(-1)\cdot f(1)=f(-1)e^{c}$ we have $f(-1)=e^{-c}$. Similarly to above, if $n\in
		\mathbb{Z},\; n>0$, we have
		$f(-n)=f\left(\underbrace{
		(-1)+(-1)+\cdots+(-1)}_{n}\right)=\underbrace{f(-1)\cdot f(-1)\cdot\cdots\cdot f(-1)}_{n}=e^{-cn}=e^{c(-n)}$
		And altogether these tell us that $\forall x\in\mathbb{Z},\;f(x)=e^{cx}$.

		Now let us take arbitrary $x\in\mathbb{Q}$. It can be written as $x=\frac{p}{q},\;p,q\in\mathbb{Z}$. According to the derivation
		in the first paragraph, $f(p/q)=f(1/q)^p=(e^{c/q})^p=e^{c\cdot \frac{p}{q}}$ and hence $\forall x\in\mathbb{Q},\;f(x)=e^{cx}$.
		Finally, by continuity of $f(x)$ and $e^{cx}$ the equality of these two functions on $\mathbb{Q}$ implies that on $\mathbb{R}$.
		}
	\item{
		\begin{enumerate}[label=(\arabic*)]
			\item{Indeed, $I_c$ is an ideal, for if $f,g\in I_c$ and $h\in C[a,b]$ we have
				\[0\in I_c,\;(-f)(c)=-f(c)=0\implies -f\in I_c\]
				\[(f+g)(0)=f(0)+g(0)=0\implies f+g\in I_c\]
				\[(hf)(0)=h(0)f(0)=0\implies hf\in I_c\]
				Moreover, it is a prime ideal, for it is definitely proper ($1\notin I_c$) and
				if $fg\notin
				I_c$ we have $(fg)(0)=f(0)g(0)\neq 0\implies fg\notin I_c$ (as the product of nonzero numbers is nonzero).
				}
			\item{We shall taken as known the fact that for any $\epsilon>0$ there exists a continuous function, which we shall denote
				by $h_{\epsilon}$, such that $h_{\epsilon}\mid_{[-\epsilon,\epsilon]}\equiv 1$ and $h_{\epsilon}$ is equal to zero
				outside $(-2\epsilon,2\epsilon)$.
				
				Now let $I\subset C[a,b]$ be a maximal ideal and let us assume it is not equal to $I_c$ for any $c\in[a,b]$.
				Then, in particular for any $c\in[a,b],\;\exists f_c\in I\mid f_c(c)\neq 0$, for otherwise we would have
				$\forall f\in I,\; f(c)=0$ for some $c\in[a,b]$ and thus $I$ would be contained in proper ideal $I_c$ and the
				inclusion would be strict (as two are unequal by assumption), thus contradicting maximality. By replacing
				$f_c$ with $-f_c$, if necessary, we may assume $\forall c\in[a,b],\;f_c(c)>0$. As every $f_c$ is continuous,
				there is some $\epsilon_c>0$, such that $f_c\mid_{(c-\epsilon_c,c+\epsilon_c)\cap [a,b]}>0$. By multiplying
				$f_c$ be $h_{\epsilon_c/2}(x-c)$ we may assume that $f_c$ vanishes outside $(c-\epsilon_c,c+\epsilon_c)\cap [a,b]$
				and is positive at the points, where it is nonzero.
				As $(c-\epsilon_c,c+\epsilon_c)\cap [a,b]$ form an open cover of $[a,b]$ (in the topology of $[a,b]$), we can
				select only finitely many of them for cover. After we will add corresponding $f_c$, we will get the function
				$f\in I$ (which is additively closed) which is positive on $[a,b]$. Therefore $\frac{1}{f}\cdot f=1\in I$ and
				hence $I=C[a,b]$ is not probper. Contradiction.
				}
		\end{enumerate}
		}
	\item{
		If $P\subset S$ is a prime ideal, $f^{-1}(P)\subset R$ is an ideal,
		as \[f(0)=0\in P\implies 0\in f^{-1}(P)\]
		\[a\in f^{-1}(P)\implies f(a)\in P\implies -f(a)=f(-a)\in P\implies -a\in f^{-1}(P)\]
		\[a,b\in f^{-1}(P)\implies f(a),f(b)\in P\implies f(a+b)=f(a)+f(b)\in P\implies a+b\in f^{-1}(P)\]
		\[a\in f^{-1}(P),x\in R\implies f(a)\in P\implies f(ax)=f(a)f(x)\in P\implies ax\in f^{-1}(P)\]
		
		If $a,b\notin f^{-1}(P)$ we have $f(a),f(b)\notin P\implies f(ab)=f(a)f(b)\notin P\implies
		ab\notin f^{-1}(P)$, hence $f^{-1}(P)$ is a prime ideal.
		}
\end{enumerate}
\end{document}
