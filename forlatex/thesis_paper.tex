\documentclass{elsarticle}

\usepackage{lineno,hyperref}
\usepackage{mathtext}
\usepackage{empheq}
\usepackage{mystyle}
\modulolinenumbers[5]

%custom theorem environments
\newtheorem{definition}{Definition}[section]
\renewcommand{\thedefinition}{\arabic{definition}}
\newtheorem{example}{\indent Example}[section]
\renewcommand{\theexample}{\arabic{example}}
\newtheorem{exercise}{Exercise}
\newtheorem{theorem}{Theorem}
\newtheorem{lemma}{Lemma}
\newtheorem{observation}{Observation}
\newtheorem*{fact}{Fact}
\newtheorem{proposition}{Proposition}
\newtheorem{corollary}[proposition]{Corollary}
\theoremstyle{remark}
\newtheorem{remark}{Remark}

\journal{Journal of Mathematical Analysis and Applications}
\bibliographystyle{elsarticle-num}

\begin{document}

\begin{frontmatter}

\title{The new criterion for the roughness of exponential dichotomy on $\mathbb{R}$}

\author[todai]{Oleksii Leontiev\corref{cor}}
\ead{leontiev@ms.u-tokyo.ac.jp}
\author[erfurt]{Petro Feketa}
\ead{petro.feketa@gmail.com}

\cortext[cor]{Corresponding author}

\address[todai]{Tokyo University}
\address[erfurt]{University of Erfurt}

\begin{abstract}
Here will be the abstract.
\end{abstract}

\begin{keyword}
keyword1 \sep keyword2 \sep keyword3
\MSC[2010] 00-01\sep  99-00
\end{keyword}

\end{frontmatter}

\linenumbers

\section{Introduction}
%TODO: history, related work
The system 
\[x'(t)=A(t)x(t)\]
where $x(t)$ and $A(t)$ are respectively vector-valued and square-matrix-valued functions defined for $t\in S\subseteq\mathbb{R}$
is said to possess \textbf{"exponential dichotomy"} on $S$ (or simply "dichotomy" in subsequent), if the phase space $R^n$
can be decomposed into the direct sum of subspaces
$E^+\oplus E^-$ and for arbitrary $x^+\in E^+,\; x^-\in E^-$ and $a,b\in S$ the following estimates hold:
\begin{equation}\begin{aligned}\label{DichotomyDef}
	\mynorm{\Omega^b_0(A)x^+}\leq K\mynorm{\Omega_0^a(A)x^+}\exp\left\{-\gamma\myabs{a-b}\right\},\;a\leq b\\
	\mynorm{\Omega^b_0(A)x^-}\leq K\mynorm{\Omega_0^a(A)x^-}\exp\left\{-\gamma\myabs{a-b}\right\},\;b\leq a\\
\end{aligned}\end{equation}
for some constants $K,\;\gamma>0$, that do not depend on $x^+,\;x^-,\;a,\;b$.

\begin{remark}The alternative definition, as given in \cite{coppel1978dichotomies} can be formulated as follows: there exist projection-operator $P$ and numbers
$K,\;\gamma>0$, such that
 for $\forall a,b\in S$
\[\mynorm{\Omega_0^a(A)P\mybra{\Omega_0^b(A)}^{-1}}\leq K e^{-\gamma(a-b)},\;a\geq b\]
\[\mynorm{\Omega_0^a(A)(I-P)\mybra{\Omega_0^b(A)}^{-1}}\leq K e^{-\gamma(b-a)},\;b\geq a\]
\end{remark}
\section{Results}
%TODO: setting
In subsequent, all matrix functions will be assumed to be continuous. 
%%In principle, results probably can be extended to partially
%%give up this assumption, but for simplicity we shall keep it.
In particular, continuity assumption will let us immediately assume
that Cauchy problem for every ODE system $x'=A(t)x$ has a unique solution on $\mathbb{R}$ for any initial conditions.
\begin{proposition}
	\label{strong}
	Assume that for the two systems 
	\begin{equation}\label{Unperturbed}
	x'=A(t)x(t)
	\end{equation}
	and
	\begin{equation}\label{Perturbed}
	x'=(A(t)+B(t))x(t)
	\end{equation}
	the following two conditions hold:
	\begin{enumerate}[(H1)]
	\item For any $t,s\in\mathbb{R}$ matrices $A(s)$ and $B(t)$ commute;
	\item $B\in L^1(\mathbb{R})\cap C(\mathbb{R})$, that is $\int_{-\infty}^{\infty}\mynorm{B(s)}\;ds<\infty$.
	\end{enumerate}
	Then, \eqref{Perturbed} has exponential dichotomy on $\mathbb{R}$ if \eqref{Unperturbed} does.
\end{proposition}
%%\begin{remark}The boundedness hypothesis of the statement \ref{strong} (the second condition) holds in particular if
%%$\int_{-\infty}^\infty\mynorm{B(s)}\;ds<+\infty$ or $\lim_{\myabs{t}\to\infty}\mynorm{B(t)}=0$.
%%\end{remark}
\begin{remark}The hypothesis of the proposition \ref{strong} about the commutativity (the first condition) holds in particular if
	$\forall t,\;B(t)\in\bigcap\limits_{s\in\mathbb{R}} Z_{\mathfrak{gl}_n}(A(s))$, where
	$Z_{\mathfrak{gl}_n}(A)$ denotes the (Lie algebra) centralizer of $A$.
	If furthermore $A(t)\equiv A$, this condition can be simply rewritten as $\forall t,\;B(t)\in Z_{\mathfrak{gl}_n}(A)$.\end{remark}
\section{Proofs}
\begin{proof}{(of proposition \ref{strong}).\;}
The desired result will follow from two observations, which we shall state now and prove later. It should be noted that the observations have
weaker hypothesis than that of the proposition, and are useful results on their own.
\begin{observation}\label{Hard}Let $A$ and $B$ be continuous (not necessary bounded matrices-valued functions on $\mathbb{R}$). Assume further
that for every $s,t\in\mathbb{R}$ $A(s)$ and $B(t)$ commute. We shall also assume that for system $x'=Bx$, every Cauchy problem has a unique
solution on $\mathbb{R}$, in other words, that the fundamental matrix (we shall denote it by $X_B$) is well-defined on $\mathbb{R}$. Then,
for arbitrary $t,s\in\mathbb{R}$, $A(t)$ and $X_B(s)$ commute.
\end{observation}
\begin{observation}\label{Easy}Let $B$ be continuous matrix-valued function, such that $B\in L^1\cap C$. Then,
the fundamental matrix $X_B$ of $x'=Bx$ is well-defined on $\mathbb{R}$, and $\exists C\forall t,\;\mynorm{X_B(t)},\mynorm{X^{-1}_B
(t)}\leq C$.
\end{observation}
Assuming these two for the moment, let us see how the proof of the proposition can be obtained from them. Consider the $x'=Bx$ system, whose
fundamental matrix (existing by observation \ref{Easy}) we shall denote by $X_B$. Let us make the variable change $x=X_Bz$. Then, we have
\[BX_Bz+X_Bz'=(X_Bz)'=x'=(A+B)X_Bz\]
Now, as by observation \ref{Hard} $A$ and $X_B$ commute, we have
\[X_Bz'=X_BAz\]
thus $z$ is the solution to $z'=Az$, hence the fundamental matrix of $x'=(A+B)x$ has the structure
\[X_{A+B}=X_BX_A\]
Now, from \cite{coppel1978dichotomies} we know, that the exponential dichotomy on $\mathbb{R}$ can be defined as the existence of projection operator $P$
and numbers $\gamma,K>0$, such that $\forall s,t\in\mathbb{R}$ we have
\[\mynorm{X(t)PX^{-1}(s)}\leq Ke^{-\gamma(t-s)},\;s\leq t\]
\[\mynorm{X(t)(I-P)X^{-1}(s)}\leq K^{-\gamma(s-t)},\;t\leq s\]
We shall show, that this estimation remains the same for perturbed system \refeq{Perturbed} with the same $P$ and $\gamma$, but perhaps different $K$
. Indeed, we shall have
\[\mynorm{X_{A+B}(t)PX^{-1}_{A+B}(s)}=\mynorm{X_B(t)X_A(s)PX^{-1}_A(t)X^{-1}_B(t)}\leq\]
\[\leq\mynorm{X_B(t)}\cdot\mynorm{X_A(s)PX^{-1}_A(t)}\cdot\mynorm{X^{-1}_B(t)}\leq \underbrace{KC^2}_{=:\tilde{K}}e^{-\gamma(t-s)}.\]
The second inequality is worked out in the same way, so it just remains to prove the observations stated above.
\end{proof}
\begin{proof}{(of the observation \ref{Easy}).\;}The existence of fundamental matrix follows immediately from the continuity assumption. The uniform
boundedness of $\mynorm{X_B}$
 follows from the \cite[Corollary 3.24]{teschlordinary} (which states that if the system $x'=Cx$ with $C$ constant is uniformly stable and
 $B\in L^1\cap C$, then $x'=(C+B)x$ is stable as well; in our case we take obviously stable system $x'=0\cdot x$ as $x'=Cx$)
 . We shall denote the upper bound obtained for $X_B$ by $M$. It is
sufficient to show that $\mynorm{X^{-1}_B(t)}\leq M$. Let us fix $\tau\in\mathbb{R}$. $X^{-1}_B(\tau)$ can be characterized as $X^{-1}_B(\tau)x(\tau)
=x(0)$, where $x$ is the solution to $x'=Bx$ or as $X^{-1}(\tau)y(0)=y(-\tau)$, where $y'=\tilde{B}y$ and $\tilde{B}(t):=B(t+\tau)$. Thus,
it's enough to show that the norm of $X_{\tilde{B}}$ is uniformly bounded by $M$ as well. The latter, however, follows from the proof
of \cite[Corollary 3.24]{teschlordinary}, as the upper bound obtained there really depends only on the
$\int_{-\infty}^\infty\mynorm{\tilde{B}(s)}\;ds=\int_{-\infty}^\infty\mynorm{B(s)}\;ds$.\end{proof}
\begin{proof}{(of the observation \ref{Hard}).\;}
We shall employ the theory of Magnus expansion, described in \cite{Moan:2008:CMS:1391929.1391932}. For convenience, we collect
the properties we need in the following proposition
\begin{fact}{(\textbf{Magnus Expansion}) from \cite{Moan:2008:CMS:1391929.1391932}}
	Assume that we have a system $x'=Bx$ with continuous $B$ and fundamental matrix $X$. Assume
	further that for $T$ we have
	\[\myabs{\int_0^T\mynorm{B(s)}\;ds}<\pi\]
	then, on $t\in[0,T)$ $X(t)=\exp(\Omega(t))$, where $\Omega(t)=\sum\limits_{k=1}^\infty\Omega_k(t)$ and $\Omega_k(t)$ is an integral
	of commutators of increasing length, such as
	\begin{align*}
	\Omega_1(t) &= \int_0^t B(t_1)\,dt_1, \\
	\Omega_2(t) &= \frac{1}{2}\int_0^t dt_1 \int_0^{t_1} dt_2\ \left[  B(t_1),B(t_2)\right], \\
	\Omega_3(t) &= \frac{1}{6} \int_0^t dt_1 \int_0^{t_1}d t_2 \int_0^{t_2} dt_3
	\Bigl(\left[B(t_1),\left[B(t_2),B(t_3)\right]\right]+\left[B(t_3),\left[  B(t_2),B(t_{1})\right]\right]\Bigr), \\
	\Omega_4(t) &= \frac{1}{12} \int_0^t dt_1 \int_0^{t_1}d t_2 \int_0^{t_2} dt_3 \int_0^{t_3} dt_4
	\Bigl(\left[\left[\left[B_1,B_2\right],B_3\right],B_4\right] \\
	&\quad+\left[B_1,\left[\left[B_2,B_3\right],B_4\right]\right]
	+\left[B_1,\left[B_2,\left[B_3,B_4\right]\right]\right]
	+\left[B_2,\left[B_3,\left[B_4,B_1\right]\right]\right]\Bigr)
	\end{align*}
\end{fact}
Having this, we shall fix $\tau\in\mathbb{R}$ and show that for any $t\in\mathbb{R}$,
\[\mysbra{A(\tau),X_B(t)}=0\]
where $\mysbra{\cdot,\cdot}$ denotes the Lie bracket. As for $t=0$ the equality holds (as $X(0)=I$ commutes with all matrices), it is enough
to show that the set where equality holds is both open and closed. As both $A$, $X$ and Lie bracket are continuous, closedness follows and it
remains thus to show, that if $\mysbra{A(\tau),X(t_0)}=0$, then equality also holds on a small neighborhood of $t_0$.

It is sufficient to show that $A(\tau)$ commutes with $\tilde{X}(s):=X(t_0+s)X^{-1}(t_0)$ for small $s$. Now, $\tilde{X}$ can be realized
as a fundamental matrix of $x'=\tilde{B}x$, where $\tilde{B}(t)=B(t+t_0)$. Thus, without loss of generality we may assume that
$t_0=0$. Now, on the small
neighborhood of $0$ the hypothesis of ~\hyperref[MagnusConvergenceFact]{fact} is satisfied, hence we may apply Magnus expansion.

Now, if $A$ and $B$ commute with $C$, then so do $A\pm B$, $AB$ and $[A,B]$. Also, if $B(t)$ commutes with $A$ for all $t$, then so does
$\int_a^bB(s)\;ds$. Thus, we see that every $\Omega_k(t)$ commutes with $A(\tau)$, as former is the integral of commutators of $B$. Finally,
as Lie bracket is continuous, we also have that $\Omega(t)$ commutes with $A(\tau)$ whenever well-defined, and so does $X(t)=\exp(\Omega(t))=
\sum_{n=0}^\infty\frac{\Omega^n(t)}{n!}$.
\end{proof}
\section{Corollaries and Examples}
\section{Further work}
\section*{References}

\bibliography{thesis_paper}

\end{document}
%cor: abstract --> intro -->keywords --> further worko
%cor: cors_and_examps: Schwartz, diagonal matrices --> ask feketa --> no o(1) counter-example
%FIXME: 

%%\bibitem{krein}
%%Ю. Л. Далецкий, М. Г. Крейн
%%\emph{Устойчивость решений дифференциальных уравнений в банаховом пространстве}.
%%Видавництво "Наука"{}, Головна редакція фізико-математичної літератури. Москва, 1970.
%%\bibitem{mitrop}
%%Митропольський Ю. А., Самойленко А. М., Кулик В. Л.
%%\emph{Исследование дихотомии линейных систем дифференциальных уравнений с помощью функций Ляпунова}.
%%АН УССР. Ін-т математики. Київ, "Наукова думка", 1990.
%%\bibitem{palmer84}
%%	Kenneth J. Palmer, {\em Exponential Dichotomies and Transversal Homoclinic Points} (1983).
%%\bibitem{palmer88}
%%	Kenneth J. Palmer, {\em Exponential Dichotomies and Fredholm Operators} (1988).
%%\bibitem{ju}
%%	Ning Ju and Stephen Wiggins, {\em On Roughness of Exponential Dichotomy} (2000).
%%\bibitem{naulin}
%%	Ra\'ul Naulin and Manuel Pinto, {\em Admissible perturbations of Exponential Dichotomy Roughness} (1997).
%%\bibitem{chow}
%%	Shui-Nee Chow and Hugo Leiva, {\em Existence and Roughness of the Exponential Dichotomy for Skew-Product Semiflow in Banach Spaces} (1994).
%%\bibitem{demidovich}
%%Demidovich B. P. \emph{Lectures on the mathematical theory of stability} --
%%Moscow (1967) (in Russian).
