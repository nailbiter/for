\documentclass[8pt]{article} % use larger type; default would be 10pt

%\usepackage[utf8]{inputenc} % set input encoding (not needed with XeLaTeX)
\usepackage[10pt]{type1ec}          % use only 10pt fonts
\usepackage[T1]{fontenc}
%\usepackage{CJK}
\usepackage{graphicx}
\usepackage{float}
\usepackage{CJKutf8}
\usepackage{subfig}
\usepackage{amsmath}
\usepackage{amsfonts}
\usepackage{hyperref}
\usepackage{enumerate}

%theorem environments configuration
\newtheorem{theorem}{Theorem}
\newtheorem{lemma}[theorem]{Lemma}
\newtheorem{proposition}[theorem]{Proposition}
\newtheorem{corollary}[theorem]{Corollary}
\newenvironment{proof}[1][Proof]{\begin{trivlist}
\item[\hskip \labelsep {\bfseries #1}]}{\qed\end{trivlist}}
\newenvironment{definition}[1][Definition]{\begin{trivlist}
\item[\hskip \labelsep {\bfseries #1}]}{\end{trivlist}}
\newenvironment{example}[1][Example]{\begin{trivlist}
\item[\hskip \labelsep {\bfseries #1}]}{\end{trivlist}}
\newenvironment{remark}[1][Remark]{\begin{trivlist}
\item[\hskip \labelsep {\bfseries #1}]}{\end{trivlist}}
\newcommand{\qed}{\nobreak \ifvmode \relax \else
\ifdim\lastskip<1.5em \hskip-\lastskip
\hskip1.5em plus0em minus0.5em \fi \nobreak
  \vrule height0.75em width0.5em depth0.25em\fi}

\newtheorem{prob}{Problem}

\newenvironment{solution}%
{\par\textbf{Solution}\space }%
{\par}

\title{Algebra II\\Homework 0}
\author{Oleksii (Alex) Leontiev\\歐立思\\3035078276\\Exchange Student (BSc)\\4th grade\\
Original School: \href{http://www.nctu.edu.tw/}{NCTU}, Taiwan}

\begin{document}
\begin{CJK}{UTF8}{bsmi}
\maketitle
\end{CJK}

\begin{theorem} Let $u$ be algebraic over $F$ with minimal polynomial $p(x)$. Then
\[[F(u):F]=\deg p(x)\]
\end{theorem}

\begin{proof}
Let $n:=\deg p(x)$. It will be
 sufficient to show that set $S:=\{u^0=1,u^1,\dots,u^{n-1}\}$ forms an $F$-basis for $F(u)$. To do this we need to show:
\begin{enumerate}
\item{$S$ spans $F(u)$ over $F$.\\ Indeed, let $g\in F(x)$. From previous result $F(x)=F[x]$ and therefore
\[g=a_0+a_1u+\dots a_m u^m,\;a_i\in F\]
Define $G(x):=a_0+a_1x+\dots+a_mx^m$. By applying Euclidean division
\[G(x)=Q(x)p(x)+r(x),\;\deg r(x)<\deg p(x)\]
Moreover, since $p(u)=0$
\[g=G(u)=Q(u)p(u)+r(u)=Q(u)\cdot 0+r(u)\implies g=G(u)=r(u)\]
But $\deg r(x)<\deg p(x)\implies g=r_0+r_1 u+\dots r_{n-1}u^{n-1}$ is expressible as linear combination of elements of $S$. 
Since $g$ was arbitrary, $S$ spans $F(u)$}
\item{$S$ is linearly independent over $F$.\\
Indeed, let $a_0+a_1u+\dots+a_{n-1}u^{n-1}=0$. Define $f(x)=a_0+a_1x+\dots a_{n-1}x^{n-1}$. Then we have $\deg f(x)<n=\deg p(x)$ and $f(u)=0$.
Hence, $f(x)\equiv 0\implies a_i=0$
From two items above, $S$ forms an $F$-basis for $F(u)$ and this finishes the proof.
}
\end{enumerate}
\end{proof}

\begin{theorem} Let $K$ be finite extension of $E$ with basis $\alpha_1,\alpha_2,\dots,\alpha_n\in E$ and $E$ in turn be finite
extension of $F$ with basis $\beta_1,\beta_2,\dots,\beta_m\in F$. Then, $K$ is finite extension of $F$ and moreover $[K:F]=m\cdot n$
\end{theorem}

\begin{proof}
It will be sufficient to show that set $S:=\{\alpha_i\beta_j\}_{i,j=1}^{n,m}$ of $mn$ elements forms an $F$-basis for $K$. As with previous result,
we need two things:
\begin{enumerate}
\item{$S$ spans $K$ over $F$.\\
Let $k\in K$. Then, since 
$\{\alpha_1,\alpha_2,\dots,\alpha_n\}$ is an $E$-basis for $K$,
\[k=\sum_{i=1}^n e_i\alpha_i,\; e_i\in E\]
Since in turn $\{\beta_1,\beta_2,\dots,\beta_m\}$ form an $F$-basis for $E$ we have
\[\forall i\in\{1,2,\dots,n\}:\quad e_i=\sum_{j=1}^m f_{ij}\beta_j,\quad f_{ij}\in F\]
Hence, finally
\[k=\sum_{i,j=1}^{n,m}f_{ij}\beta_j\alpha_i\;f_{ij}\in F\]
Since $k$ was arbitrary, $S$ spans $K$}
\item{$S$ is linearly independent over $F$.\\
	Indeed, let
	\[\sum_{i,j=1}^{n,m}f_{ij}\beta_j\alpha_i=0\;f_{ij}\in F\]
	Note that
	\[\sum_{i,j=1}^{n,m}f_{ij}\beta_j\alpha_i=\sum_{i=1}^n (\sum_{j=1}^m f_{ij}\beta_j)\alpha_i\]
	Since
	$\{\alpha_1,\alpha_2,\dots,\alpha_n\}$ is linearly independent over $E$
	\[\forall i\in\{1,2,\dots,n\},\quad \sum_{j=1}^m,\; f_{ij}\beta_j=0\]
	And because $\{\beta_1,\beta_2,\dots,\beta_m\}$ is linearly independent over $F$, we have $\forall i,j f_{ij}=0$ and therefore $S$ is
	linearly independent over $F$
}
From two items above, $S$ forms an $F$-basis for $K$ and this finishes the proof.
\end{enumerate}
\end{proof}
\end{document}
