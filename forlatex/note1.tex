\documentclass[10pt]{article} % use larger type; default would be 10pt

\usepackage{mystyle}
\usepackage{enumerate}

\newcommand{\sgn}{\mbox{\normalfont{sgn}}}
\newcommand{\Aut}{\mbox{\normalfont{Aut}}}

\begin{document}
\begin{myprop}
	Let $(\lambda,\nu)\in\C^2$ and let $F\in D'(\R^n\setminus\mycbra{0})$, satisfying
	\[(E-(\lambda-\nu-n))F=0,\quad E:=\sum_jx_j\frac{\partial}{\partial x_j}\]
	\[((\lambda-n)x_j-x_jE+\frac{1}{2}\myabs{x}^2\frac{\partial}{\partial x_j})F=0,\quad\forall1\leq j<n\]
	\[F(-x)=F(x)\]
	\[F(mx,x_n)=F(x,x_n),\quad \forall m\in O(n-1)\]
	Then if $n\geq2$,
	\[F=\begin{cases}
		\C\myabs{x_n}^{\lambda+\nu-n}\myabs{x}^{-2\nu},&\lambda+\nu-n\neq-1,-3,-5,\hdots\\
		\C\delta^{\mybra{-\lambda-\nu+n-1}}(x_n)\myabs{x}^{-2\nu},&\lambda+\nu-n=-1,-3,-5,\hdots
	\end{cases}\]
	\end{myprop}
	\begin{proof}
		To begin with, substitute first equation into the second to get
		\[(\myabs{x}^2\frac{\partial}{\partial x_j}+2\nu x_j)F=0,\quad\forall1\leq j<n\]
		and consequently
		\[\frac{\partial}{\partial x_j}\mybra{\myabs{x}^{2\nu} F}=0,\quad\forall1\leq j<n\]
		Note, that the multiplication here, as well as all the multiplications made so far make perfect sense on $\R^n\setminus
		\mycbra{0}$, as $\myabs{x}^{2\nu}$ is locally integrable on this domain for any $\nu\in\C$.
		Now, we introduce the lemma, which will be proven below.
		\begin{mylem}\label{Lemma}
			Let $D=\R^n\setminus\mycbra{0}$ or $D=\R^n,\;n\geq2$,
			$F\in D'(D)$ and $\partial/\partial x_1F=0$ on $D$. Moreover, assume that
			$F(\pm x_1,x_2,\hdots,x_n)=F(x_1,x_2,\hdots,x_n)$. Then,
			\begin{enumerate}
				\item $F=g$ for some $g\in D'(\R^{n-1})$, that is
				\begin{equation}
					(F,\phi)=\mybra{g,\tilde{x}\mapsto\int_\R\phi(x_1,\tilde{x})\;dx_1},\quad\forall \phi\in C^\infty_0(D)\label{FandG}
				\end{equation}
				\item Moreover, if for any first order differential operator (with constant coefficients)
					$P$, $PF=0$ on $D$, then $Pg=0$ on $\R^{n-1}$ (where $\partial/\partial x_1g=0$ by convention).
			\end{enumerate}
		\end{mylem}
		Using this lemma, we see that there is $g\in D'(\R)$, such that 
		\[\forall \phi\in C^\infty_0(\R^n\setminus\mycbra{0}),\quad (F\myabs{x}^{2\nu}
		,\phi)=\mybra{g,x_n\mapsto\int_{\R^{n-1}}\phi(\tilde{x},x_n)\;d\tilde{x}} \]
		In turn, the fourth equation force $g$ to be even and the first translates to (we set $a:=\lambda+\nu-n$)
		$xg'-ag=0$, which holds on $\R$, thanks to the item 2) of Lemma.

		Now, it is sufficient to show that
		\[g(x)=\begin{cases}
		\C\myabs{x}^a,&a\neq-1,-3,-5,\hdots\\
		\C\delta^{\mybra{-a-1}}(x),&a=-1,-3,-5,\hdots
		\end{cases}\]
		We shall proceed as in \cite[Sec 3.11]{gelfand}. Note, that lemma from there tells us, that
		$g$ satisfies $xg'=ag$ iff it is homogeneous of degree $a$, that is $g(\alpha\cdot)=\alpha^ag(\dot),\;\forall\alpha>0$
		.

		First, we consider the restriction of $g$ to $\R_{>0}$. It can then be multiplied by $x^{-a}$, the later
		product being generalized function as well and having zero derivative on $\R_{>0}$, from which it'll follow
		(by the argument similar to proof of lemma \ref{Lemma}, albeit simpler)
		that $g\equiv const$. Hence, $g=Cx^a$ on $\R_{>0}$ and similarly $C'\myabs{x}^a$ for $\R_{<0}$.
		
		Hence (assuming $a\neq-1,-2,-3,\hdots$)
		using the introduced in \cite{gelfand} generalized functions $x^a_+$ and $x_-^a$, we
		see that $g-C_1x_+^a-C_2x_-^a$ is supported only at zero, and hence is linear combination of derivatives
		of $\delta(x)$.

		Now, as derivatives of delta are homogeneous by themselves and homogeneous functions with different $a$'s
		 are linearly independent
		(which can be directly verified by computations), we have that $g=C_1x_+^a+C_2x^a_-$ and evenness assumption
		tells us $g=C\myabs{x}^a$, if $a\neq-1,-2,-3,\hdots$.

		Tackling now the case $a=-n=-1,-2,-3,-\hdots$, we similarly get that $g-x^{-n}$ is the linear combination of
		derivatives of delta-functions, and again by linear independence of homogeneous classes we have
		\[g(x)=Cx^{-n}+C_1\delta^{\mybra{n-1}}\]
		And again, by applying evenness assumption the result follows.
	\end{proof}
	\begin{proof} (of lemma \ref{Lemma}) To begin with, note that the statement 2. is obvious, once 1. is proven, given expression
		\ref{FandG}. So we will concentrate ourselves on proving 1.

		Let us take some fixed $\psi_0\in C^\infty_0(\R^1)$, such that
		\newcommand{\tx}{\tilde{x}}
		\newcommand{\tp}{\tilde{\phi}}
		\begin{enumerate}
			\item $\psi_0\geq0$;
			\item $\R_{>0}\supset\supp\psi_0$;
			\item $\int_{\R^1}\psi_0=1$
		\end{enumerate}
		Then, we shall define $g$ (the definition will be the same for both cases of $D$) via
		\[(g,\psi):=(F,\psi_0(x_1)\psi(\cdot))\]
		(note, that as $\supp\mybra{\psi_0(x_1)g(\cdot)}\subset\supp\psi_0\times\supp\psi$, we have $0\notin\supp\mybra{\psi_0(x_1)g(\cdot)}$).
		Thus it remains to show that
		\begin{equation}(F,\phi):=(F,\psi(x_1)\int_{-\infty}^\infty\phi(t,\tilde{x})\;dt)\quad\forall\phi\in C^\infty_0(D)\label{LemEq}\end{equation}

		Let us tackle the case $D=\R^N$ first. Note that
		\[\forall\tilde{x}\in\R^{N-1},\quad\int_{-\infty}^\infty\mycbra{\phi(x_1,\tilde{x})-\psi_0(x_1)\int_{-\infty}^\infty\phi(t,\tilde{x})\;dt}\;
		dx_1=\]
		\[\int_{-\infty}^\infty\phi(x_1,\tilde{x})\;dx_1-\int_{-\infty}^\infty\psi_0(x_1)\;dx_1\int_{-\infty}^\infty\phi(t,\tilde{x})\;dt=0\]
		Hence, if we define $\tilde{\phi}(x_1,\tilde{x}):={\phi(x_1,\tilde{x})-\psi_0(x_1)\int_{-\infty}^\infty\phi(t,\tilde{x})\;dt}$
		we would have $\int_\R\tilde{\phi}(x_1,\tilde{x})\;dx_1=0\quad\forall\tilde{x}\in\R^{N-1}$, hence
		$\overline{\phi}(x_1,\tilde{x}):=\int_{-\infty}^{x_1}\tilde{\phi}(t,\tilde{x})\;dt$ is $C^\infty_0(\R^N)$, and thus
		\[(F,\tilde{\phi})=(F,\partial/\partial x_1\overline{\phi})=-(\partial/\partial x_1 F,\overline{\phi})=0\]
		thus ending the proof for the case $D=\R^N$.

		Now, let us finally consider $D=\R^N\setminus\mycbra{0}$ case. Again, we seek to show that
		\[(F,\phi):=(F,\psi(x_1)\int_{-\infty}^\infty\phi(t,\tilde{x})\;dt)\quad\forall\phi\in C^\infty_0(D)\]
		(this time with additional assumption $0\notin\supp\phi$. As both sides of equation \ref{LemEq} are additive,
		we might use partition of unity with sets $U_i^b:=\mysetn{x\in\R^N}{b\cdot x_i >0}$, where $1\leq i\leq n$ and $b=\pm1$,
		so we may restrict ourselves to the case, where $\supp\phi\subset U_i^b$ for various $i$ and $b$. All the cases,
		except $i=1$ are handled in the same way, as in previous paragraph. To be more precise, defining the linear map on
		$C^\infty_0$ by $T(\phi):={\phi(x_1,\tilde{x})-\psi_0(x_1)\int_{-\infty}^\infty\phi(t,\tilde{x})\;dt}$, we see that
		for $\supp\phi\subset U_{i\neq 1}^b$ case, the corresponding $T(\phi)$ will also have support in $U_{i\neq 1}^b$
		and hence we can define $\overline{\phi}(x_1,\tx):=\int_{-\infty}^{x_1}T(\phi)(t,\tx)\;dt$, and so its support
		will also exclude zero, hence $(F,T(\phi))=0$. Hence, these cases are done.

		Now, assume $\supp\phi\subset U_{1}^-$. Using the assumption $F(-x_1,x_2,\hdots,x_n)=F(x_1,x_2,\hdots,x_n)$ (provided
		by hypothesis of the theorem) we may reflect $\phi$, without changing the value of either side of equation \ref{LemEq},
		and thus go directly to last case. Assume
		$\supp\phi\subset U_{1}^+=\mysetn{x\in\R^N}{x_1 >0}$. Then, $\supp T(\phi)\subset U_{1}^+$ as well and we can define
		$\overline{\phi}(x_1,\tx):=\int_{-\infty}^{x_1}T(\phi)(t,\tx)\;dt$ and we are done.
%%		$\tilde{\phi}(x_1,\tilde{x}):={\phi(x_1,\tilde{x})-\psi_0(x_1)\int_{-\infty}^\infty\phi(t,\tilde{x})\;dt}$
%%		(it will have $\int_\R\tilde{\phi}(x_1,\tilde{x})\;dx_1=0\quad\forall\tilde{x}\in\R^{N-1}$ and we will try to show
%%		$0\notin\supp\tilde{\phi}$) and 
%%		$\overline{\phi}(x_1,\tilde{x}):=\int_{-\infty}^{x_1}\tilde{\phi}(t,\tilde{x})\;dt$.
%%		The problem, however, is that it might happen that $0\in\supp\overline{\phi}$, hence $(F,\overline{\phi})$ would be undefined.
%%		We will try to fix this by adding something from $\Ker F$ to $\tilde{\phi}$.
%%
%%		Take $\epsilon,A>0$, so that $\supp\tilde{\phi}\subset\mysetn{(x_1,\tilde{x})\in\R^N}{\epsilon\leq\myabs{x_1},\mynorm{\tilde{x}}\leq A}$.
%%		Now, we define $\Phi:\mysetn{\tilde{x}\in\R^{N-1}}{\mynorm{\tx}\leq\epsilon}\ni\tx\mapsto\int_{-A}^{-\epsilon}\tp(x_1,\tx)\;dx_1=
%%		-\int_{\epsilon}^{A}\tp(x_1,\tx)\;dx_1$. 
%%		Note, furthermore, that as $F$ is assumed to be even by hypothesis, it should vanish on odd functions, and so without loss
%%		of generality we might assume that $\tp$ is even, thus forcing $\Phi$ to be odd. Now, take function $\psi_1\in C^\infty_0(\R^{N-1})$,
%%		such that
%%		\begin{enumerate}
%%			\item $\supp\psi_1\subset\mysetn{\tx\in\R^{N-1}}{\mynorm{\tx}<\epsilon}$;
%%			\item $\psi_1$: odd;
%%			\item $\psi_1\bigg|_{\mysetn{\tx}{\mynorm{\tx}\leq\epsilon/2}}=\Phi\bigg|_{\mysetn{\tx}{\mynorm{\tx}\leq\epsilon/2}}$
%%		\end{enumerate}
%%		and $\psi_2\in C^\infty_0(\R)$, such that
%%		\begin{enumerate}
%%			\item $\supp\psi_2\subset(-A,-\epsilon)$;
%%			\item $\psi_2\geq0$;
%%			\item $\int_{\R}\psi_2=1$.
%%		\end{enumerate}
%%		Then, consider the function $\tp(x_1,\tx)-\psi_2(x_1)\psi_1(\tx)+\psi_2(-x_1)\psi_1(-\tx)$. For $\mynorm{\tx}\leq\epsilon/2$, we have
%%		\[\int_{-A}^{-\epsilon}\tp(x_1,\tx)-\psi_2(x_1)\psi_1(\tx)+\psi_2(-x_1)\psi_1(-\tx)\;dx_1=\Phi(\tx)-\psi_1(\tx)=0\]
%%		and
%%		\[\int_{\epsilon}^{A}\tp(x_1,\tx)-\psi_2(x_1)\psi_1(\tx)+\psi_2(-x_1)\psi_1(-\tx)\;dx_1=-\Phi(\tx)+\psi_1(-\tx)=0\]
		%To be more precise, the sufficient condition
		%for $0\notin\supp\overline{\phi}$ is that $\tilde{\phi}
	\end{proof}
\begin{thebibliography}{9}
\bibitem{gelfand}Gelfand, I.M. and Shilov G.E., {\em Generalized functions, Volume 1:
 Properties and Operations}. Academic Press, 1964.
\end{thebibliography}
\end{document}
