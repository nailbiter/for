\documentclass[10pt]{article} % use larger type; default would be 10pt

\usepackage{mystyle}
\usepackage{enumerate}

\newcommand{\sgn}{\mbox{\normalfont{sgn}}}
\newcommand{\Aut}{\mbox{\normalfont{Aut}}}

\begin{document}
\begin{myprop}
	Let $(\lambda,\nu)\in\C^2$ and let $F\in D'(\R^n\setminus\mycbra{0})$, satisfying
	\[(E-(\lambda-\nu-n))F=0,\quad E:=\sum_jx_j\frac{\partial}{\partial x_j}\]
	\[((\lambda-n)x_j-x_jE+\frac{1}{2}\myabs{x}^2\frac{\partial}{\partial x_j})F=0,\quad\forall1\leq j<n\]
	\[F(-x)=F(x)\]
	\[F(mx,x_n)=F(x,x_n),\quad \forall m\in O(n-1)\]
	Then if $n\geq2$,
	\[F=\begin{cases}
		\C\myabs{x_n}^{\lambda+\nu-n}\myabs{x}^{-2\nu},&\lambda+\nu-n\neq-1,-3,-5,\hdots\\
		\C\delta^{-\lambda-\nu+n-1}(x_n)\myabs{x}^{-2\nu},&\lambda+\nu-n=-1,-3,-5,\hdots
	\end{cases}\]
	\end{myprop}
	\begin{proof}
		To begin with, substitute first equation into the second to get
		\[(\myabs{x}^2\frac{\partial}{\partial x_j}+2\nu x_j)F=0,\quad\forall1\leq j<n\]
		and consequently
		\[\frac{\partial}{\partial x_j}\mybra{\myabs{x}^{2\nu} F}=0,\quad\forall1\leq j<n\]
		Note, that the multiplication here, as well as all the multiplications made so far make perfect sense on $\R^n\setminus
		\mycbra{0}$, as $\myabs{x}^{2\nu}$ is locally integrable on this domain for any $\nu\in\C$.
		Now, we introduce the lemma, which will be proven below.
		\begin{mylem}
			Let $D=\R^n\setminus\mycbra{0}$ or $D=\R^n,\;n\geq2$,
			$F\in D'(D)$ and $\partial/\partial x_1F=0$ on $D$. Then,
			\begin{enumerate}
				\item $F=g$ for some $g\in D'(\R^{n-1})$, that is
			\[(F,\phi)=\mybra{g,\tilde{x}\mapsto\int_\R\phi(x_1,\tilde{x})\;dx_1},\quad\forall \phi\in C^\infty_0(D)\]
				\item Moreover, if for any first order differential operator $P$, $PF=0$ on $D$, then $Pg=0$ on $\R^{n-1}$.
			\end{enumerate}
		\end{mylem}
		Using this lemma, we see that there is $g\in D'(\R)$, such that 
		\[\forall \phi\in C^\infty_0(\R^n\setminus\mycbra{0}),\quad (F\myabs{x}^{2\nu}
		,\phi)=\mybra{g,x_n\mapsto\int_{\R^{n-1}}\phi(\tilde{x},x_n)\;d\tilde{x}} \]
		In turn, the fourth equation force $g$ to be even and the first translates to ($a:=\lambda+\nu-n$)
		$xg'-ag=0$, which holds on $\R$, thanks to the item 2) of Lemma.

		Now, it is sufficient to show that
		\[g(x)=\begin{cases}
		\C\myabs{x}^a,&a\neq-1,-3,-5,\hdots\\
		\C\delta^{-a-1}(x),&a=-1,-3,-5,\hdots
		\end{cases}\]
%%		This in turn, is sufficient to show only for big $\Re a$, as the rest can be deduced via the analytic continuation and
%%		the fact that $xg'=ag\implies x(g')'=(a-1)g'$.
		First, we consider the restriction to of $g$ to $\R_{>0}$. Suppose, we'd be able to show that it will be equal to
		$Cx^a$. Then, by the assumed evenness of $g$ we'll get that the restriction of $g$ equal to $C\myabs{x}^a$ on $\R^\times$.
	\end{proof}
%%\begin{thebibliography}{9}
%%%\bibitem{gelbaum}Gelbaum, B.R. and Olmsted, J.M.H.. Counterexamples in Analysis. Dover Publications. 2003
%%\end{thebibliography}
\end{document}
