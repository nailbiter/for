%make suugakkai_16_aux/Makefile
\documentclass[8pt,notes,notheorems]{beamer}
\mode<presentation>{\usetheme[secheader]{Boadilla}}
\usepackage{mystyle}
\usepackage{xcolor}
\usepackage{graphicx}
\usepackage{pdflscape}
\usepackage{prerex,mystyle}
\usepackage[default,scale=0.92]{opensans}
\usepackage{dot2texi}
\usepackage{tikz}
\usetikzlibrary{shapes,arrows}
\renewcommand{\seriesdefault}{sb}
\renewcommand{\ttfamily}{\fontfamily{cmtt}\fontseries{m}\selectfont}
\usepackage{geometry}
\usepackage{amsmath}
\usepackage{ruby}
\usepackage{enumerate}
\usepackage{setspace}
\usepackage{xypic}
\usepackage[all,cmtip]{xy}
\usepackage{bbm,ulem,float,mystyle}
\usepackage{caption}
\usepackage{subcaption}
\usepackage{setspace}
\usepackage{tikz-cd,array}
\usepackage{catchfilebetweentags}
\usetikzlibrary{patterns}
\usepgflibrary{arrows}

\newcommand{\red}[1]{{\color[rgb]{0.6,0,0}#1}}
\newcommand{\Sol}{\mathcal{S}\mbox{ol}}
\newcommand{\Ind}{\mbox{\normalfont Ind}}
\newcommand{\Hom}{\mbox{\normalfont Hom}}
\newcommand{\D}{\mathcal{D}}
\newcommand{\A}{\mathcal{A}}
\newcommand{\Co}{\mathbb{C}}
\newcommand{\X}{\mathbb{X}}
\renewcommand{\setminus}{\backslash}
\newcommand{\nin}{\not\in}
\newcommand{\tmop}[1]{\ensuremath{\operatorname{#1}}}
\newcommand{\tmtextbf}[1]{{\bfseries{#1}}}
\newcommand{\tmtextit}[1]{{\itshape{#1}}}
\newcommand{\mss}{//}
\newcommand{\mbb}{\backslash\backslash}
\newcommand{\mmm}{\mid\mid}
\catcode`\<=\active \def<{
\fontencoding{T1}\selectfont\symbol{60}\fontencoding{\encodingdefault}}
\catcode`\>=\active \def>{
\fontencoding{T1}\selectfont\symbol{62}\fontencoding{\encodingdefault}}
\newcommand{\assign}{:=}
\newcommand{\comma}{{,}}
\newcommand{\um}{-}
\newcommand{\sol}{\mathcal{S}ol(\R^{p,q};\lambda,\nu)}
\newcommand{\Op}{\mbox{\normalfont Op}}
\newcommand{\Res}{\operatorname{Res}\displaylimits}
\newcommand{\OpR}{\mbox{\it R}}

\setbeamertemplate{theorem}[ams style]
\setbeamertemplate{theorems}[numbered]
%\setbeamertemplate{footline}[page number]

\makeatletter
    \ifbeamer@countsect
      \newtheorem{theorem}{\translate{Theorem}}[section]
    \else
      \newtheorem{theorem}{\translate{Theorem}}
    \fi
    \newtheorem{corollary}{\translate{Corollary}}
    \newtheorem{fact}{\translate{Fact}}
    \newtheorem{lemma}{\translate{Lemma}}
    \newtheorem{problem}{\translate{Problem}}
    \newtheorem{solution}{\translate{Solution}}

    \theoremstyle{definition}
    \newtheorem{definition}{\translate{Definition}}
    \newtheorem{definitions}{\translate{Definitions}}

    \theoremstyle{example}
    \newtheorem{example}{\translate{Example}}
    \newtheorem{examples}{\translate{Examples}}


    % Compatibility
    \newtheorem{Beispiel}{Beispiel}
    \newtheorem{Beispiele}{Beispiele}
    \theoremstyle{plain}
    \newtheorem{Loesung}{L\"osung}
    \newtheorem{Satz}{Satz}
    \newtheorem{Folgerung}{Folgerung}
    \newtheorem{Fakt}{Fakt}
    \newenvironment{Beweis}{\begin{proof}[Beweis.]}{\end{proof}}
    \newenvironment{Lemma}{\begin{lemma}}{\end{lemma}}
    \newenvironment{Proof}{\begin{proof}}{\end{proof}}
    \newenvironment{Theorem}{\begin{theorem}}{\end{theorem}}
    \newenvironment{Problem}{\begin{problem}}{\end{problem}}
    \newenvironment{Corollary}{\begin{corollary}}{\end{corollary}}
    \newenvironment{Example}{\begin{example}}{\end{example}}
    \newenvironment{Examples}{\begin{examples}}{\end{examples}}
    \newenvironment{Definition}{\begin{definition}}{\end{definition}}
\makeatother

\hypersetup{colorlinks=true,urlcolor=blue}
\urlstyle{same}

\newenvironment{setting}{\begin{exampleblock}{Setting.}\it}{\end{exampleblock}}
\newenvironment{question}{\begin{block}{Problem.}\it}{\end{block}}
\newenvironment{prop}[1][]{\begin{block}{Proposition#1.}\it}{\end{block}}
\newenvironment{prop*}{\begin{block}{Proposition.}\it}{\end{block}}
\newenvironment{def*}{\begin{exampleblock}{Definition.}\it}{\end{exampleblock}}
\makeatletter
\def\th@mystyle{%
    \normalfont % body font
    \setbeamercolor{block title example}{bg=orange,fg=white}
    \setbeamercolor{block body example}{bg=orange!20,fg=black}
    \def\insertpropblockenv{exampleblock}
  	}
\makeatother
\theoremstyle{mystyle}
\newtheorem*{remark}{Remark.}

\title{Symmetry breaking operators of indefinite orthogonal groups $O(p,q)$}

% A subtitle is optional and this may be deleted

\author{Toshiyuki~Kobayashi\inst{1} \and \underline{Alex~Leontiev}\inst{2}}

\institute[Tokyo U] % (optional, but mostly needed)
{
  \inst{1}%
  Graduate School of Mathematical Sciences, Kavli IPMU\\
  The University of Tokyo
  \and
  \inst{2}%
  Graduate School of Mathematical Sciences\\
  The University of Tokyo
  }
% - Use the \inst command only if there are several affiliations.
% - Keep it simple, no one is interested in your street address.

  \date[MSJ Spring Meeting, 2017]{MSJ Spring Meeting, 2017\\Tokyo Metropolitan University\\Functional Analysis}
% - Either use conference name or its abbreviation.
% - Not really informative to the audience, more for people (including
%   yourself) who are reading the slides online

\subject{Representation Theory}

\begin{document}
\section{}
\begin{frame}\titlepage\end{frame}

\begin{frame}
		\newdir{:=}{{}}
	\begin{setting}
		%\xymatrixcolsep{5pc}
		\xymatrix{
			& \mathcal{L}_\lambda\mbox{ :conformally equivariant line bundle},\lambda\in\mathbb{C}
			\ar[d]\\
  		G=O(p+1,q+1)
		\ar@/^2pc/[r] &G/P\simeq (\Sp^p\times\Sp^q)/\left\{ \pm I \right\}\\
		P=MAN\ar@{:=}[u]_{\hspace{-0.25cm}\bigcup}
		\ar@/^2pc/[rd]^{{\begin{array}{c}\; \\\mbox{conformal transformations}\end{array}}}
		%\mbox\newline oeueou}\vspace{0.8cm}}
		&\\
	M_+N=O(p,q)\ltimes \mathbb{R}^{p,q}
	\ar@{:=}[u]_{\hspace{-0.25cm}\bigcup}
	\ar@/^2pc/[r]^{\mbox{isometries}}&
	\mathbb{R}^{p,q}=\left( \mathbb{R}^{p+q},ds^2=dx_1^2+\ldots+dx_p^2-dx_{p+1}^2-\ldots-dx_{p+q}^2 \right)\ar@{^{(}->}[uu]
	_{\mbox{conformal 
	compactification}}
	\vspace{2cm}
		}
	\end{setting}
	\begin{setting}
		\[\begin{array}{cccl}
				G:=O(p+1,q+1)&\curvearrowright &I(\lambda):=C^\infty(G/P,\mathcal{L}_\lambda)&(\lambda\in\mathbb{C})\\
		\bigcup&&&\\
		G':=O(p,q+1)&\curvearrowright &J(\nu):=C^\infty(G'/P',\mathcal{L}_\nu)&(\nu\in\mathbb{C})
		\end{array}\]
	\end{setting}
\end{frame}
\begin{frame}{Problem Statement}
	\begin{setting}
		\[\begin{array}{cccl}
				G:=O(p+1,q+1)&\curvearrowright &I(\lambda):=C^\infty(G/P,\mathcal{L}_\lambda)&(\lambda\in\mathbb{C})\\
		\bigcup&&&\\
		G':=O(p,q+1)&\curvearrowright &J(\nu):=C^\infty(G'/P',\mathcal{L}_\nu)&(\nu\in\mathbb{C})
		\end{array}\]
	\end{setting}
	\begin{def*}
		$A:I(\lambda)\to J(\nu)$ is SBO (symmetry breaking operator) $\iff$ $A$ is a continuous $G'$-homomorphism.
	\end{def*}
	\begin{question}
\begin{enumerate}
	\item Construct and classify all SBOs from $I(\lambda)$ to $J(\nu)$.
\item Find the properties of elements of $\Hom_{G'}(I(\lambda),J(\nu))$ (e.g. their images, functional equations).
\end{enumerate}
	\end{question}
	\begin{block}{The first results $\cdots$ Kobayashi and Speh [Memoirs of AMS, 2015]}
		The complete answer for these questions was obtained in the case $q=0$ (that is, $(G,G')=(O(n+1,1),O(n,1))$).
	\end{block}
\end{frame}
\begin{frame}{Strategy (Kobayashi; Kobayashi-Speh)}
	\begin{setting}
	\centerline{
		\xymatrixcolsep{0.5pc}
		\xymatrixrowsep{1pc}
		\xymatrix{G\ar@/_1pc/[d]&&\supset&&G'\ar@/^1pc/[d]\\
			I(\lambda)=\Ind_P^G(\lambda)&\ar[rr]^{A}&&&\Ind_{P'}^{G'}(\nu)=J(\nu)
	}}
	\end{setting}
	\begin{question}
		Classify all SBOs $A:I(\lambda)\to J(\nu)$.
	\end{question}
	\begin{exampleblock}{Strategy.}
		\begin{enumerate}[(1)]
			\item (``well-posed'' case)\\
				Take $(G,G',P,P')$ such that $\#\left( P'\backslash G/P \right)<\infty$ (cf. classification theory $\cdots$ Kobayashi-Matsuki 2014).
			\item (construction)\\
				Construct a meromorphic family of SBOs, $R_{\lambda,\nu}^S:I(\lambda)\to J(\nu)$ associated to each closed $P'$-invariant set $S$ in $G/P$.
			\item (classification)\\
				Classify SBOs inductively from $S=$ singleton. ($S=$ singleton $\Leftrightarrow R_{\lambda,\nu}^S$: differential SBO $\Leftarrow$ F-method).
		\end{enumerate}
	\end{exampleblock}
\end{frame}
\begin{frame}
	\begin{fact}[\cite{kobayashi2015symmetry}]
    Let $n:=p+q$. The following diagram commutes:
\begin{figure}[H]
	\centerline{
		\xymatrixcolsep{5pc}
		\xymatrix{\Hom_{G'}(I(\lambda),J(\nu))\ar[r]^{\simeq} \ar@/^2pc/[rr]^{\mathcal{S}}
		&\left( \mathcal{D}'(G/P,\mathcal{L}_{n-\lambda}) \otimes\mathbb{C}_\nu \right)^{P'}
	\ar[r]_-{F\mapsto \supp(F)}\ar[d]^{\simeq}_{\mbox{rest}}
	&2^{P'\backslash G/P}\\
	&\sol\subset\mathcal{D}'(\R^{p,q})\ar[lu]^{\mbox{Op}}_{\simeq}&
	}
}
\end{figure}
\end{fact}
\end{frame}
\begin{frame}
Note that $G$ acts on $\Xi^{p+1,q+1}:=\mysetn{(x,y)\in\R^{p+1,q+1}\setminus\left\{ 0 \right\}}{\myabs{x}^2=\myabs{y}^2}$ and on its quotient space
$X^{p,q}:=\Xi^{p+1,q+1}/\R^{\times}\simeq G/P$. Let
\[
	X:=G/P\simeq X^{p,q},\quad Y:=\mysetn{[\xi:\eta]\in G/P\simeq X^{p,q}}{\xi_{p}=0}\simeq X^{p-1,q}\]
	\[C:=\mysetn{[\xi:\eta]\in G/P\simeq X^{p,q}}{\xi_{0}=\eta_q}\simeq X^{p-1,q-1}\cup\Xi^{p,q},\quad\left\{ [0] \right\}:=\left\{ [1,0_{p+q},1] \right\}\]
\begin{theorem}[classification of closed $P'$-invariant subsets of $G/P$]
	The left $P'$-invariant closed subspaces of $G/P$ are as follows (numbers indicate codimension):\\
  \begin{figure}[H]
    \centering
    \begin{subfigure}[t]{0.3\textwidth}
	    \xymatrixrowsep{0.5pc}
	    \xymatrix{&X\ar@{-}[ld]_1\ar@{-}[rd]^1&\\Y\ar@{-}[rd]_1&&C\ar@{-}[ld]^1\\&Y\cap C\ar@{-}[dd]^{p+q-2}&\\&&\\&\{[0]\}&}
	\caption{when $p>1$}
    \end{subfigure}
    ~ %add desired spacing between images, e. g. ~, \quad, \qquad, \hfill etc. 
      %(or a blank line to force the subfigure onto a new line)
    \begin{subfigure}[t]{0.3\textwidth}
	    \xymatrixrowsep{0.5pc}
	    {\xymatrix{&X\ar@{-}[ld]_1\ar@{-}[rd]^1&\\Y\ar@{-}[rddd]_{p+q-2}&&C\ar@{-}[lddd]^{p+q-2}\\&&\\&&\\&\{[0]\}&}}
	\caption{when $p=1$}
    \end{subfigure}
\end{figure}
\end{theorem}
\end{frame}
\begin{frame}
\begin{theorem}[Construction of Symmetry Breaking Operators]
		We can construct the following families of SBOs which holomorphically depend on parameters:
\renewcommand{\arraystretch}{3.5}
\begin{center}
\begin{tabular}{|c|c|c|c|}
  \hline
  & $\tmop{Op}:\Sol(\mathbb{R}^{p,q};\lambda,\nu)\to\Hom_{G'}(I(\lambda),J(\nu))$ & defined for &
  $\mathcal{S} (\cdot)$ (generically)\\
  \hline
  $R_{\lambda, \nu}^X=$ & $\Op\left( \frac{| x_p |^{\lambda + \nu - n} | Q |^{-
  \nu}}{\Gamma \left( \frac{\lambda - \nu}{2} \right) \Gamma \left(
  \frac{\lambda + \nu - n + 1}{2} \right) \Gamma \left( \frac{1 - \nu}{2}
  \right)} \right)$ & $(\lambda,\nu)\in\mathbbm{C}^2$ & $X$\\
  \hline
  $\tilde{R}^X_{\lambda, \nu}=$ & $\Op\left(  \frac{| x_p |^{\lambda + \nu - n} | Q |^{-
  \nu}}{ \Gamma \left(
  \frac{\lambda + \nu - n + 1}{2} \right) \Gamma \left( \frac{1 - \nu}{2}
  \right)} \right)$ & $(\lambda,\nu)\in\mid \mid \mid$ & $X$\\
  \hline
  $R_{\lambda, \nu}^{\{ 0 \}}=$ & $\Op\left(  \tilde{C}_{\nu - \lambda}^{\lambda - \frac{n
  - 1}{2}} ({\Delta}_{\mathbb{R}^{p-1,q}} {\delta}_{\mathbb{R}^{p+q-1}}, \delta (x_p)) \right)$ & $(\lambda,\nu)\in/ /$ & $\{ [0]
  \}$\\
  \hline
\end{tabular}
\end{center}

\begin{itemize}
	\item $\mid \mid \mid \assign \{ (\lambda, \nu) \in \mathbbm{C}^2 \mid \nu \in
	- 2\mathbbm{Z}_{\geqslant 0} \cup (q + 1 + 2\mathbbm{Z}) \}$ \item $/ / \assign
\{ (\lambda, \nu) \in \mathbbm{C}^2 \mid \lambda - \nu \in
2\mathbbm{Z}_{\leqslant 0} \}$;
\item $Q:=x_1^2+\cdots+x_p^2-x_{p+1}^2-\cdots-x_{p+q}^2$;
\item $\tilde{C}(s,t)$ is a two-variable inflation of renormalized Gegenbauer polynomial, defined
	as in \cite{Kobayashi2014272}
\end{itemize}
	\end{theorem}
\end{frame}
\begin{frame}{Complete classification of Symmetry Breaking Operators}
	For simplicity, we treat the case $p>1$.
	\begin{theorem}
		We can find basis for $\Hom_{G'}(I(\lambda),J(\nu))$ for every $(\lambda,\nu)\in \mathbb{C}^2$. In particular, for $p>1$ we have
  \begin{eqnarray}
	  & \Hom_{G'}(I(\lambda),J(\nu))= \left\{
    \begin{array}{ll}
	    \mathbbm{C} {\tilde{\OpR}}_{\lambda, \nu}^{X} \oplus \mathbbm{C}
      {\OpR}^{\{ 0 \}}_{\lambda, \nu}, & (\lambda, \nu) \in / /\cap 
      \mid\mid\mid \\
      \mathbbm{C} \OpR^X_{\lambda, \nu}, &
      \mbox{\normalfont otherwise.}
    \end{array} \right. &  \nonumber
  \end{eqnarray}
  Note that $//\cap\mid\mid\mid\subset\mathbb{C}^2$ is a countable discrete subset.
		\label{}
	\end{theorem}
	\begin{corollary}
  \begin{eqnarray}
	  & \dim\Hom_{G'}(I(\lambda),J(\nu))= \left\{
    \begin{array}{ll}
	    2, & (\lambda, \nu) \in / /\cap 
      \mid\mid\mid \\
      1, &
      \mbox{\normalfont otherwise.}
    \end{array} \right. &  \nonumber
  \end{eqnarray}
	\end{corollary}
\end{frame}
\begin{frame}{Residue theorem}
	\begin{theorem}
		The distribution
		\[K_{\lambda,\nu}^{\mathbb{R}^{p,q}}:=\frac{\myabs{x_p}^{\lambda+\nu-n}}{\Gamma\left( \frac{\lambda+\nu-n+1}{2} \right)}\times
		\frac{\myabs{Q}^{-\nu}}{\Gamma\left( \frac{1-\nu}{2} \right)}\]
		has the pole at $(\lambda,\nu)\in//$ with the residue given by
		\[\Res_{(\lambda,\nu)\in//}K_{\lambda,\nu}^{\mathbb{R}^{p,q}}=\frac{K_{\lambda,\nu}^{\mathbb{R}^{p,q}}}{\Gamma\left( \frac{\lambda-\nu}{2} \right)}
			=\frac{ (- 1)^k k!\pi^{(n - 2) / 2} 
		}{2^{ \nu + 2 k-1}}\cdot  \frac{\sin\left( \frac{1+q-\nu}{2}\pi \right)}{\Gamma\left( \frac{\nu}{2} \right)}
	\tilde{C}_{\nu - \lambda}^{\lambda - \frac{n
  	- 1}{2}} ({\Delta}_{\mathbb{R}^{p-1,q}} {\delta}_{\mathbb{R}^{p+q-1}}, \delta (x_p))
		\]
		where $k:=\frac{\nu-\lambda}{2}$.
		Hence taking $\Op(\cdot)$ on both sides we get
  \[\OpR_{\lambda,\nu}^X  = \frac{ (- 1)^k k!\pi^{(n - 2) / 2} 
		}{2^{ \nu + 2 k-1}}\cdot  \frac{\sin\left( \frac{1+q-\nu}{2}\pi \right)}{\Gamma\left( \frac{\nu}{2} \right)}
     \OpR_{\lambda,\nu}^{ \left\{ 0 \right\} },\quad(\lambda,\nu)\in// . \]
	\end{theorem}
\end{frame}
\addtocounter{framenumber}{-1}
\begin{frame}{Functional identities}
	\begin{fact}[Knapp-Stein operator]
		There is a nontrivial $G$-invariant operator $\tilde{\mathbb{T}}_{\lambda}:I(\lambda)\to I(n-\lambda)$ which holomorphically depends on $\lambda\in \mathbb{C}$.
	\end{fact}
	\begin{theorem}
		Let $n':=n-1$. Then the following holds:
\begin{eqnarray}
	& \tilde{\mathbbm{T}}_{n' - \nu} \circ \OpR^X_{\lambda,n'-\nu}= q^{T\mathbbm{R}^n}_{\mathbbm{R}^n} (\lambda,
  \nu) \OpR^X_{\lambda,\nu} &  \nonumber\\
  & \OpR^X_{n-\lambda,\nu} \circ
  \tilde{\mathbbm{T}}_{\lambda} = q^{\mathbbm{R}^n T}_{\mathbbm{R}^n}
  (\lambda, \nu) \OpR^X_{\lambda,\nu} & \nonumber
\end{eqnarray}

\setbeamercovered{transparent}
\pause
\addtocounter{framenumber}{1}
Here, for example
\begin{eqnarray}
  & q^{\mathbbm{R}^n T}_{\mathbbm{R}^n} (\lambda, \nu) \assign \frac{2^{2
  \lambda - n} \pi^{- n / 2}}{\Gamma \left( \frac{n - \lambda}{2} \right)}
  \times \frac{\sin \left[ \pi \frac{p - \lambda + 1}{2} \right]}{\pi} \times
  \left\{ \begin{array}{ll}
    2^{1 - \lambda} \sqrt{\pi}, & n \in 2\mathbbm{Z}+ 1\\
    \Gamma \left( \frac{\lambda - n / 2 + 1}{2} \right), & n / 2 + p \in
    2\mathbbm{Z}\\
    \Gamma \left( \frac{\lambda - n / 2}{2} \right), & n / 2 + p \in
    2\mathbbm{Z}+ 1.
  \end{array} \right. &  \nonumber
\end{eqnarray}
	\end{theorem}
\end{frame}
%%\begin{frame}{How this is related to integral computation}
%%	Critical step in proving of Theorems 4--7 was the explicit computation of ``eigenvalues'' $C_{a',a,b}^{p,q}(\lambda,\nu)$.\footnote[frame]{I remember, You said the diagram is 
%%	too technical and better to be removed. I will do so in a subsequent versions, once I get it done. Nevertheless, I feel that it might be reused, so I'll type it in Tikz just in case. Any comments would be
%%appreciated.}\\
%%\begin{center}
%%	\begin{tikzpicture}
%%	\draw[color=black] (-9.0,0.75) rectangle (-6.0,-6.05);%upper
\draw[color=black] (-8.75,0.25) rectangle (-6.25,-1.25);%right in upper
\draw[color=black] (-8.5,0.0) rectangle (-6.5,-0.5);%in ``right in upper''
\draw[color=black] (-8.75,-1.75) rectangle (-6.25,-3.75);%middle in upper
\draw[color=black] (-8.5,-1.825) rectangle (-6.5,-2.325);%right in ``middle in upper''
\draw[color=black] (-8.5,-2.4) rectangle (-6.5,-2.9);%middle in ``middle in upper''
\draw[color=black] (-8.5,-2.975) rectangle (-6.5,-3.475);%left in ``middle in upper''
\draw[color=black] (-8.75,-4.5) rectangle (-6.25,-6.0);%left in upper
\draw[color=black] (-8.5,-4.85) rectangle (-6.5,-5.35);%right in ``left in upper''
\draw[color=black] (-8.5,-5.475) rectangle (-6.5,-5.975); %left in ``left in upper''

\draw[color=black] (-5.0,0.75) rectangle (-2.0,-6.05);%lower
\draw[color=black] (-4.75,0.25) rectangle (-2.25,-1.25);%right in lower
\draw[color=black] (-4.75,-4.5) rectangle (-2.25,-6.0);%left in lower

\node at (-7.25,-4.0) {\color{black}{\Huge \dots}};
\node at (-7.25,-4.7) {\color{black}{\Huge \dots}};
\node at (-2.25,-3.0) {\color{black}{\Huge \dots}};

%%\node at (-7.0,1.0) {\color{black}{$I(\lambda)$}};
%%\node at (-2.0,1.0) {\color{black}{$J(\nu)$}};

\draw[-open triangle 90] (-6.0,-2.65) to node {$R_{\lambda,\nu}^X$} (-5.0,-2.65);
\draw[-open triangle 90] (-5.5,-0.25) to node {$c^{p,q}_{0,0,0}I$} (-1.75,-0.5);
\draw[-open triangle 90] (-5.5,-2.075) to node {$c^{p,q}_{0,2,0}I$} (-1.75,-0.5);
\draw[-open triangle 90] (-6.5,-5.725) to node {$c^{p,q}_{m',m,n}I$} (-4.75,-5.25);

%%	\end{tikzpicture}
%%\end{center}
%%\end{frame}
\begin{frame}{Spherical multiple}
	\begin{theorem}
		Let $1_\lambda\in C^\infty(G/P,\mathcal{L}_\lambda)^K,1_\nu\in C^\infty(G'/P',\mathcal{L}_\nu)^{K'}$ be the spherical vectors. We then have
	\[ \OpR^X_{\lambda, \nu} 1_{\lambda} = 2^{1 -
     \lambda} \frac{\pi^{n / 2}}{\Gamma \left( \frac{\lambda}{2} \right)
     \Gamma \left(  \frac{\lambda + 1-q}{2} \right) \Gamma \left(
     \frac{q - \nu + 1}{2} \right)} 1_{\nu}. \]
	\end{theorem}
	\begin{remark}[previous results]
		 $p=q=1$: \cite{bernstein2004estimates},\\
		 $p$ general, $q=0$: \cite{kobayashi2015symmetry}.
	\end{remark}
	
\end{frame}
\begin{frame}{Analytic Proposition}
	Range characterization of SBO uses the following expansion formula, which we could not find in the literature.
	\begin{prop}[ 7]
$\begin{array}{c}
	| s + t |^{2 \nu} = \sum_{\ell, m = 0 \mid \ell + m : \tmop{even}}^{\infty} a_{\lambda,\mu,\nu}^{\ell,m} C_\ell^{\lambda} (s) C_m^{\mu} (t),\\
	\\
	a_{\lambda,\mu,\nu}^{\ell,m}= \frac{ 2^{-2\nu}(\lambda + \ell) (\mu + m)  \Gamma (\lambda + \mu + 2 \nu + 1) \Gamma (\lambda)
  \Gamma (\mu)\Gamma \left( 2\nu +
1 \right)}{\Gamma \left( \lambda + \nu + \frac{\ell -
  m}{2} + 1 \right)  \Gamma \left( \mu + \nu -
  \frac{\ell - m}{2} + 1 \right) \Gamma \left( \lambda + \mu + \nu + \frac{\ell +
  m}{2} + 1 \right)\Gamma(\nu+1-\frac{\ell+m}{2})} .
\end{array}$
	\end{prop}
	Here, $C_{\ell}^\lambda(t)$ ($\lambda\in\mathbb{C},\ell\in\N$) is the Gegenbauer polynomial.
\end{frame}
\begin{frame}[fragile]{Hierarchy}
\begin{tikzpicture}
%mypipes
%        File: intdep.tex
%     Created: 土  1 07 11:00 PM 2017 J
% Last Change: 土  1 07 11:00 PM 2017 J
%
\documentclass[a4paper,12pt]{article}
\usepackage{pdflscape}
\usepackage{prerex,mystyle}
\usepackage[left=4pt,right=4pt]{geometry}
\usepackage[default,scale=0.92]{opensans}
\usepackage{dot2texi}
\usepackage{tikz}
\usetikzlibrary{shapes,arrows}
\renewcommand{\seriesdefault}{sb}
\renewcommand{\ttfamily}{\fontfamily{cmtt}\fontseries{m}\selectfont}

\newtheorem{theorem}{Theorem}
\newtheorem{remark}[theorem]{Remark}
\newtheorem{fact}[theorem]{Fact}
\newtheorem{proposition}[theorem]{Proposition}
\newtheorem*{proposition*}{Proposition}
\newtheorem*{remark*}{Remark}



\begin{document}
%%\thispagestyle{empty}
%%\setcounter{diagheight}{50}
%%\begin{chart}\sf
%%	\grid
%%\text 10,50:{\Large Computer\\\Large Science}
%%\reqfullcourse 50,45:{1083}{Comput.\,Sci.\\Concepts}{TTh 10:00}
%%\reqhalfcourse 25,40:{1303}{Discrete\\Structures}{MWF 9:30}
%%\reqhalfcourse 30,30:{2813}{Computer\\Organiz.\,I}{MWF 8:30}
%%  \prereq 50,45,30,30:
%%  \prereq 25,40,30,30:
%%\reqhalfcourse 45,30:{2023}{Procedural\\Prog.\,Devel.}{MWF 2:30}
%%  \prereq 50,45,45,30:
%%\reqhalfcourse 65,30:{2513}{Informat.\\Systems}{TTh 1:00}
%%  \coreq 50,45,65,30:
%%\mini 10,26:{1083}
%%\reqhalfcourse 10,20:{2333}{Computab.\,\&\\Formal\,Lang.}{TTh 11:30}
%%  \prereq 25,40,10,20:
%%  \prereq 10,26,10,20:
%%\reqhalfcourse 45,20:{2013}{Software\\Engineer.\,I}{MWF 11:30}
%%  \prereq 45,30,45,20:
%%\halfcourse 55,20:{2685}{\texttt{C++}\\Program.}{no}
%%  \prereq 45,30,55,20:
%%\mini 21,16:{2013}
%%\reqhalfcourse 15,10:{3323}{Data\\Structures}{MWF 10:30}
%%  \prereq 25,40,15,10:
%%  \prereq 21,16,15,10:
%%\reqhalfcourse 25,10:{3813}{Comput.\\Organiz.\,II}{TTh 8:30}
%%  \prereq 30,30,25,10:
%%\reqhalfcourse 35,10:{3413}{Operating\\Systems\,I}{MWF 9:30}
%%  \prereq 30,30,35,10:
%%  \recomm 45,20,35,10:
%%\halfcourse 45,10:{3013}{Software\\Engineer.\,II}{MWF 11:30}
%%  \prereq 45,20,45,10:
%%\halfcourse 58,10:{3513}{Database\\Mngt.\,Sys.\,I}{MWF 8:30 pm}
%%  \prereq 65,30,58,10:
%%  \prereq 45,20,58,10:
%%\reqhalfcourse 70,10:{3503}{Sys.\,Anal.\\\&\,Design}{TTh 10:00}
%%  \prereq 65,30,70,10:
%%\end{chart}
%%
%%\begin{center}
%%\begin{minipage}{6.0in}
%%\begin{itemize}
%%\item
%%A solid arrow \solidarrow\  indicates a required prerequisite, 
%%a dotted arrow \dottedarrow\ 
%%indicates a corequisite (to be taken before or concurrently), and a
%%dashed arrow \dashedarrow\ indicates a recommended prerequisite.
%%Core courses are in \boldbox\ boxes; 
%%other courses (i.e.,~options or prerequisites)
%%are in \lightbox\ boxes.
%%\item Timetabling abbreviations: M, T, W, Th, F=Mon, Tue, Wed, Thur, Fri, resp.; eve=7:00--9:50 pm; no=not offered.
%%\end{itemize}
%%\end{minipage}
%%\end{center}
%%\newpage
\begin{landscape}
	\newgeometry{left=-5cm,top=4cm,bottom=0cm,right=0cm}
	\vspace*{5cm}
\begin{dot2tex}[mathmode,dot,scale=0.9]
  digraph G {
	  War10 -> S [label="k_2=0"];
	  DF85 -> S [label=" ",texlbl="$\kern-0cm\begin{array}[]{c}p=n,\\m=0\end{array}$"];
	  TV03 -> S [label=" ",texlbl="$\kern-1.5cm k_1=0$"];
	  War09 -> TV03;
	  //KL -> WS [color="blue",dir=both]
	  //Sp -> DS [color="blue",dir=both]
	  //WS -> DS [dir=both]
	  S -> Sp [label=" ", texlbl="$\begin{array}[]{c}
	  k=2,\\ \alpha=\beta
  \end{array}$"]
  KL -> Sp  [label=" ", texlbl="$\kern-8cm \begin{array}[]{l}\\\\\\\\\\\\A=B\end{array}$"]
	  War10 -> War10p [label=" ",texlbl="$\begin{array}[]{l}\alpha_1=\beta_1,\\\alpha_2=\beta_2,\\k_1=k_2\end{array}$"];
	  KL -> War10p[label=" ",texlbl="$\kern-2cm A+B=\nu+1$"];
	  TV03 -> TV03p [label=" ",texlbl="$\begin{array}[]{l}k_1=k_2,\\ \alpha_1=1,\\\alpha_2=\beta_2\end{array}$"];
	  KL -> TV03p [label=" ",texlbl="$\kern-3cm A=0$"];
	  DF85 -> DF85p [label=" ",texlbl="$\begin{array}[]{l}p=n=1,\\m=r=1,\\ \alpha=\beta\end{array}$"];
	  KL -> DF85p [label=" ",texlbl="$\kern-3cm\nu=2$"];
	  AAR -> S[label=" ",texlbl="$\begin{array}[]{l}\\\\\\\\\qquad\qquad\qquad j=k=0\end{array}$"];
	  AAR -> AARp [label=" ",texlbl="$\begin{array}[]{l}n=2,\\j=l=k=1,\\ \alpha=\beta\end{array}$"];
	  KL -> AARp [label=" ",texlbl="$\kern1cm\begin{array}[]{l}A=B+1\\\end{array}$"]

    KL [shape="box",label="{\mbox{[KL]}}:3"];
    S [shape="box",label="\mbox{[S]}:3"];
    War10 [shape="box",label="\mbox{\cite{warnaar2010sl3}}:4"];
    DF85 [shape="box",label="\mbox{\cite{dotsenko1985four}}:3"];
    TV03 [shape="box",label="\mbox{\cite{tarasov2003selberg}}:4"];
    War09 [shape="box",label="\mbox{\cite{warnaar2009selberg}}:2n+2"];
    War10p [shape="box",label="\mbox{\cite{warnaar2010sl3}'}:2"];
    //WS [shape="box",label="\mbox{[WS]}:3"]
    Sp  [shape="box",label="\mbox{[S]'}:2"]
    //DS  [shape="box",label="\mbox{[DS]}:3"]
    TV03p [shape="box",label="\mbox{\cite{tarasov2003selberg}'}:2"]
    DF85p [shape="box",label="\mbox{\cite{dotsenko1985four}'}:2"]
    AAR [shape="box",label="\mbox{\cite{andrews1999special}}:2"]
    AARp [shape="box",label="\mbox{\cite{andrews1999special}}':2"]
    }
\end{dot2tex}
\end{landscape}
\begin{landscape}
	\newgeometry{left=-3.7cm,top=4cm,bottom=0cm,right=0cm}
	\vspace*{5cm}
\begin{dot2tex}[mathmode,dot]//fdp
  digraph G {
	  {rank=same War10 DF85 TV03 KLg}
	  {rank=same KL S}
	  {rank=same TV03p War10pp Spp DF85pp}
	  War10 -> S [label="k_2=0"];
	  DF85 -> S [label=" ",texlbl="$\begin{array}[]{c}p=n,\\m=0\end{array}$"];
	  S -> Spp [label=" ", texlbl="$\begin{array}[]{c}k=2,\\\alpha=\beta\end{array}$"]
	  TV03 -> S [label=" ",texlbl="$k_1=0$"]
	  War10 -> War10pp [label=" ",texlbl="$\begin{array}[]{c}k_1=k_2=1,\\\alpha_1=\beta_1,\\\alpha_2=\beta_2\end{array}$"];
	  KL -> Spp  [label=" ", texlbl="$\begin{array}[]{c}\\\\A=B\end{array}$"]
	  KL -> War10pp[label=" ",texlbl="$\kern5emA+B=\nu+1$"];
	  TV03 -> TV03p [label=" ",texlbl="$\begin{array}[]{c}k_1=k_2,\\ \alpha_1=1,\\\alpha_2=\beta_2\end{array}$"];
	  KL -> TV03p [label=" ",texlbl="$\kern-1.5cm A=0$"];
	  DF85 -> DF85pp [label=" ",texlbl="$\begin{array}[]{c}m=n=1,\\ \alpha'=\beta',\\\alpha=\beta\end{array}$"];
	  KL -> DF85pp [label=" ",texlbl="$\nu=2$"];
	  KLg -> KL [label=" ",texlbl="$\begin{array}[]{c}z=1,\\l=m=0\end{array}$"];

    KLg [shape="box",label="{\mbox{[KLg]}}:3"];
    KL [shape="box",label="{\mbox{[KL]}}:3"];
    S [shape="box",label="\mbox{[S]}:3"];
    Spp [shape="box",label="\mbox{[S]''}:2"]
    War10 [shape="box",label="\mbox{\cite{warnaar2010sl3}}:4"];
    DF85 [shape="box",label="\mbox{\cite{dotsenko1985four}}:3"];
    TV03 [shape="box",label="\mbox{\cite{tarasov2003selberg}}:4"];
    War10pp [shape="box",label="\mbox{\cite{warnaar2010sl3}''}:2"];
    TV03p [shape="box",label="\mbox{\cite{tarasov2003selberg}'}:2"]
    DF85pp [shape="box",label="\mbox{\cite{dotsenko1985four}''}:2"]
    }
\end{dot2tex}
\end{landscape}
%%    a_1 -> a_2 -> a_3 -> a_4 -> a_1;
%%    a -> b [label="E=mc^2"];

Formulae above are as follows:\begin{equation*}
	\begin{array}[]{ll}
		\mbox{[KL]}:&\iint_{D_\pm}\myabs{u-v}^{-\nu}u^A(1-u)^Av^B(1-v)^B\;du\;dv=\cdots,\\
		\mbox{[S]}:&\int_{D_k}\Pi_{i=1}^kt_i^\alpha(1-t_i)^\beta\Delta^{2\gamma}(t)\;dt=\cdots,\\
		\mbox{\cite{warnaar2010sl3}}:&\int_{C_{\beta_1,\gamma}^{k_1,k_2}}\Pi_{i=1}^{k_1}t_i^{\alpha_1-1}(1-t_i)^{\beta_1}\Pi_{j=1}^{k_2}s_j^{\alpha_2}(1-s_j)^{\beta_2}
		\Delta^{2\gamma}(t)\Delta^{2\gamma}(s)\Delta^{-\gamma}(t,s)\;dt\;ds,\quad\beta_1+\beta_2=\gamma+1,\\
		\mbox{\cite{tarasov2003selberg}}:&\displaystyle\int_{C^{k_1,k_2}_{\gamma}}\prod_{i=1}^{k_1}t_i^{\alpha_1-1}\prod_{j=1}^{k_2}s_j^{\alpha_2-1}(1-s_j)^{\beta_2-1}\Delta^{2\gamma}(t)\Delta^{2\gamma}
		(s)\Delta^{-\gamma}(t,s)\;dt\;ds,\\
		\mbox{\cite{dotsenko1985four}}:&\displaystyle\int_{[0,1]^p}\int_{[1,\infty]^{n-p}}\int_{[0,1]^r}\int_{[1,\infty]^{m-r}}\prod_{i=1}^nt_i^\alpha(1-t_i)^\beta
		\prod_{j=1}^m\tau_j^{\alpha'}(1-\tau_j)^{\beta'}\Delta^{-2}(t,\tau)\Delta^{2\gamma}(t)\Delta^{2\gamma}(\tau)\;dt\;d\tau=\cdots,\\
		&{\alpha}/{\alpha'}={\beta}/{\beta'}=-\gamma,\quad\gamma\gamma'=1,\\
%%		\mbox{\cite{warnaar2010sl3}}':&\sin\mybra{\pi\beta_1}\iint_{D_-}u^{\alpha_1}(1-u)^{\beta_1}v^{\alpha_2}(1-v)^{\beta_2}\myabs{u-v}^{-\gamma}\;du\;dv+\\
%%		&+\sin\mybra{\pi\beta_2}\iint_{D_+}u^{\alpha_1}(1-u)^{\beta_1}v^{\alpha_2}(1-v)^{\beta_2}\myabs{u-v}^{-\gamma}\;du\;dv=\cdots,\quad \beta_1+\beta_2=\gamma+1,\\
%%		\mbox{[S]'}:&\iint_{D_-}\myabs{u-v}^{2\gamma}u^\alpha v^\alpha(1-u)^\beta(1-v)^\beta\;du\;dv,\\
		\mbox{[WS]}:&\;_3F_2\left(\begin{array}[]{c}
			a,1-a,c\\ d,2c-d+
		\end{array};1\right)=\cdots,\quad\mbox{a.k.a Whipple's sum},\\
		\mbox{[DS]}:&\;_3F_2\left(\begin{array}[]{c}
			a,b,c\\a-b+1,a-c+1
		\end{array};1\right)=\cdots,\quad\mbox{a.k.a Dixon's well-poised sum},\\
		\mbox{\cite{dotsenko1985four}'}:&\iint_D\myabs{u-v}^{-2}u^\alpha(1-u)^\beta v^{\alpha'}(1-v)^{\beta'}\;du\;dv,\quad \alpha/\beta=\alpha'/\beta',\\
		\mbox{\cite{andrews1999special}}:&\displaystyle\int_{[0,1]^n}\prod_{i=1}^jx_i\prod_{i=j+1}^{j+k}(1-x_i)\prod_{i=1}^kt_i^\alpha(1-t_i)^\beta\Delta^{2\gamma}(t)\;dt=\cdots,
	\end{array}
\end{equation*}

Above and below the following notations are used:\begin{equation*}
		\begin{array}[]{c}
			D_n:=\left\{ (t_i)_{i=1}^{n}\in\R^n\mid0\le t_1\le t_2\le\dots\le t_n\le1 \right\},\\
			D:=[0,1]^2,\\
			D_{\pm}:=\mysetn{\left( u,v \right)\in D}{(\pm)(u-v)>0},\\
			\displaystyle\Delta^\lambda\left( \left\{ x_i \right\}_{i=1}^n \right):=\prod_{1\le i<j\le n}\myabs{x_i-x_j}^{\lambda},\\
			\displaystyle\Delta^\lambda\left( \left\{ x_i \right\}_{i=1}^m,\left\{ y_j \right\}_{j=1}^n \right):=\prod_{i,j=1}^{m,n}\myabs{x_i-y_j}^\lambda.
		\end{array}
	\end{equation*}
	Moreover:\begin{itemize}
		\item the $\dots$ denotes the product ratio of product of Gamma functions;
		\item blue arrows imply the use of the proposition below;
	\end{itemize}
\begin{proposition*}
	\begin{equation*}
		\begin{array}[]{c}
			\iint_{D_-}u^{A_+}(1-u)^{A_-}v^{B_+}(1-v)^{B_-}\myabs{u-v}^{-\gamma}\;du\;dv=\dots\;_3F_2\left( \begin{array}[]{c}
				-A_-,A_++1,A_++B_+-\gamma+2\\
				B_-+A_++B_+-\gamma+3,A_+-\gamma+2
			\end{array};1\right)
		\end{array}
	\end{equation*}
\end{proposition*}
The literature read by me so far and referenced above is as follows:
\small
\nocite{*}
\bibliography{intdep}
\bibliographystyle{alpha}
\end{document}

\end{tikzpicture}
%%\begin{dot2tex}[mathmode,dot,scale=0.45]
%%  digraph G {
%%	  {rank=same War10 DF85 TV03 KLg}
%%	  {rank=same KL S}
%%	  {rank=same KLp}
%%	  {rank=same TV03p War10pp Spp DF85pp}
%%	  War10 -> S
%%	  DF85 -> S
%%	  S -> Spp 
%%	  TV03 -> S 
%%	  War10 -> War10pp 
%%	  KLp -> Spp  
%%	  KLp -> War10pp
%%	  TV03 -> TV03p 
%%	  KLp -> TV03p 
%%	  KL -> KLp [label=" ",texlbl="$\mbox{\Huge l=m=0}$"];
%%	  DF85 -> DF85pp
%%	  KLp -> DF85pp
%%	  KLg -> KL 
%%
%%    KLp [label="",shape="circle",color="red",style="filled"];
%%    KLg [shape="box",label="{\mbox{\Huge\color{red}{K-L}}}",color="red"];
%%    KL [shape="box",label="{\mbox{\Huge \color{red}{Proposition}}}",color="red"];
%%    S [shape="box",label="\mbox{\Huge[Selberg'44]}"];
%%    War10 [shape="box",label="\mbox{\Huge[Warnaar]}"];
%%    DF85 [shape="box",label="\mbox{\Huge Dotsenko-Fateev'85}"];
%%    TV03 [shape="box",label="\mbox{\Huge Tarasov-Varchenko'03}"];
%%    Spp [shape="point"];
%%    War10pp [shape="point"];
%%    TV03p [shape="point"];
%%    DF85pp [shape="point"];
%%    }
%%\end{dot2tex}
\end{frame}
\begin{frame}[allowframebreaks]{References}
	\bibliographystyle{apalike}
	\nocite{Selberg:411367}
	\nocite{warnaar2010sl3}
	\nocite{dotsenko1985four}
	\nocite{tarasov2003selberg}
	\nocite{kobayashi2015symmetry}
	\nocite{kobayashi2015differential2}
	\nocite{kobayashi2016differential1}
	\nocite{kobayashi2014classification}
	\nocite{kobayashi2013finite}
	\nocite{kobayashi2015program}
\bibliography{intdep}
\end{frame}
\end{document}
%talk is 15 minutes long
