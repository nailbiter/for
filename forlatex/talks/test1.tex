\documentclass{article}
\usepackage[english]{babel}
\usepackage{geometry,amsmath,amssymb,bbm,latexsym,theorem}
\geometry{letterpaper}

%%%%%%%%%% Start TeXmacs macros
\catcode`\>=\active \def>{
\fontencoding{T1}\selectfont\symbol{62}\fontencoding{\encodingdefault}}
\newcommand{\assign}{:=}
\newcommand{\comma}{{,}}
\newcommand{\nin}{\not\in}
\newcommand{\nocomma}{}
\newcommand{\tmop}[1]{\ensuremath{\operatorname{#1}}}
\newcommand{\tmrsub}[1]{\ensuremath{_{\textrm{#1}}}}
\newcommand{\tmrsup}[1]{\textsuperscript{#1}}
\newcommand{\tmtextbf}[1]{{\bfseries{#1}}}
\newcommand{\tmtextit}[1]{{\itshape{#1}}}
\newcommand{\tmtextup}[1]{{\upshape{#1}}}
\newcommand{\upl}{+}
\newenvironment{proof}{\noindent\textbf{Proof\ }}{\hspace*{\fill}$\Box$\medskip}
\newtheorem{lemma}{Lemma}
\newtheorem{proposition}{Proposition}
{\theorembodyfont{\rmfamily}\newtheorem{remark}{Remark}}
%%%%%%%%%% End TeXmacs macros

\newcommand{\D}{\mathcal{D}}
% 

\newcommand{\supp}{\tmop{supp}}
% 

\newcommand{\proofexplanation}[1]{(#1)}
% 

\newcommand{\C}{\mathbbm{C}}
\newcommand{\Z}{\mathbbm{Z}}
% 

\newcommand{\Sp}{\mathbbm{S}}
% 

\newcommand{\R}{\mathbbm{R}}
% 

\newcommand{\mybra}[1]{(#1)}
% 

\newcommand{\mysbra}[1]{\left[ #1 \right]}
% 

\newcommand{\mycbra}[1]{\left\{#1\right\}}
\newcommand{\sone}{\ensuremath{\mybra{\D' (G \times_P \C_{\lambda - n})
\otimes \C_{\nu}}^{\Delta (P')}}}
\newcommand{\Upp}{{\mysetn{(x,y){\in}{\R}\tmrsup{p,q}}{x{\neq}0,{\hspace{0.75em}}y{\neq}0}}}
\newcommand{\Stab}{O(p,q)\tmrsub{e\tmrsub{p}}}
\newcommand{\sol*}[1]{S ol(#1; {\lambda}, {\nu})}
\newcommand{\sol}{{\sol*{{\R}\tmrsup{p,q}}}}
\newcommand{\solXi}{{\sol*{{\Xi}}}}

\begin{document}

\

\

\

\

\

\

\

\

\

\

\

\

\

\

\

\

\section{Role of $K$-finite vectors}\label{sec:k-finite}

\subsection{Main results}

\begin{proposition}
  \label{k-finite:prop-claim2}For $K \in \mathcal{S} \tmop{ol}
  (\mathbbm{R}^{p, q} ; \lambda, \nu)$ there exists unique $K^{\Xi} \in
  \mathcal{D}'_{\lambda - n} (\Xi)^{\nu} \assign \left\{ f \in
  \mathcal{D}'_{\lambda - n} (\Xi) \middle| \forall p' \in P', \; f (p' \cdot)
  = \nu (p')^{} f (\cdot) \right\}$ such that $\psi^{\ast} K^{\Xi} = K$ (for
  $\psi : \mathfrak{n}_- \hookrightarrow \Xi$ embedding). Moreover, if
  $K_{\mu} \in \mathcal{S} \tmop{ol} (\mathbbm{R}^{p, q} ; \lambda (\mu), \nu
  (\mu))$ is holomorphic as generalized function on $\mathbbm{R}^{p, q}$
  depending on $\mu \in \Omega$ ($\Omega \subset \mathbbm{C}^n$ open) and
  $\lambda (\cdot), \nu (\cdot)$ holomorphic on $\Omega$, we have
  $(K_{\mu})^{\Xi}$ being holomorphic in $\mathcal{D}'_{\Gamma_{\Xi}} (\Xi)$.
\end{proposition}

\begin{proposition}
  \label{k-finite:prop-holo-to-holo}For $K \in \mathcal{S} \tmop{ol}
  (\mathbbm{R}^{p, q} ; \lambda, \nu)$ let $K^S \assign \iota^{\ast} K^{\Xi}$,
  where $K^{\Xi}$ is as in proposition \ref{k-finite:prop-claim2} and $\iota :
  \mathbbm{S}^p \times \mathbbm{S}^q \hookrightarrow \Xi$ is the embedding.
  Then, if $K_{\mu} \in \mathcal{S} \tmop{ol} (\mathbbm{R}^{p, q} ; \lambda
  (\mu), \nu (\mu))$ is holomorphic as distribution in $\mu \in \Omega$, with
  $\Omega \subset \mathbbm{C}^k$ open and $\lambda (\cdot) \comma \nu (\cdot)$
  holomorphic on $\Omega$, we have corresponding \ $(K_{\mu})^S \in
  \mathcal{D}' (\mathbbm{S}^p \times \mathbbm{S}^q)$ be holomomorphic also.
  Moreover, if $\lambda (\cdot), \nu (\cdot)$ and $(K_{\mu})^S$
  holomorphically extend to $\Omega' \supset \Omega$, $K_{\mu}$ also extends
  to an element of $\mathcal{S} \tmop{ol} (\mathbbm{R}^{p, q} ; \lambda (\mu),
  \nu (\mu))$ for $\mu \in \Omega'$ and we have $(K_{\mu})^S = \iota^{\ast}
  K_{\mu}^{\Xi}$ for $\mu \in \Omega'$.
\end{proposition}

\begin{proposition}
  \label{k-finite:prop-kfinite-extension-oneparam}Suppose $\Omega \subset
  \mathbbm{C}$ is open with $D \subset \Omega$ discrete. Suppose further that
  $M$ is a smooth manifold with fixed volume form and $D_{\mu} \in
  \mathcal{D}' (M)$ is holomorphic in $\mu \in \Omega \backslash D$. Suppose
  further that for every $f \in Z$ with $Z \subset C^{\infty}_0 (M)$ dense (in
  the sense of Frechet topology on $C_0^{\infty} (M)$) we have $\langle
  D_{\mu}, f \rangle$ being holomorphic in $\mu \in \Omega$. Then $D_{\mu}$
  extends to a holomorphic in $\mu \in \Omega$ generalized function on $M$. 
\end{proposition}

\begin{remark}
  Fixed volume form allows us to make sense of pairing of distribution and
  smooth function with compact support on $M$, as we \tmtextbf{do not} define
  $\mathcal{D}' (M)$ is dual of $C_0^{\infty} (M)$ -- see discussion in
  {\cite[sec 6.3]{hormander1983analysis}}.
\end{remark}

\begin{proposition}
  \label{k-finite:prop-KR-hook-1}For $(\lambda, \nu) \in \Omega_{- 1} \assign
  \{ \tmop{Re} (- \nu) > 0, \tmop{Re} (\lambda + \nu - n) > 0 \}$ we have
  $K_{\lambda, \nu} \assign | x_p |^{\lambda + \nu - n} | Q |^{- \nu} \in
  \mathcal{S} \tmop{ol} (\mathbbm{R}^{p, q} ; \lambda, \nu)$ by proposition
  \ref{supp-R:prop-regular}. Then, the corresponding $K_{\lambda, \nu}^S$
  (given by proposition \ref{k-finite:prop-holo-to-holo}) equals to $= |
  \xi_{p + 1} |^{\lambda + \nu - n} | \xi_{p + q + 2} - \xi_1 |^{- \nu}$.
  Moreover, for every $F \in \mathcal{H}^a (\mathbbm{S}^p) \otimes
  \mathcal{H}^b (\mathbbm{S}^q) \subset C^{\infty} (\mathbbm{S}^p \times
  \mathbbm{S}^q)$ with $a + b \in 2\mathbbm{Z}$ we have
  \begin{equation}
    \langle K^S_{\lambda, \nu}, F \rangle = \sum'_{N \in
    2\mathbbm{Z}_{\geqslant 0}} k_N \cdot \varphi_N [g_N]
    \label{k-finite:eq-hookR}
  \end{equation}
  where $g_N$ are even polynomials (``even'' in the sense $g (x, y) = g (- x,
  - y)$), $k_N$ are some entire nonzero functions in $(\lambda, \nu) \in
  \mathbbm{C}^2$ (depending only on $F$, $p, \; q$ and $N$), $\sum'$ denotes
  finite sum and
  \begin{eqnarray}
    & \varphi_N [g] (\lambda, \nu) \assign \left\{ \begin{array}{ll}
      \Gamma (\lambda + \nu - n + 1) / \Gamma \left( \frac{\lambda + \nu - n -
      N \upl 2}{2} \right) / \Gamma \left( \frac{\lambda + \nu + N - q}{2}
      \right), & p > 1\\
      1, & p = 1, N = 0\\
      0, & p = 1, N > 0
    \end{array} \right. \times &  \nonumber\\
    & \times \int_{[- 1, 1]^2} | x - y |^{- \nu} (1 - x^2)^{(q - 2) / 2} (1 -
    y^2)^{(\lambda + \nu + N - q) / 2 - 1} g (x, y) d x d y. &  \nonumber
  \end{eqnarray}
  Moreover, for every $N \in 2\mathbbm{Z}_{\geqslant 0}$ and even polynomial
  $g$ there exists $F \in \sum_{i, a_i + b_i \in 2\mathbbm{Z}}'
  \mathcal{H}^{a_i} (\mathbbm{S}^p) \otimes \mathcal{H}^{b_i} (\mathbbm{S}^q)$
  such that $\langle K_{\lambda, \nu}^S, F \rangle = k \cdot \varphi_N [g]$
  for some $k$ entire nonzero in $(\lambda, \nu) \in \mathbbm{C}^2$.
\end{proposition}

\begin{remark}
  Note that under the assumption $(\lambda, \nu) \in \Omega_{- 1}$, $\varphi_N
  [g]$ are holomorphic.
\end{remark}

\begin{proposition}
  \label{k-finite:prop-KR-hook-2}For $R$ meromorphic nonzero in $(\lambda,
  \nu) \in \mathbbm{C}^2$ and $K_{\lambda, \nu}^{\mathbbm{R}^n} \in
  \mathcal{S} \tmop{ol} (\mathbbm{R}^n ; \lambda, \nu)$ as defined in
  proposition \ref{supp-R:prop-3} for $(\lambda, \nu) \in \{ \lambda - \nu
  \nin -\mathbbm{Z}_{\geqslant 0} \}$ we have $K_{\lambda,
  \nu}^{\mathbbm{R}^n} / R$ extending to holomorphic in $(\lambda, \nu) \in
  \mathbbm{C}^2$ distribution if for every $N \in 2\mathbbm{Z}_{\geqslant 0}$
  and $g$:even polynomial we have $\varphi_N [g] / \Gamma \left( \frac{\lambda
  + \nu - n + 1}{2} \right) / \Gamma \left( \frac{1 - \nu}{2} \right) / R$
  extending to holomorphic in $(\lambda, \nu) \in \mathbbm{C}$. Moreover, for
  $(\lambda, \nu) \in \mathbbm{C}^2$ we have $K_{\lambda, \nu}^{\mathbbm{R}^n}
  / R = 0 \Leftrightarrow \forall N \in 2\mathbbm{Z}_{\geqslant 0} \forall g,
  \; \left( \varphi_N [g] / \Gamma \left( \frac{\lambda + \nu - n + 1}{2}
  \right) / \Gamma \left( \frac{1 - \nu}{2} \right) / R \right) (\lambda, \nu)
  = 0$.
\end{proposition}

\begin{proposition}
  \label{k-finite:prop-KC-hook-kfinite}Fix $\nu \in 2\mathbbm{Z}_{\geqslant 0}
  + 1$. Then there exists $M$ such that for $\lambda \in \Omega_M \assign \{
  \lambda \in \mathbbm{C} | \tmop{Re} (\lambda) > M \}$ such that for
  $\delta^{(\nu - 1)} (Q) \cdot | x_p |^{\lambda + \nu - n} \in \mathcal{S}
  \tmop{ol}_{\{ Q = 0 \}} (\mathbbm{R}^n ; \lambda, \nu)$ as in proposition
  \ref{supp-Q:prop-sol-extending}, corresponding $K^S_{\lambda, \nu}$ given by
  proposition \ref{k-finite:prop-holo-to-holo} and every $F \in \mathcal{H}^a
  (\mathbbm{S}^p) \otimes \mathcal{H}^b (\mathbbm{S}^q) \subset C^{\infty}
  (\mathbbm{S}^p \times \mathbbm{S}^q)$ with $a + b \in 2\mathbbm{Z}$ we have
  \begin{equation}
    \langle K^S_{\lambda, \nu}, F \rangle = \sum'_{N \in
    2\mathbbm{Z}_{\geqslant 0}} k_N \cdot \varphi_N [g_N]
    \label{k-finite:eq-hookC}
  \end{equation}
  where $g_N$ are even polynomials (``even'' in the sense $g (x, y) = g (- x,
  - y)$), $k_N$ are some entire nonzero functions in $(\lambda, \nu) \in
  \mathbbm{C}^2$ (depending only on $F$, $p, \; q$ and $N$), $\sum'$ denotes
  finite sum and
  \begin{eqnarray}
    & \varphi_N [g] (\lambda, \nu) \assign \left\{ \begin{array}{ll}
      \Gamma (\lambda + \nu - n + 1) / \Gamma \left( \frac{\lambda + \nu - n -
      N \upl 2}{2} \right) / \Gamma \left( \frac{\lambda + \nu + N - q}{2}
      \right), & p > 1\\
      1, & p = 1, N = 0\\
      0, & p = 1, N > 0
    \end{array} \right. \times &  \nonumber\\
    & \times \int_{- 1}^1 (1 - y^2)^{(\lambda + \nu + N - q) / 2 - 1}
    \frac{d^{\nu - 1}}{d x^{\nu - 1}} \middle|_{x = y} [(1 - x^2)^{(q - 2) /
    2} g (x, y)] d y. &  \nonumber
  \end{eqnarray}
  Moreover, for every $N \in 2\mathbbm{Z}_{\geqslant 0}$ and even polynomial
  $g$ there exists $F \in \sum_{i, a_i + b_i \in 2\mathbbm{Z}}'
  \mathcal{H}^{a_i} (\mathbbm{S}^p) \otimes \mathcal{H}^{b_i} (\mathbbm{S}^q)$
  such that $\langle K_{\lambda, \nu}^S, F \rangle = k \cdot \varphi_N [g]$
  for some $k$ entire nonzero in $(\lambda, \nu) \in \mathbbm{C}^2$.
\end{proposition}

\begin{proposition}
  \label{k-finite:prop-KC-hook-wrap}Fix $\nu \in 2\mathbbm{Z}_{\geqslant 0} +
  1$ and let
  \[ N_0 \assign \left\{ \begin{array}{ll}
       1, & p = 1\\
       \Gamma \left( \frac{\lambda + \nu - n + 1}{2} \right), & p > 1
     \end{array} \right. \]
  For $R$ meromorphic nonzero in $\lambda \in \mathbbm{C}$ and $K_{\lambda,
  \nu}^C \in \mathcal{S} \tmop{ol} (\mathbbm{R}^n ; \lambda, \nu)$ as defined
  in proposition \ref{supp-Q:prop-sol-extending} for $\lambda \in \{ \lambda -
  \nu \nin -\mathbbm{Z}_{\geqslant 0} \}$ we have $K_{\lambda, \nu}^C / R$
  extending to holomorphic in $\lambda \in \mathbbm{C}$ distribution if for
  every $N \in 2\mathbbm{Z}_{\geqslant 0}$ and $g$:even polynomial we have
  $\varphi_N [g] / N_0 / R$ extending to holomorphic in $\lambda \in
  \mathbbm{C}$. Moreover, for $\lambda \in \mathbbm{C}$ we have $K_{\lambda,
  \nu}^C / R = 0 \Leftrightarrow \forall N \in 2\mathbbm{Z}_{\geqslant 0}
  \forall g, \; (\varphi_N [g] / N_0 / R) (\lambda, \nu) = 0$.
\end{proposition}

\subsection{Auxiliary lemmas}

\begin{lemma}
  \label{k-finite:lem-good-cover}Suppose $\{ p_i' \}_i$ is a finite set of
  points of $P'$, such that $\{ p'_i \mathfrak{n}_- \mathbbm{R}^{\times} \}_i$
  form an open cover of $\Xi$. Then one can find $\{ \varphi_i \}_i$ partition
  of unity subordinate to $\{ p'_i \mathfrak{n}_- \mathbbm{R}^{\times} \}_i$
  that consists of homogeneous functions of order $0$.
\end{lemma}

\begin{proof}
  As projection $\pi : \Xi \twoheadrightarrow \Xi /\mathbbm{R}^{\times} \simeq
  \mathbbm{S}^p \times \mathbbm{S}^q / \{ \pm \}$ is an open map, $U_i \assign
  \pi (p_i' \mathfrak{n}_- \mathbbm{R}^{\times})$ will give us an open cover
  of $\mathbbm{S}^p \times \mathbbm{S}^q / \{ \pm \}$ and we can take
  partition of unity $\{ \psi_i \}_i$ subordinate to it. We can then define
  $\varphi_i (x) \assign \psi_i (\pi (x))$. These $\varphi_i$ will be smooth
  on $\Xi$ and homogeneous of order 0. As $\sum_i \psi_i = 1$ on $\Xi
  /\mathbbm{R}^{\times}$, we have $\sum_i \varphi_i = 1$ on $\Xi$. Finally, as
  $\varphi_i$ is supported inside $U_i$ and we have $\pi^{- 1} (U_i) = p_i'
  \mathfrak{n}_- \mathbbm{R}^{\times}$ ($\supseteq$ is by definition of
  $\pi^{- 1} (\cdot)$; $\subseteq$ because, $p_i' \mathfrak{n}_-
  \mathbbm{R}^{\times}$ is $\mathbbm{R}^{\times}$-cone) $\varphi_i$ is
  supported inside $p_i' \mathfrak{n}_- \mathbbm{R}^{\times}$ and hence $\{
  \varphi_i \}$ is the partition subordinate to $\{ p'_i \mathfrak{n}_-
  \mathbbm{R}^{\times} \}_i$.
\end{proof}

\begin{lemma}
  \label{k-finite:lem-compat-A}The following holds:
  \begin{enumerate}
    \item For $a \in A'$, left multiplication by $a$ is a diffemororphism of
    $\mathfrak{n}_- \mathbbm{R}^{\times}$ with itself;
    
    \item For $K \in \mathcal{S} \tmop{ol} (\mathbbm{R}^{p, q} ; \lambda,
    \nu)$ and $k \assign K^{\lambda - n} \in \mathcal{D}'_{\lambda - n}
    (\mathfrak{n}_- \mathbbm{R}^{\times})$ constructed as in lemma
    \ref{k-finite:lem-claim1}, we have $L_a k = \nu (a^{- 1}) k$, where $L_a k
    (\cdot) \assign k (a^{- 1} \cdot)$.
  \end{enumerate}
\end{lemma}

\begin{proof}
  The first item is clear, as one observes that for $a (t) \in A'$ as in
  $(\ref{def-n-nots:eq-A})$, we have
  \[ a (t) \psi (x) = e^t \cdot \psi (e^{- t} x) . \]
  For the second item, uniqueness part of lemma \ref{k-finite:lem-claim1}
  implies that it suffices to show that for $F \in \mathcal{D}'
  (\mathbbm{R}^{p, q})$ and $a (t) \in A'$ as above, we have
  \begin{equation}
    \psi^{\ast} (L_{a (t)} F^{\lambda - n}) (\cdot) = e^{(n - \lambda) t} F
    (e^t \cdot) \label{k-finite:eq-compat-A}
  \end{equation}
  as then $\lambda - \nu - n$-homogeneity of $K \in \mathcal{S} \tmop{ol}
  (\mathbbm{R}^{p, q} ; \lambda, \nu)$ would imply that $e^{(n - \lambda) t} F
  (e^t \cdot) = e^{- \nu} F (\cdot)$.
  
  In turn, as both sides $(\ref{k-finite:eq-compat-A})$ are continuous in $F$,
  we can assume $F \in C^{\infty}_0 (\mathbbm{R}^{p, q})$ and the statement
  then becomes clear.
\end{proof}

\begin{lemma}
  \label{k-finite:lem-compat-M}The following holds:
  \begin{enumerate}
    \item For $m \in M'$, left multiplication by $m$ is a diffemororphism of
    $\mathfrak{n}_- \mathbbm{R}^{\times}$ with itself;
    
    \item For $K \in \mathcal{S} \tmop{ol} (\mathbbm{R}^{p, q} ; \lambda,
    \nu)$ and $k \assign K^{\lambda - n} \in \mathcal{D}'_{\lambda - n}
    (\mathfrak{n}_- \mathbbm{R}^{\times})$ constructed as in lemma
    \ref{k-finite:lem-claim1}, we have $L_m k = \nu (m^{- 1}) k$, where $L_m k
    (\cdot) \assign k (m^{- 1} \cdot)$.
  \end{enumerate}
\end{lemma}

\begin{proof}
  Proof is almost identical to that of lemma \ref{k-finite:lem-compat-A}.
\end{proof}

\begin{lemma}
  \label{k-finite:lem-compat}Suppose $K_{} \in \mathcal{S} \tmop{ol}
  (\mathbbm{R}^{p, q} ; \lambda, \nu)$ and let $K^{\lambda - n} \in
  \mathcal{D}'_{\lambda - n} (\mathfrak{n}_- \mathbbm{R}^{\times})$ be as in
  lemma \ref{k-finite:lem-claim1}. Suppose further that for $U, V \subset
  \mathfrak{n}_- \mathbbm{R}^{\times}$: open $\mathbbm{R}^{\times}$-cones and
  $p' \in P'$ we have $p' U = V$. Then for $K_U \assign K^{\lambda - n}
  \middle|_U$ and similarly $K_V$ we have $K_U (p' \cdot) = \nu (p')^{} K_V
  (\cdot)$.
\end{lemma}

\begin{proof}
  Using $P' = M' A' N_+'$ decomposition and lemmas \ref{k-finite:lem-compat-A}
  and \ref{k-finite:lem-compat-M}, we may assume that $p' \in N_+'$. The
  statement then follows directly from lemma \ref{k-finite:lem-compat-N}.
\end{proof}

\begin{lemma}
  \label{k-finite:lem-compat-N}The conclusion of lemma
  \ref{k-finite:lem-compat} holds for $p' \in N_+'$.
\end{lemma}

\begin{proof}
  We let notation be as in the statement of lemma \ref{k-finite:lem-compat}.
  We also let $U_{\psi}, \; V_{\psi}$ to be the pullbacks of $U$ and $V$
  respectively by $\psi$ -- these are open subsets of $\mathbbm{R}^{p, q}$, as
  $\psi$ is an embedding. We let $p' = : n_{} (b)$ (with $n (\cdot)$ as in
  $(\ref{def-n-nots:eq-N+})$) for $b \in \mathbbm{R}^{p, q}$ with $b_p = 0$.
  Hypothesis and lemma \ref{k-finite:lem-compat-N-aux} imply that $c_b
  (\cdot)$ is non-zero on $U_{\psi}$. Then lemma
  \ref{k-finite:lem-compat-N-aux} implies that $V_{\psi} = \psi_b (U_{\psi})$,
  where $\psi_b (x) \assign (x - Q (x) b) / c_b (x)$.
  
  It suffices to show that for $f_V$ being generalized function on $V_{\psi}$
  we have $L_{n (b)} (| c_b |^{\lambda - n} \psi_b^{\ast} f_V)^{\lambda - n} =
  (f_V)^{\lambda - n}$ with $(\cdot)^{\lambda - n}$ as in lemma
  \ref{k-finite:lem-claim1}, as then the desired conclusion will be granted by
  item 4 of definition \ref{sol:def-sol} (note that $\nu (n (b)) = 1$). Due to
  the continuity of all the operations involved, we may assume that $f_V$ is
  smooth and then the desired equality can be equivalently rewritten as
  \begin{equation}
    | c_b |^{\lambda - n} \cdot (f_V \circ \psi_b^{})^{\lambda - n} = L_{n (-
    b)} (f_V)^{\lambda - n} \label{k-finite:eq-compat-N}
  \end{equation}
  (we view $c_b$ as function on $U$). Uniqueness part of lemma
  \ref{k-finite:lem-claim1} then tells us that for $x \in \psi
  (\mathfrak{n}_-)$ and $t \in \mathbbm{R}^{\times}$ we have $(f_V \circ
  \psi_b^{})^{\lambda - n} (t x) = | t |^{\lambda - n} f_V (\psi_b (x))$ and
  $(f_V)^{\lambda - n} (t x) = | t |^{\lambda - n} f_V (x)$. Therefore (as
  both side of $(\ref{k-finite:eq-compat-N})$ are homogeneous of order
  $\lambda - n$), it suffices to show that for arbitrary $x \in U$ we have $|
  c_b (x) |^{\lambda - n} f_V (\psi_b (x)) = (f_V)^{\lambda - n} (n (b)
  \psi_{} (x))$. As $n (b) \psi (x) = c_b (x) \psi (\psi_b (x))$ and
  $(f_V)^{\lambda - n}$ is homogeneous of degree $\lambda - n$ by
  construction, the $| c_b (x) |^{\lambda - n} f_V (\psi_b (x)) =
  (f_V)^{\lambda - n} (n (b) \psi_{} (x))$ holds and the proof is over.
\end{proof}

\begin{lemma}
  \label{k-finite:lem-c1}Fix $\nu \in 2\mathbbm{Z}_{\geqslant 0} + 1$. Then
  there exists $M$ such that for $\lambda \in \Omega_M \assign \{ \lambda \in
  \mathbbm{C} | \tmop{Re} (\lambda) > M \}$ such that for $\delta^{(\nu - 1)}
  (Q) \cdot | x_p |^{\lambda + \nu - n} \in \mathcal{S} \tmop{ol}_{\{ Q = 0
  \}} (\mathbbm{R}^n ; \lambda, \nu)$ as in proposition
  \ref{supp-Q:prop-sol-extending} we have corresponding $K^{\Xi}_{\lambda,
  \nu}$ given by proposition \ref{k-finite:prop-claim2} being equal to
  $\langle K_{\lambda, \nu}^{\Xi}, \varphi \rangle \assign \langle
  \delta^{(\nu - 1)} (\xi_{p + q + 2} - \xi_1) \nocomma, | \xi_{p + 1}
  |^{\lambda + \nu - n} \varphi \nocomma \rangle$.
\end{lemma}

\begin{proof}
  Till the end of the proof, we will call $\nu_{} \in 2\mathbbm{Z}_{\geqslant
  0} + 1$ in the statement by the name $\nu_0$. Now, application of lemma
  \ref{k-finite:lem-1} (and the observation that $| x |^{\lambda}$ enters
  arbitrary fixed $C^k (\mathbbm{R})$ class for $\tmop{Re} (\lambda)$ high
  enough) implies that there exists $M$ such that if $\lambda_0 \in \Omega_M$
  is fixed, then
  \[ K_{\nu} : \varphi \mapsto \left\langle \frac{| \xi_{p + q + 2} - \xi_1
     |^{- \nu}}{\Gamma ((1 - \nu) / 2)} \nocomma, | \xi_{p + 1} |^{\lambda_0 +
     \nu_0 - n} \varphi \nocomma \right\rangle \]
  is holomorphic in $\nu \in \Omega \assign \{ \tmop{Re} (\nu) > \tmop{Re}
  (\nu_0) - 1 \}$. We note that $\langle K_{\nu_0}, \varphi \rangle = \langle
  \delta^{(\nu - 1)} (\xi_{p + q + 2} - \xi_1) \nocomma, | \xi_{p + 1}
  |^{\lambda + \nu - n} \varphi \nocomma \rangle$ and thus the uniqueness part
  of proposition \ref{k-finite:prop-claim2} tells us that it suffices to show
  the following:
  \begin{enumerate}
    \item $\psi^{\ast} K_{\nu_0} = \delta^{(\nu - 1)} (Q) \cdot | x_p
    |^{\lambda + \nu - n}$;
    
    \item $K_{\nu_0} \in \mathcal{D}'_{\lambda - n} (\Xi)$;
    
    \item $\forall p' \in P', \; K_{\nu_0} (p' \cdot) = \nu_0 (p')^{}
    K_{\nu_0} (\cdot)$.
  \end{enumerate}
  Now, we see that for $\tmop{Re} (\nu) \gg 0$, $K_{\nu}$ becomes continuous.
  It is then easy to verify that for $\tmop{Re} (\nu) \gg 0$ $K_{\nu}$ is
  holomorphic of degree $\lambda_0 + \nu_0 - n - \nu_{}$. Hence, by
  holomorphic extension, $K_{\nu}$ is so for $\nu \in \Omega$ as well. This
  implies the second item of the list above.
  
  Homogeneity together with lemma \ref{k-finite:lem-holo-easy} imply also that
  $K_{\nu}$ is holomorphic in $\mathcal{D}'_{\Gamma_{\Xi}} (\Xi)$ and hence
  $\psi^{\ast} K_{\nu}$ is holomorphic as well. As for big $\nu$ $K_{\nu}$ is
  continuous, proposition \ref{holomorphicity-preserving:prop-pullback-cts}
  implies that $\psi^{\ast} K_{\nu} \simeq | Q |^{- \nu} / \Gamma ((1 - \nu) /
  2) \cdot | x_p |^{\lambda_0 + \nu_0 - n}$ and the holomorphicity of
  distributional product implies that $\psi^{\ast} K_{\nu_0}
  \middle|_{\mathbbm{R}^n \backslash \{ 0 \}} = \delta^{(\nu_0 - 1)} (Q) \cdot
  | x_p |^{\lambda_0 + \nu_0 - n}$ and then homogeneity (if necessary, we
  increase $\tmop{Re} (\lambda)$) implies that equality holds for
  $\mathbbm{R}^n$ as well. This shows the first item of the list above.
  
  Finally, for $\tmop{Re} (\nu) \gg 0$ when $K_{\nu}$ becomes continuous we
  see that (for $a (t) \in A$ and $n (v) \in N_+$ as in
  $(\ref{def-n-nots:eq-A})$ and $(\ref{def-n-nots:eq-N+})$ respectively)
  \begin{eqnarray}
    & F_{\nu} (\xi) : = | \xi_{p + 1} |^{\lambda_0 + \nu_0 - n} | \xi_{p + q
    + 2} - \xi_1 |^{- \nu} &  \nonumber\\
    & p' \in P', \; (L_{p'} f) (\cdot) \assign f ((p')^{- 1} \cdot) & 
    \nonumber\\
    & a (t) \in A' \Rightarrow (L_{a (t)} F_{\nu}) (\xi) = e^{- \nu t}
    F_{\nu} (\xi) &  \nonumber\\
    & m \assign \tmop{diag} (1, m', 1) \in M', \; m' \in O (p, q)_{e_p}
    \Rightarrow (L_m F_{\nu}) (\xi) = F_{\nu} (\xi) &  \nonumber\\
    & m \assign \tmop{diag} (- 1, 1, - 1) \in M' \Rightarrow (L_m F_{\nu})
    (\xi) = F_{\nu} (\xi) &  \nonumber\\
    & v \in \mathbbm{R}^{p, q}, \; n (v) \in N_+', \; v_p = 0 \Rightarrow
    (L_{n (v)} F_{\nu}) (\xi) = F_{\nu} (\xi) . &  \nonumber
  \end{eqnarray}
  By analytic continuation these relations extend to $\nu \in \Omega$ and this
  proves the third item of the list above.
\end{proof}

\begin{lemma}
  \label{k-finite:lem-c2}Fix $\nu \in 2\mathbbm{Z}_{\geqslant 0} + 1$. Then
  there exists $M$ such that for $\lambda \in \Omega_M \assign \{ \lambda \in
  \mathbbm{C} | \tmop{Re} (\lambda) > M \}$ such that for $\delta^{(\nu - 1)}
  (Q) \cdot | x_p |^{\lambda + \nu - n} \in \mathcal{S} \tmop{ol}_{\{ Q = 0
  \}} (\mathbbm{R}^n ; \lambda, \nu)$ as in proposition
  \ref{supp-Q:prop-sol-extending} we have corresponding $K^S_{\lambda, \nu}$
  given by proposition \ref{k-finite:prop-holo-to-holo} being equal to
  $\langle K_{\lambda, \nu}^S, \varphi \rangle \assign \langle \delta^{(\nu -
  1)} (\xi_{p + q + 2} - \xi_1) \nocomma, | \xi_{p + 1} |^{\lambda + \nu - n}
  \varphi \nocomma \rangle$.
\end{lemma}

\begin{proof}
  As in the proof of lemma \ref{k-finite:lem-c1}, we call $\nu$ in the
  statement by the name $\nu_0$ and consider $K_{\nu} \in \mathcal{D}' (\Xi)$
  holomorphic in $\nu \in \Omega$ with all notations being as in proof of
  lemma \ref{k-finite:lem-c1}. Similarly, we see that
  \[ \mathcal{D}' (\mathbbm{S}^p \times \mathbbm{S}^q) \ni k_{\nu} : \varphi
     \mapsto \left\langle \frac{| \xi_{p + q + 2} - \xi_1 |^{- \nu}}{\Gamma
     ((1 - \nu) / 2)} \nocomma, | \xi_{p + 1} |^{\lambda_0 + \nu_0 - n}
     \varphi \nocomma \right\rangle \]
  is holomorphic in $\nu \in \Omega$ and as for $\iota : \mathbbm{S}^p \times
  \mathbbm{S}^q \hookrightarrow \Xi$ embedding we have (by proposition
  \ref{holomorphicity-preserving:prop-pullback-cts}) $\iota^{\ast} K_{\nu} =
  k_{\nu}$ for $\tmop{Re} (\nu) \gg 0$ (when both $k_{\nu}$ and $K_{\nu}$
  become continuous) and thus this holds for $\nu \in \Omega$ by analytic
  continuation (note that $\iota^{\ast} K_{\nu}$ is holo in $\nu \in \Omega$,
  since $K_{\nu}$ is holo in $\mathcal{D}'_{\Gamma_{\Xi}} (\Xi)$, as shown in
  the proof of lemma \ref{k-finite:lem-c1}). Setting $\nu = \nu_0$ we get the
  desired result.
\end{proof}

\subsection{Proofs}

\begin{proof}
  (of prop. \ref{k-finite:prop-claim2}) As proposition
  \ref{def-n-nots:prop-ximodel} implies that $\Xi /\mathbbm{R}^{\times} \simeq
  G / P$ and we've shown in proposition \ref{doublePGP:prop-pnp} that $P' N_-
  P = G$, we see that (by compactness of $G / P$) one can find finitely many
  $p_i' \in P'$ such that $U_i \assign p_i' \cdot \mathfrak{n}_-
  \mathbbm{R}^{\times}$ cover $\Xi$. Without the loss of generality, we assume
  that $p'_0 = 1$, thus $U_0 =\mathfrak{n}_- \mathbbm{R}^{\times}$.
  
  This immediately implies for an element $F \in \mathcal{D}'_{\lambda - n}
  (\Xi)^{\nu}$ that if it vanishes on $\mathfrak{n}_- \mathbbm{R}^{\times}$,
  it vanishes everywhere. Together with the uniqueness part of lemma
  \ref{k-finite:lem-claim1}, this implies the uniqueness part.
  
  Next, we show the existence. For $p' \in P'$ and $f$ generalized function on
  open $\mathbbm{R}^{\times}$-cone $U \subset \Xi$, we let $(L_{p'} f) (\cdot)
  = f ((p')^{- 1} \cdot)$ be a generalized function defined on open cone $p'
  U$. Note that $L_{p'}$ preserves homogeneity and that $L_a L_b = L_{a b}$
  for $a, b \in P'$. For $p'$ we let $L_{p'}^{\ast}$ be a $C_0^{\infty} (U)
  \rightarrow C_0^{\infty} ((p')^{- 1} U)$ map defined by $\forall f \in
  \mathcal{D}' (U), \; \langle L_{p'} f, \varphi \rangle = \langle f,
  L_{p'}^{\ast} \varphi \rangle$.
  
  Given $K \in \mathcal{S} \tmop{ol} (\mathbbm{R}^{p, q} ; \lambda, \nu)$,
  lemma \ref{k-finite:lem-claim1} gives us $k \in \mathcal{D}'_{\lambda - n}
  (\mathfrak{n}_- \mathbbm{R}^{\times})$ which is pulled back to $K$ under
  $\psi$, so it remains to extend $k$ to an element of $\mathcal{D}'_{\lambda
  - n} (\Xi)^{\nu}$. We do this is follows: for $\{ \varphi_i \}_i$ partition
  of unity associated to $\{ U_i \}_i$ finite open cover of $\Xi$, we set
  $\langle K^{\Xi}, \varphi \rangle \assign \sum_i \nu (p'_i)^{} \langle
  \varphi_i L_{p'_i} k, \varphi \rangle$. It is clear that $K^{\Xi}$ is
  well-defined generalized function on $\Xi$. Moreover, as lemma
  \ref{k-finite:lem-good-cover} allows us to assume that $\varphi_i$ were
  taken homogeneous of degree 0, we have $\varphi_i L_{p_i'} k$ is homogeneous
  of degree $\lambda - n$, hence so is $K^{\Xi}$. Finally, it remains to show
  that for every $p' \in P'$ and $\varphi \in C^{\infty}_0 (\Xi)$ we have $\nu
  (p') \langle L_{p'} K^{\Xi}, \varphi \rangle = \langle K^{\Xi}, \varphi
  \rangle$ and that $\psi^{\ast} K^{\Xi} = K$.
  
  Assuming these are shown, we note that the remaining statement
  (holomorphicity) is readily given by the construction of $K^{\Xi}$. Indeed,
  $k \in \mathcal{D}'_{\lambda - n} (U_0)$ is holomorphic in
  $\mathcal{D}'_{\Gamma_{\Xi}}$ by lemma \ref{k-finite:lem-claim1}. As we note
  that left multiplication by $p' \in P'$ when seen as diffeomorphism of $\Xi$
  pulls $\Gamma_{\Xi}$ to itself, we see that $L_{p'_i} k (U_i)$ is
  holomorphic in $\mathcal{D}'_{\Gamma_{\Xi}} (\Xi)$ as well. Then, so is
  every $\varphi_i L_{p'_i} k$ and hence the whole $K^{\Xi}$.
  
  We first show that $\psi^{\ast} K^{\Xi} = K$. In the light of lemma
  \ref{k-finite:lem-claim1}, it suffices to show for this that $K^{\Xi}
  \middle|_{\mathfrak{n}_- \mathbbm{R}^{\times}} = k$. Now, for $\varphi \in
  C_0^{\infty} (U_0)$ we have $\langle K^{\Xi}, \varphi \rangle \assign \sum_i
  \nu (p_i') \langle L_{p_i'} k, \varphi_{(i)} \rangle$, where we let
  $\varphi_{(i)} \assign \varphi_i \varphi$. We note that $\varphi_{(i)}$ is
  supported inside $U_i \cap U_0$ and as we have $U_i = p_i' U_0$, we have by
  lemma \ref{k-finite:lem-compat} (applied to $(U, V) \assign (U_0 \cap
  (p_i')^{- 1} U_0, U_i \cap U_0)$) that $\nu (p_i') \langle L_{p_i'} k,
  \varphi_{(i)} \rangle = \langle k, \varphi_{(i)} \rangle$, hence $\langle
  K^{\Xi}, \varphi \rangle = \left\langle k, \sum_i \varphi_{(i)}
  \right\rangle = \langle k, \varphi \rangle$. This shows that $\psi^{\ast}
  K^{\Xi} = K$.
  
  We note that the argument of the previous paragraph in fact also shows that
  for $\varphi \in C^{\infty}_0 (U_i)$ we have $\langle K^{\Xi}, \varphi
  \rangle = \nu (p_i') \langle L_{p_i'} k, \varphi \rangle$.
  
  Finally, it remains to show that for every $p' \in P'$ and $\varphi \in
  C^{\infty}_0 (\Xi)$ we have
  \begin{equation}
    \nu (p') \langle L_{p'} K^{\Xi}, \varphi \rangle = \langle K^{\Xi},
    \varphi \rangle . \label{k-finite:eq-claim2}
  \end{equation}
  As $(\ref{k-finite:eq-claim2})$ is linear in $\varphi$, we can assume that
  $\varphi$ is supported in $U_i \cap p' U_j = p_i' U_0 \cap p' p_j' U_0$.
  Therefore the observation of previous paragraph implies that left and right
  hand sides of $(\ref{k-finite:eq-claim2})$ are equal to $\nu (p' p_j')
  \langle k, L_{p_j'}^{\ast} L_{p'}^{\ast} \varphi \rangle$ and $\nu (p_i')
  \langle k, L_{p_i'}^{\ast} \varphi \rangle$ respectively with
  $L_{p_j'}^{\ast} L_{p'}^{\ast} \varphi$ and $L_{p_i'}^{\ast} \varphi$ being
  supported in $p'' U_0 \cap U_0$ and $U_0 \cap (p_{}'')^{- 1} U_0$
  respectively for $p'' \assign (p' p_j')^{- 1} p_i'$. Therefore, applying
  lemma \ref{k-finite:lem-compat} we see that $\nu (p' p_j') \langle k,
  L_{p_j'}^{\ast} L_{p'}^{\ast} \varphi \rangle = \nu (p' p_j') \langle L_{p'}
  L_{p_j'} k,^{} \varphi \rangle = \nu (p' p_j') \nu (p'') \langle L_{p'}
  L_{p'_j} L_{p''} k, \varphi \rangle = \nu (p_i') \langle k, L_{p_i'}^{\ast}
  \varphi \rangle$. This proves $(\ref{k-finite:eq-claim2})$ and ends the
  proof.
\end{proof}

\begin{proof}
  (of prop. \ref{k-finite:prop-holo-to-holo}) We note that the restriction
  $\iota^{\ast} K^{\Xi}$ is well-defined by lemma
  \ref{k-finite:lem-restriction-to-S}. Holomorphicity of $(K_{\mu})^S$ follows
  by holomorphicity part of proposition \ref{k-finite:prop-claim2} and
  proposition \ref{holomorphicity-preserving:prop-pullback-holo}, so it
  remains to prove the part regarding the holomorphic extension. So suppose
  that $(K_{\mu})^S$ extends to $\mu \in \Omega' \supset \Omega$. Then,
  holomorphicity implies that for $\mu \in \Omega'$ $(K_{\mu})^S$ is still
  even as distribution on $\mathbbm{S}^p \times \mathbbm{S}^q$. Then, the
  converse part of lemma \ref{k-finite:lem-restriction-to-S} and lemma
  \ref{k-finite:lem-abs-is-holo} together with holomorphicity of tensor and
  pullback imply that we can construct $k_{\mu} \in \mathcal{D}'_{\lambda
  (\mu) - n} (\Xi)$ that is holomorphic in $\mu \in \Omega'$ and such that
  $\iota^{\ast} k_{\mu} = (K_{\mu})^S$. Then, $k_{\mu} = (K_{\mu})^{\Xi}$ for
  $\mu \in \Omega$ (as their restriction to $\mathbbm{S}^p \times
  \mathbbm{S}^q$ and homogeneity degree coincides, this is implied by lemma
  \ref{k-finite:lem-restriction-to-S}). Then, $\psi^{\ast} k_{\mu} \in
  \mathcal{D}' (\mathbbm{R}^{p, q})$ is holomorphic in $\mu$ (by proposition
  \ref{holomorphicity-preserving:prop-pullback-holo}) and coincides with
  $K_{\mu}$ for $\mu \in \Omega$, hence proposition \ref{sol:prop-holocont}
  implies that $\psi^{\ast} k_{\mu} \in \mathcal{S} \tmop{ol} (\mathbbm{R}^{p,
  q} ; \lambda (\mu), \nu (\mu))$ for $\mu \in \Omega'$. It remains to show
  that we have $(K_{\mu})^S = \iota^{\ast} K_{\mu}^{\Xi}$ for $\mu \in
  \Omega'$. But this is clear, as both sides are holomorphic in $\mu \in
  \Omega'$ and they coincide for $\mu \in \Omega$.
\end{proof}

\begin{proof}
  (of prop. \ref{k-finite:prop-kfinite-extension-oneparam}) Indeed, build the
  Laurent expansion of $K_{\mu}$ at arbitrary $\mu_0 \in D$: $K_{\mu} =
  \sum_{i = - \infty}^{\infty} K_i (\mu - \mu_0)^i$. The hypothesis then
  implies that for every $\varphi \in Z$ we have $\langle K_i, \varphi \rangle
  = 0$ for every $i \in -\mathbbm{Z}_{> 0}$ and hence we see that $K_i = 0$
  for $i < 0$ and thus $K_{\mu}$ is holomorphic at $\mu_0$. As $\mu_0$ was
  arbitrary, we are done.
\end{proof}

\begin{proof}
  (of prop. \ref{k-finite:prop-KR-hook-1}) As $K_{\lambda, \nu} \assign | x_p
  |^{\lambda + \nu - n} | Q |^{- \nu} \in \mathcal{S} \tmop{ol}
  (\mathbbm{R}^{p, q} ; \lambda, \nu)$ is continuous for $(\lambda, \nu) \in
  \Omega_0$, the uniqueness part of proposition \ref{k-finite:prop-claim2}
  (note that proposition \ref{holomorphicity-preserving:prop-pullback-cts}
  tells us that for continuous functions distributional pullback coincides
  with the usual one) implies that $K_{\lambda, \nu}^{\Xi} (\xi) = | \xi_{p +
  1} |^{\lambda + \nu - n} | \xi_{p + q + 2} - \xi_1 |^{- \nu}$. Then, pulled
  back to $\mathbbm{S}^p \times \mathbbm{S}^q$ (again, note proposition
  \ref{holomorphicity-preserving:prop-pullback-cts}) this becomes $| \xi_{p +
  1} |^{\lambda + \nu - n} | \xi_{p + q + 2} - \xi_1 |^{- \nu}, \; \xi \in
  \mathbbm{S}^p \times \mathbbm{S}^q$. This proves the first assertion.
  
  Before going further, we need to put an element $F \in \mathcal{H}^a
  (\mathbbm{S}^p) \otimes \mathcal{H}^b (\mathbbm{S}^q)$ into a more concrete
  form. Due to linearity, we can assume (using $(\xi, \eta)$ variable
  splitting for points of $\mathbbm{S}^p \times \mathbbm{S}^q$ with $\xi \in
  \mathbbm{S}^p$ and $\eta \in \mathbbm{S}^q$) that $F (\xi \comma \eta) = h_a
  (\xi) \cdot h_b (\eta)$. Then, using the explicit form of $\mathcal{H}^L
  (\mathbbm{S}^n) = \bigoplus_{N = 0}^L \mathcal{H}^N (\mathbbm{S}^{n - 1})$
  branching law given in {\cite[sec 4.2]{kobayashi2015symmetry}}, we see that
  we can assume (again, due to linearity) that
  \[ F (\xi, \eta) = | \xi' |^N \phi \left( \frac{\xi'}{| \xi' |} \right)
     \xi_1^{m'} | \eta' |^M \phi' \left( \frac{\eta'}{| \eta' |} \right)
     \eta_{q + 1}^{n'} \]
  with $(\phi, \phi') \in \mathcal{H}^N (\mathbbm{S}^{p - 1}) \times
  \mathcal{H}^M (\mathbbm{S}^{q - 1})$ and $(\xi_1, \xi') \assign \xi$,
  $(\eta', \eta_{q + 1}) \assign \eta$. As we should have $F$ being even, we
  see that we should have $m' + M + N + n' \in 2\mathbbm{Z}$.
  
  Next, we consider an embedding $\psi_{N \rightarrow S} : \mathfrak{n}_-
  \simeq \mathbbm{R}^{p, q} \ni x \mapsto \pi (1 - Q (x), 2 x, 1 + Q (x)) /
  \sqrt{R} \in \mathbbm{S}^p \times \mathbbm{S}^q$ (here $C^{\infty}
  (\mathbbm{R}^{p, q}) \ni R$ is given in bipolar coordinates as $R (r, s)
  \assign (1 - r^2 + s^2)^2 + 4 r^2$)). As $- \psi_{N \xrightarrow{} S}
  (\mathfrak{n}_-) \cup \psi_{N \xrightarrow{} S} (\mathfrak{n}_-) \subset
  \mathbbm{S}^p \times \mathbbm{S}^q$ is open dense (this is implied by the
  first item of lemma \ref{k-finite:lem-compat-N-aux}) and both $F \in
  \mathcal{H}^a (\mathbbm{S}^p) \otimes \mathcal{H}^b (\mathbbm{S}^q)$ and
  $K_{\lambda, \nu}^S$ are even continuous on $\mathbbm{S}^p \times
  \mathbbm{S}^q$ (the former is so due to $a + b \in 2\mathbbm{Z}$
  hypothesis), we see that $\langle K_{\lambda, \nu}^S, F
  \rangle_{\mathbbm{S}^p \times \mathbbm{S}^q} = \frac{1}{2} \langle \psi_{N
  \rightarrow S}^{\ast} K_{\lambda, \nu}^S, \psi_{N \rightarrow S}^{\ast} F
  \rangle_{\mathfrak{n}_-}$. Now, as one notes that canonical volume form on
  $\mathbbm{S}^p \times \mathbbm{S}^q$ is pulled back to a constant multiple
  of $| R (r, s) |^{- n / 2}$, we have (transferring to bipolar coordinates on
  $\mathbbm{R}^{p, q}$)
  \begin{eqnarray}
    & \langle \psi_{N \rightarrow S}^{\ast} K_{\lambda, \nu}^S, \psi_{N
    \rightarrow S}^{\ast} F \rangle_{\mathfrak{n}_-} \simeq
    \int_{\mathbbm{S}^{p - 1}} \phi (\omega) | \omega_p |^{\lambda + \nu - n}
    d \omega_{} \times \int_{\mathbbm{S}^{q - 1}} \phi' (\omega') d \omega'
    \times &  \nonumber\\
    & \times \int_{r, s = 0}^{\infty} r^{\lambda + \nu - n} | r^2 - s^2 |^{-
    \nu} R^{(n - \lambda) / 2} R^{- n / 2} r^{p - 1} s^{q - 1} \times & 
    \nonumber\\
    & \left| \frac{r}{\sqrt{R}} \right|^N \left| \frac{s}{\sqrt{R}} \right|^M
    \left( \frac{1 - r^2 + s^2}{\sqrt{R}} \right)^{m'} \left( \frac{1 + r^2 -
    s^2}{\sqrt{R}} \right)^{n'} d r d s &  \nonumber
  \end{eqnarray}
  Now, as we have $\int_{\mathbbm{S}^{q - 1}} \phi' (\omega') d \omega' = 0$
  unless $\phi' = \tmop{const}$, we see that $\langle \psi_{N \rightarrow
  S}^{\ast} K_{\lambda, \nu}^S, \psi_{N \rightarrow S}^{\ast} F
  \rangle_{\mathfrak{n}_-} = 0$ unless $M = 0$. Thus, in subsequent we assume
  $M = 0$.
  
  Furthermore, as explained in {\cite[lem 7.6]{kobayashi2015symmetry}},
  $\int_{\mathbb{S}^{p - 1}} \phi (\omega_{}) | \omega_p^{} |^{\lambda + \nu -
  n} d \omega_{} = 0$ if $N$ is odd or $\psi \perp \mathcal{H}^N
  (\mathbb{S}^{p - 1})^{O (p - 1)} =\mathbbm{C} \psi_N$, where $\mathcal{H}^N
  (\mathbbm{S}^{p - 1}) \ni \psi_N : \omega_{p - 1} \rightarrow
  \widetilde{\tilde{C}}_N^{p / 2 - 1} (\omega_{p - 1}^{(p)})$, with
  $\widetilde{\tilde{C}}_N^{\mu}$ denoting the renormalized Gegenbauer
  polynomial (see {\cite[(16.4)]{kobayashi2015symmetry}}). Thus in subsequent
  we may assume $N, m' + n' \in 2\mathbbm{Z}_{\geqslant 0}$.
  
  Now, as shown in {\cite[lem. 7.6]{kobayashi2015symmetry}},
  \[ \int_{\mathbb{S}^{n - 1}} | \omega_{n - 1}^{(n)} |^{\lambda + \nu - n} 
     \widetilde{\tilde{C}}_N^{\frac{n}{2} - 1} (\omega_{n - 1}^{(n)}) d
     \omega_{n - 1} \simeq \frac{\Gamma (\lambda + \nu - n + 1)}{\Gamma \left(
     \frac{\lambda + \nu - n - N + 2}{2} \right) \Gamma \left( \frac{\lambda +
     \nu - N}{2} \right)} . \]
  We note that it holds only for $n > 1$. For $n = 1$ {\cite[p.
  6]{howe1993homogeneous}} tells us that $\mathcal{H}^N (\mathbbm{S}^{q - 1})
  = 0$ for $N > 0$ and for $N = 0$ the latter integral becomes independent of
  $(\lambda, \nu)$. This gives us a first factor in the formula for $\varphi_N
  [g]$.
  
  We next compute the integral in $(r, s)$ above. Using variable change to
  coordinates $(x, y)$ given as
  \begin{eqnarray}
    & x \assign \frac{1 + (r^2 - s^2)}{\sqrt{R (r, s)}}, \quad y \assign
    \frac{1 - (r^2 - s^2)}{\sqrt{R (r, s)}}, \quad \frac{\partial (r,
    s)}{\partial (x, y)} = - \frac{1}{\sqrt{1 - x^2} \sqrt{1 - y^2} (x + y)^2}
    &  \nonumber\\
    & r = \frac{\sqrt{1 - y^2}}{x + y}, \quad s = \frac{\sqrt{1 - x^2}}{x +
    y} &  \nonumber
  \end{eqnarray}
  we can rewrite that integral as equal to (up to multiple inpendent of $m',
  n'$)
  \[ \int_{(x, y) \in D} x^{n'} y^{m'} (1 - x^2)^{(q - 2) / 2} (1 -
     y^2)^{(\lambda + \nu + N - q) / 2 - 1} | x - y |^{- \nu} d x d y, \]
  where $D \assign \{ (x, y) \in (- 1, 1)^2 | x + y > 0 \}$. But as $n' + m'
  \in 2\mathbbm{Z}_{\geqslant 0}$, this is proportional to an integral over
  $[- 1, 1]^2$.
\end{proof}

{\noindent}\tmtextbf{Fact \tmtextup{19}.
}\tmtextit{\label{k-finite:fact-hartogs}{\cite[thm.
1.2.5]{krantz1982function}} Let $\Omega \subset \mathbbm{C}^n$ be an open set
and $f : \Omega \rightarrow \mathbbm{C}$. Suppose that for each $j \in \{ 1,
2, \ldots, n \}$ and each fixed $(z_i)_{i = 1, i \neq j}^n$ we have $\{ z \in
\mathbbm{C} | (z_1, z_2, \ldots, z_{j - 1}, z, z_{j + 1}, \ldots, z_n) \in
\Omega \} \ni z \mapsto f (z_1, z_2, \ldots, z_{j - 1}, z, z_{j + 1}, \ldots,
z_n)$ is holomorphic as function of $z$. Then, $f$ is continuous on $\Omega$
and hence holomorphic on $\Omega$.}{\hspace*{\fill}}{\medskip}

\begin{proof}
  (of prop. \ref{k-finite:prop-KR-hook-2}) As $K_{\lambda,
  \nu}^{\mathbbm{R}^n} / R$ is holomorphic in $(\lambda, \nu) \in \{ \lambda +
  \nu \nin -\mathbbm{Z}_{\geqslant 0} \}$ and is a member of $\mathcal{S}
  \tmop{ol} (\mathbbm{R}^{p, q} ; \lambda, \nu)$, proposition
  \ref{k-finite:prop-holo-to-holo} gives us corresponding $k_{\lambda, \nu}
  \in \mathcal{D}' (\mathbbm{S}^p \times \mathbbm{S}^q)$. Moreover,
  proposition \ref{k-finite:prop-KR-hook-1} tells us that for $(\lambda, \nu)
  \in \Omega_{- 1}$ and $F \in \mathcal{H}^a (\mathbbm{S}^p) \otimes
  \mathcal{H}^b (\mathbbm{S}^q)$ with $a + b \in 2\mathbbm{Z}$ we have
  \[ \langle k_{\lambda, \nu}, F \rangle = \sum'_{N \in
     2\mathbbm{Z}_{\geqslant 0}}^{} k_N (\lambda, \nu) \cdot \frac{\varphi_N
     [g_N] (\lambda, \nu)}{\Gamma \left( \frac{1 - \nu}{2} \right) \Gamma
     \left( \frac{\lambda + \nu - n + 1}{2} \right) R (\lambda, \nu)} . \]
  Now, as both sides of latter equality are holomorphic in $(\lambda, \nu) \in
  \{ \lambda + \nu \nin -\mathbbm{Z}_{\geqslant 0} \}$ (the right-hand side is
  so by hypothesis), it extends to $\{ \lambda + \nu \nin
  -\mathbbm{Z}_{\geqslant 0} \}$.
  
  Now, we fix $\lambda = \lambda_0 \in \mathbbm{C}$ and let $k_{\nu} \assign
  k_{\lambda_0, \nu}$. The latter is holomorphic in $\nu \in
  \mathbbm{C}\backslash D_{}$ with $D_{} \assign \{ \nu \in \mathbbm{C} |
  \lambda_0 + \nu \in -\mathbbm{Z}_{\geqslant 0} \}$. We want to show that
  $k_{\nu}$ extends to holomorphic in $\nu \in \mathbbm{C}$ distribution.
  Proposition \ref{k-finite:prop-kfinite-extension-oneparam} tells us that it
  suffices to show that for every $F \in \mathcal{H}^a (\mathbbm{S}^p) \otimes
  \mathcal{H}^b (\mathbbm{S}^q)$ we have $\langle k_{\nu}, F \rangle$ being
  holomorphic in $\nu \in \mathbbm{C}$. Now, for $a + b \in 2\mathbbm{Z}+ 1$
  we have $k_{\nu}$ being even, while $F$ is odd and hence $\langle k_{\nu}, F
  \rangle = 0$ clearly holomorphic in $\nu$. For $a + b \in 2\mathbbm{Z}$ the
  conclusion in turn is granted by hypothesis. Repeating the same argument
  with $\nu = \nu_0$ fixed and using fact \ref{k-finite:fact-hartogs} tells us
  that $k_{\lambda, \nu}$ extends to holomorphic in $(\lambda, \nu) \in
  \mathbbm{C}^2$ distribution on $\mathbbm{S}^p \times \mathbbm{S}^q$ and thus
  proposition \ref{k-finite:prop-holo-to-holo} tells us that so does
  $K_{\lambda, \nu}^{\mathbbm{R}^n} / R$.
  
  It remains to show that for for $(\lambda, \nu) \in \mathbbm{C}^2$ we have
  $K_{\lambda, \nu}^{\mathbbm{R}^n} / R = 0 \Leftrightarrow \forall N \in
  2\mathbbm{Z}_{\geqslant 0} \forall g, \; \left( \varphi_N [g] / \Gamma
  \left( \frac{\lambda + \nu - n + 1}{2} \right) / \Gamma \left( \frac{1 -
  \nu}{2} \right) / R \right) (\lambda, \nu) = 0$. Note that proposition
  \ref{k-finite:prop-holo-to-holo} tells us that $k_{\lambda, \nu} =
  \iota^{\ast} (K_{\lambda, \nu}^{\mathbbm{R}^n} / R)^{\Xi}$ (with
  $(\cdot)^{\Xi}$ as in proposition \ref{k-finite:prop-claim2}). We first show
  the ``$\Leftarrow$''. Assuming that $\forall N \in 2\mathbbm{Z}_{\geqslant
  0} \forall g, \; \left( \varphi_N [g] / \Gamma \left( \frac{\lambda + \nu -
  n + 1}{2} \right) / \Gamma \left( \frac{1 - \nu}{2} \right) / R \right)
  (\lambda, \nu) = 0$ we see that $\langle k_{\lambda, \nu}, F \rangle = 0$
  for every $F \in \mathcal{H}^a (\mathbbm{S}^p) \otimes \mathcal{H}^b
  (\mathbbm{S}^q)$ with $a + b \in 2\mathbbm{Z}$. Arguing as in previous
  paragraph we see that $\langle k_{\lambda, \nu}, F \rangle = 0$ for $F \in
  \mathcal{H}^a (\mathbbm{S}^p) \otimes \mathcal{H}^b (\mathbbm{S}^q)$ without
  the $a + b \in 2\mathbbm{Z}$ assumption. Hence $k_{\lambda, \nu} = 0$. Lemma
  \ref{k-finite:lem-restriction-to-S} now implies that $(K_{\lambda,
  \nu}^{\mathbbm{R}^n} / R)^{\Xi} \in \mathcal{D}'_{\lambda - n} (\Xi)$ is
  zero and hence $K_{\lambda, \nu}^{\mathbbm{R}^n} / R = \psi^{\ast}
  (K_{\lambda, \nu}^{\mathbbm{R}^n} / R)^{\Xi} = 0$.
  
  Finally, we show the ``$\Rightarrow$'' direction, so we assume $K_{\lambda,
  \nu}^{\mathbbm{R}^n} / R = 0$. This implies (by the uniqueness part of lemma
  \ref{k-finite:prop-claim2}) that $(K_{\lambda, \nu}^{\mathbbm{R}^n} /
  R)^{\Xi} = 0$, hence $k_{\lambda, \nu} = \iota^{\ast} (K_{\lambda,
  \nu}^{\mathbbm{R}^n} / R)^{\Xi} = 0$. Proposition
  \ref{k-finite:prop-KR-hook-1} now implies that for every $N \in
  2\mathbbm{Z}_{\geqslant 0}$ and even polynomial $g$ there exists $F \in
  \sum_{i, a_i + b_i \in 2\mathbbm{Z}}' \mathcal{H}^{a_i} (\mathbbm{S}^p)
  \otimes \mathcal{H}^{b_i} (\mathbbm{S}^q)$ such that $\langle k_{\lambda,
  \nu}^{}, F \rangle = k \cdot \varphi_N [g] / / \Gamma \left( \frac{\lambda +
  \nu - n + 1}{2} \right) / \Gamma \left( \frac{1 - \nu}{2} \right) / R$ for
  some $k$ entire nonzero in $(\lambda, \nu) \in \mathbbm{C}^2$ (the actual
  statement of proposition \ref{k-finite:prop-KR-hook-1} was made to hold on
  $\Omega_{- 1}$, but as both sides are analytic in $(\lambda, \nu) \in \{
  \lambda - \nu \in -\mathbbm{Z}_{\geqslant 0} \}$ it extends) and thus we
  conclude that $\varphi_N [g] / \Gamma \left( \frac{\lambda + \nu - n + 1}{2}
  \right) / \Gamma \left( \frac{1 - \nu}{2} \right) / R = 0$. As $N$ and $g$
  were arbitrary, we are done.
\end{proof}

\begin{proof}
  (of prop. \ref{k-finite:prop-KC-hook-kfinite}) As in the proof of
  proposition \ref{k-finite:prop-KR-hook-1}, we may assume
  \[ F (\xi, \eta) = | \xi' |^N \phi \left( \frac{\xi'}{| \xi' |} \right)
     \xi_1^{m'} | \eta' |^M \phi' \left( \frac{\eta'}{| \eta' |} \right)
     \eta_{q + 1}^{n'} \]
  with $(\phi, \phi') \in \mathcal{H}^N (\mathbbm{S}^{p - 1}) \times
  \mathcal{H}^M (\mathbbm{S}^{q - 1})$, $(\xi_1, \xi') \assign \xi$, $(\eta',
  \eta_{q + 1}) \assign \eta$ and $m' + M + N + n' \in 2\mathbbm{Z}$.
  
  Now, by lemma \ref{k-finite:lem-restr-opendense} for an embedding
  \begin{eqnarray}
    & \psi_{N \rightarrow S}' : \{ x \in \mathbbm{R}^{p, q} \backslash \{ 0
    \} | Q (x) = 0 \} \hookrightarrow \{ (\xi, \eta) \in \mathbbm{S}^p \times
    \mathbbm{S}^q | \xi_1 = \eta_{q + 1} \} &  \nonumber\\
    & I \assign \psi'_{N \rightarrow S} (\{ x \in \mathbbm{R}^{p, q}
    \backslash \{ 0 \} | Q (x) = 0 \}) &  \nonumber
  \end{eqnarray}
  we have $\pm I \subset \{ (\xi, \eta) \in \mathbbm{S}^p \times \mathbbm{S}^q
  | \xi_1 = \eta_{q + 1} \}$ being dense and thus, as $K_{\lambda, \nu}^S$ and
  $F$ are even and for $\tmop{Re} (\lambda) \gg 0$ we have $| \xi_{p + 1}
  |^{\lambda + \nu - n} F$ having its $\nu - 1$'st normal derivative at $\{
  \xi_1 = \eta_{q + 1} \}$ being continuous, we can restrict integration of
  continuous function to an open dense subset to get
  \[ \langle K_{\lambda, \nu}^S, F \rangle = \langle K_{\lambda, \nu}^S, F
     \rangle_I \]
  Now, the map
  \begin{eqnarray}
    & D : \{ (x, y) \in (- 1, 1)^2 | x + y > 0 \} \times \mathbbm{S}^{p - 1}
    \times \mathbbm{S}^{q - 1} \ni (x, y, \omega, \omega') \mapsto (r \omega,
    s \omega') \in \{ (x, y) \in \mathbbm{R}^{p, q} | x \neq 0, y \neq 0 \} & 
    \nonumber\\
    & r : = \frac{\sqrt{1 - y^2}}{x + y}, \quad s : = \frac{\sqrt{1 - x^2}}{x
    + y} &  \nonumber
  \end{eqnarray}
  provides a diffeomorphism which pulls back $\{ x \in \mathbbm{R}^{p, q}
  \backslash \{ 0 \} | Q (x) = 0 \}$ to $\{ x = y \} \times \mathbbm{S}^{p -
  1} \times \mathbbm{S}^{q - 1}$ and as $\{ x \in \mathbbm{R}^{p, q}
  \backslash \{ 0 \} | Q (x) = 0 \} \subseteq \{ (x, y) \in \mathbbm{R}^{p, q}
  | x \neq 0, y \neq 0 \}$ we may pull back the integration above by $D$ to
  get (note that canonical volume form on $\mathbbm{S}^p \times \mathbbm{S}^q$
  is pulled back to a constant multiple of $| R (r, s) |^{- n / 2}$)
  \begin{eqnarray}
    & \langle K_{\lambda, \nu}^S, F \rangle_I \simeq \int_{\mathbbm{S}^{p -
    1}} \phi (\omega) | \omega_p |^{\lambda + \nu - n} d \omega_{} \times
    \int_{\mathbbm{S}^{q - 1}} \phi' (\omega') d \omega' \times &  \nonumber\\
    & \times \int_{\{ (x, y) \in (- 1, 1)^2 | x + y > 0 \}} (1 -
    y^2)^{(\lambda + \nu + N - q) / 2 - 1} \delta^{(\nu - 1)} (x - y) [(1 -
    x^2)^{(q - 2) / 2} g (x, y)] d x d y. &  \nonumber
  \end{eqnarray}
  
  
  One subsequently proceeds as in the proof of proposition
  \ref{k-finite:prop-KR-hook-1}.
\end{proof}

\begin{proof}
  (of prop. \ref{k-finite:prop-KC-hook-wrap}) This result is derived from
  proposition \ref{k-finite:prop-KC-hook-kfinite} in the same way as
  proposition \ref{k-finite:prop-KR-hook-2} is derived from proposition
  \ref{k-finite:prop-KR-hook-1}.
\end{proof}

\

\

\

\

\

\

\

\

\

\

\

\

\

\end{document}
