%japanese
\documentclass[8pt,notes]{beamer}
\mode<presentation>{\usetheme[secheader]{Boadilla}}
\usepackage{mystyle}
\usepackage{geometry,amsmath,amssymb,bbm,xypic}
\usepackage{xeCJK}
\usepackage{ruby}
\includecomment{versiona}
\usepackage{enumerate}
\usepackage{setspace}
\usepackage{amsmath,amssymb,bbm,xypic}
\usepackage[all,cmtip]{xy}
\usepackage{amsmath,amssymb,bbm,ulem,float,mystyle}
\usepackage{caption}
\usepackage{subcaption}
\usepackage{setspace}
\usepackage{comment}
%\excludecomment{versiona}
\includecomment{versiona}

\newcommand{\red}[1]{{\color[rgb]{0.6,0,0}#1}}
\newcommand{\Sol}{\mathcal{S}\mbox{ol}}
\newcommand{\Hom}{\mbox{Hom}}
\newcommand{\D}{\mathcal{D}}
\newcommand{\A}{\mathcal{A}}
\newcommand{\Co}{\mathbb{C}}
\newcommand{\X}{\mathbb{X}}
\renewcommand{\setminus}{-}
\newcommand{\nin}{\not\in}
\newcommand{\tmop}[1]{\ensuremath{\operatorname{#1}}}
\newcommand{\tmtextbf}[1]{{\bfseries{#1}}}
\newcommand{\tmtextit}[1]{{\itshape{#1}}}
\newcommand{\mss}{//}
\newcommand{\mbb}{\backslash\backslash}
\newcommand{\mmm}{\mid\mid}
\catcode`\<=\active \def<{
\fontencoding{T1}\selectfont\symbol{60}\fontencoding{\encodingdefault}}
\catcode`\>=\active \def>{
\fontencoding{T1}\selectfont\symbol{62}\fontencoding{\encodingdefault}}
\newcommand{\assign}{:=}
\newcommand{\comma}{{,}}
\newcommand{\um}{-}
\newcommand{\sol}{\mathcal{S}ol(\R^{p,q};\lambda,\nu)}
\newcommand{\Op}{\mbox{\normalfont Op}}
\newcommand{\OpR}{\mbox{\it R}}

\setCJKmainfont{Hiragino Mincho Pro}
\renewcommand{\thefootnote}{\fnsymbol{footnote}}
\hypersetup{colorlinks=true,urlcolor=blue}
\urlstyle{same}

\newenvironment{setting}{\begin{exampleblock}{Setting.}\it}{\end{exampleblock}}
\newenvironment{question}{\begin{block}{Problem.}\it}{\end{block}}
\newenvironment{prop}[1][]{\begin{block}{Proposition#1.}\it}{\end{block}}
\makeatletter
\def\th@mystyle{%
    \normalfont % body font
    \setbeamercolor{block title example}{bg=orange,fg=white}
    \setbeamercolor{block body example}{bg=orange!20,fg=black}
    \def\insertpropblockenv{exampleblock}
  	}
\makeatother
\theoremstyle{mystyle}
\newtheorem*{remark}{注.}

%%\newenvironment<>{setting}{%
%%  \begin{actionenv}#1%
%%      \def\insertblocktitle{Setting}%
%%      \par%
%%      \mode<presentation>{%
%%        \setbeamercolor{block title}{fg=white,bg=orange!20!black}
%%       \setbeamercolor{block body}{fg=black,bg=olive!50}
%%     }%
%%    \usebeamertemplate{block begin}}
%%{\par\usebeamertemplate{block end}\end{actionenv}}
%%\newenvironment<>{notation}{%
%%  \begin{actionenv}#1%
%%      \def\insertblocktitle{記号}%
%%      \par%
%%      \mode<presentation>{%
%%        \setbeamercolor{block title}{fg=white,bg=orange!20!black}
%%       \setbeamercolor{block body}{fg=black,bg=olive!50}
%%     }%
%%    \usebeamertemplate{block begin}}
%%{\par\usebeamertemplate{block end}\end{actionenv}}
%%\newenvironment<>{cor}{%
%%  \begin{actionenv}#1%
%%      \def\insertblocktitle{系}%
%%      \par%
%%      \mode<presentation>{%
%%        \setbeamercolor{block title}{fg=white,bg=orange!20!black}
%%       \setbeamercolor{block body}{fg=black,bg=olive!50}
%%     }%
%%    \usebeamertemplate{block begin}}
%%{\par\usebeamertemplate{block end}\end{actionenv}}

%%\title{
%%\author{レオンチエフ アレックス}
%%\institute{Tokyo U, Graduate School of Math Sciences}

\title{Symmetry breaking operators of indefinite orthogonal groups $O(p,q)$\footnote[1]{copy available at \url{nailbiter.insomnia247.nl/kansai2016.pdf}}}

% A subtitle is optional and this may be deleted

\author{Alex~Leontiev\inst{1} \and Toshiyuki~Kobayashi\inst{2}}
% - Give the names in the same order as the appear in the paper.
% - Use the \inst{?} command only if the authors have different
%   affiliation.

\institute[Tokyo U] % (optional, but mostly needed)
{
  \inst{1}%
  Graduate School of Math Sciences\\
  Tokyo U
  \and
  \inst{2}%
  Graduate School of Math Sciences, IPMU\\
  Tokyo U}
% - Use the \inst command only if there are several affiliations.
% - Keep it simple, no one is interested in your street address.

\date{MSJ Autumn Meeting, 2016}
% - Either use conference name or its abbreviation.
% - Not really informative to the audience, more for people (including
%   yourself) who are reading the slides online

\subject{Representation Theory}

\begin{document}
\begin{frame}\titlepage\end{frame}

\section{Problem Statement}
\begin{frame}
	\begin{setting}
		\begin{enumerate}
			\item 
				Let $p,q\ge1$, $G:=O(p+1,q+1)$ and $G':=O(p+1,q+1)_{e_{p+1}}\simeq O(p,q+1)$. 
			\item
				Let $P:=MAN$ and $P':=G'\cap P=M'AN'$ :max parabolic, where\\
\newcommand{\longminus}{\textemdash\textemdash}
\hspace{-1.05cm}  \begin{tabular}{l@{}}
    $N \assign \left\{ \left[ \begin{array}{lll}
      1 - Q & -^t w' & Q\\
      w & I_{p + q} & - w\\
      - Q & -^t w' & 1 + Q
    \end{array} \right] \middle| \begin{array}{c}
      (x, y) \in \mathbbm{R}^{p, q}\\
      w \assign (x, y)\\
      w' \assign (x, - y)\\
      Q \assign \frac{| x |^2 - | y |^2}{2}
    \end{array} \right\}$, $M \assign \left\{ \left[ \begin{array}{ccc}
      \epsilon & 0 & 0\\
      0 & A & 0\\
      0 & 0 & \epsilon
    \end{array} \right] \middle| \begin{array}{c}
      A \in O (p, q)\\
      \epsilon = \pm 1
    \end{array} \right\}$\\
    $N' \assign \left\{ \longminus \longminus \longminus'' \longminus
    \longminus \longminus \middle| \begin{array}{c}
      \longminus'' \longminus\\
      x_p = 0
    \end{array} \right\}$, $M' \assign \left\{ \longminus'' \longminus
    \middle| \begin{array}{c}
      \longminus'' \longminus\\
      A e_p = e_p
    \end{array} \right\}$\\
    $A \assign a (\mathbbm{R})$, \quad$a (t) \assign \left[ \begin{array}{ccc}
      \cosh (t) & 0 & \sinh (t)\\
      0 & I_{p + q} & 0\\
      \sinh (t) & 0 & \cosh (t)
    \end{array} \right]$
  \end{tabular}
			\item For $(\lambda,\nu)\in\mathbb{C}^2$ we let $I(\lambda),J(\nu)$ to be the degenerate principal series of $G,G'$ respectively, i.e.
$I(\lambda):=\{f\in C^\infty(G)\mid \forall h\in P,\;f(\cdot h)=\lambda(h) f(\cdot)\}$, where $\lambda:P\ni m\cdot a(t)\cdot n\mapsto e^{-\lambda t}$ is a $P$-representation and
with $G$-action by left multiplication, and similarly for $J(\nu)$.
		\end{enumerate}
	\end{setting}
	\begin{question}
\begin{enumerate}
\item Find the explicit basis for $\Hom_{G'}(I(\lambda),J(\nu))$.
\item Find the properties of elements of $\Hom_{G'}(I(\lambda),J(\nu))$ (e.g. their images, kernels etc.)
\end{enumerate}
	\end{question}
\end{frame}
\begin{frame}
\begin{theorem}
    The following diagram commutes:
    \vspace{-0.8cm}
\begin{figure}[H]
	\centerline{\xymatrixcolsep{5pc}\xymatrix{\Hom_{G'}(I(\lambda),J(\nu))\ar[r]^{\simeq} \ar@/^2pc/[rr]^{\mathcal{S}}
	&\left( I^{-\infty}(n-\lambda)\otimes\mathbb{C}_\nu \right)^{P'}
	\ar[r]_{F\mapsto \supp(F)}\ar[d]^{\simeq}_{\mbox{rest}}
	&2^{P'\backslash G/P}\\
	&\sol\subset\mathcal{D}'(\R^{p,q})\ar[lu]^{\mbox{Op}}_{\simeq}&
	}}
\end{figure}
\end{theorem}
\begin{theorem}
Note that $G$ acts on $\Xi^{p+1,q+1}:=\mysetn{(x,y)\in\R^{p+1,q+1}\setminus\left\{ 0 \right\}}{\myabs{x}^2=\myabs{y}^2}$ and on its quotient space
$X^{p,q}:=\Xi^{p+1,q+1}/\R^{\times}\simeq G/P$. Let
\[
	X:=G/P\simeq X^{p,q},\quad Y:=\mysetn{[\xi,\eta]\in G/P\simeq X^{p,q}}{\xi_{p}=0}\simeq X^{p-1,q}\]
	\[C:=\mysetn{[\xi,\eta]\in G/P\simeq X^{p,q}}{\xi_{0}=\eta_q}\simeq X^{p-1,q-1}\cup\Xi^{p,q},\quad\left\{ [0] \right\}:=\left\{ [1,0_{p+q},1] \right\}\]
For $p>1$, the left $P'$-invariant closed subspaces of $G/P$ are as follows:\\
  \begin{figure}[H]
	  \vspace{-0.2cm}\hspace{2cm}
    \xymatrixrowsep{0.1cm}\xymatrix{
        &Y\ar@{-}[ld]\ar@{-}[rd]&&\\
	X\ar@{-}[rd]&&Y\cap C\ar@{-}[ld]\ar@{-}[r]&\left\{ [0] \right\}\\
	&C&&
    }
\end{figure}
\end{theorem}
\end{frame}
\begin{frame}
	\begin{theorem}
		We can construct the following families of SBOs which holomorphically depend on parameters:
		\label{<++>}
	\end{theorem}<++>
\end{frame}<++>
\end{document}
