%\documentclass[10pt]{article} % use larger type; default would be 10pt
\documentclass[14pt]{extarticle} % use larger type; default would be 10pt

\usepackage{mathtext}                 % підключення кирилиці у математичних формулах
                                          % (mathtext.sty входить в пакет t2).
\usepackage[T1,T2A]{fontenc}         % внутрішнє кодування шрифтів (може бути декілька);
                                          % вказане останнім діє по замовчуванню;
                                          % кириличне має співпадати з заданим в ukrhyph.tex.
\usepackage[utf8]{inputenc}       % кодування документа; замість cp866nav
                                          % може бути cp1251, koi8-u, macukr, iso88595, utf8.
\usepackage[english,ukrainian]{babel} % національна локалізація; може бути декілька
                                          % мов; остання з переліку діє по замовчуванню. 
\usepackage{amsthm}
\usepackage{amsmath}
\usepackage{amsfonts}
\usepackage{graphicx}
\usepackage[pdftex]{hyperref}
\usepackage{caption}
\usepackage{subfig}
\usepackage{fancyhdr}
\usepackage{mystyle}
\usepackage{color}
\usepackage[table]{xcolor}
\usepackage{colortbl}

%custom command for title
\newcommand{\HRule}{\rule{\linewidth}{0.5mm}}

%for Re and Im like in the book
\renewcommand\Re{\operatorname{Re}}
\renewcommand\Im{\operatorname{Im}}

%more space after \forall and \exists
\let\oldforall\forall
\renewcommand{\forall}{\oldforall\;}
\let\oldexists\exists
\renewcommand{\exists}{\oldexists\;}

%custom theorem environments
\newtheorem{definition}{Означення}[section]
\renewcommand{\thedefinition}{\arabic{definition}}
\newtheorem{example}{\indent Приклад}[section]
\renewcommand{\theexample}{\arabic{example}}
\newtheorem{exercise}{Вправа}
\newtheorem{theorem}{Теорема}
\newtheorem{lemma}{Лема}
\newtheorem{observation}{Спостереження}
\newtheorem*{fact}{Факт}
\newtheorem{proposition}{Твердження}
\newtheorem{corollary}[proposition]{Наслідок}
\theoremstyle{remark}
\newtheorem{remark}{Зауваження}

\title{}
\author{Олексій Леонтьєв}

\begin{document}
\maketitle
\begin{proposition}Припустимо, що для довільних $t,s\in\mathbb{R}$, матриці $A(s)$, $A(t)$, $B(t)$ та $B(s)$ попарно комутують.
	Припустимо також, що $\int_{-\infty}^\infty\mynorm{B(t)}\;dt<+\infty$. Тоді система 
	\begin{equation}\label{Unperturbed}
	x'=A(t)x(t)
	\end{equation}
	має експоненціальну дихотомію на $\mathbb{R}$ тоді і лише тоді, коли її має система
	\begin{equation}\label{Perturbed}
	x'=(A(t)+B(t))x(t)
	\end{equation}
\end{proposition}
\begin{proof}Враховуючи гіпотезу про комутацію, матриціантами систем \eqref{Unperturbed} та \eqref{Perturbed} будуть
	\[X(t):=\exp\mycbra{\int_0^tA(s)\;ds}\]
	та \[\tilde{X}(t):=\exp\mycbra{\int_0^t(A(s)+B(s))\;ds}=\exp\mycbra{\int_0^tA(s)\;ds}\exp\mycbra{\int_0^tB(s)\;ds}\]
	і останні два множники в правій частині комутують при всіх $t$.
	Через симетричність умови, достатньо показати лише, що з дихотомії \eqref{Unperturbed} випливає дихотомія \eqref{Perturbed}.
	Далі, із \cite{coppel} ми знаємо, що еквівалентним означенням експоненціальної дихотомії \eqref{Perturbed}
	на $\mathbb{R}$ є існування
	такого оператора-проекції $P$ і чисел $\gamma,K>0$, що $\forall s,t\in\mathbb{R}$ маємо
	\[\mynorm{X(t)PX^{-1}(s)}\leq Ke^{-\gamma(t-s)},\;s\leq t\]
	\[\mynorm{X(t)(I-P)X^{-1}(s)}\leq K^{-\gamma(s-t)},\;t\leq s\]
	Ми покажемо, що результат залишається тим же самим (із, можливо, більшим $K$), із $X(t)$ заміненим на $\tilde{X}(t)$.
	Дійсно, матимемо тоді для $s\leq t$
	\[\mynorm{\tilde{X}(t)P\tilde{X}^{-1}(s)}=\]
	\[=\mynorm{\exp\mycbra{\int_0^tB(l)\;dl}\exp\mycbra{\int_0^tA(l)\;dl}P\exp\mycbra{-\int_0^sA(l)\;dl}
	\exp\mycbra{-\int_0^sB(l)\;dl}}\leq\]
	\[\leq\mynorm{\exp\mycbra{\int_0^tB(l)\;dl}}\cdot\mynorm{\exp\mycbra{\int_0^tA(l)\;dl}P\exp\mycbra{-\int_0^sA(l)\;dl}}\times
	\]\[\times\mynorm{\exp\mycbra{-\int_0^sB(l)\;dl}}\leq\]
	\[\leq\mynorm{\exp\mycbra{\int_0^tB(l)\;dl}}\cdot Ke^{-\gamma(t-s)}\cdot
	\mynorm{\exp\mycbra{-\int_0^sB(l)\;dl}}\leq\]
	Таким чином, достатньо показати, що
	\[\mynorm{\pm\exp\mycbra{\int_0^tB(l)\;dl}}\]
	рівномірно обмежена в $t\in\mathbb{R}$. Це, в свою чергу, слідує із
	\[\mynorm{\exp\mycbra{\int_0^tB(l)\;dl}}\leq\exp\mycbra{\mynorm{\int_0^tB(l)\;dl}}\leq\exp\mycbra{\int_0^t\mynorm{B(l)}\;dl
	}\leq\]
	\[\leq\exp\mycbra{\int_{-\infty}^{\infty}\mynorm{B(l)}\;dl}<+\infty\]
	і аналогічного аргументу для $-$.
\end{proof}
\begin{proposition}
	\label{strong}
	Припустимо, що для довільних $t,s\in\mathbb{R}$, матриці $A(s)$ та $B(t)$ комутують.
	Припустимо також, що $\int_{-\infty}^\infty\mynorm{B(t)}\;dt<+\infty$. Тоді система 
	\begin{equation}\label{Perturbed}
	x'=(A(t)+B(t))x(t)
	\end{equation}
	має експоненціальну дихотомію на $\mathbb{R}$, якщо її має система
	\begin{equation}\label{Unperturbed}
	x'=A(t)x(t)
	\end{equation}
\end{proposition}
\begin{proof}
	Бажаний результат випливатиме з двох спостережень, які ми покажемо нижче
	\begin{observation}\label{Hard}Нехай для довільних $t,s\in\mathbb{R}$, $A(t)$ та $B(s)$ комутують, де $A,B$ неперервні
	(не обов’язково обмежені) матричні функції. Припустимо також, що матриціант системи $x'=Bx$, який ми позначатимемо за
	$X_B(t)$, існує для всіх $t\in\mathbb{R}$. Тоді для довільних $t,s\in\mathbb{R}$ $A(t)$ та $X_B(s)$ комутують.\end{observation}
	\begin{observation}\label{Easy}Нехай $B$ -- неперервна матрична функція така, що $\int_{-\infty}^\infty B(s)\;ds<+\infty$. Тоді матриціант
		$X_B$ системи $x'=Bx$ існує і норми $\mynorm{X_B(t)}$ та $\mynorm{X^{-1}_B(t)}$ обмежені рівномірно в $t$ на $\mathbb{R}$.
	\end{observation}
	Дійсно, вважаючи спостереження вище вірними, розглянемо систему \ref{Perturbed} $x'=(A+B)x(t)$. Позначимо матриціант системи $x'=Bx$
	за $X_B(t)$ та зробимо заміну $x(t)=X_B(t)z(t)$ матимемо
	\[BX_Bz+X_B\dot{z}=(X_Bz)'=x'=(A+B)X_Bz\]
	оскільки за спостереженням \ref{Hard}, $A$ та $X_B$ комутують (помітимо, що умови даного твердження сильніші за умови спостереження), маємо
	\[X_B\dot{z}=X_BAz\]
	і відповідно, $z$ є розв’язком системи \ref{Unperturbed}. Таким чином, матриціант $X_{A+B}$ системи \ref{Perturbed}
	має структуру
	\[X_{A+B}=X_BX_A\]
	і ми, маючи спостереження \ref{Easy}, можемо застосувати таку ж схему оцінки, як і в попередньому твердженні
	\[\mynorm{X_{A+B}(t)PX^{-1}_{A+B}(s)}=\mynorm{X_B(t)X_A(t)PX^{-1}_A(s)X^{-1}_B(s)}\leq\]
	\[\leq\mynorm{X_B(t)}\mynorm{X_A(t)PX^{-1}_A(s)}\mynorm{X_B^{-1}(s)}\]
	\begin{proof}{(Спостереження \ref{Easy})} Рівномірна обмеженість $\mynorm{X_B(t)}$
	випливає безпосередньо з доведення Теореми 1 в \cite[\S 12]{demidovich}, де в 
	якості системи із сталою матрицею ми взяли стійку систему $x'=0x$. Позначимо отриману в доведенні зазначеною Теореми 1 верхню межу
	$\mynorm{X_B(t)}$ на $\mathbb{R}$, за
	$M$. Покажемо, що $\forall t,\;\mynorm{X^{-1}(t)}\leq M$, чим і закінчимо доведення. Дійсно, для сталого $\tau\in\mathbb{R}$,
	$X^{-1}(\tau)$ можна характеризувати як $X^{-1}(\tau)x(\tau)=x(0)$ де $x(t)$ задовольняє $x'=Bx$ або, що те ж саме, як
	$X^{-1}(\tau)y(0)=y(-\tau)$ де $y'=\tilde{B}y$, а $\tilde{B}(t):=B(t+\tau)$. Таким чином, достатньо показати, що матриціант системи
	$y'=\tilde{B}y$ обмежений зверху $M$. Це, в свою чергу, вірно, оскільки із доведення Теореми 1 бачимо, що $M$ залежить лише від
	сталої матриці (0, в нашому випадку) та $\int_{-\infty}^\infty\mynorm{B(s)}\;ds$, а $\int_{-\infty}^\infty\mynorm{B(s)}\;ds=
	\int_{-\infty}^\infty\mynorm{\tilde{B}(s)}\;ds.$
\end{proof}
\begin{proof}{(Спостереження \ref{Hard})}
	Ми будемо користуватися певними фактами без доведення. Для подальшої зручності, ми запишемо їх усі зараз тут.
	\begin{fact}{(\textbf{Розклад Магнуса},
		з \cite{moan})}\label{MagnusConvergenceFact}
		Нехай маємо систему $x'=Bx$ із неперервним $B$ і матриціантом $X$. Припустимо, що для $T$ виконується
		\[\int_0^T\mynorm{B(s)}\;ds<\pi\]
		тоді на $t\in[0,T)$ $X(t)=\exp(\Omega(t))$, де $\Omega(t)=\sum\limits_{k=1}^\infty\Omega_k(t)$ і $\Omega_k(t)$ є інтегралом 
		комутаторів зростаючої довжини, як-от
		\begin{align*}
		\Omega_1(t) &= \int_0^t B(t_1)\,dt_1, \\
		\Omega_2(t) &= \frac{1}{2}\int_0^t dt_1 \int_0^{t_1} dt_2\ \left[  B(t_1),B(t_2)\right], \\
		\Omega_3(t) &= \frac{1}{6} \int_0^t dt_1 \int_0^{t_1}d t_2 \int_0^{t_2} dt_3
		\Bigl(\left[B(t_1),\left[B(t_2),B(t_3)\right]\right]+\left[B(t_3),\left[  B(t_2),B(t_{1})\right]\right]\Bigr), \\
		\Omega_4(t) &= \frac{1}{12} \int_0^t dt_1 \int_0^{t_1}d t_2 \int_0^{t_2} dt_3 \int_0^{t_3} dt_4
		\Bigl(\left[\left[\left[B_1,B_2\right],B_3\right],B_4\right] \\
		&\quad+\left[B_1,\left[\left[B_2,B_3\right],B_4\right]\right]
		+\left[B_1,\left[B_2,\left[B_3,B_4\right]\right]\right]
		+\left[B_2,\left[B_3,\left[B_4,B_1\right]\right]\right]\Bigr)
		\end{align*}
	\end{fact}
	Зафіксуємо довільне $\tau\in\mathbb{R}$ і покажемо, що
	\[\forall t\in\mathbb{R},\;\mysbra{A(\tau),X_B(t)}=0\]
	(тут $\mycbra{\cdot,\cdot}$ позначає дужку Лі). Достатньо показати, що множина $t$, для яких це виконується є закритою і відкритою
	водночас (помітимо, що для $t=0$ твердження виконується, оскільки $X_B(0)=I$ комутує з усіма матрицями).
	Оскільки дужка Лі, $A$ та $B$ неперервні за гіпотезою, достатньо показати відкритість. Припустимо, таким чином, що для $t_0$ умова 
	комутації виконується і покажемо, що вона виконується на малому околі $t_0$.

	Достатньо показати, що для малих $s\in\mathbb{R}$, $A(\tau)$ комутує з $\tilde{X}(s):=X_B(t_0+s)X_B^{-1}(t_0)$. Останнє, в свою чергу,
	користуючись тією ж логікою, що і в кінці попереднього доведення, можна представити як матриціант системи $x'=\tilde{B}x$, де
	$\tilde{B}(t):=B(t+t_0)$. Таким чином, без втрати загальності можна вважати $t_0=0$. На малому околі 0, оскільки $B(t)$ неперервна,
	виконується умова ~\hyperref[MagnusConvergenceFact]{факту} і, таким чином, до матриціанту можна застосувати розклад Магнуса. 
	
	Оскільки
	кожен член $\Omega_k(t)$ є інтегралом комутаторів матриць, кожна з яких комутує з $A(\tau)$ то кожне $\Omega_k(t)$ комутує з $A(\tau)$.
	Дійсно, якщо $B,C$ комутують з $A$, то $BC$, а отже $[B,C]$ також комутують. Більше того,
	якщо $B(t)$ комутує з $A$ для всіх $t$, то $\int_a^bB(t)\;dt$ також комутує з $A$. Оскільки дужка Лі є неперервною, бачимо, що
	$\Omega(t)$ також комутує з $A(\tau)$. Таким чином, $X(t)=\exp(\Omega(t))=\sum\limits_{n=0}^\infty
	\frac{\Omega^n(t)}{n!}$ також комутує з $A(\tau)$ при малих $t$, що і завершує доведення.
	\end{proof}
\end{proof}
\begin{remark}Гіпотеза
	твердження \ref{strong} щодо комутації
	задовільняється, наприклад, якщо $\forall t,\;B(t)\in\bigcap\limits_{s\in\mathbb{R}} Z_{\mathfrak{gl}_n}(A(s))$, де
	$Z_{\mathfrak{gl}_n}(A)$ позначає підалгебру Лі матриць, що комутують з $A$. У випадку $A(t)\equiv A=const$, цю умову
можна записати простіше як $\forall t,\;B(t)\in Z_{\mathfrak{gl}_n}(A)$.\end{remark}
\begin{thebibliography}{9}
\bibitem{coppel}
	W. A. Coppel, {\em Dichotomies in Stability Theory} (1978).
\bibitem{moan}
	Per Christian Moan, Jitse Niesen, {\em Convergence of the Magnus series}, доступна на 
	\url{http://citeseerx.ist.psu.edu/viewdoc/summary?doi=10.1.1.63.6759}
\bibitem{demidovich}
Демидович Б. П. \emph{Лекции по математической теории устойчивости} --
Москва, 1967 г., 472 стр. с илл.
\end{thebibliography}
\end{document}
