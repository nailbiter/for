%\documentclass[10pt]{article} % use larger type; default would be 10pt
\documentclass[14pt]{extarticle} % use larger type; default would be 10pt

\usepackage{mathtext}                 % підключення кирилиці у математичних формулах
                                          % (mathtext.sty входить в пакет t2).
\usepackage[T1,T2A]{fontenc}         % внутрішнє кодування шрифтів (може бути декілька);
                                          % вказане останнім діє по замовчуванню;
                                          % кириличне має співпадати з заданим в ukrhyph.tex.
\usepackage[utf8]{inputenc}       % кодування документа; замість cp866nav
                                          % може бути cp1251, koi8-u, macukr, iso88595, utf8.
\usepackage[english,ukrainian]{babel} % національна локалізація; може бути декілька
                                          % мов; остання з переліку діє по замовчуванню. 
\usepackage{amsthm}
\usepackage{amsmath}
\usepackage{amsfonts}
\usepackage{graphicx}
\usepackage[pdftex]{hyperref}
\usepackage{caption}
\usepackage{subfig}
\usepackage{fancyhdr}
\usepackage{mystyle}
\usepackage{color}
\usepackage[table]{xcolor}
\usepackage{colortbl}

%custom command for title
\newcommand{\HRule}{\rule{\linewidth}{0.5mm}}

%for Re and Im like in the book
\renewcommand\Re{\operatorname{Re}}
\renewcommand\Im{\operatorname{Im}}

%more space after \forall and \exists
\let\oldforall\forall
\renewcommand{\forall}{\oldforall\;}
\let\oldexists\exists
\renewcommand{\exists}{\oldexists\;}

%custom theorem environments
\newtheorem{definition}{Означення}[section]
\renewcommand{\thedefinition}{\arabic{definition}}
\newtheorem{example}{\indent Приклад}[section]
\renewcommand{\theexample}{\arabic{example}}
\newtheorem{exercise}{Вправа}
\newtheorem{theorem}{Теорема}
\newtheorem*{proposition}{Твердження}
\newtheorem{remark}{Зауваження}
\title{}
\author{Олексій Леонтьєв}

\begin{document}
\maketitle
\begin{proposition}Припустимо, що для довільних $t,s\in\mathbb{R}$, матриці $A(s)$, $A(t)$, $B(t)$ та $B(s)$ попарно комутують.
	Припустимо також, що $\int_{-\infty}^\infty\mynorm{B(t)}\;dt<+\infty$. Тоді система 
	\begin{equation}\label{Unperturbed}
	x'=A(t)x(t)
	\end{equation}
	має експоненціальну дихотомію на $\mathbb{R}$ тоді і лише тоді, коли її має система
	\begin{equation}\label{Perturbed}
	x'=(A(t)+B(t))x(t)
	\end{equation}
\end{proposition}
\begin{proof}Враховуючи гіпотезу про комутацію, матриціантами систем \eqref{Unperturbed} та \eqref{Perturbed} будуть
	\[X(t):=\exp\mycbra{\int_0^tA(s)\;ds}\]
	та \[\tilde{X}(t):=\exp\mycbra{\int_0^t(A(s)+B(s))\;ds}=\exp\mycbra{\int_0^tA(s)\;ds}\exp\mycbra{\int_0^tB(s)\;ds}\]
	і останні два множники в правій частині комутують при всіх $t$.
	Достатньо показати лише, що з дихотомії \eqref{Unperturbed} випливає дихотомія \eqref{Perturbed}.
	Далі, із \cite{coppel} ми знаємо, що еквівалентним означенням експоненціальної дихотомії \eqref{Perturbed}
	на $\mathbb{R}$ є існування
	такого оператора-проекції $P$ і чисел $\gamma,K>0$, що $\forall s,t\in\mathbb{R}$ маємо
	\[\mynorm{X(t)PX^{-1}(s)}\leq Ke^{-\gamma(t-s)},\;s\leq t\]
	\[\mynorm{X(t)(I-P)X^{-1}(s)}\leq K^{-\gamma(s-t)},\;t\leq s\]
	Ми покажемо, що результат залишається тим же самим (із, можливо, більшим $K$), із $X(t)$ заміненим на $\tilde{X}(t)$.
	Дійсно, матимемо тоді для $s\leq t$
	\[\mynorm{\tilde{X}(t)P\tilde{X}^{-1}(s)}=\]
	\[=\mynorm{\exp\mycbra{\int_0^tB(l)\;dl}\exp\mycbra{\int_0^tA(l)\;dl}P\exp\mycbra{-\int_0^sA(l)\;dl}
	\exp\mycbra{-\int_0^sB(l)\;dl}}\leq\]
	\[\leq\mynorm{\exp\mycbra{\int_0^tB(l)\;dl}}\cdot\mynorm{\exp\mycbra{\int_0^tA(l)\;dl}P\exp\mycbra{-\int_0^sA(l)\;dl}}\times
	\]\[\times\mynorm{\exp\mycbra{-\int_0^sB(l)\;dl}}\leq\]
	\[\leq\mynorm{\exp\mycbra{\int_0^tB(l)\;dl}}\cdot Ke^{-\gamma(t-s)}\cdot
	\mynorm{\exp\mycbra{-\int_0^sB(l)\;dl}}\leq\]
	Таким чином, достатньо показати, що
	\[\mynorm{\pm\exp\mycbra{\int_0^tB(l)\;dl}}\]
	Це, в свою чергу, слідує із
	\[\mynorm{\exp\mycbra{\int_0^tB(l)\;dl}}\leq\exp\mycbra{\mynorm{\int_0^tB(l)\;dl}}\leq\exp\mycbra{\int_0^t\mynorm{B(l)}\;dl
	}\leq\]
	\[\leq\exp\mycbra{\int_{-\infty}^{\infty}\mynorm{B(l)}\;dl}<+\infty\]
	і аналогічного аргументу для $-$.
\end{proof}
\begin{thebibliography}{9}
\bibitem{coppel}
	W. A. Coppel, {\em Dichotomies in Stability Theory} (1978).
\end{thebibliography}
\end{document}
