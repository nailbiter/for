\documentclass[11pt]{book}
%\documentclass[8pt]{article} % use larger type; default would be 10pt

\usepackage{fontspec}
\title{Introduction to the modern Japan}
\begin{document}
\maketitle
\section*{Preface}
\subsection*{Part One}
In the spring of 2009 I was sitting in the train going from the Narita Airport to Tokyo. At the time when the train have entered the city, it was
already dark. In the twilight I have spotted another train that was carrying workers going home. The light in the train was quite bright and I
passenger have been either sitting or standing and looked very organized. For a moment I seemed to fall into the realm of Kenji Miyazawa's "Night
on the Galactic Railroad" - in the middle of the night two trains are slowly leaving towards the Milky Way. In the orange light going through the windows, the passengers in the other train have been peeling apples,
joking. And Miyazawa's Giovanni met with injured children from the Titanic in a train...\\
Saying about "Night on the Galactic Railroad", I have prepared an another book: "The Cultural History of Japan - Conflict and Agreement" (the name is tentative) to discuss this. It is worth to mention, that the railroad is
the symbol of Japan's modernization and Japanese began to rule Manchuria exactly by building a railroad. Secondly, whether he reads the book, or writing a message on mobile phone, neatly clothed Japanese always silently staying or
sitting, and talk only a bit. Also, Japanese trains and trams are very precise in their timing, which is extremely rare in Asia.\\
Starting from the Meiji Restoration there is a strong wave of rapid Westernization in Japan. But while for the other Asian countries this time
can be said to bring the beginning of setbacks and humiliation, lost wars, loss of territory, compensation payment, becoming colonies and
massacres, Japan became rich and built its military power, thus becoming a nightmare for each other Asian country. The influence of Meiji period
on Japanese far exceeds their own understanding. For example, there is a lot of things that are considered "traditional" by insightful people,
yet were established as late as during the Meiji. For example, the requirement that only the male can be the Emperor, the association of emperror's
rule with the name of the period in history, same surname shared by husband and wife etc. After the end of the World War II even despite the
disadvantageous peace treaty imposed by USA, Japan has unexpectedly transformed into strong economic power (in fact, into strong military power
as well). At that time all over Asian countries have said again: Japanese make invasion by the means of economics.\\
The period when I have been studying in  Japan (1977-1984), it has already passed through the peak of economic growth and was coming into the 
bluffing period of a bubble economics. At that time Ezra V. Fogel has already declared: "Japan is number one" (1979),
and many Japanese also were full
of self-confidence than they have been thinking about their country. They thought Japan is already staying on the top of the world. Wang Chenchih
have broken the world home run record (1977), adventurer Naomi Uemura reached the North Polo solo on a dog sled (1978), the Disneyland in Tokyo
started its operation and NHK broadcaster aired television drama "Oshin" (1983). In the entertainment industry there were
Pink Lady's "UFO", the retirement of Momoe Yamaguchi, debut of Seiko Matsuda
だョ 
時!全員集合

\subsection*{Part Two}
\part{Japan prior to the Second World War}
\chapter{The cultural heritage of Edo Period}
\section{The reasons for the success of the Meiji Restoration}
At 19th century under the massive colonization of the East by the Western countries, China, India ant other eastern countries have gradually became
either colonies, or half-colonies. As the sole exception, the Japan did not became a colony, and because of the success of the Meiji Restoration
reforms, have smoothly set foot on the way of modernization. And even till 20th century, there are still a lot of countries suffer from the late
modernization. As an example, in Iran king Pahlavi while trying to imitate Japanese Meiji Restoration, have sent a large number of people
to study abroad, thus causing revolution. At the end, Pahlavi was forced to flee overseas.\\
So, what are the reasons of the success of the Meiji Restoration? This is not an easy question, although in accordance with the present thoughts,
we may summarize as follows:
External reasons:
\begin{enumerate}
	\item{Balance in strength - one country was seldom able to make strong moves}
	\item{England did not value the importance of the Japanese market, when compared to that of China or India. Japan was though as minor
		in diplomatic sense}
\end{enumerate}
Internal reasons:
\begin{enumerate}
	\item{Blessed with the Taika reforms, Japan have accepted western culture with ease, as it did not have a feeling of strong cultural
		superiority}
	\item{Due to the system of the private schools developed during the Edo period illiteracy was low, and the standards of the national
		education were high}
	\item{Confrontation between the hans (\textit{jap.} 藩 - vassal states) gave momentum to the reforms}
	\item{Merchants of the Edo period have raised their heads, business activity have flourished; moreover under the rule of the Shogunate
		system, the national domestic market have already been formed.}
	\item{The middle class, that was ranked lower than the samurai, became the main politic force, and this was helpful for the establishment
		of the modern-type national country}
\end{enumerate}
\section{The basis for the success of Meiji Restoration - mass education of the Edo Period}
Japanese scholar Kuwabara Takeo (1904-1988) have incisively pointed out: the reasons for the success of Meiji Restoration are in the high reading
and writing ability of the Japanese people of that time. The literacy of the Meiji Restoration period is approximately 43\% for men; whereas
in the France during the Revolution literacy was lower than 30 \%; at the time of the founding of the People Republic of China (1949) about 
15\%; when the India got the independence only 10\%. By comparing these numbers one sees 
that when compared with other societies enduring great revolutions, the highest literacy rate was probably achieved in Japan during Meiji
Reconstruction. This surely can be traced back to the early built strong system of education of the Edo period - educational 
contribution of "terakoya" (\textit{jap.}
 寺子屋).\\
"Terakoya" (picture 1) have originated in the late Muromachi period (15th century), and become prominent public education institutions
during the Edo period. The name "terakoya", was originally used for the education centers in monasteries where monks have educated secular
people. But when it came to the Edo period, there were already only few terakoya operated by monasteries. According to statistics, in 1875
in Japan there were opened 15600 terakoya in total.\\
The school enrollment rate for men in schools at that time was 43\% and 10\% for woman which was better than in Europe at that time. Young students
have enrolled in the school at age from 6 to 8 and have been studying from 3 to 5 years. The curriculum of terakoya consisted of three practical
disciplines, namely reading, writing and calculations with the abacus. Such promotion of knowledge have mutually interfered with the society, 
economics and culture.
\section{Confrontation between the hans}
Satsuma (modern Kagoshima prefecture), Choshu (modern Yamaguchi prefecture), Tosa (modern Kochi prefecture) and other hans on the south-west 
of Japan have already carried out an agriculture reform, because of the farmer riots in 19th century. Afterwards, under the pressure of the
foreign powers and high sensing of crisis, they have actively implemented the political reforms. These hans, had strong army, and therefore
high political power, and were known as 雄藩 during the late Edo period. Among these hans Satsuma
and Choshu were 300 years ago defeated by the Tokugawa shogunate in the Battle of Sekigahara during the Sengoku period, therefore they were
strongly against the Tokugawa shogunate. These hans not only strengthened minister's influence to confront Tokugawa Shogunate, but also later
various senior positions in the newly established government were occupied by the samurai from these hans.\\

\chapter{Meiji Restoration and the amelioration of the culture}
\section{Collapse of the Shogunate of the Edo Period}
\section{Meiji Restoration}%page 23
In March 1868 (first year of Meiji), emperor
 
\section{Meiji Government's religious policy}%small
\subsection{Repressions on the Christians}
\subsection{Haibitsu Kishaki (\textit{jap.} 廃仏毀釈)}

\section{New government and the world}

\section{New academic system}%small
\section{Drastic changes in social structure}%small
\section{Strong army of rich country and the amelioration of the culture}%small
\section{Women's fate in the "amelioration of the culture"}%small
\end{document}
