\documentclass[pdf]{beamer}
\mode<presentation>{\usetheme[secheader]{Boadilla}}
\usepackage{mystyle}
\usepackage{amsthm,amssymb}
\usepackage{mathtools}
\usepackage{framed}
\includecomment{versiona}
\usepackage{mystyle}
\usepackage{geometry}
\usepackage{amsmath}
\usepackage{ruby}
\usepackage{enumerate}
\usepackage{setspace}
\usepackage{xypic}
\usepackage[all,cmtip]{xy}
\usepackage{bbm,ulem,float,mystyle}
\usepackage{caption}
\usepackage{subcaption}
\usepackage{setspace}
\usepackage{tikz}
\usepackage{tikz-cd,array}
\usepackage{catchfilebetweentags}
\usepackage{textcomp}
\usepackage{etoolbox}
\patchcmd{\thebibliography}{\section*{\refname}}{}{}{}
\usepackage{pifont}
\usepackage{scalerel}
\usepackage{adjustbox}
\usepackage{tcolorbox}
\usepackage{lipsum}
\usepackage{calc}
\usepackage{tikz, pgfplots}
\pgfplotsset{width=8cm,compat=1.9}
\usetikzlibrary{pgfplots.dateplot}
\usepackage{pgfplotstable}
\usepackage{graphicx}
\usepackage{amsmath}
\usepackage{amssymb}
\usepackage{relsize}
\usepackage{multirow}
\usepackage{rotating}
\usepackage{bm}
\usepackage{url}
\usepackage{mystyle}
\usepackage{enumerate}
\usepackage{geometry}
\usepackage{setspace}
\usepackage{amsmath,amssymb,bbm,xypic}
\usepackage[all,cmtip]{xy}
\usepackage{amsmath,amssymb,bbm,float,mystyle}

\newcommand{\red}[1]{{\color[rgb]{0.6,0,0}#1}}
%\setbeameroption{show only notes}

%%\newcommand{\red}[1]{{\color[rgb]{0.6,0,0}#1}}
\newcommand{\Sol}{\mathcal{S}\mbox{ol}}
\newcommand{\D}{\mathcal{D}}
\newcommand{\A}{\mathcal{A}}
\newcommand{\Co}{\mathbb{C}}
\newcommand{\X}{\mathbb{X}}
\renewcommand{\setminus}{\backslash}
\newcommand{\nin}{\not\in}
\newcommand{\Ind}{\ensuremath{\operatorname{Ind}}}
\newcommand{\tmop}[1]{\ensuremath{\operatorname{#1}}}
\newcommand{\tmtextbf}[1]{{\bfseries{#1}}}
\newcommand{\tmtextit}[1]{{\itshape{#1}}}
\newcommand{\mss}{//}
\newcommand{\mbb}{\backslash\backslash}
\newcommand{\mmm}{\mid\mid}
\catcode`\<=\active \def<{
\fontencoding{T1}\selectfont\symbol{60}\fontencoding{\encodingdefault}}
\catcode`\>=\active \def>{
\fontencoding{T1}\selectfont\symbol{62}\fontencoding{\encodingdefault}}
\newcommand{\assign}{:=}
\newcommand{\comma}{{,}}
\newcommand{\um}{-}
\newcommand{\sol}{\mathcal{S}ol(\R^{p,q};\lambda,\nu)}
\newcommand{\Op}{\mbox{\normalfont Op}}
\newcommand{\Res}{\operatorname{Res}\displaylimits}
\newcommand{\OpR}{\mbox{\it R}}

\newenvironment{setting}{\begin{exampleblock}{Setting.}\it}{\end{exampleblock}}
\newenvironment{question}{\begin{block}{Problem.}\it}{\end{block}}
%%\makeatletter
%%\newenvironment<>{proofs}[1][\proofname]{\par\def\insertproofname{#1\@addpunct{.}}\usebeamertemplate{proof begin}#2}
%%{\usebeamertemplate{proof end}}
%%\makeatother
%%
%%\makeatletter
%%\def\th@mystyle{%
%%	\normalfont % body font
%%	\setbeamercolor{block title example}{bg=orange,fg=white}
%%	\setbeamercolor{block body example}{bg=orange!20,fg=black}
%%	\def\inserttheoremblockenv{exampleblock}
%%}
%%\makeatother

\setbeamertemplate{theorem}[ams style]
\setbeamertemplate{theorem}[numbered]
\theoremstyle{mystyle}
\newtheorem{prop}{Proposition}
\theoremstyle{remark}
\newtheorem{remark}{Remark}
\newtheorem{goal}{Goal}[section]


\title[Symmetry breaking operators of $O(p,q)$]{Symmetry breaking operators of indefinite orthogonal groups $O(p,q)$}
\author[T. Kobayashi, A. Leontiev]{Toshiyuki Kobayashi, \underline{Alex Leontiev}}
\institute[Tokyo U]{
\inst{1}The University of Tokyo\\
Kavli Institute for the Physics and Mathematics of the Universe
           \and
           \inst{2}The University of Tokyo
}
\date[Inst. of Math., Acad. Sinica]{Number Theory Seminar, Institute of Mathematics, Academia Sinica}

\begin{document}
\begin{frame}\titlepage\end{frame}
%%\begin{frame}{Outline}
%%	\tableofcontents
%%\end{frame}
\begin{frame}{}
	\begin{center}
		\huge Part I: Introduction
	\end{center}
\end{frame}
\section{Prime factorization}
\begin{frame}{Prime factorization}
	Given any integer number $n$, \textbf{Prime Decomposition Theorem} tells us that it can be {\it uniquely factored} in \textbf{prime numbers}, say 
		\begin{equation*}
			23446456=2^3\cdot 11^1 \cdot 19^1\cdot 37^1\cdot 379^1.
		\end{equation*}
	\begin{problem}[Prime factorization]
		Given number $n$, determine it's {\bf prime factorization}.
	\end{problem}
	This problem is believed to be {\it difficult}. More precisely, its solution time believed to be {\it impossible} to bound by any polynomial. 

	\note{To begin with, I would like to talk about the problem that, I believe, is familiar to all the Number Theory people. That is, I would
	like to talk about the problem of prime factorization}.
\end{frame}
\begin{frame}{Multiplicities}
	We can, therefore, try to replace it with a (much simpler) ``relative'' version:
	\begin{problem}[Multiplicities]
		Given $n$ and prime $p$, determine the {\bf multiplicity} of $p$ in $n$. For example,
		\begin{equation*}
			23446456=2^{\fbox{3}}\cdot 11^{\fbox{1}} \cdot 19^{\fbox{1}}\cdot 37^{\fbox{1}}\cdot 379^{\fbox{1}}.
		\end{equation*}
	\end{problem}
\end{frame}
\section{Representation Theory}
\begin{frame}{Representation Theory}
	For compact groups, the situation is pretty much the same:
	\begin{center}
		\begin{tabular}[c]{llp{0.7\textwidth}}
			{``integers''}&$\leadsto$& {\it representations} \\&&(i.e. homomorphisms $\pi:G\to GL(V)$ for some vector space $V$)\\
			{``$m$ divides $n$''}&$\leadsto$&$\tau$ is a {\it subrepresentation} of $\pi$ \\&&(i.e. we have $G$-intertwining 1-1 map $A:\tau\xhookrightarrow{}\pi$)\\
			{``prime number''}&$\leadsto$&{\it irreducible} representations \\&&(e.g. representations with no proper nontrivial subrep's)
		\end{tabular}
	\end{center}
	\begin{remark}
		For compact groups, all irreducible representations are finitely-dimensional.
	\end{remark}
\end{frame}
\begin{frame}{Decomposition in irrep's}
	\begin{tabular}[c]{p{0.5\textwidth}lp{0.7\textwidth}}
			{``Prime decomposition theorem''}&$\leadsto$&
		\end{tabular}
		\begin{theorem}
			Given any finitely-dimensional representation $\pi$ of a compact group $G$, it has a unique decomposition:\begin{equation*}
				\begin{array}[]{c}
					\pi=\sum_{\tau\in\hat{G}}\tau^{\oplus m_{\pi}(\tau)},\\
					\hat{G}:=\left\{ \mbox{all (finitely-dimensional) irrep's of $G$} \right\},\\
					m_\pi(\tau)=\dim\Hom_{G}(\pi,\tau)=\dim\Hom_G(\tau,\pi).
				\end{array}
			\end{equation*}
		\end{theorem}
		\begin{tabular}[c]{p{0.5\textwidth}lp{0.7\textwidth}}
			{``multiplicity''}&$\leadsto$&$m_\pi(\tau)$
		\end{tabular}
\end{frame}
\begin{frame}{Branching Problem}
	Let $G$: compact group and $G'\subset G$ its subgroup. Note that if $\pi$ is an irreducible representation of $G$, its
	restriction to $G'$ $\pi\kern-0.1cm\mid_{G'}$ is {\it not necessarily} irreducible.
	\begin{problem}[Branching problem]
		Find {\bf branching rule}, which for every irrep $\pi$ of $G$ describes the decomposition
		\begin{equation*}
			\begin{array}[]{c}
			\pi\kern-0.1cm\mid_{G'}=\sum_{\tau\in\hat{G'}}\tau^{\oplus m_\pi(\tau)},\\
			m_\pi(\tau)=\dim\Hom_{G'}(\pi\kern-0.1cm\mid_{G'},\tau)=\dim\Hom_{G'}(\tau,\pi\kern-0.1cm\mid_{G'}).
			\end{array}
		\end{equation*}
	\end{problem}
\end{frame}
\begin{frame}{Kostant's Branching Theorem}
	For compact groups, recall that
	\begin{center}
	\begin{tabular}[]{lll}
		$\hat{G}$=&$\left\{ \mbox{irrep's of $G$} \right\}\simeq$&$\underbrace{\Lambda}_{\mbox{discrete set}}\subset \underbrace{\mathfrak{t}}_{\mbox{vect. sp., $\dim(\mathfrak{t})<\infty$}}$
		\kern-0.8cm:highest weights\\
		$\hat{G'}$=&$\left\{ \mbox{irrep's of $G'$} \right\}\simeq$&$\Lambda'\subset\mathfrak{s}$
	\end{tabular}
	\end{center}
	Now, the branching problem for compact groups is {\it completely solved} by:
	\begin{theorem}[Kostant's Theorem]
		Assume that $G$ and $G'$ are connected (+ some minor geometric assumption).
		Let $\pi$ be the irrep of $G$ with highest weight $\lambda$ and $\tau$ be the irrep of $G'$ with highest weight $\mu$. Then,
		\vspace{-0.3cm}
		\begin{equation*}
			\kern-0.2cm
			\begin{array}[]{c}
				m_\pi(\tau)=\sum_{w\in W_G}\varepsilon(w)\mathcal{P}\left( \sigma\left( w(\lambda+\delta_G) -\delta_G\right) -\mu\right),\\
				W_G: \mbox{Weyl group (finite group acting on $\mathfrak{t}$ and determined by $G$)},\\
				\delta_G\in \mathfrak{t},\quad \sigma:\mathfrak{t}\twoheadrightarrow\mathfrak{s},\quad \varepsilon:W\to\left\{ \pm1 \right\},\\
				\mathcal{P}:\Lambda'\to\N,\mbox{$\mathcal{P}(x)=$number of ways to write $x\in \mathfrak{s}$ as sum of elements of $\Lambda'$}
			\end{array}
		\end{equation*}
	\end{theorem}
\end{frame}
\begin{frame}{Noncompact groups}
	When $G$ and $G'$ are non-compact, the situation is much more difficult. In particular:\begin{enumerate}
		\item there are important infinitely-dimensional irreducible representations of $G,G'$ (cf. principal series rep's);
		\item it is no longer true in general that any irrep of $G$ decomposes into direct sum (integral) of those of $G'$;
		\item multiplicities $m_\pi(\tau)$ can be infinite for many $\tau$;
		\item {\it multiplicities $m_\pi(\tau)$ are no longer well-defined. In particular, in general $\dim\Hom_{G'}(\pi\kern-0.1cm\mid_{G'},\tau)\neq
			\dim\Hom_{G'}(\tau,\pi\kern-0.1cm\mid_{G'},)$}.
	\end{enumerate}
	Therefore, we settle with much more modest goal:
\end{frame}
\begin{frame}{Goal (1)}
	Therefore, we settle with much more modest goal:
	\begin{goal}[1]
		For noncompact group $G$, its noncompact subgroup $G'$ and their respectively irrep's $\pi$ and $\tau$, classify the space\begin{equation*}
			\Hom_{G'}(\pi\kern-0.1cm\mid_{G'},\tau)
		\end{equation*}
		of {\bf symmetry breaking operators} (SBO's, for short).
	\end{goal}
	It makes sense to take some concrete pair of groups for the first attempt. Perhaps, one of the simplest ``good'' pairs one can think of, is \begin{equation*}
		\left( G,G' \right)=(O(p+1,q+1),O(p,q+1)).
	\end{equation*}
	But even with a concrete pair chosen, it is too much work to classify SBOs for {\it all} $(\pi,\tau)$.
\end{frame}
\begin{frame}{Principal series}
	But even with a concrete pair chosen, it is too much work to classify SBOs for {\it all} $(\pi,\tau)$. Therefore, we restrict ourselves
	with {\bf principal series representations.}
	\begin{definition}[principal series representations]
		Let $P\subset G$ be minimal parabolic subgroup and $P=M A N$ be its Langlands decomposition (with $A\simeq \R^k$, where $k=\mbox{rank}(G)$). 
		For $(\sigma,V)$: irrep of $M$ and $\lambda\in\C^k$, define
		the representation of $P=MAN$\begin{equation*}
				\sigma_\lambda:P\ni p=m a n\mapsto \sigma(m)\exp(\myabra{\lambda,a})\in GL(V).
		\end{equation*}
		The {\bf principal series representation} $H^{\sigma,\lambda}$ of $G$ is defined as an induced representation\begin{equation*}
			\begin{array}[]{c}
				H^{\sigma,\lambda}=\Ind_P^G(\sigma_\lambda)\left( :={f\in C^\infty(G)\mid \forall(g,p)\in G\times P,f(p g)=\sigma_\lambda(p)^{-1}f(g)} \right),\\
			(g\cdot f)(\cdot)=f(\cdot g).
			\end{array}
		\end{equation*}
	\end{definition}
	\note{Principal series are good: on the one hand, they are big enough (Harish-Chandra subquotient theorem tells us that every irrep of $G$
	can be represented as a subquotient of some principal series). On the other hand, they are intimately related to the geometry of $G$.}
\end{frame}
\begin{frame}
	But even principal series representations are too big, so we use the {\bf spherical degenerate principal series}, where $\sigma$: irrep of $M$ is taken to be trivial
	and $P$ is take to be {\it maximal} parabolic, instead of {\it minimal}.
	\begin{goal}[2, final]
		For $(G,G')=(O(p+1,q+1),O(p,q+1))$ and all $(\lambda,\mu)\in\C^2$, we want to classify the vector space
		\vspace{-0.4cm}
		\begin{equation*}
			\Hom_{G'}(I(\lambda)\kern-0.1cm\mid_{G'},J(\nu)),\quad \left( I(\lambda):=\Ind_P^G(1_\lambda),J(\nu):=\Ind_{P'}^{G'}(1_\nu) \right).
		\end{equation*}
	\end{goal}
	\begin{remark}
		Previous works \cite{kobayashi2014classification}, \cite{kobayashi2013finite} imply that $\dim\Hom_{G'}(I(\lambda)\kern-0.1cm\mid_{G'},J(\nu))$ is
		bounded uniformly in $(\lambda,\mu)\in\C^2$.
	\end{remark}
	\vspace{-0.2cm}
	\begin{remark}
		In \cite{kobayashi2015symmetry} (published in Memoirs of AMS) the complete answer for these questions was obtained in the case $q=0$ (that is, $(G,G')=(O(n+1,1),O(n,1))$).
	\end{remark}
	\note{informal logic is that ``$P$ is bigger, hence induced rep'n is smaller''}
\end{frame}
\begin{frame}
	In my opinion, there are two main reasons why Kobayashi--Speh succeeded in giving the complete classification:\vspace{1em}

	First, they developed a general theory, that implies that under the geometric condition $P'N_-P=G$ we have\begin{equation*}
		Hom_{G'}(I(\lambda)\kern-0.1cm\mid_{G'},J(\nu))\simeq\underbrace{\sol}_{\mbox{solutions of some PDEs}}\subset\mathcal{D}'(\R^{p,q}).
	\end{equation*}
	But more on this later\ldots
\end{frame}
\begin{frame}
	Second, we have a very concrete geometric model for the degenerate principal series $I(\lambda),J(\nu)$:
	\newdir{:=}{{}}
	\begin{setting}
			%\xymatrixcolsep{5pc}
	\xymatrix{
		& \mathcal{L}_\lambda\mbox{ :conformally equivariant line bundle},\lambda\in\mathbb{C}
		\ar[d]\\
	G=O(p+1,q+1)
	\ar@/^2pc/[r] &G/P\simeq (\Sp^p\times\Sp^q)/\left\{ \pm I \right\}\\
	P=MAN\ar@{:=}[u]_{\hspace{-0.25cm}\bigcup}
	\ar@/^2pc/[rd]^{{\begin{array}{c}\; \\\mbox{conformal transformations}\end{array}}}
	%\mbox\newline oeueou}\vspace{0.8cm}}
	&\\
M_+N=O(p,q)\ltimes \mathbb{R}^{p,q}
\ar@{:=}[u]_{\hspace{-0.25cm}\bigcup}
\ar@/^2pc/[r]^{\mbox{isometries}}&
\mathbb{R}^{p,q}=\left( \mathbb{R}^{p+q},ds^2=dx_1^2+\ldots+dx_p^2-dx_{p+1}^2-\ldots-dx_{p+q}^2 \right)\ar@{^{(}->}[uu]
_{\mbox{conformal 
compactification}}
\vspace{2cm}
	}

	\end{setting}
%%	\begin{setting}
%%		\[\begin{array}{ccc}
%%		G:=O(p+1,q+1)&\curvearrowright &I(\lambda):=C^\infty(G/P,\mathcal{L}_\lambda)\\
%%		\bigcup&&\\
%%		G':=O(p,q+1)&\curvearrowright &J(\nu):=C^\infty(G'/P',\mathcal{L}_\nu)
%%		\end{array}\]
%%	\end{setting}
\end{frame}
%%%%%%%%%%%	PART 2
\begin{frame}{}
	\begin{center}
		\huge Part 2: Main Results
	\end{center}
\end{frame}
\begin{frame}
	\begin{fact}[\cite{kobayashi2015symmetry}]
    Let $n:=p+q$. The following diagram commutes:
\begin{figure}[H]
	\centerline{
		\xymatrixcolsep{3pc}
		\xymatrix{\Hom_{G'}(I(\lambda),J(\nu))\ar[r]^{\simeq} \ar@/^2pc/[rr]^{\mathcal{S}}
		&\left( \mathcal{D}'(G/P,\mathcal{L}_{n-\lambda}) \otimes\mathbb{C}_\nu \right)^{P'}
	\ar[r]_-{F\mapsto \supp(F)}\ar[d]^{\simeq}_{\mbox{rest}}
	&2^{P'\backslash G/P}\\
	&\sol\subset\mathcal{D}'(\R^{p,q})\ar[lu]^{\mbox{Op}}_{\simeq}&
	}
}
\end{figure}
\end{fact}
\end{frame}
\begin{frame}
Note that $G$ acts on $\Xi^{p+1,q+1}:=\mysetn{(x,y)\in\R^{p+1,q+1}\setminus\left\{ 0 \right\}}{\myabs{x}^2=\myabs{y}^2}$ and on its quotient space
$X^{p,q}:=\Xi^{p+1,q+1}/\R^{\times}\simeq G/P$. Let
\[
	X:=G/P\simeq X^{p,q},\quad Y:=\mysetn{[\xi:\eta]\in G/P\simeq X^{p,q}}{\xi_{p}=0}\simeq X^{p-1,q}\]
	\[\kern-0.32cm C:=\mysetn{[\xi:\eta]\in G/P\simeq X^{p,q}}{\xi_{0}=\eta_q}\simeq X^{p-1,q-1}\cup\Xi^{p,q},\left\{ [0] \right\}:=\left\{ [1,0_{p+q},1] \right\}\]
\begin{theorem}[classification of closed $P'$-invariant subsets of $G/P$]
	The left $P'$-invariant closed subspaces of $G/P$ are as follows (numbers indicate codimension):\\
	\vspace{-1em}
  \begin{figure}[H]
    \centering
    \begin{subfigure}[t]{0.3\textwidth}
	    \xymatrixrowsep{0.5pc}
	    \xymatrix{&X\ar@{-}[ld]_1\ar@{-}[rd]^1&\\Y\ar@{-}[rd]_1&&C\ar@{-}[ld]^1\\&Y\cap C\ar@{-}[dd]^{p+q-2}&\\&&\\&\{[0]\}&}
	\caption{when $p>1$}
    \end{subfigure}
    ~ %add desired spacing between images, e. g. ~, \quad, \qquad, \hfill etc. 
      %(or a blank line to force the subfigure onto a new line)
    \begin{subfigure}[t]{0.3\textwidth}
	    \xymatrixrowsep{0.5pc}
	    {\xymatrix{&X\ar@{-}[ld]_1\ar@{-}[rd]^1&\\Y\ar@{-}[rddd]_{p+q-2}&&C\ar@{-}[lddd]^{p+q-2}\\&&\\&&\\&\{[0]\}&}}
	    \vspace{0.6em}
	\caption{when $p=1$}
    \end{subfigure}
\end{figure}
\end{theorem}
\end{frame}
\begin{frame}
	\begin{theorem}
		We can construct the following families of SBOs which holomorphically depend on parameters:
		\vspace{-0.9em}
		\renewcommand{\arraystretch}{3.5}
\scriptsize
\begin{center}
\begin{tabular}{|c|c|c|c|}
  \hline
  & $\tmop{Op}:\Sol(\mathbb{R}^{p,q};\lambda,\nu)\to\Hom_{G'}(I(\lambda),J(\nu))$ & defined for &
  $\mathcal{S} (\cdot)$ (generically)\\
  \hline
  $R_{\lambda, \nu}^X=$ & $\Op\left( \frac{| x_p |^{\lambda + \nu - n} | Q |^{-
  \nu}}{\Gamma \left( \frac{\lambda - \nu}{2} \right) \Gamma \left(
  \frac{\lambda + \nu - n + 1}{2} \right) \Gamma \left( \frac{1 - \nu}{2}
  \right)} \right)$ & $(\lambda,\nu)\in\mathbbm{C}^2$ & $X$\\
  \hline
  $\tilde{R}^X_{\lambda, \nu}=$ & $\Op\left(  \frac{| x_p |^{\lambda + \nu - n} | Q |^{-
  \nu}}{ \Gamma \left(
  \frac{\lambda + \nu - n + 1}{2} \right) \Gamma \left( \frac{1 - \nu}{2}
  \right)} \right)$ & $(\lambda,\nu)\in\mid \mid \mid$ & $X$\\
  \hline
  $R_{\lambda, \nu}^{\{ o \}}=$ & $\Op\left(  \tilde{C}_{\nu - \lambda}^{\lambda - \frac{n
  - 1}{2}} ({\Delta}_{\mathbb{R}^{p-1,q}} {\delta}_{\mathbb{R}^{p+q-1}}, \delta (x_p)) \right)$ & $(\lambda,\nu)\in/ /$ & $\{ [0]
  \}$\\
  \hline
\end{tabular}
\end{center}
\vspace{-1em}
\begin{itemize}
	\item $\mid \mid \mid \assign \{ (\lambda, \nu) \in \mathbbm{C}^2 \mid \nu \in
	- 2\mathbbm{Z}_{\geqslant 0} \cup (q + 1 + 2\mathbbm{Z}) \}$ \item $/ / \assign
\{ (\lambda, \nu) \in \mathbbm{C}^2 \mid \lambda - \nu \in
2\mathbbm{Z}_{\leqslant 0} \}$;
\item $Q:=x_1^2+\cdots+x_p^2-x_{p+1}^2-\cdots-x_{p+q}^2$;
\item $\tilde{C}(s,t)$ is a two-variable inflation of renormalized Gegenbauer polynomial, defined as in \cite{kobayashi2015symmetry}.
\end{itemize}

	\end{theorem}
\end{frame}
\begin{frame}{Classification of SBO}
	\begin{theorem}
		We can find basis for $\Hom_{G'}(I(\lambda),J(\nu))$ for every $(\lambda,\nu)\in \mathbb{C}^2$. In particular, for $p>1$ we have
  \begin{eqnarray}
	  & \Hom_{G'}(I(\lambda),J(\nu))= \left\{
    \begin{array}{ll}
	    \mathbbm{C} {\tilde{\OpR}}_{\lambda, \nu}^{X} \oplus \mathbbm{C}
      {\OpR}^{\{ 0 \}}_{\lambda, \nu}, & (\lambda, \nu) \in / /\cap 
      \mid\mid\mid \\
      \mathbbm{C} \OpR^X_{\lambda, \nu}, &
      \mbox{\normalfont otherwise.}
    \end{array} \right. &  \nonumber
  \end{eqnarray}
  Note that $//\cap\mid\mid\mid\subset\mathbb{C}^2$ is a countable discrete subset.
		\label{}
	\end{theorem}
	\begin{corollary}
  \begin{eqnarray}
	  & \dim\Hom_{G'}(I(\lambda),J(\nu))= \left\{
    \begin{array}{ll}
	    2, & (\lambda, \nu) \in / /\cap 
      \mid\mid\mid \\
      1, &
      \mbox{\normalfont otherwise.}
    \end{array} \right. &  \nonumber
  \end{eqnarray}
	\end{corollary}
\end{frame}
\begin{frame}{Spherical multiple}
	\begin{theorem}
		Let $1_\lambda\in C^\infty(G/P,\mathcal{L}_\lambda)^K,1_\nu\in C^\infty(G'/P',\mathcal{L}_\nu)^{K'}$ be the spherical vectors. We then have
	\[ \OpR^X_{\lambda, \nu} 1_{\lambda} = 2^{1 -
     \lambda} \frac{\pi^{n / 2}}{\Gamma \left( \frac{\lambda}{2} \right)
     \Gamma \left(  \frac{\lambda + 1-q}{2} \right) \Gamma \left(
     \frac{q - \nu + 1}{2} \right)} 1_{\nu}. \]
	\end{theorem}
	\begin{remark}
		This result was previously obtained for $p=q=1$ case in \cite{bernstein2004estimates}.
	\end{remark}
\end{frame}
\begin{frame}{Residue theorem}
	\begin{theorem}
		The distribution
		\[K_{\lambda,\nu}^{\mathbb{R}^{p,q}}:=\frac{\myabs{x_p}^{\lambda+\nu-n}}{\Gamma\left( \frac{\lambda+\nu-n+1}{2} \right)}\times
		\frac{\myabs{Q}^{-\nu}}{\Gamma\left( \frac{1-\nu}{2} \right)}\]
		has the pole at $(\lambda,\nu)\in//$ with the residue given by
		{\footnotesize
		\[\Res_{(\lambda,\nu)\in//}K_{\lambda,\nu}^{\mathbb{R}^{p,q}}=\frac{K_{\lambda,\nu}^{\mathbb{R}^{p,q}}}{\Gamma\left( \frac{\lambda-\nu}{2} \right)}
			=\frac{ (- 1)^k k!\pi^{(n - 2) / 2} 
		}{2^{ \nu + 2 k-1}}\cdot  \frac{\sin\left( \frac{1+q-\nu}{2}\pi \right)}{\Gamma\left( \frac{\nu}{2} \right)}
	\tilde{C}_{\nu - \lambda}^{\lambda - \frac{n
  	- 1}{2}} ({\Delta}_{\mathbb{R}^{p-1,q}} {\delta}_{\mathbb{R}^{p+q-1}}, \delta (x_p))
		\]
		}
		where $k:=\frac{\nu-\lambda}{2}$.
		Hence taking $\Op(\cdot)$ on both sides we get
  \[\OpR_{\lambda,\nu}^X  = \frac{ (- 1)^k k!\pi^{(n - 2) / 2} 
		}{2^{ \nu + 2 k-1}}\cdot  \frac{\sin\left( \frac{1+q-\nu}{2}\pi \right)}{\Gamma\left( \frac{\nu}{2} \right)}
     \OpR_{\lambda,\nu}^{ \left\{ 0 \right\} },\quad(\lambda,\nu)\in// . \]
	\end{theorem}
\end{frame}
\begin{frame}{Functional identities}
	\begin{fact}[Knapp-Stein operator]
		There is a nontrivial $G$-invariant operator $\tilde{\mathbb{T}}_{\lambda}:I(\lambda)\to I(n-\lambda)$ which holomorphically depends on $\lambda\in \mathbb{C}$.
	\end{fact}
	\begin{theorem}
		Let $n':=n-1$. Then the following holds:
\begin{eqnarray}
	& \tilde{\mathbbm{T}}_{n' - \nu} \circ \OpR^X_{\lambda,n'-\nu}= q^{T\mathbbm{R}^n}_{\mathbbm{R}^n} (\lambda,
  \nu) \OpR^X_{\lambda,\nu} &  \nonumber\\
  & \OpR^X_{n-\lambda,\nu} \circ
  \tilde{\mathbbm{T}}_{\lambda} = q^{\mathbbm{R}^n T}_{\mathbbm{R}^n}
  (\lambda, \nu) \OpR^X_{\lambda,\nu} & \nonumber
\end{eqnarray}

\setbeamercovered{transparent}
\pause
Here, for example
\begin{equation*}
	\kern-0.1cm
  q^{\mathbbm{R}^n T}_{\mathbbm{R}^n} (\lambda, \nu) \assign \frac{2^{2
  \lambda - n} \pi^{- n / 2}}{\Gamma \left( \frac{n - \lambda}{2} \right)}
  \cdot \frac{\sin \left[ \pi \frac{p - \lambda + 1}{2} \right]}{\pi} \cdot
  \left\{ \begin{array}{l@{}l@{}}
    2^{1 - \lambda} \sqrt{\pi}, & n \in 2\mathbbm{Z}+ 1\\
    \Gamma \left( \frac{\lambda - n / 2 + 1}{2} \right), & n / 2 + p \in
    2\mathbbm{Z}\\
    \Gamma \left( \frac{\lambda - n / 2}{2} \right), & n / 2 + p \in
    2\mathbbm{Z}+ 1.
  \end{array} \right.
\end{equation*}
	\end{theorem}
\end{frame}
\begin{frame}
	{\footnotesize\bibliographystyle{apalike}
\bibliography{todai_master}}
\end{frame}
%%\begin{frame}
%%	\begin{setting}
%%		\begin{enumerate}
%%			\item 
%%				Let $p,q\ge1$, $G:=O(p+1,q+1)$ and $G':=O(p+1,q+1)_{e_{p+1}}\simeq O(p,q+1)$. 
%%			\item
%%				Let $P:=MAN$ and $P':=G'\cap P=M'AN'$ :max parabolic, where\\
%%\newcommand{\longminus}{\textemdash\textemdash}
%%\hspace{-1.05cm}  \begin{tabular}{l@{}}
%%    $N \assign \left\{ \left[ \begin{array}{lll}
%%      1 - Q & -^t w' & Q\\
%%      w & I_{p + q} & - w\\
%%      - Q & -^t w' & 1 + Q
%%    \end{array} \right] \middle| \begin{array}{c}
%%      (x, y) \in \mathbbm{R}^{p, q}\\
%%      w \assign (x, y)\\
%%      w' \assign (x, - y)\\
%%      Q \assign \frac{| x |^2 - | y |^2}{2}
%%    \end{array} \right\}$, $M \assign \left\{ \left[ \begin{array}{ccc}
%%      \epsilon & 0 & 0\\
%%      0 & A & 0\\
%%      0 & 0 & \epsilon
%%    \end{array} \right] \middle| \begin{array}{c}
%%      A \in O (p, q)\\
%%      \epsilon = \pm 1
%%    \end{array} \right\}$\\
%%    $N' \assign \left\{ \longminus \longminus \longminus'' \longminus
%%    \longminus \longminus \middle| \begin{array}{c}
%%      \longminus'' \longminus\\
%%      x_p = 0
%%    \end{array} \right\}$, $M' \assign \left\{ \longminus'' \longminus
%%    \middle| \begin{array}{c}
%%      \longminus'' \longminus\\
%%      A e_p = e_p
%%    \end{array} \right\}$\\
%%    $A \assign a (\mathbbm{R})$, \quad$a (t) \assign \left[ \begin{array}{ccc}
%%      \cosh (t) & 0 & \sinh (t)\\
%%      0 & I_{p + q} & 0\\
%%      \sinh (t) & 0 & \cosh (t)
%%    \end{array} \right]$
%%  \end{tabular}
%%			\item For $(\lambda,\nu)\in\mathbb{C}^2$ we let $I(\lambda),J(\nu)$ to be the degenerate principal series of $G,G'$ respectively, i.e.
%%$I(\lambda):=\{f\in C^\infty(G)\mid \forall h\in P,\;f(\cdot h)=\lambda(h) f(\cdot)\}$, where $\lambda:P\ni m\cdot a(t)\cdot n\mapsto e^{-\lambda t}$ is a $P$-representation and
%%with $G$-action by left multiplication, and similarly for $J(\nu)$.
%%		\end{enumerate}
%%	\end{setting}
%%\end{frame}
\end{document}
