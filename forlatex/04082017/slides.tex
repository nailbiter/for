\documentclass[pdf]{beamer}
\mode<presentation>{\usetheme[secheader]{Boadilla}}
\usepackage{mystyle}
\usepackage{amsthm,amssymb}
\usepackage{mathtools}
\usepackage{framed}
\includecomment{versiona}
\usepackage{mystyle}
\usepackage{geometry}
\usepackage{amsmath}
\usepackage{ruby}
\usepackage{enumerate}
\usepackage{setspace}
\usepackage{xypic}
\usepackage[all,cmtip]{xy}
\usepackage{bbm,ulem,float,mystyle}
\usepackage{caption}
\usepackage{subcaption}
\usepackage{setspace}
\usepackage{tikz}
\usepackage{tikz-cd,array}
\usepackage{catchfilebetweentags}
\usepackage{textcomp}
\usepackage{etoolbox}
\patchcmd{\thebibliography}{\section*{\refname}}{}{}{}
\usepackage{pifont}
\usepackage{scalerel}
\usepackage{adjustbox}
\usepackage{tcolorbox}
\usepackage{lipsum}
\usepackage{calc}
\usepackage{tikz, pgfplots}
\pgfplotsset{width=8cm,compat=1.9}
\usetikzlibrary{pgfplots.dateplot}
\usepackage{pgfplotstable}
\usepackage{graphicx}
\usepackage{amsmath}
\usepackage{amssymb}
\usepackage{relsize}
\usepackage{multirow}
\usepackage{rotating}
\usepackage{bm}
\usepackage{url}
\usepackage{mystyle}
\usepackage{enumerate}
\usepackage{geometry}
\usepackage{setspace}
\usepackage{amsmath,amssymb,bbm,xypic}
\usepackage[all,cmtip]{xy}
\usepackage{amsmath,amssymb,bbm,float,mystyle}
\usepackage{stmaryrd}

\newcommand{\red}[1]{{\color[rgb]{0.6,0,0}#1}}
%\setbeameroption{show only notes}

%%\newcommand{\red}[1]{{\color[rgb]{0.6,0,0}#1}}
\newcommand{\Sol}{\mathcal{S}\mbox{ol}}
\newcommand{\D}{\mathcal{D}}
\newcommand{\A}{\mathcal{A}}
\newcommand{\Co}{\mathbb{C}}
\newcommand{\X}{\mathbb{X}}
\renewcommand{\setminus}{\backslash}
\newcommand{\nin}{\not\in}
\newcommand{\Ind}{\ensuremath{\operatorname{Ind}}}
\newcommand{\tmop}[1]{\ensuremath{\operatorname{#1}}}
\newcommand{\tmtextbf}[1]{{\bfseries{#1}}}
\newcommand{\tmtextit}[1]{{\itshape{#1}}}
\newcommand{\mss}{//}
\newcommand{\mbb}{\backslash\backslash}
\newcommand{\mmm}{\mid\mid}
\catcode`\<=\active \def<{
\fontencoding{T1}\selectfont\symbol{60}\fontencoding{\encodingdefault}}
\catcode`\>=\active \def>{
\fontencoding{T1}\selectfont\symbol{62}\fontencoding{\encodingdefault}}
\newcommand{\assign}{:=}
\newcommand{\comma}{{,}}
\newcommand{\um}{-}
\newcommand{\sol}{\mathcal{S}ol(\R^{p,q};\lambda,\nu)}
\newcommand{\Op}{\mbox{\normalfont Op}}
\newcommand{\Res}{\operatorname{Res}\displaylimits}
\newcommand{\OpR}{\mbox{\it R}}

\newcommand{\tmtextmd}[1]{{\mdseries{#1}}}
\newcommand{\tmtextrm}[1]{{\rmfamily{#1}}}
\newcommand{\tmtextup}[1]{{\upshape{#1}}}

\newenvironment{setting}{\begin{exampleblock}{Setting.}\it}{\end{exampleblock}}
\newenvironment{question}{\begin{block}{Problem.}\it}{\end{block}}
%%\makeatletter
%%\newenvironment<>{proofs}[1][\proofname]{\par\def\insertproofname{#1\@addpunct{.}}\usebeamertemplate{proof begin}#2}
%%{\usebeamertemplate{proof end}}
%%\makeatother
%%
%%\makeatletter
%%\def\th@mystyle{%
%%	\normalfont % body font
%%	\setbeamercolor{block title example}{bg=orange,fg=white}
%%	\setbeamercolor{block body example}{bg=orange!20,fg=black}
%%	\def\inserttheoremblockenv{exampleblock}
%%}
%%\makeatother

\setbeamertemplate{theorem}[ams style]
\setbeamertemplate{theorem}[numbered]
\theoremstyle{mystyle}
\newtheorem{prop}{Proposition}
\theoremstyle{remark}
\newtheorem{remark}{Remark}
\newtheorem{goal}{Goal}[section]


\title[Symmetry breaking operators of $O(p,q)$]{Symmetry breaking operators of indefinite orthogonal groups $O(p,q)$}
\author[T. Kobayashi, A. Leontiev]{Toshiyuki Kobayashi, \underline{Alex Leontiev}}
\institute[Tokyo U]{
\inst{1}The University of Tokyo\\
Kavli Institute for the Physics and Mathematics of the Universe
           \and
           \inst{2}The University of Tokyo
}
\date[Inst. of Math., Acad. Sinica]{Number Theory Seminar, Institute of Mathematics, Academia Sinica}

\begin{document}
\begin{frame}\titlepage
	\note{
		Good morning, everyone. My name is Alex Leontiev, I am from The University of Tokyo.
		Thank you for letting me talk at today's Number Theory Seminar at Academia Sinica. This is a big honour for me.

		The topic of my today's talk is ``Symmetry breaking operators of indefinite orthogonal groups $O(p,q)$''.
		This is the result of the joint research with my advisor prof. Toshiyuki Kobayashi from The University of Tokyo.
	}
\end{frame}
%%\begin{frame}{Outline}
%%	\tableofcontents
%%\end{frame}
\section{Representation Theory of Compact groups}
\begin{frame}{}
	\begin{center}
		\huge Part I: Introduction
	\end{center}
	\note{There will be three parts in this talk: Part I, where I will try to explain why one is interested in studying symmetry breaking operators.
	Part II, where I will outline the results we have at this moment. And finally, Part III, where I will very briefly explain
	where one can go from here and what we plan to do now.
	If I am not mistaken, the talk's duration is one hour, which gives me plenty of time, so please do not hesitate to ask questions
	during the lecture.
}
\end{frame}
\begin{frame}{Prime factorization}
	Given any integer number $n$, \textbf{unique factorization theorem} tells us that it can be {\it uniquely factored} in \textbf{prime numbers}, say 
		\begin{equation*}
			23446456=2^3\cdot 11^1 \cdot 19^1\cdot 37^1\cdot 379^1.
		\end{equation*}
	\begin{problem}[Prime factorization]
		Given number $n$, determine it's {\bf prime factorization}.
	\end{problem}
	This problem is believed to be {\it difficult}. More precisely, its solution time believed to be {\it impossible} to bound by any polynomial. 

	\note{To begin with, I would like to talk about the problem that, I believe, is familiar to all the Number Theory people. 
		That is, I would
		like to talk about the problem of prime factorization. First of all, Prime Number Theorem tells us that every number can be
		decomposed uniquely (neglecting order and sign of terms) into the product of so-called prime numbers. The latter, in turn
		are characterized by the property that they cannot be decomposed further.

		Now, despite being so easily accessible, the problem is believed to be difficult when number $n$ gets big. In particular,
		it is believed to require super-polynomial time in number of digits. In other words, the time complexity of prime factorization
		cannot be uniformly bounded by any polynomial in the number of digits.

		As a side note, this fact has wide applications in cryptography. Most of the asymmetric encryption schemes in use now are build
		on top of this idea.
	}
\end{frame}
\begin{frame}{Multiplicities}
	We can, therefore, try to replace it with a (much simpler) ``relative'' version:
	\begin{problem}[Multiplicities]
		Given $n$ and prime $p$, determine the {\bf multiplicity} of $p$ in $n$. For example,
		\begin{equation*}
			23446456=2^{\fbox{3}}\cdot 11^{\fbox{1}} \cdot 19^{\fbox{1}}\cdot 37^{\fbox{1}}\cdot 379^{\fbox{1}}.
		\end{equation*}
	\end{problem}
	\note{
		On the other hand, the problem of finding multiplicity of a given prime in a given number is considerably simple: it can
		be easily solved by consecutive division, and runs in polynomial time. Finding multiplicities is much simpler than computing
		the whole decomposition.
	}
\end{frame}
\begin{frame}{Representation Theory}
	For compact groups, the situation is pretty much the same:
	\begin{center}
		\begin{tabular}[c]{llp{0.7\textwidth}}
			{``integers''}&$\leadsto$& {\it representations} \\&&(i.e. homomorphisms $\pi:G\to GL(V)$ for some vector space $V$)\\
			{``$m$ divides $n$''}&$\leadsto$&$\tau$ is a {\it subrepresentation} of $\pi$ \\&&(i.e. we have $G$-intertwining 1-1 map $A:\tau\xhookrightarrow{}\pi$)\\
			{``prime number''}&$\leadsto$&{\it irreducible} representations \\&&(e.g. representations with no proper nontrivial subrep's)
		\end{tabular}
	\end{center}
	\begin{remark}
		For compact groups, all irreducible representations are finitely-dimensional.
	\end{remark}
	\note{Going now to the representation theory, let us first about the representation theory of compact groups, where things
	are pretty much similar.

	Namely, fixing a particular compact group $G$ for now, integers may be seen as corresponding to its representations,
	namely to the actions of $G$ on a vector space $V$ by linear transformations.
	The notion of divisibility
	corresponds to that of being a subrepresentation, that is when one representation may be seen as an invariant subspace of another.
	Prime numbers, that is numbers that cannot be decomposed further, correspond to an irreducible representations, that is representations
	with no nontrivial subrepresentations.

	It is important to note at this point, that for compact groups all irreducible representations are necessarily finitely-dimensional,
	which is a particularly nice feature, not shared by non-compact groups.
}
\end{frame}
\begin{frame}{Decomposition in irreducibles}
	\begin{tabular}[c]{p{0.5\textwidth}lp{0.7\textwidth}}
			{``unique factorization theorem''}&$\leadsto$&
		\end{tabular}
		\begin{theorem}
			Given any finitely-dimensional representation $\pi$ of a compact group $G$, it has a unique decomposition:\begin{equation*}
				\begin{array}[]{c}
					\pi=\sum_{\tau\in\hat{G}}\tau^{\oplus m_{\pi}(\tau)},\\
					\hat{G}:=\left\{ \mbox{all (finitely-dimensional) irrep's of $G$} \right\},\\
					m_\pi(\tau)=\dim\Hom_{G}(\pi,\tau)=\dim\Hom_G(\tau,\pi).
				\end{array}
			\end{equation*}
		\end{theorem}
		\begin{tabular}[c]{p{0.5\textwidth}lp{0.7\textwidth}}
			{``multiplicity''}&$\leadsto$&$m_\pi(\tau)$
		\end{tabular}
		\note{For compact groups, we have the analogue of unique factorization theorem, which
		tells us that every finitely-dimensional representation of $G$ splits into a direct sum of the irreducibles.
		Similarly to the prime decomposition, we have a notion of multiplicity, that is how many times given irreducible appears in the
		decomposition.

		Note that Schur lemma tells us that
		multiplicities are equal to the dimension of $G$-intertwining operators from $\pi$ to $\tau$ or from $\tau$ to $\pi$.
		}
\end{frame}
\begin{frame}{Branching Problem}
	Let $G$: compact group and $G'\subset G$ its subgroup. Note that if $\pi$ is an irreducible representation of $G$, its
	restriction to $G'$ $\pi\kern-0.1cm\mid_{G'}$ is {\it not necessarily} irreducible.
	\begin{problem}[Branching problem]
		Find {\bf branching rule}, which for every irrep $\pi$ of $G$ describes the decomposition
		\begin{equation*}
			\begin{array}[]{c}
			\pi\kern-0.1cm\mid_{G'}=\sum_{\tau\in\hat{G'}}\tau^{\oplus m_\pi(\tau)},\\
			m_\pi(\tau)=\dim\Hom_{G'}(\pi\kern-0.1cm\mid_{G'},\tau)=\dim\Hom_{G'}(\tau,\pi\kern-0.1cm\mid_{G'}).
			\end{array}
		\end{equation*}
	\end{problem}
	\note{
		Now it is the time to recall that we are talking about the groups, so we can do something we cannot quite do with integers:
		we can switch the group and this will change the notion of irreducibility. Namely, if we take an irreducible representation of compact
		group $G$ and then consider it as a representation of its subgroup $G'$, it may well be no longer irreducible. Hence, the previous
		theorem implies that it decomposes into the direct sum of irreducibles of $G'$ and people since long ago were interested
		in how exactly it decomposes. The exact rule that explains the decomposition pattern is called ``branching rule'' and they have
		important applications. Quantum mechanics and chemistry, to name a few.

		Note that in this case multiplicities are equal to the dimension of $G'$-intertwining operators. Again, we can
		take either operators from big representation to a small one, or the other way around.
	}
\end{frame}
\begin{frame}{Kostant's Branching Theorem}
	For compact groups, recall that
	\vspace{-1em}
	\begin{center}
	\begin{tabular}[]{lll}
		$\hat{G}$=&$\left\{ \mbox{irrep's of $G$} \right\}\simeq$&$\underbrace{\Lambda}_{\mbox{discrete set}}\subset \underbrace{\mathfrak{t}}_{\mbox{vect. sp., $\dim(\mathfrak{t})<\infty$}}$
		\kern-0.8cm:highest weights\\
		$\hat{G'}$=&$\left\{ \mbox{irrep's of $G'$} \right\}\simeq$&$\Lambda'\subset\mathfrak{s}$
	\end{tabular}
	\end{center}
	Now, the branching problem for compact groups is {\it completely solved} by:
	\begin{theorem}[Kostant's Theorem]
		Assume that $G$ and $G'$ are connected (+ some minor geometric assumption).
		Let $\pi$ be the irrep of $G$ with highest weight $\lambda$ and $\tau$ be the irrep of $G'$ with highest weight $\mu$. Then,
		\vspace{-0.3cm}
		\begin{equation*}
			\kern-0.2cm
			\begin{array}[]{c}
				m_\pi(\tau)=\sum_{w\in W_G}\varepsilon(w)\mathcal{P}\left( \sigma\left( w(\lambda+\delta_G) -\delta_G\right) -\mu\right),\\
				W_G: \mbox{Weyl group (finite group acting on $\mathfrak{t}$ and determined by $G$)},\\
				\delta_G\in \mathfrak{t},\quad \sigma:\mathfrak{t}\twoheadrightarrow\mathfrak{s},\quad \varepsilon:W\to\left\{ \pm1 \right\},\\
				\mathcal{P}:\Lambda'\to\N,\mbox{$\mathcal{P}(x)=$number of ways to write $x\in \mathfrak{s}$ as sum of elements of $\Lambda'$}
			\end{array}
		\end{equation*}
	\end{theorem}
	\note{
		Luckily, for compact groups we have a general rule, named Kostant's Branching Theorem, which tells us how multiplicities can
		be computed in a combinatorial way. 
		
		As a side note, for particular pairs of groups, say $O(n)$ to $O(n-1)$ restriction of orthogonal groups
		, $U(n)$ to $U(n-1)$ restriction of unitary groups and $Sp(n)$ to $Sp(n-1)$ restriction of symplectic groups much simpler
		combinatorial rules are known respectively as Weyl rule, Murnaghan rule and Zhelobenko rule.
	}
\end{frame}
\section{Representation theory of non-compact groups}
\begin{frame}{Noncompact groups}
	When $G$ and $G'$ are non-compact, the situation is much more difficult. In particular:\begin{enumerate}
		\item there are important infinitely-dimensional irreducible representations of $G,G'$ (cf. principal series representations);
		\item it is no longer true in general that any irreducible of $G$ decomposes into the direct sum (integral) of those of $G'$;
		\item multiplicities $m_\pi(\tau)$ can be infinite for many $\tau$, even when $G'$ is maximal in $G$;
		\item {\it multiplicities $m_\pi(\tau)$ are no longer well-defined. In particular, in general $\dim\Hom_{G'}(\pi\kern-0.1cm\mid_{G'},\tau)\neq
			\dim\Hom_{G'}(\tau,\pi\kern-0.1cm\mid_{G'},)$}.
	\end{enumerate}
	Therefore, we settle with much more modest goal:
	\note{
		As much as everything is nice and clean with compact groups, it gets much worse when groups are noncompact.
		First of all, now we have important families of irreducible representations which are infinitely-dimensional.
		Second, in some situations the representation of $G$ may not split as a direct sum or integral of irreducibles of $G'$.
		Third, even if they do split, the multiplicities can get infinite for considerably many of irreducibles of $G'$.
		Finally, the multiplicities itself become ill-defined. Even in situations when splitting does not occur, following
		the hints from the previous slides, one might hope to define the multiplicities as the dimensions of intertwining operators.
		Nevertheless, in this general setting the operators from big to small representation and from small to big no longer correspond to each
		other. It is thought that the dimension of $G'$ intertwining operators from the irreducible of $G$ to the irreducible of $G'$
		is somehow a better definition, so we will pursue it.
	}
\end{frame}
\begin{frame}{Goal (1)}
	Therefore, we settle with much more modest goal:
	\begin{goal}[1]
		For noncompact group $G$, its noncompact subgroup $G'$ and their respectively irrep's $\pi$ and $\tau$, classify the space\begin{equation*}
			\Hom_{G'}(\pi\kern-0.1cm\mid_{G'},\tau)
		\end{equation*}
		of {\bf symmetry breaking operators} (SBO's, for short).
	\end{goal}
	It makes sense to take some concrete pair of groups for the first attempt. Perhaps, one of the simplest ``good'' pairs one can think of, is \begin{equation*}
		\left( G,G' \right)=(O(p+1,q+1),O(p,q+1)).
	\end{equation*}
	But even with a concrete pair chosen, it is too much work to classify SBOs for {\it all} $(\pi,\tau)$.
	\note{
		Therefore, as a reasonable Goal, one might try to classify all the intertwining operators between the irreducible of $G$
		and that of its subgroup $G'$, or, as we call them, symmetry breaking operators, or SBOs for brevity.
		The pair $(O(p+1,q+1),O(p,q+1))$ provides a good first choice, as it has the same complexification as perhaps the simplest
		pair of compact groups: the pair of orthogonal groups.

		But even with this specialization, we still need to take some particular class of irreducibles to work with.
	}
\end{frame}
\begin{frame}{Principal series}
	But even with a concrete pair chosen, it is too much work to classify SBOs for {\it all} $(\pi,\tau)$. Therefore, we restrict ourselves
	with {\bf principal series representations.}
	\begin{definition}[principal series representations]
		Let $P\subset G$ be minimal parabolic subgroup and $P=M A N$ be its Langlands decomposition (with $A\simeq \R^k$, where $k=\mbox{rank}(G)$). 
		For $(\sigma,V)$: irrep of $M$ and $\lambda\in\C^k$, define
		the representation of $P=MAN$\begin{equation*}
				\sigma_\lambda:P\ni p=m a n\mapsto \sigma(m)\exp(\myabra{\lambda,a})\in GL(V).
		\end{equation*}
		The {\bf principal series representation} $H^{\sigma,\lambda}$ of $G$ is defined as an induced representation\begin{equation*}
			\begin{array}[]{c}
				\kern-0.3cm H^{\sigma,\lambda}=\kern-0.1cm\Ind_P^G(\sigma_\lambda)\kern-0.1cm\left( :=\kern-0.1cm\left\{f\in C^\infty(G)\mid \forall(g,p)\in G\times P,f(p g)=\sigma_\lambda(p)^{-1}f(g)\right\} \right)\\
			(g\cdot f)(\cdot)=f(\cdot g).
			\end{array}
		\end{equation*}
	\end{definition}
	\note{
		We may wish to restrict ourselves to principal series, which are important families of representations, defined
		as an induction from minimal parabolic subgroup (for those who do not know
		what minimal parabolics are,
		you can think of $G$ as the group $GL_n$ of all invertible matrices,
		and then  minimal parabolic is the group of upper-triangular matrices).
		Given $P=MAN$ the Langlands decomposition of parabolic, (you can think of it as decomposing upper triangular matrix in the product of
		of diagonal and the upper-triangular with ones on a diagonal), principal series are indexed by irreducible representations of $M$
		and the vector in $k$-dimensional complex vector space, where $k$ is the dimension of $A$, known also as a real rank of $G$.

		Principal series are a good choice. 
		On the one hand, they are big enough (Harish-Chandra subquotient theorem tells us that every irrep of $G$
		can be represented as a subquotient of some principal series). On the other hand, they are intimately related to the geometry of $G$.
}
\end{frame}

\section{Motivation for the Goal}
\begin{frame}
	But even principal series representations are too big, so we switch to the
	{\bf spherical degenerate principal series}, where $\sigma$: irrep of $M$ is taken to be trivial
	and $P$ is taken to be {\it maximal} parabolic, instead of {\it minimal}.
	\begin{goal}[2, final]
		For $(G,G')=(O(p+1,q+1),O(p,q+1))$ and all $(\lambda,\mu)\in\C^2$, we want to classify the vector space
		\vspace{-0.4cm}
		\begin{equation*}
			\Hom_{G'}(I(\lambda)\kern-0.1cm\mid_{G'},J(\nu)),\quad \left( I(\lambda):=\Ind_P^G(1_\lambda),J(\nu):=\Ind_{P'}^{G'}(1_\nu) \right).
		\end{equation*}
	\end{goal}
	\vspace{-0.1cm}
	\begin{remark}
		Previous works \cite{kobayashi2014classification}, \cite{kobayashi2013finite} imply that $\dim\Hom_{G'}(I(\lambda)\kern-0.1cm\mid_{G'},J(\nu))$ is
		bounded uniformly in $(\lambda,\mu)\in\C^2$.
	\end{remark}
	\vspace{-0.2cm}
	\begin{remark}
		In \cite{kobayashi2015symmetry} (published in Memoirs of AMS) the complete answer for these questions was obtained in the case $q=0$ (that is, $(G,G')=(O(n+1,1),O(n,1))$).
	\end{remark}
	\note{But even principal series turn out to have too complicated structure, so we switch to their ``little brothers'' spherical degenerate
		principal series. Those are defined in a similar fashion, but now we take a maximal parabolic instead
		of minimal
		and trivial representation of $M$
		to begin with.

		In case of indefinite orthogonal groups we are dealing with now, spherical degenerate principal series of $G$ and $G'$ are indexed
		each by one complex parameter.

		It is in this setting that the first complete classification of SBOs in non-compact case was achieved by Toshiyuki Kobayashi and Birgit Speh
		in their book, published in Memoirs of AMS in 2015. It might be interested to notice that in their case
		maximal and minimal parabolic coincide, so that technically they obtain results for non-degenerated spherical principal series.

		Our original motivation was to extend their results to the more general $O(p,q)$ setting.
	}
\end{frame}
\begin{frame}
	$\mathcal{ABC}$ program introduced by T. Kobayashi predicted that it should be possible to classify SBOs for the pair of indefinite orthogonal groups.

	Beside that, there are two main reasons why Kobayashi--Speh succeeded in giving the complete classification:\vspace{1em}

	First, they developed a general theory, that implies that under the geometric condition $P'N_-P=G$ we have\begin{equation*}
		Hom_{G'}(I(\lambda)\kern-0.1cm\mid_{G'},J(\nu))\simeq\underbrace{\sol}_{\mbox{solutions of some PDEs}}\subset\mathcal{D}'(\R^{p,q}).
	\end{equation*}
	But more on this later\ldots
	\note{
		Before going into the details, let me briefly explain why I think Kobayashi and Speh succeeded in the first place.
		First, they developed a general theory that allows one to see symmetry breaking operators as solutions to some system of PDEs
		on the Euclidean space, thus bringing the analysis into the game.
	}
\end{frame}
\begin{frame}
	Second, we have a very concrete geometric model for the degenerate principal series $I(\lambda),J(\nu)$:
	\newdir{:=}{{}}
	\begin{setting}
			%\xymatrixcolsep{5pc}
	\xymatrix{
		& \mathcal{L}_\lambda\mbox{ :conformally equivariant line bundle},\lambda\in\mathbb{C}
		\ar[d]\\
	G=O(p+1,q+1)
	\ar@/^2pc/[r] &G/P\simeq (\Sp^p\times\Sp^q)/\left\{ \pm I \right\}\\
	P=MAN\ar@{:=}[u]_{\hspace{-0.25cm}\bigcup}
	\ar@/^2pc/[rd]^{{\begin{array}{c}\; \\\mbox{conformal transformations}\end{array}}}
	%\mbox\newline oeueou}\vspace{0.8cm}}
	&\\
M_+N=O(p,q)\ltimes \mathbb{R}^{p,q}
\ar@{:=}[u]_{\hspace{-0.25cm}\bigcup}
\ar@/^2pc/[r]^{\mbox{isometries}}&
\mathbb{R}^{p,q}=\left( \mathbb{R}^{p+q},ds^2=dx_1^2+\ldots+dx_p^2-dx_{p+1}^2-\ldots-dx_{p+q}^2 \right)\ar@{^{(}->}[uu]
_{\mbox{conformal 
compactification}}
\vspace{2cm}
	}

	\end{setting}
%%	\begin{setting}
%%		\[\begin{array}{ccc}
%%		G:=O(p+1,q+1)&\curvearrowright &I(\lambda):=C^\infty(G/P,\mathcal{L}_\lambda)\\
%%		\bigcup&&\\
%%		G':=O(p,q+1)&\curvearrowright &J(\nu):=C^\infty(G'/P',\mathcal{L}_\nu)
%%		\end{array}\]
%%	\end{setting}
	\note{
		Second, for this setting we have a very satisfying and concrete geometric model. Group $O(p+1,q+1)$ naturally acting on the $p+q+2$
		-dimensional
		space, leaves the light cone invariant. In turn, as the action is linear, we also have $O(p+1,q+1)$ acting on a quotient space of light cone
		by dilations, which in turn is diffeomorphic to the product of spheres with their opposite points identified.

		It then turns out that degenerate spherical principal series $I(\lambda)$ are realized as a space of homogeneous functions on the light
		cone. Consequently, restricting these functions to the product of spheres, $\Sp^p\times\Sp^q$, we see that $I(\lambda)$ can
		also be realized as a space of continuous functions on a product of spheres. Note that in this case, while the representation space
		stays the same, action does depend on $\lambda$.
	}
\end{frame}
%%%%%%%%%%%	PART 2
\begin{frame}{}
	\begin{center}
		\huge Part 2: Main Results
		\note{I will describe our main results now.}
	\end{center}
\end{frame}
\section{Main results (I)}
\begin{frame}
	The general theory of Kobayashi--Speh allows us to bring on the power of analysis:
	\begin{fact}[\cite{kobayashi2015symmetry}]
    Let $n:=p+q$. The following diagram commutes:
\begin{figure}[H]
	\centerline{
		\xymatrixcolsep{3pc}
		\xymatrix{\Hom_{G'}(I(\lambda),J(\nu))\ar[r]^{\simeq} \ar@/^2pc/[rr]^{\mathcal{S}}
		&\left( \mathcal{D}'(G/P,\mathcal{L}_{n-\lambda}) \otimes\mathbb{C}_\nu \right)^{P'}
		\ar[r]_-{F\mapsto \supp(F)}\ar[d]^{\rotatebox{270}{$\simeq$}}_{\mbox{rest}}
	&2^{P'\backslash G/P}\\
	&\sol\subset\mathcal{D}'(\R^{p,q})\ar[lu]^{\mbox{Op}}_{\simeq}&
	}
}
\end{figure}
\note{
%%	下のSol(Rpq)という空間がある(lambda,nu)に依存する偏微分方程式を満たす超関数空間です。ここで、左から真ん中への矢印が単純なSchwarz kernel定理で、真ん中から下へのrestという写像がEuclid 空間と同型なopen Bruhat cellに対して制限を表します。この2つの写像がベクトル空間の同型写像になります。なので、抽象的な対称性破れ作用素空間の代わりに具体的なEuclid空間上の偏微分方程式の解空間を研究してもいいということです。ポイントがもう1つあります。真ん中の空間がG/P上のP’普遍超関数空間なので、それぞれの元のサーポトがP’普遍G/P閉部分集合になります。つまり、真ん中のベクトル空間からGの両側剰余空間への写像できたということです。この両側剰余空間が有限なので、P’普遍G/P閉部分集合が大切なinvariantになります。
	\scriptsize
	As was briefly mentioned above, the general theory of Kobayashi--Speh implies that under some geometric conditions (which are satisfied
	in our case), the space of SBOs is isomorphic to the solution space of some PDE.

	More precisely, the following diagram commutes. $\sol$ below is the finitely-dimensional vector space of generalized function solution
	to some partial differential equation depending on $(\lambda,\nu)$. The map from left to the center is induced by Schwarz kernel theorem.
	It maps every symmetry breaking operator to its integral kernel. The map from center to the bottom, which we call rest, is the restriction of this
	integral kernel to open Bruhat cell, which is diffeomorphic to the Euclidean space. Both of these maps are isomorphisms of finitely-dimensional
	vector spaces and altogether this tells us that complicated and abstract space of SBOs can be replaced with more concrete space of solutions
	of a partial differential equation.

	But there is one more point to be made here.
	The middle vector space is the space of $P'$-invariant functions on $G/P$, hence the support of such a function should be a $P'$-invariant
	closed subset of $G/P$. This establishes the map from the middle vector space to the double coset space of $G$. In turn, as this double coset space
	is finite, the $P'$-invariant closed subsets of $G/P$ form an important invariant of symmetry breaking operators.
}
\end{fact}
\end{frame}
\begin{frame}
Note that $G$ acts on $\Xi^{p+1,q+1}:=\mysetn{(x,y)\in\R^{p+1,q+1}\setminus\left\{ 0 \right\}}{\myabs{x}^2=\myabs{y}^2}$ and on its quotient space
$X^{p,q}:=\Xi^{p+1,q+1}/\R^{\times}\simeq G/P$. Let
\[
	X:=G/P\simeq X^{p,q},\quad Y:=\mysetn{[\xi:\eta]\in G/P\simeq X^{p,q}}{\xi_{p}=0}\simeq X^{p-1,q}\]
	\[\kern-0.32cm C:=\mysetn{[\xi:\eta]\in G/P\simeq X^{p,q}}{\xi_{0}=\eta_q}\simeq X^{p-1,q-1}\cup\Xi^{p,q},\left\{ [0] \right\}:=\left\{ [1,0_{p+q},1] \right\}\]
\begin{theorem}[classification of closed $P'$-invariant subsets of $G/P$]
	The left $P'$-invariant closed subspaces of $G/P$ are as follows (numbers indicate codimension):\\
	\vspace{-1em}
  \begin{figure}[H]
    \centering
    \begin{subfigure}[t]{0.3\textwidth}
	    \xymatrixrowsep{0.5pc}
	    \xymatrix{&X\ar@{-}[ld]_1\ar@{-}[rd]^1&\\Y\ar@{-}[rd]_1&&C\ar@{-}[ld]^1\\&Y\cap C\ar@{-}[dd]^{p+q-2}&\\&&\\&\{[0]\}&}
	\caption{when $p>1$}
    \end{subfigure}
    ~ %add desired spacing between images, e. g. ~, \quad, \qquad, \hfill etc. 
      %(or a blank line to force the subfigure onto a new line)
    \begin{subfigure}[t]{0.3\textwidth}
	    \xymatrixrowsep{0.5pc}
	    {\xymatrix{&X\ar@{-}[ld]_1\ar@{-}[rd]^1&\\Y\ar@{-}[rddd]_{p+q-2}&&C\ar@{-}[lddd]^{p+q-2}\\&&\\&&\\&\{[0]\}&}}
	    \vspace{0.6em}
	\caption{when $p=1$}
    \end{subfigure}
\end{figure}
\end{theorem}
\note{
%%	最初の結果によって、こんな集合が分類されています。閉包(へいほう)関係[clo rel]を含めて、軌道分解を決定しました。つまり、pは1以上だったらP’普遍G/P閉部分集合が5があって、pは1と等しかったら、4つになります。CとYのX部分集合として余次元[codimension]がgeneric的に1になります。pが1以上だったら、C共通YのC又はYの中のcodimensionがgeneric的に1になって、generic的にp+q-2次元の集合になります。p=1だったら、C共通Yが原点と等しくなります。
	As a first result, we classify all such sets. We have explicitly determined the orbit decomposition of flag variety $G/P$ under the left $P'$
	-action,
	together with its closure relations. More precisely, when $p$ is bigger than one, there are five $P'$-invariant closed subsets of $G/P$
	and when $p=1$ there are four of them. The subsets $C$ and $Y$ have their generic codimension being equal to one. When $p$ is bigger than one,
	the intersection of $C$ and $Y$ has generic codimension one inside $C$ or $Y$ and is of generic dimension $p+q-2$. In turn, when $p$ equals one,
	the intersection of $C$ and $Y$ is just the origin point.
}
\end{frame}
%%\begin{frame}
%%	\begin{theorem}
%%		We can construct the following families of SBOs which holomorphically depend on parameters:
%%		\vspace{-0.9em}
%%		\renewcommand{\arraystretch}{3.5}
\scriptsize
\begin{center}
\begin{tabular}{|c|c|c|c|}
  \hline
  & $\tmop{Op}:\Sol(\mathbb{R}^{p,q};\lambda,\nu)\to\Hom_{G'}(I(\lambda),J(\nu))$ & defined for &
  $\mathcal{S} (\cdot)$ (generically)\\
  \hline
  $R_{\lambda, \nu}^X=$ & $\Op\left( \frac{| x_p |^{\lambda + \nu - n} | Q |^{-
  \nu}}{\Gamma \left( \frac{\lambda - \nu}{2} \right) \Gamma \left(
  \frac{\lambda + \nu - n + 1}{2} \right) \Gamma \left( \frac{1 - \nu}{2}
  \right)} \right)$ & $(\lambda,\nu)\in\mathbbm{C}^2$ & $X$\\
  \hline
  $\tilde{R}^X_{\lambda, \nu}=$ & $\Op\left(  \frac{| x_p |^{\lambda + \nu - n} | Q |^{-
  \nu}}{ \Gamma \left(
  \frac{\lambda + \nu - n + 1}{2} \right) \Gamma \left( \frac{1 - \nu}{2}
  \right)} \right)$ & $(\lambda,\nu)\in\mid \mid \mid$ & $X$\\
  \hline
  $R_{\lambda, \nu}^{\{ o \}}=$ & $\Op\left(  \tilde{C}_{\nu - \lambda}^{\lambda - \frac{n
  - 1}{2}} ({\Delta}_{\mathbb{R}^{p-1,q}} {\delta}_{\mathbb{R}^{p+q-1}}, \delta (x_p)) \right)$ & $(\lambda,\nu)\in/ /$ & $\{ [0]
  \}$\\
  \hline
\end{tabular}
\end{center}
\vspace{-1em}
\begin{itemize}
	\item $\mid \mid \mid \assign \{ (\lambda, \nu) \in \mathbbm{C}^2 \mid \nu \in
	- 2\mathbbm{Z}_{\geqslant 0} \cup (q + 1 + 2\mathbbm{Z}) \}$ \item $/ / \assign
\{ (\lambda, \nu) \in \mathbbm{C}^2 \mid \lambda - \nu \in
2\mathbbm{Z}_{\leqslant 0} \}$;
\item $Q:=x_1^2+\cdots+x_p^2-x_{p+1}^2-\cdots-x_{p+q}^2$;
\item $\tilde{C}(s,t)$ is a two-variable inflation of renormalized Gegenbauer polynomial, defined as in \cite{kobayashi2015symmetry}.
\end{itemize}

%%	\end{theorem}
%%	\note{
%%%%		次の結果に複数パラメーターに正則に依存する対称性破れ作用素のファミリーを3つ作ります。ここでR^X_lambda,nuのサーポトがほとんどいつもXと等しくなので、regular対称性破れ作用素と呼ばれています。Rtilda^X_lambda,nuがregular対称性破れ作用素のrenormalizationです。最後、R^{0}_lambda,nuのサーポトがいつも原点と等しくので、微分対称性破れ作用素と呼ばれています。//(ナナメのニじょうせん)と\myabs{(たての3分線)とういうのはC^2の部分多様体です。CtildeがGegenbauer多項式と関係がある2変数多項式です。
%%		Next result is the construction of three families of SBOs depending holomorphically on parameters. Here $R^X_{\lambda,\nu}$
%%		has its support being equal to the whole $X$ for generic values of parameters, hence we call it regular symmetry breaking operator.
%%		In turn, $\tilde{R}_{\lambda,\nu}^X$ is the renormalization of the regular SBO. Finally, $R^{ \left\{ o \right\}}_{\lambda,\nu}$ is
%%		supported at origin point only, hence it is called differential SBO.
%%
%%		The set double slash is the countable union of linear affine spaces. $\tilde{C}$ is the two-variable polynomial, induced by a renormalization
%%		of Gegenbauer polynomial.
%%	}
%%\end{frame}
\begin{frame}{Differential SBO}
	\begin{fact}[The classification of differential SBOs (\cite{kobayashi2015branching})]
		For
$(\lambda, \nu) \in / / \assign \{
(\lambda, \nu) \in \mathbbm{C}^2 \mid \nu - \lambda \in 2\mathbbm{N} \}$, we let
\begin{equation*}
	{R}_{\lambda, \nu}^{\{ o \}} =
\tmop{Rest}_{x_p = 0} \circ \tilde{C}_{\nu - \lambda}^{\lambda - \frac{n -
1}{2}} \left( - \Delta_{\mathbbm{R}^{p - 1, q}}, \frac{\partial}{\partial x_p}
\right)を定める。
\end{equation*}
Here $\tilde{C}(s,t)$ is a two-variable polynomial defined as in \cite[(6.5)]{Kobayashi2016}. It is the two-variable inflation
of a renormalized Gegenbauer polynomial. Then,
	$R^{ \left\{ o \right\}}_{\lambda,\nu}$ becomes the differential SBO (namely, $\mathcal{S}\mbox{upp} ({R}_{\lambda,
	\nu}^{\{ o \}}) = \{ 0 \}$), and all the differential SBOs are proportional to $R^{ \left\{ o \right\}}_{\lambda,\nu}$.
	\end{fact}
	\note{
Finally, $R^{ \left\{ o \right\}}_{\lambda,\nu}$ is
		supported at origin point only, hence it is called differential SBO.
	}
\end{frame}
\begin{frame}
	\begin{theorem}[Construction of regular SBO]
For
$(\lambda, \nu) \in \mathbbm{C}^2$, satisfying
$\tmop{Re} (\nu) < 0$ and $\tmop{Re}
(\lambda + \nu - n) > 0$, we consider the continuous function
\vspace{-0.6em}
\begin{equation*}
	| x_p |^{\lambda + \nu - n} | Q_{p, q} |^{- \nu}.
\end{equation*}
It becomes an element of $\mathcal{S} \mbox{ol} (\mathbbm{R}^{p, q} ; \lambda, \nu)$.
\vspace{-0.5em}
Let
\begin{equation*}
	\tilde{R}_{\lambda, \nu}^X \assign \tmop{Op} (| x_p
	|^{\lambda + \nu - n} | Q_{p, q} |^{- \nu}).
\end{equation*}
	\end{theorem}
\vspace{-0.4em}
\begin{theorem}[Normalization of regular SBO]
		$\tilde{R}_{\lambda, \nu}^X$ can be extended to meromorphic in $(\lambda, \nu) \in
\mathbbm{C}^2$ family of SBOs. Moreover, if we normalize it as
\vspace{-1.1em}
\begin{equation*}
	{R}_{\lambda, \nu}^X \assign \frac{1}{\Gamma
\left( \frac{\lambda - \nu}{2} \right) \Gamma \left( \frac{\lambda - \nu}{2}
\right) \Gamma \left( \frac{\lambda + \nu - n + 1}{2} \right)} \tilde{R}_{\lambda,
\nu}^X
\vspace{-0.6em}
\end{equation*}
$R_{\lambda,\nu}^X$ is then the family of SBOs, depending holomorphically on $(\lambda,\nu)\in\mathbb{C}^2$.
We also have $R^X_{\lambda,\nu}=0\iff(\lambda,\nu)$ belongs to some discrete countable subset of $\mathbb{C}^2$.
	\end{theorem}
	\note{
Here $R^X_{\lambda,\nu}$
		has its support being equal to the whole $X$ for generic values of parameters, hence we call it regular symmetry breaking operator.
	}
\end{frame}
\begin{frame}
\newcommand{\Supp}{\mathcal{S}{upp}}
	\begin{theorem}[Construction of singular SBOs]
	For $S=X,Y,C,$ $\left\{ o \right\}$, we define the parameter spaces
	$D_S$ as below. We also define symmetry breaking operators
	$R_{\lambda,\nu}^S$ or $\tilde{R}_{\lambda,\nu}^X$ from
	$I(\lambda)$ to $J(\nu)$, that are holomorphic in
$(\lambda,\nu)\in D_S$.
Each $R_{\lambda,\nu}^S$ or $\tilde{R}_{\lambda,\nu}^X$ is the renormalization of the regular SBO $\tilde{R}_{\lambda,\nu}^X$.
	For all $(\lambda,\nu)\in D_S$, we have $\Supp(R_{\lambda,\nu}^S)\subset S$ (moreover, for generic $(\lambda,\nu)$ the equality holds).
	We can explicitly determine the support of every $R_{\lambda,\nu}^S$ or $\tilde{R}_{\lambda,\nu}^X$ for every $(\lambda,\nu)$.\\
\centerline{\begin{tabular}{|l|l|}
  \hline
  ${R}_{\lambda, \nu}^S$ & $D_S$\\
  \hline
  ${\tilde{R}}_{\lambda, \nu}^X$ & $\mid \mid \mid$\\
  \hline
  ${R}_{\lambda, \nu}^Y$ & \textbackslash\textbackslash\\
  \hline
  ${R}_{\lambda, \nu}^C$ & $\mid \mid \mid$\\
  \hline
\end{tabular}}
	\end{theorem}
We now explain the notation:
\vspace{-0.5em}
	\begin{equation*}
		\begin{array}[]{l}
		\mid \mid \mid \assign \left\{ (\lambda, \nu) \in \mathbbm{C}^2 \mid
\nu \in - 2\mathbbm{N} \; \tmop{or} \; \nu \equiv q + 1 \; \tmop{mod} \; 2
\right\},\\
\backslash\backslash \assign \{ (\lambda, \nu) \in \mathbbm{C}^2 \mid
\lambda + \nu - n + 1 \in - 2\mathbbm{N} \},\\
\mid \mid \assign \{ (\lambda, \nu) \in \mathbbm{C}^2 \mid \nu \in 1 +
2\mathbbm{N} \} . 
		\end{array}
	\end{equation*}
	\note{
		singular SBOs
	}
\end{frame}
\begin{frame}
	\begin{theorem}[Classification of SBO]
		We can find basis for $\Hom_{G'}(I(\lambda),J(\nu))$ for every $(\lambda,\nu)\in \mathbb{C}^2$. In particular, we have
		\vspace{-1.4em}
		\newcommand{\SBO}{\Hom_{G'}(I(\lambda),J(\nu))}
		\begin{equation*}
	  p
	  >1\implies\SBO= \left\{
    \begin{array}{ll}
	    \mathbbm{C} {\tilde{\OpR}}_{\lambda, \nu}^{X} \oplus \mathbbm{C}
      {\OpR}^{\{ 0 \}}_{\lambda, \nu}, & (\lambda, \nu) \in / /\cap 
      \mid\mid\mid \\
      \mathbbm{C} \OpR^X_{\lambda, \nu}, &
      \mbox{\normalfont otherwise.}
    \end{array} \right. 
    \vspace{-0.9em}
  \end{equation*}
		\begin{equation*}
p=1\implies 
			\SBO = \left\{
	\begin{array}{@{}ll@{}}
		\mathbbm{C}R^X_{\lambda, \nu}, & (\lambda, \nu) \in \mathbbm{C}^2 - \mathcal{A} - \mathcal{X},\\
     \mathbbm{C} \tilde{R}^X_{\lambda, \nu} \oplus \mathbbm{C}R^{\{ o
     \}}_{\lambda, \nu}, & (\lambda, \nu) \in \mathcal{A} -
     \mathcal{X},\\
     \mathbbm{C}R^P_{\lambda, \nu} \oplus \mathbbm{C}R^C_{\lambda, \nu}, &
     (\lambda, \nu) \in \mathcal{X} - / /,\\
     \mathbbm{C}R^{\{ o \}}_{\lambda, \nu}, & (\lambda, \nu) \in \mathcal{X} \cap / /.
   \end{array} \right.
   \end{equation*}
   Note that here $\mathcal{A}:=//\cap \mid\mid\mid$ and $\mathcal{X}:=\backslash\backslash\cap\mid\mid$ are discrete countable subsets of $\mathbb{C}^2$.
		\label{}
	\end{theorem}
	\begin{corollary}
		\begin{equation*}
	  \dim\Hom_{G'}(I(\lambda),J(\nu))= \left\{
    \begin{array}{ll}
	    2, & (\lambda, \nu) \in / /\cap 
      \mid\mid\mid \\
      1, &
      \mbox{\normalfont otherwise.}
    \end{array} \right. 
  \end{equation*}
	\end{corollary}
	\note{
%%次の定理に、対称性破れ作用素が分類することができました。特に対称性破れ作用素の空間が可算無限離散集合を除いて、1次元になりますが、この集合上で2次元になります。
		Next result is the classification of symmetry breaking operators. In particular, aside from the countable discrete set
		of parameters, the dimension is 1, while at the parameters in this discrete countable set, the dimension of SBO space is 2.
	}
\end{frame}
\section{Main results (II)}
\begin{frame}{Spherical multiple}
	\begin{theorem}
		Let $1_\lambda\in C^\infty(G/P,\mathcal{L}_\lambda)^K,1_\nu\in C^\infty(G'/P',\mathcal{L}_\nu)^{K'}$ be the spherical vectors. We then have
	\[ \OpR^X_{\lambda, \nu} 1_{\lambda} = 2^{1 -
     \lambda} \frac{\pi^{n / 2}}{\Gamma \left( \frac{\lambda}{2} \right)
     \Gamma \left(  \frac{\lambda + 1-q}{2} \right) \Gamma \left(
     \frac{q - \nu + 1}{2} \right)} 1_{\nu}. \]
	\end{theorem}
	\begin{remark}
		The case corresponding to the tensor product of representations of $SL_2$ (that is, $p=q=1$ case)
		was previously shown in \cite[Lem. A.5]{bernstein2004estimates}.
		The higher rank generalization was obtained in \cite{clerc2011generalized}.
		The $q=0$ case was shown in \cite[Prop. 7.4]{kobayashi2015symmetry}.
	\end{remark}
	\note{
%%		次の結果によると、I(lambda)のK-finiteベクトルのイメージがJ(nu)のK’-finiteベクトルと比例になります。比例定数も計算できて、極点とレイテンを分けるように積公式で表せます。ちなみに、
%%		Bernstein-Reznikovの2004年の論文にp=q=1の場合に同じい結果が得ました。
		Next result shows that the image of $K$-invariant vector in $I(\lambda)$ is proportional to the $K'$-invariant vector of $J(\nu)$.
		With appropriate normalization chosen, the proportionality constant is also computed and given as a product of Gamma functions,
		which allows one to easily determine poles and zeros.

		The case $p=q=1$ was obtained by Bernstein--Reznikov in 2004. 
	}
\end{frame}
%%\begin{frame}{Residue theorem}
%%	\begin{theorem}
%%		The distribution
%%		\[K_{\lambda,\nu}^{\mathbb{R}^{p,q}}:=\frac{\myabs{x_p}^{\lambda+\nu-n}}{\Gamma\left( \frac{\lambda+\nu-n+1}{2} \right)}\times
%%		\frac{\myabs{Q}^{-\nu}}{\Gamma\left( \frac{1-\nu}{2} \right)}\]
%%		has the pole at $(\lambda,\nu)\in//$ with the residue given by
%%		{\footnotesize
%%		\[\Res_{(\lambda,\nu)\in//}K_{\lambda,\nu}^{\mathbb{R}^{p,q}}=\frac{K_{\lambda,\nu}^{\mathbb{R}^{p,q}}}{\Gamma\left( \frac{\lambda-\nu}{2} \right)}
%%			=\frac{ (- 1)^k k!\pi^{(n - 2) / 2} 
%%		}{2^{ \nu + 2 k-1}}\cdot  \frac{\sin\left( \frac{1+q-\nu}{2}\pi \right)}{\Gamma\left( \frac{\nu}{2} \right)}
%%	\tilde{C}_{\nu - \lambda}^{\lambda - \frac{n
%%  	- 1}{2}} ({\Delta}_{\mathbb{R}^{p-1,q}} {\delta}_{\mathbb{R}^{p+q-1}}, \delta (x_p))
%%		\]
%%		}
%%		where $k:=\frac{\nu-\lambda}{2}$.
%%		Hence, taking $\Op(\cdot)$ on both sides we get
%%  \[\OpR_{\lambda,\nu}^X  = \frac{ (- 1)^k k!\pi^{(n - 2) / 2} 
%%		}{2^{ \nu + 2 k-1}}\cdot  \frac{\sin\left( \frac{1+q-\nu}{2}\pi \right)}{\Gamma\left( \frac{\nu}{2} \right)}
%%     \OpR_{\lambda,\nu}^{ \left\{ 0 \right\} },\quad(\lambda,\nu)\in// . \]
%%	\end{theorem}
%%	\note{
%%%%		次の結果はregular対称性破れ作用素のkernelがナナメのニじょうせん集合上に多重度たかだか1のpoleがあって、このpoleのresidueが微分対称性破れ作用素のkernelと比例になるとういうことを表しています。先のように、比例点数を極点とレイテンを分けるように積公式で表せます
%%		Next result shows that the kernel of regular SBO has pole of multiplicity at most one, when parameters lie in the aforementioned
%%		double slash subset. The residue of this pole is proportional to the kernel of differential SBO. Again, proportionality
%%		constant is explicitly computed and given as a product of Gamma functions, allowing one to easily see zeros and poles.
%%	}
%%\end{frame}
\begin{frame}{Residue theorem}
	For $(\lambda,\nu)\in\mathbb{C}^2\setminus//$, we let
\begin{equation*}
K_{\lambda, \nu}^X \assign \frac{| x_p |^{\lambda +
\nu - n}}{\Gamma \left( \frac{\lambda + \nu - n + 1}{2} \right)} \cdot \frac{|
Q |^{- \nu}}{\Gamma \left( \frac{1 - \nu}{2} \right)} \in \mathcal{S}
\mbox{ol} (\mathbbm{R}^n ; \lambda, \nu).
\end{equation*}
Then, $\tilde{R}_{\lambda, \nu}^X = \frac{1}{\Gamma \left( \frac{\lambda -
\nu}{2} \right)} \tmop{Op} (K_{\lambda, \nu})$ can be extended to the family of SBOs holomorphic in $(\lambda,\nu)\in\mathbb{C}^2$.
\begin{theorem}[Residue theorem]
	For $(\lambda,\nu)\in//$, we let $l:=\frac{1}{2}\left( \nu-\lambda \right)\in\N$. Then,
	\vspace{-1em}
\begin{equation*}
	\tilde{R}_{\lambda, \nu}^X = \frac{(- 1)^l l!
\pi^{(n - 2) / 2}}{2^{\nu + 2 l - 1}} \cdot \frac{\sin \left( \frac{1 + q -
\nu}{2} \pi \right)}{\Gamma \left( \frac{\nu}{2} \right)} \tilde{R}_{\lambda,
\nu}^{\{ o \}}, \quad \mbox{for }(\lambda, \nu) \in //.
\end{equation*}
\end{theorem}
\begin{remark}
%%		留数公式の原型
%%		として$q=0$の場合は\cite[Thm. 12.2]{kobayashi2015symmetry}で得られた。
		The prototype of this theorem (namely, the $q=0$ case), was obtained in
		\cite[Thm. 12.2]{kobayashi2015symmetry}.
\end{remark}
	\note{
		Next result shows that the distribution $K_{\lambda,\nu}^X$ defined as on the slide,
		has pole of multiplicity at most one, when parameters lie in the aforementioned
		double slash subset. 
		After we normalize it, we see that the residue of this pole is proportional to the kernel of differential SBO. Again, proportionality
		constant is explicitly computed and given as a product of Gamma functions, allowing one to easily see zeros and poles.
	}
\end{frame}
\begin{frame}{Knapp--Stein operator}
	\begin{definition}
		\underline{Knapp--Stein operator} is the $G$-intertwining operator defined in the following way:\begin{equation*}
			\begin{array}[]{l}
				\tilde{\mathbb{T}}_\lambda^G:I(\lambda)\to I(n-\lambda)\\
				f\mapsto q_T(\lambda)\left(\myabs{Q_{p,q}}^{\lambda-n}\ast f  \right)
			\end{array}
		\end{equation*}
		Here
		$q_T (\lambda)$ can be explicitly described as the product of $\Gamma$ functions.
	\end{definition}
	Both
		$\tilde{R}_{n - \lambda, \nu}^X \circ\tilde{\mathbbm{T}}_{\lambda}^G$ and
		$\tilde{R}_{\lambda, \nu}^X$ are SBOs. Let us compare them.
		\note{Before introducing the next result, the functional identity, we recall the definition of a classical
		Knapp-Stein operator. The setting $G=G'$ of branching was well-studied before and led to the introduction of the Knapp-Stein operator.
		It is the $G$-intertwining operator from $I(\lambda)$ to $I(n-\lambda)$,
		characterized by its holomorphic dependence on its single parameter, the fact that it is nonvanishing.
		It's exact form is given as the convolution with indefinite quadratic form, but we omit the normalization constant for simplicity.
	}
\end{frame}
\begin{frame}{}
	Both
		$\tilde{R}_{n - \lambda, \nu}^X \circ\tilde{\mathbbm{T}}_{\lambda}^G$ and $\tilde{R}_{\lambda, \nu}^X$ are SBOs. Let us compare them.
		\centerline{\xymatrix{I(\lambda)\ar[d]^{\tilde{\mathbb{T}}_{\lambda}^G}\ar@{-->}[rd]^{R_{\lambda,\nu}^X}&\\
		I(n-\lambda)\ar[r]^{R^X_{n-\lambda,\nu}}&J(\nu)}}
		For $G' = O (p, q + 1)$, we similarly define $\tilde{\mathbbm{T}}_{\nu}^{G'}$:
		\centerline{
			\xymatrix{
				I(\lambda)\ar@{-->}[rd]^{R_{\lambda,\nu}^X}\ar[r]^{R_{\lambda,n-1-\nu}^X}&J(n-1-\nu)\ar[d]^{\tilde{\mathbb{T}}_{n-1-\nu}^{G'}}\\
				&J(\nu)
			}}
			\note{Now, the next two commutative diagrams convince us that Knapp-Stein operator composed with the regular SBO
			from the left or from the right should still be symmetry breaking operators and as the dimension of SBO space is
		generically equal to one, we conclude that they should be proportional. }
\end{frame}
\begin{frame}
  \begin{figure}[H]
    \centering
    \begin{subfigure}[t]{0.3\textwidth}
\xymatrix{I(\lambda)\ar[d]^{\tilde{\mathbb{T}}_{\lambda}^G}\ar@{-->}[rd]^{R_{\lambda,\nu}^X}&\\I(n-\lambda)\ar[r]^{R^X_{n-\lambda,\nu}}&J(\nu)}
    \end{subfigure}
    ~ %add desired spacing between images, e. g. ~, \quad, \qquad, \hfill etc. 
      %(or a blank line to force the subfigure onto a new line)
    \begin{subfigure}[t]{0.3\textwidth}
			\xymatrix{I(\lambda)\ar@{-->}[rd]^{R_{\lambda,\nu}^X}\ar[r]^{R_{\lambda,n-1-\nu}^X}&J(n-1-\nu)\ar[d]^{\tilde{\mathbb{T}}_{n-1-\nu}^{G'}}\\&J(\nu)}
    \end{subfigure}
    \end{figure}
    \vspace{-1em}
    \begin{theorem}[Functional identities]
	    Let
	$n=p+q\;(p,q\ge1)$ be as above. Then,
	\vspace{-1em}
	\begin{equation*}
		\begin{array}[]{c}
\tilde{R}_{n - \lambda, \nu}^X \circ
\tilde{\mathbbm{T}}_{\lambda}^G = q_X^{X T} (\lambda, \nu)
\tilde{R}^X_{\lambda, \nu} \\
\tilde{\mathbbm{T}}_{n - 1 - \nu}^{G'} \circ
\tilde{R}_{\lambda, n - 1 - \nu}^X = q_X^{T X} (\lambda, \nu)
\tilde{R}_{\lambda, \nu}^X,\mbox{ where}\\
q_X^{X T} (\lambda, \nu) = \ldots,\quad
q_X^{T X} (\lambda, \nu) = \frac{\pi^{\frac{n -
2}{2}} \sin \left( \frac{p - \nu}{2} \pi \right) 2^{1 - n - \nu}}{\Gamma
\left( \frac{n - 1 - \nu}{2} \right)} \left\{ \begin{array}{ll}
  \Gamma \left( \frac{1 - \nu}{2} \right), & p = 1\\
  1, & n \in 2\mathbbm{Z}\\
  \ldots, & \ldots
\end{array} \right.
		\end{array}
	\end{equation*}
	\end{theorem}
\vspace{-0.5em}
	\begin{remark}
		The prototype of this theorem (namely, the $q=0$ case), was obtained in
		\cite[Thm. 12.6]{kobayashi2015symmetry}.
	\end{remark}
	\note{Proportionality constant can be explicitly determined.}
\end{frame}
\begin{frame}{Image of SBOs}
	\begin{theorem}
		We can compute image of every SBOs constructed above for every $(\lambda,\nu)$.
	\end{theorem}
	\note{The last result\dots}
\end{frame}
\begin{frame}{}
	\begin{center}
		\huge Part III: Future Work
	\end{center}
	\note{Finally, I would like to talk very briefly about where one can go from here.}
\end{frame}
\begin{frame}{Future Work}
	\begin{enumerate}
		\item Branching between Zuckerman derived functors $A_{\mathfrak{q}}(\lambda)$ (cf. \cite{KO1,kobayashi1998discrete3});
		\item Branching between differential forms on spheres (cf. \cite{kobayashi2016classification,kobayashi2017symmetry});
		\item Confirming {\bf Gan-Gross-Prasad hypothesis} for $O(N+1)\shortdownarrow O(N)$ branching (cf. \cite{kobayashi2017symmetry}).
	\end{enumerate}
\end{frame}
\begin{frame}{}
	\begin{center}
		\huge Thank You for Your attention!
	\end{center}
\end{frame}
\begin{frame}[allowframebreaks]{References}
	
\begin{thebibliography}{KØSS15}

\bibitem[KKP16]{kobayashi2016classification}
Toshiyuki Kobayashi, Toshihisa Kubo, and Michael Pevzner.
\newblock Classification of differential symmetry breaking operators for
  differential forms.
\newblock {\em Comptes Rendus Mathematique}, 354(7):671--676, 2016.

\bibitem[KS17]{kobayashi2017symmetry}
Toshiyuki Kobayashi and Birgit Speh.
\newblock Symmetry breaking for orthogonal groups and a conjecture by {B}.
  {G}ross and {D}. {P}rasad.
\newblock {\em arXiv preprint arXiv:1702.00263}, 2017.

  \bibitem[Kob98b]{kobayashi1998discrete3}T.~Kobayashi. {\newblock}Discrete
  decomposability of the restriction of $A_q (\lambda)$ with respect to
  reductive subgroups III. restriction of Harish-Chandra modules and
  associated varieties. {\newblock}\tmtextit{Inventiones mathematicae},
  131(2):229--256, 1998.

  \bibitem[K{\O}03]{KO1}T.~Kobayashi and B.~{\O}rsted. {\newblock}Analysis on
  the minimal representation of \tmtextrm{O}$(p, q)$.\tmtextrm{I}. Realization
  via conformal geometry. {\newblock}\tmtextit{Adv. Math.}, 180:486--512,
  2003.

  \bibitem[CK{\O}P11]{clerc2011generalized}J.-L.~Clerc, T.~Kobayashi,
  B.~{\O}rsted , and  M.~Pevzner.{\newblock} Generalized Bernstein--Reznikov
  integrals.{\newblock} \tmtextit{Mathematische Annalen}, 349(2):395--431,
  2011.{\newblock}

\bibitem[KP16b]{Kobayashi2016}
T.~Kobayashi and M.~Pevzner.
\newblock Differential symmetry breaking operators: {I}{I}. {R}ankin--{C}ohen
  operators for symmetric pairs.
\newblock \emph{Selecta Mathematica}, \textbf{22}(2), (2016), pp. 847--911.
Available at \url{http://dx.doi.org/10.1007/s00029-015-0208-8}.

\bibitem[K93]{kobayashi1993}
T.~Kobayashi.
\newblock The restriction of ${A}_q \left( \lambda \right)$ to reductive
  subgroups.
\newblock \emph{Proc. Japan Acad. Ser. A Math. Sci.}, \textbf{69}(7), (1993),
  pp. 262--267.
Available at \url{http://dx.doi.org/10.3792/pjaa.69.262}.
  \bibitem[KM14]{kobayashi2014classification}T.~Kobayashi  and 
  T.~Matsuki.{\newblock} Classification of finite-multiplicity symmetric
  pairs.{\newblock} In \tmtextit{\tmtextrm{\tmtextup{\tmtextmd{Special Issue
  in honour of Professor Dynkin for his 90th birthday}}}},  volume~19,  pages 
  457--493. Springer, 2014.{\newblock}
  
  \bibitem[KO13]{kobayashi2013finite}T.~Kobayashi  and  T.~Oshima.{\newblock}
  Finite multiplicity theorems for induction and restriction.{\newblock}
  \tmtextit{Advances in Mathematics}, 248:921--944, 2013.{\newblock}
  
  \bibitem[Kob05]{Kobayashi2005}Toshiyuki Kobayashi.{\newblock}
  \tmtextit{Restrictions of Unitary Representations of Real Reductive Groups},
  pages  139--207.{\newblock} Birkh{\"a}user, 2005.{\newblock}
  
  \bibitem[Kob15]{kobayashi2015program}T.~Kobayashi.{\newblock} A program for
  branching problems in the representation theory of real reductive
  groups.{\newblock} In \tmtextit{\tmtextrm{\tmtextup{\tmtextmd{Special issue
  in honor of Vogan's 60th years birthday}}}},  volume  312,  pages  277--322.
  Birkh{\"a}user, 2015.{\newblock}
  
  \bibitem[KS15]{kobayashi2015symmetry}T.~Kobayashi  and  B.~Speh.{\newblock}
  \tmtextit{Symmetry Breaking for Representations of Rank One Orthogonal
  Groups},  volume \tmtextbf{238} of \tmtextit{Memoirs of the Amer. Math.
  Soc}.{\newblock} 2015.{\newblock}
  
  \bibitem[K{\O}SS15]{kobayashi2015branching}T.~Kobayashi, B.~{\O}rsted,
  P.~Somberg, and  V.~Sou{\v c}ek.{\newblock} Branching laws for verma
  modules and applications in parabolic geometry. I.{\newblock}
  \tmtextit{Advances in Mathematics}, 285:1796--1852, 2015.{\newblock}

\bibitem[K{\O}03]{KO2}
T.~Kobayashi and B.~{\O}rsted.
\newblock Analysis on the minimal representation of\/ {$\mbox{\rm O}(p,q)$}.{$\;$}{{\rm{II}}}. {B}ranching laws.
\newblock \emph{Adv. Math.}, \textbf{180}(2), (2003), pp. 513--550.
Available at \url{https://doi.org/10.1016/S0001-8708(03)00013-6}.

\bibitem[BR04]{bernstein2004estimates}
J.~Bernstein and A.~Reznikov.
\newblock Estimates of automorphic functions.
\newblock \emph{Mosc. Math. J}, \textbf{\textbf{4}}(1), (2004), pp. 19--37.
Available at \url{http://mi.mathnet.ru/eng/mmj141}.
\end{thebibliography}
\end{frame}

\begin{frame}
	\begin{setting}
		\begin{enumerate}
			\item 
				Let $p,q\ge1$, $G:=O(p+1,q+1)$ and $G':=O(p+1,q+1)_{e_{p+1}}\simeq O(p,q+1)$. 
			\item
				Let $P:=MAN$ and $P':=G'\cap P=M'AN'$ :max parabolic, where\\
\newcommand{\longminus}{\textemdash\textemdash}
\hspace{-1.05cm}  \begin{tabular}{l@{}}
    $N \assign \left\{ \left[ \begin{array}{lll}
      1 - Q & -^t w' & Q\\
      w & I_{p + q} & - w\\
      - Q & -^t w' & 1 + Q
    \end{array} \right] \middle| \begin{array}{c}
      (x, y) \in \mathbbm{R}^{p, q}\\
      w \assign (x, y)\\
      w' \assign (x, - y)\\
      Q \assign \frac{| x |^2 - | y |^2}{2}
    \end{array} \right\}$, $M \assign \left\{ \left[ \begin{array}{ccc}
      \epsilon & 0 & 0\\
      0 & A & 0\\
      0 & 0 & \epsilon
    \end{array} \right] \middle| \begin{array}{c}
      A \in O (p, q)\\
      \epsilon = \pm 1
    \end{array} \right\}$\\
    $N' \assign \left\{ \longminus \longminus \longminus'' \longminus
    \longminus \longminus \middle| \begin{array}{c}
      \longminus'' \longminus\\
      x_p = 0
    \end{array} \right\}$, $M' \assign \left\{ \longminus'' \longminus
    \middle| \begin{array}{c}
      \longminus'' \longminus\\
      A e_p = e_p
    \end{array} \right\}$\\
    $A \assign a (\mathbbm{R})$, \quad$a (t) \assign \left[ \begin{array}{ccc}
      \cosh (t) & 0 & \sinh (t)\\
      0 & I_{p + q} & 0\\
      \sinh (t) & 0 & \cosh (t)
    \end{array} \right]$
  \end{tabular}
			\item For $(\lambda,\nu)\in\mathbb{C}^2$ we let $I(\lambda),J(\nu)$ to be the degenerate principal series of $G,G'$ respectively, i.e.
$I(\lambda):=\{f\in C^\infty(G)\mid \forall h\in P,\;f(\cdot h)=\lambda(h) f(\cdot)\}$, where $\lambda:P\ni m\cdot a(t)\cdot n\mapsto e^{-\lambda t}$ is a $P$-representation and
with $G$-action by left multiplication, and similarly for $J(\nu)$.
		\end{enumerate}
	\end{setting}
\end{frame}
\end{document}
%TODO:
%	more checkpoints (<=10min)
