\documentclass[12pt]{article} % use larger type; default would be 10pt

%%\usepackage[T1,T2A]{fontenc}
%%\usepackage[utf8]{inputenc}
%%\usepackage[english,ukrainian]{babel} % може бути декілька мов; остання з переліку діє по замовчуванню. 
\usepackage{enumerate}
\usepackage{CJKutf8}
\usepackage{mystyle}
\usepackage{amsthm}

%%\usepackage{fancyhdr}
%%\pagestyle{fancy}
%%\fancyfoot[C]{text me at \href{mailto:leontiev@ms.u-tokyo.ac.jp}{leontiev@ms.u-tokyo.ac.jp} if there are mistakes/obscurities}
%%\fancyhead{}

\theoremstyle{theorem}
\newtheorem{problem}{Problem}
\newtheorem{question}{Question}
\theoremstyle{definition}
\newtheorem{answer}{My Answer}
\newtheorem{reason}{Reasons}
\theoremstyle{remark}
\newtheorem{countermeasure}{Counter-measures}
\newtheorem{remark}{Remark}
\newtheorem*{remark*}{Remark}

\title{Reflection}
\begin{document}

\begin{CJK}{UTF8}{bsmi}
	\maketitle
\end{CJK}

\section{Trivia}
\begin{center}
	\begin{tabular}[]{l|l}
		Date:&August 04, 2017\\
		Conference/Seminar's name:& Number Theory Seminar\\
		Place:& Institute of Mathematics, Academia Sinica, Taibei, Taiwan\\
		Title:&Symmetry breaking operators of indefinite orthogonal groups $O(p,q)$\\
		Expected Duration:&60 min\\
		Real Duration:&50 min\\
		Self-evaluation:& 8 out of 10\\
	\end{tabular}
\end{center}
\section{Questions asked during the talk and my answers}
\begin{remark*}
	Numbers near the questions below correspond to the numbering of slides in {\ttfamily speech.pdf}.
	Also near each question the name of the person who raised the question is listed.
\end{remark*}
\begin{question}[14, Person A\footnote{Professor Hsieh who hosted the seminar has described this person as ``non-mathematician'', but I still 
	provide his questions (except of one I was unable to understand or memorize) for completeness}]\label{question:howe}
	When was the paper by Howe and Tan written?
\end{question}
\begin{answer}
	In 70s or 80s.
\end{answer}
\begin{question}[14, Person A]
	Can you use your method to say something about the group E8?
	This group is very important in physics.
\end{question}
\begin{answer}
	We did not think about this yet, but thank you for suggestion.
\end{answer}
\begin{question}[11, Professor Hsieh]
	Shouldn't it be $f\in C^\infty(G,V)$ instead of $f\in C^\infty(G)$?
\end{question}
\begin{answer}
	Yes, sorry, my bad. This is a typo.
\end{answer}
\begin{question}[12, Professor Hsieh]
	Are degenerate spherical principal series always irreducible?
\end{question}
\begin{answer}
	No, they are reducible for discrete countable values of parameters (have shown explicitly what these values are with TeXMacs)
	and these points of reducibility are exactly where the most ``interesting'' stuff happens.
\end{answer}
\begin{question}[19, Person A]
	What is $Q_{p,q}$?
\end{question}
\begin{answer}
	Showed what $Q_{p,q}$ is with the help of TeXMacs.
\end{answer}
\begin{question}[29, Professor Hsieh (host of the seminar)]
	Can you do the same for $p$-adic case?
\end{question}
\begin{answer}
	I am sorry, I am not very familiar with $p$-adic. However, I doubt that
	our method would be much applicable. As you might have noticed, It is analysis-heavy,
	and according to my understanding, the analysis in $p$-adic does not require such sophistication.
\end{answer}
\section{Problems, their reasons and counter-measures}
\begin{problem}
	Finished the talk in 50 minutes instead of 59.
\end{problem}
\begin{reason}
	Not enough practice (especially, timing was still not perfect).
	I have started working on this talk only on Tuesday (August 2nd).
\end{reason}
\begin{countermeasure}
	Start working on talk at least 8 days before.
\end{countermeasure}

\begin{problem}
	My extremely stupid answer to Question \ref{question:howe}.
\end{problem}
\begin{reason}
	Do not remember the bibliographical information.
\end{reason}
\begin{countermeasure}
	Memorize the bibliographical information related to the talk in advance.
	At least, have it printed on a paper with me during the talk.
\end{countermeasure}

\begin{problem}
	Bad answers to questions.
\end{problem}
\begin{reason}
	Not enough knowledge.
\end{reason}
\begin{countermeasure}
	Read more literature.
\end{countermeasure}

%%\begin{thebibliography}{9}
%%\bibitem{gelbaum}Gelbaum, B.R. and Olmsted, J.M.H.. Counterexamples in Analysis. Dover Publications. 2003
%%\end{thebibliography}
\end{document}


