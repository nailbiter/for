%japanese
\documentclass[10pt]{article}
\usepackage{fontspec}
\usepackage{array, xcolor, lipsum, bibentry}
\usepackage[margin=3cm]{geometry}
\usepackage{sectsty} % Allows changing the font options for sections in a document
 
\title{\bfseries\Huge Oleksii Leontiev}
\author{alozz1991@gmail.com}
\date{}
 
\definecolor{lightgray}{gray}{0.8}
\newcolumntype{L}{>{\raggedleft}p{0.2\textwidth}}
\newcolumntype{R}{p{0.8\textwidth}}
\newcommand\VRule{\color{lightgray}\vrule width 0.5pt}
 
%font configuration
\defaultfontfeatures{Mapping=tex-text}
\setromanfont[Ligatures={Common}, Numbers={OldStyle}, Variant=01]{Helvetica} % Main text font
\sectionfont{\mdseries\upshape\Large} % Set font options for sections
\subsectionfont{\mdseries\scshape\normalsize} % Set font options for subsections
\subsubsectionfont{\mdseries\upshape\large} % Set font options for subsubsections
%\chardef\&="E050 % Custom ampersand character
 
\begin{document}
\maketitle
\vspace{1em}
\begin{minipage}[ht]{0.48\textwidth}
Graduate School of Mathematical Sciences\\
The University of Tokyo\\
3-8-1 Komaba Meguro-ku, Tokyo, Japan
\end{minipage}
\begin{minipage}[ht]{0.48\textwidth}
Ukrainian\\
December 24, 1991\\
+81 809 294 9828
\end{minipage}
\vspace{20pt}
 
\section*{Education}
\begin{tabular}{L!{\VRule}R}
2016--now&{PhD in Mathematics}, The University of Tokyo, Japan.\vspace{6pt}\\
2014--2016&{MPhil in Mathematics (thesis: {\it Study of symmetry breaking operators of indefinite orthogonal groups O(p,q)}
; advisor: Toshiyuki Kobayashi)}, The University of Tokyo, Japan.\vspace{6pt}\\
2009--2013&{BSc in Applied Math and Computer Science (double degree) \textbf{(average: 94.87)}}, National Chiao Tung University, Taiwan.\vspace{5pt}\\
2006--2009&High School Graduate, Nature Science Lyceum 145, Ukraine.\\
\end{tabular}

\section*{Honors \& Awards}
\begin{tabular}{L!{\VRule}R}
2015&Successful participant in Google Summer of Code 2015\\
2013&Successful participant in Google Summer of Code 2013\\
2013&Nominated as an Exchange Student to Hong Kong University\\
2011--2012&Diamond Project Scholarship, NCTU\\
2010--2012&Best Student in Class Award, NCTU\\
2009--2013&Golden Bamboo Scholarship Award, NCTU\\
\end{tabular}

\section*{Working Experience \& Projects}
\begin{tabular}{L!{\VRule}R}
2016--now&{\bf AJP Games}\\& 
Unity engineer at AJP Games.\\
2015--2015&{\bf Google Summer of Code 2015}\\& 
During the summer of 2015 I implemented a Tracking-Learning-Detection object tracking algorithm for the openCV library based
on the paper by Zdenek Kalal and his coauthors.
The implementation allows the user to select the object in the first frame and to have it tracked during the whole sequence.
The algorithm is known to be robust to occlusions and appearance changes.
\\
2013--2013&{\bf Google Summer of Code 2013}\\& 
I was a successful participant of Google Summer of Code 2013, the initiative, where students can have an opportunity to work on some
project for open-source company of their choice. I've submitted a proposal for Open Source Computer Vision Library (openCV) and it has passed
the competition. During the course of a project, I've implemented the generic numerical optimization module for openCV.\\
2013--2013&{\bf Pipeline for 3D models reconstruction}\\& 
Starting from my junior year I was working for Prof. Jong-Hong Chuang (Department of Computer Science, NCTU).
I was implementing the framework for the reconstruction of 3D models from the partial scans of a given body.
Pipeline included registration (that is, grouping partial scans together to get the set of
points describing the whole object), filtering and normal estimation (that is, clearing the
result from previous step and estimating normal for each point) and mesh reconstruction. I was doing mostly the first part (registration). My most notable achievement is the
naive implementation of a Chavdar Papazov’s ”Stochastic Optimization for Rigid Point
Set Registration” algorithm.
 \\
2012--2012&{\bf Privacy-Preserving Smart Meter system}\\&
This is the implementation of a paper by Dr. Hsiao-Ying Lin. The project is a
simulation in software of so-called privacy-preserving smart meter, a hypothetic hardware. This hardware, being installed in houses
can bill the electricity (network traffic, gas consumption etc.) in privacy-preserving manner, that is, without revealing the sensitive data about
per-month consumption to provider. Thus, it somehow allows seeing sums without seeing addends. This is achieved by using techniques from secure
aggregation, subfield of cryptography. I've implemented the project using openSSL library on C language in one semester.
. The code, documentation and all the rights belong to Hsiao-Ying Lin.
\\
2011--2012&{\bf "Privacy-Preserving Keyword Search" project.}\\
&
This project was done in partial fulfilment of the graduation requirements for the Computer Science department in NCTU. Thee project was done together with Sean Lin in
a team of two. Our advisors were Prof. Bao-Shuh Paul Lin (Chair Professor, Department of Computer Science, NCTU) and Dr.Hsiao-Ying Lin (Research Fellow, Information
and Communication Technology Laboratories, NCTU). Thee project was financed with
the Diamond Project Scholarship. Within one year we have delivered the prototype of
a system that was able to do the keyword search in privacy-preserving way. During
the project we have learned about important issues, such as cryptographic security and
pseudo-randomness.
\\
\end{tabular}

 
\section*{Programming Skills}
\begin{tabular}{L!{\VRule}R}
	Lisp (guile) & 8 years of experience; favourite language\\
	C/C++ & 8 years of experience\\
	Java (Android) & 6 years of experience\\
	Javascript & 2 years of experience\\
	C\#(Unity)&1 year of experience in game development
\end{tabular}
\section*{Languages}
\begin{tabular}{L!{\VRule}R}
Russian&Mother tongue\\
Ukrainian&Fluent\\
English&Fluent (TOEFL iBT score 112/120 in 2012)\\
Chinese&Intermediate (TOCFL 2012, level 3/5, score 96/100)\\
Japanese&Fluent\\
\end{tabular}

\section*{Hobbies}
Skiing, swimming, weightlifting, Android programming.

{\vspace{20pt}
\vspace{20pt}

\end{document}
