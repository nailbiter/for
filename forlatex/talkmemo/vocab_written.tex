\documentclass[12pt]{article} % use larger type; default would be 10pt

%%\usepackage[T1,T2A]{fontenc}
%%\usepackage[utf8]{inputenc}
%%\usepackage[english,ukrainian]{babel} % може бути декілька мов; остання з переліку діє по замовчуванню. 
\usepackage{enumerate}
\usepackage{xeCJK}
\usepackage{mystyle}
\usepackage{ruby}
\usepackage{longtable}
\usepackage{hyperref}

%\setCJKmainfont[AutoFakeBold=true]{Hiragino Mincho Pro} %my Mac
\setCJKmainfont{MS PGothic} %AJP windows
%\renewcommand\rubysep{-5ex}
\newcommand{\kana}[2]{\ruby{#1}{#2}}
\newcommand{\mytabra}[1]{$\myabra{\mbox{#1}}$}

\title{Useful Vocabulary}
\begin{document}
\def\A{\mytabra{A}}
\def\B{\mytabra{B}}
\def\C{\mytabra{C}}
\def\same{-------- "" --------}

	\maketitle
	\textbf{``smart'' words:}\\
	\begin{longtable}[]{l|l|l}
		wise/intelligent&\kana{賢明}{けんめい}&返事をしないのが、もっとも賢明でしょう。\\
		to prevent(t)&\kana{防ぐ}{ふせぐ}&ミスが防げる\\ 
		whether one likes or not&\kana{こ}{好}むと\kana{こ}{好}まざるとにかかわらず\\
	\end{longtable}
%%\begin{thebibliography}{9}
%%\bibitem{gelbaum}Gelbaum, B.R. and Olmsted, J.M.H.. Counterexamples in Analysis. Dover Publications. 2003
%%\end{thebibliography}
\end{document}


