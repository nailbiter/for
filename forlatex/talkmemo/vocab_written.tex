\documentclass[12pt]{article} % use larger type; default would be 10pt

%%\usepackage[T1,T2A]{fontenc}
%%\usepackage[utf8]{inputenc}
%%\usepackage[english,ukrainian]{babel} % може бути декілька мов; остання з переліку діє по замовчуванню. 
\usepackage{enumerate}
\usepackage{xeCJK}
\usepackage{mystyle}
\usepackage{ruby}
\usepackage{longtable}
\usepackage{hyperref}

\setCJKmainfont[AutoFakeBold=true]{Hiragino Mincho Pro} %my Mac
%\setCJKmainfont{MS PGothic} %AJP windows
%\renewcommand\rubysep{-5ex}
\newcommand{\kana}[2]{\ruby{#1}{#2}}
\newcommand{\mytabra}[1]{$\myabra{\mbox{#1}}$}

\title{Useful Vocabulary}
\begin{document}
\def\A{\mytabra{A}}
\def\B{\mytabra{B}}
\def\C{\mytabra{C}}
\def\same{-------- "" --------}

	\maketitle
	\textbf{``smart'' words:}\\
	\begin{longtable}[]{l|p{0.3\textwidth}|p{0.4\textwidth}}
		to raise (a question)&\kana{提起}{ていき}&
		submission date (e.g. of a paper)&\kana{投稿}{トウコウ}日&\\
		wise/intelligent&\kana{賢明}{けんめい}&返事をしないのが、もっとも賢明でしょう。\\
		to prevent(t)&\kana{防ぐ}{ふせぐ}&ミスが防げる\\ 
		whether one likes or not&\kana{こ}{好}むと\kana{こ}{好}まざるとにかかわらず\\
		appropriate & \kana{相応しい}{フサワシイ}&よりふさわしい形である\\
		joint (e.g. submit article jointly)&\kana{連名}{レンメイ}&の論文としては、よりふさわしい形である連名で提出します。\\
		to protrude (e.g. too long text) & はみ出す&はみ出しているaffiliationの部分は改行の形で直しました\\
		しかし&ただ&\\
		bibliographic data&\kana{書誌}{ショシ}データ\\
		to be listed/shown (i)&\kana{羅列}{ラレツ}する&単なるpdfが羅列する形に
		  なっておりました\\
		to occur often (i)&\kana{多発}{タハツ}する&日本語の速報を見て盗作するケース
		が多発しておりますので、\\
		to plagiarize&\kana{盗作}{トウサク}する&\\
		bulletin (e.g. of a conference)&\kana{速報}{ソクホウ}&\\
		to submit (e.g. paper)&\kana{提出}{テイシュツ}する&\\
	\end{longtable}
	\section{Mail writing}
	\textbf{usefull patterns:}\\
	\begin{longtable}[]{p{0.55\textwidth}|p{0.6\textwidth}}
		お手数をおかけいたしまして、誠に\kana{恐縮}{キョウシェく}です。& It was a shame to trouble you.
	\end{longtable}
%%\begin{thebibliography}{9}
%%\bibitem{gelbaum}Gelbaum, B.R. and Olmsted, J.M.H.. Counterexamples in Analysis. Dover Publications. 2003
%%\end{thebibliography}
\end{document}


