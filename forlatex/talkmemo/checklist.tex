%make
\documentclass[12pt]{article} % use larger type; default would be 10pt

%%\usepackage[T1,T2A]{fontenc}
%%\usepackage[utf8]{inputenc}
%%\usepackage[english,ukrainian]{babel} % може бути декілька мов; остання з переліку діє по замовчуванню. 
\usepackage{enumerate}
\usepackage{mystyle}
\usepackage{xeCJK}
\usepackage{ruby}

%\renewcommand\rubysep{-4ex}
\newcommand{\kana}[2]{\ruby{#1}{#2}}
%\setCJKmainfont{MS PGothic} %AJP windows
\setCJKmainfont[AutoFakeBold=true]{Hiragino Mincho Pro} %my Mac

%%\usepackage{fancyhdr}
%%\pagestyle{fancy}
%%\fancyfoot[C]{text me at \href{mailto:leontiev@ms.u-tokyo.ac.jp}{leontiev@ms.u-tokyo.ac.jp} if there are mistakes/obscurities}
%%\fancyhead{}

\title{Checklist for preparing the talk (when using slides)}
\begin{document}
	\maketitle
    \noindent\textbf{while preparing the slides (\underline{till} the day of the report):}
    \begin{enumerate}
	    \item 関連論文を読むこと。目的は3つがある:\begin{enumerate}
			    \item 自分の結果が他人の結果とどういう関係にあるかを理解する。 
			    \item 別の視点からの質問に対して、的確な答えをするための準備をする。
			    \item 講演の直前まで、結果を強めるための材料にする。
		    \end{enumerate}
%%        \item read related articles (the objective is threefold: to see how your results relate to others'; to
%%            be prepared to questions asked by people with another prospective; to strengthen your own results);
        \item think about what questions may be asked; list them up;
    \end{enumerate}
	\noindent\textbf{while preparing the text of the speech:}
	\begin{itemize}
		\item Carefully read Your emails, so that You do not have to repeat the same thing over and over (make written english translation
			and save it);
		\item Carefully listen to Your advices regarding the possible issues (if applicable, include them to this list);
		\item if applicable, make sure that the phrase 「この研究は小林俊行先生との\kana{共同}{キョウドウ}
            研究です.」 occurs somewhere at the beginning of the speech;
        \item prepare the {\it checkpoints} (they should be no more distant than $\frac{1}{6}$ of talk's duration
            from one another);
	\end{itemize}
    \textbf{when rehearsing the speech (\underline{starting} $7\sim10$ days before the speech):}
	\begin{itemize}
		\item make sure I speak slow, so that if the time available for the talk is $x$, I can finish in precisely $x-\myabra{\mbox{5 minutes}}$;
	\end{itemize}
	\textbf{at the day of the talk:}
	\begin{itemize}
		\item have slides saved on usb stick with no other files on it;
		\item have slides printed in color one per A4 page, one-side printing, album orientation;
	\end{itemize}
	\textbf{at the beginning of my session\footnote{\normalfont if my session is not the first one during the day, do this at the end of the previous session to ensure that
	the staff is available}:}
	\begin{itemize}
		\item ensure that my talk will be in the same place;
		\item check the equipment:\begin{itemize}
%%				\item check that I can connect my PC to the equipment (or open my file on the host PC);
				\item コンピュータ接続がスメーズできることを確認する(或いは、自分のファイルをホストパソコンで開けられることを確認する);
				\item check that quality of image is acceptable (color, visibility etc.) -- \textit{make sure that I can connect faster than other lecturers};
				\item check that letters are big enough, so that the people on the bottom row of the room can see it;
				\item check that the screen size is appropriate (i.e. that image projected is not too big or too small for the screen);
			\end{itemize}
		\item make ``sleep time'' longer, so computer will not turn off after short inactivity (terrible while showing slides);
		\item check that pages {\bf do not} turn by themselves;
	\end{itemize}
	\textbf{during my session:}
	\begin{itemize}
		\item DO NOT TALK!!!;
		\item 友達を連れて来るのはダメ!;
		\item listen carefully to other speakers; note their names; try to connect my talk's content with at least 2 more speakers' talks ();
	\end{itemize}
    \textbf{during my talk:}\begin{enumerate}
	    \item チェックポイントで早すぎたら、ただ黙って立っていること;
%%        \item if you found yourself running ahead of {\it checkpoints}, just wait in front of slide calmly;
    \end{enumerate}
	\textbf{when answering questions after the talk:}
	\begin{itemize}
		\item if do not understand the question, say 「質問の意味が良くわからないのですが」 and make sure that \textit{I perfectly understood the question};
		\item if do understand the question, say 「ご質問の日本語の意味を正しく理解できているかどうかわかりませんが」 before the answer;
	\end{itemize}
	\textbf{after the talk:}
	\begin{itemize}
		\item if there are any printed materials (booklet with abstracts etc.), purchase them;
        \item send the report for the talk;
        \item if you haven't published this yet as an article, \textbf{do not} send any parts of slides;
            and even if you send some material, make sure to take off all the formatting so to exclude
            any of Professor Kobayashi's design know-how's;
	\end{itemize}
%%\begin{thebibliography}{9}
%%\bibitem{gelbaum}Gelbaum, B.R. and Olmsted, J.M.H.. Counterexamples in Analysis. Dover Publications. 2003
%%\end{thebibliography}
\end{document}


