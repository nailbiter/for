\documentclass[12pt]{article} % use larger type; default would be 10pt

%%\usepackage[T1,T2A]{fontenc}
%%\usepackage[utf8]{inputenc}
%%\usepackage[english,ukrainian]{babel} % може бути декілька мов; остання з переліку діє по замовчуванню. 
\usepackage{enumerate}
\usepackage{xeCJK}
\usepackage{mystyle}
\usepackage{ruby}
\usepackage{longtable}
\usepackage{hyperref}

\setCJKmainfont[AutoFakeBold=true]{Hiragino Mincho Pro} %my Mac
%\setCJKmainfont{MS PGothic} %AJP windows
%\renewcommand\rubysep{-5ex}
\newcommand{\kana}[2]{\ruby{#1}{#2}}
\newcommand{\mytabra}[1]{$\myabra{\mbox{#1}}$}

\title{Useful Vocabulary}
\begin{document}
\def\A{\mytabra{A}}
\def\B{\mytabra{B}}
\def\C{\mytabra{C}}
\def\same{-------- "" --------}

	\maketitle
	\textbf{structures:}\\
	\begin{longtable}[]{p{0.55\textwidth}|p{0.6\textwidth}}
		\mytabra{A}の立場からは\mytabra{B}の一般化とみなせます&seeing from the viewpoint of \mytabra{A}, it can be seen as a generalization of \mytabra{B};\\\hline
		これを\mytabra{A}と書くことにします&we will denote it by \mytabra{A};\\\hline
		\mytabra{A}よって\mytabra{B}とみなしているのです&using \mytabra{A} we view (this) as \mytabra{B};\\\hline
		これらの課題は、\mytabra{A}の研究を\kana{深化}{シンカ}させるものです&(solving) these problems is advantageous for the deeper understanding of \mytabra{A};\\\hline
		この\mytabra{A}は\mytabra{B}の一般論を\kana{今考}{イマカンガ}えようとしている場合に\kana{当}{ア}てはめたものですA&
		this \mytabra{A} is obtained by applying the general theory of \mytabra{B} to the case under consideration;\\\hline
		\same&\B で証明された一般理論を今の特別な設 定に\kana{適用}{てきよう}すると、以下の\A を\kana{得}{エ}ます\\\hline
		we can study \mytabra{A} instead of \mytabra{B}&\mytabra{B}の代わりに\mytabra{A}を研究に\kana{置き換え}{オキカエ}ることができます\\\hline
		\same &\B の代わりに\A を研究しても\kana{同値}{ドウチ}であるということがわかります。\\\hline
		\mytabra{A} is the keypoint&\mytabra{A}が\kana{鍵}{カギ}になります\\\hline
		\mytabra{A} is important feature in applications&\mytabra{A}が\kana{応用上}{オウヨウジョウ}重要なポイントになります\\\hline
		this \A is also used in studying \B that we introduced & この\A も、ここで求めた\B の性質を調べるに用いられます\\\hline
		\A はという\kana{問}{ト}いに答えるものです&\A answers the question \B\\\hline
		I would like to start with reviewing the well-known concept of \A&最初は、ここにいらっしゃる皆様が
		よくご存知\A という\kana{概念}{ガイネン}から復習始めたいと思います\\\hline
		the case \A is much studied starting from a long time ago, so that even \B is known&
		\A の場合は古くから多くの研究があり、\B さえ知られています。\\\hline
		in slides we use the abbreviation \A to denote \B&
		スライドでは{\B}の
		\kana{頭文字}{カシラモジ}を{とって}\A と\kana{略記}{リャクキ}することにします。\\\hline
		as an a priori estimate based on \A &\A におけるアプリオリ評価として\\\hline
		\A is uniformly bounded independent of \B&
		\A が\B よらずに一様に押さえられている\\\hline
		$B\mapsto A\in C$
		&\B からその\A を対応させることによって\C への写像が定まった ことになります\\\hline
		in the language of \A, the \B is the same as \C&
		\B というのは、\A の言葉では、\C ということと同値になる\\\hline
		\A is obtained by substituting \B in \C&
		\A は\C 、そこに\B を代入して得られるものです。\\\hline
		as a complete opposite to \A, \B&
		\A と正反対で、\B\\\hline
		the next result shows \A&次の結果によると、\A になります。\\\hline
		\same&次の定理は\A とういうことを表しています。\\\hline
		if we substitute \A in place of \B, we get the \C below&
		\B と\A の役割を入れ替えたものが下の\C になります。\\\hline
		finally, using the precedings and some more results, we can deduce \A&
		最後は、これらの\kana{緒}{しょ}結果をプラスアルファの結果から\A についてもの全て決定することができます。\\\hline
		it became first attempt to attack \A& \A に\kana{挑戦}{チョウセン}する最初の\kana{試み}{こころみ}になります。\\\hline
	\end{longtable}

	\vspace{1em}
	\textbf{``smart'' words:}\\
	\begin{longtable}[]{p{0.3\textwidth}|l|p{0.3\textwidth}}
		to reduce (to simple case)&\ldots に\kana{帰着}{キチャク}\\
		&\ldots までに\kana{簡易化}{カンイカ}する\footnote{more active}\\
		非常に&\kana{極めて}{キワメテ}&この問題が極めて難しい\\
		同じように&\kana{同様}{ドウヨウ}に\\
		fundamental (e.g. research etc.)&\kana{キバン}{キバン}的な\\
		作る&\kana{構成}{コウセイ}する\\
		to determine&\kana{決定}{ケッテイ}する\\
		at most&\kana{高々}{タカダカ}&
	\kana{一次独立}{イチジドクリツナ}
な対称性破れ作用素が\kana{高々}{タカダカ}\kana{有限個}{ユウゲンコ}
しかない設定は、この問題に対するwell-posedなcaseと考えられます。	\\
		to propose (etc. program, strategy)&\kana{提唱}{テイショウ}する\\
		so-called&\kana{所謂}{イワユル}&\dots いわゆるKobayashi Program A\\
		one by one&\kana{順}{ジュン}に&帰納的に順に分類する\\
		defined/induced by \A&\dots \A によって\kana{導}{ミチビ}かれます\\
		\same&\A からの\kana{誘導}{ユウドウ}して\kana{得}{エ}られる\\
		as a first step&\kana{第一歩}{ダイイッポ}として\\
		ここに書いた&ここに述べた\\
		\same &ここに\kana{記}{シル}した\\
		もっと詳しく&より\kana{精密}{セイミツ}に\\
		方法&\kana{手法}{シュホウ}\\
		different&\kana{異なる}{コトナル}\\
		previous results&\kana{先行}{センコウ}結果\\
		at this point of time&今の\kana{時点}{ジテン}で
		in general is not \A&\kana{最早}{もはや}\A になりません\\
		色々な&\kana{種々}{シュジュ}の\\
		higher rank (e.g. group)&\kana{高階}{コウカイ}&\kana{高階}{コウカイ}の群\\
		information is not lost&情報が\kana{失われない}{うしわれない}\\
		\A を持つ&\A を\kana{有します}{ゆうします}\\
		\A that belongs  to \B&\B に\kana{属する}{ゾクスル}\A&真ん中のベクトル空間に\kana{属する}{ゾクスル}超関数F\\
		special (singular)&\kana{特異}{とくい}\\
		前に&\kana{上記}{ジョウキ}で&\kana{上記}{ジョウキ}で構成した対称性破れ作用素が\\
		to exhaust (span) all \A &\A 全てを\kana{尽くし}{ずくし}&対称性破れ作用素が全てを\kana{尽くし}{ずくし}\\
		roughly/probably&\kana{恐らく}{おそらく}&これらの証明をfull paperにして書くには、
おそらく100ページから120ページが必要であり\\
		今&\kana{現在}{げんざい}\\
		briefly&\kana{手短か}{てみじか}&最後に、\kana{手短か}{てみじか}になりますが、今後の2年間の研究計画を述べます。
	\end{longtable}


	\vspace{1em}
	\textbf{terms (math):}\\
	\begin{longtable}[]{l|p{10cm}}
		power series expansion&\kana{級数}{キュウスウ}展開\\
		Legendre associated func.&ルジャンドル\kana{陪}{バイ}関数\\
		Zuckerman derived module&Zuckerman\kana{導来函手}{どうらいかんしゅ}\kana{加群}{かぐん}\\
		normalization&\kana{正規化}{せいきか}\\
		(mathematical) term&\kana{用語}{カナ}\\
		$C^\infty$ class&$C$無限\kana{級}{キュウ}\\
		nilpo radical&\kana{冪零根基}{ベキレイコンキ}\\
		product manifold & \kana{直積}{チョクセキ}多様体\\
		indefinite orthogonal group&{不定値\kana{直交}{チョッコウ}群}\\
		(indefinite) metric&(不定値)\kana{計量}{ケイリョウ}\\
		(real) flag manifold&(\kana{実}{ジツ}$\bullet$){\kana{旗}{ハタ}多様体}\\
		pseudo-Riemannian manifold&\kana{擬}{ギ}リーマン多様体\\
		complex (e.g. parameter)&複\kana{素}{ソ}数\\
		linearly independent&\kana{一次独立}{イチジドクリツ}\\
		(maxi/mini)mal parabolic subgroup&\kana{極}{キョク}(\kana{大}{ダイ}/\kana{小}{ショウ})\kana{放物型部分群}{ホウブツガタ部分群}\\
		inductively&\kana{帰納的に}{キノウテキニ}\\
		(integral) kernel&(積分)\kana{核}{カク}\\
		system of equations&\kana{連立}{レンリツ}方程式\\
		(double) coset space&(\kana{両側}{リョウガワ})\kana{剰余}{ジョウヨ}空間\\
		coset class&\kana{剰余類}{ジョウヨルイ}\\
		proposition&\kana{命題}{メイダイ}\\
		power (e.g. of two)&\kana{冪乗}{ベキジョウ}\\
		number (e.g. of cosets)&\kana{個数}{コスウ}\\
		composition (e.g. of functions)&\kana{合成}{ゴウセイ}\\
		branching (in representation theorem)&\kana{分岐則}{ブンキソク}\\
		reductive group&\kana{簡約}{カンヤク}群\\
		surjectivity&\kana{全射性}{ゼンシャセイ}\\
		meromorphic&\kana{有理型}{ユウリケイ}\\
		special values&\kana{特殊}{とくしゅ}値\\
		multiple (e.g. of number)&\kana{乗数倍}{じょうすうばい}\\
	\end{longtable}

	\vspace{1em}
	\textbf{misc snippets:\\}
	\begin{longtable}[]{p{0.3\textwidth}|p{0.7\textwidth}}
		standard beginning &
			初めまして。東京大学、数理科学研究科のレオンチエフ・アレックスと申します。
			今日は
			不定値\kana{直交}{チョッコウ}群$O(p,q)$の
			{対称性破れ作用素}というタイトルでお話ししたいと思います。
			最初は、設定から始めます。\\\hline
		when going to details&
		最後に\kana{手法}{シュホウ}について少し\kana{触}{フ}れましょう。\\\hline
		although we can prove more general, for simplicity of presentation we present special case&
		実はもう少し一般の公式も証明できているのですが、ここでの発表では記述を簡単にするため、\kana{少し}{スコシ}\\\hline
		let us list the important questions concerning these&これに関して重要な問題を\kana{挙}{ア}げましょう\\\hline
	\end{longtable}
	\textbf{speech texts used for the preparation of above:\\}
	\begin{enumerate}
		\item \href{https://drive.google.com/open?id=0Bx9ORoAf44_QZmhON1lWekR2LWs}{\texttt{math17talk.pdf}}
		\item \href{https://drive.google.com/file/d/0Bx9ORoAf44_QQU54WEUyUktWTm8/view?usp=sharing}{\texttt{symposium\_16.pdf}}
		\item \href{https://drive.google.com/file/d/0Bx9ORoAf44_QNHBmSWxCRkJoVUE/view?usp=sharing}{\texttt{学振スピーチ.pdf}}
		\item \href{https://drive.google.com/file/d/0Bx9ORoAf44_QVDU0THBCamt1SW8/view?usp=sharing}{\texttt{Kansai\_talk.pdf}}
		\end{enumerate}
%%\begin{thebibliography}{9}
%%\bibitem{gelbaum}Gelbaum, B.R. and Olmsted, J.M.H.. Counterexamples in Analysis. Dover Publications. 2003
%%\end{thebibliography}
\end{document}


