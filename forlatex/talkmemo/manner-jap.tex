%make
%mail manners
\documentclass[12pt]{article} % use larger type; default would be 10pt

%%\usepackage[T1,T2A]{fontenc}
%%\usepackage[utf8]{inputenc}
%%\usepackage[english,ukrainian]{babel} % може бути декілька мов; остання з переліку діє по замовчуванню. 
\usepackage{enumerate}
\usepackage{mystyle}
\usepackage{xeCJK}
\usepackage{ruby}

%\renewcommand\rubysep{-4ex}
\newcommand{\kana}[2]{\ruby{#1}{#2}}
%\setCJKmainfont{MS PGothic} %AJP windows
\setCJKmainfont[AutoFakeBold=true]{Hiragino Mincho Pro} %my Mac

%%\usepackage{fancyhdr}
%%\pagestyle{fancy}
%%\fancyfoot[C]{text me at \href{mailto:leontiev@ms.u-tokyo.ac.jp}{leontiev@ms.u-tokyo.ac.jp} if there are mistakes/obscurities}
%%\fancyhead{}

\title{Checklist for preparing the talk (when using slides)}
\begin{document}
\section{Basic mail template (when writing to 小林先生)}
\begin{verbatim}
小林先生

最近ご指導どうもありがとうございます!
お世話になっておりました。D2のアレックスです。

...

宜しくお願い申し上げます。
アレックス
\end{verbatim}
\section{Basic mail template (when writing to other professors)}
\begin{verbatim}
千葉大学
 渚 先生

お世話になっております。
(お手数をおかけいたしまして、誠に恐縮です。どうもありがとうございます。)

...

(遅くなって申し訳ございません。)
どうぞよろしくお願いします。
レオンチエフ アレックス
\end{verbatim}
\section{Special rules when answering to 小林先生}
\begin{enumerate}
	\item send both my version and incoming mail as cut-and-paste!;
\end{enumerate}
\end{document}


