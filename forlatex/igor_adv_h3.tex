\documentclass[8pt]{article} % use larger type; default would be 10pt

%\usepackage[utf8]{inputenc} % set input encoding (not needed with XeLaTeX)
\usepackage[10pt]{type1ec}          % use only 10pt fonts
\usepackage[T1]{fontenc}
%\usepackage{CJK}
\usepackage{graphicx}
\usepackage{float}
\usepackage{CJKutf8}
\usepackage{subfig}
\usepackage{amsmath}
\usepackage{amsfonts}
\usepackage{hyperref}
\usepackage{enumerate}
\usepackage{enumitem}
\usepackage{a4wide}

\newcommand{\norm}[1]{\left|\left|#1\right|\right|}

\title{Advanced Calculus, Exercise 16}
\author{}
\begin{document}
\maketitle
\begin{enumerate}
	\item{Indeed,\[f'(0)=\lim_{x\to0}\left(x+2x^2\sin\frac{1}{x}\right)/x=1+\lim_{x\to0}2x\sin\frac{1}{x}=1\neq 0\]
		Furthermore, $f$ is not locally invertible near the origin. For if it would be injective and continuous on some neighborhood of $(a,b)\ni0$,
		it would have to be (non-strictly) monotonic on that neighborhood. 
		
		To see this, assume, it is not so. Then, $f(x)$ cannot be constant
		(otherwise, it would be monotonic) and we may WLOG assume $\exists x,y\in (a,b),\;x<y,\;f(x)<f(y)$. 
		
		We claim $\forall u,v\in (a,b),\;u<v,\; f(u)<f(v)\implies f
		\text{ is increasing on } [u,v]$, for if $u<u'<v'<v,\;f(u')>f(v')$ and $f(u')>f(u)$
		, the value $\left(f(u')+\max\{f(u),f(v')\}\right)/2$ would be attained on disjoint intervals
		$(u,u')$ and $(u',v')$ by intermediate value theorem,
		thus contradicting injectivity of $f$. Consequently, if $u<u'<v'<v,\;f(u')>f(v')$ and $f(u')<f(u)\implies f(v')<f(v)$,
		the value $\left(f(v')+\min\{f(u'),f(v)\}\right)/2$ by IVT would be attained on $(u',v')$ and $(v,v)$, violating injectivity. And case $f(u)=f(u')$ is
		impossible, again by injectivity. This proves the claim and also shows that $f$ is increasing on $[x,y]$.

		Now, we shall show that $\forall z\in(a,b),\; z>y\implies f(z)>f(y)$ and by previous claim this will imply that $f$ should be increasing on $[y,b)$. For
		if $z_0>x,\;f(z_0)<f(x)$ ($f(z_0)=f(x)$ impossible by injectivity) IVT would imply that
		value $\left(f(y)+\max\{f(x),f(z)\}\right)/2$ is attained on both $(x,y)$ and $(y,z)$, violating injectivity. Similarly, $f$ is increasing on $(a,x]$ and
		therefore on $(a,b)$ (for example, if $\forall u\in(a,x] f(z)\leq f(x)$ and $\forall v\in [x,y] f(x)\leq f(v)\implies f(u)\leq f(v)$).

		All the reasoning above shows that if $f$ would be invertible near the origin, it would have to be monotonic there, and hence have derivative of same sign.
		However, as for $x\neq 0\;f'(x)=1-2\cos (1/x)+4x\sin(1/x)$, we see that for $u_n:=1/(\pi+2\pi n),\;v_n:=1/(2\pi n),\;n\in\mathbb{N}$ we have 
		$u_n,v_n\to 0$ and moreover for big $n$ $f'(u_n)>1,\;f'(v_n)<-1/2$. This shows that derivative of $f$ does not have constant sign on any neighborhood of $0$
		and also accidentally shows that $f'$ is not continuous at $0$. The latter violates the assumptions of inverse function theorem and thus it is not applicable
		to this example, so we have no contradiction.
		}
	\item{For convenience, we denote $g(x_1,x_2,x_3):=(1+x_1+x_2+x_3)^{-1}$. Indeed,
		\begin{gather*}
			Jf(x_1,x_2,x_3)=\begin{vmatrix}g-x_1g^2&-x_2g^2&-x_3g^2\\-x_1g^2&g-x_2g^2&-x_3g^2\\-x_1g^2&-x_2g^2&g-x_3g^2\end{vmatrix}=\\
				=(g-x_1g^2)(g-x_2g^2)(g-x_3g^2)-2x_1x_2x_3g^3-x_1x_2g^2(g-x_3g)-x_1x_3g^2(g-x_2g)-x_2x_3g^2(g-x_1g)=\\
				=g^3-g^4(x_1+x_2+x_3)=g^4(g^{-1}-x_1-x_2-x_3)=g^4
		\end{gather*}
		}
	\item{\begin{enumerate}[label=(\alph*)]
			\item{Indeed,
			\begin{gather*}
				\frac{\partial(x,y)}{\partial(r,\theta)}=
				\begin{vmatrix}\frac{\partial x}{\partial r}&\frac{\partial y}{\partial r}\\\frac{\partial x}{\partial\theta}&
					\frac{\partial y}{\partial\theta}\end{vmatrix}=
				\begin{vmatrix}\cos\theta&\sin\theta\\-r\sin\theta&r\cos\theta\end{vmatrix}=r\cos^2\theta+r\sin^2\theta
					=r
			\end{gather*}
				}
			\item{Similarly to above,
			\begin{gather*}
				\frac{\partial(x,y,z)}{\partial(r,\theta,\phi)}=
				\begin{vmatrix}
					\frac{\partial x}{\partial r}&\frac{\partial y}{\partial r}&\frac{\partial z}{\partial r}\\
					\frac{\partial x}{\partial\theta}&\frac{\partial y}{\partial\theta}&\frac{\partial z}{\partial\theta}\\
					\frac{\partial x}{\partial\phi}&\frac{\partial y}{\partial\phi}&\frac{\partial z}{\partial\phi}\\
				\end{vmatrix}=
				\begin{vmatrix}
					-\cos\theta\sin\phi&\sin\theta\sin\phi&\cos\phi\\
					r\sin\theta\sin\phi&r\cos\theta\sin\phi&0\\
					-r\cos\theta\cos\phi & r\sin\theta\cos\phi&-r\sin\phi
				\end{vmatrix}=\\
				=\cos\phi\begin{vmatrix}
					r\sin\theta\sin\phi&r\cos\theta\sin\phi\\
					-r\cos\theta\cos\phi&r\sin\theta\cos\phi\\
				\end{vmatrix}-r\sin\phi\begin{vmatrix}
				-\cos\theta\sin\phi&\sin\theta\sin\phi\\
				r\sin\theta\sin\phi&r\cos\theta\sin\phi\\
				\end{vmatrix}=\\
				=r^2\cos\phi\sin\phi\cos\phi\begin{vmatrix}\sin\theta&\cos\theta\\-\cos\theta&\sin\theta\end{vmatrix}-
				r^2\sin\phi\sin^2\phi\begin{vmatrix}-\cos\theta&\sin\theta\\\sin\theta&\cos\theta\end{vmatrix}=\\
				=r^2\sin\phi(\cos^2\phi+\sin^2\phi)=r^2\sin\phi
			\end{gather*}
				}
		\end{enumerate}
		}
	\item{Note, that \[\frac{\partial(u,v)}{\partial(x,y)}=\begin{vmatrix}2x&-2y\\2y&2x\end{vmatrix}=4(x^2+y^2)>0\text{ whenever } (x,y)\neq 0\]
			Hence, by inverse function theorem (as mapping is obviously $C^1$, even $C^{\infty}$) map is locally invertible on $\mathbb{R}^2\setminus\{(0,0)\}$.
			}
		\item{First, note that \[J(x,y)=\frac{1}{(x^2+y^2)^2}\begin{vmatrix}2x(x^2+y^2)-2x(x^2-y^2)&-2y(x^2+y^2)-2y(x^2-y^2)\\
			y(x^2+y^2)-2x^2y&x(x^2+y^2)-2y^2x\end{vmatrix}\implies J(0,1)=0\]
			Hence, even if the local inverse will exist, it will be non-smooth. And even such inverse does not exist, for all the points
			of the set $\{(0,y)\in\mathbb{R}^2\mid y\neq 0\}$ are mapped to $(-1,0)$. Since any neighborhood of $(0,1)$ contains infinitely many members of
			that set, mapping is not injective on any neighborhood of $(0,1)$, hence is not locally invertible near the point of interest. Thus, answer is negative
			}
\end{enumerate}
\end{document}
