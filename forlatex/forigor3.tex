
\documentclass[10pt]{article} % use larger type; default would be 10pt

%%\usepackage[T1,T2A]{fontenc}
%%\usepackage[utf8]{inputenc}
%%\usepackage[english,ukrainian]{babel} % може бути декілька мов; остання з переліку діє по замовчуванню. 
\usepackage{enumerate}
\usepackage{CJKutf8}
\usepackage{mystyle}

\title{}
\author{}
\begin{document}
\maketitle
%104 год номер 4, 103 год номер 6, 101 год номер 8
\begin{myprob}
	Let $C$ be a simple closed curve \dots.
\end{myprob}
Indeed, Stokes theorem tells us (we shall denote by $A$ the region enclosed by $C$)
\[\int_Czdx-2xdy+3ydz=\int_A\mbox{curl}(zdx-2xdy+3ydz)\cdot\mathbf{n}=\]
\[=\int_A(3dxdy+dxdz-2dydz)\mathbf{n}\]
and as normal vector for plane $x+y+z=1$ is $\mathbf{n}=(1/\sqrt{3},1/\sqrt{3},1/\sqrt{3})$ we have
\[\int_A(3dxdy+dxdz-2dydz)\mathbf{n}=\mynorm{A}\frac{3+1-2}{\sqrt{3}}\]
where $\mynorm{A}$ denotes the area of $A$ and this expression clearly depends only on area.

\begin{myprob}
	Find the positive oriented simple \dots.
\end{myprob}
Let us denote the region that this curve encloses as $A\subset\R^2$ and apply Stokes theorem that would give us
\[\int_C(y^3-y)dx-2x^3dy=\int_A(3y^2-1)dydx-6x^2dxdy=\int_A(1-3y^2-6x^2)dxdy\]
as we are interested in maximizing, integration should not be done over the regions where $1-3y^2-6x^2<0$ and should be done over
the whole region where $1-3y^2-6x^2\geq0$, hence $A=\mysetn{(x,y)\in\R^2}{1-3y^2-6x^2}$, and $C$ is its boundary, which is the ellipse
\[3y^2+6x^2=1\]
\begin{myprob}
	Let $\vec{F}=(P,Q,R)$\dots
\end{myprob}
We are going to prove this statement. It is in fact enough to show that for any two surface "pulled" on $\Gamma$ the integration
will give the same result. In turn, for this it is sufficient to show that integrating $\vec{F}$ over the surface of any closed solid
will give zero. To show, this for some particular solid $V$ we may apply Stokes theorem to see that
\[\oiint_\partial \vec{F}\cdot\vec{n}_{\Sigma}dA=\iiint_V\mybra{\frac{\partial P}{\partial x}+\frac{\partial Q}{\partial y}+\frac{\partial R}
{\partial z}}dxdydz=0\]
as expression in under the rightmost integral is zero by hypothesis.

\end{document}
