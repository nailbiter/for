\documentclass[12pt]{article} % use larger type; default would be 10pt

\usepackage{enumerate}
\usepackage{mystyle}
\usepackage{amsthm}
\usepackage{xeCJK}

%%\usepackage{fancyhdr}
%%\pagestyle{fancy}
%%\fancyfoot[C]{text me at \href{mailto:leontiev@ms.u-tokyo.ac.jp}{leontiev@ms.u-tokyo.ac.jp} if there are mistakes/obscurities}
%%\fancyhead{}

\theoremstyle{theorem}
\newtheorem{problem}{Problem}
\newtheorem{question}{Question}
\theoremstyle{definition}
\newtheorem{answer}{My Answer}
\newtheorem{reason}{Reasons}
\theoremstyle{remark}
\newtheorem{countermeasure}{Counter-measures}
\newtheorem{remark}{Remark}
\newtheorem*{remark*}{Remark}

\setCJKmainfont{Hiragino Mincho Pro}

\title{Reflection}
\begin{document}

	\maketitle

\section{Trivia}
\begin{center}
	\begin{tabular}[]{l|l}
		Date:&September 21, 2017\\
		Conference/Seminar's name:& 2017 年 9 月河野研集中セミナー\\
		Place:& 東大駒場、数理棟、128教室\\
		Title:&{2つのゲーゲンバウアー多項式に関連する積分公式について}\\
		Expected Duration:&90 min\\
        Real Duration:& \textasciitilde 65/75 min\footnotemark\\
		Self-evaluation:& 6 out of 10\\
        Voice record:&{\ttfamily alex.mp3} in \url{https://www.dropbox.com/s/srkyolm6uidydmq/voice.zip?dl=0}
	\end{tabular}
\end{center}\footnotetext{
        before questions and discussion/after all questions and discussion}
\section{Timing}
\begin{center}
	\begin{tabular}[]{l|l|l}
        checkpoint name (see {\ttfamily mytalk.pdf})&expected time&real time\\\hline
        prop3&15 min&15 min\\
        propconvform&30 min&30 min\\
        exdf&45 min&\\
        lemredpf&60 min&\\
        m2&75 min&\\
        bib&90 min&\\
	\end{tabular}
\end{center}

\section{Questions asked during the talk and my answers}
\begin{question}[Tamori--kun]
    It was mentioned that the integral expressions presented have original appeared
    in computations of (generalized) eigenvalues of SBO's. Could you explain precisely, which one occured?
\end{question}
\begin{answer}
    In fact, both the very first one (the very special one)
    \begin{equation}
        \int_{-1}^1\int_{-1}^1\myabs{s-t}^{2\nu}(1-s^2)^{\lambda-\frac{1}{2}}(1-t^2)^{\beta-\frac{1}{2}}dsdt
        \label{eqn:special}
    \end{equation}
    and the very general one\begin{equation}
        \label{eqn:general}
        \int_{-1}^1\int_{-1}^1\myabs{s-tx}^{2\nu}(1-s^2)^{\lambda-\frac{1}{2}}(1-t^2)^{\beta-\frac{1}{2}}
        C^\lambda_{\ell}(s)C^\mu_m(t)dsdt
    \end{equation}
    appeared. The latter one appeared somehow indirectly:
    more precisely, the eigenvalues of SBO in our setting can be given as follows
    \begin{equation}
        \frac{\Gamma\Gamma\cdots\Gamma}{\Gamma\Gamma\cdots\Gamma}\int_{-1}^1(1-x^2)^\alpha\cdot
        \eqref{eqn:general}ds
    \end{equation}
    Now, I have said before that being able to compute eigenvalues is important for studying SBO's (for the simplest example,
    we can say that SBO is zero if and only if all of its eigenvalues are). I have also said
    that the integral expressions we have derived appeared in these eigenvalues, so you might have had
    impression that we can completely compute eigenvalues of SBO's. Unfortunately, this is not so. Indeed,
    using the beta integral, the latter expression can be rewritten as
    \begin{equation*}
        \frac{\Gamma\Gamma\cdots\Gamma}{\Gamma\Gamma\cdots\Gamma}{}_3F_2(;1).
    \end{equation*}
    Now, as I have mentioned before, if all five complex parameters of ${}_3F_2(;1)$ are 
    independent, it can be shown that it cannot be represented as the product of Gamma functions.
    Unfortunately, it turns out that precisely all 5 parameters are independent in the latter formula,
    hence we cannot express eigenvalues of SBOs as product of Gamma functions in general.
    
    Nevertheless, we still might be able to determine when ${}_3F_2(;1)$ gets zero, but this is 
    a bit more involved and we are working on this now.
\end{answer}
\begin{question}[Professor Kohno]
    Can we further generalize the results (perhaps, by using integral formulae with more complex
    parameters, cf. Aomoto--Gelfand integral etc.)
\end{question}
\begin{answer}
    Actually, we might be able to. I have said before that one of the questions that
    motivated us was ``Can we generalize \eqref{eqn:special}?'' However, it was a bit of a lie,
    because as I just said to Tamori--san, even the most general of our integral expressions appeared
    in the study of SBOs and this (rather than the mere generalization) was our original motivation.
    Hence, it may well be that further generalization is possible.
\end{answer}
\section{Problems, their reasons and counter-measures}
\begin{problem}
\end{problem}
\begin{reason}[]
\end{reason}
\begin{countermeasure}
\end{countermeasure}
\section{Miscellaneous comments}
\begin{enumerate}
    \item I would like to extremely thank Tamori--kun. His questions gave additional ten minutes to
        my talk, so it was a bit less embarrassing.
    \item This was my second time to use TeXmacs for the talk (the first one was Representation Theory
        Symposium in Okinawa in November). I consider this attempt to be the first successful one.
        Although overall talk went bad, the main reason is the time management, not technical restraints
        (somehow opposite to what happened in Okinawa).
\end{enumerate}
\end{document}
