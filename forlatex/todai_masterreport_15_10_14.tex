
\documentclass[10pt]{article} % use larger type; default would be 10pt

%%\usepackage[T1,T2A]{fontenc}
%%\usepackage[utf8]{inputenc}
%%\usepackage[english,ukrainian]{babel} % може бути декілька мов; остання з переліку діє по замовчуванню. 
\usepackage{enumerate}
\usepackage{CJKutf8}
\usepackage{mystyle}
\usepackage{amscd}
\usepackage{extarrows}

\newcommand{\D}{\mathcal{D}}

\theoremstyle{definition}
\newtheorem{mydef}{Definition}[section]
\theoremstyle{remark}
\newtheorem*{remark}{Remark}

\title{}
\author{Alex Leontiev}
\begin{document}
\newcommand{\sone}{$\mybra{\D'(G\times_P\C_{\lambda-n})\otimes\C_\nu}^{\Delta(P')}$}
\newcommand{\Upp}{\mysetn{(x,y)\in\R^{p,q}}{x\neq0,\;y\neq0}}
\newcommand{\Stab}{O(p,q)_{e_p}}
\newcommand{\sol}[1][\R^{p,q}]{\mathcal{S}ol(#1;\lambda,\nu)}
\newcommand{\solXi}{\sol[\Xi]}
\maketitle
\section{Definitions and Notations}
We fix $p,q\in\Z_{\geq0}$, $n:=p+q$, $G:=O(p+1,q+1)$ and $G':=O(p+1,q+1)_{e_{p+1}}:=\mysetn{g\in G}{g\cdot e_{p+1}=e_{p+1}}\simeq
O(p,q+1)$.

We then set,
\[M:=\mysetn{ \begin{bmatrix}\epsilon&0&0\\0&A&0\\0&0&\epsilon\end{bmatrix}}{A\in O(p,q),\;\epsilon=\pm1}\]
\[N_+:=\mysetn{
\begin{bmatrix}
	1-\frac{ \myabs{ v }^2-\myabs{ w }^2 }{2}&-v^T&w^T&\frac{ \myabs{ v }^2-\myabs{ w }^2 }{2}\\
	v&I_{ p }&0&-v\\w&0&I_q&-w\\-\frac{ \myabs{ v }^2-\myabs{ w }^2 }{2}&-v^T&w^T&1+\frac{ \myabs{ v }^2-\myabs{ w }^2 }{2}\\
\end{bmatrix}}{v\in\R^p,\;w\in\R^q}\]
\[A:=\mysetn{a'(t):= 
	\begin{bmatrix}
		\cosh(t)&0&\sinh(t)\\0&I_{ p+q }&0\\\sinh(t)&0&\cosh(t)
	\end{bmatrix}
}{t\in\R}.\]
We also let $M':=M\cap G'$, $N_+':=N_+\cap G'$, $A':=A\cap G'=A$ and $P':=M'A'N'=P\cap G'$.

We next set $\R^{ p+1,q+1 }\supset\Xi:=\mysetn{ (x,y)\in\R^{ p+1,q+1 }\setminus \left\{ 0 \right\}}{\myabs{x}=\myabs{y}}$.
$\Xi$ is clearly invariant under left multiplication by the elements of $G$.
\begin{mydef}
	Having $(\lambda,\nu)\in\C^2$ and given $f\in \D'(\Xi)$ we say that $f\in\solXi$
	if $\forall t\in\R^\times,\;f(t\cdot)=\myabs{t}^{\lambda-n}f(\cdot)$ and 
	$\forall m'a'(t)n'\in P'$
	we have $f(m'a'(t)n'\cdot)=e^{ t\nu }f(\cdot)$.
\end{mydef}

We further let $X:=\Sp^p\times\Sp^q/\left\{ \pm1 \right\}$. Note that the action of $G$
on $\Xi$ induces the action on $X=\Xi/\R^\times$. It turns out that the stabilizer $P\subset G$ of $p_+:=[1,0_{p+q},1]\in X$
is the maximal parabolic subgroup of $G$, hence $X\simeq G/P$ via $gP\mapsto g\cdot p_+$. Having maximal parabolic subgroup
fixed, we may also fix $\lambda\in\C$ and define a $G$-equivariant line bundle
$\mathcal{L}_{\lambda}:=G\times_P\C_\lambda$ over $G/P\simeq X$. We shall be interested
in the vector space $\mybra{\D'(G\times_P\C_{\lambda-n} )\otimes\C_\nu}^{\Delta(P')}$ for $(\lambda,\nu)\in\C^2$ fixed.

We also define the embedding \[\R^{ p,q }\ni(u,v)\mapsto \mathfrak{n}_-(u,v):=\mysbra{\frac{
	(1-\myabs{ u }^2+\myabs{ v }^2,2u,2v,1+\myabs{ u }^2
-\myabs{ v }^2)}{\sqrt{(1-\myabs{u}^2+\myabs{v}^2)^2+4\myabs{v}^2}}}\in X\simeq G/P\].
	For $x,y\in\R^{ p,q }$
	we set $Q(x,y):=x^TI_{ p,q }y$ and $Q(x):=Q(x,x)$.
\begin{mydef}
 For $F\in\D'(U)$, where $U\subset\R^{ p,q }$ is an open set, we say that
	$F$ is $N_+'$-invariant on $U$ if 
	$\forall b\in\R^{ p,q }$ with $b_p=0$ and $x_0\in U$ such that
	$\frac{ x_0-Q(x)b }{ 1-2Q(x_0,b)+Q(x_0)Q(b) }\in U$ and the expression makes sense (i.e. 
	the denominator is non-zero) we have \[\myabs{ 1-2Q(b,x)+Q(x)Q(b) }^{ \lambda-n }F
	\mybra{\frac{ x-Q(x)b }{ 1-2Q(x,b)+Q(x)Q(b)}}=F(x)\] equality holding for $x$ near $x_0$.
\end{mydef}
%%\begin{remark}
%%It should be noted that although the map $\mysetn{b\in\R^{p,q}}{b_p=0}\times\R^{p,q}\ni(b,x)\mapsto
%%\frac{ x-Q(x)b }{ 1-2Q(x,b)+Q(x)Q(b)}\in\R^{p,q}$ {\it does not} form an action of $N_{+}'$ on $\R^{p,q}$ in the sense
%%of a Lie group acting on manifold (indeed, this map doesn't even always make sense), if one uses the concept of a local group
%%of a local transformation group, as introduced, say in \cite{olver2000applications}, this ``action'' can be interpreted.
%%Consequently, it induces the corresponding action on $\D'(U)$ and all in all, the word ``invariant'' in the above definition makes
%%sense.
%%\end{remark}
\begin{mydef}
		 For $F\in\D'(U)$, where $U\subset\R^{ p,q }$ is the open set,
			we say that $F\in\sol[U]$ if the following holds:
			\begin{enumerate}
				\item if $x_0\in U$ and $-x_0\in U$, then $F(x)=F(-x)$ for $x$ near $x_0$;
				\item if $(m,x_0,m\cdot x_0)\in \Stab\times U\times U$, then $F(x)=F(m\cdot x)$ for
					$x$ near $x_0$, where $\Stab:=\mysetn{g\in O(p,q)}{g\cdot e_p=e_p}$;
				\item if $(\alpha,x_0,\alpha x_0)\in \R_{>0}\times U\times U$, then $\alpha^{
					\lambda-\nu-n}F(x)=F(\alpha x)$ for $x$ near $x_0$;
				\item $F$ is $N_+'$-invariant on $U$.
			\end{enumerate}
\end{mydef}
\begin{remark}
	If we trivialize $G\times_P\C_{\lambda-n}$ over simply-connected open $\mathfrak{n}_-(\R^{p,q})$ (the image of $\R^{p,q}$
	under the embedding $\mathfrak{n}_-:\R^{p,q}\to G/P$ defined above),
	these are precisely elements of $\sol[\R^{p,q}]$
	that are the pullbacks of elements of 
	$\mybra{\D'(G\times_P\C_{\lambda-n})\otimes\C_\nu}^{\Delta(P')}$
	from $X\simeq G/P$.
	In fact, due to the $P'\cdot\mathfrak{n}_-(\R^{ p,q })=X$ holding, given the element of $\sol[\R^{p,q}]$,
	it can be uniquely matched to an element of
	$\mybra{\D'(G\times_P\C_{\lambda-n} )\otimes\C_\nu}^{\Delta(P')}$
	on $X$.
\end{remark}
\section{Non-equivalence of $N_+'$-invariance and $\mathfrak{n}'_+$-invariance}
It is tempting to replace in the latter definition
the fourth condition (invariance under $N_+'$) with its differential (which we'll call $\mathfrak{n}_+'$-invariance)
, which will then give the family of differential equations that need to be satisfied:
\begin{equation}
\mysbra{(\lambda-n)\varepsilon_jx_j-\varepsilon_jx_jE+\frac{ 1 }{ 2 }Q(x)\frac{ \partial}{ \partial x_j } }F=0
	\label{Ndiff}
\end{equation}
where $E:=\sum_j x_j\frac{ \partial }{ \partial x_j }$ and $\varepsilon_j=+1$ for $1\leq j\leq p$ and $=-1$ otherwise.
However, if $p,q>0$ the resulting system of conditions will be strictly weaker than the original one. Indeed, having
fixed $(\lambda,\nu)\in\C^2$ such that $\Re(-\nu)>0$ and $\Re(\lambda+\nu-n)>0$ consider two distributions
\[K_+(x):=\myabs{ x_p }^{ \lambda+\nu-n }Q_+^{ -\nu }(x)\]
\[K_-(x):=\myabs{ x_p }^{ \lambda+\nu-n }Q_-^{ -\nu }(x)\]
with $Q^{-\nu}_{\pm}(x):=\mathbf{1}_{\left\{ \pm Q>0 \right\}}Q^{\nu}(x)$, where $\mathbf{1}_A(x)$ denotes the
characteristic function of a set $A$.
It is directly verified that both satisfy first three properties together with differential of the fourth one. Nevertheless,
they cannot be pullbacks of elements of \sone. Indeed, $P'$ acting on $G/P$
has only one open orbit, which under $\mathfrak{n}_-(\cdot)$ is pulled back to $\R^{ p,q }\supset\mycbra{ Q\neq0 }$.
Hence, if pullback of element of \sone is supported on subset of $\mycbra{ Q\neq0 }$
it should be supported on the whole $\mycbra{ Q\neq0 }$. However, $\supp(K_+)=\mycbra{ Q\geq0 }$ and
$\supp(K_-)=\mycbra{ Q\leq0 }$. Hence, none of these is pullback of element of 
$\mybra{\D'(G\times_P\C_{\lambda-n} )\otimes\C_\nu}^{\Delta(P')}$ distribution.
\section{The kernel of regular symmetry breaking operator}
In this section we construct an element of $\sol$, that is supported on $\R^{ p,q }$ under some mild
assumptions on the complex parameters. More precisely,
\begin{myprop}\label{prop-regular}
	Let $(\lambda,\nu)\in\C^2$ be such that $\Re(\lambda+\nu-n)>0$ and $\Re(-\nu)>0$. Then
	$K_{ \lambda,\nu }^{ \R^n }(x):=\myabs{ x_p}^{ \lambda+\nu-n }\myabs{ Q(x) }^{ -\nu }$ is a continuous function,
	which when seen as a generalized function on $\R^{ p,q }$, is supported on $\R^{ p,q }$ and is an element of $\sol$.
\end{myprop}
For the proof, we will need one technical result.
\begin{mylem}\label{lem-3}
	Let $f\in \solXi$. Then it's pullback under $\iota:R^{p,q}\ni(u,v)\mapsto
	(1-\myabs{ u }^2+\myabs{ v }^2,2u,2v,1+\myabs{ u }^2-\myabs{ v }^2)\in\Xi$ gives an element of $\sol$.
\end{mylem}
\begin{myproof}
	We recall that the restriction is a continuous function 
	\[r:\D'(\Xi)\supset D_{\lambda-n}:=\mysetn{f\in\D'(\Xi)}{f(t\cdot)=t^{\lambda-n}f(\cdot)}\to\D'(\R^{p,q})\]
	such that for $\varphi\in C(\Xi)\cap D_{\lambda-n}$ we have $r(\varphi)(x)=\varphi(\iota(x))$.
	Note that $\solXi\subset D_{\lambda-n}$ by definition.
	
	We have to show that for $f\in\sol[\Xi]$
	\begin{enumerate}
		\item $r(f)(x)=r(f)(-x)$;
		\item if $m\in O(p,q)_{e_p}$, then $r(f)(x)=r(f)(m\cdot x)$;
		\item if $\alpha>0$, then $\alpha^{\lambda-\nu-n}r(f)(x)=r(f)(\alpha x)$;
		\item $\forall (b,x_0)\in\R^{ p,q }\times\R^{p,q}$ with $b_p=0$ and
				${ 1-2Q(x_0,b)+Q(x_0)Q(b) }\neq0$
				we have \[\mybra{ 1-2Q(b,x)+Q(b)Q(b) }^{ \lambda-n }r(f)
				\mybra{\frac{ x_0-Q(x)b }{ 1-2Q(x_0,b)+Q(x_0)Q(b)}}=r(f)(x)\] equality holding for $x$ near $x_0$.
	\end{enumerate}
	Now, stated and proven below lemma \ref{lem-3-1} grants first and the second item. Similarly, lemmas \ref{lem-3-2} 
	and \ref{lem-3-3} (both stated and proven below) grant third and fourth item respectively.
\end{myproof}
\begin{mylem}\label{lem-3-1}
	Let $f\in \solXi$. Then it's pullback $r(f)$ under $\iota:R^{p,q}\ni(u,v)\mapsto
	(1-\myabs{ u }^2+\myabs{ v }^2,2u,2v,1+\myabs{ u }^2-\myabs{ v }^2)\in\Xi$ satisfies $r(f)(x)=r(f)(-x)$
	and $r(f)(m\cdot)=r(f)(\cdot)$ for $\forall m\in O(p,q)_{e_p}$.
\end{mylem}
\begin{myproof}
	We start with proving the first assertion. Consider the diagram
\[\begin{CD}
		D_{\lambda-n}     @>F>>  D_{\lambda-n}\\
@VVrV        @VVrV\\
\D'(\R^{p,q})     @>\widetilde{F}>>  \D'(\R^{p,q})
\end{CD}\]
Where $F(f)(x):=f(-m_0x)$ and $\widetilde{F}(f)(x)=f(-x)$ where $m_0:=
\begin{bmatrix}-1&0&0\\0&I_{p+q}&0\\0&0&-1\end{bmatrix}\in M'$.
As this diagram is commutative if we start with an element of $C(\Xi)\cap D_{\lambda-n}$ and the latter subspace
is dense in $D_{\lambda-n}$, we see that the diagram above commutes. As $f\in\solXi$ should satisfy $f(m_0\cdot)=f(\cdot)$ and
$f(-x)=f(x)$, we see that $F(f)=f$ and this proves the first assertion.

The derivation of the
second assertion follows the similar pattern. We fix arbitrary $\widetilde{m}\in O(p,q)_{e_p}$ and let $F(f)(x)=f(m x)$ with 
$m:=\begin{bmatrix}1&0&0\\0&\widetilde{m}&0\\0&0&1\end{bmatrix}\in M'$ and $\widetilde{F}(f)(x):=f(\widetilde{m} x)$. Finally,
we use the fact that $f(m\cdot)=f(\cdot)$ for $f\in\solXi$.
\end{myproof}
\begin{mylem}\label{lem-3-2}
	Let $f\in \solXi$. Then it's pullback $r(f)$ under $\iota:R^{p,q}\ni(u,v)\mapsto
	(1-\myabs{ u }^2+\myabs{ v }^2,2u,2v,1+\myabs{ u }^2-\myabs{ v }^2)\in\Xi$ satisfies 
	$\alpha^{\lambda-\nu-n}r(f)(x)=r(f)(\alpha x)$ for $\forall\alpha>0$.
\end{mylem}
\begin{myproof}
	We consider the same commutative diagram, as in proof of lemma \ref{lem-3-1}, except that we let $F(f)(x):=f(a_tx)$
	where $a_t
	\begin{bmatrix}
		\cosh(t)&0&\sinh(t)\\0&I_{ p+q }&0\\\sinh(t)&0&\cosh(t)
	\end{bmatrix}\in A'$ for some fixed $t\in\R$ and let $\widetilde{F}(f)(x):=e^{(\lambda-n)t}f(e^{-t}x)$.

	Similarly to proof of lemma \ref{lem-3-1}, diagram turns out to be commutative and as for $f\in\solXi$ we should have
	$F(f)(\cdot)=e^{\nu t}f(\cdot)$, hence we have $r(f)(e^t\cdot)=e^{(\lambda-\nu-n)t}f(\cdot)$, which is equivalent to what
	we wanted to show.
\end{myproof}
\begin{mylem}\label{lem-3-3}
	Let $f\in \solXi$. Then for it's pullback $r(f)$ under $\iota:R^{p,q}\ni(u,v)\mapsto
	(1-\myabs{ u }^2+\myabs{ v }^2,2u,2v,1+\myabs{ u }^2-\myabs{ v }^2)\in\Xi$ we have
	$\forall (b,x_0)\in\R^{ p,q }\times\R^{p,q}$ with $b_p=0$ and
	${ 1-2Q(x_0,b)+Q(x_0)Q(b) }\neq0$
	we have \[\mybra{ 1-2Q(b,x)+Q(b)Q(b) }^{ \lambda-n }r(f)
	\mybra{\frac{ x_0-Q(x)b }{ 1-2Q(x_0,b)+Q(x_0)Q(b)}}=r(f)(x)\] equality holding for $x$ near $x_0$.
\end{mylem}
\begin{myproof}%NB: in fact, here a *stronger* statement is proven: not ``some'' nbhd of x_0  as in hypothesis, but ``any''
	So let $b\in\R^{p,q}$ be fixed with $b_p=0$ and let $c_b(x):=1-2Q(b,x)+Q(x)Q(b)$. Let
	$U:=\subset\mysetn{x\in\R^{p,q}}{c_b(x)\neq0}$ (it's an open set). Note that for $\psi_b(x):=(x-Q(x)b)/c_b(x)$ 
	we have \[c_{-b}(\psi_b(x))=\frac{1}{c_b(x)}\neq0\]
	hence we have $\psi_{-b}\circ\psi_b=id$ equality holding on $U$. Thus, we see that $\psi_b$ on $U$ is a local diffeomorphism,
	hence $V:=\psi_b(U)$ is an open set and repeating the argument above with $b$ and $-b$ interchanged we see that $\psi_b$
	is a diffeomorphism $U\diffsm V$.

	We now consider the diagram
\[\begin{CD}
D_{\lambda-n}     @>F>>  D_{\lambda-n}\\
@VVrV        @VVrV\\
\D'(\R^{p,q})  @.     \D'(\R^{p,q})\\
@VV{\cdot\big|_V}V        @VV{\cdot\big|_U}V\\
\D'(V)     @>\widetilde{F}>>  \D'(U)
\end{CD}\]
Where $F(f)(\cdot):=f(n\cdot)$ with $n:=
\begin{bmatrix}
	1-Q(b)/2&-b^TI_{p,q}&Q(b)/2\\
	b&I_{p+q}&-b\\
	-Q(b)/2&-b^TI_{p,q}&1+Q(b)/2
\end{bmatrix}\in N_+'$ where $I_{p,q}$ denotes the diagonal matrix of $p+q$ columns
whose first $p$ diagonal entries are $+1$ and next $q$ are $-1$, and $\widetilde{F}(f)(\cdot):=\myabs{c_b(\cdot)}^{\lambda-n}
f(\psi_b(\cdot))$.

The diagram above commutes if we start with an element of $C(\Xi)\cap D_{\lambda-n}$ (note that $n\cdot\iota(x)=c_b(x)\iota(\psi_b(x))
$), hence commutes in general. Now, the statement to be shown is granted by the fact that for $f\in\solXi$ we should have
$f(n\cdot)=f(\cdot)$.
\end{myproof}
\begin{myproof}(of {\bf Proposition \ref{prop-regular}})\newline
	First, due to the assumption on complex parameters, we see that $K_{ \lambda,\nu }^{ \R^n }$ is indeed well-defined as a
	distribution. The statement about support is also evident.\par
	Now, let's consider distribution induced by continuous function on $\Xi$ given by
	$\Xi\ni\xi\mapsto\myabs{\xi_p}^{\lambda+\nu-n}\myabs{\xi_{p+q+2}-\xi_1}^{-\nu}$. We easily see that 
	it is well-behaved under $P'$ on $\Xi$ and pulls back to the multiple of $K_{ \lambda,\nu }^{ \R^n }$
	under the $\iota:R^{p,q}\to\Xi$. Hence,
	lemma \ref{lem-3} implies the desired conclusion.
\end{myproof}
\section{The kernel of singular symmetry breaking operator supported on $\left\{ x_p=0 \right\}$}
In this section we (again, for parameters $(\lambda,\nu)$ in appropriate subset of $\C^2$)
explicitly construct distribution on $\R^{p,q}$ with support in $\left\{ x_p=0 \right\}$ and belongs to $\sol$.
\begin{myprop}
	Fix $k\in\Z_{\ge0}$ and let $\lambda\in\C$ be determined by $\nu\in\C$ so that $\lambda+\nu-n=-1-2k$ holds.
	Then for $\Re(\nu)<-k$ 
	\[K_{\lambda,\nu}^P(\varphi):=\sum_{i=0}^k\frac{(-1)^i(2k)!(\nu)_i}{(2k-2i)!i!}
		\myabra{\delta^{(2k-2i)}(x_p),\myabs{\tilde{Q}}^{-\nu-i}\varphi}
	\] with $(\nu)_i:=\nu(\nu-1)\cdots(\nu-i+1)$ and $\tilde{Q}(x,y):=\myabs{x}^2-\myabs{y}^2$ for $(x,y)\in\R^{p-1,q}$,
	is well-defined distribution on $\R^{p,q}$ which is supported in $\left\{ x_p=0 \right\}$ and is well-behaved
	under $P'$.
\end{myprop}
\begin{myproof}
	First well-definedness is pretty obvious, as $\myabs{\tilde{Q}}^{(-\nu-i)}$ is at least in class $C^{k-i}\subset C^0$.
	The claim about support then gets obvious as well.

	As for being an element of $\sol$, the first three properties of definition are evident (note that $\tilde{Q}$ is
	invariant under $\Stab$), only invariance
	under $N_+'$ requires more detailed explanation. It is sufficient thus to show (fixing $x_0\in\R^{p,q}$
	and $b\in\R^{p,q}$ such that $b_p=0$ and $c_b(x_0):=1-2Q(b,x_0)+Q(x_0)Q(b)\neq0$ and letting 
	$\psi_b(x):=(x-Q(x)b)/c_b(x)$ for $x$ near $x_0$) that for $F_{\nu,i}:=\delta^{(2i)}(x_p)\myabs{\tilde{Q}}^{-\nu}$ we have
	\begin{equation}
	F_{\nu,i}\left( \frac{x-Q(x)b}{c_b(x)}\right)=c_b^{1+2i+\nu}(x)F_{\nu,i}(x) 
		\label{eq:sing_p}
	\end{equation}
	
	It is known, however, from \cite[ch. I, sec. 3.5]{gelfand1980distribution}
	that $\delta^{(2i)}(x)$ with $i\geq0$ is proportional to $\lim_{\nu\to-2i-1}\myabs{x}^{\nu}/\Gamma\left( \frac{\nu+1}{2}
	\right)$. Hence if we fix $\nu$ with $\Re(\nu)<0$,
	and set $\tilde{F}_{\mu,\nu}:=\myabs{x_p}^{\mu}\myabs{\tilde{Q}}^{-\nu}/\Gamma\left( \frac{\mu
	+1}{2} \right)$, we see that $\tilde{F}_{-2i-1,\nu}=F_{\nu,i}$ up to proportionality. Moreover, for $\Re(\mu)>0$,
	$\tilde{F}_{\mu,\nu}\in L^1_{loc}$ and clearly satisfies \eqref{eq:sing_p}. Then, an analogue of 
	\cite[prop. 3.18]{kobayashi2015symmetry}
	implies that so does $\tilde{F}_{\mu,\nu}$ and hence so does $F_{\nu,i}$ and this ends the proof.
\end{myproof}

\section{The kernel of singular symmetry breaking operator supported on $\left\{ Q=0 \right\}$}
In this section, we construct the distribution on $\R^{p,q}$ which is supported on
$R^{p,q}\supset\left\{ Q=0 \right\}$ hypersurface and belongs to $\sol$,
under some mild assumption on the complex parameters. More precisely,
\begin{myprop}
	Suppose $\nu\in2\Z_{\geq0}+1$ and $\lambda\in\Omega_\nu:=
	\mysetn{\lambda\in\C}{\Re(\lambda-\nu+n)\notin\Z_{\leq1}}$. Then, the following holds:
	\begin{enumerate}
		\item 
	There exists $F\in\sol$ such that it is supported on $\left\{ Q=0 \right\}$.
	When $\Re(\lambda)\gg0$ restricted to $\R^{p,q}\setminus\left\{ 0 \right\}$ we have $F(\varphi)=\myabra{
		\delta^{\nu-1}(Q),\myabs{x_p}^{\lambda+\nu-n}\varphi}$;
	\item Moreover, if $\nu\notin2\Z_{\geq0}-1$ such generalized function does not exist.
	\end{enumerate}
	\label{thm:sing_q}
\end{myprop}
We first state several auxiliary results, which we need to prove the latter proposition, then finish a proof with the help of them.
Afterwards we will prove the auxiliary results.
\begin{mylem}
	If $\nu\notin2\Z_{\geq0}+1$, $F\in\sol[\R^{p,q}\setminus\left\{ 0 \right\}]$ with $\supp(F)=\left\{ Q=0 \right\}$
	then $F=0$.
	\label{lem:sing_q_1}
\end{mylem}
\begin{mylem}
	Suppose $\nu\in2\Z_{\geq0}+1$. Then there exists $F\in\sol[\R^{p,q}\setminus\left\{ 0 \right\}]$
	such that $\supp(F)=\left\{ Q=0 \right\}$.
	\label{lem:sing_q_2}
\end{mylem}
%proof: construct a distribution, show that its restriction to U' for ``good'' parameters coincides with 
%another function and that one is a residue of function that well behaves under P'
\begin{mylem}
	If $F\in\D'(\mysetn{(x,y)\in\R^{p,q}}{x\neq0,\;y\neq0}$, $\supp(F)=\left\{ Q=0 \right\}$
	, then $F$ extends to $\tilde{F}\in\D'(\R^{p,q}\setminus\left\{ 0 \right\})$ with $\supp(\tilde{F})=\left\{ Q=0 \right\}$
	and any two such extensions would coincide.
	Moreover, if $F\in\sol[\Upp]$, then $\tilde{F}\in\sol[\R^{p,q}\setminus\left\{ 0 \right\}]$.
	\label{lem:sing_q_3}
\end{mylem}
\begin{mylem}
	Suppose $F\in\D'(\R^{p,q})$, $F\Big|_{\R^{p,q}\setminus\left\{ 0 \right\}}$ is $N_+'$-invariant on $\R^{p,q}
	\setminus\left\{ 0 \right\}$ and $F$ satisfies \eqref{Ndiff} on $\R^{p,q}$. Then, is $F$ $N_{+}'$-invariant
	on $R^{p,q}$.
	\label{lem:sing_q_4}
\end{mylem}
\begin{myfact}{\bf\cite[thm 3.2.3]{hormander1983analysis}}\newline
	Suppose $u\in\D'(\R^n\setminus\left\{ 0 \right\})$ is homogeneous of degree $a\notin\left\{ -n,-n-1,\dots \right\}$.
	Then $u$ has a homogeneous of the same degree extension to $\tilde{u}\in\D'(\R^n)$ and any two such extensions would coincide.
	Moreover, if $P$ is homogeneous polynomial then $\tilde{Pu}=P\tilde{u}$. Moreover, if in addition to above $a\neq -n+1$,
	then $\partial_j\tilde{u}=\tilde{\partial_ju}$.
	\label{fact:sing_q}
\end{myfact}
\begin{myproof}{\bf (of Proposition \ref{thm:sing_q})}\newline
	The second item is readily granted by Lemma \ref{lem:sing_q_1}. Now, if $\nu\in2\Z_{\geq0}+1$, Lemma \ref{lem:sing_q_2}
	grants the existence of $F\in\sol[\R^{p,q}\setminus\left\{ 0 \right\}]$ with support in $\left\{ Q=0 \right\}$.
	Moreover, as elements of $\sol$ are homogeneous with degree $\lambda-\nu-n$, Fact 
	\ref{fact:sing_q}
	implies the existence of an extension $\tilde{F}\in\D'(\R^{p,q})$ that satisfies \eqref{Ndiff} and items 1 and 2
	from definition of $\sol$ (the latter is true because $\tilde{F}$ satisfies
	discrete symmetries due to uniqueness part of Fact \ref{fact:sing_q}) and differential of item 2).
	Hence, due to lemma \ref{lem:sing_q_3} $\tilde{F}$ is also $N_+'$-invariant, hence is in $\sol$.
\end{myproof}
Before starting the next proof, one more fact to be stated.
\begin{myfact}\proofexplanation{\cite[Thm 2.2.4]{hormander1983analysis}}
	Let $X_i$ be the family of open subsets of $\R^n$ and $f_i\in\D'(X_i)$. Suppose further
	that $\forall i,j,\;f_i\Big|_{X_i\cap X_j}=f_j\Big|_{X_i\cap X_j}$. Then there exists unique
	$f\in\D'(\bigcup_iX_i)$ such that $f\Big|_{X_i}=f_i$.
	\label{fact:sing_q_1}
\end{myfact}
\begin{myproof}\proofexplanation{of Lemma \ref{lem:sing_q_3}}
	Note that $\R^{p,q}\setminus\left\{ 0 \right\}=A\cup B:
	=\mysetn{(x,y)\in\R^{p,q}}{x\neq0,\;y\neq0}\cup\mysetn{(x,y)\in\R^{p,q}}
	{\myabs{x}\neq\myabs{y}}$. If we define $\tilde{F}$ as $F$ on $A$ and 0 on $B$, the Fact \ref{fact:sing_q_1}
	shows that $\tilde{F}$ is a well-defined extension, and this proves the existence part. Regarding the uniqueness part,
	whatever extension $\tilde{F}$ would be, it should be equal to $F$ on $A$ and 0 on $B$, hence the uniqueness part 
	of Fact \ref{fact:sing_q_1} grants the uniqueness. This proves the conclusion in the first sentence of Lemma being proven.

	Now, assume that $F\in\sol[A]$. Then both $\tilde{F}\Big|_A=F\in\sol[A]$ and
	$F\Big|_B=0\in\sol[B]$. This grants the desired conclusion.
\end{myproof}
One more fact before the next proof.
\begin{myfact}\proofexplanation{\cite[Thm 2.1.3]{hormander1983analysis}}
	Suppose $\phi\in C^\infty(X\times Y)$ where $X,\,Y\in\R^m,\,\R^n$ are open sets and
	$\phi(x,y)=0$ if $x\notin K\subset X$, where $K$ is compact. Then for any $u\in\D'(X)$
	$y\mapsto u(\phi(\cdot,y))$ is smooth function and $\frac{\partial}{\partial y_i}\left( y\mapsto u(\phi(\cdot,y)) \right)
	=u\left( \frac{\partial}{\partial y_i}u(\cdot,y) \right)$.
	\label{fact:sing_q_2}
\end{myfact}
\begin{myproof}\proofexplanation{of Lemma \ref{lem:sing_q_4}}
	Fix arbitrary $b\in\R^{p,q}$ with $b_p=0$ and let $c_b(x):=1-2Q(b,x)+Q(b)Q(x)$, $\psi_b(x):=(x-Q(x)b)/c_b(x)$.
	What needs to be shown is that near $x=0$ we have $F(x)=F(\psi_b(x))c_b^{\lambda-n}(x)$.

	Let us take open $\R^{p,q}\supset U\ni\left\{ 0 \right\}$ such that $\forall(t,x)\in(-1/2,3/2)\times U$
	we have $c_{tb}(x)\neq0$ and let us further fix arbitrary $\phi\in C_0^\infty(V)$. Define
	$(-3/2,1/2)\ni t\mapsto f(t):=\myabra{F(\psi_{tb}(x))c_{tb}^{\lambda-n},\phi}\in\R$.
	Recalling what $F(\psi_{tb}(x))c_{tb}^{\lambda-n}$ means in a distribution sense,
	we may write $\myabra{F(\psi_{tb}(x))c_{tb}^{\lambda-n},\phi}=\myabra{F,\phi_t}$ for $\phi_t\in C^\infty( (-1/2,3/2)\times V)$
	the deformation of $\phi$. Further restricting if necessary $\supp(\phi)$ we may ensure that $\phi_t$
	vanishes outside $(-1/2,3/2)\times K$ for some compact $K\times V$. Fact \ref{fact:sing_q_2} then tells us that
	$f$ is smooth and $f'(t)=\myabra{D_bF,\phi_t}$, where $D_b=\sum_jb_j\mysbra{
{2\nu\varepsilon_jx_j+\frac{ 1 }{ 2 }Q(x)\frac{ \partial}{ \partial x_j } }
	}$
	But then the hypothesis of the lemma implies that $D_bF=0$, hence $f'(t)=0$, hence $f(1)=f(0)$ and
	this implies the desired conclusion.
\end{myproof}
Before the next proofs, we will need a few more results. As usual, we first state the lemmas, then finish the proofs with their
help and then finally prove the lemmas.
\begin{myfact}\proofexplanation{\cite[Thm. 2.3.5]{hormander1983analysis}}
	\label{fact:sing_q_3}
	Let $x=(x',x'')$ be a splitting of the variables in $\R^n$ in two groups. If $u\in\D'(\R^n)$ is of order
	$k$, with compact support and the latter is contained in the plane $x'=0$,
	\[u=\sum_{\myabs{\alpha}\leq k}u_{\alpha}\delta^{(\alpha)}(x')\]
	with $u_\alpha$ is distribution with compact support and of order $k-\myabs{\alpha}$ in the $x''$ variables.
\end{myfact}
\begin{remark}
	Using the cutoff function, this fact can be applied to distributions $u$ of non-compact support, the sum 
	$u=\sum_{\myabs{\alpha}\leq k}u_{\alpha}\delta^{(\alpha)}(x')$
	then becoming locally compact. Also, such extension is easily seen to be unique.
\end{remark}
\begin{mydef}
	Let's fix $\nu\in\Z_{>0}$ and $\lambda\in\C$. Let $f\in\D'(\R^p\setminus\left\{ 0 \right\})$ be the restriction to 
	$\R^n\setminus\left\{ 0 \right\}$ of the distribution $\myabs{x_p}^{\lambda+\nu-n}/\Gamma(\frac{\lambda+1}{2})\in\D'(\R^n)$
	(see \cite[ch. III, sec. 3.2, 3.3]{gelfand1980distribution} for definition of distribution $\myabs{x}^\lambda$).
	Let $g\in\D'(\R_{>0}\times\Sp^{p-1})$ be pullback of $f$ via the polar coordinates.

	We introduce the coordinate system to parametrize $\Upp$.
	These will be $(\mu,s,\omega_{p-1},\omega_{q-1})$ coordinates given by $\mysetn{(x,y)\in\R^{p,q}}{
	x\neq0,\;y\neq0}\ni(x,y)=(\sqrt{s}\omega_{p-1},\sqrt{\mu s}\omega_{q-1}$ with $(\mu,s,\omega_{p-1},\omega_{q-1})\in
	\R_{>0}\times\R_{>0}\times \Sp^{p-1}\times\Sp^{q-1}$.

	We define $K^C_{\lambda,\nu}\in\D'(\R_{>0}\times\R_{>0}\times\Sp^{p-1}\times\Sp^{q-1})$ to be the distribution
	$K^C_{\lambda,\nu}:=\delta^{(i-1)}(\mu-1)\otimes s^{-\nu}g\otimes 1_{\Sp^{q-1}}
	$ in variables $(\mu,s,\omega_{p-1},\omega_{q-1})\in\R_{>0}\times\R_{>0}\times\Sp^{p-1}\times\Sp^{q-1}$.
\end{mydef}
\begin{mylem}
	\label{lem:sing_q_6}
	If $F\in\sol[\Upp]$ has $\supp(F)=\left\{ Q=0 \right\}$,
	then $F$ is a multiple of $K^C_{\lambda,\nu}$.
\end{mylem}
\begin{mylem}
	\label{lem:sing_q_7}
	for $\nu\in\Z_{>0}$
	$K^C_{\lambda,\nu}$ as distribution on $\Upp$ satisfies the first three items of definition of $\sol[\Upp]$.
	Moreover, it is $N_+'$-invariant iff $\nu$ is odd.
\end{mylem}
\begin{myproof}\proofexplanation{of Lemma \ref{lem:sing_q_1}}
	In the light of lemma \ref{lem:sing_q_3} it suffices to prove that $F\Big|_{\mysetn{(x,y)\in\R^{p,q}}{x\neq0,\;y\neq0}}=0$.
	Abusing notation, from now and till the end of the proof we will call this restriction $F$.
	Furthermore, as $\supp(F)\subset\left\{ Q=0 \right\}$ and latter subset in the $(\mu,s,\omega_{p-1},\omega_{q-1})$ coordinates
	become $\left\{ \mu=1 \right\}$, fact \ref{fact:sing_q_3}
	implies that in $(\mu,s,\omega_{p-1},\omega_{q-1})$ coordinates 
	$F=\sum_{i\geq0}\delta^{(i)}(\mu-1)u_i$ with sum being locally finite.

	Now, as $F\in\sol[\mysetn{(x,y)\in\R^{p,q}}{x\neq0,\;y\neq0}]$, this in particular
	implies $N_+'$-invariance, hence equations \eqref{Ndiff}. Writing elements of $\R^{p,q}$
	as $(x,y)$ the last $q$ of these equations get written as \[\left[ -2\nu y_j+(\myabs{x}^2-\myabs{y}^2)\frac{\partial}{
	\partial y_j} \right]F=0,\quad1\leq j\leq q\]
	In turn, in bipolar coordinates $(x,y)=(r_1\omega_{p-1},r_2\omega_{q-1})$, where $\partial y_j/\partial r=
	y_j/r_2$ these get written as
	\[\mysbra{-2\nu \frac{y_j^2}{r_2}+(r_1^2-r_2^2)\frac{\partial y_j}{\partial r_2}\frac{\partial}{\partial y_j}}F=0,\quad
	1\leq j\leq q\]
	and summing these up and writing in $(\mu,s,\omega_{p-1},\omega_{q-1})$ we get
	\[\mysbra{\nu+(\mu-1)\frac{\partial}{\partial\mu}}F=0\]
	Substituting this in 
	$F=\sum_{i\geq0}\delta^{(i)}(\mu-1)u_i$ and keeping in mind the formula $(\mu-1)\frac{\partial}{\partial\mu}\delta^{(i)}
	(\mu-1)=-(i+1)\delta^{(i)}(\mu-1)$ we get (invoking the uniqueness part of lemma \ref{fact:sing_q_3})
	$(\nu-(i+1))u_i=0$ this implies that $F=0$, unless $\nu\in\Z_{>0}$.

	In case $\nu\in\Z_{>0}$, however, lemmas \ref{lem:sing_q_6} and \ref{lem:sing_q_7} grant the desired conclusion anyway.
\end{myproof}
\begin{myproof}\proofexplanation{of Lemma \ref{lem:sing_q_2}}
	Lemma \ref{lem:sing_q_7} readily implies that $K^C_{\lambda,\nu}\in\sol[\Upp]$
	and it obviously satisfies $\supp(K^C_{\lambda,\nu})=\left\{ Q=0 \right\}$.
	Then, lemma \ref{lem:sing_q_3} gives the desired conclusion.
\end{myproof}
One more technical result needs to be stated and proved before the next proof:
\begin{mylem}
	\label{lem:sing_q_8}
	Assume that $\nu\in\Z_{>0}$ is fixed. Then, for $\Re(\lambda)\gg0$ we have
	$K_{\lambda,\nu}^C\Big|_{\Upp}$ is proportional to a distribution
	$G(\varphi):=\myabra{\delta^{(\nu-1)}(Q),\varphi\myabs{x_p}^{\lambda+\nu-n}}$
	(here $\delta^{i}(Q)$ is defined as in
	\cite[ch. III, sec. 2.1]{gelfand1980distribution} and for $\Re(\lambda)$ big enough $\myabs{x_p}^{\lambda+\nu-n}$ becomes
	smooth enough, so that the definition there can be used).
\end{mylem}
\begin{myproof}
	To prove the desired proportionality we note that when restricted to $\Upp$, $\delta^{(\nu-1)}(Q)\simeq
	\delta^{(\nu-1)}(\myabs{y}^2-\myabs{x}^2)=\myabs{x}^{-2\nu}\delta^{(\nu-1)}(\mu-1)$, as explained in
	\cite[ch. III, sec 1.7]{gelfand1980distribution} (here $\simeq$ means proportionality). Therefore, in 
	$(\mu,s,\omega_{p-1},\omega_{q-1}
	)$ we can write $G(\phi)=\myabra{\delta^{(\nu-1)}(\mu-1),s^{\nu}\myabs{x_p}^{\lambda+\nu-n}\phi}$ and this
	perfectly coincides with the definition of $K_{\lambda,\nu}^C$.
\end{myproof}
\begin{myproof}\proofexplanation{of Lemma \ref{lem:sing_q_7}}
	Due to the holomorphicity, we can assume that $\Re(\lambda)\gg0$.
	Moreover, in the light of lemma \ref{lem:sing_q_3} it suffices to work with the restriction $K^C_{\lambda,\nu}\Big|_{\Upp}$,
	which we will call $K$ for brevity from now and till the end of the proof. Lemma \ref{lem:sing_q_8} then tells
	us that we can work with distribution $G\in\D'(\Upp)$ as defined there. When we will need to emphasize complex parameters
	involved in definition, we will write $G=G_{\lambda,\nu}$.

	The residue information obtained in \cite[ch. III, sec 2.2]{gelfand1980distribution} tells us that restricted to 
	$\R^{p,q}\setminus\left\{ 0 \right\}$ we have
	\[\delta^{(2k)}(Q)=\frac{\myabs{Q}^{-\nu}}{\Gamma\left( \frac{-\nu+1}{2} \right)}\Big|_{\nu=2k+1},\quad k\in\Z_{>0}\]
	\[\delta^{(2k-1)}(Q)=\frac{Q_+^{-\nu}-Q_-^{-\nu}}{\Gamma\left( \frac{-\nu+2}{2} \right)}\Big|_{\nu=2k},\quad k\in\Z_{>0}\]
	and right hand sides are holomorphic near $\nu$ positive odd and even respectively. Hence
	\[G_{\lambda_0,\nu_0}=\lim_{\nu\to\nu_0=2k+1}\frac{\myabs{Q}^{-\nu}\myabs{x_p}^{\lambda+\nu-n}}
		{\Gamma\left( \frac{-\nu+1}{2} \right)},\quad k\in\Z_{>0}\]
	\[G_{\lambda_0,\nu_0}=\lim_{\nu\to\nu_0=2k}\frac{\mybra{Q_+^{-\nu}-Q_-^{-\nu}}\myabs{x_p}^{\lambda+\nu-n}}
		{\Gamma\left( \frac{-\nu+2}{2} \right)},\quad k\in\Z_{>0}\]
	where in two last equalities in right-hand side we set $\lambda=\lambda_0+\nu_0-\nu$.
	Now for $0<\myabs{\nu-\nu_0}\ll1$ the enumerators of right-hand sides of the last two equalities are regular functionals
	and as functions they are readily seen to satisfy first three items of definition of $\sol[\Upp]$.

	Fix $b\in\R^{p,q}$ with $b_p=0$, and set $c_b(x):=1-2Q(b,x)+Q(b)Q(x)$, $\psi_b(x):=(x-Q(x)b)/c_b(x)$.
	As $Q(\psi_b(x))=Q(x)/c_b(x)$,
	we see that for $G_1:={\myabs{Q}^{-\nu}\myabs{x_p}^{\lambda+\nu-n}}$ when $0<\myabs{\nu-(2k+1)}\ll1$
	\[G_1(\psi_b(x))=\myabs{c_b(x)}^{\lambda-n}G_1(x)\]
	whereas for $G_2:={\mybra{Q_+^{-\nu}-Q_-^{-\nu}}\myabs{x_p}^{\lambda+\nu-n}}$ when $0<\myabs{\nu-2k}\ll1$
	and $c_b(x)<0$ we have
	\[G_2(\psi_b(x))=-\myabs{c_b(x)}^{\lambda-n}G_2(x)\]
	and due to holomorphicity similar relations hold for $G_{\lambda_0,\nu_0}$ when $\nu_0$ is odd or even respectively,
	which grants the desired conclusion.
\end{myproof}
The following fact needs to be stated before the next proof
\begin{myfact}\proofexplanation{\cite[Thm. 3.1.4']{hormander1983analysis}}
	\label{fact:sing_q_4}
	Let $u\in\D'(Y\times I)$ with $Y\subset\R^n$ open and $I\subset\R$ interval. Assume further that
	$\partial_\nu=0$. Then, there exists a distribution $u_0\in\D'(Y)$ such that $u(\varphi)=\int_Iu_0(x\mapsto\varphi(x,t))\;dt$.
\end{myfact}
\begin{myproof}\proofexplanation{of lemma \ref{lem:sing_q_6}}
	Again, applying fact \ref{fact:sing_q_3}, in $(\mu,s,\omega_{p-1},\omega_{q-1})$ coordinates $F$ should be of
	the form $F=\sum_{i\geq0}\delta^{(i)}(\mu-1)u_i$ with sum being locally finite and 
	$u_i\in\D'(\R_{>0}\times\Sp^{p-1}\times
	\Sp^{q-1})$ being independent of $\mu$.
	Similarly as in proof of lemma \ref{lem:sing_q_1},
	we can conclude that $F=\delta^{(\nu-1)}(\mu-1)u$ globally with
	$u\in\D'(\R_{>0}\times\Sp^{p-1}\times
	\Sp^{q-1})$ being independent of $\mu$.

	Furthermore, being an element of $\sol{\Upp}$
	implies homogeneity with degree
	$\lambda-\nu-n$ and hence that the Euler equation $EF=(\lambda-\nu-n)F$ should hold, with Euler operator $E$
	in $(\mu,s,\omega_{p-1},\omega_{q-1})$ coordinates being written as $E=2s\frac{\partial}{\partial s}$, hence
	$\frac{\partial}{\partial s}u=\frac{\lambda-\nu-n}{2}u$. Hence, applying fact \ref{fact:sing_q_4}
	we see that $s^{-\frac{\lambda-\nu-n}{2}}u$ is independent of $s$, hence we may write $u=s^{\frac{\lambda-\nu-n}{2}}v$
	with $v\in\D'(\Sp^{p-1}\times\Sp^{q-1})$ independent of $s$.

	As $F$ has to be invariant under left multiplication by $O(q)\subset \Stab$, we see that in fact $v$ is independent
	of variables of $\Sp^{q-1}$, hence may be treated as $v\in\D'(\Sp^{p-1})$.
	To finish the proof, it now 
	suffices to show that under the polar coordinates transformation, $\tilde{v}:=s^{\lambda+\nu-n}v\in
	\D'(\R_{>0}\times\Sp^{p-1})$
	becomes proportional to $\myabs{x_p}^{\lambda+\nu-n}/\Gamma\left( \frac{\lambda+\nu-n}{2} \right)$. For this it suffices
	to show that $(\partial/\partial x_i)\tilde{v}=0$ and that $\tilde{v}$ is proportional of degree $\lambda+\nu-n$
	(the latter is obvious).

	To derive the required equalities we will need to understand how equations \eqref{Ndiff} get written in $(\mu,s,\omega_{p-1},
	\omega_{q-1})$ coordinates and how $(\partial/\partial x_i)\tilde{v}=0$ get written. We assume that we fix two 
	parametrizations $\R^{p-1}\supset U\ni\left( z_i \right)_{i=1}^{p-1}\mapsto\left( \omega_{p-1}^{(j)}(z) \right)_{j=1}^{p}\in
	\Sp^{p-1}\subset\R^p$ and 
	$\R^{q-1}\supset U\ni\left( w_i \right)_{i=1}^{q-1}\mapsto\left( \omega_{q-1}^{(j)}(w) \right)_{j=1}^{q}\in
	\Sp^{q-1}\subset\R^q$. We will denote the first order derivatives of these by $D_{p-1}$ and $D_{q-1}$ respectively
	(these being $p-1\times p$ and $q-1\times q$ matrices respectively). We will also use shorthands to denote column-vectors:
	$\partial/\partial z:=\left[ \frac{\partial}{\partial z_i} \right]_i$ and 
	$\partial/\partial w:=\left[ \frac{\partial}{\partial w_j} \right]_j$

	We will start with the former task. Under the polar parametrization $x=\sqrt{s}
	\cdot \omega_{p-1}$ the desired equality $(\partial/\partial x_i)\tilde{v}=0$ for $1\leq i\leq p-1$
	gets written as \[ \left[ \frac{1}{\sqrt{s}}\omega_{p-1}^{(i)}2s\frac{\partial}{\partial s}
	+\frac{1}{\sqrt{s}}\left( J_{p-1}\frac{\partial}{\partial z} \right)_{i}\right]\tilde{v}=0,\quad 1\leq i\leq p-1\]
	as we further know that $\tilde{v}$ is homogeneous of degree $\lambda+\nu-n$ and Euler operator is written as $E=2s\frac{
	\partial}{\partial s}$, this can be rewritten as
	\begin{equation}
	\left[ \omega_{p-1}^{(i)}(\lambda+\nu-n)
		+\left( J_{p-1}\frac{\partial}{\partial z} \right)_{i}\right]\tilde{v}=0,\quad 1\leq i\leq p-1
		\label{eq:sing_q_dx}
	\end{equation}

	Furthermore, differentiating condition 2 of definition of $\sol{\Upp}$ for $F\in\sol{\Upp}$ we see that (using the usual
	splitting $(x,y)\in\R^{p,q}$) we should have
	\[ \left[ x_i\partial y_j + y_j\partial x_i \right]F=0,\quad 1\le i\le p-1,\;1\leq j\leq q\]
	which in $(\mu,s,\omega_{p-1},\omega_{q-1})$ gets written as
	\[ \sqrt{s}\omega^{(i)}_{p-1}\left[ \omega^{(j-1)}_{q-1}\frac{2\sqrt{\mu}}{\sqrt{s}}\cdot\frac{\partial}{\partial\mu}+
	\frac{1}{\sqrt{s\mu}}\left( J_{q-1}\frac{\partial}{\partial w} \right)_j\right]F+\]\[
	\sqrt{s\mu}\omega^{(j)}_{q-1}\left[ \omega_{p-1}^{(i)}\left( -2\mu/\sqrt{s}\frac{\partial}{\partial\mu}+2\sqrt{s}\frac
	{\partial}{\partial s} \right)+\frac{1}{\sqrt{s}}\left( J_{p-1}\frac{\partial}{\partial z} \right)_i \right]F=0.\]
	As $F$ is independent of $w$ (as shown above) this gets rewritten as
	\[ \omega^{(i)}_{p-1}\left[ \omega^{(j-1)}_{q-1}\frac{2}{\sqrt{s}}\cdot\frac{\partial}{\partial\mu}
	\right]F+
	\omega^{(j)}_{q-1}\left[ \omega_{p-1}^{(i)}\left( -2\mu/\sqrt{s}\frac{\partial}{\partial\mu}+2\sqrt{s}\frac
	{\partial}{\partial s} \right)+\frac{1}{\sqrt{s}}\left( J_{p-1}\frac{\partial}{\partial z} \right)_i \right]F=0.\]
	We also can choose $j$, so that $\omega^{(j)}_{q-1}\neq0$ locally, hence we can divide it and $\sqrt{s}$ out to get
	\[ \omega^{(i)}_{p-1}\left[ {2}\cdot\frac{\partial}{\partial\mu}
	\right]F+
	\left[ \omega_{p-1}^{(i)}\left( -2\mu\frac{\partial}{\partial\mu}+2{s}\frac
	{\partial}{\partial s} \right)+\left( J_{p-1}\frac{\partial}{\partial z} \right)_i \right]F=0.\]
	where homogeneity of order $\lambda-\nu-n$
	of $F$ renders this into
	\[ \omega^{(i)}_{p-1} {2}(1-\mu)\cdot\frac{\partial}{\partial\mu}F+
	\left[ \omega_{p-1}^{(i)}\left( \lambda-\nu-n
	\right)+\left( J_{p-1}\frac{\partial}{\partial z} \right)_i \right]F=0.\]
	substituting here $F=\delta^{(\nu-1)}(\mu-1)u$ and using $(\mu-1)\frac{\partial}{\partial\mu}\delta^{(i)}(\mu-1)=
	-(i+1)\delta^{(i+1)}(\mu-1)$, we get
	\[ 2\nu\omega^{(i)}_{p-1}u+\left[ \omega_{p-1}^{(i)}\left( \lambda-\nu-n
	\right)+\left( J_{p-1}\frac{\partial}{\partial z} \right)_i \right]u=0.\]
	and finally substituting $u=s^{-2\nu}\tilde{v}$ one gets
	\[ \left[ \omega_{p-1}^{(i)}\left( \lambda+\nu-n
	\right)+\left( J_{p-1}\frac{\partial}{\partial z} \right)_i \right]\tilde{v}=0,\;1\leq i\leq p-1.\]
	and as the latter is precisely \eqref{eq:sing_q_dx}, this finishes the proof.
\end{myproof}
\nocite{Kobayashi201489}
\bibliography{todai_master}
\bibliographystyle{plain}
%%\begin{thebibliography}{9}
%%\bibitem{gelbaum}Gelbaum, B.R. and Olmsted, J.M.H.. Counterexamples in Analysis. Dover Publications. 2003
%%\end{thebibliography}
\end{document}
%TODO: we can also prove statements about ``proportional up to S''
