
\documentclass[10pt]{article} % use larger type; default would be 10pt

%%\usepackage[T1,T2A]{fontenc}
%%\usepackage[utf8]{inputenc}
%%\usepackage[english,ukrainian]{babel} % може бути декілька мов; остання з переліку діє по замовчуванню. 
\usepackage{enumerate}
\usepackage{CJKutf8}
\usepackage{mystyle}

\renewcommand{\S}{\mathcal{S}}
\newcommand{\mIndex}{\mbox{Index}\;}
\def \hat {\widehat}

\title{Course 901-45 by Bent Ørsted\\Final Report}
\author{Alex Leontiev, 45-146044}
\begin{document}
\begin{CJK}{UTF8}{bsmi}
\maketitle
\end{CJK}
\section{Introduction}
As a topic for this report I've settled with solving problems 3-7 page 15-16 of \cite{met-sobolev}
and 1-4 page 47-48 of \cite{met-fa}. Let me briefly indicate why such a choice was made. First
, I admit that I'm not (at least, yet) directly related to the area of Pseudodifferential operators or Index Theory, and the whole
course was rather a great training in analytic techniques (e.g. Fourier Transform, Sobolev Spaces, Schwartz spaces, PDO etc.) for me
(from this viewpoint, IT is a particularly good mix of analysis, algebra and geometry, the great place to try one's ability at all
of these), these are the things I'll definitely use in nearest future, perhaps just in a bit different way. Thus said, not having
done (or doing) any research related to IT or PDO I couldn't come up with my own report topic, so I had to choose one from the list.

Topics on the list upon closer inspection could be roughly divided into three categories: "more-details-with-proof", "problem-solving"
and "topic connected with course" (e.g. especially the last two items about alternative construction of heat kernel and dets of
Laplace-type operators). I've decided not to go with the latter, estimating my skills and time available, as well as I did not want
to distract from the main content of the course for me: the training. On the other hand, first category seemed rather dull, as
proofs in \cite{gilkey} are really quite detailed as for me and match well with my background
, so I did not see much possibility there either.
%%Nevertheless, in "Notes
%%on Reading \cite{gilkey}" I've included a more detailed proofs of the few places which seemed non-trivial for me during the reading.

Thus said, my only option was solving the problems. These are the content of the next section.
\section{Problems}
Before going to problems, let us point out that definitions in \cite{gilkey} are {\it a bit} different when compared to those
of \cite{met-sobolev} and \cite{met-fa}, so for completeness we have explicitly 
collected the definitions we use in section Definitions and
Basic Properties (we'll use the ones in \cite{gilkey}) and proved the equivalence of these with \cite{met-fa,met-sobolev}.
\myprobshort{Problem 1, p.47, \cite{met-fa}}{Let $H\subset L^2(S^1)$ denote the image (equipped with restriction of norm
	of $L^2(S^1)$) of $L^2(S^1)$ under the
	\[Pf(\theta)=\int_{ n=0 }^\infty \hat{ f }(n)e^{ in\theta }\]
	that becomes $H\to H$ operator. For $\phi\in C(S^1)$ let $T_\phi$ be the $H\to H$ operator defined by $T(f)=P(\phi u)$.
Show that $T_{ E_k }T_{ E_l }-T_{ E_{ k+l } }$ is compact for $k,l\in\Z$ and $E_k(\cdot):=\exp(ik\;\cdot)$.}
Indeed, due to the Fourier transform and its inverse forming a bijection $L^2(S^1)\leftrightarrow l^2(\Z)$ as shown in
\cite{met-fourier}, we can write for $H\ni f=\sum_{ k\in\Z }\hat{ f }(k)e^{ ik\theta }$ (assuming $\hat{ f }(k)=0$ for $k<0$)
\[T_{E_k}(f)=P(e^{ ik\theta} \sum_{ j\in\Z }\hat{ f }(j)e^{ ji\theta })=P(\sum_{ j\in\Z }\hat{ f }(j-k)e^{ ji\theta })=\sum_{ j\geq0}
\hat{ f }(j-k)e^{ ji\theta }\]
and similarly
\[T_{ E_k }T_{E_l}(f)=P(e^{ ik\theta} \sum_{ j\geq0 }\hat{ f }(j-l)e^{ ji\theta })=
P(\sum_{ j\geq k }\hat{ f }(j-l-k)e^{ ji\theta })=\sum_{ j\geq\max\mycbra{ k,0 }} \hat{ f }(j-l-k)e^{ ji\theta }\]
and thus
\[(T_{ E_k }T_{ E_l }-T_{ E_kE_l })\mybra{ \sum_{ j\in\Z }\hat{ f }(j)e^{ ij\theta } }=
\sum_{ j\geq\max\mycbra{ k,0 }} \hat{ f }(j-l-k)e^{ ji\theta }-
\sum_{ j\geq0 } \hat{ f }(j-l-k)e^{ ji\theta }
\]
and hence we see that image of $T_{ E_k }T_{ E_l }-T_{ E_kE_l }$ is finitely dimensional, hence operator is compact.
\myprobshort{Problem 2, p.47, \cite{met-fa}}{Show that for $\phi,\psi\in C(S^1)$ $T_{ \phi }T_{ \psi }-T_{ \phi\psi }$
is compact on $H$.}
First, let us note that (recall that we put the norm on $H$ so that $\mynorm{ \cdot }_H:=\mynorm{ \cdot }_{ L^2(S^1) }$)
$\mynorm{ T_\phi(f) }^2_{ L^2 }=\mynorm{ P(\phi f) }^2_{ L^2 }=\mynorm{ \sum_{ k\geq0}\widehat{ \phi f }(k)
e^{ ik\theta }}^2_{ L^2 }=\sum_{ k\geq0 }\myabs{\widehat{ \phi f }(k)}^2\leq\sum_{ k\in\Z }\myabs{\widehat{ \phi f }(k)}^2=
\mynorm{ \phi f }^2_{ L^2 }\leq M^2\mynorm{ f }^2_{ L^2 }$ and hence $\mynorm{ T_\phi }\leq M:=\sup_{ x\in S^1 }\myabs{ \phi(x) }$.\par
Now, it suffices to be able to approximate any $\phi\in C(S^1)$ with finite sums of the form 
$\sum_{ n\in\Z }c_nE_n$ in $\mynorm{ \cdot }_\infty$ norm.
Indeed, assuming such sequences of approximations $\phi_n$ and $\psi_n$ were constructed for $\phi$ and $\psi$ respectively,
we see that $\phi_n\psi_n\to\phi\psi$ in sup-norm as well, and hence due to the computations in the previous paragraph,
$T_{ \phi_n }T_{ \psi_n }-T_{ \phi_n\psi_n }\to T_\phi T_\psi-T_{ \phi\psi }$ in operator norm. As compact operators
form a closed set under operator norm and $T_{ \phi_n }T_{ \psi_n }-T_{ \phi_n\psi_n }$ are compact (due to the fact that
$T_{ af+bg }=aT_f+bT_g$ and since operators $T_{ E_k }T_{ E_l }-T_{ E_kE_l }$ were shown to be compact in the previous problem),
$T_\phi T_\psi-T_{ \phi\psi }$ is compact as well.\par
Finally, approximating $\phi\in C(S^1)$ in sup-norm is achieved with the application of Stone-Weierstrass theorem 
for compact manifold $S^1$ and subalgebra of $C(S^1)$ given by finite linear combinations of $E_n$'s. Indeed, the latter
space of these finite linear combinations is easily seen to be a subalgebra (as $E_nE_m=E_{ n+m }$), it contains constant function
($cE_0$), closed under conjugation (as $\overline{ E_l }=E_{ -l }$) and separates points (as for given pair $(\theta,\theta')\in\R^2$
we have $\exp(i\theta)=\exp(i\theta')\iff \frac{ \theta-\theta' }{ 2\pi }\in\Z$ and the latter is equivalent to saying
that $\theta$ and $\theta'$ correspond to the same point on $S^1$, hence $E_1=\mybra{\theta\mapsto e^{ in\theta }}$ is injective on $S^1$).
\myprobshort{Problem 3, p.47, \cite{met-fa}}{Show that if $\phi\in C(S^1)$ is nowhere-vanishing, then
$T_\phi$ is Fredholm on $H$.}
Due to \cite[Proposition 7.1]{met-fa} it suffices to find $S_{ 1,2 }:H\to H$ such that $S_1T_\phi-I_H$ and $T_\phi S_2-I_H$ are
both compact. Now, taking $S_1:=S_2:=T_{ \phi^{ -1 } }$ apparently gives the required, as
$T_{ \phi^{ -1 } }T_\phi-T_{ \phi\phi^{ -1 } }=T_{ \phi^{ -1 } }T_\phi-I_H$ is compact by the solution of previous problem and
similarly for the other side.
\myprobshort{Problem 4, p.48, \cite{met-fa}}{Suppose $\phi\in C(S^1)$ is nowhere-vanishing function having degree $k\in\Z$,
	that is being homotopic to $E_k$ through continuous maps of $S^1$ to $\C\setminus\mycbra{ 0 }$. Show that this implies
	$\mIndex T_\phi=\mIndex T_{ E_k }$. Compute the index explicitly by describing $\Ker T_{ E_k }$ and $\Ker T^*_{ E_k }$.
}
Being homotopic means that there is continuous
$F:[0,1]\times[0,1]\to\C\setminus\mycbra{ 0 }$ and $F(0,\cdot)=\phi(\cdot)$ and $F(1,\cdot)=
E_k(\cdot)$. We will show $\mIndex T_\phi=\mIndex T_{ E_k }$ by showing that $t\mapsto \mynorm{ F(t,\cdot) }_\infty$ is continuous
(hence $t\mapsto T_{ F(t,\cdot) }$ is a $[0,1]\to\mathcal{L}(H)$ continuous map that connects $T_\phi$ and $T_{ E_k }$). So
let $\epsilon>0$ and $t_0\in[0,1]$ be given. Given point $(x,y)\in[0,1]\times[0,1]$ and $r>0$ we shall denote by $B_r(x,y)$
the intersection $(x-r,x+r)\times(y-r,y+r)\cap[0,1]\times[0,1]$.
Due to the continuity of $F$ for every $s\in[0,1]$ there is $\delta_s>0$
such that $\forall (x,y)\in B_{ \delta_s}(t_0,s)$ we have $\myabs{ F(x,y)-F(t_0,s) }<\epsilon$. As $[0,1]$ is compact,
it can be covered by finite number of intervals $(s_i-\delta_{s_i},s_i+\delta_{s_i})$ for some finite number of $s_i\in[0,1]$.
Now, taking $\delta:=\min\mycbra{ \delta_{ s_i } }$ we see that $\forall(x,y)\in(t_0-\delta,t_0+\delta)\times[0,1]\cap[0,1]\times[0,1]
,\;\myabs{ F(x,y)-F(t_0,y) }\leq\myabs{ F(x,y)-F(t_0,\exists s_i) }+\myabs{ F(t_0,y)-F(t_0,s_i) }<2\epsilon$ and 
hence $\forall x\in(t_0-\delta,t_0-\delta)\cap[0,1],\;\mynorm{ F(x,\cdot)-F(t_0,\cdot) }_\infty<2\epsilon$. This ends proof
of $\mIndex T_\phi=\mIndex T_{ E_k }.$\par
Therefore, all that remains is to compute the index of $T_{ E_k }$. In fact, in the light of \cite[Proposition 7.6]{met-fa}
we have (as $E_mE_n=E_{ m+n }$) $\mIndex T_{ E_{ k\pm1 } }=\mIndex T_{ E_k }+\mIndex T_{ E_{ \pm1 } }$ and so it suffices
to handle the computation only for $T_{ E_{ \pm1 } }$. Moreover, as $T^*_{ E_k }=T_{ E_{ -k } }$ 
only considering $T_{ E_{\pm1} }$ would be sufficient. To see this latter statement, recall that $T_{ E_n }\mybra{ \sum_{ k\in\Z } 
\hat{ f }(k)e^{ ik\theta }}=\sum_{ k\geq0 }\hat{ f }(k-n)e^{ ik\theta }
$ as was shown above (where we assume $\hat{ f }(k)=0$ for negative $k$). Hence (we again put
$\hat{ g }(k)=0$ for negative $k$)
, $\myabra{ T_{ E_k }f,g }=
\sum_{ n\in\Z }\hat{ f }(n-k)\overline{ \hat{ g }(n) }=\sum_{ n\in\Z }\hat{ f }(n)\overline{ \hat{ g }(n+k) }=\myabra{ f,T_{ E_{ -k }
} g}$.\par
Nevertheless, despite what was said above, we now do calculations for $\mIndex T_{ E_k }$ explicitly by describing
$\Ker T_{ E_k }$ and using the formula $\mIndex T_{ E_k }=\Ker T_{ E_k }-\Ker T^*_{ E_k }=\Ker T_{ E_k }-\Ker T_{ E_{ -k } }$,
supported by the observation above that $T^*_{ E_k }=T_{ E_{ -k } }$. As was shown above, $T_{ E_n }\mybra{ \sum_{ k\in\Z } 
\hat{ f }(k)e^{ ik\theta }}=\sum_{ k\geq0 }\hat{ f }(k-n)e^{ ik\theta }$ and for the latter to be zero we need
$\hat{ f }(k)=0$ for $\forall k\geq -n$. Hence if $n\geq0$ we have $\Ker T_{ E_n }=0$, while for $n<0$ we will have
$\Ker T_{ E_n }$ having basis $\mycbra{ E_i }_{ i=0}^{ -n-1 }$, hence forming an $n$-dimensional vector space. Thus said,
$\mIndex T_{ E_n }=-n$.
\myprobshort{Problem 3, p.15, \cite{met-sobolev}}{Consider the projection $P$ defined by \[Pf(\theta)=\sum_{ n=0 }^\infty
	\hat{f}(n)e^{ in\theta },\;\hat{ f }(k)=\frac{ 1 }{ 2\pi }\int_{ S^1 }f(\theta)e^{ -ik\theta }\;d\theta\]
	Show that $P:H^s(S^1)\to H^s(S^1)$ for all $s\in\R$.
}
As shown in \cite{met-sobolev} (specializing to $\mathbb{ T }^1=S^1$ case), $u\in H^s(S^1)\iff 
\sum_{ m\in\Z }\myabs{ \hat{ u }(m) }^2(1+m^2)^s<\infty$
. Now, if we let $f=Pu$ we see (by the uniqueness of Fourier coefficients) that $\hat{ f }(n)=0$ for $n<0$
and $\hat{ f }(n)=\hat{ u }(n)$ for $n\geq0$. Thus, if $u\in H^s(S^1)$, then $
\sum_{ m\in\Z }\myabs{ \hat{ f }(m) }^2(1+m^2)^s=\sum_{ m\geq0 }\myabs{ \hat{ u }(m) }^2(1+m^2)^s\leq
\sum_{ m\in\Z }\myabs{ \hat{ u }(m) }^2(1+m^2)^s<\infty$, hence $f\in H^s(S^1)$.
\myprobshort{Problem 4, p.15, \cite{met-sobolev}}{For $a\in C^\infty(S^1)$ let $M_af(\theta)=a(\theta)f(\theta)$ (hence $M_a:
	H^s(S^1)\to H^s(S^1)$). Consider the commutator $\mysbra{ P,M_a }=PM_a-M_aP$ and show that
	\[\mysbra{ P,M_a }f=\sum_{ k\geq0,m>0 }\hat{ a }(k+m)\hat{ f }(-m)e^{ ik\theta }-
	\sum_{ k>0,m\geq0 }\hat{ a }(-k-m)\hat{ f }(m)e^{ -ik\theta }\]
	and deduce that $[P,M_a]:H^s(S^1)\to H^s(S^1)$ for any $s\in\R$.}
	Indeed, $PM_af=\sum_{ k\geq0 }\hat{ af }(k)e^{ ik\theta }=\sum_{ k\geq0 }\sum_{ n\in\Z }\hat{a}(k-n)\hat{f}(n)e^{ ik\theta }$
	On the other hand, $M_aPf=\sum_{ k\in\Z }\widehat{ a\cdot Pf }(k)e^{ ik\theta }=
	\sum_{ k\in\Z }\sum_{ n\in\Z }\hat{ a }(k-n)\hat{ Pf }(n)e^{ ik\theta }=
	\sum_{ k\in\Z }\sum_{ n\geq0 }\hat{ a }(k-n)\hat{ f }(n)e^{ ik\theta }
	$. Thus
	\[\mysbra{ P,M_a }f=
	\sum_{ k\geq0 }\sum_{ n\in\Z }\hat{a}(k-n)\hat{f}(n)e^{ ik\theta }
-\sum_{ k\in\Z }\sum_{ n\geq0 }\hat{ a }(k-n)\hat{ f }(n)e^{ ik\theta }=\]
\[=
	\sum_{ k\geq0 }\sum_{ n<0 }\hat{a}(k-n)\hat{f}(n)e^{ ik\theta }
-\sum_{ k<0 }\sum_{ n\geq0 }\hat{ a }(k-n)\hat{ f }(n)e^{ ik\theta }=\]
\[
	\sum_{ k\geq0 }\sum_{ n>0 }\hat{a}(k+n)\hat{f}(-n)e^{ ik\theta }
-\sum_{ k>0 }\sum_{ n\geq0 }\hat{ a }(-k-n)\hat{ f }(n)e^{ -ik\theta }
\]
(manipulations with sums are justified by the fact that
as $a\in C^\infty(S^1)$ we have $\forall N,\;\sup_{ k\in\Z }(1+k^2)^{ N/2 }\myabs{ \hat{ a }(k) }<\infty$) and
this gives the desired formula
$\mysbra{ P,M_a }f=\sum_{ k\geq0,m>0 }\hat{ a }(k+m)\hat{ f }(-m)e^{ ik\theta }-
\sum_{ k>0,m\geq0 }\hat{ a }(-k-m)\hat{ f }(m)e^{ -ik\theta }$\par
Now, we want to show that $\mysbra{ P,M_a }f\in C^\infty(S^1)$, which is equivalent to showing that for any $N>0\;\exists c_N$ such
that
\begin{equation}\label{Prob6}\forall k\geq0,\;
	(1+k^2)^{ N/2 }\myabs{\sum_{ m>0 }\hat{ a }(k+m)\hat{ f }(-m)}+(1+k^2)^{ N/2 }
	\myabs{\sum_{ m\geq0 }\hat{ a }(-k-m)\hat{ f }(m)}<c_N
\end{equation}
(we've consciously increased the constant term of $[P,M_a]f$ to make formula more symmetric).
This is not difficult, as ($a\preceq b$ is used to denote the $a\leq C\cdot b+B$, where $B$ and $C$ do not depend on $m$ or $k$), 
\[(1+k^2)^{ N/2 }\myabs{\sum_{ m>0 }\hat{ a }(k+m)\hat{ f }(-m)}\preceq
	(1+k^2)^{ N/2 }\sum_{ m>0 }
\myabs{ \hat{ a }(k+m)(1+m^2)^{ -s/2}}^2\cdot\sum_{m>0}\myabs{\hat{f}(-m)}^2(1+m^2)^{ s}\preceq\]
\[\preceq 
(1+k^2)^{ N/2 }\sum_{ m>0 }\myabs{ \hat{ a }(k+m)}^2(1+m^2)^{ -s}\preceq\]
\[\preceq\sum_{ m>0 }\myabs{ \hat{ a }(k+m)}^2(1+(m+k)^2)^{ \myabs{ s }+N/2+1}\frac{ 1 }{ 1+(m+k)^2 }\preceq
	\sum_{ m>0 }\frac{ 1 }{ 1+(m+k)^2 }
\]
and as the last term is uniformly bounded in $k$, we
see that the first addend on the left hand side of \ref{Prob6} is uniformly bounded in $k$. Similarly, so does
the second one, thus ending the proof.
\myprobshort{Problem 5, p.15, \cite{met-sobolev}}{Let $a_j,\,b_j\in C^\infty(S^1)$, and consider $T_j:=M_{ a_j }P+M_{ b_j }(I-P).$
	Show that \[T_1T_2=M_{ a_1a_2 }P+M_{ b_1b_2 }(I-P)+R\]
	where for each $s\in\R$ we have $R:H^s(S^1)\to C^\infty(S^1)$.
}
Indeed, as the previous problem tells us, modulo the "infinitely smoothing operators" (i.e. the linear space of operators
$H$ such that $\forall s\in\R$ we have $H:H^s(S^1)\to C^\infty(S^1)$) $P$ and $M_a$ commute for $a\in C^\infty(S^1)$
(hence $M_a$ also commutes with $I-P$ modulo "infinitely smoothing operators"). Hence,
modulo "infinitely smoothing operators" we can write
\[T_1T_2-(M_{ a_1a_2 }P+M_{ b_1b_2 }(I-P))\equiv(M_{ a_1 }P+M_{ b_1 }(I-P))(M_{ a_2 }P+M_{ b_2 }(I-P))
-(M_{ a_1a_2 }P+M_{ b_1b_2 }(I-P))\equiv\]
\[\equiv M_{ a_1 }M_{ a_2 }P^2+M_{ b_1 }M_{ a_2 }(I-P)P+M_{ a_1 }M_{ b_1 }P(I-P)+M_{ b_1 }M_{ b_1 }(I-P)^2
-(M_{ a_1a_2 }P+M_{ b_1b_2 }(I-P))\]
and since $M_{ a_1 }M_{ a_2 }=M_{ a_1a_2 }$, $M_{ b_1 }M_{ b_2 }=M_{ b_1b_2 }$, $P(I-P)=(I-P)P=0$, $P^2=P$ and $(I-P)^2=I-P$
this implies that
\[M_{ a_1 }M_{ a_2 }P^2+M_{ b_1 }M_{ a_2 }(I-P)P+M_{ a_1 }M_{ b_1 }P(I-P)+M_{ b_1 }M_{ b_1 }(I-P)^2-(M_{ a_1a_2 }P+M_{ b_1b_2 }(I-P))\equiv0\]
and this is just the desired conclusion.
\section{Definitions and Basic Properties}
\begin{mydef}Given the inner product space $V$ with the inner product $\myabra{ \cdot,\cdot }$ we
	shall call the {\bf completion of $V$} the set of Cauchy sequences in $V$ under the equivalence relation
	$\mycbra{ v_n }\sim\mycbra{ u_n }$ if $\mynorm{ u_n-v_n }\to0$, vector space structure defined element-wise for sequences
	and inner product defined as limit of $\myabra{ u_n,v_n }$ (latter can be seen to be Cauchy in $\C$).
\end{mydef}
\begin{mydef}For $s\in\R$ we define 
	$H^s(\R^n)$ to be the completion of $\mathcal{ S }$ with respect to the norm
	$\mynorm{u}':=\int_{\R^n}(1+\myabs{\xi}^2)^{s}\myabs{\hat{u}}^2\;d\xi$
	\end{mydef}
\begin{myremark}This definition coincides with that of \cite{met-sobolev}. More precisely, $H^s(\R^n)$ is
isomorphic to 
$H:=\mysetn{u\in\mathcal{S}'(\R^n)}{(1+\myabs{\xi}^2)^{s/2}\widehat{u}\in L^2(\R^n)}$ (equipped with
a norm that sets $\mynorm{ u }$ being an $L^2$-norm of element $(1+\myabs{\xi}^2)^{s/2}\widehat{u}$)
. Indeed, every element of $\S$ is naturally included in $\S'$ (we shall denote the inclusion by $i:S\rightarrow S'$
and we note that it commutes with multiplication by polynomialy bounded functions and Fourier transform) 
and then it belongs to $H$.

Moreover, we see that $L^2$ norm of $(1+\myabs{ \xi }^2)^{ s/2 }\hat{ u }$ is precisely
$\mynorm{ u }'$ (because $(1+\myabs{ \xi }^2)^{ s/2 }\hat{i(u)}=i((1+\myabs{ \xi }^2)^{ s/2 }\hat{ u })$)
. Hence, if $\mycbra{ u_n }$ is Cauchy sequence in $\S$ with respect to $\mynorm{ \cdot }'$, it will
also be Cauchy in $H$ and thus (due to the completeness of $L^2(\R^n)$)
$\mycbra{ (1+\mynorm{ \xi }^2)^{ s/2 }\hat{ u_n } }$ converges to unique
$f\in L^2(\R^n)$. But then $f\in\S'\supset L^2(\R^n)$ and hence $(1+\mynorm{ \xi }^2)^{ -s/2 }f\in\S'$ as well. \par
Thus, the inverse Fourier Transform of the latter is an element of $H$ and $u_n$ converges to it in norm of $H$.
We shall consider so constructed limit of $u_n$ in $H$ as corresponding to the Cauchy sequence $u_n\in\S$.
The equality of $\mynorm{ \cdot }'$ and norm of $H$
implies that the mapping will give the same element for two $\mynorm{ \cdot }'$-Cauchy
sequences $\mycbra{ u_n }$ and $\mycbra{ u_n' }$ in $\S$, if $\mynorm{ u_n-u_n' }\to0$. Hence, the above construction is
a well-defined unitary map from the completion of $\S$ with respect to $\mynorm{ \cdot }'$ to $H$, the map which we'll
denote by $T$. It remains to show that it is onto (injectivity is granted by unitarity).\par
So let $u\in H$, hence $(1+\myabs{ \xi }^2)^{ s/2 }\hat{ u }\in L^2(\R^n)$ and can thus be approximated (in $L^2$-norm) with
the sequence $\mycbra{ u_n }_n$ of elements of $C_0^{ \infty }(\R^n)$. Hence, setting $v_n:=\mathcal{ F }^{ -1 }\mybra{ u_n(1+
\myabs{ \xi })^{ -s/2 } }\in\mathcal{ S }$ we have $\mycbra{ v_n }_n$ forming a Cauchy sequence with respect to the $\mynorm{ \cdot }'
$ norm, hence defining an element of $H^s(\R^n)$ which (according to the remarks above) will get mapped to $u\in H$. This proves
the surjectivity.
\end{myremark}
\begin{thebibliography}{9}
	\bibitem{gilkey}P. B. Gilkey, {\it Invariance Theory, The Heat Equation, and the Atiyah-Singer Index Theorem,} CRC Press 1995.
	\bibitem{met-sobolev}Notes by Michael E. Taylor about Sobolev Spaces at \url{http://www.unc.edu/math/Faculty/met/chap4.pdf}.
	\bibitem{met-fourier}Notes by Michael E. Taylor about Fourier Analysis at \url{http://www.unc.edu/math/Faculty/met/chap3.pdf}.
	\bibitem{met-fa}Notes by Michael E. Taylor about Outline of Functional Analysis
		at \url{http://www.unc.edu/math/Faculty/met/appenda.pdf}.
\end{thebibliography}
\end{document}
%Sobolev(R^d) is the same def--> (1)
%Sobolev(M) (remark: [Gilkey:notWritten]) --> 
%write up all defs --> (Hilbert, Sobolev(M))
%Problem_1-4
%Problem 5-8
%read book --> write up book reading problems[1](2) --> write up others

%finish def(1) --> check manifold --> clean up(how many pages) --> 2-4 --> (1)
