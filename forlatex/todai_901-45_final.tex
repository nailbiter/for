
\documentclass[10pt]{article} % use larger type; default would be 10pt

%%\usepackage[T1,T2A]{fontenc}
%%\usepackage[utf8]{inputenc}
%%\usepackage[english,ukrainian]{babel} % може бути декілька мов; остання з переліку діє по замовчуванню. 
\usepackage{enumerate}
\usepackage{CJKutf8}
\usepackage{mystyle}

\renewcommand{\S}{\mathcal{S}}

\title{Course 901-45 by Bent Ørsted\\Final Report}
\author{Alex Leontiev, 45-146044}
\begin{document}
\begin{CJK}{UTF8}{bsmi}
\maketitle
\end{CJK}
\section{Introduction}
As a topic for this report I've settled with solving problems 3-7 page 15-16 of \cite{met-sobolev}
and 1-4 page 47-48 of \cite{met-fa}. Let me briefly indicate why such a choice was made. First
, I admit that I'm not (at least, yet) directly related to the area of Pseudodifferential operators or Index Theory, and the whole
course was rather a great training in analytic techniques (e.g. Fourier Transform, Sobolev Spaces, Schwartz spaces, PDO etc.) for me
(from this viewpoint, IT is a particularly good mix of analysis, algebra and geometry, the great place to try one's ability at all
of these), these are the things I'll definitely use in nearest future, perhaps just in a bit different way. Thus said, not having
done (or doing) any research related to IT or PDO I couldn't come up with my own report topic, so I had to choose one from the list.

Topics on the list upon closer inspection could be roughly divided into three categories: "more-details-with-proof", "problem-solving"
and "topic connected with course" (e.g. especially the last two items about alternative construction of heat kernel and dets of
Laplace-type operators). I've decided not to go with the latter, estimating my skills and time available, as well as I did not want
to distract from the main content of the course for me: the training. On the other hand, first category seemed rather dull, as
proofs in \cite{gilkey} are really quite detailed as for me and match well with my background
, so I did not see much possibility there either. Nevertheless, in "Notes
on Reading \cite{gilkey}" I've included a more detailed proofs of the few places which seemed non-trivial for me during the reading.

Thus said, my only option was solving the problems. These are the content of the next section.
\section{Problems}
Before going to problems, let us point out that definitions in \cite{gilkey} are {\it a bit} different when compared to those
of \cite{met-sobolev} and \cite{met-fa}, so for completeness we have explicitly 
collected the definitions we use in section Definitions and
Basic Properties (we'll use the ones in \cite{met-fa} and \cite{met-sobolev}) and proved the equivalence of these with \cite{gilkey}.
\section{Notes on Reading \cite{gilkey}}
\section{Definitions and Basic Properties}
\begin{mydef}Given the inner product space $V$ with the inner product $\myabra{ \cdot,\cdot }$ we
	shall call the {\bf completion of $V$} the set of Cauchy sequences in $V$ under the equivalence relation
	$\mycbra{ v_n }\sim\mycbra{ u_n }$ if $\mynorm{ u_n-v_n }\to0$, vector space structure defined element-wise for sequences
	and inner product defined as limit of $\myabra{ u_n,v_n }$ (latter can be seen to be Cauchy in $\C$).
\end{mydef}
\begin{mydef}For $s\in\R$ we define $H^s(\R^n):=\mysetn{u\in\mathcal{S}'(\R^n)}{(1+\myabs{\xi}^2)^{s/2}\hat{u}\in L^2(\R^n)}$
	\end{mydef}
\begin{myremark}This definition coincides with that of \cite{gilkey}. More precisely, there $H^s(\R^n)$ is
isomorphic to completion of $\mathcal{S}$ with respect to norm $\mynorm{u}':=\int_{\R^n}(1+\myabs{\xi}^2)^{s}\myabs{\hat{u}}^2\;d\xi
$ as Hilbert spaces. Indeed, every element of $\S$ is naturally included in $\S'$ (we shall denote inclusion by $i:S\rightarrow S'$
and we note that it commutes with multiplication by polynomialy bounded functions and Fourier transform) 
and then it belongs to $H^s(\R^n)$.

Moreover, we see that $L^2$ norm of $(1+\myabs{ \xi }^2)^{ s/2 }\hat{ u }$ is precisely
$\mynorm{ u }'$ (because $(1+\myabs{ \xi }^2)^{ s/2 }\hat{i(u)}=i((1+\myabs{ \xi }^2)^{ s/2 }\hat{ u })$)
. Hence, if $\mycbra{ u_n }$ is Cauchy sequence in $\S$ with respect to $\mynorm{ \cdot }'$, it will
also be Cauchy in $H^s(\R^n)$ and thus (due to the completeness of $L^2(\R^n)$)
$\mycbra{ (1+\mynorm{ \xi }^2)^{ s/2 }\hat{ u_n } }$ converges to unique
$f\in L^2(\R^n)$. But then $f\in\S'\supset L^2(\R^n)$ and hence $(1+\mynorm{ \xi }^2)^{ -s/2 }f\in\S'$ as well. 

Thus, the inverse Fourier Transform of the latter is an element of $H^s(\R^n)$ and $u_n$ converges to it in norm of $H^s(\R^n)$.
We shall consider so constructed limit of $u_n$ in $H^s(\R^n)$ as corresponding to the Cauchy sequence $u_n\in\S$.
The equality of $\mynorm{ \cdot }'$ and norm of $H^s(\R^n)$
implies that the mapping will give the same element for two $\mynorm{ \cdot }'$-Cauchy
sequences $\mycbra{ u_n }$ and $\mycbra{ u_n' }$ in $\S$, if $\mynorm{ u_n-u_n' }\to0$. Hence, the above construction is
a well-defined unitary map from the completion of $\S$ with respect to $\mynorm{ \cdot }'$ to $H^s(\R^n)$, the map which we'll
denote by $T$. It remains to show that it is onto (injectivity is granted by unitarity).

So let $u\in H^s(\R^n)$
\end{myremark}
\begin{thebibliography}{9}
	\bibitem{gilkey}P. B. Gilkey, {\it Invariance Theory, The Heat Equation, and the Atiyah-Singer Index Theorem,} CRC Press 1995.
	\bibitem{met-sobolev}Notes by Michael E. Taylor about Sobolev Spaces at \url{http://www.unc.edu/math/Faculty/met/chap4.pdf}.
	\bibitem{met-fa}Notes by Michael E. Taylor about Outline of Functional Analysis
		at \url{http://www.unc.edu/math/Faculty/met/appenda.pdf}.
\end{thebibliography}
\end{document}
%Sobolev(R^d) is the same def--> 
%Sobolev(M) (remark: [Gilkey:notWritten]) --> 
%write up all defs --> 
%Problem_1-4
%Problem 5-8
%read book --> write up book reading problems
