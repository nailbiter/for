\documentclass[12pt,fleqn]{article} % use larger type; default would be 10pt

\usepackage{mystyle}
\newtheorem{prob}{Завдання}
\newcommand{\ds}{\;ds}
\newcommand{\dt}{\;dt}
\newcommand{\dx}{\;dx}
\newcommand{\dta}{\;d\tau}


\newtheorem{myulem}[mythm]{Лема}

\renewenvironment{myproof}[1][Доведення]{\begin{trivlist}
\item[\hskip \labelsep {\bfseries #1}]}{\myqed\end{trivlist}}

	\title{}
\begin{document}
\maketitle
For every map given, determine whether it's linear or not. If linear, find representation matrix in bases given. Find bases for kernel and
image.
\begin{enumerate}
\item \[f:\mathbb{R}\to\mathbb{R}\]
	\[f(x)=\max\mycbra{x,1}\]
Basis for both domain and codomain is $\mycbra{1}$.
\\\textbf{Solution.} Well, indeed, map is not linear. In particular, rule 
\[f(\alpha x)=\alpha f(x)\]
does not hold for $\alpha=2$ and $x=1$, as then $f(2x)=f(2)=1$, whereas $2f(x)=2f(1)=2$.
\item \[f:\mathbb{R}^2\to\mathbb{R}\]
	\[f(x,y)=\left\{\begin{array}{ll}
		\frac{x^2}{y},&y\neq0\\
		0,&y=0
	\end{array}\right.\]
	with basis for $\mathbb{R}^2$ given by $\mycbra{(1,0),(0,1)}$ and for $\mathbb{R}$ as in previous problem.
\\\textbf{Solution.} Well, indeed, map is not linear. In particular, rule 
\[f(x+y)=f(x)+f(y)\]
does not holds. Take $x=(1,0)\in\mathbb{R}^2$, whereas $y=(0,1)\in\mathbb{R}^2$. Then, 
\[f(x)=f(1,0)=0\]
\[f(y)=f(0,1)=\frac{0}{0^2}=0\]
whereas
\[f(x+y)=f(1,1)=\frac{1}{1^2}=1\]
\item \[f:\mathbb{R}^+\mapsto\mathbb{R}^+\]
	\[f(x)=8x^2\]
	where the vector space structure on $\mathbb{R}^+$ is given by $a+_{\mathbb{R}^+}b=ab$ and $\alpha*_{\mathbb{R}^+}b=b^{\alpha},\;\alpha
	\in\mathbb{R}$. Basis for $\mathbb{R}^+$ is given by $\mycbra{e}$.
	\\\textbf{Solution.} 
	This map is also nonlinear, as 
	\[f(x+_{\mathbb{R}}y)=f(xy)=8x^2y^2\]
	whereas
	\[f(x)+_{\mathbb{R}}f(y)=8x^2+_{\mathbb{R}}8y^2=64x^2y^2\]
	and in general $8x^2y^2\neq64x^2y^2$ (in particular, for $x=y=1$, equality does not hold).
\item \[f:P_3\to P_3\]
	\[f(p)=\frac{d}{dx}p\]
	where by $P_3$ I mean the vector space of polynomials of degree no more than 3 with basis given by
	\[\mycbra{1,x,x^2}\]
	\\\textbf{Solution.} 
	This map is linear. Indeed for arbitrary $p(x)=a+bx+cx^2,\;p_0(x)=a_0+b_0x+c_0x^2$, we have
	\[f(p+p_0)=f((a+bx+cx^2)+(a_0+b_0x+c_0x^2))=\]\[=\frac{d}{dx}((a+a_0)+(b+b_0)x+(c+c_0)x^2)=(b+b_0)+2(c+c_0)x\]
	whereas
	\[f(p)+f(p_0)=b+2cx+b_0+2c_0x\]
	which is precisely the same.
	\[f(\alpha p)=\frac{d}{dx}(\alpha a+\alpha bx+\alpha cx^2)=\alpha b+2\alpha cx\]
	and 
	\[\alpha f(p)=\alpha(b+2cx)\]
	which is the same.\\
	Let's find the basis for kernel of $f$. For $p(x)=a+bx+cx^2$ we have
	$f(p)=b+2cx$
	which is zero if and only if $b=c=0$, therefore $\Ker f=\mysetn{p\in P_3}{p(x)=a+0\cdot x+0\cdot x^2}$ which is
	equal to the span of $1$, which hence forms a basis.\\
	We see from the formula for $f(p)$ that it will always be a polynomial of the first degree, as it does not contain the $x^2$
	term. All such polynomials are in the span of $1$ and $x$. Conversely, both $1$ and $x$ belong to the image,
	as $L(x)=1$ and $L(x^2/2)=x$, hence image also contains their span. Thus, image and span of $1$ and $x$ coincide. As
	latter two vectors are independent, they form a basis.\\%TODO: show how to find matrix representation
	Finally, let's find matrix representation in bases given. We recall that in general, given the linear map $L:V\to W$,
	together with basis $\mycbra{v_1,v_2,\hdots,v_n}$ for $V$ and $\mycbra{w_1,w_2,\hdots,w_m}$ for $W$, matrix representation
	for $L$ is $m$-by-$n$ matrix, whose $i$-th column (for $i\in\mycbra{1,2,\hdots,n}$) is given by representation of
	$L(v_i)$ in basis $\mycbra{w_1,w_2,\hdots,w_m}$. Let's demonstrate this with an example (in subsequent, we won't be so 
	thorough). Note, that the \textit{order} of vectors in basis is \textbf{important!}
	\[f(1)=0=0\cdot1+0\cdot x+0\cdot x\]
	this gives us the first column of our 3-by-3 representation
	\[\begin{pmatrix}0&*&*\\0&*&*\\0&*&*\end{pmatrix}\]
	\[f(x)=1=1\cdot1+0\cdot x+0\cdot x\]
	this gives us the second column of our 3-by-3 representation
	\[\begin{pmatrix}0&1&*\\0&0&*\\0&0&*\end{pmatrix}\]
	\[f(x^2)=2x=0\cdot1+2\cdot x+0\cdot x\]
	that gives us the remaining column
	\[\begin{pmatrix}0&1&1\\0&0&2\\0&0&0\end{pmatrix}\]
\item \[\mathbb{R}^3\to\mathbb{R}^2\]
	\[f(x,y,z)=(z,y)\]
	and basis for domain $\mathbb{R}^3$ is given by
	\[\mycbra{(1,0,0),(0,1,0),(0,0,1)}\]
	basis for codomain $\mathbb{R}^2$ is given by
	\[\mycbra{(1,0),(0,1)}\]
	\\\textbf{Solution.} This is linear function, as
	\[f((x,y,z)+(x_0,y_0,z_0))=f(x+x_0,y+y_0,z+z_0)=(z+z_0,y+y_0)\]
	whereas 
	\[f(x,y,z)+f(x_0,y_0,z_0)=(z,y)+(z_0,y_0)=(z+z_0,y+y_0)\]
	which is the same and
	\[f(\alpha(x,y,z))=f(\alpha x,\alpha y,\alpha z)=(\alpha z,\alpha y)\]
	and
	\[\alpha f(x,y,z)=\alpha(z,y)=(\alpha z,\alpha y)\]
	which is the same.\\
	Thus said, let's find the basis for kernel. $f(x,y,z)=(z,y)=0$ if and only if $z=y=0$, thus
	$\Ker f=\mysetn{(x,y,z)\in\mathbb{R}^3}{y=z=0}$ is spanned by vector $(1,0,0)$, which forms a basis.\\
	The image, in turn cannot be larger than $\mathbb{R}^2$ and as it contains $(1,0)$ (since $f(0,0,1)=(1,0)$)
	and $(0,1)$ (since $f(0,1,0)=(0,1)$), which are the basis vectors for $\mathbb{R}^2$, it covers the whole $\mathbb{R}^2$
	with basis $(1,0)$ and $(0,1)$.\\
	Finally, we compute the matrix representation. As
	\[f(1,0,0)=(0,0)=0\cdot(1,0)+0\cdot(0,1)\]
	\[f(0,1,0)=(0,1)=0\cdot(1,0)+1\cdot(0,1)\]
	\[f(0,0,1)=(1,0)=1\cdot(1,0)+0\cdot(0,1)\]
	and thus the matrix representation is
	\[\begin{pmatrix}0&0&1\\0&1&0\end{pmatrix}\]
\item \[f:\mathbb{R}^3\to\mathbb{R}^3\]%prob 6,p.44 lin alg
	\[f(x,y,z)=(2x-11y+6z,1x-7y+4z,2x-y)\]
	where basis for both domain and codomain is given by
	\[\mycbra{(2,3,5),(0,1,2),(1,0,0)}\]
	\\\textbf{Solution.} 
	This is linear transformation, as can be verified directly, like in the previous example. Its kernel is the solution
	to system
	\[\begin{cases}2a-11b+6c=0\\1a-7b+4c=0\\2a-b=0\end{cases}\]
	This can be written in form of an augmented matrix as
	\newcommand{\myexplainiii}[3]{#1\xrightarrow{\text{#2}}#3}
	\newcommand{\myexplainiv}[4]{#1\xrightarrow{\begin{subarray}{c}\text{#2}\\\text{#3}\end{subarray}}#4}
	\[\left(\begin{array}{rrr|r}2&-11&6&0\\1&-7&4&0\\2&-1&0&0\end{array}\right)\]
	One possible row echelon form of it is
	\[\left(\begin{array}{rrr|r}1&-7&4&0\\0&1&0&0\\0&0&1&0\end{array}\right)\]
	from which we see that the only solution is zero-vector, hence the basis for kernel of $f$ is an \textit{empty set}.
	(\textbf{note: }this can be seen simply as a convention, but also may be argued logically, but \textbf{the basis
	of a trivial vector space is an empty set}).\\
	Next, image is the set of triples $(\alpha,\beta,\gamma)$, such that the following system has solution
	\[\begin{cases}2a-11b+6c=\alpha\\1a-7b+4c=\beta\\2a-b=\gamma\end{cases}
	\]
	This can be written in form of an augmented matrix as
	\[\left(\begin{array}{rrr|r}2&-11&6&\alpha\\1&-7&4&\beta\\2&-1&0&\gamma\end{array}\right)\]
	As we know, however, the sum of dimensions of null-space and column-space of matrix should be equal to number of columns
	(3, in this case). And since as we've seen about dimension of null-space is zero, column space should have dimension
	three, that is it is the whole $\mathbb{R}^3$, and so is the image of $f$.\\
	Finally, let's do matrix representation. 
	\[f(2,3,5)=(1,1,1)=(2,3,5)-2(0,1,2)-(1,0,0)\]
	\[f(0,1,2)=(1,1,-1)=(2,3,5)-3(0,1,2)-(1,0,0)\]
	\[f(1,0,0)=(2,1,2)=0(2,3,5)+(0,1,2)+(1,0,0)\]
	where expressions of vectors in terms of basis are obtained, as usual, via solving the system of linear equations. For
	example, to find $a$, $b$ and $c$, such that $(1,1,1)=a(2,3,5)+b(0,1,2)+c(1,0,0)$ we solve the system
	\[\begin{cases}2a+0b+1c=1\\3a+1b+0c=1\\5a+2b+0c=1\end{cases}\]
	Therefore, the matrix expression of $f$ is
	\[\left(\begin{array}{rrr}1&1&0\\-2&-3&1\\-1&-1&1
	\end{array}\right)\]
\item \[f:\mathbb{R}^3\to\mathbb{R}^3\]%prob 2,p.40 lin alg
	\[f(x,y,z)=(2x-y-z,x-2y+z,x+y-2z)\]
	where the basis for domain is given by
	\[\mycbra{(1,0,0),(0,1,0),(0,0,1)}\]
	whereas the basis for codomain is given by
	\[\mycbra{(2,1,1),(-1,-2,1),(-1,1,-2)}\]
	\\\textbf{Solution.} 
	In the same way as in example before the previous, $f$ can be checked to be linear. Again, kernel is formed by triples
	$(x,y,z)$ that satisfy the system
	\[\begin{cases}2x-y-z=0\\x-2y+z=0\\x+y-2z=0\end{cases}\]
	This can be written in form of an augmented matrix as
	\[\left(\begin{array}{rrr|r}2&-1&-1&0\\1&-2&1&0\\1&1&-2&0\end{array}\right)\]
	with one possible row echelon form as
	\[\left(\begin{array}{rrr|r}1&1&-2&0\\0&1&-1&0\\0&0&0&0\end{array}\right)\]
	Thus, allowing $z$ to take any values we deduce $y=z$ and $x=z$, so kernel is
	\[\begin{pmatrix}x\\y\\z\end{pmatrix}=\begin{pmatrix}z\\z\\z\end{pmatrix}=z\begin{pmatrix}1\\1\\1\end{pmatrix}\]
	with basis consisting of a sole vector $(1,1,1)$. \\
	Second, image is the set of triples $(\alpha,\beta,\gamma)$, such that the following system has solution
	\[\begin{cases}2x-y-z=\alpha\\x-2y+z=\beta\\x+y-2z=\gamma\end{cases}\]
	This can be written in form of an augmented matrix as
	\[\left(\begin{array}{rrr|r}2&-1&-1&\alpha\\1&-2&1&\beta\\1&1&-2&\gamma\end{array}\right)\]
	Interchanging third row with the first gives
	\[\left(\begin{array}{rrr|r}1&1&-2&\gamma\\1&-2&1&\beta\\2&-1&-1&\alpha\end{array}\right)\]
	Subtracting first row from the second once and from the third twice gives
	\[\left(\begin{array}{rrr|r}1&1&-2&\gamma\\0&-3&3&\beta-\gamma\\0&-3&3&\alpha-2\gamma\end{array}\right)\]
	Thus we see that $(\alpha,\beta,\gamma)$ are subject to sole requirement $\beta-\gamma=\alpha-2\gamma\iff
	\alpha=\beta+\gamma$. Thus, taking $\beta$ and $\gamma$ to be arbitrary real numbers, we get
	\[\begin{pmatrix}\alpha\\\beta\\\gamma\end{pmatrix}=\begin{pmatrix}\beta+\gamma\\\beta\\\gamma\end{pmatrix}=
		\beta\begin{pmatrix}1\\1\\0\end{pmatrix}+\beta\begin{pmatrix}1\\0\\1\end{pmatrix}\]
	thus the basis for range is given by $\mycbra{(1,0,1),(1,1,0)}$.\\
	Finally, the matrix representation is what remains to be done.
	\[f(1,0,0)=(2,1,1)\]
	\[f(0,1,0)=(-1,-2,1)\]
	\[f(0,0,1)=(-1,1,-2)\]
	Hence, the matrix expression of $f$ with given bases for domain and codomain is
	\[\left(\begin{array}{rrr}1&0&0\\0&1&0\\0&0&1\end{array}\right)\]
\item \[f:M_2(\mathbb{R})\to M_2(\mathbb{R})\]
	\[f\mybramatii{a}{b}{c}{d}=\mybramatii{0}{1}{2}{3}\mybramatii{a}{b}{c}{d}\mybramatii{3}{4}{5}{6}\]
	where by $M_2(\mathbb{R})$ I mean the vector space of all $2\times2$ matrices. And the basis is given by
	\[\mycbra{\mybramatii{1}{0}{0}{0},\mybramatii{0}{1}{0}{0},\mybramatii{0}{0}{1}{0},\mybramatii{0}{0}{0}{1}}\]
	\\\textbf{Solution.} 
	This function is linear. Indeed, by the properties of multiplication for any $A$ and $B$ in $M_2(\mathbb{R})$ we have
	\[\begin{array}{rr}f(A+B)=&\myexplain{as by $ABC$ we usually mean $A(BC)$}\\[2em]
	=\mybramatii{0}{1}{2}{3}\mysbra{(A+B)\mybramatii{3}{4}{5}{6}}=&\myexplain{by $A(B+C)=AB+AC$}\\[2em]
	=\mybramatii{0}{1}{2}{3}\mysbra{A\mybramatii{3}{4}{5}{6}+B\mybramatii{3}{4}{5}{6}}=&\myexplain{by $(A+B)C=AC+BC$}\\[2em]
	=\mybramatii{0}{1}{2}{3}A\mybramatii{3}{4}{5}{6}+\mybramatii{0}{1}{2}{3}B\mybramatii{3}{4}{5}{6}&=
	f(A)+f(B)\end{array}\]
	and similarly
	\[\begin{array}{rr}
	f(\alpha A)=\mybramatii{0}{1}{2}{3}\mysbra{(\alpha A)\mybramatii{3}{4}{5}{6}}=&\myexplain{by $(\alpha A)B=\alpha(AB)$}\\[2em]
	\mybramatii{0}{1}{2}{3}\mysbra{\alpha \mycbra{A\mybramatii{3}{4}{5}{6}}}=&\myexplain{by $A(\alpha B)=\alpha(AB)$}\\[2em]
	\alpha\mysbra{\mybramatii{0}{1}{2}{3}{\mycbra{A\mybramatii{3}{4}{5}{6}}}}&=\alpha f(A)
	\end{array}\]
	Now, let's do the matrix representation
	\[f\mybramatii{1}{0}{0}{0}=
	\left(\begin{array}{rr}
	0&0\\
	6&8\\\end{array}\right)\]

	\[f\mybramatii{0}{1}{0}{0}=
\left(\begin{array}{rr}
0&0\\
10&12\\
\end{array}\right)\]

	\[f\mybramatii{0}{0}{1}{0}=
\left(\begin{array}{rr}
3&4\\
9&12\\
\end{array}\right)\]

	\[f\mybramatii{0}{0}{0}{1}=
\left(\begin{array}{rr}
5&6\\
15&18\\
\end{array}\right)=5\mybramatii{1}{0}{0}{0}+6\mybramatii{0}{1}{0}{0}+15\mybramatii{0}{0}{1}{0}+18\mybramatii{0}{0}{0}{1}\]
This gives us matrix representation for $f$:
\[L=\left(\begin{array}{rrrr}
	0&0&3&5\\
	0&0&4&6\\
	6&10&9&15\\
	8&12&12&18\\
\end{array}\right)\]
Now, as the kernel of $f$ corresponds to null-space of this matrix, and latter is unaffected by row operations, we determine it
by performing gaussian elimination on $A$. It turns out that $L$ has reduced row echelon form equal to identity matrix, hence
it's null-space is zero and has basis equal to empty set. As sum of dimensions of column-space and null-space should be equal
to number of columns (four, in this case), column-space should have dimension 4, hence range of $f$ coincides with the whole $M_2(
\mathbb{R})$ and it's basis can be taken to be the same, as given in problem statement.
\item \[f:M_3(\mathbb{R})\to M_3(\mathbb{R})\]%prob 4,p.42 lin alg
	\[f\mybramatiii{a}{b}{c}{d}{e}{f}{g}{h}{i}=\mybramatiii{a}{d}{g}{b}{e}{h}{c}{f}{i}\]
	where by $M_3(\mathbb{R})$ I mean the vector space of all $3\times3$ matrices. And the basis is given by
	\[\left\{
	\mybramatiii{1}{0}{0}{0}{0}{0}{0}{0}{0},
	\mybramatiii{0}{1}{0}{0}{0}{0}{0}{0}{0},
	\mybramatiii{0}{0}{1}{0}{0}{0}{0}{0}{0},
	\mybramatiii{0}{0}{0}{1}{0}{0}{0}{0}{0},
	\mybramatiii{0}{0}{0}{0}{1}{0}{0}{0}{0},\right.\]\[\left.
	\mybramatiii{0}{0}{0}{0}{0}{1}{0}{0}{0},
	\mybramatiii{0}{0}{0}{0}{0}{0}{1}{0}{0},
	\mybramatiii{0}{0}{0}{0}{0}{0}{0}{1}{0},
	\mybramatiii{0}{0}{0}{0}{0}{0}{0}{0}{1}
	\right\}\]
	\\\textbf{Solution.} 
	This function is linear. Indeed, 
	\[f\mysbra{\mybramatiii{a}{b}{c}{d}{e}{f}{g}{h}{i}+
	\mybramatiii{a'}{b'}{c'}{d'}{e'}{f'}{g'}{h'}{i'}}=f\mybramatiii{a+a'}{b+b'}{c+c'}{d+d'}{e+e'}{f+f'}{g+g'}{h+h'}{i+i'}=\]
	\[=\mybramatiii{a+a'}{d+d'}{g+g'}{b+b'}{e+e'}{h+h'}{c+c'}{f+f'}{i+i'}=
	\mybramatiii{a}{d}{g}{b}{e}{h}{c}{f}{i}+
	\mybramatiii{a'}{d'}{g'}{b'}{e'}{h'}{c'}{f'}{i'}=\]\[=f\mybramatiii{a}{b}{c}{d}{e}{f}{g}{h}{i}+
	f\mybramatiii{a'}{b'}{c'}{d'}{e'}{f'}{g'}{h'}{i'}\]
	and 
	\[f\mysbra{\alpha\mybramatiii{a}{b}{c}{d}{e}{f}{g}{h}{i}}=
	f\mybramatiii{\alpha a}{\alpha b}{\alpha c}{\alpha d}{\alpha e}{\alpha f}{\alpha g}{\alpha h}{\alpha i}=\]
	\[=\mybramatiii{\alpha a}{\alpha d}{\alpha g}{\alpha b}{\alpha e}{\alpha h}{\alpha c}{\alpha f}{\alpha i}=
	\alpha\mybramatiii{ a}{ d}{ g}{ b}{ e}{ h}{ c}{ f}{ i}=\alpha f{\mybramatiii{a}{b}{c}{d}{e}{f}{g}{h}{i}}\]
	Next, let's compute the matrix representation. As
	\[f\left(\begin{array}{rrr}
	1&0&0\\
	0&0&0\\
	0&0&0\\
	\end{array}\right)
	=\left(\begin{array}{rrr}
	1&0&0\\
	0&0&0\\
	0&0&0\\
	\end{array}\right)
	\]
	\[f\left(\begin{array}{rrr}
	0&1&0\\
	0&0&0\\
	0&0&0\\
	\end{array}\right)
	=\left(\begin{array}{rrr}
	0&0&0\\
	1&0&0\\
	0&0&0\\
	\end{array}\right)
	\]
	\[f\left(\begin{array}{rrr}
	0&0&1\\
	0&0&0\\
	0&0&0\\
	\end{array}\right)
	=\left(\begin{array}{rrr}
	0&0&0\\
	0&0&0\\
	1&0&0\\
	\end{array}\right)
	\]
	\[f\left(\begin{array}{rrr}
	0&0&0\\
	1&0&0\\
	0&0&0\\
	\end{array}\right)
	=\left(\begin{array}{rrr}
	0&1&0\\
	0&0&0\\
	0&0&0\\
	\end{array}\right)
	\]
	\[f\left(\begin{array}{rrr}
	0&0&0\\
	0&1&0\\
	0&0&0\\
	\end{array}\right)
	=\left(\begin{array}{rrr}
	0&0&0\\
	0&1&0\\
	0&0&0\\
	\end{array}\right)
	\]
	\[f\left(\begin{array}{rrr}
	0&0&0\\
	0&0&1\\
	0&0&0\\
	\end{array}\right)
	=\left(\begin{array}{rrr}
	0&0&0\\
	0&0&0\\
	0&1&0\\
	\end{array}\right)
	\]
	\[f\left(\begin{array}{rrr}
	0&0&0\\
	0&0&0\\
	1&0&0\\
	\end{array}\right)
	=\left(\begin{array}{rrr}
	0&0&1\\
	0&0&0\\
	0&0&0\\
	\end{array}\right)
	\]
	\[f\left(\begin{array}{rrr}
	0&0&0\\
	0&0&0\\
	0&1&0\\
	\end{array}\right)
	=\left(\begin{array}{rrr}
	0&0&0\\
	0&0&1\\
	0&0&0\\
	\end{array}\right)
	\]
	\[f\left(\begin{array}{rrr}
	0&0&0\\
	0&0&0\\
	0&0&1\\
	\end{array}\right)
	=\left(\begin{array}{rrr}
	0&0&0\\
	0&0&0\\
	0&0&1\\
	\end{array}\right)
	\]
	we have matrix representation of $f$ being equal to
	\[\left(\begin{array}{rrrrrrrrr}
	1&0&0&0&0&0&0&0&0\\
	0&0&0&1&0&0&0&0&0\\
	0&0&0&0&0&0&1&0&0\\
	0&1&0&0&0&0&0&0&0\\
	0&0&0&0&1&0&0&0&0\\
	0&0&0&0&0&0&0&1&0\\
	0&0&1&0&0&0&0&0&0\\
	0&0&0&0&0&1&0&0&0\\
	0&0&0&0&0&0&0&0&1\\
	\end{array}\right)\]
	As can be verified directly, its reduced row echelon form is the identity matrix, hence kernel of $f$ is trivial vector
	space, and the basis can be taken to be empty, while its range is the whole $M_3(\mathbb{R})$, with basis that can be taken
	to be the same as in problem statement.

\item \[f:M_2(\mathbb{R})\to M_2(\mathbb{R})\]%prob 4,p.42 lin alg
	\[f(X)=\mybramatii{2}{-1}{1}{0}X+X\mybramatii{-3}{1}{-4}{1}\]
	where by $M_2(\mathbb{R})$ I mean the vector space of all $2\times2$ matrices. And the basis is given by
	\[\mycbra{\mybramatii{1}{0}{0}{0},\mybramatii{0}{1}{0}{0},\mybramatii{0}{0}{1}{0},\mybramatii{0}{0}{0}{1}}\]
	\\\textbf{Solution.}
	Again, this function is linear
	\[\begin{array}{rl}f(X+Y)
		=\mybramatii{2}{-1}{1}{0}(X+Y)+(X+Y)\mybramatii{-3}{1}{-4}{1}=&\myexplain{by $A(B+C)=AB+AC$}\\[2em]
		&\myexplain{and $(A+B)C=AC+BC$}\\[2em]
		=\mybramatii{2}{-1}{1}{0}X+\mybramatii{2}{-1}{1}{0}Y+X\mybramatii{-3}{1}{-4}{1}+
		Y\mybramatii{-3}{1}{-4}{1}=&\myexplain{by $A+B=B+A$}\\[2em]
		=\mybramatii{2}{-1}{1}{0}X+X\mybramatii{-3}{1}{-4}{1}+\mybramatii{2}{-1}{1}{0}Y+
		Y\mybramatii{-3}{1}{-4}{1}=&f(X)+f(Y)
	\end{array}\]
	and
	\[\begin{array}{rr}f(\alpha X)=
		\mybramatii{2}{-1}{1}{0}(\alpha X)+(\alpha X)\mybramatii{-3}{1}{-4}{1}=&\myexplain{by $\alpha AB=A(\alpha B)
		=(\alpha A)B$}\\[2em]
		=\alpha\mysbra{\mybramatii{2}{-1}{1}{0}X}+\alpha\mysbra{X\mybramatii{-3}{1}{-4}{1}}=&\myexplain{by $\alpha A+
		\alpha B=\alpha(A+B)$ for vector spaces}\\[2em]
		=\alpha\mysbra{\mybramatii{2}{-1}{1}{0}X+X\mybramatii{-3}{1}{-4}{1}}&=\alpha f(X)
	\end{array}\]
	Next, matrix form. As we have
	\[f\left(\begin{array}{rr}
	1&0\\
	0&0\\
	\end{array}\right)
	=\left(\begin{array}{rr}
	-1&1\\
	1&0\\
	\end{array}\right)
	\]
	\[f\left(\begin{array}{rr}
	0&1\\
	0&0\\
	\end{array}\right)
	=\left(\begin{array}{rr}
	-4&3\\
	0&1\\
	\end{array}\right)
	\]
	\[f\left(\begin{array}{rr}
	0&0\\
	1&0\\
	\end{array}\right)
	=\left(\begin{array}{rr}
	-1&0\\
	-3&1\\
	\end{array}\right)
	\]
	\[f\left(\begin{array}{rr}
	0&0\\
	0&1\\
	\end{array}\right)
	=\left(\begin{array}{rr}
	0&-1\\
	-4&1\\
\end{array}\right)=0\mybramatii{1}{0}{0}{0}-1\mybramatii{0}{1}{0}{0}-4\mybramatii{0}{0}{1}{0}+1\mybramatii{0}{0}{0}{1}\]
And thus matrix representation is
\[\left(\begin{array}{rrrr}
-1&-4&-1&0\\
1&3&0&-1\\
1&0&-3&-4\\
0&1&1&1\\
\end{array}\right)
\]
As null-space is unchanged by matrix operations, we can perform gaussian elimination to find it (and hence, the kernel of $f$).
One possible reduced row echelon form is.
\[\left(\begin{array}{rrrr}1&0&-3&-4\\0&1&1&1\\0&0&0&0\\0&0&0&0\end{array}\right)\]
Thus we see that basis of kernel can be taken to be the pair of quadruples $(4,-1,0,1)$ and $(3,-1,1,0)$, which corresponds
to matrices
\[\left(\begin{array}{rr}4&-1\\0&1\end{array}\right)\mbox{ and }\left(\begin{array}{rr}3&-1\\1&0\end{array}\right)\]
Image of $f$ is, therefore two-dimensional and to find a basis of it, it suffices to guess two linearly independent matrices
in it. In particular,
	\[f\left(\begin{array}{rr}
	1&0\\
	0&0\\
	\end{array}\right)
	=\left(\begin{array}{rr}
	-1&1\\
	1&0\\
	\end{array}\right)
	\]
	and 
	\[f\left(\begin{array}{rr}
	0&1\\
	0&0\\
	\end{array}\right)
	=\left(\begin{array}{rr}
	-4&3\\
	0&1\\
	\end{array}\right)
	\]
	would do.
\end{enumerate}
\end{document}
