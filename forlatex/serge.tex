\documentclass[8pt]{article} % use larger type; default would be 10pt

%\usepackage[utf8]{inputenc} % set input encoding (not needed with XeLaTeX)
\usepackage[10pt]{type1ec}          % use only 10pt fonts
\usepackage[T1]{fontenc}
%\usepackage{CJK}
\usepackage{graphicx}
\usepackage{float}
\usepackage{CJKutf8}
\usepackage{subfig}
\usepackage{amsmath}
\usepackage{amsfonts}
\usepackage{hyperref}
\usepackage{enumerate}
\usepackage{enumitem}

\newtheorem{prob}{Problem}

\newenvironment{solution}%
{\par\textbf{Solution}\space }%
{\par}

\title{Statistics}
\date{}

\begin{document}
\maketitle
\section{Chapter 6}
\begin{prob}6.63\end{prob}
\begin{solution}
	Random sample itself is a (ordered list of) random variables. Therefore, sample mean, being the function of a random sample is naturally
	random variable on its own. Thus, it has its own distribution, which is called \textbf{sampling distribution}. Later, of course, depends
	on a size $n$ of a random sample.
\end{solution}
\begin{prob}6.64\end{prob}
\begin{solution}
\begin{enumerate}[label=\alph*.]
	\item{\[\dbinom{6}{2}=15\]}
	\item{Calling funds $a$,$b$,$c$,$d$,$e$,$f$ respectively possible samples are: $(a,b),(a,c),(a,d),(a,e),(a,f),(b,c),$
		$(b,d),(b,e),(b,f),$
		$(c,d),(c,e),(c,f),(d,e),(d,f),(e,f)$.
		}
	\item{For samples in order listed above sample means are: 40, 38, 38, 37, 39.5,
		37, 37, 36, 38.5, 35, 34, 36.5, 34, 36.5, 35.5. Hence probability function is
		\[\begin{array}{l r}
			f(x)=\frac{2}{15}, & \mbox{if } x=34,\\
			f(x)=\frac{1}{15}, & \mbox{if } x=35,\\
			f(x)=\frac{1}{15}, & \mbox{if } x=35.5,\\
			f(x)=\frac{1}{15}, & \mbox{if } x=36,\\
			f(x)=\frac{2}{15}, & \mbox{if } x=36.5,\\
			f(x)=\frac{3}{15}, & \mbox{if } x=37,\\
			f(x)=\frac{2}{15}, & \mbox{if } x=38,\\
			f(x)=\frac{1}{15}, & \mbox{if } x=38.5,\\
			f(x)=\frac{1}{15}, & \mbox{if } x=39.5,\\
			f(x)=\frac{1}{15}, & \mbox{if } x=40,\\
			f(x)=0, & \mbox{otherwise}
		\end{array}\]
		}
	\item{Indeed, both mean of sample distribution and population mean are equal to $36\frac{5}{6}$
		}
\end{enumerate}
\end{solution}
\begin{prob}6.65\end{prob}
\begin{solution}
	Central limit theorem states basically that for samples, whose size is sufficiently big, the corresponding sample mean's sample
	distribution will be almost \textbf{normal} with the sample mean, equal to that of population and variation, that converges to zero.
	This has very important consequence, as it says that for big sample sizes mean is very good estimator of mean (without making any
	assumptions about population distribution) and moreover that behaviour of this estimator, its distribution, is independent
	of population distribution (for big $n$), which is, of course, very strong and somewhat unexpected result.
\end{solution}
\begin{prob}6.66\end{prob}
\begin{solution}
	Since population's distribution is normal, the distribution of sample mean will be normal as well (with mean $\hat{\mu}=\mu=420$ and
	sample variance $\hat{\sigma}=\sigma/\sqrt{n}=20$). Hence,
\begin{enumerate}[label=\alph*.]
	\item{
	\[P(\bar{X}>450)=P(420+20Z>450)=P(Z>1.5)=0.06681\]
	where $Z$ denotes standard normal random variable
		}
	\item{Similarly, \[P(400<\bar{X}<450)=P(-1<Z<1.5)=0.77453475\]}
	\item{Similarly,
		\[P(\bar{X}>x)=0.1\iff P(420+20Z\leq x)=0.9\iff P(Z\leq (x-420)/20)=0.9\]
		Now, by looking up the table, we see that $(x-420)/20\approx 1.29\implies x\approx 445.80$
		}
	\item{By symmetry of normal distribution $x\approx 420-(445.80-420)=394.2$}
	\item{As we know, \[s^2\sim \frac{\sigma^2}{n-1}\chi_{n-1}^2=\frac{1250}{3}\chi_{24}^2\]
		Hence\[P(s\leq x)=0.95\iff P(\frac{1250}{3}\chi_{24}^2\leq x^2)=0.95\]
		By looking up in the table we see that $\frac{3}{1250}x^2\approx 36.45$ and therefore $x\approx 123.23$
		}
	\item{Similarly to previous item,
		\[P(s\leq x)=0.05\iff P(\frac{1250}{3}\chi_{24}^2\leq x^2)=0.05\]
		Hence $\frac{3}{1250}x^2\approx 13.88$ and therefore $x\approx 76.04$
		}
	\item{Probability would be smaller. This is because $\bar{X}$ while still having normal distribution (with mean 420),
		would have smaller variance.
		Hence, probability that it exceeds particular fixed value would be smaller.
		\begin{figure}[H]
		\centering
		\includegraphics[width=\textwidth]{../Downloads/20130123_204851.jpg}
		\caption{Probability would become smaller}
		\end{figure}
		}
\end{enumerate}
\end{solution}

\begin{prob}6.67\end{prob}
\begin{solution}
\begin{enumerate}[label=\alph*.]
	\item{Similarly to previous problem, $P\approx 0.1586$

		}
	\item{Using same methods as in previous problem, $x\approx 53.59$
		}
	\item{Again,
		\[P(s^2<x)=0.9\iff P(\frac{100}{4}\chi_3^2<x)=0.9\]
		From tables, $\frac{4}{100}x\approx 6.28\implies x\approx 157$
		}
	\item{Similarly, $x/25\approx 0.59\implies x\approx 14.75$}
	\item{The probability that a single call takes more than 65 minutes is $0.308$. Hence, if we consider each of four members of the sample
		as an individual elements,
		\[P=\dbinom{4}{4}(0.308)^4(1-0.308)^0+\dbinom{4}{3}(0.308)^3(1-0.308)^1=0.08987\]
		}
\end{enumerate}
\end{solution}

\end{document}
