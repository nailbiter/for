
\documentclass[10pt]{article} % use larger type; default would be 10pt

%%\usepackage[T1,T2A]{fontenc}
%%\usepackage[utf8]{inputenc}
%%\usepackage[english,ukrainian]{babel} % може бути декілька мов; остання з переліку діє по замовчуванню. 
\usepackage{enumerate}
\usepackage{CJKutf8}
\usepackage{enumerate}
\usepackage{mystyle}
\usepackage{amsmath}

\newcommand{\diag}{\mbox{diag}}

\author{Alex Leontiev, 45-146044}
\title{Discussion}
\begin{document}
\begin{CJK}{UTF8}{min}
\maketitle
小林先生\\

公式(6.3.4)を真似て
$n_a$の$\tilde{C}$への作用は
\[n_a\begin{pmatrix}x_0\\x\\x_{p+q-1}\end{pmatrix}=\begin{pmatrix}x_0-^txw_0a\\x\\x_{p+q-1}-^txw_0a\end{pmatrix}+
\frac{x_0-x_{p+q-1}}{2}\begin{pmatrix}-Q(a)\\2a\\-O(a)\end{pmatrix}\]\\
($w_0:=\diag(I_p,-I_q)$という記号が使っています)

ですので、
\[n_a\begin{pmatrix}0\\1\\0\\0\\\vdots\\0\\1\\0\end{pmatrix}
=\begin{pmatrix}1\\0\\0\\\vdots\\0\\-1\end{pmatrix}\]
ということが必ず成り立たたないと思います。\\

何故かと言うと、
\[n_a\begin{pmatrix}0\\1\\0\\0\\\vdots\\0\\1\\0\end{pmatrix}=
\begin{pmatrix}x_0-^txw_0a\\x\\x_{p+q-1}-^txw_0a\end{pmatrix}=
\begin{pmatrix}\star\\1\\0\\0\\\vdots\\0\\1\\\star\end{pmatrix}\neq
\begin{pmatrix}1\\0\\0\\\vdots\\0\\-1\end{pmatrix}
\]
からです。\\

ですので、私の昨日の反例がやはり正しいと思います。\\

アレックス
\end{CJK}
\begin{thebibliography}{9}
\bibitem{kmano}Toshiyuki Kobayashi, and Gen Mano. 
	{\em The Schrödinger model for the minimal representation of the indefinite orthogonal group $O(p, q)$}. American Mathematical Soc., 2011.
\end{thebibliography}
\end{document}
