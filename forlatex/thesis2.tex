\documentclass[12pt,fleqn]{article} % use larger type; default would be 10pt

\usepackage{color}
\usepackage[table]{xcolor}
\usepackage{colortbl}

\usepackage{mystyle}
\newcommand{\ds}{\;ds}
\newcommand{\dt}{\;dt}
\newcommand{\dx}{\;dx}
\newcommand{\dta}{\;d\tau}

\title{Exponential Dichotomy}
\author{Alex Leontiev}
\begin{document}
\maketitle
\rowcolors{1}{white}{gray}
\begin{center}
\newcommand{\mygraycenteredcell}[1]{\multicolumn{1}{c|}{\cellcolor{gray}#1}}
\newcommand{\mygraycenteredcello}{\cellcolor{gray}}

\begin{tabular}{ |l| p{0.35\textwidth} | p{0.35\textwidth} | p{0.35\textwidth}| }
\mygraycenteredcello&\multicolumn{2}{c|}{\cellcolor{gray}Dichotomy Conditions}&\mygraycenteredcello\\\hline
&\mygraycenteredcell{Necessary}&\mygraycenteredcell{Sufficient}&\mygraycenteredcell{Roughness Conditions}\\\hline\hline
$\mathbb{R}^+$&\textbf{both iff} $BC^1(\mathbb{R})\ni x\mapsto \dot{x}-Ax\in BC(\mathbb{R})$ is Fredholm operator.
& & \\\hline\hline
$\mathbb{R}$&\textbf{iff} Exponential dichotomy on both $\mathbb{R}^+$ and $\mathbb{R}^-$ and $\mathbb{R^n}$ is sum
of stable and unstable subspace.& & \\\hline
\end{tabular}
\end{center}
\end{document}
