%\documentclass[10pt]{article} % use larger type; default would be 10pt
\documentclass[14pt]{extarticle} % use larger type; default would be 10pt

\usepackage{mathtext}                 % підключення кирилиці у математичних формулах
                                          % (mathtext.sty входить в пакет t2).
\usepackage[T1,T2A]{fontenc}         % внутрішнє кодування шрифтів (може бути декілька);
                                          % вказане останнім діє по замовчуванню;
                                          % кириличне має співпадати з заданим в ukrhyph.tex.
\usepackage[utf8]{inputenc}       % кодування документа; замість cp866nav
                                          % може бути cp1251, koi8-u, macukr, iso88595, utf8.
\usepackage[english,ukrainian]{babel} % національна локалізація; може бути декілька
                                          % мов; остання з переліку діє по замовчуванню. 

%\usepackage{sectsty}   %in order to make chapter headings and title centered
%\chapterfont{\centering}

\usepackage{amsthm}
\usepackage{amsmath}
\usepackage{amsfonts}
\usepackage{graphicx}
\usepackage[pdftex]{hyperref}
\usepackage{caption}
\usepackage{subfig}
\usepackage{fancyhdr}
\usepackage{mystyle}
\usepackage{color}
\usepackage[table]{xcolor}
\usepackage{colortbl}

%custom command for title
\newcommand{\HRule}{\rule{\linewidth}{0.5mm}}

%for Re and Im like in the book
\renewcommand\Re{\operatorname{Re}}
\renewcommand\Im{\operatorname{Im}}

%more space after \forall and \exists
\let\oldforall\forall
\renewcommand{\forall}{\oldforall\;}
\let\oldexists\exists
\renewcommand{\exists}{\oldexists\;}

%custom theorem environments
\newtheorem{definition}{Означення}[section]
\renewcommand{\thedefinition}{\arabic{definition}}
\newtheorem{example}{\indent Приклад}[section]
\renewcommand{\theexample}{\arabic{example}}
\newtheorem{exercise}{Вправа}
\newtheorem{theorem}{Теорема}
\newtheorem{lemma}{Лема}
\newtheorem{observation}{Спостереження}
\newtheorem*{fact}{Факт}
\newtheorem{proposition}{Твердження}
\newtheorem{corollary}[proposition]{Наслідок}
\theoremstyle{remark}
\newtheorem{remark}{Зауваження}

\begin{document}
\begin{titlepage}
	\addtolength{\voffset}{-3cm}
	\setlength{\footskip}{5.5cm}
	\thispagestyle{fancy}
	\fancyfoot[C]{м. Київ -- 2014}
	\begin{center}
		\textsc{\Large Київський Національний Університет імені Тараса Шевченка}\\[0.5cm]
		\textsc{\Large Механіко-математичний факультет}\\[0.5cm]
		\textsc{\Large Кафедра інтегральних та диференціальних рівнянь}\\[1.5cm]

		\textsc{\Large курсова робота}\\
		\textsc{\Large на тему:}\\[0.5cm]

		% Title
		\HRule \\[0.4cm]
		{ \huge \bfseries Експоненціально дихотомічні лінійні системи диференціальних рівнянь}\\[0.4cm]

		\HRule \\[1.5cm]

		% Author and supervisor
		\begin{minipage}{0.4\textwidth}
			\begin{flushleft} \large
				\emph{Студент:}\\
				\textsc{Леонтьєв} Олексій Костянтинович
			\end{flushleft}
		\end{minipage}
		\begin{minipage}{0.4\textwidth}
			\begin{flushright} \large
				\emph{Керівник:} \\
				\textsc{Фекета} Петро Володимирович
			\end{flushright}
		\end{minipage}

		\vfill

		% Bottom of the page
		{\large \today}
	\end{center}
\end{titlepage}
\tableofcontents
\section*{Вступ}
\addcontentsline{toc}{section}{Вступ}
Експоненціальна дихотомія є явищем, що має місце в системах однорідних лінійних диференціальних рівнянь. Вивчення цього явища
закладено ще роботами Ж. Адамара та O. Перрона. В той час як дихотомія при першому погляді на означення здається ненатуральним
і складним концептом, вона тісно пов’язана з експоненціальною поведінкою розв’язків рівняння і дає інформацію про існування рівномірно обмежених
в розв’язків однорідних систем із збуреннями, якщо останні також рівномірно обмежені.

Попри всю свою цікавість і корисність, автору відома лише одна монографія, присвячена дихотомії \cite{coppel}, в той час як
більшість вводить її просто як допоміжне
означення для встановлення інших результатів. Це робить роботу з пошуку інформації дещо більш важкою, адже необхідні факти і леми
розподілені по великій кількості літератури, більшість з якої, у свою чергу, невідома автору цієї роботи.

Окрім чудової монографій Copell'а, у
даній роботі використовувалася література: Ю. Л. Далецкий, М. Г. Крейн
\emph{Устойчивость решений дифференциальных уравнений в банаховом пространстве} та Митропольський Ю. А., Самойленко А. М., Кулик В. Л.
\emph{Исследование дихотомии линейных систем дифференциальных уравнений с помощью функций Ляпунова}. З них перша, як видно з назви,
досліджує широкий спектр явищ, пов’язаних із стійкістю розв’язків диференціальних рівнянь (в тому числі, і дихотомію) в дещо
більш загальному контексті банахових просторів, в той час як друга використовує дихотомію для дослідження інваріантних торів.

Цікавим напрямом дослідження є так звана "грубість" дихотомії, тобто критерії на матриці збурення $B$ такі, що дихотомія системи $x'=Ax$ зберігається
для системи $x'=(A+B)x$. Як відомо (і видно з таблиці нижче), дихотомія на $R$, є значно "менш" грубою, ніж дихотомія на $\mathbb{R}$, адже
класи збурень, які її зберігають, вужчі. Тому, природно, цікаво знайти достатньо широкі критерії для збурників для збереження дихотомії на $
\mathbb{R}$.

Під час читання літератури, авторові спало на думку побудувати наступну таблицю. У першій колонці наведено необхідні критерії
дихотомії, у другій -- достатні. Знаком \textbf{iff} відмічені достатні критерії, які є еквівалентними. В третій колонці
наведено умови на матрицю $B=B(t)$ таку, щоб система $x'=(A+B)x$ зберігала дихотомію на $\mathbb{R}^+$ чи $\mathbb{R}$, якщо
система $x'=Ax$ її мала.
\rowcolors{1}{white}{gray}
\begin{center}
\newcommand{\mygraycenteredcell}[1]{\multicolumn{1}{c|}{\cellcolor{gray}#1}}
\newcommand{\mygraycenteredcello}{\cellcolor{gray}}

\begin{tabular}{ |l| p{0.30\textwidth} | p{0.30\textwidth} | p{0.30\textwidth}| }
	\mygraycenteredcello&\multicolumn{2}{c|}{\cellcolor{gray}Критерії дихотомії}&\mygraycenteredcello\\\hline
&\mygraycenteredcell{Необхідні}&\mygraycenteredcell{Достатні}&\mygraycenteredcell{Критерії грубості}\\\hline\hline

$\mathbb{R}^+$&
&
\textbf{iff} $BC^1(\mathbb{R})\ni x\mapsto \dot{x}-Ax\in BC(\mathbb{R})$ є оператором Фредгольма. \cite{palmer88}&
$\mynorm{B}_\infty<\delta$ \cite{coppel}\\\hline

&
&
\textbf{iff}$x'=Ax+f$ має принаймні один обмежений розв’язок для кожної обмеженої  $f$. \cite[\S IV, Насл. 3.1]{krein}&
$\lim_{t\to+\infty}\mynorm{B(t)}=0$\cite{coppel}\\\hline\hline
$\mathbb{R}$&
$x'=Ax+f$ має єдиний обмежений розв’язок для кожної $f$ неперервної та обмеженої на $\mathbb{R}$.\ref{MysticThmRemark}&
\textbf{iff}\qquad Дихотомія на $\mathbb{R}^+$ та $\mathbb{R}^-$, а також $\mathbb{R}^n$ є сумою стійкого і нестійкого підпросторів.
	\cite[Proposition 2.1]{palmer84} &
$\mynorm{B}_\infty<\delta$\cite{coppel}\\\hline
&
&
\textbf{iff}існує симетрична форма $S(t)$, така що похідна $V(t,x):=\mysca{S(t)x}{x}$ в силу системи
$\mysca{\mysbra{\dot{S}+SA+A^*S}x}{x}\leq-\beta\mynorm{x}^2,\;\beta>0$. \cite[Теор. 1.1, 1.2]{mitrop}&
\\\hline
\end{tabular}
\end{center}
\begin{remark}\label{MysticThmRemark}\end{remark} Дуже схоже, що цей критерій також є достатнім. Зокрема, відповідне твердження сформульоване
	в \cite[\S IV, Насл. 3.2]{krein}. Проте доведення не дане і автор цієї курсової виявився нездатним його відтворити. 
	До того ж, цей факт використовується без доведення в \cite[ст. 37]{mitrop} як "очевидний". Проте авторові все ще не відоме доведення.
У цій роботі сформульовано і доведено два результати, які дають нові умови грубості. Що цікаво, критерії ці обмежують $L^1$-норму матриці-збурника
(тобто, $L_1(B):=\int_{-\infty}^{\infty}\mynorm{B(t)}\;dt$) і, у світлі того факту, що із двох тверджень $L_1(B)<+\infty$ та $\lim\limits_{t\to
\pm\infty}\mynorm{B(t)}=0$ жоден не є сильнішим за інший, ці результати є дійсно новими, хоча й тривіальними.
\section{Базові поняття}
Експоненціальна дихотомія -- це явище, яке спостерігається в спеціальному виді систем, так званих експоненціально дихотомічних.
Ми дамо означення і декілька пояснень.

Експоненціально дихотомічні системи являють собою особливий тип однорідних систем виду
\equation\label{LinHomSysDef}\frac{dx}{dt}=A(t)x,\;x(t)\in\mathbb{R}^n\endequation

де $t\in\mathbb{R},\; A(t)$ - обмежена неперервна матрична функція на $\mathbb{R}$. 

Перш ніж продовжити, варто зробити декілька зауважень стосовно систем такого типу загалом:
\begin{remark}\label{AllNormsAreEqRemark}
	Ми вважаємо відомим і використовуємо без доведення той факт, що всі норми на $\mathbb{R}^n$ є еквівалентними,
	тобто якщо $\mynorm{\cdot}$ і $\mynorm{\cdot}'$ є двома нормами на $\mathbb{R}^n$, то існує пара $m,M>0$, залежна
	лише від $n,\;\mynorm{\cdot}$ та $\mynorm{\cdot}'$ і така, що виконуються нерівності $\forall x\in\mathbb{R}^n,\;m\mynorm{x}\leq
	\mynorm{x}'\leq M\mynorm{x}$. Таким чином, усюди, де ми будемо казати що певний об’єкт лінійної алгебри (матриця, вектор тощо) або ж
	функція, значеннями якої є такі об’єкти є обмеженою, ми не будемо уточнювати, яка норма мається на увазі і в доведеннях будемо 
	використовувати ті норми, які є доречними в даному контексті
	, особливо не промовляючи це. Окремо зауважимо, що у випадку операторів так звані "операторні норми" також
	є нормами, якщо ми використаємо стандартне ототожнення оператора на $\mathbb{R}^n$ з вектором в $\mathbb{R}^{n\times n}$.
\end{remark}
\begin{remark}\label{SolsExistAndUniqRemark}
	Для довільних $t_0\in\mathbb{R},\;x_0\in\mathbb{R}^n$ існує єдиний визначений на 
$\mathbb{R}$ розв'язок $y:\mathbb{R}\mapsto\mathbb{R}^n$, що задовольняє початковим умовам $y(t_0)=x_0$.
Це випливає з теореми Піка, яка стверджує, що функція $F(t,x)$, що набуває значень в $\mathbb{R}^n$, визначена на 
$\Omega=\left\{(t,x) \mid \myabs{t-t_0}\leq a;\;\mynorm{x-x_0}_1:=\max_{1\leq i\leq n}\myabs{x^i-x^i_0}\leq b\right\}$ і відповідає умові Ліпшиця по другому аргументу (тобто 
$\exists M,\;\forall (t,x),\;(t,y)\in\Omega\;\mynorm{F(t,x)-F(t,y)}\leq M\mynorm{x-y}$), то задача Коші $\frac{dx}{dt}=F(t,x),\;x(t_0)=x_0$
має єдиний розв’язок принаймні на інтервалі $\myabs{t-t_0}\leq h:=\min\left\{a;\frac{b}{M}\right\}$. Таким чином, нам лише потрібно показати,
що $F(t,x):=A(t)x$ задовольняє умові Ліпшиця по другому аргументу при даній гіпотезі стосовно $A(t)$. Оскільки $\forall t\in\mathbb{R}$
операторна норма матриці $A(t)$, узгоджена з нормою $\mynorm{\cdot}_2$ на $\mathbb{R}^n$ не перевищує певне $M>0$
(див. попереднє зауваження \ref{AllNormsAreEqRemark}), маємо 
$\forall (t,x),\;(t,y)\in\mathbb{R}\times\mathbb{R}^n,\;\mynorm{F(t,x)-F(t,y)}_2=\mynorm{A(t)(x-y)}\leq M\mynorm{x-y}_2$ (помітимо,
що через еквівалентність норм неважливо, яку конкретно норму ми використовуємо в умові Ліпшиця в даний момент)
\end{remark}
\begin{remark}
	Ми позначатимемо матриціант системи \ref{LinHomSysDef} як $\Omega_a^b(A)$. За означенням це лінійний
	оператор на $\mathbb{R}$ і $\Omega_a^b(A)(x_0):=x(b)$, де $x(t)$ - це розв’язок задачі Коші $\frac{dx}{dt}=
	A(t)x,\;x(a)=x_0$. Через однорідність задачі, матриціант є дійсно лінійним (тому надалі ми
	писатимемо $\Omega_a^b(A)x_0$ замість $\Omega_a^b(A)(x_0)$) і попереднє зауваження \ref{SolsExistAndUniqRemark}
	гарантує, що він завжди визначений однозначно. За означенням, $Omega^t_t(A)=I_n$.
\end{remark}
\begin{remark}
Наведемо декілька властивостей матриціанта, які будуть корисні у подальшому разом із стислими доведеннями:
\begin{itemize}
	\item{Знову ж таки, безпосередньо з означення і єдиності розв’язку маємо $\Omega_b^c(A)\cdot\Omega_a^b(A)=\Omega_a^c(A)$}
	\item{З попередньої властивості $\Omega_a^b(A)\cdot\Omega_b^a(A)=\Omega_b^b(A)=I_n$, тому $\Omega_b^a(A)=\left(\Omega_a^b(A)\right)^{-1}$
		і ми бачимо, що $\Omega_a^b(A)$ є автоморфізмом $\mathbb{R}^n$}
\end{itemize}
\end{remark}

Таким чином, система виду \ref{LinHomSysDef} називається \textbf{експоненціально дихотомічною} на $A\subset\mathbb{R}$
(або просто "дихотомічною" в подальшому) якщо векторний простір $\mathbb{R}^n$ розкладається
у векторну суму двох підпросторів $E^+\oplus E^-$ і для довільних $x^+\in E^+,\; x^-\in E^-$ і $a,b\in A$ мають місце наступні оцінки:
\begin{equation}\begin{aligned}\label{DichotomyDef}
	\mynorm{\Omega^b_0(A)x^+}\leq K\mynorm{\Omega_0^a(A)x^+}\exp\left\{-\gamma\myabs{a-b}\right\},\;a\leq b\\
	\mynorm{\Omega^b_0(A)x^-}\leq K\mynorm{\Omega_0^a(A)x^-}\exp\left\{-\gamma\myabs{a-b}\right\},\;b\leq a\\
\end{aligned}\end{equation}
для сталих $K,\;\gamma>0$, що не залежать від $x^+,x^-,a,b$

\begin{remark}Альтернативне означення, як дано в \cite{coppel} формулюється так: існує оператор-проекція $P$ і числа $K,\;\gamma>0$ такі, що
	для $\forall a,b\in A$
	\[\mynorm{\Omega_0^t(A)P\mybra{\Omega_0^s(A)}^{-1}}\leq K e^{-\gamma(a-b)},\;a\geq b\]
	\[\mynorm{\Omega_0^t(A)(I-P)\mybra{\Omega_0^s(A)}^{-1}}\leq K e^{-\gamma(b-a)},\;b\geq a\]
\end{remark}
\begin{remark}
	Через причини, пояснені в зауваженні \ref{AllNormsAreEqRemark}, ми не уточнюємо, яка норма використовується в означенні
	\ref{DichotomyDef} і будемо використовувати у доведеннях ту, яка буде доречною.
\end{remark}
%remark - intuitive meaning
\begin{remark}
Немає жодної втрати загальності, якщо ми вважатимемо, що $0\in A$. Дійсно, дане довільне $o\in A$ (нас не цікавить тривіальний
випадок $A=\emptyset$) ми можемо в означенні \ref{DichotomyDef} переписати $\Omega_0^a(A)x^+$ як $\Omega_o^a(A)\Omega_0^o(A)x^+$
Таким чином, якщо ми позначимо $\Omega_0^o(A)x^+=:\tilde{x}^+\in \tilde{E}^+:=\Omega_0^o(A)E^+$. Оскільки $\Omega_0^o(A)$ є автоморфізмом,
ми все ще маємо $\mathbb{R}^n=\tilde{E}^+\oplus \tilde{E}^-$, де $\tilde{E}^-:=\Omega_0^o(A)E^-$. Ми отримаємо, після таких переозначень
\begin{equation*}
	\tag{\ref*{DichotomyDef}$'$}
	\begin{aligned}
	\mynorm{\Omega^b_o(A)\tilde{x}^+}\leq K\mynorm{\Omega_o^a(A)\tilde{x}^+}\exp\left\{-\gamma\myabs{a-b}\right\},\;a\leq b\\
	\mynorm{\Omega^b_o(A)\tilde{x}^-}\leq K\mynorm{\Omega_o^a(A)\tilde{x}^-}\exp\left\{-\gamma\myabs{a-b}\right\},\;b\leq a\\
\end{aligned}\end{equation*}

Таким чином, зсунувши все на $o$ ми можемо вважати $o\in A$ нулем або, що те ж саме, що $0\in A$.

Щоб зрозуміти означення, можна почати зі слабшої версії, тобто зафіксувати $a=0$. Це дасть нам
\begin{equation*}
	\tag{\ref*{DichotomyDef}$''$}
	\begin{aligned}
	\mynorm{\Omega^b_0(A){x}^+}\leq K\mynorm{{x}^+}\exp\left\{-\gamma\myabs{b}\right\},\; b\geq 0\\
	\mynorm{\Omega^b_0(A){x}^-}\leq K\mynorm{{x}^-}\exp\left\{-\gamma\myabs{b}\right\},\;b\leq 0\\
\end{aligned}\end{equation*}

Таким чином, якщо розв’язок починається в $E^+$, він (принаймні)
експоненціально
згасає при $b\to+\infty$ (може, швидше), в той час як розв’язки, які стартують з $E^-$,
(принаймні) експоненціально згасають на $b\to-\infty$ (можливо швидше).

З іншого боку, зафіксувавши $b=0$ отримаємо
\begin{equation*}
	\tag{\ref*{DichotomyDef}$'''$}
	\begin{aligned}
	\frac{1}{K}\exp\left\{\gamma\myabs{a}\right\}\mynorm{{x}^+}\leq \mynorm{\Omega_0^a(A){x}^+},\; a\leq 0\\
	\frac{1}{K}\exp\left\{\gamma\myabs{a}\right\}\mynorm{{x}^+}\leq \mynorm{\Omega_0^a(A){x}^-},\;a\geq 0\\
\end{aligned}\end{equation*}

Тобто розв’язки, що почалися в $E^+$ принаймні 
експоненціально зростають при $a\to-\infty$, ті, що почалися в $E^-$ проявляють аналогічну поведінку при 
$a\to+\infty$. 

Підсумовуючи вищесказане, нерівності \ref{DichotomyDef} надають нам інформацію про асимптотичну поведінку розв’язків, залежно
від їх положення в момент часу $t=0$. Разом із фактом, що увесь фазовий простір $\mathbb{R}^n=E^+\oplus E^-$ і тим, що 
розв’язки \ref{LinHomSysDef} утворюють векторний простір, це дає нам суттєву інформацію про асимптотичну поведінку усіх розв’язків
системи \ref{LinHomSysDef} і пов’язяних з нею, як ми побачимо в подальшому.
\end{remark}
\begin{remark}
	Хоча означення дихотомії \ref{DichotomyDef} було сформульоване для довільного $A\subset\mathbb{R}$, на практиці лише три випадки є 
	найбільш цікавими: дихотомія на $A=\mathbb{R}^+:=[0,+\infty)$
	, на $A=\mathbb{R}^-:=(-\infty,0]$ і на $A=\mathbb{R}$. Нижче ми спробуємо пояснити, чому це так.

	По-перше, у випадку обмеженого $A$ дихотомія виконується тривіально для кожної системи виду \ref{LinHomSysDef} із обмеженою, неперервною
	$A$. Дійсно, ми покажемо що у випадку $A\subset[-M,M]$ дихотомія виконується для $E^+:=\mathbb{R}^n$ та $E^-:=\mycbra{0}$. Ми маємо
	довести існування певних $K,\;\gamma>0$, таких що
	\[\forall x\in\mathbb{R}^n,\;a,b\in A,\;a\leq b\implies 
	\mynorm{\Omega^b_0(A){x}}\leq K\mynorm{\Omega_0^a(A){x}}\exp\left\{-\gamma\myabs{a-b}\right\}\\
	\]

	Ми покажемо більш сильне твердження:
	\[\tag{*}\label{InterestingARemarkBddDesiredStatement}
	\forall x\in\mathbb{R}^n,\;a,b\in [-M,M],\;
	\mynorm{\Omega^b_0(A){x}}\leq K\mynorm{\Omega_0^a(A){x}}\exp\left\{-\myabs{a-b}\right\}\\
	\]
	
	Оскільки для $x=0$ нерівність очевидно виконується незалежно від $K$ та $\gamma$, у подальшому ми припустимо $x\neq 0$. Більше того,
	оскільки норма лінійна, можна припустити $\mynorm{x}=1$. Ми введемо додатні функції
	\[m(r):=\min_{\mynorm{x}=1}\mynorm{\Omega^r_0(A)x}\]
	\[M(r):=\max_{\mynorm{x}=1}\mynorm{\Omega^r_0(A)x}\]
	
	Оскільки $\Omega_0^r(A)x$ є неперервною по $r$ і $x$, $m(r)$ та $M(r)$ є неперервними на $\mathbb{R}$. Таким чином, оскільки в 
	бажаному твердженні \ref{InterestingARemarkBddDesiredStatement} $\mynorm{\Omega^b_0(A)x}\leq M(b)$ і $\mynorm{\Omega^a_0(A)x}
	\geq m(a)$, твердження \ref{InterestingARemarkBddDesiredStatement} випливатиме з сильнішого твердження
	\[\tag{$\star$}\label{InterestingARemarkStronger}
	\forall x\neq 0,\;a,b\in [-M,M],\;
	M(b)\leq K\cdot m(a)\exp\left\{-\myabs{a-b}\right\}\\
	\]

	Оскільки $M(b)$ та $m(a)$ є неперервними і додатними, функція 
	\[F(a,b):=\frac{M(b)}{m(a)\exp\left\{-\myabs{a-b}\right\}}\]
	визначена на $[-M,M]\times[-M,M]$ є також неперервною, а отже обмежена згори, оскільки її область визначення компактна. Іншими словами,
	існує $K$ таке, що $\forall (a,b)\in[-M,M]\times[-M,M] F(a,b)\leq K$, що еквівалентно твердженню
	\ref{InterestingARemarkStronger}. Це показує, що обмежені $A$ не є цікавими для теорії експоненціальної дихотомії.

	По-друге, $A$ можна вважати закритою множиною. Дійсно, нехай $a_0,b_0\in\overline{A}$ (
	найменша закрита множина, що містить $A$) і система \ref{LinHomSysDef} є дихотомічною на $A$,
	тобто рівняння \ref{DichotomyDef} виконуються для $a,b\in A$. Ми покажемо, що ці рівняння виконуються і для $a_0,b_0$, тому останні можна
	вважати елементами $A$, а саме $A$ - закритим.

	Випадок $a_0=b_0$ тривіальний, адже рівняння \ref{DichotomyDef} очевидно виконуються якщо $K\geq 1$. Але остання умова випливає з дихотомії
	на $A$, адже ми можем взяти $a=b=0\in A$ і просто записати умови дихотомії, які в такому разі тягнуть за собою умову $K\geq 1$. Таким чином,
	можемо припустити $a_0\neq b_0$. Оскільки випадки $a_0<b_0$ та $a_0>b_0$ є симетричними, достатньо розглянути лише перший. Оскільки $a_0,
	b_0\in\overline{A}$, маємо $a_n,b_n\in A,\; a_n\to a_0,\;b_n\to b_0$. Оскільки за припущенням $a_0<b_0$, для великих $n$ маєм
	$a_n<b_n$. Переходячи для підпослідовності, можна вважати, що остання умова виконується для всіх $a_n,\;b_n$. Тому
	\[\forall x^+\in E^+,\;\mynorm{\Omega^{b_n}_0(A)x^+}\leq K\mynorm{\Omega_0^{a_n}(A)x^+}\exp\left\{-\gamma\myabs{a_n-b_n}\right\}\]
	Оскільки для $\Omega_y^z(A)x$ є неперервною в $y,\;z$ та $x$, отримуємо $\Omega^{a_n}(A)x^+\to\Omega^{a_0}(A)x^+$ i
	$\Omega^{b_n}(A)x^+\to\Omega^{b_0}(A)x^+$. Додавши до цього очевидне $\exp\left\{-\gamma\myabs{a_n-b_n}\right\}\to
	\exp\left\{-\gamma\myabs{a_0-b_0}\right\}$, маємо
	\[\forall x^+\in E^+,\;\mynorm{\Omega^{b_0}_0(A)x^+}\leq K\mynorm{\Omega_0^{a_0}(A)x^+}\exp\left\{-\gamma\myabs{a_0-b_0}\right\}\]

	Таким чином, нерівність \ref{DichotomyDef} виконується і для $a_0< b_0$, а тому їх можна вважати елементами $A$.

	По-третє, з суто технічної точки зору, умови експоненціальної дихотомії дають більший ефект у випадку зв’язної множини $A\subset\mathbb{R}$.
	Таким чином, природно зосередитися на випадку $A$ замкненого (див. вище) інтервалу. З поясненого вище, цей інтервал має бути необмеженим,
	якщо нас цікавлять нетривіальні випадки. Таким чином, інтервал має містити нуль (з попереднього зауваження) і окіл хоча б однієї
	з бескінечностей: $+\infty$ або $-\infty$ (можливо, обидва). Ці 3 випадки і приводять до трьох найбільш цікавих значень для $A$, на
	яких варто зосередитись: дихотомія на $\mathbb{R}^+$, на $\mathbb{R}^-$ або на $\mathbb{R}$
\end{remark}
\section{Результати}
\begin{proposition}\label{weak}Припустимо, що для довільних $t,s\in\mathbb{R}$, матриці $A(s)$, $A(t)$, $B(t)$ та $B(s)$ попарно комутують.
	Припустимо також, що $\int_{-\infty}^\infty\mynorm{B(t)}\;dt<+\infty$. Тоді система 
	\begin{equation}\label{Unperturbed}
	x'=A(t)x(t)
	\end{equation}
	має експоненціальну дихотомію на $\mathbb{R}$ тоді і лише тоді, коли її має система
	\begin{equation}\label{Perturbed}
	x'=(A(t)+B(t))x(t)
	\end{equation}
\end{proposition}
\begin{proposition}
	\label{strong}
	Припустимо, що для довільних $t,s\in\mathbb{R}$, матриці $A(s)$ та $B(t)$ комутують.
	Припустимо також, що $\int_{-\infty}^\infty\mynorm{B(t)}\;dt<+\infty$. Тоді система 
	\begin{equation}\label{Perturbed}
	x'=(A(t)+B(t))x(t)
	\end{equation}
	має експоненціальну дихотомію на $\mathbb{R}$, якщо її має система
	\begin{equation}\label{Unperturbed}
	x'=A(t)x(t)
	\end{equation}
\end{proposition}
\begin{remark}Гіпотеза
	твердження \ref{strong} щодо комутації
	задовільняється, наприклад, якщо $\forall t,\;B(t)\in\bigcap\limits_{s\in\mathbb{R}} Z_{\mathfrak{gl}_n}(A(s))$, де
	$Z_{\mathfrak{gl}_n}(A)$ позначає підалгебру Лі матриць, що комутують з $A$. У випадку $A(t)\equiv A=const$, цю умову
можна записати простіше як $\forall t,\;B(t)\in Z_{\mathfrak{gl}_n}(A)$.\end{remark}
\section{Доведення}
\begin{proof}{(твердження \ref{weak})}Враховуючи гіпотезу про комутацію, матриціантами систем \eqref{Unperturbed} та \eqref{Perturbed} будуть
	\[X(t):=\exp\mycbra{\int_0^tA(s)\;ds}\]
	та \[\tilde{X}(t):=\exp\mycbra{\int_0^t(A(s)+B(s))\;ds}=\exp\mycbra{\int_0^tA(s)\;ds}\exp\mycbra{\int_0^tB(s)\;ds}\]
	і останні два множники в правій частині комутують при всіх $t$.
	Через симетричність умови, достатньо показати лише, що з дихотомії \eqref{Unperturbed} випливає дихотомія \eqref{Perturbed}.
	Далі, із \cite{coppel} ми знаємо, що еквівалентним означенням експоненціальної дихотомії \eqref{Perturbed}
	на $\mathbb{R}$ є існування
	такого оператора-проекції $P$ і чисел $\gamma,K>0$, що $\forall s,t\in\mathbb{R}$ маємо
	\[\mynorm{X(t)PX^{-1}(s)}\leq Ke^{-\gamma(t-s)},\;s\leq t\]
	\[\mynorm{X(t)(I-P)X^{-1}(s)}\leq K^{-\gamma(s-t)},\;t\leq s\]
	Ми покажемо, що результат залишається тим же самим (із, можливо, більшим $K$), із $X(t)$ заміненим на $\tilde{X}(t)$.
	Дійсно, матимемо тоді для $s\leq t$
	\[\mynorm{\tilde{X}(t)P\tilde{X}^{-1}(s)}=\]
	\[=\mynorm{\exp\mycbra{\int_0^tB(l)\;dl}\exp\mycbra{\int_0^tA(l)\;dl}P\exp\mycbra{-\int_0^sA(l)\;dl}
	\exp\mycbra{-\int_0^sB(l)\;dl}}\leq\]
	\[\leq\mynorm{\exp\mycbra{\int_0^tB(l)\;dl}}\cdot\mynorm{\exp\mycbra{\int_0^tA(l)\;dl}P\exp\mycbra{-\int_0^sA(l)\;dl}}\times
	\]\[\times\mynorm{\exp\mycbra{-\int_0^sB(l)\;dl}}\leq\]
	\[\leq\mynorm{\exp\mycbra{\int_0^tB(l)\;dl}}\cdot Ke^{-\gamma(t-s)}\cdot
	\mynorm{\exp\mycbra{-\int_0^sB(l)\;dl}}\leq\]
	Таким чином, достатньо показати, що
	\[\mynorm{\pm\exp\mycbra{\int_0^tB(l)\;dl}}\]
	рівномірно обмежена в $t\in\mathbb{R}$. Це, в свою чергу, слідує із
	\[\mynorm{\exp\mycbra{\int_0^tB(l)\;dl}}\leq\exp\mycbra{\mynorm{\int_0^tB(l)\;dl}}\leq\exp\mycbra{\int_0^t\mynorm{B(l)}\;dl
	}\leq\]
	\[\leq\exp\mycbra{\int_{-\infty}^{\infty}\mynorm{B(l)}\;dl}<+\infty\]
	і аналогічного аргументу для $-$.
\end{proof}
\begin{proof}{(Твердження \ref{strong})}
	Бажаний результат випливатиме з двох спостережень, які ми покажемо нижче
	\begin{observation}\label{Hard}Нехай для довільних $t,s\in\mathbb{R}$, $A(t)$ та $B(s)$ комутують, де $A,B$ неперервні
	(не обов’язково обмежені) матричні функції. Припустимо також, що матриціант системи $x'=Bx$, який ми позначатимемо за
	$X_B(t)$, існує для всіх $t\in\mathbb{R}$. Тоді для довільних $t,s\in\mathbb{R}$ $A(t)$ та $X_B(s)$ комутують.\end{observation}
	\begin{observation}\label{Easy}Нехай $B$ -- неперервна матрична функція така, що $\int_{-\infty}^\infty B(s)\;ds<+\infty$. Тоді матриціант
		$X_B$ системи $x'=Bx$ існує і норми $\mynorm{X_B(t)}$ та $\mynorm{X^{-1}_B(t)}$ обмежені рівномірно в $t$ на $\mathbb{R}$.
	\end{observation}
	Дійсно, вважаючи спостереження вище вірними, розглянемо систему \ref{Perturbed} $x'=(A+B)x(t)$. Позначимо матриціант системи $x'=Bx$
	за $X_B(t)$ та зробимо заміну $x(t)=X_B(t)z(t)$ матимемо
	\[BX_Bz+X_B\dot{z}=(X_Bz)'=x'=(A+B)X_Bz\]
	оскільки за спостереженням \ref{Hard}, $A$ та $X_B$ комутують (помітимо, що умови даного твердження сильніші за умови спостереження), маємо
	\[X_B\dot{z}=X_BAz\]
	і відповідно, $z$ є розв’язком системи \ref{Unperturbed}. Таким чином, матриціант $X_{A+B}$ системи \ref{Perturbed}
	має структуру
	\[X_{A+B}=X_BX_A\]
	і ми, маючи спостереження \ref{Easy}, можемо застосувати таку ж схему оцінки, як і в попередньому твердженні
	\[\mynorm{X_{A+B}(t)PX^{-1}_{A+B}(s)}=\mynorm{X_B(t)X_A(t)PX^{-1}_A(s)X^{-1}_B(s)}\leq\]
	\[\leq\mynorm{X_B(t)}\mynorm{X_A(t)PX^{-1}_A(s)}\mynorm{X_B^{-1}(s)}\]
	\begin{proof}{(Спостереження \ref{Easy})} Рівномірна обмеженість $\mynorm{X_B(t)}$
	випливає безпосередньо з доведення Теореми 1 в \cite[\S 12]{demidovich}, де в 
	якості системи із сталою матрицею ми взяли стійку систему $x'=0x$. Позначимо отриману в доведенні зазначеною Теореми 1 верхню межу
	$\mynorm{X_B(t)}$ на $\mathbb{R}$, за
	$M$. Покажемо, що $\forall t,\;\mynorm{X^{-1}(t)}\leq M$, чим і закінчимо доведення. Дійсно, для сталого $\tau\in\mathbb{R}$,
	$X^{-1}(\tau)$ можна характеризувати як $X^{-1}(\tau)x(\tau)=x(0)$ де $x(t)$ задовольняє $x'=Bx$ або, що те ж саме, як
	$X^{-1}(\tau)y(0)=y(-\tau)$ де $y'=\tilde{B}y$, а $\tilde{B}(t):=B(t+\tau)$. Таким чином, достатньо показати, що матриціант системи
	$y'=\tilde{B}y$ обмежений зверху $M$. Це, в свою чергу, вірно, оскільки із доведення Теореми 1 бачимо, що $M$ залежить лише від
	сталої матриці (0, в нашому випадку) та $\int_{-\infty}^\infty\mynorm{B(s)}\;ds$, а $\int_{-\infty}^\infty\mynorm{B(s)}\;ds=
	\int_{-\infty}^\infty\mynorm{\tilde{B}(s)}\;ds.$
\end{proof}
\begin{proof}{(Спостереження \ref{Hard})}
	Ми будемо користуватися певними фактами без доведення. Для подальшої зручності, ми запишемо їх усі зараз тут.
	\begin{fact}{(\textbf{Розклад Магнуса},
		з \cite{moan})}\label{MagnusConvergenceFact}
		Нехай маємо систему $x'=Bx$ із неперервним $B$ і матриціантом $X$. Припустимо, що для $T$ виконується
		\[\int_0^T\mynorm{B(s)}\;ds<\pi\]
		тоді на $t\in[0,T)$ $X(t)=\exp(\Omega(t))$, де $\Omega(t)=\sum\limits_{k=1}^\infty\Omega_k(t)$ і $\Omega_k(t)$ є інтегралом 
		комутаторів зростаючої довжини, як-от
		\begin{align*}
		\Omega_1(t) &= \int_0^t B(t_1)\,dt_1, \\
		\Omega_2(t) &= \frac{1}{2}\int_0^t dt_1 \int_0^{t_1} dt_2\ \left[  B(t_1),B(t_2)\right], \\
		\Omega_3(t) &= \frac{1}{6} \int_0^t dt_1 \int_0^{t_1}d t_2 \int_0^{t_2} dt_3
		\Bigl(\left[B(t_1),\left[B(t_2),B(t_3)\right]\right]+\left[B(t_3),\left[  B(t_2),B(t_{1})\right]\right]\Bigr), \\
		\Omega_4(t) &= \frac{1}{12} \int_0^t dt_1 \int_0^{t_1}d t_2 \int_0^{t_2} dt_3 \int_0^{t_3} dt_4
		\Bigl(\left[\left[\left[B_1,B_2\right],B_3\right],B_4\right] \\
		&\quad+\left[B_1,\left[\left[B_2,B_3\right],B_4\right]\right]
		+\left[B_1,\left[B_2,\left[B_3,B_4\right]\right]\right]
		+\left[B_2,\left[B_3,\left[B_4,B_1\right]\right]\right]\Bigr)
		\end{align*}
	\end{fact}
	Зафіксуємо довільне $\tau\in\mathbb{R}$ і покажемо, що
	\[\forall t\in\mathbb{R},\;\mysbra{A(\tau),X_B(t)}=0\]
	(тут $\mysbra{\cdot,\cdot}$ позначає дужку Лі). Достатньо показати, що множина $t$, для яких це виконується є закритою і відкритою
	водночас (помітимо, що для $t=0$ твердження виконується, оскільки $X_B(0)=I$ комутує з усіма матрицями).
	Оскільки дужка Лі, $A$ та $B$ неперервні за гіпотезою, достатньо показати відкритість. Припустимо, таким чином, що для $t_0$ умова 
	комутації виконується і покажемо, що вона виконується на малому околі $t_0$.

	Достатньо показати, що для малих $s\in\mathbb{R}$, $A(\tau)$ комутує з $\tilde{X}(s):=X_B(t_0+s)X_B^{-1}(t_0)$. Останнє, в свою чергу,
	користуючись тією ж логікою, що і в кінці попереднього доведення, можна представити як матриціант системи $x'=\tilde{B}x$, де
	$\tilde{B}(t):=B(t+t_0)$. Таким чином, без втрати загальності можна вважати $t_0=0$. На малому околі 0, оскільки $B(t)$ неперервна,
	виконується умова ~\hyperref[MagnusConvergenceFact]{факту} і, таким чином, до матриціанту можна застосувати розклад Магнуса. 
	
	Оскільки
	кожен член $\Omega_k(t)$ є інтегралом комутаторів матриць, кожна з яких комутує з $A(\tau)$ то кожне $\Omega_k(t)$ комутує з $A(\tau)$.
	Дійсно, якщо $B,C$ комутують з $A$, то $BC$, а отже $[B,C]$ також комутують. Більше того,
	якщо $B(t)$ комутує з $A$ для всіх $t$, то $\int_a^bB(t)\;dt$ також комутує з $A$. Оскільки дужка Лі є неперервною, бачимо, що
	$\Omega(t)$ також комутує з $A(\tau)$. Таким чином, $X(t)=\exp(\Omega(t))=\sum\limits_{n=0}^\infty
	\frac{\Omega^n(t)}{n!}$ також комутує з $A(\tau)$ при малих $t$, що і завершує доведення.
	\end{proof}
\end{proof}
\section{Висновки}
Можливими подальшими напрямами автор бачить наступне:\begin{enumerate}
	\item Довести, що експоненціальна дихотомія на $\mathbb{R}$ еквівалентна тому, що для довільної неперервної обмеженої на $\mathbb{R}$
		вектор-функції $f$ система $x'=Ax+f$ має єдиний обмежений розв’язок. Подумати, чи не є зайвою умова єдиності (тобто, чи залишиться
		критерій достатнім після її вилучення).
	\item Посилити доведені твердження, вилучивши (або послабивши) гіпотезу щодо комутації.
\end{enumerate}
\begin{thebibliography}{9}
\bibitem{krein}
Ю. Л. Далецкий, М. Г. Крейн
\emph{Устойчивость решений дифференциальных уравнений в банаховом пространстве}.
Видавництво "Наука"{}, Головна редакція фізико-математичної літератури. Москва, 1970.
\bibitem{mitrop}
Митропольський Ю. А., Самойленко А. М., Кулик В. Л.
\emph{Исследование дихотомии линейных систем дифференциальных уравнений с помощью функций Ляпунова}.
АН УССР. Ін-т математики. Київ, "Наукова думка", 1990.
\bibitem{palmer84}
	Kenneth J. Palmer, {\em Exponential Dichotomies and Transversal Homoclinic Points} (1983).
\bibitem{palmer88}
	Kenneth J. Palmer, {\em Exponential Dichotomies and Fredholm Operators} (1988).
\bibitem{ju}
	Ning Ju and Stephen Wiggins, {\em On Roughness of Exponential Dichotomy} (2000).
\bibitem{naulin}
	Ra\'ul Naulin and Manuel Pinto, {\em Admissible perturbations of Exponential Dichotomy Roughness} (1997).
\bibitem{chow}
	Shui-Nee Chow and Hugo Leiva, {\em Existence and Roughness of the Exponential Dichotomy for Skew-Product Semiflow in Banach Spaces} (1994).
\bibitem{coppel}
	W. A. Coppel, {\em Dichotomies in Stability Theory} (1978).
\bibitem{moan}
	Per Christian Moan, Jitse Niesen, {\em Convergence of the Magnus series}, доступна на 
	\url{http://citeseerx.ist.psu.edu/viewdoc/summary?doi=10.1.1.63.6759}
\bibitem{demidovich}
Демидович Б. П. \emph{Лекции по математической теории устойчивости} --
Москва, 1967 г., 472 стр. с илл.
\end{thebibliography}
\end{document}
