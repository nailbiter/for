\documentclass[a4paper]{article} % use larger type; default would be 10pt

\usepackage{enumerate}
\usepackage{setspace}
\usepackage{geometry,amsmath,amssymb,bbm}

\usepackage{xeCJK}
\setCJKmainfont{IPAMincho}

\newcommand{\assign}{:=}
\newcommand{\comma}{{,}}
\newcommand{\nin}{\not\in}
\newcommand{\tmop}[1]{\ensuremath{\operatorname{#1}}}
\newcommand{\tmtextbf}[1]{{\bfseries{#1}}}
\newcommand{\tmtextit}[1]{{\itshape{#1}}}
\newenvironment{itemizedot}{\begin{itemize} \renewcommand{\labelitemi}{$\bullet$}\renewcommand{\labelitemii}{$\bullet$}\renewcommand{\labelitemiii}{$\bullet$}\renewcommand{\labelitemiv}{$\bullet$}}{\end{itemize}}
%%%%%%%%%% End TeXmacs macros

\newcommand{\D}{\mathcal{D}} \newcommand{\supp}{\tmop{supp}}
\newcommand{\proofexplanation}[1]{(#1)}
\newcommand{\C}{{\mathbbm{C}}}\newcommand{\Z}{{\mathbbm{Z}}}
\newcommand{\Sp}{{\mathbbm{S}}} \newcommand{\R}{{\mathbbm{R}}}
\newcommand{\mybra}[1]{(#1)} \newcommand{\mysbra}[1]{\left[#1\right]}
\newcommand{\mycbra}[1]{\left\{#1\right\}}

\title{Title}
\author{レオンチエフ オレックシィ}
\begin{document}

\begin{titlepage}
{\huge\begin{spacing}{1.2}
\begin{center}
	平成27年度
	\end{center}
	\vspace{2cm}
	修士論文題目
	\begin{center}
	{\Huge Study of symmetry breaking operators of indefinite orthogonal groups O(p,q)}\\
	(不定値直交群O(p,q)の対称性の破れ作用素の研究)\\
\vspace{6cm}
\begin{tabular}{ll}
学生証番号&45-146044\\
フリガナ & レオンチエフ オレクシィ\\
氏名&Leontiev Oleksii
\end{tabular}
	\end{center}
\end{spacing}
}
\end{titlepage}
%\setlength{\parindent}{1cm}
\begin{spacing}{1.5}
\begin{center} 修論内容の要旨\end{center}
\noindent 修士論文題目\\ 
Study of symmetry breaking operators of indefinite orthogonal groups O(p,q)\\
(不定値直交群O(p,q)の対称性の破れ作用素の研究)\\
氏名\quad Leontiev Oleksii
\end{spacing}\vspace{0.5cm}

$G$をLie群、$G'$を$G$の閉部分群とする。表現論においてよく研究される問題として、Gの表現とG’の表現の関係を調べるということがある。
この問題には2つの方向がある。すなわち、$G'$からGの表現を構成するという方向と、$G$の表現を$G'$の表現に分解するという方向である。\par
最初の問題について、$G'$の表現から$G$の表現を作る標準的な方法として、誘導と呼ばれるものがある。
特に、G’の自明表現を誘導すると、$G$の$L^2(G/G')$上の表現ができる。
たとえばプランシュレルの公式を導き出すなどの、この空間$L^2(G/G')$を研究する分野は調和解析と呼ばれている。
また、多くの数学者によって研究されてきたものとして、G’の自明表現の$G'\times G'$への誘導を調べるということが挙げられる
(ここで、$G'$は対角埋め込みによって$G'\times G'$の部分群と考えることができる)。この表現は、I.M.Gelfand氏とその学生によって50年代に、Harish-Chandra氏によって70年代に、大島利雄氏、P.Delorme氏、小林俊行氏によって80-90年代に研究されてきた。

その一方で、$G$の表現$\pi$の、$G'$の表現への分解を調べることは難しい問題である。
$\pi$が無限次元で、$G'$が非コンパクトという、最も一般的な設定下においては、少なくとも90年代以前には一般的な設定での本格的な研究は
されていなかった。
この問題の一番基本的な場合として、$\pi$が$G'$の表現の離散和になる場合が挙げられる。
$G'$の表現$\tau$がこの離散和に何回現れるか、という数を重複度と呼ぶが、この重複度は組み合わせ論的な技術によって計算できる。
しかし、一般的な場合に以下のような問題点がある:
\begin{enumerate}
\item 一般には$\pi$は$G'$の表現の離散和になるとは限らないので、このような計算をすることはできない;
\item 重複度の定義も変更する必要がある。一つの良い定義として、対称性の破れ次元$m(\pi,\tau)\assign\dim\tmop{Hom}_{G'}(\pi |_{G'},
 \tau)$がある;
\item この重複度も有限になるとは限らないという難しさもある (ただし、重複度が常に有限になるような$(G,G')$の組は2014年に
小林俊行氏と松木敏彦氏によって分類されている)。
\end{enumerate}

本論文に研究されたのは、$( G, G') = ( O ( p + 1, q), O ( p,q))$の場合である。
この設定は特によいとういうことが、最近証明された(小林俊行ー大島利雄、Sun-Zhu)
とれあえず、この組は、上述の分類により、重複度は有限になる。
更に最近、$O ( n + 1, 1) \supset O ( n, 1)$という特別な場合に、すべての対称性の破れ作用素が
2014、2015年に
小林俊行氏とB. Speh氏によって完全に分類された。

一方で、この場合は表現論はもちろん、数論と(Gross-Prasad予想、Rankin-Cohen微分作用素)、共形幾何学と(Juhlの共形微分作用素)
関係もある。$O ( n + 1, 1) \supset O ( n, 1)$の場合にはde-Sitter空間上の調和解析と補系列などと関係がある。もう一つ
重要な例は$O ( 2, 2) \supset O ( 2, 1)$である。この場合は$SL_2(\mathbbm{R})$の表現上の
不変三重型形式を見つける問題と同等である。

小林俊行氏とB. Speh氏によって発展された一般な手法によって、$( G, G') = ( O ( p + 1, q), O ( p,q))$の場合の
対称性の破れ作用素の研究するのは本論文の目標である。具体的には、以下の問題を考える\\

{\noindent}\tmtextbf{問\textbf{1}.}\tmtextit{与えられた $( \lambda, \nu) \in
\mathbbm{C}^2$に対して、対称性の破れ作用素の空間 $\tmop{Hom}_{G'} ( I (
\lambda), J ( \nu))$ を具体的に求めよ。特に、この空間の基底を具体的に求めよ。ここで、$I(\lambda):=G\times_P\mathbb{C}_\lambda$
と$J(\nu):=G'\times_{P'}\mathbb{C}_\nu$は$G$と$G'$の退化主系列である。}\\

{\noindent}小林俊行とBirgit Spehの結果から従う命題として、以下のものがある。

{\noindent}\textbf{命題\textbf{2}.}\tmtextit{(論文のprop. 9.4)
任意の$( \lambda, \nu) \in \mathbbm{C}^2$ に対して、ベクトル空間の同型 $\tmop{Hom}_{G'} ( I (
\lambda), J ( \nu)) \simeq \mathcal{S} \tmop{ol} ( \mathbbm{R}^{p, q} ;
\lambda, \nu)$がある。ここで、$\mathcal{S} \tmop{ol} ( \mathbbm{R}^{p, q} ; \lambda,
\nu)$ は以下の条件を満たす超関数$F \in \mathcal{D}' (
\mathbbm{R}^{p, q})$の空間である:
\begin{enumerate}
 \item $F$は斉 $\lambda-\nu-n$-次である;
 \item $F$は偶関数である;
\item $\forall m \in O ( p, q)_{e_p} \assign \{ m \in O ( p, q) | m \cdot e_p = e_p \},\quad F ( m \cdot) = F ( \cdot)$が成立する;
 \item $b, x_0 \in \mathbbm{R}^{p, q}$, $b_p = 0$ と $c_b
 ( x_0) : = 1 - 2 Q ( b, x_0) + Q ( x_0) Q ( b) \neq 0$であれば、 $x_0$の近傍上で
\begin{eqnarray}
& | c_b ( \cdot) |^{\lambda - n} F ( \psi_b ( \cdot)) = F (\cdot)
\; \mbox{が成立}& \nonumber\\
& \mbox{ここで、}\;\psi_b ( x) \assign \frac{x - Q (x) b}{c_b ( x)} . & \nonumber
\end{eqnarray}
\end{enumerate}}

{\noindent}本論文の主結果は次のように述べられる:\\

{\noindent}\tmtextbf{命題\textbf{3}.}\tmtextit{(論文の sec. 8)
$Q$を$(p,q)$-二次形式、$P,P'$を$G,G'$の最大故物型部分群とする。
\begin{enumerate}
 \item $P' \backslash G/ P$の軌道を全て具体的に分類できる (論文の
prop. 8.1);
 \item $P' \backslash G/P$の軌道の$\mathbbm{R}^{p, q} \simeq N_- \hookrightarrow G / P$ 埋め込みの
引き戻しは次のようになる:
 \[ \left\{ \begin{array}{ll}
 \{ x_p \neq 0, Q \neq 0 \} \sqcup \{ x_p \neq 0, Q = 0 \} \sqcup \{ x_p
 = 0, Q \neq 0 \} \sqcup \{ 0 \}, & p = 1\\
 \{ x_p \neq 0, Q \neq 0 \} \sqcup \{ x_p \neq 0, Q = 0 \} \sqcup \{ x_p
 = 0, Q \neq 0 \} \sqcup \{ x_p = 0, Q = 0 \} \backslash \{ 0 \} \sqcup
 \{ 0 \}, & p > 1
 \end{array} \right. \]
 ただし、$\{ x_p \neq 0, Q \neq 0 \} \assign \{ x \in \mathbbm{R}^{p, q} |
 Q ( x) \neq 0, x_p \neq 0 \}$とする;
 
 \item $P' N_- P = G$。
\end{enumerate}}

{\noindent}\tmtextbf{注\textbf{4}.}三番目の結果は同型$\tmop{Hom}_{G'} ( I ( \lambda), J ( \nu)) \simeq
\mathcal{S} \tmop{ol} ( \mathbbm{R}^{p, q} ; \lambda, \nu)$の証明に使われている。
二番めの結果から$\mathcal{S} \tmop{ol} (
\mathbbm{R}^{p, q} ; \lambda, \nu)$元の台は次のようにしかならないことが分かる: $\{ 0
\}$, $P \assign \{ x \in \mathbbm{R}^{p, q} | x_p = 0 \}$, $C \assign \{ x \in
\mathbbm{R}^{p, q} | Q ( x) = 0 \}$, $P \cap C$, $C \cup P$又は$\mathbbm{R}^{p, q}$。

以下では$\mathcal{S} \tmop{ol}_S ( \mathbbm{R}^{p,
q} ; \lambda, \nu):=\{u\in\mathcal{S} \tmop{ol} ( \mathbbm{R}^{p,
q} ; \lambda, \nu)\mid \supp(u)\subset S\}$という記法を使う。\\

{\noindent}\tmtextbf{命題\textbf{5}.}\tmtextit{次のような結果が成り立つ:
\begin{enumerate}
 \item (prop. 12.2)$\lambda, \nu \in \mathbbm{C}$は$\lambda - \nu \nin -\mathbbm{Z}_{\geqslant 0}$を満たさとする。そうならば、超関数の掛け算
 \[ \frac{| x_p |^{\lambda + \nu - n}}{\Gamma ( ( \lambda + \nu - n + 1) /
 2)} \cdot \frac{| Q |^{- \nu}}{\Gamma ( ( 1 - \nu) / 2)} \in \mathcal{D}'
 ( \mathbbm{R}^{p, q} \backslash \{ 0 \}) \]
 は$\mathcal{S} \tmop{ol} (
 \mathbbm{R}^{p, q} ; \lambda, \nu)$の元$K_{\lambda, \nu}^{\mathbbm{R}^n}$に延長できる。台も具体的に記述できる
 (論文のprop. 12.3)。
 さらに、ある稠密開部分集合$X\subset\mathbb{C}^2$が存在して、$(\lambda,\nu)\in X$ならば、台は$\mathbb{R}^{p,q}$になる;
 
 \item (prop. 14.4) $\nu \nin
 2\mathbbm{Z}_{\geqslant 0} + 1$であれば、$\mathcal{S} \tmop{ol}_{\{ 0 \}} (
 \mathbbm{R}^{p, q} ; \lambda, \nu) =\mathcal{S} \tmop{ol}_C (
 \mathbbm{R}^{p, q} ; \lambda, \nu)$である。$\nu \in
 2\mathbbm{Z}_{\geqslant 0} + 1$に対して、$\lambda \in \{ \lambda \in
 \mathbbm{C} | \lambda - \nu \in -\mathbbm{Z}_{\geqslant 0} \}$に超関数掛け算
 \[ \left\{ \begin{array}{ll}
 \delta^{( \nu - 1)} ( Q) \cdot | x_p |^{\lambda + \nu - n}, & p = 1\\
 \delta^{( \nu - 1)} ( Q) \cdot \frac{| x_p |^{\lambda + \nu -
 n}}{\Gamma ( ( \lambda + \nu - n + 1) / 2)}, & p > 1
 \end{array} \right. \]
 は$\mathcal{S} \tmop{ol}_C (
 \mathbbm{R}^{p, q} ; \lambda, \nu) \backslash\mathcal{S} \tmop{ol}_{\{ 0 \}}
 ( \mathbbm{R}^{p, q} ; \lambda, \nu)$の元$K_{\lambda, \nu}^C$に延長できる。
台も具体的に記述できる
 (論文のprop. 14.4)。
 ある稠密開部分集合$X\subset\mathbb{C}$が存在して、$\lambda\in X$ならば台、は$C$になる;
 
 \item (本論文のsec. 13) $\lambda + \nu - n = - 1 - 2 k, \; k \in
 \mathbbm{Z}_{\geqslant 0}$とする。ある離散集合$Y\subset\mathbbm{C}$が存在して、
 $\nu\nin Y$ならば
 \begin{eqnarray}
 & K^P_{\lambda, \nu} \assign \sum_{i = 0}^k \frac{(- 1)^i (2 k) !
 (\nu)_i}{(2 k - 2 i) !i!} \delta^{(2 k - 2 i)} (x_p) \otimes \tilde{Q}_i
 \in \mathcal{D}' ( \mathbbm{R}^{p, q} ; \lambda, \nu) & \nonumber\\
 & (\nu)_i \assign \nu (\nu + 1) \ldots (\nu + i - 1), & \nonumber\\
 & \tilde{Q}_i \assign \left\{ \begin{array}{ll}
 \tilde{Q}_+^{- \nu - i} + \tilde{Q}_-^{- \nu - i}, & i \in 2 \Z_{\ge 0},
 p > 2\\
 \tilde{Q}_+^{- \nu - i} - \tilde{Q}_-^{- \nu - i}, & i \in 2 \Z_{\ge 0}
 + 1, p > 2\\
 | \tilde{Q} |^{- \nu - i}, & i \in 2 \Z_{\ge 0}, p = 1\\
 - | \tilde{Q} |^{- \nu - i}, & i \in 2 \Z_{\ge 0} + 1, p = 1
 \end{array} \right. & \nonumber
 \end{eqnarray}
 ($\tilde{Q}$を $( p - 1, q)$-二次形式である)は
 $\mathcal{S} \tmop{ol}_P ( \mathbbm{R}^{p, q} ; \lambda, \nu)$の元になる。台も具体的に記述できる
 (論文の sec. 18)。
 ある稠密開部分集合$X\subset\mathbb{C}$が存在して、$\nu\in X$ならば台、は$P$になる。
\end{enumerate}}{\hspace*{\fill}}{\medskip}

以下のように、
$K_{\lambda,
\nu}^{\mathbbm{R}^n}$、$K_{\lambda, \nu}^C$ と $K_{\lambda, \nu}^P$に正規化することで、解析接続できる。

{\noindent}\tmtextbf{命題\textbf{7}.} \tmtextit{$p, q \in
 \mathbbm{Z}_{\geqslant 1}$とする。次のような結果が成り立つ:
\begin{enumerate}
 \item (sec. 17) $\nu \in 2\mathbbm{Z}_{\geqslant 0}+1$に対して、 次のような有理型関数 $N$ が存在する。$K_{\lambda, \nu}^C / N$が全ての$\lambda \in \mathbbm{C}$に解析接続できる。
 $K_{\lambda, \nu}^C / N$が$\mathcal{S} \tmop{ol}_C ( \mathbbm{R}^{p, q} ; \lambda, \nu)$ の0でないの元になり、
 台も具体的に記述できる。
 
 \item (sec. 18) $k \assign - ( \lambda + \nu - n + 1) / 2$
に対して, 次のような有理型関数 $N$ が存在する。
$K_{\lambda, \nu}^P / N$
が全ての$\nu\in \mathbbm{C}$に解析接続できる 
(ここで、$\lambda+\nu-n=-1-2k,\;k\in\mathbb{Z}_{\ge0}$の条件を用いて、$K_{\lambda, \nu}^P$は$\nu$の関数と見ている)
$K_{\lambda, \nu}^P / N$が$\mathcal{S}\tmop{ol}_P ( \mathbbm{R}^{p, q} ;
 \lambda, \nu)$の0でないの元になる。台も具体的に述べられる。
 
 \item (sec. 19) $q \in 2\mathbbm{Z}$とする。次のような有理型関数 $N$ が存在する。$K_{\lambda, \nu}^{\mathbbm{R}^n} / N$ が全ての$(\lambda,\nu) \in \mathbbm{C}^2$に解析接続できる。
$K_{\lambda, \nu}^{\mathbbm{R}^n} / N$が$\mathcal{S}\tmop{ol} ( \mathbbm{R}^{p, q} ;
 \lambda, \nu)$の元になる。
 ある離散集合$Z\subset\mathbb{C}^2$が存在して、その外で
 $K_{\lambda, \nu}^{\mathbbm{R}^n}\neq0$。
\end{enumerate}}{\hspace*{\fill}}{\medskip}
\end{document}
