\documentclass[11pt]{article} % use larger type; default would be 10pt
\usepackage[10pt]{type1ec}          % use only 10pt fonts
\usepackage[T1]{fontenc}
\usepackage{graphicx}
\usepackage{enumerate}
\usepackage{float}
\usepackage{CJKutf8}
\usepackage{subfig}
\usepackage{amsmath}
\usepackage{listings}
\usepackage{amsfonts}
\usepackage{hyperref}
\usepackage{enumitem}
\newtheorem{prob}{Problem}

\newenvironment{solution}%
{\par\textbf{Solution}\space }%
{\par}

\title{Introduction to Networks\\Homework 2}
\author{歐立思\\
9822058\\Department of Applied Mathematics}
\begin{document}
\begin{CJK}{UTF8}{bsmi}
\maketitle
\end{CJK}
\begin{prob}
	Exercise 3%           1101 1110 1010 1101 1011 1110 1110 1111
\end{prob}
\begin{solution}
According to the table on the page 82 of the textbook, the resulting 4B/5B cipher is
\[11011\:11100\:10110\:11011\:10111\:11100\:11100\:11101\]
Assuming that NRZI starts out low, we get the following NRZI signal
\begin{figure}[H]
\centering
\subfloat{\includegraphics[width=1.3\textwidth]{9822058_homemork2_table}}
\caption{Resultant NRZI}
\end{figure}
\end{solution}
\begin{prob}
	Exercise 11
\end{prob}
\begin{solution}
	Assume without loss of generality, that we are dealing with even parity scheme, that is all rows and columns of encoded frame have
	even number of bits. Now, assume that we have (at most) 3 bits corrupted. We may further assume that exactly 3 bits are corrupted,
	since one bit corruption is obviously detectable, and if the two bits are corrupted, they will lie in different rows (or different columns)
	and therefore these rows (columns) will have corrupted parity. Thus, we restrict ourself to the case when \textit{exactly} 3 bits are
	corrupted. If all of them lie in the same row, then the parity will be
	incorrect in three columns to which they belong, and error will be detected. Situation is the same if they lie in three different rows,
	in one column, or in three different columns. Therefore, the only situation of interest is when they lie exactly in 2 different rows
	and columns. That is, they lie in the vertices of rectangle with sides parallel to the sides of a frame. In this case this rectangle will
	have one vertex uncorrupted, therefore there will be one column (row) with only one bit corrupted, and this will give detectable parity
	corruption.\\
	Note, that the above reasoning about rectangle
	does not work for the case when four bits are corrupted and indeed this sort of error is undetectable by
	the two-dimensional parity scheme.
\end{solution}
\begin{prob}
	Exercise 15
\end{prob}
\begin{solution}
	First of all, notice that it is sufficient to prove the case when we have only two blocks, as other cases may be proven by induction
	based on this one. Therefore, we need to show that if
	\[[A,B]+'[C,D]=[X,Y]\]
	where $+'$ denotes ones complement addition, then
	\[[B,A]+'[D,C]=[Y,X]\]
	We shall refer to these equations as to "first addition" and "second addition".
	Let us consider few cases based on what happens in the first addition
	(in subsequent, we denote byte addition with carry disregarded simply by $+$)
	\begin{description}
		\item[No carry in $A+C$, no carry in $B+D$]{This is easy case, as in this case $X=A+C$, $Y=B+D$, and equality obviously
			holds}
		\item[Carry in $B+D$ but it didn't provoke carry in $A+C$]{In this case $Y=B+D$ and $X=A+C+1$ and we do not have carry for the
			first addition. On the other hand, for the second addition we will have carry wrapped and we similarly get $Y=B+D$ (and
			this provokes carry) and $X=A+C+1$ (add one because of carry)}
		\item[Carry in $B+D$ provoked carry in $A+C$]{Due to the carry provoked by $B+D$, $X=A+C+1$ and $Y=B+D+1$ due to the carry wrapped.
			Similar situation happens in the second addition, and equality holds}
		\item[No carry in $B+D$, but carry in $A+C$]{This will provoke wrapped carry in the first addition. Therefore, one shall be added
			to $D+B$. Now, it may happen that addition of one will provoke carry. If this happens, then $X=A+C+1$, $Y=B+D+1$ and 
			the same happens in the second addition, as $A+C$ will provoke carry in $B+D$ and final result will be the same.
			If addition of one two $B+D$ in first addition did not provoke carry, we have $X+A+C$, $Y=B+D+1$. What about the second
			addition, it will be similar to the second case considered (that is, carry in $A+C$ will not provoke carry in $B+D$)}
	\end{description}
	Since cases above exhaust all possibilities, equality is proven. Now, let us refer to the case when we have more than two addends. Let
	us proceed by induction on the number of addends. Statement shown above gives base case. That is, we want to show that if
	\[[A_1,B_1]+'\dots+'[A_n,B_n]+'[A_{n+1},B_{n+1}]=[X_{n+1},Y_{n+1}]\]
	then
	\[[B_1,A_1]+'\dots+'[B_n,A_n]+'[B_{n+1},A_{n+1}]=[Y_{n+1},X_{n+1}]\]
	Moreover, from the inductive assumption
	\[[A_1,B_1]+'\dots+'[A_n,B_n]=[X_{n},Y_{n}]\]
	\[[B_1,A_1]+'\dots+'[B_n,A_n]=[Y_{n},X_{n}]\]
	Therefore,
	\[[A_1,B_1]+'\dots+'[A_n,B_n]+'[A_{n+1},B_{n+1}]=[X_n,Y_n]+'[A_{n+1},B_{n+1}]\]
	\[[B_1,A_1]+'\dots+'[B_n,A_n]+'[B_{n+1},A_{n+1}]=[Y_{n},X_{n}]+'[B_{n+1},A_{n+1}]\]
	And from the above statement, if
	\[[X_n,Y_n]+'[A_{n+1},B_{n+1}]=[X_{n+1},Y_{n+1}]\]
	then
	\[[Y_n,X_n]+'[B_{n+1},A_{n+1}]=[Y_{n+1},X_{n+1}]\]
\end{solution}
\begin{prob}
	Exercise 19
\end{prob}
\begin{solution}
	\begin{enumerate}[label=(\alph*)]
		\item{The polynomial that should be transmitted is
			\[x^{23}+x^{21}+x^{20}+x^{17}+x^{14}+x^{11}+x^9+x^8+x^7+x^4+x+1\]
			which gives message 101100100100101110010011 to transmit
			}
		\item{The receiver will get 001100100100101110010011 which corresponds to 
			$x^{21}+x^{20}+x^{17}+x^{14}+x^{11}+x^9+x^8+x^7+x^4+x+1$. After the division modulo two
			on $x^8+x^2+x+1$ the remainder will be $x^7+x^5+x^4+x^2+x\neq 0$} Thus, receiver will be notified about the error.
	\end{enumerate}
\end{solution}
\begin{prob}
	Exercise 22
\end{prob}
\begin{solution}
	If only NAKs would be used, then we would have to implement the following timeouts:
	\begin{itemize}
		\item{Timeout for the receiver. Receiver should wait specified period of time for a next frame, after that it sends
			NAK}
		\item{Timeout for the sender. If sender does not receive NAK during some period after the frame was sent, it assumes that the 
			frame was delivered succesfully}
	\end{itemize}
	Using NAKs only has many disadvantages. To begin with, sender should wait the full timeout whether the frame was send successfully (which
	happens often on a good channel). In addition, if the NAK was lost, the frame will also be lost. Finally, if NAK comes too late previous
	frame will be lost.
\end{solution}
\begin{prob}
	Exercise 31
\end{prob}
\begin{solution}
	\begin{enumerate}[label=(\alph*)]
		\item{See the corresponding figure
\begin{figure}[H]
\centering
\subfloat{\includegraphics[width=1.3\textwidth]{net3_p31_a}}
\caption{Timeline when frame 4 is lost}
\end{figure}
			}
		\item{See the corresponding figure
\begin{figure}[H]
\centering
\subfloat{\includegraphics[width=1.3\textwidth]{net3_p31_b}}
\caption{Timeline when frame 4 is lost}
\end{figure}
			}
	\end{enumerate}
\end{solution}
\begin{prob}
	Exercise 37
\end{prob}
\begin{solution}
We assume that RWS=4 as well.
\begin{enumerate}[label=(\alph*)]
	\item{
\begin{tabular}{|c||c|c||c|c|}
	Time & Sent from A & Arrive to A & Sent from B & Arrive to B\\
	\hline
	0 & 1,2,3,4 & & & \\
	1 & & & ACK1& 1\\
	2 & 5&ACK1 & ACK2& 2\\
	3 & 6 & ACK2 & ACK3 & 3\\
	4 & 7 & ACK3 & ACK4 & 4\\
	5 & 8 & ACK4 & ACK5 & 5\\
	\hline
\end{tabular}\\
The queue at R grows to the size 4.
		}
\end{enumerate}
\end{solution}
\begin{prob}
	Exercise 43
\end{prob}
\begin{solution}
	\begin{enumerate}[label=(\alph*)]
		\item{The combinations of backs off that will allow are A to win are $(0\times T,1\times T),
(0\times T,2\times T),(0\times T,3\times T),(1\times T,2\times T),(1\times T,3\times T)
			$, thus probability is 0.675
			}
		\item{By the reasoning similar to above, the probability is 0.81
			}
		\item{In general, when B backs off for time equal to $0\times T,\dots (2^n-1)\times T$, while A backs of for $0, T$, A's probability
			to win is \[\frac{2^n-1+2^n-2}{2^{n+1}}=1-\frac{3}{2^{n+1}}\]
			Therefore, A's probability to win all remaining races is
			\[\prod_{n=4}^\infty (1-\frac{3}{2^{n+1}})=\exp(\sum_{n=4}^\infty \ln(1-\frac{3}{2^{n+1}})\]
			Now, since for $x$ such that $0<x<1/2$ we have $-3x/2<\ln(1-x)$ we get
			\[\exp(\sum_{n=4}^\infty \ln(1-\frac{3}{2^{n+1}})\geq\exp(\sum_{n=4}^\infty -9/2^{n+2})=\exp(-18(1-1/64))>2\times 10^{-8}\]
			}
		\item{Frame $B_1$ is simply never sent}
	\end{enumerate}
\end{solution}
\begin{prob}
	Exercise 51
\end{prob}
\begin{solution}
	Bandwidth will be (we count packet's size as a unit)
	\[\frac{1}{5+N/2}=\frac{2}{N+10}\]
\end{solution}
\begin{prob}
	Exercise 54
\end{prob}
\begin{solution}
Let me outline a few reasons, as on the page 136 of a textbook
\begin{itemize}
	\item{Node cannot send and receive at the same time, because the signal emitted will be strong and will blind receiving 
		circuitry
		}
	\item{The pair of nodes may (and often will be) divided by an obstacle, or simply be too far away and therefore will be unable
		to know that they transmit at the same time}
\end{itemize}
\end{solution}
\end{document}
%Why is collision detection more complex in wireless networks
%than in wired networks such as Ethernet?
