\documentclass[12pt]{article}

\usepackage{amsmath}
\usepackage{amsfonts}

\newenvironment{proof}[1][Proof]{\begin{trivlist}
\item[\hskip \labelsep {\bfseries #1}]}{\end{trivlist}}

\newcommand{\qed}{\nobreak \ifvmode \relax \else
      \ifdim\lastskip<1.5em \hskip-\lastskip
      \hskip1.5em plus0em minus0.5em \fi \nobreak
      \vrule height0.75em width0.5em depth0.25em\fi}

\oddsidemargin 0 truemm \evensidemargin 0 truemm \marginparsep 0pt
\topmargin -50pt \textheight 240 truemm \textwidth 160 truemm
\parindent 0em \parskip 1ex
\newcommand{\ds}{\displaystyle}
\newcommand{\rto}{\rightarrow}
\newcommand{\defeq}{\stackrel{\tiny{\mathbf{def}}}{=\!\!=}}

\begin{document}           % End of preamble and beginning of text.
\pagestyle{myheadings} \thispagestyle{empty} \markright{}

\begin{center}
{\bf THE CHINESE UNIVERSITY OF HONG KONG}\\
{\bf Department of Mathematics}\\
{\bf Course Code: MATH1010} \\
{\bf Liu Beibei} \\
\end{center}

Function $f$ satisfies functional equation $f(x+y)=f(x)+f(y)\quad(\forall x,y\in\mathbb{R})$, and $f$ is continuous at $x=0$, then there is only one
solution $f(x)=ax$ satisfying the equation ($a$ is a constant).
\begin{proof}From the functional equation, we have $f(nx)=nf(x)\quad(\forall n\in\mathbb{N})$. Replace $x$ with $\frac{x}{n}$ in this formula,
	then we get $f(x)=nf\left(\frac{x}{n}\right)$,
	$$f\left(\frac{1}{n}x\right)=\frac{1}{n}f(x)\quad(\forall n\in\mathbb{N})$$
	Repeat the procedure above, we can obtain
	$$f\left(\frac{m}{n}x\right)=\frac{m}{n}f(x)\quad(\forall n\in\mathbb{N})$$
	Since $f(x)=f(0+x)=f(0)+f(x)$, $f(0)=0$. Therefore, $$f(x)+f(-x)=f(0)=0$$
	Which means $f(x)=-f(-x)$. When $c$ is a rational number, we can obtain $$f(cx)=cf(x)$$
	Any irrational number is the limit of a sequence of rational numbers. Therefore for any irrational number $c$, there exists a sequence
	of rational numbers $\left\{c_n\right\}$ so that $c_n\to c$. Then
	$$f(cx)-c_nf(x)=f(cx)-f(c_nx)=f(cx-c_nx)$$\[=f[(c-c_n)x]\]
	When $n\to\infty$, since $f$ is continuous at $x=0$, $f[(c-c_n)x]\to f(0)=0$, which implies $f(cx)=cf(x)\quad(\forall c\in\mathbb{R})$. Then
	$f(x)=f(x\cdot 1)=xf(1)$.
\qed\end{proof}

	{\bf Note:} This question is chosen from the book “typical problems and methods in mathematical
	analysis” written by Liwen Pei.
\end{document}
