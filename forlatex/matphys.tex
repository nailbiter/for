\documentclass[12pt]{article} % use larger type; default would be 10pt

\usepackage{mathtext}                 % підключення кирилиці у математичних формулах
                                          % (mathtext.sty входить в пакет t2).
\usepackage[T1,T2A]{fontenc}         % внутрішнє кодування шрифтів (може бути декілька);
                                          % вказане останнім діє по замовчуванню;
                                          % кириличне має співпадати з заданим в ukrhyph.tex.
\usepackage[utf8]{inputenc}       % кодування документа; замість cp866nav
                                          % може бути cp1251, koi8-u, macukr, iso88595, utf8.
\usepackage[english,russian,ukrainian]{babel} % національна локалізація; може бути декілька
                                          % мов; остання з переліку діє по замовчуванню. 
\usepackage{mystyle}

\newtheorem{prob}{Завдання}
\newcommand{\ds}{\;ds}
\newcommand{\dt}{\;dt}
\newcommand{\dx}{\;dx}
\newcommand{\dta}{\;d\tau}
\newcommand{\extr}{\mbox{\normalfont extr}}

\newtheorem{myulem}[mythm]{Лема}

\renewenvironment{myproof}[1][Доведення]{\begin{trivlist}
\item[\hskip \labelsep {\bfseries #1}]}{\myqed\end{trivlist}}

\title{Рівняння математичної фізики (9 семестр)}
\author{Олексій Леонтьєв}

\begin{document}
\def\dx{\Delta x}
\def\dt{\Delta t}
\def\dX{\frac{\partial}{\partial x}}
\maketitle
\begin{prob}Поставити крайову задачу про поздовжні коливання пружного однорідного стержня довжини $l$, якщо починаючи з моменту
$t=0$ стрижень відчуває дію направленої вздовж осі $x$ сили об’ємної густини $R(x,t)$, лівий кінець рухається за заданим законом,
до правого кінця приєднана зосереджена маса $m$ і до нього прикладена повздовжна сила $\Phi(t)$. Початкові відхилення перерізів стрижня та
швидкості їх руху відсутні.\end{prob}
Традиційно, ми позначимо густину стержня за $\rho$, площу перерізу за $S$, а модуль Юнга -- за $E$. Закон, за яким рухається лівий кінець стержня,
ми позначимо як $\alpha(t)$. Це відразу дає нам можливість крайову умову для лівого кінця:
\[u\big|_{x=0}=\alpha(t)\]
Щодо правого кінця, запишемо принцип Даламбера для елемента стрижня $[l-\Delta x,l]$. Нагадаємо, що він, за умовою, має масу $m+\rho S\Delta x$
\[(\rho S\Delta x+m) u_{tt}(l-\Delta x+\theta_1\Delta x,t)=-ESu_x(l-\Delta x,t)+R(l-\Delta x+\Delta x\theta_3,t)S\Delta x+\Phi(t)\]
і при $\Delta x\to0$ це перетворюється просто на
\[m u_{tt}(l,t)=\Phi(t)-ESu_x(l,t)\]
Таким чином, остаточно задача записується як
\[\begin{cases}
u\big|_{x=0}=\alpha(t)\\
(m u_{tt}+ESu_x)\big|_{x=l}=\Phi(t)\\
\rho u_{tt}=Eu_{xx}+R(x,t)
\end{cases}\]
\begin{prob}Поставити крайову задачу про визначення температури однорідного стрижня $0\leq x\leq l$, вздовж якого рівномірно розподілені теплові
	джерела об’ємної густини $F(x,t)$, якщо його початкова температура є довільною функцією $x$; розглянути випадок, коли на кінці стрижня
	$x=0$ зосереджена маса $m$ з того ж матеріалу, що і стрижень, і на ньому відбувається конвективний теплообмін з навколишнім середовищем 
	температури $u_2$, на кінець $x=l$ подається ззовні тепловий потік $Q(t)$. На бічній поверхні відбувається конвективний теплообмін з 
	навколишнім середовищем нульової температури.
\end{prob}
Позначимо за $\rho$, $S$, $p$, $c$ і $k$ відповідно густину, площу поздовжнього перерізу, периметр поздовжнього перерізу, теплоємність і 
теплопровідність стержня. Ми почнемо з того, що запишемо рівняння теплообміну для ділянки $[x,x+\dx]$ \textit{всередині} стрижня.
Рівняння, дане в конспекті
\[\dX\mybra{k u_x}S\dx\dt+FS\dx\dt=c\rho Su_t\dx\dt\]
має бути доповнене ще одним членом в лівій частині
, щоб врахувати теплообмін через бічну поверхню. Згідно із законом Ньютона-Ріхмана, цей член буде рівним
\[dQ_4=h[0-u(x,t)]p\dx\dt\]
де $p\dx$ -- площа бічної поверхні циліндра, який являє собою бічна поверхня розглядаємої ділянки, а $h$ -- певний коефіцієнт. Таким чином, маємо
\[\dX\mybra{k u_x}S\dx\dt+FS\dx\dt-hup\dx\dt=c\rho Su_t\dx\dt\]
\[ku_{xx}S+FS-hup=c\rho Su_t\]
Залишається записати граничні умови. Щодо правого кінця, то як показано в конспекті, гранична умова пишеться як
\[u_x\big|_{x=l}=\frac{Q(t)}{kS}\]
Щодо лівого кінця, використовуючи Закон Ньютона-Ріхмана, маємо
\[h(u_2-u(0,t))S+FS\dx\dt+ku_xS\dt=(m+\rho S\dx)cu_t(0,t)\]
\[h(u_2-u(0,t))S+ku_xS=mcu_t(0,t)\]
і таким чином, задача матиме вигляд
\[\begin{cases}
ku_{xx}S+FS-hup=c\rho Su_t\\
u_x\big|_{x=l}=\frac{Q(t)}{kS}\\
(h(u_2-u)S-mcu_t+ku_xS)\big|_{x=0}=0
\end{cases}\]
\end{document}
