\documentclass[8pt]{article} % use larger type; default would be 10pt

\usepackage[margin=1in]{geometry}
\usepackage{graphicx}
\usepackage{float}
\usepackage{subfig}
\usepackage{amsmath}
\usepackage{amsfonts}
\usepackage{hyperref}
\usepackage{enumerate}
\usepackage{enumitem}

\usepackage{mystyle}
\newcommand{\dx}{\;dx}
\newcommand{\dy}{\;dy}
\newcommand{\dz}{\;dz}
\newcommand{\du}{\;du}
\newcommand{\dv}{\;dv}
\newcommand{\dw}{\;dw}
\newcommand{\dr}{\;dr}
\newcommand{\dph}{\;d\phi}
\newcommand{\drh}{\;d\rho}
\newcommand{\dt}{\;d\theta}

\title{Math 1540\\University Mathematics for Financial Studies\\2013-14 Term 1\\Suggested solutions for\\
Sec. 15.4--15.8}
\begin{document}
\maketitle
\section{Section 15.4}
\begin{description}
	\item[\# 22.]{{\it Change the Cartesian integral into an equivalent polar integral. Then evaluate the integral.}
		\[\int_1^2\int_0^{\sqrt{2x-x^2}}\frac{1}{(x^2+y^2)^2}\dx\dy\]
		Note, that the region of integration is the "triangle" with vertices $(1,0)$, $(2,0)$ and $(1,1)$. In polar 
		coordinates, they have their respective coordinates $(\theta,r)$ equal to $(0,1)$, $(0,2)$ and $(\pi/4,\sqrt{2})$
		respectively. The curves joining $(1,1)$ and $(2,0)$ has Cartesian equation $y=\sqrt{2x-x^2}$ and thus in
		polar coordinates can be written as $r=2\cos\theta$, while line joining $(1,0)$ and $(1,1)$ is
		$x=1$ in Cartesian coordinates, hence $r\cos\theta=1$ in polar. This allows us to rewrite integral as
		\[\int_1^2\int_0^{\sqrt{2x-x^2}}\frac{1}{(x^2+y^2)^2}\dx\dy=\int_0^{\pi/4}\int_{1/\cos\theta}^{2/\cos\theta}\frac{1}
		{r^4}\dr\dt=\int_0^{\pi/4}=\int_0^{\pi/4}\frac{r^{-3}}{-4}\bigg|_{1/\cos\theta}^{2/\cos\theta}\dt=\]
		\[=\frac{7}{32}\int_0^{\pi/4}\cos^3\theta\dt=\frac{7}{32}\int_0^{1/\sqrt{2}}(1-\sin^2\theta)\;d(\sin\theta)=
		\frac{7}{32}\mybra{\frac{1}{\sqrt{2}}-\frac{1}{6\sqrt{2}}}=\frac{35}{192\sqrt{2}}\]
		}
	\item[\# 25.]{{\it Sketch the region of integration and convert each polar integral or sum of integrals to a Cartesian
		integral or sum of integrals.}
		\[\int_0^{\pi/4}\int_0^{2\sec\theta}r^5\sin^2\theta\dr\dt\]
		The region looks like\\%TODO
		Hence, in Cartesian coordinates integral becomes
		\[\int_0^{\pi/4}\int_0^{2\sec\theta}r^5\sin^2\theta\dr\dt=\int_0^2\int_0^xy^2\mybra{x^2+y^2}^{\frac{3}{2}}\dy\dx\]
		}
	\item[\# 30.]{{\it Find the area of the region enclosed by the positive $x$-axis and spiral $r=4\theta/3$, $0\leq\theta\leq
		2\pi$. The region looks like a snail shell.}\\
		To begin with, the region looks like\\%TODO
		Second, using the formula for area in polar coordinates from the textbook, we get
		\[A=\int_0^{2\pi}\int_0^{4\theta/3}r\dr=\int_0^{2\pi}\frac{8\theta^2}{9}\dt=\frac{64\pi^3}{27}\]
		}
\section{Section 15.5}
	\item[\# 22.]{{\it Rewrite the integral as an equivalent iterated integral}
		Let us just write answers together with brief comments. It is important to note, that while this sort of problems
		can be rather tricky, it is important to first try to {\it imagine} how the region looks like (the best is to have
		a picture, but solid understanding should also do).%TODO: more explanation
		\begin{enumerate}[label=(\bfseries\alph*)]
			\item $\int_0^1\int_0^1\int_{\sqrt{z}}^1\dy\dz\dx$
			\item $\int_0^1\int_0^1\int_{\sqrt{z}}^1\dy\dx\dz$
			\item $\int_0^1\int_{\sqrt{z}}^1\int_0^1\dx\dy\dz$
			\item $\int_0^1\int_0^{y^2}\int_0^1\dx\dz\dy$
			\item $\int_0^1\int_0^1\int_0^{y^2}\dz\dx\dy$
		\end{enumerate}
		}
	\item[\# 29.]{{\it Find the volume of the region common to the interiors of the cylinders $x^2+y^2=1$ and $x^2+z^2=1$.\\}
		We will use the picture given in textbook to find a volume of solid it describes. As it is only one-eighth of what
		we need, we'll multiply the answer by 8. We will integrate in Cartesian coordinates, first with respect to $x$,
		then $y$ and then $z$. This order is appropriate in the sense, that for each cylinder intersection with plane
		$x=a$ is a in infinite strip, parallel to the axis of cylinder. The intersection of such strips will be a rectangle,
		and it is easy to find the area of a rectangle. So,
		\[V=8\int_0^1\int_0^{\sqrt{1-x^2}}\int_0^{\sqrt{1-x^2}}\dz\dy\dx=8\int_0^1(1-x^2)\dx=\frac{16}{3}\]
		}
	\item[\# 44.]{{\it Evaluate the integral by changing order of integration in an appropriate way.}
		\[\int_0^2\int_0^{4-x^2}\int_0^x\frac{\sin 2z}{4-z}\dy\dz\dx\]
		It is reasonable to try to move $\dz$ to the outmost layer, so we can do two inner integrals effortlessly (as 
		integrable expression is independent on $x$ and $y$).
		Simple interchange of two variable integral gives us
		\[\int_0^2\int_0^{4-x^2}\int_0^x\frac{\sin 2z}{4-z}\dy\dz\dx=
		\int_0^4\int_0^{\sqrt{4-z}}\int_0^x\frac{\sin 2z}{4-z}\dy\dx\dz=\int_0^4\int_0^{4-z}\frac{\sin 2z}{2(4-z)}\;d(x^2)\dz
		=\]\[=\int_0^4\frac{\sin 2z}{2}\dz=\frac{1}{4}(1-\cos8)\]
		}
\section{Section 15.7}
	\item[\# 34.]{{\it Find the spherical coordinate limits for integral that calculates the volume of a given solid, and then
		evaluate the integral. The solid given is bounded below by the hemisphere $\rho=1,\;z\geq0$, and above by the
		cardioid of revolution $\rho=1+\cos\phi$.}\\
		Let us proceed sequentially.\\
		{\it The $\rho$-limits of integration.} $\rho$ can go from $1$ to $1+\cos\phi$.\\
		{\it The $\phi$-limits of integration.} $\phi$ can go from $0$ to $\pi/2$, as solid occupies upper half-space.\\
		{\it The $\theta$-limits of integration.} $\theta$ can be anything from $0$ to $2\pi$.\\
		Altogether, this allows us to set up and evaluate the integral
		\[V=\int_0^{2\pi}\int_0^{\pi/2}\int_1^{1+\cos\phi}\rho^2\sin\phi\drh\dph\dt=2\pi\int_0^{\pi/2}\sin\phi\frac
		{\cos^3\phi+3\cos^2\phi+3\cos\phi}{3}\dph=\]
		\[=\frac{2\pi}{3}\int_0^1(\cos^3\phi+3\cos^2\phi+3\cos\phi)\;d(\cos\phi)=\frac{2\pi}{3}\mybra{\frac{1}{3}+1+\frac{3}
		{2}}=\frac{17\pi}{9}\]
		}
	\item[\# 47.]{{\it Find the volume of the solid.\\}
		Cylindrical coordinates are the most natural for this problem, as one of the curves is given in polar coordinates,
		other involves $x^2+y^2$ term. Thus, we can set up the integral
		\[V=\int_0^{\pi/2}\int_0^{\sin\theta}\int_0^{\sqrt{1-r^2}}r\dz\dr\dt=\frac{1}{2}\int_0^{\pi/2}\int_0^{\sin^2\theta}
		\sqrt{1-r^2}\;d(r^2)\dt=\frac{1}{2}\int_0^{\pi/2}\int_{\cos^2\theta}^1\sqrt t\;dt\dt=\]
		\[=\frac{1}{3}\int_0^{\pi/2}1-\cos^3\theta\dt=\frac{1}{3}\mybra{\frac{\pi}{2}-\frac{2}{3}}\]
		}
	\item[\# 58.]{{\it Find the volume of the region enclosed by the cylinder $x^2+y^2=4$ and the planes $z=0$ and $x+y+z=4$.}\\
		The cylindrical coordinates are the most natural here. Before proceeding, however, note that planes $z=0$
		and $x+y+z=4$ intersect along the line $x+y=4$, and this line lies completely outside circle $x^2+y^2\leq 4$ (as
		$x^2+y^2\leq 4\implies x+y\leq\myabs{x}+\myabs{y}=\sqrt{x^2+y^2+2\myabs{x}\myabs{y}}\leq \sqrt{2x^2+2y^2}<2\sqrt{2}<4$
		). Hence, above $x^2+y^2\leq 4$ we have $4-x-y\geq 0$, hence we can set up the integral
		\[V=\int_0^{2\pi}\int_0^2\int_0^{4-\rho\cos\theta-\rho\sin\theta}\dz\drh\dt=\int_0^{2\pi}16-8\cos\theta-8\sin\theta\dt
		=32\pi\]
		}
\section{Section 15.8}
	\item[\# 8.]{{\it Use the transformation and parallelogram $R$ in Exercise 4 to evaluate the integral}
		\[\iint\limits_R2(x-y)\dx\dy\]
		Following the suggestion in the textbook, we employ the substitution
		\[u=2x-3y,\;v=-x+y\iff x=-u-3v,\;y=-u-2v\]
		Which has Jacobian
		\[\frac{\partial(x,y)}{\partial(u,v)}=\left|\begin{array}{ll}-1&-3\\-1&-2\end{array}\right|=-1\]
		and transforms $R$ to the parallelogram $R'$, bounded by $v=0$, $v=1$, $3v+u=3$ and $3v+u=0$. Thus, we have
		\[\iint\limits_R2(x-y)\dx\dy=\iint\limits_{R'}-2v\du\dv=\int_0^1\int_{-3v}^{3-3v}-2v\du\dv=\int_0^1-6v\dv=-3\]
		}
	\item[\# 16.]{{\it Use the transformation $x=u^2-v^2$, $y=2uv$ to evaluate the integral}
		\[\int_0^1\int_0^{2\sqrt{1-x}}\sqrt{x^2+y^2}\dy\dx\]
		As suggested by the hint, we shall employ substitution $x=u^2-v^2$, $y=2uv$. To begin with, note that
		\[x^2+y^2=u^4-2u^2v^2+v^4+4u^2v^2=(u^2+v^2)^2\implies \sqrt{x^2+y^2}=u^2+v^2\]
		and thus
		\[u^2=\frac{\sqrt{x^2+y^2}+x}{2},\;v^2=\frac{\sqrt{x^2+y^2}-x}{2}\]
		Now, as on the region of integration we have $0\geq y=2uv$, we may assume $u,v\geq0$ and hence the region of 
		integration, being bounded by $y=0$, $x=0$ and $y=2\sqrt{1-x}$ in $xy$-plane, becomes bounded by
		$v=0$, $u=v$ and $u=1$. As Jacobian is
		\[\frac{\partial(x,y)}{\partial(u,v)}=\left|\begin{array}{ll}
			2u&-2v\\2v&2u
		\end{array}\right|=4u^2+4v^2\]
		Thus integral becomes
		\[\int_0^1\int_0^{2\sqrt{1-x}}\sqrt{x^2+y^2}\dy\dx=\int_0^1\int_0^u4(u^2+v^2)^2\dv\du=\int_0^14u^4+\frac{52}{15}u^5\du
		=\frac{4}{5}+\frac{52}{90}=\frac{58}{45}\]
		}
	\item[\# 24.]{{\it Led $D$ be the region in $xyz$-space defined by the inequalities
		\[1\leq x\leq2,\;0\leq xy\leq2,\;0\leq z\leq1.\]
		Evaluate
		\[\iiint\limits_D(x^2y+3xyz)\dx\dy\dz\]
		by applying the transformation
		\[u=x,\;v=xy,\;w=3z\]
		and integrating over an appropriate region $G$ in $uvw$-space.\\}
		To begin with, in new coordinates the region is given by
		\[1\leq u\leq2,\;0\leq v\leq 2,\;0\leq w\leq 3\]
		Next, Jacobian is given by
		\[\frac{\partial(x,y,z)}{\partial(u,v,w)}=\left|\begin{array}{lll}
			1&0&0\\-\frac{1}{u^2}&\frac{1}{u}&0\\0&0&\frac{1}{3}
		\end{array}\right|=\frac{1}{3u}\]
		Hence, integral becomes
		\[\iiint\limits_D(x^2y+3xyz)\dx\dy\dz=\int_1^2\int_0^2\int_0^3(uv+vw)\frac{1}{3u}\dw\dv\du=\]
		\[=2+2\cdot\frac{3}{2}\ln2=2+3\ln2\]
		}
\end{description}
\end{document}
%Sec 15.4: #22, 25, 30
%Sec 15.5: #22, 29, 44
%Sec 15.7: #34, 47, 58
%Sec 15.8: #8, 16, 24
