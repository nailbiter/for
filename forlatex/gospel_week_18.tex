\documentclass[10pt]{article} % use larger type; default would be 10pt

\usepackage{CJKutf8}
\usepackage[margin=0.7in]{geometry}
\usepackage[russian,english]{babel} % національна локалізація; може бути декілька
\usepackage{setspace}
\usepackage{mdframed}
\usepackage[utf8]{inputenc}       % кодування документа; замість cp866nav

\title{Divine Liturgy\\Reading from The Gospels}
\date{Week 18 after the Pentecost\vspace{-5ex}}
\begin{document}
\pagenumbering{gobble}
\begin{otherlanguage*}{russian}
\maketitle
\end{otherlanguage*}
\large\singlespacing
\framebox[\textwidth]{
\begin{minipage}[t]{0.45\textwidth}
\begin{otherlanguage*}{russian}
\textbf{Лк., 35 зач., VIII, 5-15.}\\
вышел сеятель сеять семя свое, и когда он сеял, иное упало при дороге и было потоптано, и птицы небесные поклевали его;\\
а иное упало на камень и, взойдя, засохло, потому что не имело влаги;\\
а иное упало между тернием, и выросло терние и заглушило его;\\
а иное упало на добрую землю и, взойдя, принесло плод сторичный. Сказав сие, возгласил: кто имеет уши слышать, да слышит!\\
Ученики же Его спросили у Него: что бы значила притча сия?\\
Он сказал: вам дано знать тайны Царствия Божия, а прочим в притчах, так что они видя не видят и слыша не разумеют.\\
Вот что значит притча сия: семя есть слово Божие;\\
 упавшее при пути, это суть слушающие, к которым пото́м приходит диавол и уносит слово из сердца их, чтобы они не уверовали и не спаслись;\\
а упавшее на камень, это те, которые, когда услышат слово, с радостью принимают, но которые не имеют корня, и временем веруют, а во время искушения отпадают;\\
а упавшее в терние, это те, которые слушают слово, но, отходя, заботами, богатством и наслаждениями житейскими подавляются и не приносят плода;\\
а упавшее на добрую землю, это те, которые, услышав слово, хранят его в добром и чистом сердце и приносят плод в терпении. Сказав это, Он возгласил: кто имеет уши слышать, да слышит!
\end{otherlanguage*}
\end{minipage}
\hfill
\begin{minipage}[t]{0.45\textwidth}
\textbf{Luke 8:5 -- 8:15.}\\
A sower went out to sow his seed: and as he sowed, some fell by the way side; and it was trodden down, and the fowls of the air devoured it.\\
And some fell upon a rock; and as soon as it was sprung up, it withered away, because it lacked moisture.\\
And some fell among thorns; and the thorns sprang up with it, and choked it.\\
And other fell on good ground, and sprang up, and bare fruit an hundredfold. And when he had said these things, he cried, He that hath ears to hear, let him hear.\\
And his disciples asked him, saying, What might this parable be?\\
And he said, Unto you it is given to know the mysteries of the kingdom of God: but to others in parables; that seeing they might not see, and hearing they might not understand.\\
Now the parable is this: The seed is the word of God.\\
Those by the way side are they that hear; then cometh the devil, and taketh away the word out of their hearts, lest they should believe and be saved.\\
They on the rock are they, which, when they hear, receive the word with joy; and these have no root, which for a while believe, and in time of temptation fall away.\\
And that which fell among thorns are they, which, when they have heard, go forth, and are choked with cares and riches and pleasures of this life, and bring no fruit to perfection.\\
But that on the good ground are they, which in an honest and good heart, having heard the word, keep it, and bring forth fruit with patience.\\
\end{minipage}}
\newpage\huge
\singlespacing
\framebox[\textwidth]{
\begin{minipage}[t]{\textwidth}
\begin{CJK}{UTF8}{bsmi}
\textbf{路加福音 8:5 -- 8:15.}\\
「有一個撒種的出去撒種。他撒的時候,有的落在路旁,被人踐踏,天上的飛鳥又來把它吃掉了。\\
有的落在磐石上,一出來就枯乾了,因為得不着滋潤。\\
有的落在荊棘裏,荊棘跟它一同生長,把它擠住了。\\
又有的落在好土裏,生長起來,結實百倍。」耶穌說完這些話,大聲說:「有耳可聽的,就應當聽!」\\
門徒問耶穌這比喻是甚麼意思。\\
他說:「上帝國的奧祕只讓你們知道,至於別人,就用比喻,要\\
他們看也看不見,\\
聽也不明白。」\\
「這比喻是這樣的:種子就是上帝的道。\\
那些在路旁的,就是人聽了道,隨後魔鬼來,從他們心裏把道奪去,以免他們信了得救。\\
那些在磐石上的,就是人聽道,歡喜領受,但沒有根,不過暫時相信,等到碰上試煉就退後了。\\
那落在荊棘裏的,就是人聽了道,走開以後,被今生的憂慮、錢財、宴樂擠住了,結不出成熟的子粒來。\\
那落在好土裏的,就是人聽了道,並用純真善良的心持守它,耐心等候結果實。」 \\
\end{CJK}
\end{minipage}}
\end{document}
