\documentclass[12pt]{article} % use larger type; default would be 10pt

\usepackage{mathtext}                 % підключення кирилиці у математичних формулах
                                          % (mathtext.sty входить в пакет t2).
\usepackage[T1,T2A]{fontenc}         % внутрішнє кодування шрифтів (може бути декілька);
                                          % вказане останнім діє по замовчуванню;
                                          % кириличне має співпадати з заданим в ukrhyph.tex.
\usepackage[utf8]{inputenc}       % кодування документа; замість cp866nav
                                          % може бути cp1251, koi8-u, macukr, iso88595, utf8.
\usepackage[english,russian,ukrainian]{babel} % національна локалізація; може бути декілька
                                          % мов; остання з переліку діє по замовчуванню. 
\usepackage{mystyle}
\usepackage{gensymb}

\newtheorem{prob}{Завдання}
\newcommand{\ds}{\;ds}
\newcommand{\dt}{\;dt}
\newcommand{\dx}{\;dx}
\newcommand{\dy}{\;dy}
\newcommand{\dta}{\;d\tau}
\newcommand{\extr}{\mbox{\normalfont extr}}

\newtheorem{myulem}[mythm]{Лема}

\renewenvironment{myproof}[1][Доведення]{\begin{trivlist}
\item[\hskip \labelsep {\bfseries #1}]}{\myqed\end{trivlist}}

\newcommand{\px}[1]{\mybra{\vec{#1}}_x}
\newcommand{\py}[1]{\mybra{\vec{#1}}_y}

\title{Теоретична механіка (10 семестр)}
\author{Олексій Леонтьєв}

\begin{document}
\maketitle
\begin{prob}Задача 2.29\end{prob}
	\mypic{0.8}{teormech_2_29.png}
По-перше, знайдемо силу опору в точці $D$, яку ми позначимо $R_D$. Записуючи рівність моментів сил для балки $AB$ відносно
вісі, що перпендикулярна площині рисунка і проходить через $A$, маємо
\[R_D\sqrt{2}=5\cdot(2+1)\implies R_D=10.6066\mbox{кН}\]
Далі, сила опору $R_A$ має врівноважувати суму сил, що діють на балку $AB$, тобто
\[\vec{R_A}=\vec{F}-\vec{R_D}\]
\[\mybra{\vec{R_A}}_x=R_D/\sqrt{2}=7.5\mbox{кН}\]
\[\mybra{\vec{R_A}}_y=R_D/\sqrt{2}-F=2.5\mbox{кН}\]
і таким чином,
\[R_A=\sqrt{\mybra{R_A}_x^2+\mybra{R_A}_y^2}=7.905\mbox{кН}\]
\begin{prob}Задача 4.10\end{prob}
\mypic{0.8}{teormech_4_10.png}
Записавши рівність моментів для балки $AB$ відносно осі, що перпендикулярна площині малюнка і проходить через $B$, бачимо, що
\[N_A=mg/2=50\mbox{Н}\]
адже лише $N_A$ та вага балки мають ненульовий момент відносно $B$ і вони діють паралельно. Знаючи $N_A$, згадаємо, що всі
сили, що діють на балку (а це $N_A$, вага балки, вага сила реакції нитки $CB$, яку
ми позначимо за $F_P$ і яка рівна за модулем вазі $P$, та $N_B$) мають врівноважитись і тому сума $\vec{F_P}+\vec{N_B}$
рівна силі в 25Н, що діє строго вверх (а саме такою є $-(\vec{mg}+\vec{N_A})$) і оскільки $F_P$ діє
строго вздовж $CB$, а $N_B$ строго перпендикулярно цьому напряму, їх модулі є проекціями $-(\vec{mg}+\vec{N_A})$ на
$CB$ і перпендикулярний йому напрям. Таким чином,
\[F_P=50\sin({30^\circ})=25\mbox{Н}\]
\[N_B=50\cos({30^\circ})=43.3012\mbox{Н}\]
\begin{prob}Задача 8.25\end{prob}
\mypic{0.8}{teormech_8_25.png}
Перш за все, запишемо рівність моментів для полички $ABCD$, відносно точки $H$. Матимемо (позначивши точку перетину діагоналей
прямокутника $ABCD$ за $O$)
\[\vec{HK}\times \vec{R_K}=-\mybra{\vec{HO}\times\vec{mg}+\vec{HD}\times{S}}\]
або в покомпонентному записі, позначивши 
\[\vec{R_K}=\mybra{X_K,Y_K,Z_K}\]
і враховуючи
\[\vec{HK}=(0,100,0)\]
\[\vec{HO}=\mybra{30,50,0}\]
\[\vec{mg}=\mybra{0,0,-800}\]
\[\vec{HD}=\mybra{60,125,0}\]
матимемо
\[(100Z_K,0,-100X_K)\approx-\mybra{10002.5 -1.20000000000073 -66670.0}\]
\[\implies Z_K=-100\mbox{Н},\;X_K=-666.7\mbox{Н}\]
Записуючи рівність моментів відносно $K$, отримаємо відповідні вирази для $\vec{R_K}$
\[\implies Z_H=500\mbox{Н},\;X_H=-133.3\mbox{Н}\]
\begin{prob}Задача 18.11\end{prob}
\mypic{0.4}{teormech_18_11.png}
Для початку, ми введемо систему координат як вказано на малюнку. Її центром є проекція точки $O$ на вісь руху повзуна $PB$.
Нам потрібно буде знайти відстань $OP$ для подальших розрахунків. Ми знайдемо її з наступних геометричних міркувань. 
Обидва трикутника $\Delta AOC$ та $\Delta CPB$ є прямими рівнобічними і таким чином
\[OC=OA\sqrt{2}=20\sqrt{2}\]
\[PC=BC/\sqrt{2}=\frac{AC+AB}{\sqrt{2}}=60\sqrt{2}\]
\[OP=PC-OC=40\sqrt{2}\]
Далі, можемо записати рівняння руху точки $A$ (вважаючи, що в початковий момент $t=0$ вона знаходилась в точці перетину
кола і відрізка $OC$)
\[A(t)=(x_A(t),y_A(t))=\mybra{x_0+20\cos\omega t,20\sin\omega t},\;\omega:=20,\;x_0:=OP=40\sqrt{2}\]
далі, запишемо рівняння руху точки $B$. Оскільки абсциса незмінна, нас цікавить лише ордината
\[y_B(t)=y_A(t)+\sqrt{AB^2-x_A(t)^2}=20\sin\omega t+\sqrt{100^2-(x_0+20\sin^2\omega t)^2}\]
Відповідно, оскільки нас цікавить такий момент часу $t_0$, що $\alpha_0=\omega t_0=\pi/4$, маємо
\[w_b=\frac{d^2}{dt^2}\mybra{20\sin\omega t+\sqrt{100^2-(x_0+20\sin^2\omega t)^2}}\bigg|_{t_0=\pi/40}=-565.6854\mbox{ см/с}^2\]
Далі, рівняння руху точки $A$ відносно точки $B$ (тобто вважаючи останню нерухомою) запишеться як
\[\mybra{\tilde{x}_A(t),\tilde{y}_A(t)}=\mybra{x_A(t),y_A(t)}-\mybra{x_B(t),y_B(t)}=\mybra{x_0+20\cos\omega t,-\sqrt{100^2-(x_0+
20\cos\omega t)^2}}\]
Далі, знайдемо закон, за яким змінюється кутове положення цього руху
\[R(t)\mybra{\cos\alpha(t),\sin\alpha(t)}=\mybra{\tilde{x}_A(t),\tilde{y}_A(t)},\;R(t):=\sqrt{\tilde{x}^2(t)+\tilde{y}^2(t)}\implies\]
\[\implies \alpha(t)=\arccos\mybra{\frac{\tilde{x}(t)}{\sqrt{\tilde{x}^2(t)+\tilde{y}^2(t)}}}=\arccos\mybra{\frac{x_0+20\cos\omega t}
{100}}\]
і таким чином
\[\omega=\frac{d}{dt}\alpha(t_0)=2\mbox{ рад/с}\]
\[\epsilon=\frac{d^2}{dt^2}\alpha(t_0)=16\mbox{ рад/с}^2\]
\begin{prob}Задача 23.26\end{prob}
Почнемо
\mypic{0.4}{teormech_23_36.png}
Загальне прискорення складається із нормальної проекції $a_n$ та тангенціальної $a_\tau$
\[a_n={\overbrace{(\underbrace{30\omega+v}_{\mbox{сумарна т. швидкість}})^2/30}^{\mbox{н. прискорення}=
v^2/r}
-\underbrace{2\cdot30\epsilon}_{\mbox{сила Коріоліса}}}\]
\[a_\tau=30\epsilon\]
\[w_a=\sqrt{a_n^2+a_\tau^2}=1018.675610\mbox{ см/с}^2\]
\end{document}
