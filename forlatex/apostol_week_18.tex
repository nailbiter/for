\documentclass[10pt]{article} % use larger type; default would be 10pt

\usepackage[utf8]{inputenc}       % кодування документа; замість cp866nav
\usepackage[margin=0.8in]{geometry}
\usepackage[russian,english]{babel} % національна локалізація; може бути декілька
\usepackage{setspace}
\usepackage{CJKutf8}
\usepackage{mdframed}

\title{Divine Liturgy\\Reading from Epistles}
\date{Week 18 after the Pentecost\vspace{-3ex}}
\begin{document}
\pagenumbering{gobble}
\begin{otherlanguage*}{russian}
\maketitle
\end{otherlanguage*}
\large\doublespacing
\framebox[\textwidth]{
\begin{minipage}[t]{0.45\textwidth}
\begin{otherlanguage*}{russian}
\textbf{ 2 Кор., 188 зач., IX, 6-11.}\\
При сем скажу: кто сеет скупо, тот скупо и пожнет; а кто сеет щедро, тот щедро и пожнет.\\
Каждый уделяй по расположению сердца, не с огорчением и не с принуждением; ибо доброхотно дающего любит Бог.\\
Бог же силен обогатить вас всякою благодатью, чтобы вы, всегда и во всем имея всякое довольство, были богаты на всякое доброе дело,\\
как написано: расточил, раздал нищим; правда его пребывает в век.\\
Дающий же семя сеющему и хлеб в пищу подаст обилие посеянному вами и умножит плоды правды вашей,\\
так чтобы вы всем богаты были на всякую щедрость, которая через нас производит благодарение Богу. \\
\end{otherlanguage*}
\end{minipage}
\hfill
\begin{minipage}[t]{0.45\textwidth}
\textbf{2 Corinthians 9:6 -- 9:11}\\
But this I say, He which soweth sparingly shall reap also sparingly; and he which soweth bountifully shall reap also bountifully.\\
Every man according as he purposeth in his heart, so let him give; not grudgingly, or of necessity: for God loveth a cheerful giver.\\
And God is able to make all grace abound toward you; that ye, always having all sufficiency in all things, may abound to every good work:\\
(As it is written, He hath dispersed abroad; he hath given to the poor: his righteousness remaineth for ever.\\
Now he that ministereth seed to the sower both minister bread for your food, and multiply your seed sown, and increase the fruits of your righteousness;)\\
Being enriched in every thing to all bountifulness, which causeth through us thanksgiving to God.\\
\end{minipage}}
\newpage\huge\doublespacing
\framebox[\textwidth]{
\begin{minipage}[t]{\textwidth}
\begin{CJK}{UTF8}{bsmi}
\textbf{哥林多後書 9:6 -- 9:11}\\
還有一點:「少種的少收;多種的多收。」\\
各人要隨心所願,不要為難,不要勉強,因為上帝愛樂捐的人。\\
上帝能將各樣的恩惠多多加給你們,使你們凡事常常充足,能多做各樣善事。\\
如經上所記:\\
「他施捨,賙濟貧窮;\\
他的義行存到永遠。」\\
那賜種子給撒種的,賜糧食給人吃的,必多多加給你們種地的種子,又增添你們仁義的果子。\\
你們必凡事富足,能多多施捨,使人藉着我們而生感謝上帝的心。 \\
\end{CJK}
\end{minipage}}
\end{document}
