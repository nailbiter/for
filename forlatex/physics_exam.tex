\documentclass[12pt]{article} % use larger type; default would be 10pt

\usepackage{mathtext}                 % підключення кирилиці у математичних формулах
                                          % (mathtext.sty входить в пакет t2).
\usepackage[T1,T2A]{fontenc}         % внутрішнє кодування шрифтів (може бути декілька);
                                          % вказане останнім діє по замовчуванню;
                                          % кириличне має співпадати з заданим в ukrhyph.tex.
\usepackage[utf8]{inputenc}       % кодування документа; замість cp866nav
                                          % може бути cp1251, koi8-u, macukr, iso88595, utf8.
\usepackage[english,russian,ukrainian]{babel} % національна локалізація; може бути декілька
                                          % мов; остання з переліку діє по замовчуванню. 

\newtheorem{prob}{Завдання}
\newcommand{\ds}{\;ds}
\newcommand{\dt}{\;dt}
\newcommand{\dx}{\;dx}
\newcommand{\dv}{\;dv}
\newcommand{\dpp}{\;dp}
\newcommand{\dta}{\;d\tau}
\let\oldint\int
\renewcommand{\int}{\oldint\limits}
\let\phi\varphi
\newcommand{\extr}{\mbox{\normalfont extr}}

\usepackage{mystyle}

\newtheorem{myulem}[mythm]{Лема}

\renewenvironment{myproof}[1][Доведення]{\begin{trivlist}
\item[\hskip \labelsep {\bfseries #1}]}{\myqed\end{trivlist}}

\title{Методи мат. фізики (9 семестр)\\Білет №9}
\author{Олексій Леонтьєв}

\begin{document}
\maketitle
\section{Друге начало термодинаміки}
В своїй найбільш примітивній формі, друге начало термодинаміки стверджує, що {\bf в замкненій системі ентропія $S$ не спадає, тобто що $dS\geq0$.} 
Нагадаємо, що для квазирівноважних ентропія може бути введена за формулою
\[dS=\frac{dQ}{T}\]
і це поняття (ентропії) може бути узагальнено для нерівноважних процесів, для яких вона є чимось на зразок міри "хаотичності" процесу. Нагадаємо,
що перше начало термодинаміки, як по суті є наслідком закону збереження енергії, стверджує, що для квазистатичного процесу має місце відношення
\[dQ=dU+dA\]
і, таким чином, разом із визначенням ентропії та другим началом, це можна записати як
\[TdS\geq dU+dA\]
Перше начало стверджує, що при перетіканні енергії від однієї системи до іншої енергія зберігається, в той час як друге начало вказує {\it напрямок
} енергетичного потоку -- від теплого тіла до холодного, адже саме при рівній температурі двох тіл ентропія досягає найвищого значення.
\section{Потік вектору напруженості електростатичного поля. Теорема Гауса.}
Нехай $S$ -- двовимірний орієнтовний многовид (поверхня), що знаходиться в полі $\myvec{E}=\myvec{E}(\myvec{r})$. Для такого многовиду в кожній точці
його може бути визначено нормаль $\myvec{n}=\myvec{n}(\myvec{r})$, і таким чином, можна ввести поняття {\bf потоку вектору напруженості через
поверхню $S$} як поверхневий інтеграл
\[\Phi=\iint_S(\myvec{E},\myvec{n})\;dS\]
Теорема Гауса, в свою чергу, стверджує, що для замкненої поверхні $S=\partial W$, яка є границею відкритої множини $W\subset\mathbb{R}^3$,
 і коли $\myvec{n}$ направлено назовні, має місце співвідношення
\[\oiint_S(\myvec{E},\myvec{n})\;dS=\frac{1}{\epsilon_0}\iiint_W\rho(\myvec{r})\;dV\]
де інтеграл в правій частині розуміється як сума, якщо всередині $W$ присутні дискретні заряди.

Це -- інтегральна форма теореми Гауса. Існує також і диференціальна форма, яка може бути отримана з інтегральної застосуванням теореми
Гауса-Остроградського (яка, в свою чергу, є узагальненням теореми Стокса), яка для векторного поля $\myvec{E}=\myvec{E}(\myvec{r})$ і замкненої
поверхні $S=\partial W$ записується як
\[\oiint_S(\myvec{E},\myvec{n})\;dS=\iiint_W\mybra{\myvec{\nabla},\myvec{E}}\;dV,\quad\mybra{\myvec{\nabla},\myvec{E}}
\mbox{ -- дивергенція}\]
Враховуючи інтегральну форму теореми Гауса, отримаємо що для довільної відкритої множини має місце
\[\iiint_W\mybra{\myvec{\nabla},\myvec{E}}\;dV=\frac{1}{\epsilon_0}\iiint_W\rho(\myvec{r})\;dV\]
а отже підінтегральні вирази мають бути рівними
\[\mybra{\myvec{\nabla},\myvec{E}}=\frac{\rho(\myvec{r})}{\epsilon_0}\]

Доведення інтегральної теореми Гауса проводиться спочатку для одного точкового заряду з суто геометричних міркувань, потім узагальнюється на
випадок скінченної суми зарядів, адже як $\myvec{E}$, так і права частина інтегральної теореми Гауса є адитивними по $\rho$, а потім
і для неперервного розподілу переходом до границі.
\end{document}
