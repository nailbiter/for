\documentclass[8pt]{article} % use larger type; default would be 10pt

\usepackage[10pt]{type1ec}          % use only 10pt fonts
\usepackage[T1]{fontenc}
\usepackage{graphicx}
\usepackage{float}
\usepackage{CJKutf8}
\usepackage{subfig}
\usepackage{amsmath}
\usepackage{amsfonts}
\usepackage{hyperref}
\usepackage{enumerate}
\usepackage{enumitem}
\usepackage[T1,T2A]{fontenc}
\usepackage[utf8]{inputenc}
\usepackage[english,ukrainian]{babel}

%theorem environments configuration
\newtheorem{problem}{Задача}
\newenvironment{solution}%
{\par\textbf{Розв'язок}\space }%
{\par}

%custom theorems for saving typing
\renewcommand{\P}{\mathbb{P}}

\title{Завдання на залік з Теорії Ймовірності\\Зимова Сессія 2013}
\author{Студент 4го курсу\\Механіко-математичного факультету КНУ\\Заочної форми\\Олексій Леонтьєв}

\begin{document}
\maketitle
\begin{problem}
Нехай є урна з $n$ кулями невідомого кольору. При цьому всі припущення стосовно складу урни є однаково ймовірними. Перша витягнута куля виявляється
білою. Яка ймовірність того, що друга витягнута теж виявиться білою?
\end{problem}
\begin{solution}
Варто зробити декілька зауважень, перед тим як почати робити власне розрахунки. На жаль, задача сформульована дещо нечітко, тому нам доведеться
зробити деякі припущення щодо інтерпретації умови. Ми розуміємо, що ці припущення (у випадку, якщо зроблені невірно) одразу роблять весь
розв'язок некоректним. Проте єдине, що ми можемо зробити -- це чітко сформулювати ці припущення, для зручності перевіряючого. По-перше, 
інтерпретації вимагає твердження про \textit{"всі припущення стосовно складу урни є однаково ймовірними"}. Ми вважаємо, це означає, що
\textbf{ймовірність того, що в урні $m$ ($1\leq m\leq$) білого кольору рівна $1/n$}. Друге, ми вважаємо, що \textbf{перша куля не поверталася
до урни перед вийняттям другої}. Тепер, після того як припущення (можливо, невідповідні оригінальному задуму автора задачі) зроблені, ми можемо
переходити до, відверто кажучи, нескладних розрахунків.\\
Позначим події, що складаються з того, що перша витягнутя куля виявилася білою, друга білою і всього в урна $m$ білих куль як $A$, $B$ та
$W_m$ відповідно. В цих термінах, нас цікавить $\mathbb{P}(B|A)$. Проте
\[\P(B|A)=\P(B|A,W_1)\P(W_1|A)+\P(B|A,W_2)\P(W_2|A)+\ldots+\P(B|A,W_n)\P(W_n|A)\]
\[\P(W_i|A)=\frac{\P(A|W_i)\P(W_i)}{\sum_{j=1}^n\P(A|W_j)\P(W_j)}=\frac{\frac{i}{n^2}}{\sum_{j=1}^n \frac{j}{n^2}}=\frac{2i}{n(n+1)}\]
%=\frac{1}{n}\sum_{i=1}^n\P(B|A,W_i)\]
зрозуміло, що для $i\geq1$
\[\P(B|A,W_i)=\frac{\P(A\cap B|W_i)}{\P(A|W_i)}=\frac{\frac{i}{n}\cdot\frac{i-1}{n-1}}{\frac{i}{n}}=\frac{i-1}{n-1}\]
Тому
\[\P(B|A)=\sum_{i=1}^n \frac{2i(i-1)}{n(n-1)(n+1)}=\frac{2}{n(n-1)(n+1)}\left(\frac{2n^3+3n^2+n}{6}-\frac{n(n+1)}{2}\right)=\]
\[=\frac{2}{3}\]
\end{solution}
\end{document}
