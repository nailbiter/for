\documentclass[12pt]{article}
\usepackage[english]{babel}
\usepackage{amsmath,amssymb,graphicx,bbm,hyperref,latexsym,theorem}
\usepackage{mathtools}
\usepackage{dot2texi}
\usepackage{tikz}
\usepackage{blindtext,titlefoot}
\usepackage[inline]{showlabels}

\newcommand{\myabs}[1]{\left|#1\right|}
\newcommand{\N}{\mathbb{N}}
\newcommand{\C}{\mathbb{C}}
\newcommand{\R}{\mathbb{R}}
\renewcommand{\Re}{\operatorname{Re}}
\renewcommand{\Im}{\operatorname{Im}}

\numberwithin{equation}{section}

%%%%%%%%%% Start TeXmacs macros
\newcommand{\mynorm}[1]{\left\|#1\right\|}
\catcode`\<=\active \def<{
\fontencoding{T1}\selectfont\symbol{60}\fontencoding{\encodingdefault}}
\newcommand{\assign}{:=}
\newcommand{\comma}{{,}}
\newcommand{\nin}{\not\in}
\newcommand{\tmdummy}{$\mbox{}$}
\newcommand{\tmop}[1]{\ensuremath{\operatorname{#1}}}
\newcommand{\tmscript}[1]{\text{\scriptsize{$#1$}}}
\newcommand{\tmtextbf}[1]{{\bfseries{#1}}}
\newcommand{\tmtextit}[1]{{\itshape{#1}}}
\newcommand{\tmtextmd}[1]{{\mdseries{#1}}}
\newcommand{\tmtextrm}[1]{{\rmfamily{#1}}}
\newcommand{\tmtextup}[1]{{\upshape{#1}}}
\newenvironment{proof*}[1]{\noindent\textbf{\textit{#1\ }}}{\hspace*{\fill}$\Box$\medskip}
\newtheorem{corollary}{Corollary}[section]
{\theorembodyfont{\rmfamily}\newtheorem{example}[corollary]{Example}}
\newtheorem{lemma}[corollary]{Lemma}
\newtheorem{proposition}[corollary]{Proposition}
{\theorembodyfont{\rmfamily}\newtheorem{remark}[corollary]{Remark}}
\newtheorem{theorem}[corollary]{Theorem}
%%%%%%%%%% End TeXmacs macros

%proofreading symbols
\newcommand{\mykana}[2]{#1}
\newcommand{\doubt}[1]{\fbox{#1}}
\newcommand{\pause}{$\bullet$}
\newcommand{\slowly}[1]{\dashuline{#1}}
\newcommand{\continuously}[1]{\underline{#1}}
\newcommand{\badword}[1]{\uwave{#1}}

%my commants
\newcommand{\mygrammarfootnote}[1]{}

\newcommand{\sectionsep}{{\sectionalsep}}
\newcommand\blfootnote[1]{%
	  \begingroup
	  \renewcommand\thefootnote{\footnote{#1}%
    \addtocounter{footnote}{-1}%
      \endgroup
      }}

\title{\bf Double Gegenbauer expansion of $\myabs{s+t}^\alpha$}
\author{Toshiyuki Kobayashi \and Alex Leontiev}

\begin{document}
\unmarkedfntext{2010 Mathematics Subject Classification. Primary 22E46; Secondary 33C45, 53C35.\\
(or should it be 42C05, 33C45, 33C05?)}
\makeatletter
\renewcommand{\thefootnote}{\ifcase\value{footnote}\or*)\or
**)\or(***)\or(****)\or(\#)\or(\#\#)\or(\#\#\#)\or(\#\#\#\#)\or($\infty$)\fi}
\makeatother

    \begin{center}
        {\large\bf Double Gegenbauer expansion of $\myabs{s+t}^\alpha$\par}
\vspace{1em}
{\small By Toshiyuki \textsc{Kobayashi}\footnote{Kavli IPMU and Graduate School of Mathematical Sciences, The University of Tokyo}\footnotemark and Alex \textsc{Leontiev}\footnotemark[\value{footnote}]}
\footnotetext{Graduate School of Mathematical Sciences, The University of Tokyo}
\end{center}
\vspace{2\baselineskip}
\renewcommand{\thefootnote}{\arabic{footnote}}
%%\begin{abstract}
%%\end{abstract}
{
{
	\hspace{1.5cm}\begin{minipage}[]{0.8\textwidth}
  \quad\textbf{Abstract:}\quad
  Motivated by the study of symmetry breaking operators for indefinite
  orthogonal groups, we give a Gegenbauer expansion of the two variable
  function $| s + t |^{\alpha}$ in terms of the ultraspherical polynomials
  $C_{\ell}^{\lambda} (s)$ and $C^{\mu}_m (t)$.
  
  Specializations and limits of the expansion are discussed in the context of
  specializations of the Selberg integral and its
  generalization\mygrammarfootnote{maybe, ``generalization'' should be in plural (i.e.
  ``generalizations'')?} by Dotsenko, Fateev, Tarasov, Varchenko, and Warnaar
  among others.

\quad\textbf{Key words:}\quad
Gegenbauer polynomial, Sobolev inequality, Hermite polynomial, Selberg integral\mygrammarfootnote{maybe, dot here?}
\end{minipage}
}}

\vspace{2em}
In this article, we give an expansion of the power $\myabs{s+t}^\alpha$ by two Gegenbauer polynomials $C^\lambda_\ell(s)$ and $C^\mu_m(t)$ with independent parameters $\lambda$ and $\mu$.
\unmarkedfntext{The instruction ``\underline{Remark} estimate'' was written on the last correction. I am not sure I understand what it means. Could You explain it to me once again?
I am extremely sorry.}
\section{Main results}
Let $C_{\ell}^{\lambda} (s)$ be the Gegenbauer polynomial of degree $\ell$. For $\ell,m\in\N$, we set
\begin{equation*}
     \displaystyle a_{\lambda, \mu, \nu}^{\ell, m} \assign\displaystyle \frac{2^{- 2 \nu} (\lambda + \ell)
    (\mu + m) \Gamma (\lambda + \mu + 2 \nu + 1) \Gamma (\lambda) \Gamma (\mu)
    \Gamma (2 \nu + 1)}{\Gamma \left( \lambda + \nu + \frac{\ell - m}{2} + 1
    \right) \Gamma \left( \mu + \nu - \frac{\ell - m}{2} + 1 \right) \Gamma
    \left( \lambda + \mu + \nu + \frac{\ell + m}{2} + 1 \right) \Gamma \left(
    \nu + 1 - \frac{\ell + m}{2} \right)} .
\end{equation*}
\begin{theorem}
	\label{thm:1-1}
	Let $\lambda,\mu$ be positive numbers and $\nu\in\C$ satisfying $2\Re \nu>\lambda+\mu+2$. Then
  \begin{equation*}
	  \begin{array}[]{c}
		  | s + t |^{2 \nu} = \displaystyle\sum_{\scalebox{0.6}{$\begin{array}[]{l}
			  \ell,m=0\\
			  l+m:\tmop{even}
		  \end{array}$}}^{\infty} a_{\lambda, \mu, \nu}^{\ell, m} C_{\ell}^{\lambda}
    (s) C_m^{\mu} (t), \\
	  \end{array}
  \end{equation*}
  where the right-hand side converges absolutely and uniformly in $(s,t)\in[-1,1]^2$.
\end{theorem}
By the Sobolev-type estimate for the Gegenbauer expansion given in Proposition \ref{prop:1718105},
Theorem \ref{thm:1-1} is deduced from the integral formula (Proposition \ref{cor:1}),
which is a special case at $x=1$ of the following.
\begin{theorem}
  \label{main-thm}Suppose $\ell, m \in \mathbbm{N}$ such that $\ell + m$ is
  even. For $\lambda, \mu, \nu \in \mathbbm{C}$ with $\tmop{Re} \lambda,
  \tmop{Re} \mu, \tmop{Re} \nu > - 1 / 2$, and for $- 1 \leqslant x \leqslant
  1$, we have
  \begin{multline}
     \displaystyle\int_{- 1}^1 \displaystyle\int_{- 1}^1 | s + xt |^{2 \nu} u_{\ell}^{\lambda} (s)
    u_m^{\mu} (t) d s d t   \\
     \displaystyle= \frac{(- \nu)_{\frac{\ell + m}{2}} (- 1)^{\frac{\ell + m}{2}}
    \pi^{\frac{3}{2}} \Gamma \left( \nu + \frac{1}{2} \right) x^m{}_2 F_1
    \left( \begin{array}{c}
      \frac{\ell + m}{2} - \nu, \frac{m - \ell}{2} - \nu - \lambda\\
      \mu + m + 1
    \end{array} ; x^2 \right)}{\Gamma (\mu + m + 1) \Gamma \left( \lambda +
    \nu + \frac{\ell - m}{2} + 1 \right)},  \label{eqn:main}  
  \end{multline}
  where $(y)_j \assign y (y + 1) (y + 2) \cdots (y + j - 1)$ is the Pochhammer 
  symbol, and we set
\[ u_{\ell}^{\lambda} (s) \assign \frac{2^{2 \lambda - 1} \ell ! \Gamma
   (\lambda)}{\Gamma (2 \lambda + \ell)} (1 - s^2)^{\lambda - \frac{1}{2}}
   C_{\ell}^{\lambda} (s) . \]
\end{theorem}
\begin{remark}
	Note that even slightly more general formula
	\begin{multline}
		\displaystyle\int_{- 1}^1 \displaystyle\int_{- 1}^1 | s + xt |^{2 \nu}\operatorname{sgn}^{\frac{1\pm1}{2}}\left( s+xt \right) u_{\ell}^{\lambda} (s)
    u_m^{\mu} (t) d s d t
     \\
     =\begin{cases}
      \frac{(- \nu)_{\frac{\ell + m}{2}} (- 1)^{\frac{\ell + m}{2}}
    \pi^{\frac{3}{2}} \Gamma \left( \nu + \frac{1}{2} \right) x^m{}_2 F_1
    \left( \begin{array}{c}
      \frac{\ell + m}{2} - \nu, \frac{m - \ell}{2} - \nu - \lambda\\
      \mu + m + 1
    \end{array} ; x^2 \right)}{\Gamma (\mu + m + 1) \Gamma \left( \lambda +
    \nu + \frac{\ell - m}{2} + 1 \right)},&\left(\ell+m\equiv\frac{1\pm1}{2}\operatorname{mod}2  \right)\\
    0,&\left( \operatorname{otherwise} \right)
     \end{cases}
		\nonumber
	\end{multline}
	can be shown.
\end{remark}

Note that as the Gegenbauer polynomial $g (s) \assign C_{\ell}^{\lambda} (s)$
satisfies the second-order differential equation
\begin{eqnarray}
  & (1 - s^2) g'' - (2 \lambda + 1) s g' + n (n + 2 \lambda) g = 0, & 
  \nonumber
\end{eqnarray}
$f (s) \assign u_{\ell}^{\lambda} (s)$ satisfies
\begin{eqnarray}
  & (1 - s^2)^2 f'' + (1 - s^2) (1 - 2 \lambda - (2 \lambda + 1) s) f' + ((2
  s + 1) (\lambda^2 - 1 / 4) + \ell (\ell + 2 \lambda) (1 - s^2)) f = 0. & 
  \nonumber
\end{eqnarray}
\begin{remark}
  Since $u_m^{\mu} (- t) = (- 1)^m u_m^{\mu} (t)$, Theorem \ref{main-thm} is
  reduced to the case $0 \leqslant x \leqslant 1$.
\end{remark}

Selberg-type integrals are related to (finite-dimensional) representation
theory of semisimple Lie algebras, see {\cite{forrester2008importance}},
{\cite{tarasov2003selberg}} and references therein. On the other hand, Theorem
\ref{main-thm} and Corollary \ref{cor:1} will be used in the study of symmetry
breaking operators for infinite-dimensional representations when we extend the
work {\cite{kobayashi2015symmetry}} to indefinite orthogonal groups $O (p,
q)$. This will be done in a separate paper.

Special cases and the limit case of Theorem \ref{main-thm} will be discussed
in Section \ref{sec:4} \doubt{and $\bullet$}.

\section{Proof of the main theorem}\label{sec:2}

In this section we prove that Theorem \ref{main-thm}
is deduced from the special case $\ell=m=0$, namely, from
the following
integral formula $(\ref{eqn:stz})$.
Proposition \ref{prop:2}
will be proved in Section
\ref{sec:3}.

\begin{proposition}
  \label{prop:2}For $a, b, c \in \mathbbm{C}$ such that $\tmop{Re} a,
  \tmop{Re} b, \tmop{Re} c > 0$ and for $- 1 \leqslant x \leqslant 1$, we have
  \begin{multline}
     \displaystyle\int_{- 1}^1 \displaystyle\int_{- 1}^1 | s + t x |^{2 c - 1} (1 - s^2)^{a - 1} (1 -
    t^2)^{b - 1} d s d t = \frac{\sqrt{\pi} \Gamma (a) \Gamma (b) \Gamma
    (c)}{\Gamma (a + c) \Gamma \left( b + \frac{1}{2} \right)}{}_2 F_1 \left(
    \begin{array}{c}
      - c + \frac{1}{2}, - a - c + 1\\
      b + \frac{1}{2}
    \end{array} ; x^2 \right) .  \label{eqn:stz}
  \end{multline}
\end{proposition}

\begin{proof*}{Proof of Theorem \ref{main-thm}}
  The Rodrigues formula for the Gegenbauer polynomial (see
  {\cite[(6.4.14)]{andrews2000special}} for instance) shows
  \begin{equation} u_{\ell}^{\lambda} (t) = \frac{(- 1)^{\ell} 2^{- \ell} \sqrt{\pi}}{\Gamma
     \left( \lambda + \ell + \frac{1}{2} \right)} \cdot \frac{d^{\ell}}{d
     t^{\ell}} (1 - t^2)^{\lambda + \ell - \frac{1}{2}} . 
     \label{eqn:Rod} \end{equation}
  Then\mygrammarfootnote{maybe, we need a comma here?} the left-hand side of
  $(\ref{eqn:main})$ amounts to
  \begin{eqnarray}
    & \displaystyle\frac{2^{- \ell - m} \pi}{\Gamma \left( \lambda + \ell + \frac{1}{2}
    \right) \Gamma \left( \mu + m + \frac{1}{2} \right)} I_{\ell, m} (x), & 
    \nonumber
  \end{eqnarray}
  where we set
  \[ I_{\ell, m} (x) \assign \displaystyle\int_{- 1}^1 \displaystyle\int_{- 1}^1 | s + tx |^{2 \nu}
     \frac{\partial^{\ell}}{\partial s^{\ell}} (1 - s^2)^{\lambda + \ell -
     \frac{1}{2}} \frac{\partial^m}{\partial t^m} (1 - t^2)^{\mu + m -
     \frac{1}{2}} d s d t. \]
  Suppose $\tmop{Re} \nu > \frac{\ell + m}{2}$, $\tmop{Re} \lambda >
  \frac{1}{2}$ and $\tmop{Re} \mu > \frac{1}{2}$. Then
  integration by parts gives
  \begin{eqnarray}
    & I_{\ell, m} (x) = \displaystyle\int_{- 1}^1 \displaystyle\int_{- 1}^1 (1 - s^2)^{\lambda + \ell -
    \frac{1}{2}} (1 - t^2)^{\mu + m - \frac{1}{2}} \frac{\partial^{\ell +
    m}}{\partial s^{\ell} \partial t^m} | s + t x |^{2 \nu} d s d t. 
    \label{eqn:derst} & 
  \end{eqnarray}
  Since $\ell + m \in 2\mathbbm{N}$ we have
  \[ \frac{\partial^{\ell + m}}{\partial s^{\ell} \partial t^m} | s + tx |^{2
     \nu} = (- 2 \nu)_{\ell + m} x^m | s + t x |^{2 \nu - \ell - m}, \]
  and therefore $(\ref{eqn:derst})$ equals
  \begin{eqnarray}
    & (- 2 \nu)_{\ell + m} x^m \displaystyle\int_{- 1}^1 \displaystyle\int_{- 1}^1 | s + t x |^{2 \nu -
    \ell - m} (1 - s^2)^{\lambda + \ell - \frac{1}{2}} (1 - t^2)^{\mu + m -
    \frac{1}{2}} d s d t. &  \nonumber
  \end{eqnarray}
  Applying Proposition \ref{prop:2} with $(a,
  b, c) = \left( \lambda + \ell + \frac{1}{2}, \mu + m + \frac{1}{2}, \nu +
  \frac{1}{2} (1 - \ell - m) \right)$, we see\mygrammarfootnote{``that''?}
  the equation \eqref{eqn:main} holds in the domain of $(\lambda,\mu,\nu)$ that we treated. Now Theorem \ref{main-thm} follows
  by analytic continuation.
\end{proof*}

\section{Proof of Proposition \ref{prop:2}}\label{sec:3}
In this section we show Proposition \ref{prop:2}.
We use the following two lemmas.
\begin{lemma}
  \label{lem4}For $a, b, z \in \mathbbm{C}$ with $\tmop{Re} a, \tmop{Re} b >
  0$ and for $- 1 \leqslant x \leqslant 1$ we have
  \begin{equation}
     \displaystyle\int_{- 1}^1 (1 + tx)^{a - 1} (1 - t^2)^{b - 1} d t = B \left(
    \frac{1}{2}, b \right){}_2 F_1 \left( \begin{array}{c}
      \displaystyle\frac{1 - a}{2}, \frac{2 - a}{2}\\
      b + \frac{1}{2}
    \end{array} ; x^2 \right).
    \label{eqn:lem31}
  \end{equation}
\end{lemma}
\begin{lemma}
  \label{lem:Fisum}Let $a, b, c \in \mathbbm{C}$ with $b + \frac{1}{2} \nin
  -\mathbbm{N}$. Then the series\mygrammarfootnote{maybe, we need a comma here?}
  \[ G (a, b, c ; \zeta) \assign \displaystyle\sum_{i = 0}^{\infty} \frac{(a)_i (1 -
     a)_i}{2^i i! (d)_i}{}_2 F_1 \left( \begin{array}{c}
       \displaystyle\frac{1 - d - i}{2}, \frac{2 - d - i}{2}\\
       b + \frac{1}{2}
     \end{array} ; \zeta \right) \]
  converges when $| \zeta | < 1$, and we have the following closed formula:
  \begin{equation} G (a, b, c ; \zeta) = \frac{2^{1 - d} \sqrt{\pi} \Gamma (d)}{\Gamma
     \left( \frac{a + d}{2} \right) \Gamma \left( \frac{1 - a + d}{2} \right)}
    {}_2 F_1 \left( \begin{array}{c}
       \displaystyle1 - \frac{a + d}{2}, \frac{1 + a - d}{2}\\
       b + \frac{1}{2}
   \end{array} ; \zeta \right) .  \label{eqn:iF}\end{equation}
\end{lemma}

Postponing the verification of Lemmas \ref{lem4} and \ref{lem:Fisum}, we first
show Proposition \ref{prop:2}.

\begin{proof*}{Proof of Proposition \ref{prop:2}}
  By the change of variables $s = 1 - (1 + x) t$, the interval $0 \leqslant t
  \leqslant 1$ is transformed onto $(- 1 \leqslant) - x \leqslant s \leqslant
  1$, and thus Euler's integral representation of the hypergeometric function
  $_2 F_1$ shows
  \begin{eqnarray}
    & \displaystyle\int_{- 1}^1 (s + x)_+^{2 c - 1} (1 - s^2)^{a - 1} d s = 2^{a - 1} B (2
    c, a) (1 + x)^{2 c + a}{}_2 F_1 \left( \begin{array}{c}
      1 - a, 2 c\\
      2 c + a
    \end{array} ; \frac{1 + x}{2} \right), &  \nonumber
  \end{eqnarray}
  where $y_+^{\lambda} \assign | y |^{\lambda}$ for $y > 0$; $= 0$ for $y
  \leqslant 0$.
  
  Therefore the left-hand side of $(\ref{eqn:stz})$ equals:
  \begin{multline}
     2 \displaystyle\int_{- 1}^1 \displaystyle\int_{- 1}^1 (s + tx)_+^{2 c - 1} (1 - s^2)^{a - 1} (1 -
    t^2)^{b - 1} d s d t   \\
     = 2^a B (2 c, a) \displaystyle\int_{- 1}^1 (1 + tx)^{2 c + a - 1}{}_2 F_1 \left(
    \begin{array}{c}
      1 - a, a\\
      2 c + a
    \end{array} ; \frac{1 + tx}{2} \right) (1 - t^2)^{b - 1} d t.   
	  \nonumber
  \end{multline}
  Fix $\varepsilon > 0$. Assume $| x | \leqslant 1 - 2 \varepsilon$. Then
  $\left| \frac{1 + tx}{2} \right| \leqslant 1 - \varepsilon$. Expanding the
  hypergeometric function as a uniformly convergent power series of
  $\frac{1}{2} (1 + tx)$, we can rewrite the integral in the right-hand side
  as
  \begin{eqnarray}
    & \displaystyle\sum_{i = 0}^{\infty} \frac{(1 - a)_i (a)_i}{2^i i! (2 c + a)_i}
    \displaystyle\int_{- 1}^1 (1 + t x)^{2 c + a - 1 + i} (1 - t^2)^{b - 1} d t. & 
    \nonumber
  \end{eqnarray}
  Owing to Lemma \ref{lem4}, this is equal to
  \begin{eqnarray}
    & B \left( \frac{1}{2}, b \right) \displaystyle\sum_{i = 0}^{\infty} \frac{(1 - a)_i
    (a)_i}{2^i i! (2 c + a)_i}{}_2 F_1 \left( \begin{array}{c}
      \displaystyle\frac{1 - 2 c - a - i}{2}, \frac{2 - 2 c - a - i}{2}\\
      b + \frac{1}{2}
    \end{array} ; x^2 \right) . &  \nonumber
  \end{eqnarray}
  Now (\ref{eqn:stz}) follows from Lemma \ref{lem:Fisum} with $\zeta = x^2$
  and $d = 2 c + a$\mygrammarfootnote{maybe, we need a comma here?} and from the
  duplication formula of the Gamma function $\Gamma (2 c) = \pi^{-
  \frac{1}{2}} 2^{2 c - 1} \Gamma (c) \Gamma \left( c + \frac{1}{2} \right)$.
\end{proof*}

\begin{proof*}{Proof of Lemma \ref{lem4}}
  By Euler's integral representation of $_2 F_1$ again,
  the left-hand side of \eqref{eqn:lem31} amounts to
  \begin{eqnarray}
    & \displaystyle2^{2 b - 1} (1 +
    x)^{a - 1} B (b, b)_2 F_1 \left( \begin{array}{c}
      1 - a, b\\
      2 b
    \end{array} ; \frac{2 x}{1 + x} \right) .  \label{eqn:quad} & 
  \end{eqnarray}
  Applying the quadratic transformation of $_2 F_1$ (cf. {\cite[Thm.
  3.13]{andrews2000special}}):
  \begin{eqnarray}
    & \;_2 F_1 \left( \begin{array}{c}
      \displaystyle1 - a, b\\
      2 b
    \end{array} ; u \right) = \left( 1 - \frac{z}{2} \right)^{a - 1}{}_2 F_1
    \left( \begin{array}{c}
      \displaystyle\frac{1 - a}{2}, \frac{2 - a}{2}\\
      b + \frac{1}{2}
    \end{array} ; \displaystyle\left( \frac{u}{2 - u} \right)^2 \right), &  \nonumber
  \end{eqnarray}
  with $u = \frac{2 z}{1 + z}$, we get the desired result after a small
  computation using the duplication formula of the Gamma function.
\end{proof*}

\begin{proof*}{Proof of Lemma \ref{lem:Fisum}}
  We list some elementary identities for the Pochhammer symbol $(y)_j =
  \frac{\Gamma (y + j)}{\Gamma (y)}$:
  \begin{eqnarray}
    & y_j (1 - y)_{- j} & = \; (- 1)^j,  \label{eqn:p1}\\
    & \left( \frac{y}{2} \right)_j \left( \frac{1 + y}{2} \right)_j & = \;
    2^{- 2 j} (y)_{2 j},  \label{eqn:p2}\\
    & (y)_i (1 - y)_{2 j} & = \; (1 - y - i)_{2 j} (y - 2 j)_i . 
    \label{eqn:p3}
  \end{eqnarray}
  To prove the equation \eqref{eqn:iF}, we first show the following.
  \begin{eqnarray}
    & G (a, b, d ; \zeta) = \displaystyle\sum_{j = 0}^{\infty} \frac{(1 - d)_{2 j}
    \zeta^j}{2^{2 j} j! \left( b + \frac{1}{2} \right)_j}{}_2 F_1 \left(
    \begin{array}{c}
      a, 1 - a\\
      d - 2 j
    \end{array} ; \frac{1}{2} \right) . &  \label{eqn:Fijsum}
  \end{eqnarray}
  Indeed, by expanding the hypergeometric function as a power series and by
  using $(\ref{eqn:p2})$ with $y = 1 - d - i$, we have
  \begin{eqnarray}
    & G (a, b, d ; \zeta) = \displaystyle\sum_{i = 0}^{\infty} \displaystyle\sum_{j = 0}^{\infty}
    \frac{(a)_i (1 - a)_i}{2^{i + 2 j} i!j! (d)_i}  \frac{(1 - d - i)_{2
    j}}{\left( b + \frac{1}{2} \right)_j} \zeta^j, &  \nonumber
  \end{eqnarray}
  which is equal to the right-hand side of $(\ref{eqn:Fijsum})$ by
  $(\ref{eqn:p3})$ with $y = d$.
  
  As $\;_2 F_1 \left( \begin{array}{c}
    a, 1 - a\\
    c
  \end{array} ; \displaystyle\frac{1}{2} \right) =\displaystyle \frac{2^{1 - c} \sqrt{\pi} \Gamma
  (c)}{\Gamma \left( \frac{a + c}{2} \right) \Gamma \left( \frac{c - a + 1}{2}
  \right)}$ (see {\cite[Thm. 5.4]{andrews2000special}} for instance), we can
  continue as
  \begin{eqnarray}
    & \begin{array}{ll}
      (\ref{eqn:Fijsum}) & = \displaystyle\sum_{j = 0}^{\infty} \frac{(1 - d)_{2 j}
      \zeta^j}{2^{2 j} j! \left( b + \frac{1}{2} \right)_j} \cdot \frac{2^{1 -
      d + 2 j} \sqrt{\pi} \Gamma (d - 2 j)}{\Gamma \left( \frac{a + d}{2} - j
      \right) \Gamma \left( \frac{1 - a + d}{2} - j \right)}\\
      &\displaystyle = \frac{2^{1 - d} \sqrt{\pi} \Gamma (d)}{\Gamma \left( \frac{a +
      d}{2} \right) \Gamma \left( \frac{1 - a + d}{2} \right)} \displaystyle\sum_{j =
      0}^{\infty} \frac{\left( 1 - \frac{a + d}{2} \right)_j \left( \frac{1 +
      a - d}{2} \right)_j}{j! \left( b + \frac{1}{2} \right)_j} \zeta^j,
    \end{array} &  \nonumber
  \end{eqnarray}
  where we have used $(\ref{eqn:p1})$ with $y = 1 - d, 1 - \frac{1}{2} (a +
  d),$ and $\frac{1}{2} (1 + a - d)$ in the second equality. Hence Lemma
  \ref{lem:Fisum} is proven.
\end{proof*}

\section{Sobolev-type estimate for Gegenbauer expansion}\label{sec:Sobolev}
In this section we formulate a Sobolev-type estimate for Gegenbauer expansion, by which
Theorem \ref{thm:1-1} follows ready\mygrammarfootnote{readily?} from the special value of the integral formula (Theorem \ref{main-thm}),
as we shall see in Section \ref{sec:pfThm}.

We begin with a basic {set-up}.
If $\lambda>-\frac{1}{2}$ and $\lambda\neq0$, the Gegenbauer polynomials $\left\{ C_n^\lambda(x) \right\}_{n\in\N}$
form an orthogonal basis in the Hilbert space $L_\lambda^2:=L^2\left( (-1,1),(1-x^2)^{\lambda-\frac{1}{2}}dx \right)$
with the norm\begin{equation}
	\label{eqn:vnlmd}
	v_n^\lambda\assign\left\|C_n^\lambda(x)\right\|^2_{L^2_\lambda}=\frac{\pi2^{1-2\lambda}\Gamma(n+2\lambda)}{n!(n+\lambda)\Gamma(\lambda)^2}.
\end{equation}Therefore, for any $f\in L_\lambda^2$, we have an $L^2$-expansion\begin{equation}
	\label{eqn:aGegen}
	f(x)=\displaystyle\sum_{n=0}^\infty a_n(f)C_n^\lambda(x)
\end{equation}where $a_n(f)\in\C$ is\mygrammarfootnote{are?} given by
\begin{equation*}
	a_n(f)=\frac{1}{v_n^\lambda}\displaystyle\int_{-1}^1f(x)C_n^\lambda(x)(1-x^2)^{\lambda-\frac{1}{2}}dx.
\end{equation*}
\begin{proposition}
	\label{prop:1718105}(Sobolev-type inequality for Gegenbauer expansion)
	Suppose $\lambda>-\frac{1}{2}$ and $\lambda\neq0$. Then there exists $D_\lambda>0$ with the following property:
	let $N\equiv N(\lambda)$ be the smallest integer satisfying $N>\lambda+1$. Then
	\begin{equation*}
		\mynorm{f(x)}_{L^\infty(-1,1)}\le D_\lambda\left( \mynorm{f(x)}_{L^2_\lambda}+\mynorm{f^{(N)}(x)}_{L^2_{\lambda+N}} \right)	
	\end{equation*}
	for any $f(x)\in L_\lambda^2$ such that the $N$th derivative $f^{(N)}(x)$ belongs to $L^2_{\lambda+N}$. Moreover, the Gegenbauer expansion
	\eqref{eqn:aGegen} converges absolutely and uniformly in $[-1,1]$ for any such $f(x)$.
\end{proposition}
The rest of this section is devoted to the proof of Proposition \ref{prop:1718105}. We begin with the following.
\begin{lemma}
	Suppose $\lambda\neq0$ and $\lambda>-\frac{1}{2}$. Let $N\in\N$. Then there is a constant $d_{\lambda,N}>0$ such that \begin{equation*}
		\mynorm{C_n^\lambda(x)}_{L^\infty(-1,1)}\le d_{\lambda,N}\mynorm{C^{\lambda+N}_{n-N}(x)}_{L^2_{\lambda+N}}\mbox{ for all $n\ge N$.}
	\end{equation*}
	\label{lem:1718109}
\end{lemma}
\begin{proof*}{Proof.}
	In view that\begin{equation*}
		\myabs{C_n^\lambda(x)}\le C_n^\lambda(1)=\frac{\Gamma(n+2\lambda)}{n!\Gamma(2\lambda)}\mbox{ for all $-1\le x\le 1$,}
	\end{equation*}we see Lemma \ref{lem:1718109} from \eqref{eqn:vnlmd} and from the asymptotic behaviour of the $\Gamma$ function:\begin{equation*}
		\Gamma(n+a)/\Gamma(n)\sim n^a.
	\end{equation*}
\end{proof*}

We are ready to complete the proof of Proposition \ref{prop:1718105}.

\begin{proof*}{Proof of Proposition \ref{prop:1718105}.}
	Iterating the differential formula\begin{equation*}
		\frac{d}{dx}C_n^\lambda(x)=2\lambda C^{\lambda+1}_{n-1}(x),
	\end{equation*}we {get} the following $L^2$-expansion:\begin{equation*}
		f^{(N)}(x)=2^N(\lambda)_N\displaystyle\sum_{n=0}^\infty a_n(f)C_{n-N}^{\lambda+N}(x).
	\end{equation*}Thus, for all $n\ge N$, we have\begin{equation*}
		\myabs{a_n(f)}\le\frac{1}{2^N(\lambda)_N}\frac{\mynorm{f^{(N)}(x)}_{L^2_{\lambda+N}}}{\mynorm{C^{\lambda+N}_{n-N}(x)}_{L^2_{\lambda+N}}}.
	\end{equation*}By Lemma \ref{lem:1718109}\begin{equation*}
		\myabs{a_n(f)}\mynorm{C^\lambda_n(x)}_{L^\infty(-1,1)}\le\frac{d_\lambda}{2^N(\lambda)_N}\mynorm{f^{(N)}(x)}_{L^2_{\lambda+N}}n^{\lambda-N}.
	\end{equation*}Therefore\mygrammarfootnote{comma here?} the right-hand side of \eqref{eqn:aGegen} converges uniformly in $[-1,1]$ because $\lambda-N<-1$.

	For $0\le n<N$, we use\begin{equation*}
		\myabs{a_n(f)}\sqrt{v_n^\lambda}\le\mynorm{f}_{L_\lambda^2}
	\end{equation*}to conclude\begin{equation*}
		\left(\sum_{n=0}^{N-1}+\sum_{n=N}^\infty  \right) a_n(f)\mynorm{C_n^\lambda(x)}_{L^\infty(-1,1)}\le D_\lambda\left( \mynorm{f}_{L^2_\lambda}+\mynorm{f^{(N)}}_{L^2_{\lambda+N}} \right),
	\end{equation*}where we set\begin{equation*}
		D_\lambda:=\max\left( \frac{d_\lambda}{2^N(\lambda)_N}\displaystyle\sum_{n=N}^{\infty}n^{\lambda-N},\left\{ \frac{\mynorm{C_n^\lambda(x)}_{L^\infty(-1,1)}}{\sqrt{v_n^\lambda}} \right\}_{n=0,\cdots,N-1} \right).
	\end{equation*}Hence Proposition \ref{prop:1718105} is proved.
\end{proof*}
\section{Proof of Theorem \ref{thm:1-1}}\label{sec:pfThm}
We are ready to complete the proof of Theorem \ref{thm:1-1}.
The substitution of $x = 1$ in Theorem \ref{main-thm} yields:
\begin{proposition}
  \label{cor:1}{\tmdummy}
  
%%  \begin{multline}
%%    \displaystyle\int_{- 1}^1 \displaystyle\int_{- 1}^1 | s + t |^{2 c-1} (1 - s^2)^{a-1
%%	    } (1 - t^2)^{b-1} C_{\ell}^{a-\frac{1}{2}} (s)
%%	    C_m^{b-\frac{1}{2}} (t) d s d t   \\
%%	    \displaystyle = \frac{(\frac{1}{2}- c)_{\frac{\ell + m}{2}} (- 1)^{\frac{\ell + m}{2}}
%%    \pi^{\frac{1}{2}} (2a-1)_{\ell} (2 b-1)_m \Gamma \left( 
%%    a \right) \Gamma \left( b \right) \Gamma \left(
%%    c \right) \Gamma (a+b+2c-1)}{\ell !m!
%%    \Gamma \left( a+c + \frac{\ell - m}{2}  \right) \Gamma \left(
%%    b+c - \frac{\ell - m}{2}  \right) \Gamma \left( a+b +
%%    c + \frac{\ell + m}{2} -\frac{1}{2} \right)} . 
%%    \end{multline}
    \begin{multline}
    \displaystyle\int_{- 1}^1 \displaystyle\int_{- 1}^1 | s + t |^{2 \nu} (1 - s^2)^{\lambda -
    \frac{1}{2}} (1 - t^2)^{\mu - \frac{1}{2}} C_{\ell}^{\lambda} (s)
    C_m^{\mu} (t) d s d t   =a^{\ell,m}_{\lambda,\mu,\nu}v_\ell^\lambda v^\mu_m\\
    \displaystyle = \frac{(- \nu)_{\frac{\ell + m}{2}} (- 1)^{\frac{\ell + m}{2}}
    \pi^{\frac{1}{2}} (2 \lambda)_{\ell} (2 \mu)_m \Gamma \left( \lambda +
    \frac{1}{2} \right) \Gamma \left( \mu + \frac{1}{2} \right) \Gamma \left(
    \nu + \frac{1}{2} \right) \Gamma (\lambda + \mu + 2 \nu + 1)}{\ell !m!
    \Gamma \left( \lambda + \nu + \frac{\ell - m}{2} + 1 \right) \Gamma \left(
    \mu + \nu - \frac{\ell - m}{2} + 1 \right) \Gamma \left( \lambda + \mu +
    \nu + \frac{\ell + m}{2} + 1 \right)}  \label{eqn:cor:1} .
  \end{multline}
\end{proposition}
%8 star
Owing to Proposition \ref{prop:1718105}, we can deduce
Theorem \ref{thm:1-1}
from Proposition \ref{cor:1}
under a slightly weaker assumption that $\lambda,\mu\in\R$ and $\nu\in\C$ satisfy\begin{equation*}
	\lambda,\mu>-\frac{1}{2};\quad\lambda,\mu\neq0;\quad2\Re \nu>\lambda+\mu+2,
\end{equation*}because for any $m,n\in\N$ with $m\le\lambda+2$ and $n\le\mu+2$, we have\begin{equation*}
	\frac{\partial^{m+n}}{\partial s^m\partial t^n}\myabs{s+t}^{2\nu}\in L^2\left( (-1,1)^2,(1-s^2)^{\lambda+m}(1-t^2)^{\mu+n}dsdt \right).
\end{equation*}
Hence Theorem \ref{thm:1-1} is proved.

\section{Limit case and special values}\label{sec:4}
In this section, we examine the relationship between Theorem
\ref{main-thm} and some known integral formul{\ae} by Selberg, Dotsenko,
Fateev, Tarasov Varchenko, Warnaar among others when the parameters take
special values. The hierarchy of the formul{\ae} treated here is summarized in
Figure \ref{table}.

For this, we limit ourselves to the special case of Theorem \ref{main-thm}
with $(\ell, m, x) = (0, 0, -1)$, or equivalently, of Proposition \ref{cor:1}
with $(\ell, m) = (0, 0)$:
\begin{multline}  \label{eqn:lm0}
   \displaystyle\int_{- 1}^1 \displaystyle\int_{- 1}^1 | s - t |^{2 \nu} (1 - s^2)^{\lambda -
  \frac{1}{2}} (1 - t^2)^{\mu - \frac{1}{2}} d s d t \\
  = \frac{\pi^{\frac{1}{2}}
  \Gamma \left( \lambda + \frac{1}{2} \right) \Gamma \left( \mu + \frac{1}{2}
  \right) \Gamma \left( \nu + \frac{1}{2} \right) \Gamma (\lambda + \mu + 2
  \nu + 1)}{\Gamma (\lambda + \nu + 1) \Gamma (\mu + \nu + 1) \Gamma (\lambda
  + \mu + \nu + 1)} .
\end{multline}
\begin{example}
  \label{ex:1}(Selberg integral {\cite{Selberg:411367}}) The Selberg integral
  \begin{multline}
     \displaystyle\int_0^1 \ldots \displaystyle\int_0^1 \displaystyle\prod_{i = 1}^n t_i^{\alpha - 1} (1 -
    t_i)^{\beta - 1} \left| \displaystyle\prod_{1 \leqslant i < j \leqslant n} (t_i - t_j)
    \right|^{2 \nu} d t_1 \cdots d t_n    \\
     = \displaystyle\prod_{j = 1}^n \frac{\Gamma (\alpha + (j - 1) \nu) \Gamma (\beta +
    (j - 1) \nu) \Gamma (1 + j \nu)}{\Gamma (\alpha + \beta + (n + j - 2) \nu)
    \Gamma (1 + \nu)}   
	 \label{eqn:selberg} 
  \end{multline}
  is a generalization of the Euler beta integral. A special case of
  $(\ref{eqn:selberg})$ with $(n, \alpha, \beta) = \left( 2, \lambda +
  \frac{1}{2}, \lambda + \frac{1}{2} \right)$ says
  \begin{equation}
     \displaystyle\int_{- 1}^1 \displaystyle\int_{- 1}^1 (1 -
    s^2)^{\lambda - \frac{1}{2}} (1 - t^2)^{\lambda - \frac{1}{2}} | s - t
    |^{2 \nu} d s d t 
    = \displaystyle\frac{2^{4\lambda+2\nu}\Gamma \left( \lambda + \frac{1}{2} \right)^2}{\Gamma (2
    \lambda + 1 + \nu)} \cdot \frac{\Gamma \left( \lambda + \nu + \frac{1}{2}
    \right)^2 \Gamma (1 + 2 \nu)}{\Gamma (2 \lambda + 2 \nu + 1) \Gamma (1 +
    \nu)},
    \label{eqn:spec_selberg}
  \end{equation}
  after a change of variables $(t_1, t_2) = \left( \frac{1 + s}{2}, \frac{1 +
  t}{2} \right)$. This coincides with the special case of Theorem
  \ref{main-thm} with $(\ell,m,x,\mu)=(0,0,-1,\lambda)$, namely, $\lambda=\mu$ in 
  \eqref{eqn:lm0}.
\end{example}

\begin{example}
  \label{ex:2}(Warnaar integral) The integral formula (1.4) in
  {\cite{warnaar2010sl3}} in the special case $(k_1, k_2, \alpha_1, \beta_1,
  \alpha_2 \comma \beta_2, \gamma) = \left( 1, 1, \lambda + \frac{1}{2},
  \lambda + \frac{1}{2}, \mu + \frac{1}{2}, \mu + \frac{1}{2} \comma \lambda +
  \mu \right)$ implies the following equation
  \begin{multline}
     \left( \displaystyle\int \displaystyle\int_{0 \leqslant s < t
    \leqslant 1} + \frac{\cos (\pi \lambda)}{\cos (\pi \mu)} \displaystyle\int \displaystyle\int_{0
    \leqslant t < s \leqslant 1} \right) (1 - s^2)^{\lambda - \frac{1}{2}} (1
    - t^2)^{\mu - \frac{1}{2}} | s - t |^{- \lambda - \mu} d s d t
    \\
    \displaystyle= \frac{{2^{\lambda + \mu}}\Gamma \left( \lambda + \frac{1}{2} \right) \Gamma \left(
    \frac{1}{2} - \mu \right) \Gamma \left( \mu + \frac{1}{2}
    \right)^2}{\Gamma (\lambda + 1 - \mu) \Gamma (\mu + 1 - \lambda) \Gamma
    (\lambda + \mu + 1)}.
      \label{eqn:spec_warnaar}
  \end{multline}
  This coincides with the special case of Theorem \ref{main-thm} with $(\ell,
  m, x, \nu) = \left( 0, 0, 1, - \frac{\lambda + \mu}{2} \right)$, namely,
  $\lambda+\mu+2\nu=0$ in
  \eqref{eqn:lm0}.
\end{example}

\begin{example}
  \label{ex:3}($\mathfrak{s}\mathfrak{l}_3$ Selberg integral of Tarasov and
  Varchenko) The integral formula (3.4) in {\cite{tarasov2003selberg}} in the
  special case $(k_1, k_2, \alpha, \beta_1, \beta_2, \gamma) = \left( 1, 1,
  \lambda + \frac{1}{2}, \lambda + \frac{1}{2}, 1, - 2 \nu \right)$ reduces to
  the following equation
  \begin{equation}
    \displaystyle\displaystyle\int_{- 1}^1 \displaystyle\int_{- 1}^1 (1 -
    s^2)^{\lambda - \frac{1}{2}} (t - s)_+^{2 \nu} d s d t = \frac{2^{2\lambda+2\nu+1}\Gamma
    \left( \lambda + \frac{1}{2} \right) \Gamma \left( \frac{3}{2} + \lambda +
2 \nu \right)}{(1 + 2 \nu) \Gamma (2 + 2 \lambda + 2 \nu)} \mbox{.\footnotemark}
\label{eqn:spec_tv}
  \end{equation}
  This coincides with the special case of Theorem \ref{main-thm} with $(\ell,
  m, x, \mu) = \left( 0, 0, 1, \frac{1}{2} \right)$, namely, $\mu=\frac{1}{2}$ in \eqref{eqn:lm0}.
\end{example}

\begin{example}
  \label{ex:4}(Dotsenko-Fateev integral) The integral formula (A1)$=$(A35) in
  {\cite{dotsenko1985four}} in the special case $(n, m, \alpha, \beta, \rho) =
  \left( 1, 1, \mu - \frac{1}{2}, \mu - \frac{1}{2}, - \frac{\mu -
  \frac{1}{2}}{\lambda - \frac{1}{2}} \right)$ reduces to the following
  equation
  \begin{equation}
     \displaystyle\int_{- 1}^1 \displaystyle\displaystyle\int_{- 1}^1 (1 - s^2)^{\lambda
    - \frac{1}{2}} (1 - t^2)^{\mu - \frac{1}{2}} | s - t |^{- 2} d s d t
    = \frac{ 2^{2\lambda+2\mu-1} \Gamma \left( \lambda + \frac{1}{2} \right)^2 \Gamma \left(
    \mu + \frac{1}{2} \right)^2}{(1-\lambda - \mu ) \Gamma (2 \lambda) \Gamma
    (2 \mu)} .
    \label{eqn:spec_df}
  \end{equation}
  This coincides with the special case of Theorem \ref{main-thm} with $(\ell,
  m, x, \nu) = (0, 0, 1, - 1)$, namely, $\nu=-1$ in \eqref{eqn:lm0}.
\end{example}

The hierarchy of the integral formul{\ae} in Examples \ref{ex:1}-\ref{ex:4}
and Theorem \ref{main-thm} is summarized as follows:\footnote{should I include
the following in the diagram: Corollary \ref{cor:Hermite} (and its relation
with the Mehta integral); relation with the results of
{\cite{kobayashi2011schrodinger}}?}
\begin{figure}[h]
	\begin{dot2tex}[mathmode,dot,scale=0.55]//fdp
  digraph G {
	  {rank=same War10 DF85 TV03 KLg}
	  {rank=same KL S}
	  {rank=same TV03p War10pp Spp DF85pp}
	  War10 -> S [label="k_2=0"];
	  DF85 -> S [label=" ",texlbl="$\begin{array}[]{c}p=n,\\m=0\end{array}$"];
	  S -> Spp [label=" ", texlbl="$\begin{array}[]{c}k=2,\\\alpha=\beta\end{array}$"]
	  TV03 -> S [label=" ",texlbl="$k_1=0$"]
	  War10 -> War10pp [label=" ",texlbl="$\begin{array}[]{c}k_1=k_2=1,\\\alpha_1=\beta_1,\\\alpha_2=\beta_2\end{array}\kern2cm$"];
	  KL -> Spp  [label=" ", texlbl="$\begin{array}[]{c}\\\\A=B\end{array}$"]
	  KL -> War10pp[label=" ",texlbl="$\kern5emA+B=\nu+1$"];
	  TV03 -> TV03p [label=" ",texlbl="$\kern2cm\begin{array}[]{c}k_1=k_2,\\ \alpha_1=1,\\\alpha_2=\beta_2\end{array}$"];
	  KL -> TV03p [label=" ",texlbl="$\kern-1.5cm A=0$"];
	  DF85 -> DF85pp [label=" ",texlbl="\kern2cm$\begin{array}[]{c}m=n=1,\\ \alpha'=\beta',\\\alpha=\beta\end{array}$"];
	  KL -> DF85pp [label=" ",texlbl="$\nu=2$"];
	  KLg -> KL [label=" ",texlbl="$\begin{array}[]{c}x=1,\\\ell=m=0\end{array}$"];

    KLg [shape="box",label="{\mbox{Thm.~\ref{main-thm}}}:3"];
    KL [shape="box",label="{\mbox{Cor.~\ref{cor:1}}}:3"];
    S [shape="box",label="\mbox{\cite{Selberg:411367}}:3"];
    Spp [shape="box",label="\mbox{\eqref{eqn:spec_selberg}}:2"]
    War10 [shape="box",label="\mbox{\cite{warnaar2010sl3}}:4"];
    DF85 [shape="box",label="\mbox{\cite{dotsenko1985four}}:3"];
    TV03 [shape="box",label="\mbox{\cite{tarasov2003selberg}}:4"];
    War10pp [shape="box",label="\mbox{\eqref{eqn:spec_warnaar}}:2"];
    TV03p [shape="box",label="\mbox{\eqref{eqn:spec_tv}}:2"]
    DF85pp [shape="box",label="\mbox{\eqref{eqn:spec_df}}:2"]
    }
\end{dot2tex}

\caption{Hierarchy of various integral formul{\ae}\label{table}}
\end{figure}
\section{Limiting case}\label{sec:limit}
In this section we discuss the limiting case of our integral formula.
%3 star star
Taking the limit in $(\ref{eqn:cor:1})$ as both $\lambda$ and
$\mu$ tend to be
zero, we obtain
\begin{corollary}
  \label{cor:170599}For $\rho \in \mathbbm{C}$ with $\tmop{Re} \rho > 0$ and
  $r \in \{ 0, 1 \}$,
  \begin{multline}
     | \cos \varphi + \cos \psi |^{\rho} \tmop{sgn}^r (\cos \varphi + \cos
    \psi)   \\
     = \displaystyle\sum_{\tmscript{\begin{array}{c}
      \ell, m = 0\\
      \ell \equiv m + r \tmop{mod} 2
    \end{array}}}^{\infty} \frac{2^{2 - \rho} \Gamma (\rho + 1)^2}{(1 +
    \delta_0^{\ell}) (1 + \delta_0^m) \displaystyle\prod_{\delta, \varepsilon \in \{ \pm 1
    \}} \Gamma \left( 1 + \frac{1}{2} (\rho + \delta \ell + \varepsilon m)
    \right)} \cos \ell \varphi \cos m \psi . 
  \end{multline}
  where
  $\delta_0^\ell$ denotes the Kronecker delta, i.e.,
  $\delta_0^{\ell} \assign 1$ if $\ell = 0$ and $= 0$ otherwise.
\end{corollary}
%10 star
Since the Hermite polynomial $H_n (x)$ is obtained as a limit of the
Gegenbauer polynomial:
\begin{eqnarray}
  & H_n (x) = n! \displaystyle\lim_{\lambda \rightarrow \infty} \lambda^{- \frac{n}{2}}
  C_n^{\lambda} \left( \displaystyle\frac{x}{\sqrt[]{\lambda}} \right), &  \nonumber
\end{eqnarray}
we can deduce the following integral formula of the Hermite polynomial from
Proposition \ref{cor:1}:

\begin{corollary}
  \label{cor:Hermite}Suppose $\ell, m \in \mathbbm{N}$ with $\ell + m$
  even.
  \begin{multline}
     \displaystyle\int_{- \infty}^{\infty} \displaystyle\int_{- \infty}^{\infty} | s+xt |^{2 \nu}
    e^{- s^2 - t^2} H_{\ell} (s) H_m (t) d s d t \\= (- \nu)_{\frac{\ell +
    m}{2}} (- 1)^{\frac{\ell + m}{2}} 2^{\ell + m} \pi^{\frac{1}{2}} \Gamma
    \left( \frac{1}{2} + \nu \right) (x^2 + 1)^{\nu - \frac{\ell + m}{2}} x^m.
    \end{multline}
\end{corollary}

\begin{example}
  (Mehta integral {\cite{mehta2004random}}) The Mehta integral
  \begin{equation*}
     \frac{1}{(2 \pi)^{n / 2}} \displaystyle\int_{\mathbbm{R}^n} \displaystyle\prod_{i = 1}^n e^{-
    t_i^2 / 2}_{} \displaystyle\prod_{1 \leqslant i < j \leqslant n} | t_i - t_j |^{2 \nu}
    d t_1 \cdots d t_n
     = \displaystyle\prod_{j = 1}^n \frac{\Gamma (1 + j \nu)}{\Gamma (1 + \nu)}
  \end{equation*}
  in special case $n = 2$ implies the following equation
  \begin{eqnarray}
    & \displaystyle\frac{1}{2 \pi} \displaystyle\int_{- \infty}^{\infty} \displaystyle\int_{- \infty}^{\infty} e^{-
    \frac{x^2 + y^2}{2}} | x - y |^{2 \nu} d x d y = \frac{\Gamma (1 + 2
    \nu)}{\Gamma (1 + \nu)} . &  \nonumber\\
    &  &  \nonumber
  \end{eqnarray}
  This coincides with the special case of Corollary \ref{cor:Hermite} with
  $(\ell, m, x) = (0, 0, -1)$.
\end{example}


{\bf Acknowledgement.} The first author was partially supported by the Grant-in-Aid for Scientific Research (A) 25247006.
\begin{thebibliography}{CK{\O}P11}
\expandafter\ifx\csname urlstyle\endcsname\relax
  \providecommand{\doi}[1]{doi:\discretionary{}{}{}#1}\else
  \providecommand{\doi}{doi:\discretionary{}{}{}\begingroup
  \urlstyle{rm}\Url}\fi

\bibitem[BR04]{bernstein2004estimates}
J.~Bernstein and A.~Reznikov.
\newblock Estimates of automorphic functions.
\newblock \emph{Mosc. Math. J}, \textbf{\textbf{4}}(1), (2004), pp. 19--37.
Available at \url{http://mi.mathnet.ru/eng/mmj141}.

\bibitem[CK{\O}P11]{clerc2011generalized}
J.-L. Clerc, T.~Kobayashi, B.~{\O}rsted and M.~Pevzner.
\newblock Generalized {B}ernstein--{R}eznikov integrals.
\newblock \emph{Math. Ann.}, \textbf{349}(2), (2011), pp. 395--431.
Available at \url{https://doi.org/10.1007/s00208-010-0516-4}.

\bibitem[HT93]{howe1993homogeneous}
R.~E. Howe and E.-C. Tan.
\newblock Homogeneous functions on light cones: the infinitesimal structure of
  some degenerate principal series representations.
\newblock \emph{Bull. Amer. Math. Soc.}, \textbf{28}(1), (1993), pp. 1--74.
Available at \url{http://www.ams.org/journals/bull/1993-28-01/S0273-0979-1993-00360-4/S0273-0979-1993-00360-4.pdf}.

\bibitem[J09]{juhl2009families}
A.~Juhl.
\newblock \emph{Families of {C}onformally {C}ovariant {D}ifferential
  {O}perators, {Q}-curvature and {H}olography}, \emph{Progr. Math,}
  \textbf{275}.
\newblock Springer Science \& Business Media (2009).
Available at \url{http://www.springer.com/in/book/9783764398996}.

\bibitem[K15]{kobayashi2015program}
T.~Kobayashi.
\newblock A program for branching problems in the representation theory of real
  reductive groups.
\newblock \emph{Progr. Math.}, \textbf{312}, (2015), pp. 277--322.
\newblock In: \emph{{\normalfont Special issue in honor of Vogan's 60th years
  birthday}}.
Available at \url{https://doi.org/10.1007/978-3-319-23443-4_10}.

\bibitem[K{\O}03]{KO2}
T.~Kobayashi and B.~{\O}rsted.
\newblock Analysis on the minimal representation of\/ {$\mbox{\rm O}(p,q)$}.{$\;$}{{\rm{II}}}. {B}ranching laws.
\newblock \emph{Adv. Math.}, \textbf{180}(2), (2003), pp. 513--550.
Available at \url{https://doi.org/10.1016/S0001-8708(03)00013-6}.

\bibitem[KO13]{kobayashi2013finite}
T.~Kobayashi and T.~Oshima.
\newblock Finite multiplicity theorems for induction and restriction.
\newblock \emph{Adv. Math.}, \textbf{248}, (2013), pp. 921--944.
Available at \url{http://dx.doi.org/10.1016/j.aim.2013.07.015}.

\bibitem[K93]{kobayashi1993}
T.~Kobayashi.
\newblock The restriction of ${A}_q \left( \lambda \right)$ to reductive
  subgroups.
\newblock \emph{Proc. Japan Acad. Ser. A Math. Sci.}, \textbf{69}(7), (1993),
  pp. 262--267.
Available at \url{http://dx.doi.org/10.3792/pjaa.69.262}.

\bibitem[K98]{10.2307/120963}
T.~Kobayashi.
\newblock Discrete decomposability of the restriction of ${A}_q(\lambda)$ with
  respect to reductive subgroups {I}{I}: Micro-local analysis and asymptotic
  {K}-support.
\newblock \emph{Annals of Mathematics}, \textbf{147}(3), (1998), pp. 709--729.
Available at \url{http://dx.doi.org/10.2307/120963}.

\bibitem[K14]{KOBAYASHI2014272}
T.~Kobayashi.
\newblock F-method for symmetry breaking operators.
\newblock \emph{Differential Geometry and its Applications}, \textbf{33},
  (2014), pp. 272 -- 289.
Available at \url{http://dx.doi.org/10.1016/j.difgeo.2013.10.003}.

\bibitem[K16]{kobayashi16birth}
T.~Kobayashi.
\newblock \emph{Birth of new branching problems}.
\newblock
  日本数学会70年記念 総合講演・企業特別講演アブストラクト, pp. 65--92,
  日本数学会, 2016.
Available at \url{http://www.ms.u-tokyo.ac.jp/~toshi/texpdf/tk2016p-msj70.pdf}.

\bibitem[K{\O}SS15]{kobayashi2015branching}
T.~Kobayashi, B.~{\O}rsted, P.~Somberg and V.~Sou{\v{c}}ek.
\newblock Branching laws for verma modules and applications in parabolic
  geometry. {I}.
\newblock \emph{Adv. Math.}, \textbf{285}, (2015), pp. 1796--1852.
Available at \url{http://dx.doi.org/10.1016/j.aim.2015.08.020}.

\bibitem[KP16a]{kobayashi2016differential1}
T.~Kobayashi and M.~Pevzner.
\newblock Differential symmetry breaking operators: I. {G}eneral theory and
  {F}-method.
\newblock \emph{Selecta Mathematica}, \textbf{22}(2), (2016), pp. 801--845.
Available at \url{http://dx.doi.org/10.1007/s00029-015-0207-9}.

\bibitem[KP16b]{Kobayashi2016}
T.~Kobayashi and M.~Pevzner.
\newblock Differential symmetry breaking operators: {I}{I}. {R}ankin--{C}ohen
  operators for symmetric pairs.
\newblock \emph{Selecta Mathematica}, \textbf{22}(2), (2016), pp. 847--911.
Available at \url{http://dx.doi.org/10.1007/s00029-015-0208-8}.

\bibitem[KS15]{kobayashi2015symmetry}
T.~Kobayashi and B.~Speh.
\newblock \emph{Symmetry {B}reaking for {R}epresentations of {R}ank {O}ne
  {O}rthogonal {G}roups}, \emph{Memoirs of the Amer. Math. Soc,} \textbf{238}.
\newblock American Mathematical Society (2015).
Available at \url{http://dx.doi.org/10.1090/memo/1126}.

\end{thebibliography}


\end{document}
