
NIntegrate::slwcon: Numerical integration converging too slowly; suspect one of the following: singularity, value of the integration is 0, highly oscillatory integrand, or WorkingPrecision too small.

NIntegrate::slwcon: Numerical integration converging too slowly; suspect one of the following: singularity, value of the integration is 0, highly oscillatory integrand, or WorkingPrecision too small.

NIntegrate::slwcon: Numerical integration converging too slowly; suspect one of the following: singularity, value of the integration is 0, highly oscillatory integrand, or WorkingPrecision too small.

General::stop: Further output of NIntegrate::slwcon will be suppressed during this calculation.

                                                                                                                                                                                                                                                                                                                                                                                                                                                                                                                                                                                          -6             -76                 -12
NIntegrate::eincr: The global error of the strategy GlobalAdaptive has increased more than 2000 times. The global error is expected to decrease monotonically after a number of integrand evaluations. Suspect one of the following: the working precision is insufficient for the specified precision goal; the integrand is highly oscillatory or it is not a (piecewise) smooth function; or the true value of the integral is 0. Increasing the value of the GlobalAdaptive option MaxErrorIncreases might lead to a convergent numerical integration. NIntegrate obtained -2.86147 10   + 5.47256 10    I and 3.32536 10    for the integral and error estimates.

                                                                                                                                                                                                                                                                                                                                                                                                                                                                                                                                                                                         -9            -77                 -12
NIntegrate::eincr: The global error of the strategy GlobalAdaptive has increased more than 2000 times. The global error is expected to decrease monotonically after a number of integrand evaluations. Suspect one of the following: the working precision is insufficient for the specified precision goal; the integrand is highly oscillatory or it is not a (piecewise) smooth function; or the true value of the integral is 0. Increasing the value of the GlobalAdaptive option MaxErrorIncreases might lead to a convergent numerical integration. NIntegrate obtained 4.41704 10   - 3.6939 10    I and 2.50805 10    for the integral and error estimates.

                                                                                                                                                                                                                                                                                                                                                                                                                                                                                                                                                                                         -8             -77                 -12
NIntegrate::eincr: The global error of the strategy GlobalAdaptive has increased more than 2000 times. The global error is expected to decrease monotonically after a number of integrand evaluations. Suspect one of the following: the working precision is insufficient for the specified precision goal; the integrand is highly oscillatory or it is not a (piecewise) smooth function; or the true value of the integral is 0. Increasing the value of the GlobalAdaptive option MaxErrorIncreases might lead to a convergent numerical integration. NIntegrate obtained 3.87685 10   + 1.86149 10    I and 2.80757 10    for the integral and error estimates.

General::stop: Further output of NIntegrate::eincr will be suppressed during this calculation.
%% AMS-LaTeX Created with the Wolfram Language : www.wolfram.com

\documentclass{article}
\usepackage{amsmath, amssymb, graphics, setspace}

\newcommand{\mathsym}[1]{{}}
\newcommand{\unicode}[1]{{}}

\newcounter{mathematicapage}
\begin{document}

\[0.000550565\]

\end{document}

%% AMS-LaTeX Created with the Wolfram Language : www.wolfram.com

\documentclass{article}
\usepackage{amsmath, amssymb, graphics, setspace}

\newcommand{\mathsym}[1]{{}}
\newcommand{\unicode}[1]{{}}

\newcounter{mathematicapage}
\begin{document}

\[\int _{-1}^1\int _{-1}^1\frac{2^{-2+2 \lambda +2 \mu } \left(1-s^2\right)^{-\frac{1}{2}+\lambda } \left(1-t^2\right)^{-\frac{1}{2}+\mu } \text{Abs}[s-t
z]^{2 \nu } l! m! \text{Gamma}[\lambda ] \text{Gamma}[\mu ] \text{GegenbauerC}[l,\lambda ,s] \text{GegenbauerC}[m,\mu ,t]}{\text{Gamma}[l+2 \lambda
] \text{Gamma}[m+2 \mu ]}dtds\]

\end{document}

%% AMS-LaTeX Created with the Wolfram Language : www.wolfram.com

\documentclass{article}
\usepackage{amsmath, amssymb, graphics, setspace}

\newcommand{\mathsym}[1]{{}}
\newcommand{\unicode}[1]{{}}

\newcounter{mathematicapage}
\begin{document}

\[\frac{i^{l-m} \pi ^{3/2} z^m \text{Gamma}\left[\frac{1}{2}+\nu \right] \text{Hypergeometric2F1}\left[\frac{l+m}{2}-\nu ,\frac{1}{2} (-l+m)-\lambda
-\nu ,1+m+\mu ,z^2\right] \text{Pochhammer}\left[-\nu ,\frac{l+m}{2}\right]}{\text{Gamma}[1+m+\mu ] \text{Gamma}\left[1+\frac{l-m}{2}+\lambda +\nu
\right]}\]

\end{document}

%% AMS-LaTeX Created with the Wolfram Language : www.wolfram.com

\documentclass{article}
\usepackage{amsmath, amssymb, graphics, setspace}

\newcommand{\mathsym}[1]{{}}
\newcommand{\unicode}[1]{{}}

\newcounter{mathematicapage}
\begin{document}

\[\text{[1.45, 1.5]}\]

\end{document}

%% AMS-LaTeX Created with the Wolfram Language : www.wolfram.com

\documentclass{article}
\usepackage{amsmath, amssymb, graphics, setspace}

\newcommand{\mathsym}[1]{{}}
\newcommand{\unicode}[1]{{}}

\newcounter{mathematicapage}
\begin{document}

\[6\]

\end{document}

%% AMS-LaTeX Created with the Wolfram Language : www.wolfram.com

\documentclass{article}
\usepackage{amsmath, amssymb, graphics, setspace}

\newcommand{\mathsym}[1]{{}}
\newcommand{\unicode}[1]{{}}

\newcounter{mathematicapage}
\begin{document}

\[\{6,7,8\}\]

\end{document}

%% AMS-LaTeX Created with the Wolfram Language : www.wolfram.com

\documentclass{article}
\usepackage{amsmath, amssymb, graphics, setspace}

\newcommand{\mathsym}[1]{{}}
\newcommand{\unicode}[1]{{}}

\newcounter{mathematicapage}
\begin{document}

\[\text{[0, 1]}\]

\end{document}

%% AMS-LaTeX Created with the Wolfram Language : www.wolfram.com

\documentclass{article}
\usepackage{amsmath, amssymb, graphics, setspace}

\newcommand{\mathsym}[1]{{}}
\newcommand{\unicode}[1]{{}}

\newcounter{mathematicapage}
\begin{document}

\[\text{[0, 10]}\]

\end{document}

%% AMS-LaTeX Created with the Wolfram Language : www.wolfram.com

\documentclass{article}
\usepackage{amsmath, amssymb, graphics, setspace}

\newcommand{\mathsym}[1]{{}}
\newcommand{\unicode}[1]{{}}

\newcounter{mathematicapage}
\begin{document}

\[\text{[1.45, 1.475]}\]

\end{document}

%% AMS-LaTeX Created with the Wolfram Language : www.wolfram.com

\documentclass{article}
\usepackage{amsmath, amssymb, graphics, setspace}

\newcommand{\mathsym}[1]{{}}
\newcommand{\unicode}[1]{{}}

\newcounter{mathematicapage}
\begin{document}

\[\text{[1.475, 1.5]}\]

\end{document}

%% AMS-LaTeX Created with the Wolfram Language : www.wolfram.com

\documentclass{article}
\usepackage{amsmath, amssymb, graphics, setspace}

\newcommand{\mathsym}[1]{{}}
\newcommand{\unicode}[1]{{}}

\newcounter{mathematicapage}
\begin{document}

\[\text{[1.5, 1.525]}\]

\end{document}

