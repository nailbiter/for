\documentclass[12pt]{article} % use larger type; default would be 10pt

\usepackage{mathtext}                 % підключення кирилиці у математичних формулах
                                          % (mathtext.sty входить в пакет t2).
\usepackage[T1,T2A]{fontenc}         % внутрішнє кодування шрифтів (може бути декілька);
                                          % вказане останнім діє по замовчуванню;
                                          % кириличне має співпадати з заданим в ukrhyph.tex.
\usepackage[utf8]{inputenc}       % кодування документа; замість cp866nav
                                          % може бути cp1251, koi8-u, macukr, iso88595, utf8.
\usepackage[english,russian,ukrainian]{babel} % національна локалізація; може бути декілька
                                          % мов; остання з переліку діє по замовчуванню. 
\usepackage{mystyle}

\newtheorem{prob}{Завдання}
\newcommand{\ds}{\;ds}
\newcommand{\dt}{\;dt}
\newcommand{\dx}{\;dx}
\newcommand{\dta}{\;d\tau}
\newcommand{\extr}{\mbox{\normalfont extr}}

\newtheorem{myulem}[mythm]{Лема}

\renewenvironment{myproof}[1][Доведення]{\begin{trivlist}
\item[\hskip \labelsep {\bfseries #1}]}{\myqed\end{trivlist}}

\title{Рівняння математичної фізики (10 семестр)}
\author{Олексій Леонтьєв}

\begin{document}
\def\dx{\Delta x}
\def\dt{\Delta t}
\def\dX{\frac{\partial}{\partial x}}
\maketitle
\begin{prob}Звести до канонічного вигляду запропоноване рівняння
	\[u_{xx}+2u_{xy}-2u_{xz}-4u_{yz}+2u_{yt}+u_{zz}=0\]
\end{prob}
Це рівняння від чотирьох змінних і для початку ми запишемо його в матричній формі
\newcommand{\px}{\frac{\partial}{\partial x}}
\newcommand{\py}{\frac{\partial}{\partial y}}
\newcommand{\pz}{\frac{\partial}{\partial z}}
\newcommand{\pt}{\frac{\partial}{\partial t}}
\[\begin{bmatrix}\px&\py&\pz&\pt
\end{bmatrix}\begin{bmatrix}1&1&-1&0\\1&0&-2&1\\-1&-2&1&0\\0&1&0&0\end{bmatrix}\begin{bmatrix}\px\\\py\\\pz\\\pt\end{bmatrix}u=0\]
Далі ця задача еквівалентна задачі діагоналізації квадратичної форми. Достатньо знайти власні числа і відповідні власні вектори матриці:
\[p(\lambda)=\det\mybra{\begin{bmatrix}1&1&-1&0\\1&0&-2&1\\-1&-2&1&0\\0&1&0&0\end{bmatrix}-\lambda I}=
\begin{vmatrix}1-\lambda&1&-1&0\\1&-\lambda&-2&1\\-1&-2&1-\lambda&0\\0&1&0&-\lambda\end{vmatrix}=0\]
\[0=\begin{vmatrix}1-\lambda&1&-1&0\\1&-\lambda&-2&1\\-1&-2&1-\lambda&0\\0&1&0&-\lambda\end{vmatrix}=-\lambda\cdot\begin{vmatrix}1-\lambda&1&-1\\
1&-\lambda&-2\\-1&-2&1-\lambda\end{vmatrix}+1\cdot\begin{vmatrix}1-\lambda&1&-1\\-1&-2&1-\lambda\\0&1&0\end{vmatrix}=\]
\[=-\lambda\mysbra{-\lambda(1-\lambda)^2+\cancel{2}+\cancel{2}+\lambda-4(\cancel{1}-\lambda)-(1-\lambda)}-\mysbra{(1-\lambda)^2-1)}=\]
\[=-\lambda\mysbra{(-\lambda^3+2\lambda^2+5\lambda-1)+(\lambda-2)}=-\lambda(-\lambda^3+2\lambda^2+6\lambda-3)\]
Власними числами є $\lambda=0,\;\lambda=-1.92542,\;\lambda=0.448071,\;\lambda=3.47735$ із відповідними нормованими власними векторами
\[\begin{bmatrix}\frac{1}{\sqrt{3}}&0&\frac{1}{\sqrt{3}}&\frac{1}{\sqrt{3}}\end{bmatrix}^T\]
\[\begin{bmatrix}0.0940008358887456&-0.767721407064694&-0.492729757029737&0.398729319870311\end{bmatrix}^T\]
\[\begin{bmatrix}-0.64761477077145&0.310989432774467&-0.0464483513012581&0.69406284444757\end{bmatrix}^T\]
\[\begin{bmatrix}0.488288727356506&0.56025776763911&-0.649405027883951&0.161116300527445\end{bmatrix}^T\]
Таким чином, роблячи лінійну заміну
\[\begin{bmatrix}a\\b\\c\\d\end{bmatrix}=\]\[=
\begin{bmatrix}\frac{1}{\sqrt{3}}&0&\frac{1}{\sqrt{3}}&\frac{1}{\sqrt{3}}\\
0.0940008358887456&-0.767721407064694&-0.492729757029737&0.398729319870311\\
-0.64761477077145&0.310989432774467&-0.0464483513012581&0.69406284444757\\
0.488288727356506&0.56025776763911&-0.649405027883951&0.161116300527445\end{bmatrix}\begin{bmatrix}x\\y\\z\\t\end{bmatrix}\]
Рівняння перетворюється на
\[-u_{bb}+u_{cc}+u_{dd}=0\]
Це рівняння параболічного типу.
\begin{prob}Звести рівняння до канонічного вигляду в кожній з областей, де зберігається його тип
	\[xyu_{xx}+u_{yy}=0,\;x<0,\;y>0\]
\end{prob}
Це рівняння гіперболічного типу, оскільки в заданій області $xy<0$. Характеристичне рівняння має вигляд
\[xy(dy)^2+(dx)^2=0\]
\[dx/dy=\pm\sqrt{-xy}\]
\[\sqrt{-x}=\pm\frac{1}{3}y^{\mysfrac{3}{2}}+C\]
Робимо, таким чином, заміну змінних
\[\xi=\sqrt{y}+\frac{1}{3}(-x)^{\mysfrac{3}{2}}\]
\[\eta=\sqrt{y}-\frac{1}{3}(-x)^{\mysfrac{3}{2}}\]
%\[y=\mybra{\frac{\xi+\eta}{2}}^2,\;x=-\mybra{\frac{\xi-\eta}{2}}^{\mysfrac{2}{3}}\]
\[u_x=u_\xi\xi_x+u_\eta\eta_x=u_\xi\mybra{-\frac{1}{2}\sqrt{-x}}+u_\eta\mybra{\frac{1}{2}\sqrt{-x}}\]
\[u_y=u_\xi\xi_y+u_\eta\eta_y=u_\xi\frac{1}{2\sqrt{y}}+u_\eta\frac{1}{2\sqrt{y}}\]
\[u_{xx}=-\frac{1}{4\sqrt{-x}}\mybra{u_\eta-u_\xi}+\frac{1}{2}\sqrt{-x}\mybra{u_{\eta\xi}\xi_x+u_{\eta\eta}\eta_x-
u_{\xi\xi}\xi_x-u_{\xi\eta}\eta_x}=\]\[=-\frac{1}{4\sqrt{-x}}\mybra{u_\eta-u_\xi}-\frac{x}{4}\mybra{\cancel{-u_{\eta\xi}}
+u_{\eta\eta}+u_{\xi\xi}+\cancel{u_{\xi\eta}}}\]
\[u_{yy}=-\frac{1}{4}y^{-\mysfrac{3}{2}}(u_\xi+u_\eta)+\frac{1}{2\sqrt{y}}
\mybra{u_{\eta\xi}\xi_y+u_{\eta\eta}\eta_y+u_{\xi\xi}\xi_y+u_{\xi\eta}\eta_y}=\]\[=-\frac{1}{4}y^{-\mysfrac{3}{2}}(u_\xi+u_\eta)+\frac{1}{4{y}}
\mybra{{u_{\eta\xi}}+u_{\eta\eta}+u_{\xi\xi}+{u_{\xi\eta}}}\]
Підставляючи це в рівняння, маємо
\end{document}
