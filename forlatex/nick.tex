\documentclass[8pt]{article} % use larger type; default would be 10pt

\usepackage{textcomp} %for copyleft symbol
\usepackage{mathtext}                 % підключення кирилиці у математичних формулах
                                          % (mathtext.sty входить в пакет t2).
\usepackage[T1,T2A]{fontenc}         % внутрішнє кодування шрифтів (може бути декілька);
                                          % вказане останнім діє по замовчуванню;
                                          % кириличне має співпадати з заданим в ukrhyph.tex.
\usepackage[utf8]{inputenc}       % кодування документа; замість cp866nav
                                          % може бути cp1251, koi8-u, macukr, iso88595, utf8.
\usepackage[english,ukrainian]{babel} % національна локалізація; може бути декілька
%\usepackage{CJK}
\usepackage{graphicx}
\usepackage{float}
\usepackage{CJKutf8}
\usepackage{subfig}
\usepackage{amsmath}
\usepackage{amssymb}
\usepackage{amsthm}
\usepackage{amsfonts}
\usepackage{hyperref}
\usepackage{enumerate}
\usepackage{enumitem}

%put subscript under lim and others
\let\oldlim\lim
\renewcommand{\lim}{\displaystyle\oldlim}
\let\oldmin\min
\renewcommand{\min}{\displaystyle\oldmin}
\let\oldmax\max
\renewcommand{\max}{\displaystyle\oldmax}

\newtheorem*{prob}{Завдання}

\usepackage{mystyle}

\title{}
\begin{document}
\maketitle
\begin{prob}4\end{prob}
	Оскільки і крива, і поверхня проходять через початок координат, порядок дотику принаймні нульовий.
	Далі, дотична площина до поверхні в початку
	координат розраховується як
	\[0=d(xz^2-xz+y-y^2)\bigg|_{(0,0,0)}=dy\]
	Таким чином, дотична площина має рівняння $y=0$. Відповідно, дотична лінія до кривої має рівняння $x=y=0$ і лежить в цій площині -- 
	порядок дотику принаймні перший. Спільна дотична площина це $y=0$.

	Загальна точка на кривій має рівняння $(t^3,t^4,t)$, її проекція на $y=0$ $(t^3,0,t)$ і пряма, що проходить через ці точки перетинає
	криву в точці $(t^3,y,t)$, де $t^5-t^4+y-y^2=0\implies y=\frac{1-\sqrt{1-4(t^4-t^5)}}{2}$ (адже $y\to0$ при $t\to0$, тому з $\pm$ вибрали
	$-$). Відповідно, маємо
	\[d(t)\frac{1-\sqrt{1-4(t^4-t^5)}}{2}-t^4=\frac{1}{2}\frac{(1-2t^4)^2-(1-4(t^4-t^5)}{(1-2t^4)^2+(1-4(t^4-t^5)}=\]
	\[=\frac{1}{2}\frac{4t^8+4t^5}{(1-t^4)^2+(1-4(t^4-t^5))}=O(t^5)\]
	Але $t$ не є натуральною параметризацією кривої. Її довжина
	\[s(t)=L(0,t)=\int_0^t\sqrt{(3t^2)^2+(4t^3)^2+1^2}\;dt=O(t^4)\]
	і тому $d(s)=O(t^5)=O(s^{5/4})$ і ми маємо дотик лише першого порядку.
\begin{prob}8\end{prob}
	\[L=\int_{-1}^1\sqrt{\mysbra{\frac{d}{du}\mybra{(-1)\sin u}}^2+
	\mysbra{\frac{d}{du}\mybra{(-1)\cos u}}^2+
	\mysbra{\frac{d}{du}\mybra{-u}}^2}\;du=\]
	\[=\int_{-1}^1\sqrt{\cos^2u+\sin^2u+1}\;du=2\sqrt{2}\]
\begin{prob}9\end{prob}
	$u+4v=0\iff u=-4t,\;v=t$ і $u-3v=0\iff u=3t,\;v=t$. Точка перетину (0,0) і нам лише необхідно знайти кут $\alpha$ між
	(-4,1) та (3,1)
	\[\cos\alpha=\frac{(-4,1)\cdot(3,1)}{\mynorm{(-4,1)}\mynorm{(3,1)}}=\frac{-12+1}{\sqrt{17}\sqrt{10}}\]
\begin{prob}12\end{prob}
	По-перше, розрахуємо координати нормального вектора до поверхні в її загальній точці $(v\sin u,v\cos u,-u)$:
	\[\begin{vmatrix}i&j&k\\
		\frac{d}{du}(v\sin u) & 
		\frac{d}{du}(v\cos u) & 
		\frac{d}{du}(-u) \\
		\frac{d}{dv}(v\sin u) & 
		\frac{d}{dv}(v\cos u) & 
		\frac{d}{dv}(-u) \\
	\end{vmatrix}=(\cos u,-\sin u,v)=:N(u,v)\]
	Таким чином, якщо $(a,b,c)$ вектор -- довжина його проекції на дотичну площину поверхні в точці  $(v\sin u,v\cos u,-u)$
	запишеться як
	\[P(a,b,c;u,v):=\frac{\mynorm{(a,b,c)\times N(u,v)}}{\mynorm{N(u,v)}}\]
	де $\times$ позначає векторний добуток.

	Далі, візьмемо, наприклад, координатну лінію $f_1(u):=(a\sin u, a\cos u,-u)$. Її довжина
	\[s(u)=L(0,u)=\int_0^u\sqrt{(a\cos u)^2+(-a\sin u)^2+(-1)^2}\;du=u\sqrt{a^2+1}\]
	Таким чином, натуральна параметризація
	\[f_1(s)=\mybra{a\sin\frac{s}{\sqrt{a^2+1}},a\cos\frac{s}{\sqrt{a^2+1}},-\frac{s}{\sqrt{a^2+1}}}\]
	Друга похідна
	\[f_1''(s)=\mybra{\frac{-a}{a^2+1}\sin\frac{s}{\sqrt{a^2+1}},\frac{-a}{a^2+1}\cos\frac{s}{\sqrt{a^2+1}},0}\]
	Геодезична кривина, відповідно
	\[P\mybra{\frac{-a}{a^2+1}\sin\frac{s}{\sqrt{a^2+1}},\frac{-a}{a^2+1}\cos\frac{s}{\sqrt{a^2+1}},0;u,v}=\dots\]
\end{document}
