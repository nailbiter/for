%stoic ethic
\documentclass[14pt]{extarticle} % use larger type; default would be 10pt
\usepackage{fontspec}
\usepackage{array, xcolor, lipsum, bibentry}
\usepackage[margin=3cm]{geometry}
\usepackage{hyperref}
\usepackage{fancyhdr}
%\usepackage[T1,T2A]{fontenc}         % внутрішнє кодування шрифтів (може бути декілька);
%\usepackage[utf8]{inputenc}       % кодування документа; замість cp866nav
\usepackage[english,russian,ukrainian]{babel} % національна локалізація; може бути декілька

 
\title{\bfseries\Huge Oleksii Leontiev}
\author{inp9822058@cs.nctu.edu.tw}
\date{}
 
\definecolor{lightgray}{gray}{0.8}
\newcolumntype{L}{>{\raggedleft}p{0.2\textwidth}}
\newcolumntype{R}{p{0.8\textwidth}}
\newcommand\VRule{\color{lightgray}\vrule width 0.5pt}
 
%font configuration
\defaultfontfeatures{Mapping=tex-text}
\setromanfont[Ligatures={Common}, Numbers={OldStyle}, Variant=01]{Times New Roman} % Main text font
\chardef\&="E050 % Custom ampersand character

\begin{document}
\begin{titlepage}
	\addtolength{\voffset}{-2cm}
	%\setlength{\footskip}{5.5cm}
	\thispagestyle{fancy}
	\fancyfoot[C]{м. Київ -- 2014}
	\begin{center}
		\newcommand{\HRule}{\rule{\linewidth}{0.5mm}}
		\textsc{\Large Київський Національний Університет імені Тараса
		Шевченка}\\[1.5cm]

		% Title
		\HRule \\[0.4cm]
		{ \huge \bfseries Вплив на людину джерел електромагнітного
		випромінювання}\\[0.4cm]

		\HRule \\[1.5cm]

		% Author 
			\begin{flushright} \large
				Робота студента\\
				5-го курсу\\
				механіко-математичного факультету\\
				заочної форми навчання\\
				\textsc{Леонтьєва} Олексія Костянтиновича
			\end{flushright}

		\vfill

		% Bottom of the page
		{\large 31 березня 2014 р.}
	\end{center}
\end{titlepage}
\section{Вступ}
В процесі еволюції біосфера постійно перебуває під впливом електромагнітного і
магнітного полів Землі, космічних променів. Нині людство широко використовує
штучні джерела ЕМП у різних галузях науки і техніки (термообробка,
радіолокація, радіозв’язок, у мобільному і стільниковому зв’язку,
радіонавігації, медицині і т. ін).

Основним джерелом ЕМП є трансформатори, ЛЕП, антенні пристрої радіотелевізійних
станцій, та інше електричне устаткування, що працює у широкому діапазоні
частот.

Устаткування, що генерує електромагнітну енергію, випромінює в оточуючий
простір електромагнітні хвилі зі швидкістю близькою до швидкості світла (3108
м/с). Основними параметрами ЕМП є довжина хвилі, частота коливань і швидкість
розповсюдження.
Електромагнітне поле навколо джерела випромінювання хвиль умовно поділяться на
три діапазони:

\begin{itemize}
	\item ближня (зона індукції);
    \item проміжна (зона інтерференції);
\item    дальня (хвильова або зона випромінювання).
\end{itemize}

Електромагнітну енергію використовують у радіо-, радіорелейному і космічному
зв'язках, радіолокації, радіонавігації, на телебаченні, у металургії та
металообробній промисловості для індукційного плавлення, зварювання,
напилювання металів, у деревообробній, текстильній, легкій та харчовій
промисловості, у радіоспектроскопії, сучасній обчислювальній техніці, медицині
тощо.
У виробничих приміщеннях джерелами електромагнітного випромінювання є
неекрановані робочі елементи високочастотних установок (індуктори,
конденсатори, високочастотні трансформатори, фідерні лінії, батареї
конденсаторів, котушки коливальних контурів тощо). При експлуатації ВЧ-, ДВЧ-,
УВЧ-передавачів на радіо- та телецентрах джерелами електромагнітного
випромінювання є високочастотні генератори, антенні комутатори, пристрої
складання потужностей електромагнітного поля, комунікації (від генератора до
антенного пристрою), антени.
Ступінь опромінення працюючих залежить від кількості розміщуваних у приміщенні
передавачів (в окремих зонах, на радіо- та телецентрах їх може бути до 20), їх
потужності, ступеня екранування, розміщення окремих блоків всередині приміщення
і поза його межами.
Для всіх видів зв'язку джерелом електромагнітного випромінювання є
радіолокаційні станції, зокрема генератори, фідерні лінії, антени, окремі блоки
енергії електромагнітного поля ЗВЧ- та НВЧ-діапазонів.
Впливу енергії НВЧ-діапазону працівники зазнають при регулюванні, настроюванні
та випробовуванні радіолокаційних станцій (РЛС), у цехах заводів і ремонтних
майстерень. Основним джерелом випромінювання в цехах заводу є відкриті антенні
системи. Під час випробовування РЛС на полігонах або їх експлуатації в
цивільній авіації умови праці операторів сприятливіші, оскільки більшу частину
робочого часу вони перебувають в екранованих кабінах.
В умовах виробництва електромагнітне випромінювання характеризується
різноманітністю режимів генерації та варіантів дій працівників (випромінювання
у ближній зоні, зоні індукції, загальне і місцеве, яке часто діє разом з іншими
несприятливими факторами навколишнього середовища). Випромінювання може бути
ізольоване (від одного джерела ЕМП), поєднане (від кількох джерел ЕМП одного
частотного діапазону), змішане (від кількох джерел ЕМП різних частотних
діапазонів) та комбіноване (коли одночасно діє інший несприятливий фактор). Дія
ЕМП може бути постійною або переривчастою. Остання, у свою чергу, може бути
періодичною та аперіодичною. Прикладом переривчастої періодичної дії ЕМП є
випромінювання від антен РЛС, які працюють у режимі кругового огляду або
сканування. Дії ЕМП може зазнавати як усе тіло працівника (загальне
опромінення), так і окремі його частини (локальне або місцеве опромінення).

Проблема електромагнітного забруднення навколишнього середовища постала лише
тоді коли було виявлено небезпечний вплив ЕМП на здоров'я людини.
\section{Біологічна дія електромагнітного поля на людину}
Розрізняють дві форми негативного впливу на організм людини електромагнітного
випромінювання діапазону радіочастот — гостру і хронічну, яка, у свою чергу,
поділяється на три ступені: легкий, середній і тяжкий. Хронічна форма
характеризується функціональними порушеннями нервової, серцево-судинної та
інших систем організму, що проявляються астенічним синдромом, і вегетативними
порушеннями, переважно серцево-судинної системи.

Особи, які перебувають під впливом хронічного випромінювання ЕМП, частіше (в
1,9 раза чоловіки та в 1,5 раза жінки), ніж ті, хто не зазнає опромінення,
скаржаться на незадовільний стан здоров'я, у тому числі на головний біль (в 1,5
раза чоловіки та в 1,3 раза жінки), біль у серці (в 1,8 раза чоловіки та в 1,5
раза жінки), серцебиття, загальну слабкість, сонливість, шум у вухах,
парестезію тощо.

Електромагнітне випромінювання -- потужний фізичний подразник. Різні організми
мають різну чутливість до природних та антропогенних (штучних) ЕМП: характер і
вираженість біологічного ефекту залежать від параметрів ЕМП і рівня організації
біосистеми. Міліметрові хвилі ЕМП впливають переважно на рецепторний апарат,
хвилі більшої довжини — на центральну нервову систему.
Радіочастотне випромінювання різні органи і системи організму поглинають
по-різному: істотне значення мають їх форма та лінійні розміри, орієнтація
відносно джерела ЕМП. Первинні зміни функцій центральної нервової системи і
пов'язані з ними порушення спричинюють біологічні ефекти на рівні органів і
систем. Тривала дія високих рівнів електромагнітного випромінювання призводить
до перенапруження адаптаційно-компенсаторних механізмів, істотних відхилень
функцій органів і систем, порушення обміну речовин і ферментативної активності,
гіпоксії, органічних змін. Оскільки у виробничому середовищі електромагнітне
випромінювання діє, як правило, в комплексі з іншими факторами, його вплив на
організм людини посилюється.

Захисно-пристосувальні реакції, що з'являються у людини під впливом
електромагнітного випромінювання, мають неспецифічний характер. Найчастіше
пристосувальними реакціями є збудження центральної нервової системи і
підвищення рівня обміну речовин.

Ефекти від впливу на біологічні тканини людини електромагнітного випромінювання
радіочастотного діапазону малої потужності поділяються на теплові й нетеплові.
Тепловий ефект може виявлятись у людини або підвищенням температури тіла, або
вибірковим (селективним) нагріванням окремих його органів, терморегуляція яких
утруднена (жовчного і сечового міхурів, шлунка, кишок, яєчок, кришталиків,
склистого тіла та ін.). Дія електромагнітного випромінювання на біологічний
об'єкт виявляється тоді, коли інтенсивність випромінювання нижча від теплових
порогових його значень, тобто спостерігаються нетеплові ефекти або специфічна
дія радіохвиль, яка визначається інформаційним аспектом електромагнітного
випромінювання, що сприймається організмом і залежить від властивостей джерела
ЕМП та каналу зв'язку. Очевидно, що інформаційні процеси відіграють також певну
роль при тепловій дії електромагнітного поля на організм. Крім того, дія
електромагнітного випромінювання малої Інтенсивності призводить до локального
нагрівання — мікро-нагрівання.
Умовно розрізняють такі механізми біологічної дії ЕМП:
\begin{itemize}
\item безпосередня дія на тканини та органи, коли змінюється функція центральної
нервової системи і пов'язана з нею нейрогуморальна регуляція;
\item рефлекторні зміни нейрогуморальної регуляції;
\item поєднання основних механізмів патогенезу, дії ЕМП з переважним порушенням
обміну речовин, активності ферментів. 
\end{itemize}
Питома вага кожного з цих механізмів
визначається фізичними та біологічними змінами в організмі людини.
В окремих випадках у людини з'являються біль у серці, задишка, серцебиття,
запаморочення, підвищена пітливість, посилюється функція щитовидної залози,
порушується менструальний цикл у жінок і спостерігається статева слабкість у
чоловіків; змінюється формула крові (зменшується кількість лейкоцитів і
тромбоцитів). Одним із специфічних уражень людини є катаракта, яка може
виникнути або одразу після опромінення, або через 3-6 днів, або розвиватися
поступово впродовж кількох років. Катаракта спричинюється нагріванням
кришталика до температури понад допустимі фізіологічні межі. Окрім катаракти
можливе пошкодження строми рогівки і кератит.
Отже, вплив електромагнітного випромінювання має системний характер і потребує
відповідних системних заходів захисту від нього.
Постійне електричне (електростатичне) поле як фактор впливу на людину
Джерелами постійного електричного (електростатичного) поля (ЕСП) є енергетичні
установки для електротехнологічних процесів, які застосовують у народному
господарстві (електрогазоочищення, електростатична сепарація руд і матеріалів,
електростатичне нанесення лакофарбових матеріалів). Заряди статичної електрики
виникають при подрібненні, деформації речовин, переміщенні тіл, сипких
матеріалів, при інтенсивному перемішуванні, кристалізації, випаровуванні тощо.
Електростатичне поле утворюється електричним полем нерухомих електричних
зарядів, з якими воно взаємодіє, і є найпоширенішим класом стаціонарних
фізичних полів в енергетичних установках та електротехнічних процесах.
Електростатичне поле може існувати як власне електричне поле (поле нерухомих
зарядів) або стаціонарне електричне поле (електричне поле постійного струму).
Електростатичне поле характеризується напруженістю і потенціалом окремих його
точок. Напруженість ECU (Е) — це відношення сили, що діє в полі на точковий
заряд, до величини цього заряду. Одиниця напруженості ЕСП — вольт на метр
(В/м). Напруженість ЕСП не залежить від властивостей середовища, де існує це
поле.

Електростатичні заряди одного знака і поля можуть виникати при виготовленні та
обробці діелектричних матеріалів. Це явище, що називається статичною
електризацією, може відігравати негативну роль.
При експлуатації енергосистем ЕСП утворюються поблизу діючих електроустановок,
розподільних пристроїв та ЛЕП надвисокої напруги постійного струму. В окремих
випадках напруженість ЕСП збільшується в разі іонізації повітря, що виникає при
появі корони на проводах високовольтних ЛЕП постійного струму. При цьому в
повітрі навколо ЛЕП утворюються озон і оксиди азоту.
У зоні високовольтних ЛЕП постійного струму напругою 400, 750 та 1150 кВ
напруженість ЕСП на рівні землі коливається в межах 10-50 В/м. В умовах
виробництва напруженість ЕСП коливається від одиниці до сотень кіловольт на
метр. Висока напруженість ЕСП (до 10 кВ/м) реєструється на пультах управління,
при електростатичному фарбуванні виробів в ізольованих камерах.
При виробництві пластмаси (виготовленні лінолеуму, плівок, паперового пластику
тощо) напруженість ЕСП досягає 240-500 кВ/м. У деревообробній промисловості
напруженість ЕСП на робочих місцях може досягати 140кВ/м. Основним обладнанням,
яке генерує ЕСП, є різноманітні модифікації шліфувальних і полірувальних
верстатів. На шліфувальних верстатах електростатичні заряди утворюються в
місцях зіткнення шліфувальної стрічки з притискним пристроєм і поверхнею
оброблюваного виробу, на полірувальних — у місцях зіткнення полірувального
барабана з поверхнею оброблюваного виробу.
У целюлозно-паперовій промисловості напруженість ЕСП на окремих робочих місцях
може коливатися в межах 60-150 кВ/м, оскільки основою при виробництві паперу є
речовини з вираженими діелектричними властивостями (каніфоль, целюлоза,
парафін, деревна
маса та ін). Електризація відбувається під час сушіння, обробки та намотування
паперу на сортувальних верстатах.
У текстильній промисловості ЕСП зумовлюються широким використанням хімічних
волокон, які мають діелектричні властивості. Електростатичні заряди внаслідок
електризації текстильних волокон (тертя між собою та ниткопровідною гарнітурою)
виникають упродовж всієї технологічної операції. Висока напруженість ЕСП
(120-160 кВ/м) спостерігається на сушильно-ширильних, термофіксацій-них,
стригальних, друкувальних та інших апертурно-оброблюваль-них машинах.

Дія електромагнітних хвиль на організм залежить від інтенсивності джерела,
тривалості опромінення, довжини хвиль, характеру випромінювання (безперервне,
імпульсне) та режиму опромінення (постійне, інтермітуюче).

Основою функціонування організму є дуже слабкі біоелектричні струми, що
синхронізують природні біологічні режими.

Штучні ЕМП якщо співпадають з частотами біологічних ритмів мозку або
біоелектричною активністю серця чи інших органів людини можуть призвести до
десинхронізації функціональних процесів в організмі.

Встановлено, що ЕМП (особливо високовольтні ЛЕП) при тривалій дії здатні
викликати рак, лейкемію, пухлини мозку, розсіяний склероз та інші тяжкі
захворювання. Встановлено , що ЕМП змінюють гени та генофонд усього живого.

Механізм біологічної дії на організм людини полягає як у тепловому, так і
нетепловому специфічному ефекті, теплова дія ЕМП проявляються у підвищенні
температури тіла, а також локальному, вибірковому нагріванні тканин, органів,
клітин унаслідок переходу електромагнітної енергії у теплову.

На людину впливають перемінні ЕМП, статичні струми та ЕМП, що їх супроводять.
Багато полімерних матеріалів накопичують електричні заряди, джерелом статичного
струму може бути одяг людини, що легко електризується за рахунок тертя.

Електризація тіла людини позначається на нервовій системі. Людина стає
роздратованою, надмірно втомлюється, відчуває головні болі або алергічні
реакції.

Напруженість ЕМП величиною 300-1000В/см чинить негативний вплив на організм
людини, а в діапазоні 5000-10000В/см викликає загибель тварин.

Інтенсивність опромінення ЕМП у мешканців міста значно вища, ніж у мешканців
села. У містах утворюються зони, напруженість ЕМП у, яких в десятки та сотні
разів перевищує електромагнітний фон природних зелених зон та сільських
поселень.

\section{Засоби захисту}
Остаточно весь механізм впливу ЕМП на організм людини, ще не зовсім досконало
вивчений, але відомо, що його шкідлива дія проявляється на всіх рівнях --
субклітинному, окремих органах та організмі в цілому. Встановлена кореляційна
залежність між народженням дітей з хворобою Дауна та опромінення їх батьків
НВЧ-енергією.

ЕМП підлягають нормуванню через свою негативну дію на організм людини. Закон
``Про забезпечення санітарного і епідемічного благополуччя населення'' (1996р.)
передбачає норми й правила захисту населення від впливу електромагнітного
випромінювання.

Порогову інтенсивність теплової дії електромагнітних хвиль нормують залежно від
діапазону частот, окремо -- за електричною і магнітною складовою ЕМП.
Біологічний вплив ЕСП залежить від його тривалості, форми струмопровідних
частин обладнання, розміщення робочого місця відносно джерела випромінювання,
кліматичних умов тощо. Експериментальне на тваринах встановлено, що ЕСП впливає
на нервову, серцево-судинну, ендокринну та інші системи організму. Зокрема,
було зареєстровано зміни електричної активності кори великого мозку та
умовно-рефлекторної діяльності. Електростатичне поле спричинює зміни
артеріального тиску, що мають нестійкий і фазовий характер, швидкості зсідання
крові, вмісту сульфгідрильних груп у крові.
Вплив ЕСП на працівників призводить до проявів у них дратівливості, головного
болю, порушення сну, зниження 
апетиту, порушення загальної функції центральної нервової системи, зміни
частоти серцевих скорочень 
(найчастіше у вигляді брадикардії) і вуглеводного, ліпідного, білкового та
мінерального обмінів, а також до зниження активності ферментів.
Заходи захисту від статичної електрики спрямовані на зменшення генерації
електричних зарядів або на їх відведення з наелектризованого матеріалу за
рахунок підвищення його електропровідності. Ці заходи передбачають заземлення
металевих і електропровідних елементів обладнання, встановлення нейтралізаторів
статичної електрики, збільшення поверхневої та об'ємної електропровідності
діелектриків. Заземленню підлягають елементи обладнання, в яких утворюються
електричні заряди, та ізольовані електропровідні ділянки технологічних
установок. Пристрої для захисту від статичної електрики майже завжди
поєднуються із захисними заземлювальними пристроями.
Найефективнішим із зазначених заходів боротьби зі статичною електрикою є
збільшення поверхневої та об'ємної електропровідності
діелектриків. Збільшення відносної вологості повітря до 60-75 \% значно
підвищує поверхневу електропровідність діелектричних гідрофільних матеріалів
(адсорбують на своїй поверхні тонку плівку вологи). На цьому принципі базується
застосування антистатичних речовин (гігроскопічних і поверхнево-активних --
ПАР). Поверхнево-активні речовини наносять на поверхню або вводять у масу
матеріалу (останнє раціональніше, оскільки сприяє тривалому зберіганню
полімерами антистатичних властивостей).
Нейтралізувати електричні заряди можна також за допомогою іонізації повітря.
Для цього використовують нейтралізатори статичної електрики, принцип роботи
яких полягає у створенні поблизу наелектризованих матеріалів позитивних і
негативних іонів.
Для антистатичного захисту можна використовувати ще і принцип екранування за
допомогою металевих листів. При цьому поле, що утворюється на стінках екрана,
нейтралізує зовнішнє поле. Для того щоб електричні заряди з тіла людини швидше
відводилися в землю, застосовують підлоги з електропровідним покриттям. До
індивідуальних засобів захисту тіла людини від статичної електрики належать
антистатичні халати, заземлювальні браслети для рук, антистатичне взуття та ін.
Вибираючи такі засоби, слід враховувати особливості технологічного процесу,
фізико-хімічні властивості оброблюваного матеріалу, мікроклімат приміщень тощо.
\begin{thebibliography}{9}
\bibitem{zhidecky}
В. Ц. Жидецький {\em Основи охорони праці}, Львів, ``Афіша'', 2005
\bibitem{eliseev}
Єлісєєв А.Т. {\em Охорона праці.} -- К., 1995.
\bibitem{moskaliova}
В.М. Москальова. {\em Охорона праці.} -- НУВГП , 2009.
\end{thebibliography}
\end{document}%229 room
