\documentclass[8pt]{article} % use larger type; default would be 10pt

%\usepackage[utf8]{inputenc} % set input encoding (not needed with XeLaTeX)
%\usepackage{CJK}
\usepackage[margin=1in]{geometry}
\usepackage{graphicx}
\usepackage{float}
\usepackage{subfig}
\usepackage{amsmath}
\usepackage{amsfonts}
\usepackage{hyperref}
\usepackage{enumerate}
\usepackage{enumitem}
\usepackage{harpoon}

\usepackage{mystyle}

\title{Homework 3, Math 5111}
\author{Alex Leontiev, 1155040702, CUHK}
\begin{document}
\maketitle
\begin{enumerate}[label=\bfseries Problem \arabic*.]
	\item{
		\begin{enumerate}[label=(\arabic*).]
			\item{Note that $(\sigma(\alpha)\alpha^{-1})^n=\frac{\sigma(\alpha^n)}{\alpha^n}=\frac{\sigma(a)}{a}$, as $\alpha^n=a$. Now, as
				$a\in F$, $\sigma\in Gal(E/F)$, $\sigma$ fixes elements of $F$, in particular $a$. Hence,
				$(\sigma(\alpha)\alpha^{-1})^n=\frac{\sigma(a)}{a}=\frac{a}{a}=1$ and thus $\sigma(\alpha)\alpha^{-1}\in\mu_n$, as it is
				the root of equation $a^n-1=0$ over $E$, but this equation splits in $F$ (as it has there $n$ distinct roots by assumption and it
				cannot have more than $n$ roots in any extension field), hence $\sigma(\alpha)\alpha^{-1}\in F$ and hence in $\mu_n$.

				Now, if $\beta$ is another root of $x^n-a$, we should have $\beta=\mu\alpha,\;\mu\in\mu_n$, as multiplying $\alpha$ with all elements
				of $\mu_n$ shall give us $n$ distinct elements (as $\myabs{\mu_n}=n$ and $a\neq 0\implies\alpha\neq 0$) all of which will satisfy
				$x^n-a$ (because $(\mu\alpha)^n=\mu^n\alpha^n=1\cdot a=a$, if $\mu\in\mu_n$) and latter equation can have no more than $n$ roots.
				Therefore, $\beta=\mu\alpha$ and hence $\frac{\beta}{\alpha}=\mu\in F$ and as $\sigma$ fixes $F$, $\sigma\left(\frac{\beta}{\alpha}
				\right)=\frac{\beta}{\alpha}\implies \sigma(\alpha)\alpha^{-1}=\sigma(\beta)\beta^{-1}$ as required.
				}
			\item{Although as computations done in a previous item imply, this definition of $\Phi$ will not depend on a particular $\alpha$ root
				of $x^n-a=0$ in $E$, for convenience we shall assume it fixed from now on. Now, given $\tau,\;\sigma\in Gal(E/F)$ arbitrary,
				not that as $\sigma(\alpha)/\alpha\in\mu_n\subset F$, we should have $\tau(\sigma(\alpha)/\alpha)=\sigma(\alpha)/\alpha$ and hence
				$\tau(\sigma(\alpha))=\tau(\alpha)\sigma(\alpha)/\alpha$ and thus $(\tau\circ\sigma)(\alpha)/\alpha=\tau(\alpha)/\alpha\cdot
				\sigma(\alpha)/\alpha$ which can be written as $\Phi(\tau\circ\sigma)=\Phi(\tau)\cdot\Phi(\sigma)$. This proves that $\Phi$ is indeed
				a group homomorphism.

				Now, if $\Phi(\sigma)=\sigma(\alpha)/\alpha=1$
				for $\sigma\in Gal(E/F)$, this means that $\forall\alpha,\;\alpha^n=a\implies\sigma(\alpha)=\alpha$, thus $\sigma$ fixes
				zeros of $x^n-a=0$. As splitting field $E$ is generated by zeros of $x^n-a$, the fact that $\sigma$ fixes these zeros as well as $F$
				implies that $\sigma$ fixes $E$, that is $\sigma=id_E$ and this shows injectiveness.
				}
			\item{As $\mbox{char }F=0$, $E/F$ is a separable extension. As $E$ is a splitting field over $F$ (of $x^n-a=0$), it is also normal extension,
				thus it is Galois and hence $[E:F]=\myabs{G}$. Now, as there is an injective mapping $\Phi:G\mapsto \mu_n$ we have validated
				in the previous item, $\myabs{G}$ is the divisor of $\myabs{\mu_n}=n$ and hence $[E:F]$ divides $n$ as well, as $\myabs{G}=[E:F]$.
				}
		\end{enumerate}
		}
	\item{
		\begin{enumerate}[label=(\arabic*).]
			\newcommand{\M}{GL_n(F_q)}
			\item{Let us consider how many choices for each row of arbitrary $M\in GL_n(F_q)$ we have, starting from the first row. Recall, that $M\in
				GL_n(F_q)$ if and only if all rows of it are linearly independent. We have $q^n-1$ choices for the first row, as it is basically
				can be anything except of the zero vector. 
				
				Now, the second row should be linearly independent of the first one, or equivalently,
				does not have to belong to subspace spanned by the first row. The size of that subspace is is $q$, as we can multiply
				the first row with any scalar in $F_q$ to get elements in subspace spanned by it, and conversely any element of the span can be obtained
				in this way (including zero element, which belongs to span). Hence, we have $q^n-q$ choices for the second row.

				Continuing, the third row should not belong to the subspace spanned by the first two rows, and that subspace has size exactly
				$q^2$ (corresponding to $a,b\in F_q$ in linear combinations $a\myvec{r}_1+b\myvec{r}_2$ that form the span of the first two rows
				$\myvec{r}_1$ and $\myvec{r}_2$ of $M$) -- hence $q^n-q^2$ choices, etc.

				The final answer is
				\[\myabs{GL_n(F_q)}=(q^n-1)\cdot(q^n-q^2)\cdot\dots\cdot(q^n-q^{n-1})=q^{\frac{n(n-1)}{2}}(q^n-1)(q^{n-1}-1)\dots(q-1)\]
				}
			\item{Note, that the size of a Sylow $p$-subgroup of $\M$ is the maximal degree of $p$ that divides $\myabs{\M}=q^{\frac{n(n-1)}{2}}(q^n-1)(q^{n-1}-1)\dots(q-1)$
				and hence is $q^{\frac{n(n-1)}{2}}$, as $q^i-1\equiv -1\mbox{ mod }p$ for any $i\geq 1$. Therefore, it is just enough
				to exhibit a subgroup of $\M$ of this size.

				We claim that the set of {\it upper triangular matrices with all diagonal elements equal to 1} will do. They do form a subgroup, as
				they are a subset of $\M$ (in fact, each of them has determinant equal to 1)
				closed under multiplication, identity belongs to this set as well as inverse of any element (this is true, as
				can be seen most easily by computing inverse via the matrix of cofactors). And the order of this subgroup is precisely
				$q^{\frac{n(n-1)}{2}})$, as we have zeros below the diagonal, ones on diagonal and places above the diagonal can be filled with
				anything -- and we have $\frac{n(n-1)}{2}$ of such places.
				}
			\item{We shall count $F_q$-subspaces in the following way. First, every set of $k$ linearly independent vectors
				in $F_q^n$ generates some element of $G_{n,k}(F_q)$, but many sets generate the same space. Thus, we shall compute
				the number of sets of $k$ linearly independent vectors in $F_q^n$. Following the logic in the first item of this
				problem, we have $q^n-1$ ways to pick the first vector, $q^n-q$ to pick the second and so on, giving the final
				answer
				\[N_1=(q^n-1)(q^n-q^2)\dots(q^n-q^{k-1})\]
				Now, given any particular $V\in G_{n,k}(F_q)$, let us count how many bases it has. Fix any particular one,
				say $\myvec{v}_1,\myvec{v}_2,\dots,\myvec{v}_k$, then any other (say, 
				$\myvec{u}_1,\myvec{u}_2,\dots,\myvec{u}_k$) can be bijectively identified with an element of $M\in GL_k(F_q)$,
				namely by $M(\myvec{v}_i)=\myvec{u}_i$. Every basis can be mapped to such an element of $GL_k(F_q)$, and conversely
				given such a matrix, it can be made into a basis in the following way. Assume $M=(a_{i,j})_{i,j=1}^k$ and then
				define $u_i=\sum_{j=1}^k a_{i,j}\myvec{v}_j$. As these two correspondences are the inverses of each other, the 
				correspondence is bijective, hence
				\[\myabs{G_{n,k}(F_q)}=\frac{N_1}{\myabs{GL_k(F_q)}}=\frac{(q^n-1)(q^n-q^2)\dots(q^n-q^{k-1})}
				{(q^k-1)(q^{k}-q)\dots(q^k-q^{k-1})}=\]
				\[=\frac{(q^n-1)(q^{n-1}-1)\dots(q^{n-k+1}-1)q^{\frac{k(k-1)}{2}}}{(q^k-1)(q^{k-1}-1)
				\dots(q-1)q^{\frac{k(k-1)}{2}}}=\]\[=
				\frac{(q^n-1)(q^{n-1}-1)\dots(q-1)}{(q^k-1)(q^{k-1}-1)\dots(q-1)(q^{n-k}-1)(q^{n-k-1}-1)\dots(q-1)}\]
				}
			\item{To begin with, let us assume that $f_{n,k}(x)$ are defined to be zero unless when $0\leq k\neq n$, as otherwise
				given rational expression for them has no sense. Let us furthermore assume the inductive formula
				$f_{n,k}(x)=f_{n-1,k-1}(x)+x^kf_{n-1,k}(x)$ that we shall prove below. Using it, it can be seen
				inductively on $n$ that $f_{n-1,k-1}(x)$ is a polynomial, as right hand side is a polynomial in $f_{n-1,\ast}(x)$ and the inductive
				base with $n\leq 0$ following from definition (in particular, $f_{0,0}(x)=\frac{1}{1\cdot 1}=1$)
				
				Furthermore, it can be shown using the aforementioned recursive formula that $f_{n,k}(1)=\binom{n}{k}$ (again, assuming that
				$\binom{n}{k}$ defined as $\frac{n!}{k!(n-k)!}$ if $0\leq k\leq n$ and zero otherwise). For indeed, we may again proceed by induction
				on $n$. The base case follows for $0\leq k\leq n$ (which is the only thing that requires proof) we have 
				\[f_{n,k}(1)=f_{n-1,k-1}(1)+x^k\cdot 0=\binom{n-1}{k-1}=1\]
				if $n=k$ and
				\[f_{n,k}(1)=f_{n-1,k-1}(1)+1^k\cdot f_{n-1,k}(1)=\binom{n-1}{k-1}+\binom{n-1}{k}=\frac{(n-1)!}{(k-1)!(n-k)!}+\frac{(n-1)!}{k!(n-k-1)!}
				=\]\[=\frac{n!}{k!(n-k)!}\left(\frac{k}{n}+\frac{n-k}{n}\right)=\binom{n}{k}\]
				if $n<k$

				\newcommand{\x}[1]{(x^{#1}-1)}
				Therefore, let us prove the recursive formula $f_{n,k}(x)=f_{n-1,k-1}(x)+x^kf_{n-1,k}(x)$ that is the only piece
				we are lacking now to finish. Again, the case $n<0$ just follows from definition, the case $n=k\geq 0$ is straightforward, as
				\[f_{n,n}(x)=\frac{(x^n-1)(x^{n-1}-1)\dots(x-1)}{(x^n-1)(x^{n-1}-1)\dots(x-1)\cdot1}=1=
				\frac{(x^{n-1}-1)(x^{n-2}-1)\dots(x-1)}{(x^{n-1}-1)(x^{n-2}-1)\dots(x-1)\cdot1}=\]\[=f_{n-1,n-1}(x)+x^n\cdot 0
				=f_{n-1,n-1}(x)+x^n\cdot f_{n-1,n}(x)\]
				So the only remaining case is $0\leq k<n$. We have
				\[f_{n-1,k-1}(x)+x^kf_{n-1,k}(x)=\frac{\x{n-1}\x{n-2}\dots\x{1}}{\x{k-1}\x{k-2}\dots\x{1}\x{n-k}\x{n-k-1}\dots\x{1}}+\]
				\[+x^k\frac{\x{n-1}\x{n-2}\dots\x{1}}{\x{k}\x{x-1}\dots\x{1}\x{n-k-1}\x{n-k-2}\dots\x{1}}=\]
				\[=\frac{\x{n}\x{n-1}\dots\x{1}}{\x{k}\x{k-1}\dots\x{1}\x{n-k}\x{n-k-1}\dots\x{1}}\left(\frac{x^k-1}{x^n-1}+x^k\frac{x^{n-k}-1}{x^n-1}
				\right)=\]
				\[=\frac{\x{n}\x{n-1}\dots\x{1}}{\x{k}\x{k-1}\dots\x{1}\x{n-k}\x{n-k-1}\dots\x{1}}\left(\frac{x^k-1}{x^n-1}+\frac{x^n-x^k}{x^n-1}
				\right)=f_{n,k}(x)\]
				}
		\end{enumerate}
		}
	\item{
		\begin{enumerate}[label=(\arabic*).]
			\newcommand{\E}{\mathbb{C}((x^{\frac{1}{n}}))}
			\newcommand{\F}{\mathbb{C}((x))}
			\item{As $\mbox{char }\mathbb{C}((x))=\mbox{char }\mathbb{C}=0$, this extension is separable. As it is a splitting field of $t^n-x=0$ over
				$\mathbb{C}((x))$, it is normal. The last claim is true, as $t^n-x$ splits in $\mathbb{C}((x^{\frac{1}{n}}))$, with roots
				being the products $x^{\frac{1}{n}}\cdot\mu$ for all possible $\mu\in\mu_n:=\mysetn{\mu\in\mathbb{C}}{\mu^n=1}$. As
				$\myabs{\mu_n}=n$, there are $n$ such distinct products and they exhaust zero set of $t^n-x=0$, as latter equation cannot
				have more than $n$ roots. To finish the verification of a claim that $\mathbb{C}((x^{\frac{1}{n}}))$ is indeed a splitting field
				of $t^n-x=0$ over $\mathbb{C}((x))$, it is sufficient to show that $\E$ is generated over $\F$ by $x^{\frac{1}{n}}$.
				Now, given arbitrary $\sum_{i=-m}^{\infty}a_nx^{\frac{i}{n}}\in\E$, part $\sum_{i=-m}^0a_nx^{\frac{i}{n}}$ is a finite sum
				of products of elements of $\F$ with $x^{-\frac{1}{n}}$ and hence belongs to $\F(x^{\frac{1}{n}})$. Now, the remaining
				part $\sum_{i=0}^\infty a_nx^{\frac{i}{n}}$ can be realized as a finite sum
				$\sum_{k=0}^{n-1}\left((x^{\frac{1}{n}})^k\sum_{i=0}^\infty a_{in+k}x^i\right)$ and as every addend belongs to
				$\F(x^{\frac{1}{n}})$, the whole $\sum_{i=-m}^{\infty}a_nx^{\frac{i}{n}}$ belongs to $\F(x^{\frac{1}{n}})$ and as it was
				arbitrary, $E\subset\F(x^{\frac{1}{n}})$ which ends the proof of a claim.

				Thus $\E/\F$ is normal and separable, and hence Galois. 
				}
			\item{To begin with, note that upon assigning $E:=\E$ and $F:=\F$ we find ourselves completely within
				a setting of Problem 1. Applying it's results, we see that $G:=Gal(\E/\F)$ can be thought
				as a subgroup of a cyclic group $\mu_n$ of $n$-th roots of unity, and thus $G$ is cyclic. Therefore,
				it is completely determined by knowing its order and below we attempt to show that it's order is $n$,
				so it is cyclic of order $n$.

				In order to show that $\myabs{G}=n$, it is sufficient to explicitly exhibit $n$ different
				automorphisms of $\E$ that fix $\F$. We claim that $\sigma_i$ defined as $\sigma_i\mid_\mathbb{C}
				=id_\mathbb{C}$
				and $\sigma_i(x^{\frac{1}{n}})=g^ix^{\frac{1}{n}}$ (where $g$ is an arbitrary generator of $\mu_n$)
				for $1\leq i\leq n$ will do. Now, each two of them are distinct, as they all map $x^{\frac{1}{n}}$
				to distinct values. Besides, each of them is an automorphism of $\E$, as $\sigma_i(\E)\subset\E$
				and $\sigma_i$ is well-behaved under multiplication and addition and maps $1$ to $1$. Finally,
				each $\sigma_i$ fixes $\F$, as $\sigma_i$ fixes $\mathbb{C}$ and $\sigma_i(x)=(\sigma(x^{\frac{1}{n}})
				^n)=g^n(x^{\frac{1}{n}})^n=1\cdot x=x$.
				}
			\item{For the intermediate field $\F\subset K\subset\E$ let us define the correspondence in the following way. Assume $[\E:K]=d\mid n$
				then let us associate with it $H:=\mysetn{g^{\frac{nk}{d}}}{k\in\mathbb{Z}}\subset G$ (where $g$ is the generator of $G$) -- the 
				subgroup of $G$ of order $d$. $H$ is indeed a Galois group for $E/K$, as it is the only subgroup of $G$ of order $[\E:K]=d$. 
				
				Conversely, given $H\subset G$ of order $\myabs{H}=d\mid n$, let us define the corresponding intermediate field $K$ to be
				$\F(x^{\frac{d}{n}})$. $K$ will indeed be a field that is fixed by $H$ as it is of degree $\frac{n}{d}$ (similarly to how we shown
				above that $[\F(x^{\frac{1}{n}}):\F]=n$) over $\F$, hence $[\E:K]=d$, hence it's Galois group will be a subgroup of $G$ of order
				$d$, but there is only one such subgroup, namely $H$.
				}
		\end{enumerate}
		}
\end{enumerate}
\end{document}
