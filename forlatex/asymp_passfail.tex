\documentclass[12pt]{article} % use larger type; default would be 10pt

\usepackage{mathtext}                 % підключення кирилиці у математичних формулах
                                          % (mathtext.sty входить в пакет t2).
\usepackage[T1,T2A]{fontenc}         % внутрішнє кодування шрифтів (може бути декілька);
                                          % вказане останнім діє по замовчуванню;
                                          % кириличне має співпадати з заданим в ukrhyph.tex.
\usepackage[utf8]{inputenc}       % кодування документа; замість cp866nav
                                          % може бути cp1251, koi8-u, macukr, iso88595, utf8.
\usepackage[english,russian,ukrainian]{babel} % національна локалізація; може бути декілька
                                          % мов; остання з переліку діє по замовчуванню. 
\usepackage{mystyle}

\newtheorem{prob}{Завдання}
\newcommand{\ds}{\;ds}
\newcommand{\dt}{\;dt}
\newcommand{\dx}{\;dx}
\newcommand{\dta}{\;d\tau}
\newcommand{\extr}{\mbox{\normalfont extr}}

\newtheorem{myulem}[mythm]{Лема}

\renewenvironment{myproof}[1][Доведення]{\begin{trivlist}
\item[\hskip \labelsep {\bfseries #1}]}{\myqed\end{trivlist}}

\title{Варіаційне числення (9 семестр)}
\author{Олексій Леонтьєв}

\begin{document}
\maketitle
\begin{prob}Довести і сформулювати теорему про стійкість за першим наближенням.\end{prob}%TODO
\begin{prob}Намалювати фазовий портрет та дослідити на стійкість%TODO
	\[\begin{cases}\dot{x}=x\\\dot{y}=2x-y\end{cases}\]\end{prob}
\begin{prob}Дослідити на стійкість положення рівноваги рівняння:\[\dot{x}=x^2-3x+2\]\end{prob}
Почнемо з того, що знайдемо положення рівноваги рівняння. Оскільки воно є автономним, положення рівноваги відповідають
нулям правої частини, тобто $x=2$ та $x=1$. Далі, ми будемо досліджувати на стійкість за допомогою теореми про стійкість за першим
наближенням. Для $x=2$, маємо $\frac{\partial}{\partial x}(x^2-3x+2)\bigg|_{x=2}=1$, а отже, інтерпретуючи це як матрицю
$1\times1$, бачимо, що вона має власне число 2 із додатною дійсною частиною, а отже розв’язок $x=2$ не є стійким. В той же час, 
розв’язок $x=1$ є стійким, оскільки $\frac{\partial}{\partial x}(x^2-3x+2)\bigg|_{x=1}=-1$, і єдине власне число цієї $1\times1$ 
матриці має від’ємну дійсну частину.
\begin{prob}Дослідити на стійкість положення рівноваги наступної системи:
	\[\begin{cases}\dot{x}=y\\\dot{y}=-\sin x\end{cases}\]\end{prob}
	Знову ж таки, ми почнемо із знаходження положень рівноваги,
	які відповідають розв’язкам системи \[\begin{cases}y=0\\\sin x=0\end{cases}\]
	а отже мають вигляд $y=0,\;x=\pi n,\;n\in\mathbb{Z}$. Застосуємо теорему про стійкість за першим наближенням:
	\[\frac{\partial}{\partial (x,y)} \begin{bmatrix}y\\-\sin x\end{bmatrix}\bigg|_{(x,y)=(\pi n,0)}
		=\begin{bmatrix}0&1\\-\cos x&0\end{bmatrix}\bigg|_{(x,y)=(\pi n,0)}=\begin{bmatrix}0&1\\(-1)^{n+1}&0\end{bmatrix}\]
	При $n$ непарному характеристичне рівняння матиме вигляд $\lambda^2-1$, тому $\lambda=1$ буде характеристичним числом,
	а отже положення рівноваги $(x,y)=(\pi(2n+1),0)$ є нестійкими. У випадку ж парного $n$ характеристичними числами будуть
	$\pm i$, і теорема про стійкість за першим наближенням не дає відповіді і потребується пряме аргументування.\\

\end{document}
