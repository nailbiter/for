\documentclass[12pt]{article} % use larger type; default would be 10pt

\usepackage{mathtext}                 % підключення кирилиці у математичних формулах
                                          % (mathtext.sty входить в пакет t2).
\usepackage[T1,T2A]{fontenc}         % внутрішнє кодування шрифтів (може бути декілька);
                                          % вказане останнім діє по замовчуванню;
                                          % кириличне має співпадати з заданим в ukrhyph.tex.
\usepackage[utf8]{inputenc}       % кодування документа; замість cp866nav
                                          % може бути cp1251, koi8-u, macukr, iso88595, utf8.
\usepackage[english,russian,ukrainian]{babel} % національна локалізація; може бути декілька
                                          % мов; остання з переліку діє по замовчуванню. 
\usepackage{mystyle}

\newtheorem{prob}{Завдання}
\newcommand{\ds}{\;ds}
\newcommand{\dt}{\;dt}
\newcommand{\dx}{\;dx}
\newcommand{\dta}{\;d\tau}
\newcommand{\extr}{\mbox{\normalfont extr}}

\newtheorem{myulem}[mythm]{Лема}

\renewenvironment{myproof}[1][Доведення]{\begin{trivlist}
\item[\hskip \labelsep {\bfseries #1}]}{\myqed\end{trivlist}}

\title{Варіаційне числення (9 семестр)}
\author{Олексій Леонтьєв}

\begin{document}
\maketitle
\begin{prob}Довести і сформулювати теорему про стійкість за першим наближенням.\end{prob}%TODO
	{\it Формулювання. }Теорема про стійкість за першим наближенням стосується автономних систем виду
	\[\frac{dy}{dt}=f(y),\;f\in C^2(\mathbb{R}^n),\;f(0)=0\]
	Позначивши $A:=\frac{\partial f}{\partial y}(0)$, теорема стверджує:
	\begin{enumerate}
		\item Якщо всі власні числа $A$ над $\mathbb{C}$ мають від’ємні дійсні частини, розв’язок системи $x\equiv0$ є
			асимптотично стійким.
		\item Якщо $A$ має власне число над $\mathbb{C}$ із додатною власною частиною, розв’язок $x\equiv0$ є нестійким.
	\end{enumerate}
	{\it Доведення}(взяте з \cite{perestyuk}).
	Усюди, де ми згадуватимемо "власні числа", матиметься на увазі "власні числа над $\mathbb{C}$".
	\begin{enumerate}
		\item Ми позначатимемо $g(x):=f(x)-Ax$. Помітимо, що для певного $C>0$, $g(x)\leq C\mynorm{x}^2$ на околі $0$,
			оскільки градієнт $f$ неперервний за умовою. За Теоремою 5.2 з \cite{perestyuk}, для $m=2$, існує
			квадратна форма $V(x)=x^TVx$, така що \[\mysca{2Vx}{Ax}=-\mynorm{x}^2\] Помітимо, що всі власні
			числа $V(x)$ -- невід’ємні. Дійсно, якби якесь з них було б від’ємним, тобто $Ax_0=\lambda x_0,\;\lambda<0,\;
			x_0\neq0$
			, для системи $\dot{x}=Ax$, 
			$V(x)$ задовольняла б умові теореми Ляпунова про нестійкість, адже похідна в силу системи $\dot{V}_{A}
			=\mysca{2Vx}{Ax}=-\mynorm{x}^2$ є від’ємною для $x\neq0$ і $V(0)=0$, але в кожному виколотому околі $0$ міститься
			$sx_0$ для $s>0$ малого, а отже $V(sx_0)=s^2\lambda\mynorm{x_0}^2<0$. Далі, жодне з чисел $V$ не може бути нульовим,
			адже якщо $Vx_0=0,\;x_0$, для розв’язка $x(t)$ системи $\dot{x}=Ax$ що починається в $x_0$, оскільки похідна $V(x)$ в силу
			системи спадає, $V(x(t))$ ставала б від’ємною у суперечність із тільки-но доведеним. Такими чином, $V(x)=x^TVx$ додатно
			визначена.

			Далі, ми стверджуємо, що $V(x)$ є функцією Ляпунова для системи $\dot{x}=Ax+g(x)$, адже похідна в силу системи
			\[\dot{V}_f(x)=\mysca{2Vx}{Ax+g(x)}=-\mynorm{x}^2+2\mysca{Vx}{g(x)}\]
			і оскільки \[\myabs{\mysca{Vx}{g(x)}}\leq\mynorm{V}\mynorm{x}^3C=O\mybra{\mynorm{x}^3}\]
			$\dot{V}_f(x)<0$ для малих $\mynorm{x}>0$.\qed
		\item За теоремою 5.5 з \cite{perestyuk}, існує форма $V(x)=x^TVx$, яка не є від’ємно сталою і така, що
			\[\mysca{2Vx}{Ax}=\alpha V(x)+\mynorm{x}^2,\;\alpha>0\]
			Похідна $V(x)$ в силу системи рівна
			\[\dot{V}_f(x)=\alpha V(x)+\underbrace{\mynorm{x}^2+\mysca{2Vx}{g(x)}}_{=:W(x)}\]
			Помітимо, що $W(x)>0$ для малих $\mynorm{x}>0$, адже $\mysca{2Vx}{g(x)}=O(\mynorm{x}^3)$.
			Покажемо, що $x\equiv0$ не є стійким розв’язком. Нехай $\epsilon>0$ мале задане. Тоді для довільного $\delta>0$ малого
			існує $x_0$, таке що $0<\mynorm{x_0}<\delta$ і $V(x_0)=:V_0>0$. Для розв’язку системи $\dot{x}=f(x)$ із
			початковими умовами $x(0)=x_0$ припустимо (сподіваючись досягти протиріччя), що $\mynorm{x(t)}<\epsilon$, таким чином
			$W(x(t))\geq0$ і ми маємо
			\[\frac{d}{dt}V(x(t))=\alpha V(x(t))+W(x(t))\]
			тому
			\[V(x(t))=e^{\alpha t}\mysbra{\int_0^te^{-\alpha t}W(x(t))\;dt+V_0}\]
			і отже $V(x(t))\geq V_0e^{\alpha t}$, стає необмеженою у протиріччя із обмеженістю $x(t)$.\qed
	\end{enumerate}
\begin{prob}Намалювати фазовий портрет та дослідити на стійкість
	\[\begin{cases}\dot{x}=x\\\dot{y}=2x-y\end{cases}\]\end{prob}
	Власними значеннями матриці 
	\[\begin{bmatrix}1&0\\2&-1\end{bmatrix}\]
		є $\lambda=+1$ із власним вектором $\begin{bmatrix}1&1\end{bmatrix}^T$ і $\lambda=-1$ із власним вектором $\begin{bmatrix}0&1
		\end{bmatrix}^T$. Таким чином, фазовий портрет виглядає як.\\
		\mypic{0.9}{asypm_phasep.JPG}
		Точка рівноваги -- нестійка.

\begin{prob}Дослідити на стійкість положення рівноваги рівняння:\[\dot{x}=x^2-3x+2\]\end{prob}
Почнемо з того, що знайдемо положення рівноваги рівняння. Оскільки воно є автономним, положення рівноваги відповідають
нулям правої частини, тобто $x=2$ та $x=1$. Далі, ми будемо досліджувати на стійкість за допомогою теореми про стійкість за першим
наближенням. Для $x=2$, маємо $\frac{\partial}{\partial x}(x^2-3x+2)\bigg|_{x=2}=1$, а отже, інтерпретуючи це як матрицю
$1\times1$, бачимо, що вона має власне число 2 із додатною дійсною частиною, а отже розв’язок $x=2$ не є стійким. В той же час, 
розв’язок $x=1$ є стійким, оскільки $\frac{\partial}{\partial x}(x^2-3x+2)\bigg|_{x=1}=-1$, і єдине власне число цієї $1\times1$ 
матриці має від’ємну дійсну частину.
\begin{prob}Дослідити на стійкість положення рівноваги наступної системи:
	\[\begin{cases}\dot{x}=y\\\dot{y}=-\sin x\end{cases}\]\end{prob}
	Знову ж таки, ми почнемо із знаходження положень рівноваги,
	які відповідають розв’язкам системи \[\begin{cases}y=0\\\sin x=0\end{cases}\]
	а отже мають вигляд $y=0,\;x=\pi n,\;n\in\mathbb{Z}$. Застосуємо теорему про стійкість за першим наближенням:
	\[\frac{\partial}{\partial (x,y)} \begin{bmatrix}y\\-\sin x\end{bmatrix}\bigg|_{(x,y)=(\pi n,0)}
		=\begin{bmatrix}0&1\\-\cos x&0\end{bmatrix}\bigg|_{(x,y)=(\pi n,0)}=\begin{bmatrix}0&1\\(-1)^{n+1}&0\end{bmatrix}\]
	При $n$ непарному характеристичне рівняння матиме вигляд $\lambda^2-1$, тому $\lambda=1$ буде характеристичним числом,
	а отже положення рівноваги $(x,y)=(\pi(2n+1),0)$ є нестійкими. У випадку ж парного $n$ характеристичними числами будуть
	$\pm i$, і теорема про стійкість за першим наближенням не дає відповіді і потребується інша аргументація. 
	
	Достатньо дослідити
	стійкість розв’язку $(0,0)$, адже всі інші розв’язки виду $(x,y)=(2\pi n,0)$ можуть бути зведеними до цього за допомогою
	заміни $\tilde{x}=x-2\pi n$ і $\tilde{y}=y$.
	Розглянемо функцію Ляпунова $V(x,y)=y^2/2+1-\cos x$. За побудовою, $V(0,0)=0$ і $V(x,y)>0$ на достатньо малому
	виколотому околі $(0,0)$ (наприклад, $(-\pi/2,\pi/2)\times\mathbb{R}\setminus\mycbra{(0,0)}$). Далі, похідна в силу системи
	\[\dot{V}_f=\sin x\cdot y +y(-\sin x)=0\]
	Таким чином, розв’язок $(0,0)$ є стійким, але не є асимптотично стійким, адже для довільно близьких до $(0,0)$ початкових
	умов $(x_0,y_0)$
	, розв’язки будуть залишатися на рівні $\mysetn{(x,y)\in\mathbb{R}^2}{V(x,y)=c:=V(x_0,y_0)>0}$, а тому не можуть прямувати до
	$(0,0)$, через неперервність $V(x,y)$.
\begin{prob}Дослідити на стійкість за допомогою теореми Ляпунова нульовий розв’язок
	\[\begin{cases}\dot{x}=2y-x\\\dot{y}=x-2y-y^3\end{cases}\]\end{prob}
	Спробуємо функцію Ляпунова $V(x,y)=x^2/2+y^2$. Вона є додатною на малому виколотому околі $(0,0)$ і $V(0,0)=0$, а похідна
	в силу системи
	\[\dot{V}_f=x(2y-x)+2y(x-2y-y^3)=-x^2+4xy-4y^2-2y^4=-(x-2y)^2-2y^4\]
	таким чином, похідна є недодатною, і, більше того, рівна нулю лише коли $x-2y=y=0\iff x=y=0$, а отже розв’язок є асимптотично
	стійким.
\begin{prob}Дослідити при яких значеннях параметрів нульовий розв’язок буде асимптотично стійким:
	\[\begin{cases}\dot{x}=ax-2y+x^2\\\dot{y}=x+y+xy\end{cases}\]\end{prob}
	Для початку,
	\[\frac{\partial}{\partial(x,y)}\begin{bmatrix}ax-2y+x^2\\x+y+xy\end{bmatrix}\bigg|_{(0,0)}=\begin{bmatrix}a&-2\\1&1\end{bmatrix}\]
	має характеристичне рівняння 
	\[\chi(\lambda)=\lambda^2-(a+1)\lambda+(a+2)\]
	Таким чином, при $a<-2$ рівняння має від’ємний і додатний корені, тому за стійкістю за першим наближенням, нульовий розв’язок нестійкий.
	Далі, при $-2<a<-1$ рівняння має два від’ємні корені, тому розв’язок асимптотично стійкий, а при $a>-1$, маємо два додатні корені, а отже
	нестійкість за першим наближенням.

	Таким чином, залишається класифікувати $a=-2$ і $a=-1$. Почнемо з $a=-2$. Зробивши заміну
	\[\begin{bmatrix}x\\y\end{bmatrix}=\begin{bmatrix}-2&-1\\1&1\end{bmatrix}\begin{bmatrix}u\\v\end{bmatrix}\]
	приведемо систему до виду
	\[\begin{cases}u'=-u-(2u+v)u\\v'=-(2u+v)v\end{cases}\]
	Маємо нестійкість, адже для довільного розв’язку із початковими умовами $(u_0,v_0)=(0,v_0)$, маємо $u'=0,\;v'=-v^2\implies
	v=1/(t+1/v_0)\to\infty$.

	Для $a=-1$ ми спробуємо довести, що асимптотичної стійкості нема. Зробимо заміну
	\[\begin{bmatrix}x\\y\end{bmatrix}=\begin{bmatrix}-1-i&-1+i\\1&1\end{bmatrix}\begin{bmatrix}u\\v\end{bmatrix}\]
	надалі, ми будемо розглядати комплекснозначні функції як можливі розв’язки рівняння -- це не впливає на твердження, яке ми хочемо довести.
	Система перетвориться на
	\[\begin{cases}u'=-iu+
		\mybra{-u^2-iu^2+iuv-uv}
		\\v'=iv+
		\mybra{-v^2-iuv+iv^2-uv}
	\end{cases}\]
	таким чином, ми бачимо, що для розв’язка із початковими умовами $v(0)=0$, матимемо $v(t)=v(0)=0$, а отже
	\[u'=-iu+(-1-i)u^2\]
	і таким чином
	\[\ln\frac{u}{u+\alpha}=-it,\;\alpha:=\frac{1}{2}+\frac{i}{2}\]
	\[\frac{u}{u+\alpha}=A(\cos t-i\sin t)\]
	\[u=\frac{\alpha s(t)}{1-s(t)},\;s(t):=A(\cos t-i\sin t)\]
	оскільки $s(t)$ періодичне, $u(t)$ також періодичне і не буде прямувати до нуля.
\begin{prob}Показати, що для довільного розв’язку рівняння
	\[\ddot{x}+(4+e^{-t^2})x=0\]
	$x(t)$ та $\dot{x}(t)$ -- обмежені на $[0,+\infty)$.\end{prob}
	Рівняння еквівалентне системі
	\[\begin{cases}\dot{x}=y\\\dot{y}=-(4+e^{-t^2})x\end{cases}\]
	і таким чином достатньо довести обмеженість її розв’язків, що еквівалентно стійкості тривіального розв’язку системи. Оскільки стійкість
	тривіального розв’язку зберігається при $L^1$ збуреннях, а \[\begin{bmatrix}0&0\\-e^{-t^2}&0\end{bmatrix}\]
	є саме таким збуренням, оскільки $\int_0^\infty e^{-t^2}\;dt\leq\int_0^\infty e^{-t}\;dt=1<+\infty$, то достатньо довести стійкість 
	нульового розв’язку системи
	\[\begin{bmatrix}\dot{x}\\\dot{y}\end{bmatrix}=\begin{bmatrix}0&1\\-4&0\end{bmatrix}\begin{bmatrix}x\\y\end{bmatrix}\]
	але нульовий розв’язок цієї системи стійкий (адже розв’язками є $\alpha\cos 2t+\beta\sin 2t$), доведення завершено.
\begin{thebibliography}{9}
\bibitem{demidovich}
Демидович Б. П. \emph{Лекции по математической теории устойчивости} --
Москва, 1967 г., 472 стр. с илл.
\bibitem{perestyuk}
М. О. Перестюк, О. С. Чернікова \emph{Лекции по математической теории устойчивости}, \url{
http://www.mechmat.univ.kiev.ua/dload/pos/teor_stij.pdf}.
\end{thebibliography}
\end{document}
