%platex
\documentclass[reqno,12pt]{pja00} % use larger type; default would be 10pt

\usepackage[dvipdfmx]{hyperref}
\usepackage[normalem]{ulem}
\usepackage{enumerate}
\usepackage{geometry}
\usepackage{setspace}
\usepackage{amsmath,amssymb,xypic}
\usepackage[all,cmtip]{xy}
\usepackage{amsmath,amssymb,float,mystyle}
\usepackage[normalem]{ulem}
\usepackage{caption}
\usepackage{setspace}
\usepackage{multirow}
\usepackage[table]{xcolor}
\usepackage{minibox}
\usepackage{subcaption}
\captionsetup{compatibility=false}
\usepackage{float}

\catcode`\<=\active \def<{
\fontencoding{T1}\selectfont\symbol{60}\fontencoding{\encodingdefault}}
\catcode`\>=\active \def>{
\fontencoding{T1}\selectfont\symbol{62}\fontencoding{\encodingdefault}}
\newcommand{\assign}{:=}
\newcommand{\comma}{{,}}
\newcommand{\nin}{\not\in}
\newcommand{\tmop}[1]{\ensuremath{\operatorname{#1}}}
\newcommand{\tmtextit}[1]{{\itshape{#1}}}
\newcommand{\um}{-}

\newtheorem{theorem}{Theorem}[section]
\newcommand{\sol}{\mathcal{S}\!{\it ol}(\R^{p,q};\lambda,\nu)}
\newcommand{\Hom}{\mbox{\normalfont Hom}}
\newcommand{\Sol}{\mathcal{S}\!{\it ol}}
\newcommand{\Ind}{\mbox{\normalfont Ind}}
\newcommand{\Supp}{\mathcal{S}\!{\it upp}}
\newtheorem{remark}[theorem]{Remark}
\newtheorem{fact}[theorem]{Fact}
\newtheorem{proposition}[theorem]{Proposition}
\theoremstyle{definition}
\newtheorem{definition}[theorem]{Definition}

\theoremstyle{exampstyle} \newtheorem{examp}[theorem]{Theorem}

\catcode`\<=\active \def<{
\fontencoding{T1}\selectfont\symbol{60}\fontencoding{\encodingdefault}}
\catcode`\>=\active \def>{
\fontencoding{T1}\selectfont\symbol{62}\fontencoding{\encodingdefault}}
\newcommand{\dueto}[1]{\textup{\textbf{(#1) }}}
\newcommand{\tmrsub}[1]{\ensuremath{_{\textrm{#1}}}}
\newcommand{\tmrsup}[1]{\textsuperscript{#1}}
\newcommand{\tmtextbf}[1]{{\bfseries{#1}}}
\newcommand{\Op}{\mbox{\normalfont Op}}
\newcommand{\Res}{\operatorname{Res}\displaylimits}
\newcommand{\OpR}{\mbox{\it R}}
\renewcommand{\Q}{Q_{p,q}}
\newcommand{\IlambdaGprime}{I(\lambda)\kern-0.3em\mid_{G'}}
\newcommand{\SBO}{\Hom_{G'}\left(\IlambdaGprime,J(\nu) \right)}

\let\oldsetminus\setminus
\let\setminus-

\setlength{\parskip}{0.4em}
\setlength{\parindent}{2em}

\newcommand{\even}{2\Z}
\newcommand{\odd}{2\Z+1}
\newcommand{\teven}{\mbox{\textrm{: even}}}
\newcommand{\todd}{\mbox{\textrm{: odd}}}
\newcommand{\tevenText}[1]{\vspace{-3cm}$\begin{array}{l}\nu\teven\\\nu#1\end{array}$}
\newcommand{\toddText}[1]{\vspace{-3cm}$\begin{array}{l}\nu\todd\\\nu#1\end{array}$}
\newcommand{\mm}{\mid\mid}
\newcommand{\bb}{\backslash\backslash}
\renewcommand{\ss}{//}

\begin{document}

\title{Symmetry breaking operators for the restriction of representations of indefinite orthogonal groups $O(p,q)$}
\Author{1}{Toshiyuki}{Kobayashi}
\Author{2}{Alex}{Leontiev}
\affiliation{1}{Kavli IPMU and Graduate School of Mathematical Sciences, The University of Tokyo}
\affiliation{2}{Graduate School of Mathematical Sciences, The University of Tokyo}
\KeyWords{ {Representation theory}{reductive group}{branching law}{broken symmetry}{conformal geometry}{symmetry breaking operator}{Verma module}}
\Subject[2010]{22E46; 33C45, 53C35}

  \maketitle
\begin{abstract}
	For the pair $(G, G') =(O(p+1, q+1), O(p,q+1))$, we construct and give a complete classification of intertwining operators (\textit{symmetry breaking operators})
	between
most degenerate spherical
principal series representations of 
$G$ and those of the subgroup $G'$, extending the results of Kobayashi--Speh in the real rank one case where $q=0$
 [Memoirs of Amer. Math. Soc. 2015].
Functional identities and residue formul\ae\, of the regular symmetry breaking operators are also provided 
explicitly.
The results contribute to Program C of branching problems suggested by the first author [Progr. Math. 2015].
\end{abstract}

\section{Branching problem}

Suppose $G \supset G'$ are reductive groups and $\pi$ is an irreducible
representation of $G$. 
The restriction of $\pi$ to the subgroup $G'$ is no more irreducible in general as a representation
of $G'$. If $G$ is compact, then any irreducible $\pi$ is finite-dimensional and splits
into a finite direct sum
\[ \pi\!\mid_{G'} = \bigoplus_{\pi' \in \widehat{G'}} m (\pi, \pi') \pi' \]
of irreducibles $\pi'$ of $G'$ with multiplicities $m(\pi,\pi')$. These multiplicities have been studied
by various techniques including combinatorial algorithms.

However, 
for noncompact $G'$ and for infinite-dimensional $\pi$,
the restriction $\pi|_{G'}$
is not always a direct sum of irreducible representations, see \cite{kobayashi1998discrete2,
kobayashi1998discrete3} for details.
In order to define the ``multiplicity'' in this generality, we recall that, associated to a continuous representation $\pi$ of a Lie group on a Banach space $\mathcal{H}$, 
a continuous representation $\pi^\infty$ is defined on the Fr\'echet space $\mathcal{H}^\infty$ of $C^\infty$-vectors of $\mathcal{H}$.
Given another representation $\pi'$ of a subgroup $G'$, we consider the space of continuous $G'$-intertwining operators ({\it symmetry breaking operators})
\begin{equation}\label{eq:1}
	\Hom_{G'}\left( \pi^\infty\!\mid_{G'}, \left( \pi' \right)^\infty\right).\tag{1.1}
\end{equation}
If both $\pi$ and $\pi'$ are admissible representations of finite 
length of reductive Lie groups $G$ and $G'$, respectively, then the dimension of the space \eqref{eq:1} is determined by the underlying
$(\mathfrak{g},K)$-module $\pi_K$ of $\pi$ and the $(\mathfrak{g}',K')$-module $\pi'_{K'}$ of $\pi'$, and is independent of the choice of Banach globalizations by the 
Casselman--Wallach theory
\cite[Chap.\ 11]{wallach1988real2}. We denote by $m(\pi,\pi')$ the dimension of \eqref{eq:1}, and call it the {\it multiplicity} of $\pi'$ in the restriction $\pi\!\mid_{G'}$.

The above definition of the multiplicity $m(\pi,\pi')$ makes sense 
for nonunitary representations, too. 

In general, $m(\pi,\pi')$ may be infinite, even when $G'$ is a 
maximal reductive subgroup of $G$
({\it e.g.}\;symmetric pairs), see \cite{Kobayashi2014}.
By the theory of real spherical spaces \cite{kobayashi2013finite}, the geometric criterion for finite multiplicities was established in \cite{Kobayashi2014} and \cite{kobayashi2013finite}:

\begin{fact}\label{fact:1} Let $(G,G')$ be a pair of real reductive Lie groups with complexification $(G_{\C},G'_{\C})$.
	\begin{enumerate}[(1)]
		\item The multiplicity $m(\pi,\pi')$ is finite for all irreducible representations $\pi$ of $G$ and all irreducible representations $\pi'$ of $G'$ if and only if
			a minimal parabolic subgroup of $G'$ has an open orbit on the real flag variety of $G$.
		\item The multiplicity $m(\pi,\pi')$ is uniformly bounded if and only if a Borel subgroup of $G_{\C}'$ has an open orbit on the complex flag variety of $G_{\C}$.
	\end{enumerate}
\end{fact}
The classification of symmetric pairs $(G, G')$ satisfying the above geometric criteria
was accomplished in \cite{kobayashi2014classification}.

On the other hand, switching the order in \eqref{eq:1}, we may also consider another space
\begin{equation*}
	\Hom_{G'}\left( \left( \pi' \right)^\infty,\pi^\infty\kern-0.1cm\mid_{G'} \right)\mbox{ or }\Hom_{\mathfrak{g}',K'}\left( \pi_{K'}',\pi_{K}\kern-0.1cm\mid_{\mathfrak{g}',K'} \right).
\end{equation*}
The study of these objects is closely related to the theory of discretely decomposable restrictions \cite{kobayashi1998discrete2,kobayashi1998discrete3}

\section{$\mathcal{A}\mathcal{B}\mathcal{C}$ program for branching}

In {\cite{kobayashi2015program}} the first author suggested a program
for studying the restriction of representations of reductive groups, which may be summarized
as follows:
\begin{description}
  \item[$(\mathcal{A})$] $\mathcal{A}$bstract features of the restriction;
  
  \item[$(\mathcal{B})$] $\mathcal{B}$ranching law of $\pi\!\mid_{G'}$;
  
  \item[$(\mathcal{C})$] $\mathcal{C}$onstruction of symmetry breaking operators.
\end{description}
Program $\mathcal{A}$ aims for establishing the general theory of the restrictions $\pi\!\mid_{G'}$
({\it e.g.} spectrum, multiplicity), which would single out the {\it good} triples $\left( G,G',\pi \right)$. In turn, we could expect concrete and detailed study of those restrictions
$\pi\!\mid_{G'}$ through Programs $\mathcal{B}$ and $\mathcal{C}$.

The main theme of this work is Program ${\mathcal{C}}$ for certain standard
representations with focus on symmetry breaking operators (SBO for short) as follows:
\begin{description}
  \item[$(\mathcal{C}1)$] Construct SBOs explicitly;
  \item[$(\mathcal{C}2)$] Classify all SBOs;
  \item[$(\mathcal{C}3)$] Find residue formul\ae\, for SBOs;
  \item[$(\mathcal{C}4)$] Study functional equations among SBOs;
  \item[$(\mathcal{C}5)$] Determine the images of subquotients by SBOs.
\end{description}
The subprogram $(\mathcal{C}1) - (\mathcal{C}5)$ was proposed by
Kobayashi--Speh in their book {\cite{kobayashi2015symmetry}} with
a complete answer %to $(\mathcal{C}1) - (\mathcal{C}5)$ 
for the
pair $(G, G') = (O (n + 1, 1), O (n, 1))$ of real rank one groups.


In this note we describe the multiplicities for degenerate spherical principal series representations $\pi=I(\lambda)$ of $G$ and $\pi'=J(\nu)$ of $G'$ for the pair
of higher real rank groups\begin{equation}\tag{2.1}\label{eq:2}
	(G,G')=\left( O(p+1,q+1),O(p,q+1) \right)\kern-0.1cm,
\end{equation}
and give an answer to $\left( \mathcal{C}1)-(\mathcal{C}4 \right)$. The subprogram $\left( \mathcal{C}5 \right)$ will be discussed
in a subsequent paper together with an application to the restriction problem of Zuckerman derived functor modules.

Concerning Program $\mathcal{A}$,  Fact \ref{fact:1} assures the following {\it a priori} estimate:\begin{equation*}
	m(\pi,\pi')\mbox{ is uniformly bounded}
\end{equation*}
if the pair of Lie algebras $(\mathfrak{g},\mathfrak{g}')$ is a real form of $(\mathfrak{sl}(n+1,\C),\mathfrak{gl}(n,\C))$
or $(\mathfrak{o}(n+1,\C),\mathfrak{o}(n,\C))$, in particular, if $(G,G')$ is of the form \eqref{eq:2}.
\section{Settings}
Let $G=O(p+1,q+1)$ be the automorphism group of the quadratic form
on $\R^{p+q+2}$ of signature $(p+1,q+1)$ defined by
\begin{equation*}
	Q_{p+1,q+1}(x)
		=x_0^2+\cdots+x_{p}^2-x_{p+1}^2-\cdots-x_{p+q+1}^2.
\end{equation*}

A degenerate spherical principal series representation $I(\lambda):=\Ind_P^G(\C_{\lambda})$ with parameter $\lambda\in\C$ of $G$ is induced from
a character $\C_\lambda$ of a maximal parabolic subgroup $P=MAN_+$
with Levi part
$M A \simeq O (p, q) \times \{ \pm 1 \} \times \mathbb{R}$.
We realize $I(\lambda)$ on the space of $C^\infty$ sections
of the $G$-equivariant line bundle\[
	\mathcal{L}_\lambda=G\times_{P}\C_\lambda\to G/P
\]
so that $I(\lambda)$ itself is the smooth Fr\'echet globalization of moderate growth.
Our parametrization is chosen in a way that
$I(\lambda)$ contains a finite-dimensional submodule if $-\lambda\in2\N$ and a finite-dimensional quotient if $\lambda-\left( p+q\right)\in2\N$ ({\it cf.} \cite{howe1993homogeneous}).

Let $G'=O(p,q+1)$ be the stabilizer of the basis element $e_p$. Similarly to $I(\lambda)$,
we denote by $J(\nu):=\Ind_{P'}^{G'}\left( \C_\nu \right)$ the representation of $G'$
induced from a character $\C_\nu$ of a
maximal parabolic
subgroup $P'$ of $G'$ with Levi part $O(p-1,q)\times\left\{ \pm1 \right\}\times\R$.

The representation $I (\lambda)$ arises from conformal
geometry as follows. We endow the direct product manifold $\Sp^p\times\Sp^q$ with the pseudo-Riemannian structure $g_{\Sp^p}\oplus\left( -g_{\Sp^q} \right)$ of signature $(p,q)$.
Then the group $G=O(p+1,q+1)$ acts as conformal diffeomorphisms on $\Sp^p\times\Sp^q$, and also on its quotient space $X=\left( \Sp^p\times\Sp^q \right)/\Z_2$
by identifying
the direct product of
antipodal points. By
the general theory of conformal {groups}, one has a natural family of representations $\varpi_\lambda$ on $C^\infty(X)$
with parameter $\lambda\in\C$ \cite[Sect.\ 2]{KO1}. Then $X$ identifies with $G/P$, and $\varpi_\lambda$ identifies with $I(\lambda)$. Thus the
branching problem in our setting arises from the
conformal construction of representations for $(X,Y)=\left( \left(\Sp^p\times\Sp^q  \right)/\Z_2,\left( \Sp^{p-1}\times\Sp^q \right)/\Z_2 \right)$.

\section{Multiplicity formul\ae}
In this section we determine explicitly the multiplicity
\begin{equation*}
	m(I(\lambda),J(\nu))=\dim\Hom_{G'}\left( I(\lambda)\kern-0.1cm\mid_{G'},J(\nu) \right).
\end{equation*}
We shall find $m(I(\lambda),J(\nu))>0$ for all $\lambda,\nu\in\C$. More precisely, we
define four subsets of $\C^2$ as below:
\begin{align*}
	 \mid \mid \mid &\assign \{ (\lambda, \nu) \in \mathbb{C}^2 \mid \nu \in
	- 2\mathbb{N} \cup (q + 1 + 2\mathbb{Z}) \},\\
 \backslash\backslash&\assign\mysetn{(\lambda,\nu)\in\C^2}{n-1-\lambda-\nu\in2\N},\\
 / / &\assign\mysetn{(\lambda, \nu) \in \mathbb{C}^2}{\nu-\lambda \in2\N },\\
 \mid\mid&\assign\mysetn{(\lambda,\nu)\in\C^2}{\nu\in1+2\N},
\end{align*}
and two subsets of $\Z^2$ by 
\begin{equation*}
	\mathcal{A}:=//\cap\mid\mid\mid\mbox{ and }\mathcal{X}:=\mid\mid\cap\backslash\backslash.
\end{equation*}
\begin{theorem}
	Let $(G,G')$ be as in \eqref{eq:2} with $p,q\ge1$. Then\begin{equation*}
		m(I(\lambda),J(\nu))\in\left\{ 1,2 \right\}
	\end{equation*}
	for all $\lambda,\nu\in\C$. Furthermore, $m(I(\lambda),J(\nu))=2$ if and only if one of the following conditions holds:
	\begin{enumerate}[C{a}se 1.]
		\item $p>1$. $(\lambda,\nu)\in\mathcal{A}$.
		\item $p=1$ and $q$ is odd. $(\lambda,\nu)\in\mathcal{A}\cup\mathcal{X}$.
		\item $p=1$ and $q$ is even. $(\lambda,\nu)\in\mathcal{A}\cup\mathcal{X}-\mathcal{X}\cap//$.
	\end{enumerate}
	\label{thm:multiplicity}
\end{theorem}
We shall construct explicitly all the symmetry breaking operators in Section \ref{sec:constr}.
\section{Double coset space $P'\backslash G/P$}
In general, as we observe in Fact \ref{fact:1} (and Fact \ref{fact1} below), the double coset space $P'\backslash G/P$ plays a fundamental role in analyzing symmetry breaking operators\begin{equation*}
	\Ind_P^G(\sigma)\to\Ind_{P'}^{G'}(\tau),
\end{equation*}where $\sigma$ is a representation of a parabolic subgroup $P$ of $G$ 
and $\tau$ is that of a parabolic subgroup $P'$ of $G'$. The description of the double coset space $P'\backslash G/P$
is nothing but the Bruhat decomposition if $G'=G$; 
the Iwasawa decomposition if $G'$ is a maximal compact subgroup $K$ of $G$ 
where $P'$ equals automatically  $K$. 

In this section we give a description
of $P'\backslash G/P$ together with its closure relation in the setting where $(G,G',P,P')$ is given {as} in Section 3.
Then the natural action of $G=O(p+1,q+1)$ on $\R^{p+q+2}$ preserves the isotropic cone
\begin{equation*}
	\Xi:=\mysetn{x\in\R^{p+q+2}\setminus\left\{ 0 \right\}}{Q_{p+1,q+1}(x)=0},
\end{equation*}
inducing the $G$-action on its quotient space
\begin{equation*}
	X:=\Xi/\R^{\times}\simeq \left( \Sp^p\times\Sp^q \right)/\Z_2.
\end{equation*}
We define the subvarieties of $X$ by 
\begin{align*}
	Y&\assign\mysetn{[x]\in X}{x_p=0},\\
	C&\assign\mysetn{[x]\in X}{x_0=x_{p+q+1}}.
\end{align*}
Let $P$ be the stabilizer of the point \begin{equation*}
	o:=\left[ 1:0:\dots:0:1 \right]\in X,
\end{equation*}and $P':=P\cap G'$. Then $X$ and $Y$ are identified with the real flag varieties $G/P$ and $G'/P'$, respectively.
\begin{theorem}[description of $P'\backslash G/P$]%[classification of closed $P'$-invariant subsets of $G/P$]
	\label{thm:cloclassif}
	Suppose $p,q\ge1$.
	The left $P'$-invariant closed subsets of $G/P$ are described in the following Hasse diagram. Here 
	$
	\begin{array}{l}
	        \xymatrixrowsep{0.5pc}
		\xymatrix{A\ar@{-}[d]^m\\B}
	\end{array}
	$
	means that $A\supset B$ and that the subvariety $B$ is of codimension $m$ in $A$.\\
  \begin{figure}[h]
    \hspace{-0.6cm}
    \begin{subfigure}[t]{0.3\textwidth}
	    \hspace{0.2cm}
	    \xymatrixrowsep{0.5pc}
	    \xymatrix{&X\ar@{-}[ld]_1\ar@{-}[rd]^1&\\Y\ar@{-}[rd]_1&&C\ar@{-}[ld]^1\\&C\cap Y\ar@{-}[dd]^{p+q-2}&\\&&\\&\{[o]\}&}
	\caption{when $p>1$}
    \end{subfigure}
     %add desired spacing between images, e. g. ~, \quad, \qquad, \hfill etc. 
      %(or a blank line to force the subfigure onto a new line)
    \hspace{-1.6cm}\begin{subfigure}[t]{0.3\textwidth}
	    \hspace{0.8cm}
	    \xymatrixrowsep{0.5pc}
	    {\xymatrix{&X\ar@{-}[ld]_1\ar@{-}[rd]^1&\\Y\ar@{-}[rddd]_{p+q-1}&&C\ar@{-}[lddd]^{p+q-1}\\&&\\&&\\&\{[o]\}&}}
	    \vspace{0.2cm}
	\caption{when $p=1$}
    \end{subfigure}
    \hspace{-1cm}
\end{figure}
\end{theorem}
\section{Construction of symmetry breaking operators\label{sec:constr}}
Let $n:=p+q$. The slice of $\Xi$ by the hyperplane $x_0+x_{p+q+1}=2$
defines the coordinates $\left( x_1,\dots,x_n \right)\in\R^n$ of the open Bruhat cell $U$ of $G/P$, and induces the $N$-picture of the representation $I(\lambda)$\begin{equation*}
	\iota_\lambda^*:I(\lambda)\hookrightarrow C^\infty(\R^n)
\end{equation*}via the trivialization $\mathcal{L}_\lambda\kern-0.1cm\mid_{U}\simeq\R^n\times\C$.

We shall realize
a symmetry breaking operator $T$ in the $N$-pictures of $I(\lambda)$ and $J(\nu)$, and
find a distribution $K_T\in\mathcal{D}'\left( \R^n \right)$ such that
\begin{equation*}
\iota^*_\nu(Tf)(x')=
\operatorname{Rest}_{x_p=0}\circ\int_{\R^n}K_T(x-y)(\iota_\lambda^*f)(y)dy
\end{equation*}
\begin{figure}[h]
\centering
\hspace{1.2cm}
	\xymatrixcolsep{0.2pc}
	\xymatrix
	{
		&I(\lambda)\ar[d]_{\rotatebox{90}{$\sim$}}^{\iota_\lambda^{*}}\ar[rr]^T &&J(\nu)\ar[d]_{\rotatebox{90}{$\sim$}}^{\iota_\nu^{*}}&\\
		C^\infty(\R^n)\supset&\iota_\lambda^*\left( I(\lambda) \right)\ar@{-->}[rr]&&\iota_\nu^{*}\left( J(\nu) \right)&\subset C^\infty(\R^{n-1})\vspace*{-1cm}\\
	}
\end{figure}
for all $f\in I(\lambda)$, where $x''\in\R^{n-1}$. 

In order to analyze the 
distribution kernels $K_T$
of symmetry breaking operators $T$, we begin with:
\begin{definition}\label{def2}We let $O(p-1,q)$ act on $\R^n$ $(n=p+q)$ by leaving $x_p$ invariant. We define $\sol$ 
	to be the space of distributions $F\in\mathcal{D}'(\R^n)$ satisfying the following three conditions:
\begin{enumerate}[(1)]
    \item $F$ is $O(p-1,q)$-invariant and $F(x)=F(-x)$;
    \item $F$ is homogeneous of degree $\lambda-\nu-n$;
    \item $F$ is $N_+'$-invariant.
  \end{enumerate}
\end{definition}
Applying the general result proven in \cite[Chap.\ 3]{kobayashi2015symmetry} to our particular setting, we get the following:
\begin{fact}[{\cite[Thm.\ 3.16]{kobayashi2015symmetry}}]\label{fact1}
Recall 
	$n=p+q\;(p,q\ge1)$. Then the following diagram commutes:\\
	\xymatrixcolsep{0.0pc}
	\hspace*{-1.3cm}\xymatrix{
		\SBO\ar[r]^{\simeq}&\left( \mathcal{D}'(G/P,\mathcal{L}_{n-\lambda}) \otimes\mathbb{C}_\nu \right)^{P'}
\ar[dl]^{\simeq}_{\mbox{\rm Rest}}\\
{\hspace{1.65cm}\sol\subset\mathcal{D}'(\R^n)}\ar[u]^{\mbox{Op}}_{\simeq}\\
}
\end{fact}
For $T\in\SBO$, 
a closed subset $\Supp(T)$ in $P'\backslash G/P$
is defined to be the support of the distribution kernel
 $K_T\in\left( \mathcal{D}'\left( G/P,\mathcal{L}_{n-\lambda} \right)\otimes\C_\nu \right)^{P'}$.
We recall from \cite[Lem.\ 2.22]{kobayashi2016differential1} that
$T$ is a {\it differential} symmetry breaking operator 
if and only if $\Supp(T)$ is a singleton.

Conversely, for each closed subset $S$ of $P'\backslash G/P$, we construct a family $R^S_{\lambda,\nu}$ of SBOs such that:
\begin{itemize}
	\item The domain $D_S$ of the definition of $R_{\lambda,\nu}^S$ is
		either the whole $\C^2$, or is a countable
		union of one-dimensional complex affine spaces;
	\item $R_{\lambda,\nu}^S$ depends holomorphically on $(\lambda,\nu)\in D_S$;
	\item $\Supp(R_{\lambda,\nu}^S)\subset S$ for every $(\lambda,\nu)\in D_S$, and the equality holds for generic points in $D_S$.
\end{itemize}
The relationship among these operators is discussed in Section 8.
We omit here SBOs associated to $S=C\cap Y$ for $p=1$;
$S=C$ or $Y$ for $p>1$, as they are covered by the ``residues'' of the other operators,
and are not used for the classification
 in Theorem \ref{thm:classif}.

Here is a summary of the symmetry breaking operators that we construct below.\\
\begin{equation*}
\begin{array}[]{lll|l|l}
	R_{\lambda,\nu}^S&=&\Op\left(K_{\lambda,\nu}^S  \right)&D_S&\\
	\hline
	R_{\lambda,\nu}^X&=&\Op\left(K_{\lambda,\nu}^X  \right)&\C^2&\mbox{Theorem \ref{thm:regular}}\\
	\tilde{R}_{\lambda,\nu}^X&=&\Op\left(\tilde{K}_{\lambda,\nu}^X  \right)&\mid\mid\mid&\mbox{Theorem \ref{thm:X2}}\\
	R_{\lambda,\nu}^Y&=&\Op\left(K_{\lambda,\nu}^Y  \right)&\backslash\backslash&\mbox{Theorem \ref{thm:singY}}\\
	R_{\lambda,\nu}^C&=&\Op\left(K_{\lambda,\nu}^C  \right)&\mid\mid&\mbox{Theorem \ref{thm:singC}}\\
	R_{\lambda,\nu}^{\mycbra{o}}&=&\Op\left(K_{\lambda,\nu}^{\mycbra{o}}  \right)&//&\mbox{Fact \ref{fact:singo}}\\
\end{array}
\end{equation*}

\begin{theorem}[regular symmetry breaking operator]\label{thm:regular}
	Suppose $n=p+q$ with $p,q\ge1$.
	\begin{enumerate}[(1)]
		\item There exists a family of symmetry breaking operators $R_{\lambda,\nu}^X\in\Hom_{G'}(I(\lambda)\kern-0.1cm\mid_{G'},J(\nu))$ that depends 
			holomorphically on $(\lambda,\nu)$ in the entire $\C^2$ with the distribution kernel
 $K_{\lambda,\nu}^X(x)$ given by
\begin{equation*}
		\frac{1}{\Gamma\left( \frac{\lambda-\nu}{2}\right)\Gamma\left( \frac{\lambda+\nu-n+1}{2}\right)\Gamma\left( \frac{1-\nu}{2}   \right)}\myabs{x_p}^{\lambda+\nu-n}
		\myabs{\Q}^{-\nu}.
\end{equation*}
\item 
	$R^X_{\lambda,\nu}$ vanishes if and only if $(\lambda,\nu)$ belongs to the discrete set $\mathcal{A}$ for $p>1$, $\mathcal{A}\cup\mathcal{X}$ for $p=1,q$ odd
	and $\mathcal{A}\cup\mathcal{X}-\mathcal{X}\cap//$ for $p=1,q$ even.
\item 
	$\Supp(R_{\lambda,\nu}^X)\subset Y,C$ or $\left\{ o \right\}$ if $(\lambda,\nu)\in\backslash\backslash,\mid\mid$ or $//$, respectively, and $\Supp(R_{\lambda,\nu}^X)=X$
	otherwise.
	\end{enumerate}
\end{theorem}
The above normalization of $R^X_{\lambda,\nu}$ is optimal in the sense that
the zeros of $R^X_{\lambda,\nu}$ form a subset of codimension two in $\mathbb C^2$.
Next, we renormalize $R_{\lambda,\nu}^X$ in the place where
$R_{\lambda,\nu}^X$ vanishes. For this, we observe
that $\Gamma\left( \frac{\lambda-\nu}{2} \right)$ is holomorphic in $\C^2\setminus //$,
 and therefore
\begin{equation*}
	\tilde{K}_{\lambda,\nu}^X:=\Gamma( \frac{\lambda-\nu}{2})K_{\lambda,\nu}^X
	=\frac{\myabs{x_p}^{\lambda+\nu-n}\myabs{\Q}^{-\nu}}{\Gamma\left( \frac{\lambda+\nu-n+1}{2} \right)\Gamma\left( \frac{1-\nu}{2} \right)}
\end{equation*}
makes sense if $(\lambda,\nu)\in\C^2\setminus //$. 
Moreover, in light of the fact that $K_{\lambda,\nu}^X$ vanishes 
on $\mathcal{A}=\mid\mid\mid\cap//$,
we obtain its analytic continuation on $\mid\mid\mid$ as follows.
\begin{theorem}[renormalized operator $\tilde{R}_{\lambda,\nu}^X$]
	\begin{enumerate}[(1)]
		\item The renormalized symmetry breaking operator
\begin{equation*}
	\tilde{R}_{\lambda,\nu}^X:=\Op(\tilde{K}_{\lambda,\nu}^X)\in\Hom_{G'}\left( I(\lambda)\kern-0.1cm\mid_{G'},J(\nu) \right)
\end{equation*}
is defined for $(\lambda,\nu)\in\mid\mid\mid$ that depends holomorphically 
			on $\lambda$ in the entire $\C$ for each fixed $\nu$.
		\item $\tilde{R}^X_{\lambda,\nu}$ vanishes if and only if $p=1$, $q$ even and $(\lambda,\nu)\in\mathcal{X}\setminus//$.
	\end{enumerate}
	\label{thm:X2}
\end{theorem}

Let $N\colon\R\to\Z$ be a discontinuous function defined
by $N(x):=x$ if $x \in \mathbb N$; $=0$ otherwise.

Associated to closed subsets $Y$ and $C$ in $P'\backslash G/P$ we introduce families of
\textit{singular} SBOs. For later purpose, we discuss only the case $p=1$.
\begin{theorem}[singular symmetry breaking operators $R_{\lambda,\nu}^Y$]\label{thm:singY}
	Suppose $p=1$ and $q\ge1$. For $(\lambda,\nu)\in\backslash\backslash$, we fix $k:=\frac{1}{2}\left( q-\lambda-\nu \right)\in\N$. Then there exists a family of symmetry breaking operators
	$R_{\lambda,\nu}^Y$ that depends holomorphically on $\nu$ in the entire plane $\C$ with the distribution kernel $K_{\lambda,\nu}^Y$ given by\begin{equation*}
		\frac{1}{\Gamma\left( \frac{\lambda-\nu}{2}+N\left( k-\frac{q}{2} \right) \right)}\delta^{(2k)}(x_p)\myabs{\Q}^{-\nu}.
	\end{equation*}
\end{theorem}
\begin{theorem}[singular symmetry breaking operators $R_{\lambda,\nu}^C$]\label{thm:singC}
	Suppose $p=1$ and $q\ge1$. For $(\lambda,\nu)\in\mid\mid$, we fix $m:=\frac{1}{2}\left(\nu-1 \right)\in\N$. Then there exists a family of symmetry breaking operators
	$R_{\lambda,\nu}^C$ that depends holomorphically on $\lambda$ in the entire plane $\C$ with the distribution kernel $K_{\lambda,\nu}^C$ given by
	\begin{equation*}
	\frac{1}{\Gamma\left( \frac{\lambda-\nu}{2}+N\left(\nu- \frac{q}{2} \right) \right)}\myabs{x_p}^{\lambda+\nu-n}\delta^{(2m)}(\Q).
	\end{equation*}
\end{theorem}

The differential symmetry breaking operators
 $R_{\lambda,\nu}^{\left\{ o \right\}}\colon
C^\infty\left( \R^n \right)\to C^\infty\left( \R^{n-1} \right)$
	were previously found in \cite[Thms.\ 5.1.1 and 5.2.1]{juhl2009families} for $q=0$ 
and in \cite[Thm.\ 4.3]{kobayashi2015branching}
	for general $p,q$ by a different approach. 
See also \cite{kokupe2016forms} for further generalization.
\begin{fact}\label{fact:singo}
	Suppose $(\lambda,\nu)\in//$. We set $l:=\frac{1}{2}\left( \nu-\lambda \right)\in\N$.
	We define $R_{\lambda,\nu}^{ \left\{ o \right\}}$ by
	\begin{equation*}
		\mbox{\normalfont Rest}_{x_p=0}\;\circ
	\sum_{j=0}^la_j\left( \lambda,\nu \right)\left(- \Delta_{\mathbb{R}^{p-1,q}} \right)^j\left( \frac{\partial}{\partial x_p} \right)^{2l-2j}
	\end{equation*}
	where $a_j(\lambda,\nu)$ is given by\begin{equation*}
		a_j(\lambda,\nu)=\frac{(-1)^j2^{2l-2j}}{j!(2l-2j)!}\prod_{i=1}^{l-j}\left(\frac{\lambda+\nu-n-1}{2} 
		+i \right).
	\end{equation*}
	Then $R_{\lambda,\nu}^{ \left\{ o \right\}}\in\Hom_{G'}\left( I(\lambda)\kern-0.1cm\mid_{G'},J(\nu) \right)$.
	The coefficients $a_j(\lambda,\nu)$ gives rise to a Gegenbauer polynomial\begin{equation*}
		\tilde{C}_{2l}^{\lambda+\frac{n-1}{2}}(t)=\sum_{j=0}^la_j(\lambda,\nu)t^{2l-2j}
	\end{equation*}
	renormalized as $\tilde{C}_{2l}^{\lambda+\frac{n-1}{2}}(0)=\left( -1 \right)^l/l!$.

	Its distribution kernel is given by
	\begin{equation*}
		K_{\lambda,\nu}^{ \left\{ o \right\} }:=\sum_{j=0}^la_j(\lambda,\nu)\left( -\Delta_{\R^{p-1,q}} \right)^j\delta_{\R^{p+q-1}}\delta^{\left( 2l-2j \right)}(x_p).
	\end{equation*}
\end{fact}
\begin{remark}\label{rmk:thm:construction}
	The operators $R_{\lambda,\nu}^Y$, $R_{\lambda,\nu}^C$ and $R^{ \left\{ o \right\}}_{\lambda,\nu}$ do not vanish.
\end{remark}
These SBOs are not always linearly independent, but exhaust all SBOs. We provide explicit
basis for $\SBO$ for every $(\lambda,\nu)\in \mathbb{C}^2$:
\begin{theorem}[classification of SBOs]\label{thm:classif}
	The vector space $\Hom_{G'}\left( I(\lambda)\kern-0.1cm\mid_{G'},J(\nu) \right)$ is spanned by the operators as below.
	\begin{enumerate}[(1)]
		\item Suppose $p=1$ and $q\ge1$.
			\begin{equation*}
\left\{
   \begin{array}{ll}
	   R^X_{\lambda, \nu}, & \mbox{\normalfont if }(\lambda, \nu) \notin \mathcal{A}\cup\mathcal{X},\\
      \tilde{R}^X_{\lambda, \nu} , R^{\{ o
      \}}_{\lambda, \nu}, & \mbox{\normalfont if }(\lambda, \nu) \in \mathcal{A} -\mathcal{X},\\
     R^Y_{\lambda, \nu} , R^C_{\lambda, \nu}, &
     \mbox{\normalfont if }(\lambda, \nu) \in \mathcal{X} - / /,\\
     R^{\{ o \}}_{\lambda, \nu}, & \mbox{\normalfont if }(\lambda, \nu) \in \mid \mid
     \cap \backslash\backslash \cap / /.
   \end{array} \right.
			\end{equation*}
		\item Suppose $p\ge2$ and $q\ge1$.
			\begin{equation*}
\left\{
   \begin{array}{ll}
      \tilde{R}^X_{\lambda, \nu} , R^{\{ o
     \}}_{\lambda, \nu}, & \mbox{\normalfont if }(\lambda, \nu) \in \mathcal{A},\\
     R^X_{\lambda, \nu}, & \mbox{otherwise.}
   \end{array} \right. 
			\end{equation*}
	\end{enumerate}
\end{theorem}
\section{Spectrum of SBOs}
The representation $I(\lambda)$ of $G$ contains a one-dimensional subspace of spherical vectors ({\it i.e.} $K$-fixed vectors), and likewise $J(\nu)$ of $G'$.
Let $\mathbf{1}_\lambda\in I(\lambda),\mathbf{1}_\nu\in J(\nu)$ be the spherical vectors normalized by $\mathbf{1}_\lambda(e)=\mathbf{1}_\nu(e)=1$. With this normalization, we have:
\begin{theorem}[spectrum for spherical vectors]\label{thm:spherical}
	Let $n=p+q\;(p,q\ge1)$ as before.
\[ \OpR^X_{\lambda, \nu} \mathbf{1}_{\lambda} =  \frac{2^{1 -
\lambda}\pi^{n / 2}}{\Gamma \left( \frac{\lambda}{2} \right)
\Gamma \left(  \frac{\lambda + 1-q}{2} \right) \Gamma \left(
\frac{q - \nu + 1}{2} \right)} \mathbf{1}_{\nu}. \]
\end{theorem}
\begin{remark}
	Theorem \ref{thm:spherical} was known in \cite[Lem.\ A.5]{bernstein2004estimates} for $p=q=1$ and in \cite[Prop.\ 7.4]{kobayashi2015symmetry} for $q=0$.
 	Another generalization was given in \cite[Thm.\ 1.1]{clerc2011generalized}
	for higher dimensional cases.
\end{remark}
\section{Residue formul\ae\, of symmetry breaking operators}
The regular symmetry breaking operators $R_{\lambda,\nu}^X$ have two complex parameters $(\lambda,\nu)\in\C^2$, whereas the singular operators $R_{\lambda,\nu}^Y$, $R_{\lambda,\nu}^C$, and
$R_{\lambda,\nu}^{ \left\{ o \right\}}$ are defined
for $(\lambda,\nu)\in\backslash\backslash$, $\mid\mid$ and $//$, respectively. We find the relationship
among these operators as explicit residue formul\ae. 

Let $\left( x \right)_j$ be the Pochhammer symbol defined by
\begin{equation*}
	\left( x \right)_j=x(x+1)\dots(x+j-1).
\end{equation*}
\begin{proposition}\label{prop}
	Suppose $p=1$.\begin{enumerate}[(1)]
		\item For $(\lambda,\nu)\in\backslash\backslash$, we set $k=\frac{1}{2}\left( q-\lambda-\nu \right)\in\N$. Then
\begin{equation*}
R_{\lambda,\nu}^X=\frac{(-1)^kk!}{(2k)!}\frac{\left( \frac{\lambda-\nu}{2} \right)_{N\left(k-\frac{q}{2}  \right)}}{\Gamma\left( \frac{1-\nu}{2}\right) }R_{\lambda,\nu}^Y\mbox{ if }(\lambda,\nu)\in\backslash\backslash.
\end{equation*}
\item For $(\lambda,\nu)\in\mid\mid$, we set $m:=\frac{1}{2}\left( \nu-1 \right)\in\N$. Then
\begin{equation*}
R_{\lambda,\nu}^X=\frac{(-1)^mm!}{(2m)!}\frac{\left( \frac{\lambda-\nu}{2} \right)_{N\left( \nu-\frac{q}{2} \right)}}{\Gamma\left( \frac{\lambda+\nu-n+1}{2}\right)}R_{\lambda,\nu}^C\mbox{ if }(\lambda,\nu)\in\mid\mid.
\end{equation*}
	\end{enumerate}
\end{proposition}
\begin{theorem}[residue formula]
\label{thm:residue}
	Let $n=p+q\;(p,q\ge1)$.
	For $(\lambda,\nu)\in//$, we set $l:=\frac{1}{2}\left( \nu-\lambda \right)\in\N$. Then we have for $(\lambda,\nu)\in//$
  \[\OpR_{\lambda,\nu}^X  = \frac{ (- 1)^l l!\pi^{(n - 2) / 2} 
		}{2^{ \nu + 2 l-1}}\cdot  \frac{\sin\left( \frac{1+q-\nu}{2}\pi \right)}{\Gamma\left( \frac{\nu}{2} \right)}
	\OpR_{\lambda,\nu}^{ \left\{ o \right\} }. \]
	\end{theorem}

Proposition \ref{prop} treats easier cases as the subvarieties $Y$ and $C$ are of
 codimension one in $X$ (see Theorem~\ref{thm:cloclassif}),
whereas Theorem~\ref{thm:residue} is more involved.
	\begin{remark}
		The residue formula in the case $q=0$ was given in \cite[Thm.\ 12.2]{kobayashi2015symmetry}.
	\end{remark}
	\section{Functional identities among SBOs}
		Let $n:=p+q$ as before.
	We recall that there exist nonzero Knapp--Stein intertwining operators\begin{equation*}
		\tilde{\mathbb{T}}_\lambda^G:I(\lambda)\to I(n-\lambda)
	\end{equation*}
	with holomorphic parameter $\lambda\in\C$ by the distribution kernel in the $N$-picture normalized as follows:\begin{equation*}
		\begin{array}[]{c}
			\frac{1}{\Gamma\left( \frac{\lambda-n+1}{2} \right)\Gamma\left( \frac{2\lambda-n+2}{4} \right)\Gamma\left( \frac{2\lambda-n}{4} \right)}\cdot{\myabs{\Q}^{\lambda-n}} \times\\
		\left\{\begin{array}[]{@{}l@{}l@{}}
			\Gamma\left( \frac{\lambda-n+2}{2} \right),&\mbox{if $\min(p,q)=0$,}\\
			1,&\mbox{if $p,q>0$, $p\not\equiv q$ mod 2}\\
			\Gamma\left( \frac{2\lambda-n}{4} \right),&\mbox{if $p,q>0$, $p-q\equiv 2$ mod 4}\\
			\Gamma\left( \frac{2\lambda-n+2}{4} \right),&\mbox{if $p,q>0$, $p-q\equiv0$ mod 4}
		\end{array}\right.
		\end{array}
	\end{equation*}

Similarly, we write
		$\tilde{\mathbb{T}}_{\nu}^{G'}:J(\nu)\to J(n-1-\nu)$
	for the Knapp--Stein intertwining operator for $G'$.

\begin{theorem}[functional identities]
  \begin{equation*}
	\begin{array}[]{c}
		\tilde{\mathbb{T}}^{G'}_{n-1 - \nu} \circ R^X_{\lambda, n-1 - \nu}
                  =\frac{\pi^{\frac{n - 3}{2}}\sin\left( \frac{p-\nu}{2} \pi\right)}{\Gamma\left( \frac{n-1-\nu}{2} \right)} a
  (\lambda, \nu) R^X_{\lambda, \nu},
\\
		 R_{n - \lambda, \nu}^X \circ \tilde{\mathbb{T}}^G_{\lambda} = 
  \frac{\pi^{-\frac{n}{2}-1}\sin\left( \frac{p-\lambda+1}{2}\pi \right)}{2^{n-2\lambda}\Gamma\left( \frac{n-\lambda}{2} \right)}
  b
  (\lambda, \nu) R_{\lambda, \nu}^X, 
			\end{array}
 \end{equation*}
 for any $\lambda,\nu\in\C$, where
  \begin{equation*}
	  \begin{array}[]{ll}
		  a(\lambda,\nu)&=\left\{\begin{array}[]{ll}
			  2^{\frac{1-n}{2}}\Gamma\left( \frac{1-\nu}{2} \right),&\mbox{\normalfont if $p=1$},\\
			  2^{\frac{1-n}{2}},&\mbox{\normalfont if $p>1$,$p\equiv q$ mod 2},\\
			  \Gamma\left( \frac{n-2\nu}{2} \right),&\mbox{\normalfont if $p>1$, $p-q\equiv1$ mod 4},\\
			  \Gamma\left( \frac{n-2\nu-2}{4} \right),&\mbox{\normalfont if $p>1$, $p-q\equiv3$ mod 4},\\
		  \end{array}\right.\\
		  b(\lambda,\nu)&=\left\{\begin{array}[]{ll}
			  2^{-\frac{n}{2}},&\mbox{\normalfont\kern0.76cm if $p\equiv q+1$ mod 2},\\
			  \Gamma\left( \frac{2\lambda-n+2}{4} \right),&\mbox{\kern0.76cm\normalfont if $p-q\equiv0$ mod 4},\\
			  \Gamma\left( \frac{2\lambda-n}{4} \right),&\mbox{\kern0.76cm\normalfont if $p-q\equiv2$ mod 4.}\\
		  \end{array}\right.
	  \end{array}
  \end{equation*}
	\end{theorem}
	\begin{remark}
		The functional identities in the case $q=0$ were proven in \cite[Thm.\ 12.6]{kobayashi2015program}.
	\end{remark}

We have given all the constants in this note as {\textit{multiplicatve formul\ae}}
so that we can tell the zeros explicitly. 
Their representation-theoretic interpretation serves as a clue
in the subprogram ($\mathcal C$-5).

	A detailed proof will appear elsewhere.

	{\bf Acknowledgement.} The first author was partially supported by the Grant-in-Aid for Scientific Research (A) 25247006.
\nocite{kobayashi1998discrete2}
\nocite{kobayashi2015program}
\small
\begin{thebibliography}{14}
\expandafter\ifx\csname urlstyle\endcsname\relax
  \providecommand{\doi}[1]{doi:\discretionary{}{}{}#1}\else
  \providecommand{\doi}{doi:\discretionary{}{}{}\begingroup
  \urlstyle{rm}\Url}\fi

\bibitem[1]{bernstein2004estimates}
J.~Bernstein and A.~Reznikov.
\newblock Estimates of automorphic functions.
\newblock \emph{{\normalfont Mosc. Math. J}}, \textbf{\textbf{4}}, (2004),
  pp.~19--37.

  \bibitem[2]{clerc2011generalized}
J.-L. Clerc, T.~Kobayashi, B.~{\O}rsted and M.~Pevzner.
\newblock Generalized {B}ernstein--{R}eznikov integrals.
\newblock \emph{{\normalfont Math.~Ann.}}, \textbf{349}, (2011),.
\href{http://dx.doi.org/10.1007/s00208-010-0516-4}{pp.~395--431}.

\bibitem[3]{howe1993homogeneous}
R.~E. Howe and E.-C. Tan.
\newblock Homogeneous functions on light cones: the infinitesimal structure of
  some degenerate principal series representations.
\newblock \emph{{\normalfont Bull.~Amer.~Math.~Soc.}}, \textbf{28},
  (1993), pp.~1--74.

\bibitem[4]{juhl2009families}
A.~Juhl.
\newblock \emph{Families of {C}onformally {C}ovariant {D}ifferential
  {O}perators, {Q}-curvature and {H}olography}, \emph{{\normalfont Progr.~ Math.},} \textbf{275},
\newblock Birkh{\"a}user (2009).
\newblock ISBN 978-3-7643-9900-9.

\bibitem[5]{kobayashi1998discrete2}
T.~Kobayashi.
\newblock Discrete decomposability of the restriction of {$A_q(\lambda)$} with
  respect to reductive subgroups {II}: Micro-local analysis and asymptotic
  {K}-support.
  \newblock \emph{{\normalfont Ann. Math. (2)}}, \textbf{147}, (1998),
\href{http://dx.doi.org/10.2307/120963}{pp.~709--729}.

\bibitem[6]{kobayashi1998discrete3}
T.~Kobayashi.
\newblock Discrete decomposability of the restriction of {$A_q(\lambda)$} with
  respect to reductive subgroups {III}. {R}estriction of {H}arish-{C}handra
  modules and associated varieties.
\newblock \emph{{\normalfont Invent. Math.}}, \textbf{131}, (1998), 
\href{http://dx.doi.org/10.1007/s002220050203}{pp.~229--256}.

\bibitem[7]{Kobayashi2014}
T.~Kobayashi.
\newblock {S}hintani functions, real spherical manifolds, and
  symmetry breaking operators.
  \newblock \emph{{\normalfont Dev.~Math.}}, \textbf{37}, (2014),
 \href{http://dx.doi.org/10.4171/OWR/2014/3}{pp.~127--159}.

\bibitem[8]{kobayashi2015program}
T.~Kobayashi.
\newblock A program for branching problems in the representation theory of real
  reductive groups.
\newblock \emph{{\normalfont Progr.~Math.}}, \textbf{312}, (2015), 
\href{http://dx.doi.org/10.1007/978-3-319-23443-4_10}{pp.~277--322}.
\newblock In: \emph{{\normalfont Special issue in honor of Vogan's 60th years
  birthday}}.

\bibitem[9]{kokupe2016forms}
T.~Kobayashi, T.~Kubo, and M.~Pevzner,
\newblock 
Conformal symmetry breaking operators for anti-de Sitter spaces.
preprint, 
\href{https://arxiv.org/abs/1610.09475}{arXiv:1610.09475}.

\bibitem[10]{kobayashi2014classification}
T.~Kobayashi and T.~Matsuki.
\newblock Classification of finite-multiplicity symmetric pairs.
\newblock \emph{{\normalfont Transformation Groups}}, \textbf{19}, (2014),
\href{http://dx.doi.org/10.1007/s00031-014-9265-x}{pp.~457--493}.
\newblock In: \emph{{\normalfont Special Issue in honour of Dynkin
  for his 90th birthday}}.


  \bibitem[11]{KO1}
T.~Kobayashi and B.~{\O}rsted.
\newblock Analysis on the minimal representation of\/ {${\rm
  O}(p,q)$}.{\;}{{\rm{I}}. Realization via conformal geometry}.
\newblock \emph{\normalfont Adv.~Math.}, \textbf{180}, (2003),
\href{http://dx.doi.org/10.1016/S0001-8708(03)00012-4}{pp.~486--512}.

\bibitem[12]{kobayashi2015branching}
T.~Kobayashi, B.~{\O}rsted, P.~Somberg and V.~Sou{\v{c}}ek.
\newblock Branching laws for verma modules and applications in parabolic
  geometry. {I}.
\newblock \emph{{\normalfont Adv.~Math.}}, \textbf{285}, (2015),
\href{http://dx.doi.org/10.1016/j.aim.2015.08.020}{pp.~1796--1852}.

\bibitem[13]{kobayashi2013finite}
T.~Kobayashi and T.~Oshima.
\newblock Finite multiplicity theorems for induction and restriction.
\newblock \emph{{\normalfont Adv.~Math.}}, \textbf{248}, (2013), 
 \href{http://dx.doi.org/10.1016/j.aim.2013.07.015}{pp.~921--944}.

\bibitem[14]{kobayashi2016differential1}
T.~Kobayashi and M.~Pevzner.
\newblock Differential symmetry breaking operators: I. {G}eneral theory and
  {F}-method.
\newblock \emph{{\normalfont Selecta Math.}}, \textbf{22}, (2016),
\href{http://dx.doi.org/10.1007/s00029-015-0207-9}{pp.~801--845}.

\bibitem[15]{kobayashi2015symmetry}
T.~Kobayashi and B.~Speh.
\newblock \emph{Symmetry {B}reaking for {R}epresentations of {R}ank {O}ne
  {O}rthogonal {G}roups}, \emph{{\normalfont Memoirs of the Amer.~Math.~Soc},}
  \textbf{\href{http://dx.doi.org/10.1090/memo/1126}{238}}, (2015).
\newblock ISBN 978-1-4704-1922-6.

\bibitem[16]{wallach1988real2}
N.~Wallach.
\newblock \emph{Real Reductive Groups II}, \emph{{\normalfont Pure and Applied
  Mathematics},} \textbf{132},
\newblock Academic {P}ress (1992).
\newblock ISBN 978-0127329611.

\end{thebibliography}
\end{document}
