\documentclass[12pt]{article} % use larger type; default would be 10pt

%%\usepackage[T1,T2A]{fontenc}
%%\usepackage[utf8]{inputenc}
%%\usepackage[english,ukrainian]{babel} % може бути декілька мов; остання з переліку діє по замовчуванню. 
\usepackage{enumerate}
\usepackage{CJKutf8}
\usepackage{mystyle}
\usepackage{enumitem,amssymb}
\usepackage[]{hyperref}
\newlist{todolist}{itemize}{2}
\setlist[todolist]{label=$\square$}
\usepackage{pifont}
\newcommand{\cmark}{\ding{51}}%
\newcommand{\xmark}{\ding{55}}%
\newcommand{\done}{\rlap{$\square$}{\raisebox{2pt}{\large\hspace{1pt}\cmark}}%
\hspace{-2.5pt}}
\newcommand{\wontfix}{\rlap{$\square$}{\large\hspace{1pt}\xmark}}

%%\usepackage{fancyhdr}
%%\pagestyle{fancy}
%%\fancyfoot[C]{text me at \href{mailto:leontiev@ms.u-tokyo.ac.jp}{leontiev@ms.u-tokyo.ac.jp} if there are mistakes/obscurities}
%%\fancyhead{}

\title{Documents included for Visa Application\\Provided from Japan\\For Visiting Relatives/Acquaintances}
\begin{document}

\begin{CJK}{UTF8}{bsmi}
	\maketitle
\end{CJK}

\begin{todolist}
	%\item[\done] 
	\item Letter of reason for invitation
	\item Schedule of stay (日本語)
	\item List of visa applications (日本語、見本: \cite{applicantlist})
	\item Letter of guarantee
\end{todolist}

\begin{thebibliography}{9}
	\bibitem{applicantlist}\url{http://www.mofa.go.jp/j_info/visit/visa/pdfs/application5.pdf}
\end{thebibliography}
\end{document}


