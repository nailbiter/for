\documentclass[12pt]{article} % use larger type; default would be 10pt

\usepackage{enumerate}
\usepackage{xeCJK}
\usepackage{mystyle}
\usepackage{ruby}
\usepackage{longtable}
\usepackage{hyperref}
\usepackage{mystyle}
\usepackage{enumitem,amssymb}
\usepackage[]{hyperref}
\newlist{todolist}{itemize}{2}
\setlist[todolist]{label=$\square$}
\usepackage{pifont}
\newcommand{\cmark}{\ding{51}}%
\newcommand{\xmark}{\ding{55}}%
\newcommand{\done}{\rlap{$\square$}{\raisebox{2pt}{\large\hspace{1pt}\cmark}}%
\hspace{-2.5pt}}
\newcommand{\wontfix}{\rlap{$\square$}{\large\hspace{1pt}\xmark}}

\setCJKmainfont[AutoFakeBold=true]{Hiragino Mincho Pro} %my Mac
%\setCJKmainfont{MS PGothic} %AJP windows
%\renewcommand\rubysep{-5ex}
\newcommand{\kana}[2]{\ruby{#1}{#2}}
\newcommand{\mytabra}[1]{$\myabra{\mbox{#1}}$}

\title{Documents included for Visa Application\\Provided from Japan\\For Visiting Relatives/Acquaintances}
\begin{document}
\maketitle
\begin{todolist}
	%\item[\done] 
	\item[\done] Letter of reason for invitation (日本語、見本: \cite{reasonletter});
	\item Schedule of stay (日本語、見本: \cite{schedule});
	\item[\done] List of visa applications (日本語、見本: \cite{applicantlist});
	\item[\done] Letter of guarantee (日本語、見本: \cite{guarantee});
	\item[\done] Copy of both side of the ``Residence card'';
	\item[\done] Certificate of Receipt of a Scholarship (学振奨学金、コーピ、日本語);
	\item[\done] Certificate of student registration (コーピ、日本語);
	\item[\done] Certificate of residence (日本語);
	\item[\done] Certification of the Balance of Deposit (2口座、英語);4
	\item[\done] Copy of the last page of passbook (コーピ、日本語).
\end{todolist}

\begin{thebibliography}{9}
	\bibitem{reasonletter}\url{http://www.mofa.go.jp/files/000137089.pdf}
	\bibitem{schedule}\url{http://www.mofa.go.jp/j_info/visit/visa/pdfs/application7.pdf}
	\bibitem{applicantlist}\url{http://www.mofa.go.jp/j_info/visit/visa/pdfs/application5.pdf}
	\bibitem{guarantee}\url{http://www.mofa.go.jp/j_info/visit/visa/pdfs/application2.pdf}
\end{thebibliography}
\end{document}
