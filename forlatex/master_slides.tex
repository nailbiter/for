\documentclass[8pt,notes]{beamer}
\mode<presentation>{\usetheme[secheader]{Boadilla}}
\usepackage{mystyle}
\usepackage{geometry,amsmath,amssymb,bbm,xypic}
\usepackage{xeCJK}
\usepackage{ruby}
\includecomment{versiona}

\newcommand{\red}[1]{{\color[rgb]{0.6,0,0}#1}}
\newcommand{\Sol}{\mathcal{S}\mbox{ol}}
\newcommand{\Hom}{\mbox{Hom}}
\newcommand{\D}{\mathcal{D}}
\newcommand{\A}{\mathcal{A}}
\newcommand{\Co}{\mathbb{C}}
\renewcommand{\setminus}{-}
\newcommand{\assign}{:=}
\newcommand{\comma}{{,}}
\newcommand{\nin}{\not\in}
\newcommand{\tmop}[1]{\ensuremath{\operatorname{#1}}}
\newcommand{\tmtextbf}[1]{{\bfseries{#1}}}
\newcommand{\tmtextit}[1]{{\itshape{#1}}}
\newcommand{\mss}{//}
\newcommand{\mbb}{\backslash\backslash}
\newcommand{\mmm}{\mid\mid}

\setCJKmainfont{IPAMincho}

\newenvironment{setting}{\begin{exampleblock}{設定.}\it}{\end{exampleblock}}
\newenvironment{question}{\begin{block}{問.}\it}{\end{block}}
\newenvironment{prop}[1][]{\begin{block}{命題#1.}\it}{\end{block}}
\makeatletter
\def\th@mystyle{%
    \normalfont % body font
    \setbeamercolor{block title example}{bg=orange,fg=white}
    \setbeamercolor{block body example}{bg=orange!20,fg=black}
    \def\insertpropblockenv{exampleblock}
  	}
\makeatother
\theoremstyle{mystyle}
\newtheorem*{remark}{注.}

%%\newenvironment<>{setting}{%
%%  \begin{actionenv}#1%
%%      \def\insertblocktitle{Setting}%
%%      \par%
%%      \mode<presentation>{%
%%        \setbeamercolor{block title}{fg=white,bg=orange!20!black}
%%       \setbeamercolor{block body}{fg=black,bg=olive!50}
%%     }%
%%    \usebeamertemplate{block begin}}
%%{\par\usebeamertemplate{block end}\end{actionenv}}
\newenvironment<>{notation}{%
  \begin{actionenv}#1%
      \def\insertblocktitle{記号}%
      \par%
      \mode<presentation>{%
        \setbeamercolor{block title}{fg=white,bg=orange!20!black}
       \setbeamercolor{block body}{fg=black,bg=olive!50}
     }%
    \usebeamertemplate{block begin}}
{\par\usebeamertemplate{block end}\end{actionenv}}
\newenvironment<>{cor}{%
  \begin{actionenv}#1%
      \def\insertblocktitle{系}%
      \par%
      \mode<presentation>{%
        \setbeamercolor{block title}{fg=white,bg=orange!20!black}
       \setbeamercolor{block body}{fg=black,bg=olive!50}
     }%
    \usebeamertemplate{block begin}}
{\par\usebeamertemplate{block end}\end{actionenv}}

\title{\ruby{不定値直交}{ふていあたいちょっこ}群 O(p,q) の対称性の破れ作用素}
\author{Alex Leontiev}

\begin{document}
\begin{frame}\titlepage\end{frame}
\begin{frame}{Outline}
	\tableofcontents
\end{frame}

\section{設定}
\begin{frame}
	\begin{setting}
		$p\ge1,\;q\ge2$, $G:=O(p+1,q+1)$, $G':=O(p+1,q+1)_{e_{p+1}}\simeq O(p,q+1)$.
	\end{setting}
	\begin{question}
	{与えられた $( \lambda, \nu) \in
	\mathbbm{C}^2$に対して、\ruby{対称性}{たいしょうせい}の破れ作用素の空間 $\tmop{Hom}_{G'} ( I (
	\lambda), J ( \nu))$ を具体的に求めよ。特に、この空間の\ruby{基底}{きてい}を具体的に求めよ。ここで、$I(\lambda):=G\times_P\mathbb{C}_\lambda$
	と$J(\nu):=G'\times_{P'}\mathbb{C}_\nu$は$G$と$G'$の\ruby{退化主系列}{たいかしゅけいれつ}である。}	
	\end{question}
\end{frame}
\section{Equations for kernels}
\begin{frame}
	\begin{prop}[ (prop. 9.4)]
		\begin{equation*}
			\Hom_{G'}(I\left( \lambda \right),J(\nu))\simeq \mybra{\D'(G/P,G\times_P\mathbb{C}_{n-\lambda})
			\otimes\mathbb{C}_\nu}^{
			\Delta P'}\simeq\Sol(\R^{p,q};\lambda,\nu)
		\end{equation*}
ここで、$\mathcal{S} \tmop{ol} ( \mathbbm{R}^{p, q} ; \lambda,
\nu)$ は以下の条件を満たす超関数$F \in \mathcal{D}' (
\mathbbm{R}^{p, q})$の空間である:
\begin{enumerate}
	\item $F$は\ruby{斉}{せい} $\lambda-\nu-n$-次である;
 \item $F$は偶関数である;
\item $\forall m \in O ( p, q)_{e_p} \assign \{ m \in O ( p, q) | m \cdot e_p = e_p \},\quad F ( m \cdot) = F ( \cdot)$が成立する;
 \item $b, x_0 \in \mathbbm{R}^{p, q}$, $b_p = 0$ と $c_b
 ( x_0) : = 1 - 2 Q ( b, x_0) + Q ( x_0) Q ( b) \neq 0$であれば、 $x_0$の近傍上で
\begin{eqnarray}
& | c_b ( \cdot) |^{\lambda - n} F ( \psi_b ( \cdot)) = F (\cdot)
\; \mbox{が成立}& \nonumber\\
& \mbox{ここで、}\;\psi_b ( x) \assign \frac{x - Q (x) b}{c_b ( x)} . & \nonumber
\end{eqnarray}
\end{enumerate}
	\end{prop}
	\begin{prop}[ (prop. 9.5)]
		Suppose $F_\mu\in\D'(U)$
		is holomorphic in $\mu\in\Omega\subset\Co^k$, $\lambda(\cdot),\nu(\cdot)$ are holo on $\Omega$ and
		for $\mu\in\Omega'\subset\Omega$ we have $F_\mu\in\Sol(U;\lambda(\mu),\nu(\mu))$. Then this is
		so also for $\mu\in\Omega$.
	\end{prop}
\end{frame}
\section{$P'\backslash G/P$}
\begin{frame}
\begin{prop}[ (prop. 8.1, 8.2)]
	\begin{equation*}
	G/P=\left\{ \begin{array}{ll}
	P' ( 0_p, 1, 0_q, 1) \sqcup P' ( 1, 0_{p + q}, - 1) \sqcup P' ( 0_p, 1,
	1, 0_q) \sqcup P' ( 1, 0_{p + q}, 1), & p = 1,\\
	P' ( 0_p, 1, 0_q, 1) \sqcup P' ( 1, 0_{p + q}, - 1) \sqcup P' ( 0_p, 1,
	1, 0_q) \sqcup&\\
	\sqcup P' ( 1, 0_{p + q}, 1) \sqcup P' ( 0_{p - 1}, 1, 0, 1,
	0_q), & p > 1
	\end{array} \right.
	\end{equation*}
	and these are pulled back to $\R^n-P\cup C,\;P-C,\;C\cap P-\left\{ 0 \right\}
	,\;{0}$ and $C-P$ respectively (note that $P\cap C=\left\{ 0 \right\}$ iff $p=1$)
	via Bruhat cell parametrization and only the first one is open dense in $\R^n$.
	Here $P:=\mycbra{x_{p}=0}$ and $C:=\mysetn{(x,y)\in\R^{p,q}}{\myabs{x}_{p}=\myabs{y}_{q}}$ closed subsets of $\R^n$.
\end{prop}
\begin{cor}
	\begin{itemize}
		\item $P'N_-P=G$ (prop. 8.3);
		\item Supports can only be $\R^{p,q}$, $C$, $P$, $C\cap P$, $C\cup P$ or $\left\{ 0 \right\}$.
	\end{itemize}
\end{cor}
\end{frame}
\section{Exact sequences}
\begin{frame}
	We have the following diagram.
	\begin{equation*}
\xymatrix{&\mathbbm{R}^n&\\
&C\cup P\ar@{-}[u]&\\
C\ar@{-}[ur]&&P\ar@{-}[ul]\\
&C\cap P\ar@{-}[ul]\ar@{-}[ur]&\\
&\{0\}\ar@{-}[u]&}
\end{equation*}
Therefore, as we want to know $\Sol(\R^n;\lambda,\nu)$, we proceed in steps:
\end{frame}
\begin{frame}
	\begin{enumerate}
		\item 
			$0\rightarrow\Sol_{ \left\{ 0 \right\} }(\R^n;\lambda,\nu)\hookrightarrow\Sol_C(\R^n;\lambda,\nu)
			\rightarrow\Sol_C(\R^n\setminus\left\{ 0 
			\right\};\lambda,\nu)$
		\item 
			$0\rightarrow\Sol_{C}(\R^n;\lambda,\nu)\hookrightarrow\Sol(\R^n;\lambda,\nu)
			\rightarrow\Sol(\R^n\setminus C;\lambda,\nu)$
	\end{enumerate}
	\begin{prop}[ (prop. 14.2)]
		\begin{equation*}
			\Sol_C\left( \R^n\setminus\left\{ 0 \right\};\lambda,\nu \right)=\Co
			\begin{cases}
				0,&\nu\nin2\Z_{\ge0}+1\\
				\delta^{(\nu-1)}(Q)\cdot\myabs{x_p}^{\lambda+\nu-n},&p=1,\nu\in2\Z_{\ge0}+1\\
				\delta^{(\nu-1)}(Q)\cdot\frac{\myabs{x_p}^{\lambda+\nu-n}}{\Gamma\left( \frac{
				\lambda+\nu-n+1}{2} \right)},&p>1,\nu\in2\Z_{\ge0}+1.
			\end{cases}
		\end{equation*}
	\end{prop}
	\begin{prop}[ (prop. 11.1)]
		\begin{equation*}
			\Sol\left( \R^n\setminus C;\lambda,\nu \right)=
			\Co\myabs{Q}^{-\nu}\frac{\myabs{x_p}^{\lambda+\nu-n}}{\Gamma\left( \frac{\lambda+\nu-n+1}{2} \right)}.
		\end{equation*}
	\end{prop}
	The nontrivial part therefore is to know for a given pair of parameters $(\lambda,\nu)\in\Co^2$ whether the last morphism
	is onto or not. We still cannot do this at the moment. As a preliminary, we need to construct as many as possible families of
	kernels.
\end{frame}
\section{Kernel of regular SBO}
\begin{frame}
\begin{prop}[ (prop. 12.1)]
	For $\Re(\lambda+\nu-n),\Re(-\nu)>0$ we have continuous function\begin{equation*}
		\myabs{x_p}^{\lambda+\nu-n}\myabs{Q}^{-\nu}\in\Sol(\R^{p,q};\lambda,\nu)
	\end{equation*}
\end{prop}
\begin{prop}[ (prop. 12.2, 12.3)]
	For $(\lambda,\nu)\in\mathbb{C}^2$ with $\lambda-\nu\nin\Z_{\ge1}$ the {\bf well-defined} product of generalized
	functions\begin{equation*}
		\frac{\myabs{x_p}^{\lambda+\nu-n}}{\Gamma\left( \frac{\lambda+\nu-n+1}{2} \right)}\cdot
		\frac{\myabs{Q}^{-\nu}}{\Gamma\left( \frac{1-\nu}{2} \right)}\in\D'(\R^{p,q}\setminus\left\{ 0 \right\})
	\end{equation*}
	has the {\bf unique}
	extension to a homogeneous $K^{\R^n}_{\lambda,\nu}\in\Sol(\R^{p,q};\lambda,\nu)$.
	Moreover,\begin{equation*}
		\supp\left(K^{\R^n}_{\lambda,\nu}  \right)=\begin{cases}
			\R^n,&(\lambda,\nu)\in\mmm^c\cap\mbb^c\\
			C,&(\lambda,\nu)\in\mmm\cap\mbb^c\\
			P,&(\lambda,\nu)\in\mmm^c\cap\mbb\\
			\emptyset,&p=1,(\lambda,\nu)\in\mmm\cap\mbb\\
			P\cap C,&p>1,(\lambda,\nu)\in\mmm\cap\mbb
		\end{cases}
	\end{equation*}
\end{prop}
\end{frame}
\begin{frame}
\begin{notation}
\begin{gather*}
	\slash\slash:=\mysetn{\left( \lambda,\nu \right)\in\Co^2}{\lambda-\nu\in-2\Z_{\ge0}}=\mysetn{(\lambda,\nu)\in\Co^2}{
		\Gamma\left( \frac{\lambda-\nu}{2} \right)=\infty},\\
\backslash\backslash:=\mysetn{\left( \lambda,\nu \right)\in\Co^2}{\lambda+\nu-n+1\in-2\Z_{\ge0}}=\mysetn{(\lambda,\nu)\in\Co^2}{
		\Gamma\left( \frac{\lambda+\nu-n+1}{2} \right)=\infty},\\
\mathbb{X}:={\backslash\backslash}\cap{\slash\slash},\\
\mmm:=\mysetn{(\lambda,\nu)\in\Co^2}{\nu\in2\Z_{\ge0}+1}=\mysetn{(\lambda,\nu)\in\Co^2}{
		\Gamma\left( \frac{1-\nu}{2} \right)=\infty}.
\end{gather*}
\end{notation}
\end{frame}
\section{Kernel of singular SBO supported on $C$}
\begin{frame}
	\begin{prop}[ (prop. 14.4)]
		For $\nu\in2\Z_{\ge0}+1$ and $\lambda\in\Co$ such that $\lambda-\nu\nin-\Z_{\ge1}$ the\begin{equation*}
  \left\{ \begin{array}{ll}
 \delta^{( \nu - 1)} ( Q) \cdot | x_p |^{\lambda + \nu - n}, & p = 1\\
 \delta^{( \nu - 1)} ( Q) \cdot \frac{| x_p |^{\lambda + \nu -
 n}}{\Gamma ( ( \lambda + \nu - n + 1) / 2)}, & p > 1
 \end{array} \right. 
		\end{equation*}
	has the {\bf unique}
	extension to a homogeneous $K^C_{\lambda,\nu}\in\Sol_C(\R^{p,q};\lambda,\nu)$. 
	\end{prop}
\end{frame}
\section{Kernel of singular SBO supported on $P$}
\begin{frame}
	\begin{prop}[ (prop. 13.1)]
		For $k\in\Z_{\ge0}$, $\Re(\nu)\ll0$ and $\lambda$ defined by $\lambda+\nu-n=-1-2k$, the product\begin{equation*}
			\delta^{(2k)}(x_p)\cdot\myabs{Q}^{-\nu}\in\D'(\R^{p,q}\setminus\left\{ 0 \right\})
		\end{equation*}
	has the {\bf unique}
	extension to a homogeneous $K^P_{\lambda,\nu}\in\Sol_P(\R^{p,q};\lambda,\nu)$.
	\end{prop}
	\begin{prop}[ (lem. 13.2)]
		For $k,\lambda$ as above and $\Re(\nu)\ll0$ there is an equality
			 \begin{eqnarray}
			 & K^P_{\lambda, \nu} = \sum_{i = 0}^k \frac{(- 1)^i (2 k) !
			 (\nu)_i}{(2 k - 2 i) !i!} \delta^{(2 k - 2 i)} (x_p) \otimes \tilde{Q}_i
			 & \nonumber\\
			 & (\nu)_i \assign \nu (\nu + 1) \ldots (\nu + i - 1), & \nonumber\\
			 & \tilde{Q}_i \assign \left\{ \begin{array}{ll}
			 \tilde{Q}_+^{- \nu - i} + \tilde{Q}_-^{- \nu - i}, & i \in 2 \Z_{\ge 0},
			 p > 1\\
			 \tilde{Q}_+^{- \nu - i} - \tilde{Q}_-^{- \nu - i}, & i \in 2 \Z_{\ge 0}
			 + 1, p > 1\\
			 | \tilde{Q} |^{- \nu - i}, & i \in 2 \Z_{\ge 0}, p = 1\\
			 - | \tilde{Q} |^{- \nu - i}, & i \in 2 \Z_{\ge 0} + 1, p = 1
			 \end{array} \right. & \nonumber
			 \end{eqnarray}
			 and therefore $K^P_{\lambda,\nu}$ can be extended to all $\nu\in\Co$ outside some discrete set.
	\end{prop}
\end{frame}
\section{Differential symmetry breaking operators}
\begin{frame}
\begin{prop}[ (prop. 15.1)]
	\begin{equation*}
		\Sol_{\left\{ 0 \right\}}(\R^{p,q};\lambda,\nu)=\begin{cases}0,&\lambda-\nu\nin2\Z_{\le0}\\
			\Co \tilde{C}_{\nu-\lambda}^{\lambda-(n-1)/2}\mybra{\tilde{\Delta},\frac{\partial}{\partial x_p}},&
			\lambda-\nu\in2\Z_{\le0}\\
		\end{cases}
	\end{equation*}
	here,\begin{eqnarray}
		&\tilde{C}^\mu_N(s,t):=s^{N/2}\tilde{C}^\mu_N\mybra{\frac{t}{\sqrt{s}}}&\nonumber\\
		&\tilde{C}^\mu_N(t):=\frac{\Gamma(\mu)}{\Gamma(\mu+N/2)}\frac{\Gamma(2\mu+N)}{\Gamma(N+1)\Gamma(2\mu)}
		{}_2F_1\mybra{2\mu+N,-N;\mu+\frac{1}{2};\frac{1-t}{2}}&\nonumber
	\end{eqnarray}
\end{prop}
\end{frame}
\section{Normalization of $K^P_{\lambda,\nu}$}
\begin{frame}
	Assume $n\in2\Z$.
	\begin{prop}[ (prop. 18.1, 18.7)]
		Let $\tilde{K}^P_{\lambda,\nu}:=K^P_{\lambda,\nu}/N$ where \begin{equation*}
			N:=\Gamma\left( \frac{n-1}{2}-\nu-k \right)\times\begin{cases}
				1,&p=1\\
				\Gamma\left( \frac{1-\nu}{2} \right),&p>1.
			\end{cases}
		\end{equation*}
		Then $\tilde{K}^P_{\lambda,\nu}$ is nonzero and holomorphic in $\nu\in\Co$. Moreover,
		\begin{equation*}
			\supp\left( \tilde{K}_{\lambda,\nu}^P \right)=\begin{cases}
				\left\{0 \right\},&\left(\lambda,\nu \right)\in \mathbb{X}(\iff\nu\in\frac{n-1}{2}-k+\Z_{\ge0})\\
				\left\{ Q=0 \right\}\cap\left\{ x_p=0 \right\},&p>1,\;\nu\in1+2\Z_{\ge0}\\
				\left\{ x_p=0 \right\},&\mbox{otherwise}.
			\end{cases}
		\end{equation*}
	\end{prop}
\end{frame}
\begin{frame}
	Assume $n\in2\Z+1$ and $q\in2\Z$.
	\begin{prop}[ (prop. 18.2, 18.3, 18.7)]
		Let $\tilde{K}^P_{\lambda,\nu}:=K^P_{\lambda,\nu}/N$ where \begin{equation*}
			N:=\begin{cases}
				\Gamma\left( \max\left\{ 0,\frac{n-1}{2}-k \right\}-\nu \right),&p=1\\
				\Gamma\left(\mybra{\left.\max\left\{ 2,\left( \frac{n-1}{2}-k \right)' \right\}-\nu}\right/2\right)
				\Gamma\left( \frac{1-\nu}{2} \right),&p>1
			\end{cases}
		\end{equation*}
		(where $a'$ denotes the smallest even integer which is $\ge a$).
		Then $\tilde{K}^P_{\lambda,\nu}$ is nonzero and holomorphic in $\nu\in\Co$. Moreover,
		\begin{align*}
			p=1\implies&\supp\left( \tilde{K}_{\lambda,\nu}^P \right)=\begin{cases}
				\left\{0 \right\},&\nu\in\max\left\{ 0,\frac{n-1}{2}-k \right\}+\Z_{\ge0}\\
				\left\{ x_p=0 \right\},&\mbox{otherwise}.
			\end{cases}\\
			p>1\implies&\supp\left( \tilde{K}_{\lambda,\nu}^P \right)=\begin{cases}
				\left\{0 \right\},&\nu\in 2\Z\cap\left[\max\left\{ \frac{n-1}{2}-k ,1\right\}+\Z_{\ge0}\right]\\
				\left\{ Q=0 \right\}\cap\left\{ x_p=0 \right\},&\nu\in1+2\Z_{\ge0}\\
				\left\{ x_p=0 \right\},&\mbox{otherwise}.
			\end{cases}
		\end{align*}
	\end{prop}
\end{frame}
\begin{frame}
	Assume $n\in2\Z+1$ and $q\in2\Z+1$.
	\begin{prop}[ (prop. 18.5, 18.6, 18.7)]
		Let $\tilde{K}^P_{\lambda,\nu}:=K^P_{\lambda,\nu}/N$ where \begin{equation*}
			N:=\begin{cases}
				\Gamma\left( \max\left\{ 0,\frac{n-1}{2}-k \right\}-\nu \right),&p=1\\
				\Gamma\left( \frac{\left( \frac{n-1}{2}-k \right)'-\nu}{2}
				\right)\Gamma\left( \frac{1-\nu}{2} \right),&p>1
			\end{cases}
		\end{equation*}
		(where $a'$ denotes the smallest even integer which is $\ge a$).
		Then $\tilde{K}^P_{\lambda,\nu}$ is nonzero and holomorphic in $\nu\in\Co$. Moreover,
		\begin{align*}
			p=1\implies&\supp\left( \tilde{K}_{\lambda,\nu}^P \right)=\begin{cases}
				\left\{0 \right\},&\nu\in\max\left\{ 0,\frac{n-1}{2}-k \right\}+\Z_{\ge0}\\
				\left\{ x_p=0 \right\},&\mbox{otherwise}.
			\end{cases}\\
			p>1\implies&\supp\left( \tilde{K}_{\lambda,\nu}^P \right)=\begin{cases}
				\left\{0 \right\},&\nu\in 2\Z_{\ge0}+\left( \frac{n-1}{2}-k \right)'\\
				\left\{ Q=0 \right\}\cap\left\{ x_p=0 \right\},&\nu\in2\Z+1,\;
				1\le\nu<\left( \frac{n-1}{2}-k \right)'\\
				\left\{ x_p=0 \right\},&\mbox{otherwise}.
			\end{cases}
		\end{align*}
	\end{prop}
\end{frame}
\section{Normalization of $K^C_{\lambda,\nu}$}
\begin{frame}
	\begin{prop}[ (prop. 16.8, 16.9)]
		Fix $\nu\in2\Z_{\ge0}+1$. Suppose $R$ is meromorphic in $\lambda\in\Co$. Then $K^C_{\lambda,\nu}/R(\lambda)$
		is holomorphic in $\lambda\in\Co$ if for every $N\in2\Z_{\ge0}$ and every $g\in\Co\left[ x,y \right]$ even
		polynomial we have $\varphi_N\left[ g \right]\left( \lambda,\nu \right)/R(\lambda)$ being holo in $\lambda\in\Co$ for
		\begin{align*}
  \varphi_N [ g] ( \lambda, \nu) \assign \left\{ \begin{array}{ll}
    \Gamma ( \lambda + \nu - n + 1) / \Gamma \left( \frac{\lambda + \nu - n -
    N + 2}{2} \right) / \Gamma \left( \frac{\lambda + \nu + N - q}{2}
    \right), & p > 1\\
    1, & p = 1, N = 0\\
    0, & p = 1, N > 0
  \end{array} \right. \times\\
  \times \int_{- 1}^1 ( 1 - y^2)^{( \lambda + \nu + N - q) / 2 - 1}
  \frac{d^{\nu - 1}}{d x^{\nu - 1}} |_{x = y} [ ( 1 - x^2)^{( q - 2) / 2} g (
  x, y)] d y.
		\end{align*}
		Moreover, for $\lambda\in\Co$ we have $K^C_{\lambda,\nu}/R=0$ iff for every $N\in2\Z_{\ge0}$ and $g$ even poly
		we have $\varphi_N[g](\lambda,\nu)/R(\lambda)=0$.
	\end{prop}
	\end{frame}\begin{frame}
	\begin{prop}[ (prop. 17.1, 17.2)]
		For fixed $\nu\in2\Z_{\ge0}+1$ let \begin{equation*}
	N \assign \left\{ \begin{array}{ll}
     \Gamma \left( \frac{\lambda - \nu}{2} \right), & q \in 2\mathbbm{Z}+ 1\\
     \Gamma \left( \frac{\lambda - \min \{ \nu, q - \nu \}}{2} \right) & q \in
     2\mathbbm{Z}, \; p = 1\\
     1, & q \in 2\mathbbm{Z}, p > 1.
   \end{array} \right. 
		\end{equation*}
	Then $\tilde{K}^C_{\lambda,\nu}:=K^C_{\lambda,\nu}/N$ is a nonzero holomorphic in $\lambda\in\Co$ distribution.
	Moreover, $\supp\left( \tilde{K}^C_{\lambda,\nu} \right)$ 
	equals to\begin{equation*}
	\left\{ \begin{array}{lll}
     \{ 0 \}, & p = 1, q \in 2\mathbbm{Z}+ 1, & \lambda - \nu \in -
     2\mathbbm{Z}_{\geqslant 0}\\
     \{ Q = 0 \}, & p = 1, q \in 2\mathbbm{Z}+ 1, & \lambda - \nu \nin -
     2\mathbbm{Z}_{\geqslant 0}\\
     \{ 0 \}, & p = 1, q \in 2\mathbbm{Z}, & \lambda - \min \{ \nu, q - \nu \}
     \in - 2\mathbbm{Z}_{\geqslant 0}\\
     \{ Q = 0 \}, & p = 1, q \in 2\mathbbm{Z}, & \lambda - \min \{ \nu, q -
     \nu \} \nin - 2\mathbbm{Z}_{\geqslant 0}\\
     \{ 0 \}, & p > 1, q \in 2\mathbbm{Z}+ 1, & \lambda - \nu \in -
     2\mathbbm{Z}_{\geqslant 0}\\
     \{ Q = 0 \}, & p > 1, q \in 2\mathbbm{Z}_{} + 1, & \lambda - \nu \nin -
     2\mathbbm{Z}_{\geqslant 0}, \lambda + \nu - n \nin -
     2\mathbbm{Z}_{\geqslant 0} - 1\\
     \{ Q = 0 \} \cap \{ x_p = 0 \}, & p > 1, q \in 2\mathbbm{Z}_{} + 1, &
     \lambda - \nu \nin - 2\mathbbm{Z}_{\geqslant 0}, \lambda + \nu - n \in -
     2\mathbbm{Z}_{\geqslant 0} - 1\\
     \{ Q = 0 \}, & p > 1, q \in 2\mathbbm{Z}, & \lambda + \nu - n \nin -
     2\mathbbm{Z}_{\geqslant 0} - 1\\
     \{ Q = 0 \} \cap \{ x_p = 0 \}, & p > 1, q \in 2\mathbbm{Z}, & \lambda +
     \nu - n \in - 2\mathbbm{Z}_{\geqslant 0} - 1
   \end{array} \right. 
		\end{equation*}
	\end{prop}
\end{frame}
\section{Normalization of $K^{\R^n}_{\lambda,\nu}$}
\begin{frame}
	\begin{prop}[ (prop. 16.5, 16.7)]
		Suppose $R$ is meromorphic in $(\lambda,\nu)\in\Co^2$. Then $K^{\R^n}_{\lambda,\nu}/R(\lambda,\nu)$
		is holomorphic in $(\lambda,\nu)\in\Co$ if for every $N\in2\Z_{\ge0}$ and every $g\in\Co\left[ x,y \right]$ even
		polynomial we have $\varphi_N\left[ g \right]\left( \lambda,\nu \right)/\Gamma\left( \frac{\lambda+\nu-n+1}{2} \right)
		/\Gamma\left( \frac{1-\nu}{2} \right)
		/R(\lambda,\nu)$ being holo in 
		$(\lambda,\nu)\in\Co^2$ for
		\begin{align*}
  \varphi_N [ g] ( \lambda, \nu) \assign \left\{ \begin{array}{ll}
    \Gamma ( \lambda + \nu - n + 1) / \Gamma \left( \frac{\lambda + \nu - n -
    N +2 2}{2} \right) / \Gamma \left( \frac{\lambda + \nu + N - q}{2}
    \right), & p > 1\\
    1, & p = 1, N = 0\\
    0, & p = 1, N > 0
  \end{array} \right. \times\\
  \times \int_{[ - 1, 1]^2} | x - y |^{- \nu} ( 1 - x^2)^{( q - 2) / 2} ( 1
  - y^2)^{( \lambda + \nu + N - q) / 2 - 1} g ( x, y) d x d y.\\
		\end{align*}
		Moreover, for $(\lambda,\nu)\in\Co^2$ we have $K^{\R^n}
		_{\lambda,\nu}/R(\lambda,\nu)=0$ iff for every $N\in2\Z_{\ge0}$ and $g$ even poly
		we have $\varphi_N[g](\lambda,\nu)/\Gamma\left( \frac{\lambda+\nu-n+1}{2} \right)
		/\Gamma\left( \frac{1-\nu}{2} \right)/R(\lambda,\nu)=0$.
	\end{prop}
	\begin{prop}[ (prop. 19.2)]
	Assume $q\in2\Z$.
		Let $N:=\Gamma\left( \frac{\lambda-\nu}{2} \right)$. Then $\tilde{K}_{\lambda,\nu}^{\R^n}:=K_{\lambda,\nu}^{
			\R^{n}}/N$ is holomorphic in $\left( \lambda,\nu \right)\in\Co^2$ and vanishes on a discrete
			subset of $\Co^2$.
	\end{prop}
\end{frame}
\end{document}
