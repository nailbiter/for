\documentclass[12pt]{article} % use larger type; default would be 10pt

\usepackage{stmaryrd}
\usepackage{enumerate}
\usepackage{geometry}
\usepackage{setspace}
\usepackage{amsmath,amssymb,bbm,xypic}
\usepackage[all,cmtip]{xy}
\usepackage{amsmath,amssymb,bbm,float,mystyle}
\usepackage[normalem]{ulem}
\usepackage{caption}
\usepackage{subcaption}
\usepackage{setspace}
\usepackage{comment}
\usepackage{catchfilebetweentags}
\usepackage{multirow}
\usepackage[table]{xcolor}
\includecomment{versiona}
\usepackage{tikz}
\usepackage{bashful}
\usetikzlibrary{patterns}
\usepackage{bbm}
\newcommand{\Conv}{\mathop{\scalebox{1.5}{\raisebox{-0.2ex}{$\ast$}}}}

%%%%%%%%%% Start TeXmacs macros
\catcode`\<=\active \def<{
\fontencoding{T1}\selectfont\symbol{60}\fontencoding{\encodingdefault}}
\catcode`\>=\active \def>{
\fontencoding{T1}\selectfont\symbol{62}\fontencoding{\encodingdefault}}
\newcommand{\assign}{:=}
\newcommand{\comma}{{,}}
\newcommand{\nin}{\not\in}
\newcommand{\tmop}[1]{\ensuremath{\operatorname{#1}}}
\newcommand{\tmtextit}[1]{{\itshape{#1}}}
\newcommand{\um}{-}

\newtheorem{theorem}{Theorem}
\newcommand{\sol}{\mathcal{S}\!{\it ol}(\R^{p,q};\lambda,\nu)}
\newcommand{\Hom}{\mbox{\normalfont Hom}}
\newcommand{\Sol}{\mathcal{S}\!{\it ol}}
\newcommand{\Ind}{\mbox{\normalfont Ind}}
\newcommand{\Supp}{\mathcal{S}\!{\it upp}}
\newtheorem{remark}[theorem]{Remark}
\newtheorem{corollary}[theorem]{Corollary}
\newtheorem{fact}{Fact}
%\newtheorem{definition}{Definition}
\theoremstyle{definition}
\newtheorem{definition}{Definition}

\makeatletter
\newtheoremstyle{exampstyle}
  {\topsep} % Space above
  {\topsep} % Space below
  { {\addtolength{\@totalleftmargin}{3.5cm}
     \addtolength{\linewidth}{-3.5cm}
        \parshape 1 3.5em \linewidth}} % Body font
  {-2.5cm} % Indent amount
  {\bfseries} % Theorem head font
  {.} % Punctuation after theorem head
  {.5em} % Space after theorem head
  {} % Theorem head spec (can be left empty, meaning `normal')
\makeatother

\theoremstyle{exampstyle} \newtheorem{examp}[theorem]{Theorem}

\catcode`\<=\active \def<{
\fontencoding{T1}\selectfont\symbol{60}\fontencoding{\encodingdefault}}
\catcode`\>=\active \def>{
\fontencoding{T1}\selectfont\symbol{62}\fontencoding{\encodingdefault}}
\newcommand{\dueto}[1]{\textup{\textbf{(#1) }}}
\newcommand{\tmrsub}[1]{\ensuremath{_{\textrm{#1}}}}
\newcommand{\tmrsup}[1]{\textsuperscript{#1}}
\newcommand{\tmtextbf}[1]{{\bfseries{#1}}}
\newtheorem{proposition}{Proposition}
\newcommand{\Op}{\mbox{\normalfont Op}}
\newcommand{\Res}{\operatorname{Res}\displaylimits}
\newcommand{\OpR}{\mbox{\it R}}
\renewcommand{\Q}{Q_{p,q}}
\newcommand{\IlambdaGprime}{I(\lambda)\kern-0.3em\mid_{G'}}
\newcommand{\SBO}{\Hom_{G'}\left(\IlambdaGprime,J(\nu) \right)}
\renewcommand{\setminus}{-}
%%%%%%%%%% End TeXmacs macros

\setlength{\parskip}{0.4em}
\setlength{\parindent}{2em}

\newcommand{\even}{2\Z}
\newcommand{\odd}{2\Z+1}
\newcommand{\teven}{\mbox{\textrm{: even}}}
\newcommand{\todd}{\mbox{\textrm{: odd}}}
\newcommand{\tevenText}[1]{\vspace{-3cm}$\begin{array}{l}\nu\teven\\\nu#1\end{array}$}
\newcommand{\toddText}[1]{\vspace{-3cm}$\begin{array}{l}\nu\todd\\\nu#1\end{array}$}
\newcommand{\mm}{\mid\mid}
\newcommand{\bb}{\backslash\backslash}
\renewcommand{\ss}{//}
%%%%%%%%%% End TeXmacs macros


\begin{document}

\title{Symmetry breaking of indefinite orthogonal subgroups $O(p,q)$}

  %%%% 講演者1
  \author{小林俊行}
  \author{レオンチエフ アレックス}

  %%%% 講演者2

  %%%% 日付
%  \date{2012年3月26日}

  %%%% 謝辞、キーワード、MSCコード  

  \maketitle
\section{Branching problem}

Suppose $G \supset G'$ are reductive groups and $\pi$ is irreducible
representation of $G$. If we restrict $\pi$ to $G'$, in general it is no
longer irreducible.\\
\begin{figure}[H]
	\xymatrixrowsep{0.2pt}
	\xymatrixcolsep{0.5cm}
	\centering
	\hspace{2.8cm}\xymatrix{
		\pi:&G\ar[r]&GL_{\C}(V)&(\dim V=\infty)\\
		&\bigcup&&\\
		&G'\ar@{-->}[uur]_{\pi\big|_{G'}}&&\\
	}
\end{figure}
\begin{center}
	\fbox{{\textbf{Branching problem}} (in a wider sense) = Understand $\pi\!\mid_{G'}.$}
\end{center}
These are well-studied (e.g. combinatorial algorithm) for $\pi$:
finitely-dimensional and $G$: compact. In this setting, $\pi$ always splits
into a direct sum
\[ \pi\!\mid_{G'} = \bigoplus_{\tau \in \widehat{G'}} m (\pi, \tau) \tau \]
of irreducibles $\tau$ of $G'$.

However, when $\dim\pi = \infty$ and $G, G'$ are non-compact, the situation
becomes much more involved and was not studied seriously before Kobayashi's
theory appeared in 90s. In particular, several examples that show that
behaviour is very wild in general were constructed {\cite{Kobayashi2008}}.

{\textbf{SBOs:}} an idea to understand ``indecomposable'' restriction $\pi\!
\mid_{G'}$ is to compare it with irreducible representations $\tau$ of the
small group $G'$, i.e. to study the space
\[ \tmop{Hom}_{G'} (\pi\!\mid_{G'}, \tau) \]
of {\textbf{symmetry breaking operators}} (SBOs, for short).

\section{$\mathcal{A}\mathcal{B}\mathcal{C}$ program for branching}

In {\cite{kobayashi2015program}} T. Kobayashi introduced the far-reaching
program for studying branching of noncompact groups, which can be summarized
as follows:
\begin{description}
  \item[($\mathcal{A}$)] $\mathcal{A}$bstract features of the representation
  (i.e. we want to find triples $(G, G', \pi)$, so that $\pi\!\mid_{G'}$ is
  manageable);
  
  \item[$(\mathcal{B})$] $\mathcal{B}$ranching law of $\pi\!\mid_{G'}$;
  
  \item[$(\mathcal{C})$] $\mathcal{C}$onstruction of SBOs.
\end{description}
The main theme of this work is part $\mathcal{C}$ for ``standard
representations'' with focus on:
\begin{description}
  \item[$(\mathcal{C}1)$] Construct SBOs;
  
  \item[$(\mathcal{C}2)$] Classify all SBOs;
  
  \item[$(\mathcal{C}3)$] Study functional equations among SBOs;
  
  \item[$(\mathcal{C}4)$] Find residue formulae for SBOs;
  
  \item[$(\mathcal{C}5)$] Find images of SBOs.
\end{description}
The subprogram $(\mathcal{C}1) - (\mathcal{C}5)$ was proposed by
Kobayashi-Speh in their book {\cite{kobayashi2015symmetry}}. Further, they
gave a complete answer to $(\mathcal{C}1) - (\mathcal{C}5)$ for real rank 1
pair $(G, G') = (O (n + 1, 1), O (n, 1))$.

\tmtextbf{Goal}: extend this to higher rank case $(G, G') = (O (p + 1, q + 1),
O (p, q + 1))$. The class of the ``standard'' representations we are working
with are \tmtextbf{degenerate spherical principal series representations}:
\begin{eqnarray}
  & I (\lambda) \assign \tmop{Ind}_P^G (\mathbbm{C}_{\lambda}), \quad \lambda
  \in \mathbbm{C}, &  \nonumber\\
  & J (\nu) \assign \tmop{Ind}_{P'}^{G'} (\mathbbm{C}_{\nu}), \quad \nu \in
  \mathbbm{C}. &  \nonumber
\end{eqnarray}
where $P \subset G$ is the maximal parabolic subgroup with the Levi part
\[ M A \simeq O (p, q) \times \{ \pm 1 \} \times \mathbbm{R}, \]
$P' = P \cap G'$ is a maximal parabolic of $G'$.

\fbox{Conformal viewpoint:} Geometrically $I (\lambda)$ arise from conformal
geometry:
\begin{eqnarray}
  & X \assign \mathbbm{S}^p \times \mathbbm{S}^q / \pm &  \nonumber\\
  & \rightsquigarrow \tmop{Conf} (X) = O (p + 1, q + 1) = G \curvearrowright
  &  \nonumber\\
  & \curvearrowright \mathcal{L}_{\lambda} \rightarrow X = G / P :
  \tmop{conformally} \tmop{equivariant} \tmop{line} \tmop{bundle} & 
  \nonumber\\
  & \rightsquigarrow G \curvearrowright I (\lambda) = C^{\infty} (X,
  \mathcal{L}_{\lambda}) . &  \nonumber
\end{eqnarray}
{\noindent}\tmtextbf{Remark. }Works {\cite{kobayashi2013finite}},
{\cite{kobayashi2014classification}} regarding the part $\mathcal{A}$ (a
priori estimate) for this setting, imply that $\dim (\tmop{Hom}_{G'} (I
(\lambda), J (\nu)))$ is uniformly bounded in $(\lambda, \nu) \in
\mathbbm{C}^2$.{\hspace*{\fill}}{\medskip}


\begin{spacing}{0.8}
	This work aims to generalize the results of \cite{kobayashi2015symmetry}
	regarding the symmetry breaking in the $O(n+1,1)\downarrow O(n,1)$ case. This is joint work with T. Kobayashi.
	What we call ``symmetry breaking operators'' (SBOs, for short) in this work are the $G'$ intertwining operators between the spherical degenerate principal series representations $I(\lambda)$
	of $G=O(p+1,q+1)$ and $J(\nu)$ of its closed subgroup $G'\simeq O(p,q+1)$ induced from maximal parabolic subgroups.
\end{spacing}

\section{Main results}
Below we will denote by $Q_{p+1,q+1}$ the indefinite quadratic form of signature $(p+1,q+1)$ on $\R^{p+1,q+1}$.
$G$ will denote the group of transformations preserving $Q_{p+1,q+1}$. As $G$ preserves $Q_{p+1,q+1}$, it also preserves
$\Xi^{p+1,q+1}:=\mysetn{x\in\R^{p+1,q+1}\setminus\left\{ 0 \right\}}{Q_{p+1,q+1}(x)=0}$、and we will denote the stabilizer subgroup of 
$[e_0+e_{p+q+1}]\in X^{p,q}:=\Xi^{p+1,q+1}/\R^{\times}$ by
$P$. $G'$ is the stabilizer subgroup of $e_p$ in $G$ and $P':=G\cap P$.

\[
	\mbox{Let }X:=G/P\simeq X^{p,q},\quad Y:=\mysetn{[\xi,\eta]\in G/P\simeq X^{p,q}}{\xi_{p}=0}\simeq X^{p-1,q}\]
	\[C:=\mysetn{[\xi,\eta]\in G/P\simeq X^{p,q}}{\xi_{0}=\eta_q}\simeq X^{p-1,q-1}\cup\Xi^{p,q},\quad\left\{ o \right\}:=\left\{ [1,0_{p+q},1] \right\}.
  \]

\begin{theorem}
For $p, q \geqslant 1$, $P'$-invariant closed subspaces of $G/P$ are given as follows:\\
  \begin{figure}[H]
    \centering
    \begin{subfigure}{0.3\textwidth}
	\xymatrix{&X\ar@{-}[ld]\ar@{-}[rd]&\\Y\ar@{-}[rd]&&C\ar@{-}[ld]\\&Y\cap C\ar@{-}[d]&\\&\{o\}&}
	\caption{$p>1$}
    \end{subfigure}
    ~ %add desired spacing between images, e. g. ~, \quad, \qquad, \hfill etc. 
      %(or a blank line to force the subfigure onto a new line)
    \begin{subfigure}{0.3\textwidth}
	\raisebox{40mm}
	{\xymatrix{&X\ar@{-}[ld]\ar@{-}[rd]&\\Y\ar@{-}[rd]&&C\ar@{-}[ld]\\  &\{o\}&}}
	\vspace{0.38cm}
	\caption{$p=1$}
    \end{subfigure}
\end{figure}
\end{theorem}

Define the maps $\Op$ and $\mathcal{S}upp$ via the diagram:
\begin{figure}[H]
	  
	\centerline{\xymatrixcolsep{7pc}\xymatrix{\Hom_{G'}(I(\lambda),J(\nu))\ar[r]^{\simeq} \ar@/^2pc/[rr]^{\mathcal{S}upp}
	&\left( \mathcal{D}'(G/P,\mathcal{L}_{n-\lambda})\otimes\mathbb{C}_\nu \right)^{P'}
	\ar[r]_{F\mapsto \supp(F)}\ar[d]^{\simeq}_{\mbox{rest}}
	&2^{P'\backslash G/P}\\
	&\sol\subset\mathcal{D}'(\R^{p,q})\ar[lu]^{\mbox{Op}}_{\simeq}&
	}}
	  
\end{figure}
\begin{theorem}[construction of SBOs]\label{thm:construction}
	For $S=X,Y,C,$ and $\left\{ o \right\}$, the following operators $R_{\lambda,\nu}^S$ and $\tilde{R}_{\lambda,\nu}^X$ are symmetry breaking operators from 
$\IlambdaGprime$ to $J(\nu)$, which depend holomorphically on $(\lambda,\nu)\in D_S$. Moreover, $\Supp(R_{\lambda,\nu}^S)\subseteq S$, with equality holding generically in $(\lambda,\nu)$.

\hspace{-2.0cm}\begin{tabular}{@{}|@{}l@{}|@{}l@{}|l@{}|@{}l@{}|}
  \hline
  $R_{\lambda,\nu}^S$& $\tmop{Op} : 
  \Sol(\mathbbm{R}^{p, q} ; \lambda, \nu)
  \rightarrow \tmop{Hom}_{G'} (I (\lambda), J (\nu))$ & $D_S\,$ &
  $\,{\Supp} (\cdot)=$\\
  \hline
  $R_{\lambda, \nu}^X =$ & $ \frac{1}{\Gamma \left( \frac{\lambda - \nu}{2} \right) \Gamma \left(
  \frac{\lambda + \nu - n + 1}{2} \right) \Gamma \left( \frac{1 - \nu}{2}
  \right)}{\tmop{Op} \left(| x_p |^{\lambda + \nu - n}
  | \Q |^{- \nu} \right)}$ & $\mathbbm{C}^2$ & 
  %supp-X
$\left\{\begin{array}{llllll}
	\varnothing, &  &  &  &  & p > 1, (\lambda, \nu) \in\mathcal{A}
  \\
  C & \cap & Y & \cap & \{ 0 \}, & p > 1, (\lambda, \nu) \nin 
  \mathcal{A}\\
  \uparrow &  & \uparrow &  & \uparrow & \\
  \mid \mid &  & \backslash\backslash &  & / / & \\
  \varnothing, &  &  &  &  & p = 1, (\lambda, \nu) \in \mathcal{A} \cup \mathcal{X}\\
  C & \cap & Y & \cap & \{ 0 \}, & p = 1, (\lambda, \nu) \nin \mathcal{A} \cup \mathcal{X}\\
  \uparrow &  & \uparrow &  & \uparrow & \\
  \mid \mid &  & \backslash\backslash &  & / / & 
\end{array}\right.$
  \\
  \hline
  $\tilde{R}^X_{\lambda, \nu} =$ & $\frac{1}{\Gamma \left( \frac{\lambda + \nu - n + 1}{2}
  \right) \Gamma \left( \frac{1 - \nu}{2} \right)}{\tmop{Op} \left( | x_p |^{\lambda +
  \nu - n} | \Q |^{- \nu}\right)} $ & $
   \mid \mid \mid$ &
   %supp-tX
$\left\{\begin{array}{llll}
  C & \cap & Y, & p > 1\\
  \uparrow &  & \uparrow & \\
  \mid \mid &  & \backslash\backslash & \\
  \varnothing, &  &  & p = 1, (\lambda, \nu) \in \mathcal{X}- / /\\
  C & \cap & Y, & p = 1, (\lambda, \nu) \nin \mathcal{X}- / /\\
  \uparrow &  & \uparrow & \\
  \mid \mid &  & \backslash\backslash & 
\end{array}\right.$
   \\
  \hline
  $R_{\lambda, \nu}^Y =$ & $\frac{(-1)^k k! q_Y^X (\lambda, \nu)}{\Gamma \left( \frac{\lambda - \nu}{2}
  \right) }{\tmop{Op} \left( \delta^{(2k)}(x_p)
  | \Q |^{- \nu}  \right)}$ & $
  \backslash\backslash$ & 
%supp-P
$\left\{\begin{array}{llllll}
  Y, &  &  &  &  & p = 1, (\lambda, \nu) \in L_{\tmop{even}}\\
  Y & \cap & \{ 0 \}, &  &  & p = 1, \tmop{otherwise}\\
  &  & \uparrow &  &  & \\
  &  & / / &  &  & \\
  Y & \cap & \{ 0 \} & \cap & C. & p > 1\\
  &  & \uparrow &  & \uparrow & \\
  &  & / / - \mid \mid \mid &  & \mid \mid & 
\end{array}\right.$
\\
  \hline
  $R_{\lambda, \nu}^C =$ & $\frac{(-1)^m m! q_C^X (\lambda, \nu)}{\Gamma \left( \frac{\lambda - \nu}{2}
  \right) \Gamma \left( \frac{\lambda + \nu - n + 1}{2} \right) }{\tmop{Op} \left( | x_p |^{\lambda + \nu - n}\delta^{(2m)}\left( \Q \right)
    \right)}$ & $ \mid \mid$ &
%supp-C
    $\left\{\begin{array}{llllll@{}}
  C & \cap & Y & \cap & \{ 0 \}, & p > 1, q : \tmop{odd}\\
  &  & \uparrow &  & \uparrow & \\
  &  & \backslash\backslash &  & / / & \\
  C & \cap & Y, &  &  & p > 1, q : \tmop{even}\\
  &  & \uparrow &  &  & \\
  &  & \backslash\backslash &  &  & \\
  C, &  &  &  &  & p = 1, (\lambda, \nu) \in / / - \bb
  \\
  C & \cap & \{ 0 \}. &  &  & p = 1, \tmop{otherwise}\\
  &  & \uparrow &  &  & \\
  &  & / / &  &  & 
\end{array}\right.$
\\
  \hline
  $R_{\lambda, \nu}^{\{ o \}} =$ & 
  $\tmop{Op} \left( \tilde{C}_{\nu -
  \lambda}^{\lambda - \frac{n - 1}{2}} \left(-\Delta_{\mathbbm{R}^{p - 1, q}}
  \delta_{\mathbbm{R}^{p + q - 1}}, \delta (x_p)\right) \right)
  $ & $
  / /$ & $\{ o \}$\\
  \hline
\end{tabular}
Let us explain the notation in the table.
\begin{itemize}
	\item $\mid \mid \mid \assign \{ (\lambda, \nu) \in \mathbbm{C}^2 \mid \nu \in
	- 2\mathbbm{N} \cup (q + 1 + 2\mathbbm{Z}) \},\quad \backslash\backslash:=\mysetn{(\lambda,\nu)\in\C^2}{\lambda+\nu-n+1\in-2\N}$;
\item $/ / \assign
\{ (\lambda, \nu) \in \mathbbm{C}^2 \mid \lambda - \nu \in
-2\N \},\quad \mid\mid:=\mysetn{(\lambda,\nu)\in\C^2}{\nu\in1+2\N}$;
\item $\mathcal{A}:=//\cap\mid\mid\mid$, $\mathcal{X}:=\backslash\backslash \cap \mid \mid$;
\item $(\lambda, \nu) \in L_{\tmop{even}} \assign \{ (\lambda, \nu) \in / / \mid \nu
\in -\mathbbm{N} \}$ as in \cite{kobayashi2015symmetry};
\item $\tilde{C}(s,t)$ is a polynomial of two-variable's, which obtained by inflation of the renormalized Gegenbauer polynomial, defined as in \cite[(16.3)]{kobayashi2015symmetry}.
\end{itemize}
\end{theorem}
We set $m:=\frac{1}{2}\left( \nu-1 \right)\in\N$ for $(\lambda,\nu)\in\mm$ and $k:=\frac{1}{2}\left( n-1-\lambda-\nu \right)\in\N$ for $(\lambda,\nu)\in\bb$.
For $p=1$ we define $q_C^X(\lambda,\nu)$ and $q_Y^X(\lambda,\nu)$ by
\ExecuteMetaData[.master_extract.tex]{residue}
\vspace{-3ex}
\begin{theorem}[classification of SBOs]
  We have\vspace{-1.5ex}
\[ p = 1 \Rightarrow \SBO = \left\{
   \begin{array}{ll}
     \mathbbm{C}R^X_{\lambda, \nu}, & (\lambda, \nu) \nin (/ /
     \cap \mid \mid \mid) - (\mid \mid \cap \backslash\backslash),\\
     \mathbbm{C} \tilde{R}^X_{\lambda, \nu} \oplus \mathbbm{C}R^{\{ o
     \}}_{\lambda, \nu}, & (\lambda, \nu) \in (/ / \cap \mid \mid \mid) -
     (\mid \mid \cap \backslash\backslash),\\
     \mathbbm{C}R^P_{\lambda, \nu} \oplus \mathbbm{C}R^C_{\lambda, \nu}, &
     (\lambda, \nu) \in (\mid \mid \cap \backslash\backslash) - / /,\\
     \mathbbm{C}R^{\{ o \}}_{\lambda, \nu}, & (\lambda, \nu) \in \mid \mid
     \cap \backslash\backslash \cap / /.
   \end{array} \right. \]
\[ p > 1 \Rightarrow \SBO = \left\{
   \begin{array}{ll}
     \mathbbm{C} \tilde{R}^X_{\lambda, \nu} \oplus \mathbbm{C}R^{\{ o
     \}}_{\lambda, \nu \lambda, \nu}, & (\lambda, \nu) \in / / \cap \mid \mid
     \mid,\\
     \mathbbm{C}R^X_{\lambda, \nu}, & \tmop{otherwise.}
   \end{array} \right. \]
\end{theorem}
\vspace{-2ex}
\begin{theorem}[spectrum for spherical vectors]
Let $\mathbbm{1}_\lambda\in I(\lambda)^K,\mathbbm{1}_\nu\in J(\nu)^{K'}$ be the spherical vectors normalized so that $\mathbbm{1}_\lambda(e)=\mathbbm{1}_\nu(e)=1$. We then have\vspace{-3.0ex}
	\[ \OpR^X_{\lambda, \nu} 1_{\lambda} = 2^{1 -
     \lambda} \frac{\pi^{n / 2}}{\Gamma \left( \frac{\lambda}{2} \right)
     \Gamma \left( - \frac{q}{2} + \frac{\lambda + 1}{2} \right) \Gamma \left(
     \frac{q - \nu + 1}{2} \right)} 1_{\nu}.\]
\end{theorem}
\vspace{-2ex}
\begin{theorem}[residue formula]
	For $(\lambda,\nu)\in//$, we set $l:=\frac{1}{2}\left( \nu-\lambda \right)\in\N$. Then we have
  \[\tilde{R}_{\lambda,\nu}^X  = \frac{ (- 1)^l l!\pi^{(n - 2) / 2} 
		}{2^{ \nu + 2 l-1}}\cdot  \frac{\sin\left( \frac{1+q-\nu}{2}\pi \right)}{\Gamma\left( \frac{\nu}{2} \right)}
	\tilde{R}_{\lambda,\nu}^{ \left\{ o \right\} }.\]
	\end{theorem}
	\begin{definition}
		Let $n:=p+q\;(p,q\ge1)$ as before.
		The {\it Knapp-Stein operator} is a $G$-intertwining holomorphic operator defined as
		\begin{equation*}
		\tilde{\mathbb{T}}_{\lambda}^G:I(\lambda)\ni f\mapsto q_T(\lambda)\myabs{\Q}^{\lambda-n}\Conv f\in I(n-\lambda)
		\end{equation*}
		where $q_T(\lambda,\nu)$ is explicitly given by Gamma factors.
		For $G'=O(p,q+1)$ we similarly define $\tilde{\mathbb{T}}^{G'}_\nu:J(\nu)\to J(n-1-\nu)$.
	\end{definition}
	\begin{theorem}
		
		We have:
\begin{eqnarray}
    & \tilde{\mathbbm{T}}^{G'}_{n-1 - \nu} \circ \tilde{R}^X_{\lambda, n' - \nu} =\pi^{\frac{n - 3}{2}} q^{T X}_X
  (\lambda, \nu) \tilde{R}^X_{\lambda, \nu}, &  \nonumber\\
  & \tilde{R}_{n - \lambda, \nu}^X \circ \tilde{\mathbbm{T}}^G_{\lambda} =2^{2\lambda-n}\pi^{-\frac{n}{2}-1} q^{X T}_X
  (\lambda, \nu) \tilde{R}_{\lambda, \nu}^X, &  \nonumber
  \end{eqnarray}
  where
  \begin{gather*}
  q^{T X}_X (\lambda, \nu) \assign\frac{\sin\left( \frac{p-\nu}{2} \pi\right)}{\Gamma\left( \frac{n-1-\nu}{2} \right)} \ \left\{
	  \begin{array}{@{}l@{}l}
    {\Gamma\left( \frac{n / 2 - \nu}{2} \right)}, & \frac{n - 1}{2}+p\todd,\\
    \dots&
  \end{array} \right.
\end{gather*}
and $q_X^{TX}(\lambda,\nu)$ is similarly explicitly given by Gamma functions.
	\end{theorem}
	\begin{theorem}
		For every $(\lambda,\nu)\in\C^2$ we have computed image of every SBO constructed in Theorem \ref{thm:construction}.
	\end{theorem}
\begin{theorem}[$G'$-invariant maps between Zuckerman modules $\pi_{\pm,\lambda}^{p,q}$]\label{thm:Aq}
	Let $n=p+q,\;(p,q\ge1)$ and $n':=n-1$.
	The dimensions of $\Hom_{G'}\left(\pi_{\pm,{n}/{2}-\lambda}^{p+1,q+1}\kern-0.3em\mid_{G'} ,\pi_{\pm,\nu-{n'}/{2}}^{p,q+1} \right)$
	are as follows:\newline
\ExecuteMetaData[.master_extract.tex]{Aq}\\\vspace{\baselineskip}
\end{theorem}
\begin{remark}
Theorem \ref{thm:Aq} generalizes \cite[Thms. 12.1 and 1.3]{kobayashi2015symmetry}.
\end{remark}
\nocite{kobayashi2015program}
\small
\bibliography{todai_master}
\bibliographystyle{mystyle}
\end{document}
%TODO:
