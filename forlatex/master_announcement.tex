%platex
\documentclass[12pt]{msjproc} % use larger type; default would be 10pt

\usepackage{enumerate}
\usepackage{setspace}
\usepackage{amsmath,amssymb,bbm,xypic}
\usepackage[all,cmtip]{xy}
\usepackage{amsmath,amssymb,bbm,ulem,float,mystyle}
\usepackage{caption}
\usepackage{subcaption}
\usepackage{setspace}
\usepackage{comment}
%\excludecomment{versiona}
\usepackage{catchfilebetweentags}
\includecomment{versiona}

%%%%%%%%%% Start TeXmacs macros
\catcode`\<=\active \def<{
\fontencoding{T1}\selectfont\symbol{60}\fontencoding{\encodingdefault}}
\catcode`\>=\active \def>{
\fontencoding{T1}\selectfont\symbol{62}\fontencoding{\encodingdefault}}
\newcommand{\assign}{:=}
\newcommand{\comma}{{,}}
\newcommand{\nin}{\not\in}
\newcommand{\tmop}[1]{\ensuremath{\operatorname{#1}}}
\newcommand{\tmtextit}[1]{{\itshape{#1}}}
\newcommand{\um}{-}

\newtheorem{theorem}{Theorem}
\newcommand{\sol}{\mathcal{S}\!{\it ol}(\R^{p,q};\lambda,\nu)}
\newcommand{\Hom}{\mbox{\normalfont Hom}}
\newcommand{\Sol}{\mathcal{S}\!{\it ol}}
\newcommand{\Ind}{\mbox{\normalfont Ind}}
\newcommand{\Supp}{\mathcal{S}\!{\it upp}}
\newtheorem{remark}[theorem]{Remark}
\newtheorem{corollary}[theorem]{Corollary}
\newtheorem{fact}{Fact}
%\newtheorem{definition}{Definition}
\theoremstyle{definition}
\newtheorem{definition}{Definition}

\makeatletter
\newtheoremstyle{exampstyle}
  {\topsep} % Space above
  {\topsep} % Space below
  { {\addtolength{\@totalleftmargin}{3.5cm}
     \addtolength{\linewidth}{-3.5cm}
        \parshape 1 3.5em \linewidth}} % Body font
  {-2.5cm} % Indent amount
  {\bfseries} % Theorem head font
  {.} % Punctuation after theorem head
  {.5em} % Space after theorem head
  {} % Theorem head spec (can be left empty, meaning `normal')
\makeatother

\theoremstyle{exampstyle} \newtheorem{examp}[theorem]{Theorem}

\catcode`\<=\active \def<{
\fontencoding{T1}\selectfont\symbol{60}\fontencoding{\encodingdefault}}
\catcode`\>=\active \def>{
\fontencoding{T1}\selectfont\symbol{62}\fontencoding{\encodingdefault}}
\newcommand{\dueto}[1]{\textup{\textbf{(#1) }}}
\newcommand{\tmrsub}[1]{\ensuremath{_{\textrm{#1}}}}
\newcommand{\tmrsup}[1]{\textsuperscript{#1}}
\newcommand{\tmtextbf}[1]{{\bfseries{#1}}}
\newtheorem{proposition}{Proposition}
\newcommand{\Op}{\mbox{\normalfont Op}}
\newcommand{\Res}{\operatorname{Res}\displaylimits}
\newcommand{\OpR}{\mbox{\it R}}
\renewcommand{\Q}{Q_{p,q}}
\newcommand{\IlambdaGprime}{I(\lambda)\kern-0.3em\mid_{G'}}
\newcommand{\SBO}{\Hom_{G'}\left(\IlambdaGprime,J(\nu) \right)}
\renewcommand{\setminus}{-}
%%%%%%%%%% End TeXmacs macros

\setlength{\parskip}{0.4em}
\setlength{\parindent}{2em}

\newcommand{\even}{2\Z}
\newcommand{\odd}{2\Z+1}
\newcommand{\teven}{\mbox{\textrm{: even}}}
\newcommand{\todd}{\mbox{\textrm{: odd}}}
\newcommand{\tevenText}[1]{\vspace{-3cm}$\begin{array}{l}\nu\teven\\\nu#1\end{array}$}
\newcommand{\toddText}[1]{\vspace{-3cm}$\begin{array}{l}\nu\todd\\\nu#1\end{array}$}
\newcommand{\mm}{\mid\mid}
\newcommand{\bb}{\backslash\backslash}
\renewcommand{\ss}{//}

\begin{document}

\title{Symmetry breaking of indefinite orthogonal groups $O(p,q)$}

  %%%% 講演者1
  \author{小林俊行}{東京大学}
  \author{レオンチエフ アレックス}{東京大学}

  %%%% 講演者2

  %%%% 日付
%  \date{2012年3月26日}

  %%%% 謝辞、キーワード、MSCコード  

  \maketitle

\begin{spacing}{0.8}
	This work aims to generalize the results of \cite{kobayashi2015symmetry}
	regarding the symmetry breaking in the $O(n+1,1)\downarrow O(n,1)$ case. This is joint work with T. Kobayashi.
	What we call ``symmetry breaking operators'' (SBOs, for short) in this work are the $G'$ intertwining operators between the spherical degenerate principal series representations $I(\lambda)$
	of $G=O(p+1,q+1)$ and $J(\nu)$ of its closed subgroup $G'\simeq O(p,q+1)$ induced from maximal parabolic subgroups.
\end{spacing}

Below we will denote by $Q_{p+1,q+1}$ the indefinite quadratic form of signature $(p+1,q+1)$ on $\R^{p+1,q+1}$.
$G$ will denote the group of transformations preserving $Q_{p+1,q+1}$. As $G$ preserves $Q_{p+1,q+1}$, it also preserves
$\Xi^{p+1,q+1}:=\mysetn{x\in\R^{p+1,q+1}\setminus\left\{ 0 \right\}}{Q_{p+1,q+1}(x)=0}$、and we will denote the stabilizer subgroup of 
$[e_0+e_{p+q+1}]\in X^{p,q}:=\Xi^{p+1,q+1}/\R^{\times}$ by
$P$. $G'$ is the stabilizer subgroup of $e_p$ in $G$ and $P':=G\cap P$.

\[
	\mbox{Let }X:=G/P\simeq X^{p,q},\quad Y:=\mysetn{[\xi,\eta]\in G/P\simeq X^{p,q}}{\xi_{p}=0}\simeq X^{p-1,q}\]
	\[C:=\mysetn{[\xi,\eta]\in G/P\simeq X^{p,q}}{\xi_{0}=\eta_q}\simeq X^{p-1,q-1}\cup\Xi^{p,q},\quad\left\{ [o] \right\}:=\left\{ [1,0_{p+q},1] \right\}.
  \]

\begin{theorem}
$p, q \geqslant 1$に対して、
  $G/P$の$P'$不変閉集合と{包含関係}が以下の通りになる:\\
  \begin{figure}[H]
	  
    \centering
    \begin{subfigure}{0.3\textwidth}
	\xymatrix{&X\ar@{-}[ld]\ar@{-}[rd]&\\Y\ar@{-}[rd]&&C\ar@{-}[ld]\\&Y\cap C\ar@{-}[d]&\\&\{[0]\}&}
	\caption{$p>1$の時}
    \end{subfigure}
    ~ %add desired spacing between images, e. g. ~, \quad, \qquad, \hfill etc. 
      %(or a blank line to force the subfigure onto a new line)
    \begin{subfigure}{0.3\textwidth}
	\raisebox{40mm}
	{\xymatrix{&X\ar@{-}[ld]\ar@{-}[rd]&\\Y\ar@{-}[rd]&&C\ar@{-}[ld]\\&\{[0]\}&}}
	\caption{$p=1$の時}
    \end{subfigure}
\end{figure}
\end{theorem}

以下では
の2つの写像$\Op$と$\mathcal{S}$を定義する:
\begin{figure}[H]
	  
	\centerline{\xymatrixcolsep{7pc}\xymatrix{\Hom_{G'}(I(\lambda),J(\nu))\ar[r]^{\simeq} \ar@/^2pc/[rr]^{\mathcal{S}upp}
	&\left( \mathcal{D}'(G/P,\mathcal{L}_{n-\lambda})\otimes\mathbb{C}_\nu \right)^{P'}
	\ar[r]_{F\mapsto \supp(F)}\ar[d]^{\simeq}_{\mbox{rest}}
	&2^{P'\backslash G/P}\\
	&\sol\subset\mathcal{D}'(\R^{p,q})\ar[lu]^{\mbox{Op}}_{\simeq}&
	}}
	  
\end{figure}
\begin{theorem}[construction of SBOs]\label{thm:construction}
	For $S=X,Y,C,$ and $\left\{ o \right\}$, the following operators $R_{\lambda,\nu}^S$ and $\tilde{R}_{\lambda,\nu}^X$ are symmetry breaking operators from $\IlambdaGprime$ to $J(\nu)$, which depend holomorphically on $(\lambda,\nu)\in D_S$. Moreover, $\Supp(R_{\lambda,\nu}^S)\subset S$, and are given explicitly as follows.\\
\ExecuteMetaData[.master_extract.tex]{table}\vspace{\baselineskip}
Let us explain the notation in the table.
\begin{itemize}
	\item $\mid \mid \mid \assign \{ (\lambda, \nu) \in \mathbbm{C}^2 \mid \nu \in
	- 2\mathbbm{N} \cup (q + 1 + 2\mathbbm{Z}) \},\quad \backslash\backslash:=\mysetn{(\lambda,\nu)\in\C^2}{\lambda+\nu-n+1\in-2\N}$;
\item $/ / \assign
\{ (\lambda, \nu) \in \mathbbm{C}^2 \mid \lambda - \nu \in
-2\N \},\quad \mid\mid:=\mysetn{(\lambda,\nu)\in\C^2}{\nu\in1+2\N}$;
\item $\tilde{C}(s,t)$ is a polynomial of two-variable's, which obtained by inflation of the renormalized Gegenbauer polynomial, defined as in \cite[(16.3)]{kobayashi2015symmetry}.
\end{itemize}
\end{theorem}
We set $m:=\frac{1}{2}\left( \nu-1 \right)\in\N$ for $(\lambda,\nu)\in\mm$ and $k:=\frac{1}{2}\left( n-1-\lambda-\nu \right)\in\N$ for $(\lambda,\nu)\in\bb$.
For $p=1$ we define $q_C^X(\lambda,\nu)$ and $q_Y^X(\lambda,\nu)$ by
\ExecuteMetaData[.master_extract.tex]{residue}
\begin{theorem}[対称性破れ作用素の分類]
  $p > 1$に対して
  \begin{eqnarray}
	  & \Hom_{G'}(I(\lambda),J(\nu))= \left\{
    \begin{array}{ll}
      \mathbbm{C} {\OpR}_{\lambda, \nu}^{X} \oplus \mathbbm{C}
      {\OpR}^{\{ 0 \}}_{\lambda, \nu}, & (\lambda, \nu) \in / /\cap 
      L\\
      \mathbbm{C} \OpR^X_{\lambda, \nu}, &
      \mbox{\normalfont otherwise.}
    \end{array} \right. &  \nonumber
  \end{eqnarray}
  特に、$\dim\Hom_{G'}(I(\lambda),J(\nu))=1\iff(\lambda,\nu)\in//\cap L$。
  ここで、$//\assign \{ (\lambda, \nu) \in \mathbbm{C}^2 |
  \lambda - \nu = - 2 k \in - 2\mathbbm{Z}_{\geqslant 0} \}$。$L\subset\mathbb{C}^2$は
  複数余次元$1$の部分集合(具体形は省略する)。
  $p=1$に対して同様の命題が成り立つ。
\begin{versiona}
\[ \mathcal{S} \tmop{ol} (\mathbbm{R}^n ; \lambda, \nu) = \left\{
   \begin{array}{ll}
     \tilde{K}_{\lambda, \nu}^{\mathbbm{R}^n}, & (\lambda, \nu) \in
     \mathbbm{C}^2 - (/ / \cap L) - (| | \cap \backslash\backslash)\\
     \widetilde{\tilde{K}}_{\lambda, \nu}^{\mathbbm{R}^n} \oplus \tilde{K}^{\{
     0 \}}_{\lambda, \nu}, & (\lambda, \nu) \in (/ / \cap L) - (| | \cap
     \backslash\backslash)\\
     \tilde{K}_{\lambda, \nu}^P \oplus \tilde{K}_{\lambda, \nu}^C, & (\lambda,
     \nu) \in (| | \cap \backslash\backslash) - / /\\
     \tilde{K}^{\{ 0 \}}_{\lambda, \nu}, & (\lambda, \nu) \in \mid
     \mid \cap \backslash\backslash \cap / /
   \end{array} \right. \]
\[ L \assign \left\{ \begin{array}{ll}
     \{ (\lambda, \nu) \in \mathbbm{C}^2 | \nu \in \mathbbm{Z}_{\leqslant 0}
     \cup (2\mathbbm{Z}_{\geqslant 0} + 1) \}, & q \in 2\mathbbm{Z}\\
     \{ (\lambda, \nu) \in \mathbbm{C}^2 | \nu \in 2\mathbbm{Z} \}, & q \in
     2\mathbbm{Z}+ 1.
   \end{array} \right. \]
\end{versiona}
\end{theorem}

\begin{theorem}[$K$不変ベクトルにおける``固有値'']
	正規化された$K$不変ベクトル$1_{\lambda} \in I (\lambda)$と$K'$不変ベクトル$1_{\nu} \in I (\nu)$に対して
	\[ \OpR^X_{\lambda, \nu} 1_{\lambda} = 2^{1 -
     \lambda} \frac{\pi^{n / 2}}{\Gamma \left( \frac{\lambda}{2} \right)
     \Gamma \left( - \frac{q}{2} + \frac{\lambda + 1}{2} \right) \Gamma \left(
     \frac{q - \nu + 1}{2} \right)} 1_{\nu} \quad\mbox{が成り立つ。}\]
\end{theorem}
\begin{versiona}
	\begin{remark}
\[ \begin{array}{ll}
     \int_{x, y = - 1}^1 | x - y |^{- \nu} (1 - x^2)^A (1 - y^2)^B d x d y =
     \frac{\Gamma \left( \frac{1 - \nu}{2} \right) \hspace{0.25em} \sqrt{\pi} 
     \hspace{0.25em} \Gamma (A + 1)  \hspace{0.25em} \Gamma (B + 1) 
     \hspace{0.25em} \Gamma (B + A - \nu + 2)}{\Gamma \left( \frac{2
     \hspace{0.25em} A - \nu + 3}{2} \right)  \hspace{0.25em} \Gamma \left(
     \frac{2 \hspace{0.25em} B - \nu + 3}{2} \right)  \hspace{0.25em} \Gamma
     \left( \frac{2 \hspace{0.25em} B + 2 \hspace{0.25em} A - \nu + 4}{2}
     \right)} & \\
     \int_{x, y = - 1}^1 | x - y |^{- \nu} \tmop{sgn} (x - y) (1 - x^2)^A y (1
     - y^2)^B d x d y = - \frac{\Gamma \left( 1 - \frac{\nu}{2} \right) 
     \hspace{0.25em} \sqrt{\pi}  \hspace{0.25em} \Gamma (A + 1) 
     \hspace{0.25em} \Gamma (B + 1)  \hspace{0.25em} \Gamma (B + A - \nu +
     2)}{\Gamma \left( \frac{2 \hspace{0.25em} A - \nu + 2}{2} \right) 
     \hspace{0.25em} \Gamma \left( \frac{2 \hspace{0.25em} (B + 1) - \nu +
     2}{2} \right)  \hspace{0.25em} \Gamma \left( \frac{2 \hspace{0.25em} (B +
     1) + 2 \hspace{0.25em} A - \nu + 3}{2} \right)} & 
   \end{array} \]
	\end{remark}
\end{versiona}

\begin{theorem}[留数定理]
  $(\lambda, \nu) \in / / $に対して
  $\OpR^{X}_{\lambda, \nu} = q_{\{ 0 \}}^{X}
  (\lambda, \nu) \OpR^{\{ 0 \}}_{\lambda, \nu}$が成り立つ。
\end{theorem}
\begin{versiona}
\begin{enumerate}
  \item For $(\lambda, \nu) \in / / \assign \{ (\lambda, \nu) \in
  \mathbbm{C}^2 | \lambda - \nu = - 2 k \in - 2\mathbbm{Z}_{\geqslant 0} \}$
  we have $\tilde{K}^{\mathbbm{R}^n}_{\lambda, \nu} = q_{\{ 0
  \}}^{\mathbbm{R}^n} (\lambda, \nu) \tilde{K}^{\{ 0 \}}_{\lambda, \nu}$,
  where
  \[ q_{\{ 0 \}}^{\mathbbm{R}^n} (\lambda, \nu) = 2^{- \nu - 2 l} (- 1)^l l!
     \pi^{(n - 4) / 2} \Gamma \left( 1 - \frac{\nu}{2} \right) \left[ \sin
     \frac{\pi q}{2} + \sin (\pi (\nu - q / 2)) \right] ; \]
  \item For $(\lambda, \nu) \in \mid \mid \assign \{ (\lambda, \nu) \in
  \mathbbm{C}^2 | \nu = - 1 - 2 k \in - 1 - 2\mathbbm{Z}_{\geqslant 0} \}$ we
  have $\tilde{K}^{\mathbbm{R}^n}_{\lambda, \nu} = q_C^{\mathbbm{R}^n}
  (\lambda, \nu) \tilde{K}^C_{\lambda, \nu}$, where
  \[ q_C^{\mathbbm{R}^n} (\lambda, \nu) = \frac{(- 1)^k k!}{(2 k) !} \left\{
     \begin{array}{ll}
       1, & q \in 2\mathbbm{Z}+ 1\\
       \Gamma \left( \frac{\lambda - \min \{ \nu, q - \nu \}}{2} \right) /
       \Gamma \left( \frac{\lambda - \nu}{2} \right), & q \in 2\mathbbm{Z}, p
       = 1\\
       1 / \Gamma \left( \frac{\lambda - \nu}{2} \right), & q \in
       2\mathbbm{Z}, p > 1
     \end{array} \right. ; \]
  \item For $(\lambda, \nu) \in \backslash\backslash \assign \{ (\lambda, \nu)
  \in \mathbbm{C}^2 | \lambda + \nu - n = - 1 - 2 k \in - 1 -
  2\mathbbm{Z}_{\geqslant 0} \}$ we have $\tilde{K}^{\mathbbm{R}^n}_{\lambda,
  \nu} = q_P^{\mathbbm{R}^n} (\lambda, \nu) \tilde{K}^P_{\lambda, \nu}$, where
  \[\hspace*{-3cm} q_C^{\mathbbm{R}^n} (\lambda, \nu) = \frac{(- 1)^k k!}{(2 k) !} \left\{
     \begin{array}{ll}
       1, & n \in 2\mathbbm{Z}, p > 1\\
       \frac{\Gamma \left( \frac{\max \{ 2, ((n - 1) / 2 - k)' \} - \nu}{2} \right)}{\Gamma \left( \frac{\lambda - \nu}{2} \right)}
       , & n-1,q \in 2\mathbbm{Z}+
       1, p > 1 ; \; a' \assign \min \{ x | a \leqslant x
       \in 2\mathbbm{Z} \}\\
       \frac{\Gamma \left( \frac{((n - 1) / 2 - k)' - \nu}{2} \right) }{ \Gamma
       \left( \frac{\lambda - \nu}{2} \right)}
       , & n \in 2\mathbbm{Z}+ 1, q \in
       2\mathbbm{Z}+ 1, p > 1 ; \; a' \assign \min \{ x | a \leqslant x \in
       2\mathbbm{Z}+ 1 \}\\
       1 / \Gamma \left( \frac{1 - \nu}{2} \right), & p = 1, n \in
       2\mathbbm{Z}\\
       \frac{\Gamma \left( \max \left\{ \frac{n - 1}{2} - k, 0 \right\} - \nu
       \right)}{\Gamma \left( \frac{1 - \nu}{2} \right)\Gamma \left(
       \frac{\lambda - \nu}{2} \right)}
        , & p = 1, n \in 2\mathbbm{Z}+ 1
     \end{array} \right. \]
\end{enumerate}
\end{versiona}
\begin{theorem}[factorization identities]
  $\tilde{\mathbbm{T}}_{\lambda} : I (\lambda) \rightarrow I (n -
  \lambda)$ をKnapp-Stein作用素とする。$(\lambda, \nu) \in \mathbbm{C}^2$に対して
  $\tilde{\mathbbm{T}}_{n - 1 - \nu} \circ \OpR_{\lambda,
    n - 1 - \nu}^{X} = q^{T X}_{X}
    (\lambda, \nu) \OpR_{\lambda, \nu}^{X}$と$ \OpR_{n - \lambda, \nu}^X \circ
    \tilde{\mathbbm{T}}_{\lambda} = q^{X T}_{X}
    (\lambda, \nu) \OpR_{\lambda, \nu}^{X}$が成り立つ。
\end{theorem}
\begin{remark}
	上に記述された $N_C(\lambda,\nu)$、$N_Y(\lambda,\nu)$、$q_{ \left\{ 0 \right\}}^{X}(\lambda,\nu)$、$q_X^{TX}(\lambda,\nu)$、$q_X^{XT}(\lambda,\nu)$
	は$\Gamma$関数を用いて明示的に記述される。
\end{remark}
\begin{versiona}
	
The following holds:

\begin{eqnarray}
  & \tilde{\mathbbm{T}}_{n - 1 - \nu} \circ \tmop{Op} (\tilde{K}_{\lambda, n
  - 1 - \nu}^{\mathbbm{R}^n}) = q^{T\mathbbm{R}^n}_{\mathbbm{R}^n} (\lambda,
  \nu) \tmop{Op} (\tilde{K}_{\lambda, \nu}^{\mathbbm{R}^n}) &  \nonumber\\
  & \tmop{Op} (\tilde{K}_{n - \lambda, \nu}^{\mathbbm{R}^n}) \circ
  \tilde{\mathbbm{T}}_{\lambda} = q^{\mathbbm{R}^n T}_{\mathbbm{R}^n}
  (\lambda, \nu) \tmop{Op} (\tilde{K}_{\lambda, \nu}^{\mathbbm{R}^n}) & 
  \nonumber\\
  & q^{T\mathbbm{R}^n}_{\mathbbm{R}^n} (\lambda, \nu) \assign \pi^{\frac{n -
  1}{2}} \left\{ \begin{array}{ll}
    \frac{\sqrt{\pi} 2^{1 - q + \nu}}{\Gamma \left( \frac{q - \nu}{2} \right)
    \Gamma \left( \frac{\nu + 1}{2} \right)}, & p = 1\\
    \frac{\sin \left( \pi \frac{p - \nu}{2} \right)}{\pi} \times
    \frac{\sqrt{\pi} 2^{2 - n + \nu}}{\Gamma \left( \frac{n - 1 - \nu}{2}
    \right)}, & n - 1 \in 1 + 2\mathbbm{Z}\\
    \frac{\sin \left( \pi \frac{p - \nu}{2} \right)}{\pi} \times \frac{\Gamma
    \left( \frac{n / 2 - \nu}{2} \right)}{\Gamma \left( \frac{n - 1 - \nu}{2}
    \right)}, & \frac{n - 1}{2} + p - 1 \in 2\mathbbm{Z}\\
    \frac{\sin \left( \pi \frac{p - \nu}{2} \right)}{\pi} \times \frac{\Gamma
    \left( \frac{n / 2 - \nu - 1}{2} \right)}{\Gamma \left( \frac{n - 1 -
    \nu}{2} \right)}, & \frac{n - 1}{2} + p - 1 \in 1 + 2\mathbbm{Z}
  \end{array} \right. &  \nonumber\\
  & q^{\mathbbm{R}^n T}_{\mathbbm{R}^n} (\lambda, \nu) \assign \frac{2^{2
  \lambda - n} \pi^{- n / 2}}{\Gamma \left( \frac{n - \lambda}{2} \right)}
  \times \frac{\sin \left[ \pi \frac{p - \lambda + 1}{2} \right]}{\pi} \times
  \left\{ \begin{array}{ll}
    2^{1 - \lambda} \sqrt{\pi}, & n \in 2\mathbbm{Z}+ 1\\
    \Gamma \left( \frac{\lambda - n / 2 + 1}{2} \right), & n / 2 + p \in
    2\mathbbm{Z}\\
    \Gamma \left( \frac{\lambda - n / 2}{2} \right), & n / 2 + p \in
    2\mathbbm{Z}+ 1
  \end{array} \right. &  \nonumber
\end{eqnarray}
\end{versiona}
\begin{theorem}[対称性破れ作用素の像]
%%	$\nu\notin\Z$ならば、$R_{\lambda,\nu}^X$が全射になる。与えられた$\nu\in\Z$に対した、$R_{\lambda,\nu}^X$の像
%%	を具体的に決定できる。
	$\nu$が整数でなければ、$J(\nu)$は既約である。
	この場合は$R_{\lambda,\nu}^X$は$(\mathfrak{g},K)$-加群のレベルで全射となる。
	$\nu$が既約でない場合は全射とは限らないが、その像が、
	どのような$G'$の表現としてどのような表現であるか $(\mathfrak{g},K)$-加群のレベルで
	完全に記述することができる。
\end{theorem}
\begin{versiona}
	
\begin{theorem}
  \label{images:prop-criterion}Let $\tilde{K}_{\lambda, \nu}^{\mathbbm{R}^n}
  \in \mathcal{S} \tmop{ol} (\mathbbm{R}^n ; \lambda, \nu)$ be kernel of''
  regular SBO for $(\lambda, \nu) \in \mathbbm{C}^2$. Then, for
  $\pi_{\lambda}, \pi_{\nu}'$ as in lemma \ref{images:lem-bastheor} and $a, b
  \in \mathbbm{Z}_{\geqslant 0}$ with $a + b \in 2\mathbbm{Z}$ (we also assume
  $a = 0, 1$ in $p = 1$ case, as $\mathcal{H}^a (\mathbbm{S}^0) \neq 0
  \Leftrightarrow a \in \{ 0, 1 \}$) we have $(\pi_{\nu}')^{- 1}
  (\mathcal{H}^a (\mathbbm{S}^{p - 1}) \otimes \mathcal{H}^b (\mathbbm{S}^q))
  \tmop{image} (\tmop{Op} (\tilde{K}_{\lambda, \nu}^{\mathbbm{R}^n}))$ iff
  $\forall N \in 2\mathbbm{Z}_{\geqslant 0}$
  \begin{eqnarray}
    & \frac{\Gamma \left( \frac{a + b + \nu}{2} \right) \Gamma \left(
    \frac{\lambda - \nu + N}{2} \right) \Gamma \left( \frac{1 + \lambda + \nu
    - n + N}{2} \right)}{\Gamma \left( \frac{\lambda - \nu}{2} \right) \Gamma
    \left( \frac{1 + \lambda - n + \nu}{2} \right) \Gamma \left( \frac{\nu}{2}
    \right)} \times \frac{1}{\Gamma \left( \frac{1 - a + b - \nu + q}{2}
    \right)  \hspace{0.17em} \Gamma \left( \frac{a + b + \lambda + N}{2}
    \right)  \hspace{0.17em} \Gamma \left( \frac{1 + a - b + \lambda - q +
    N}{2} \right)} = 0 &  \nonumber
  \end{eqnarray}
  and if $(\pi_{\nu}')^{- 1} (\mathcal{H}^a (\mathbbm{S}^{p - 1}) \otimes
  \mathcal{H}^b (\mathbbm{S}^q)) \tmop{image} (\tmop{Op} (\tilde{K}_{\lambda,
  \nu}^{\mathbbm{R}^n}))$, then $(\pi_{\nu}')^{- 1} (\mathcal{H}^a
  (\mathbbm{S}^{p - 1}) \otimes \mathcal{H}^b (\mathbbm{S}^q)) \cap
  \tmop{image} (\tmop{Op} (\tilde{K}_{\lambda, \nu}^{\mathbbm{R}^n})) =
  \varnothing$.
\end{theorem}

\begin{remark}
  We will use the shorthand $\tmop{image} (\tilde{K}_{\lambda, \nu})$ to
  denote $(a, b) \in \mathbbm{Z}_{\geqslant 0}$ with $a + b \in 2\mathbbm{Z}$
  such that $(\pi_{\nu}')^{- 1} (\mathcal{H}^a (\mathbbm{S}^{p - 1}) \otimes
  \mathcal{H}^b (\mathbbm{S}^q)) \subset \tmop{image} (\tmop{Op}
  (\tilde{K}_{\lambda, \nu}^{\mathbbm{R}^n}))$.
\end{remark}

\begin{theorem}
  \label{images:prop-main}Let $\mathcal{I} \assign \{ (a, b) \in
  \mathbbm{Z}_{\geqslant 0} | a + b \in 2\mathbbm{Z} \}$. We then have
  $\tmop{image} (\tilde{K}_{\lambda, \nu}) =\mathcal{I}$ if $\nu \nin
  \mathbbm{Z}$. Moreover:
  
  for $p > 1$:
\newcommand{\noplus}{}
  \begin{enumerate}
    \item Suppose $p$:odd and $q$:even. We then have:
    \begin{enumerate}
      \item If $\nu \in - 2\mathbbm{Z}_{\geqslant 0}$ we have
      \[ \tmop{image} (\tilde{K}_{\lambda, \nu}) = \left\{ \begin{array}{ll}
           A^{+ +} \assign \{ (a, b) \in \mathcal{I} | a + b \leqslant - \nu
           \}, & \tmop{otherwise}\\
           \varnothing, & (\lambda, \nu) \in / /
         \end{array} \right. \]
      \item If $\nu \in - 1 - 2\mathbbm{Z}_{\geqslant 0}$ we have
      \begin{eqnarray}
        & \tmop{image} (\tilde{K}_{\lambda, \nu}) = \left\{ \begin{array}{ll}
          \varnothing, & (\lambda, \nu) \in / /\\
          A^{- +} \cap A^{\noplus + -}, & (\lambda, \nu) \in
          \backslash\backslash - / /\\
          A^{+ -}, & (\lambda, \nu) \nin \backslash\backslash \cup / /
        \end{array} \right. &  \nonumber\\
        & A^{- +} \assign \{ (a, b) \in \mathcal{I} | b - a \leqslant - \nu +
        p - 2 \} &  \nonumber\\
        & \mathcal{I}= (A^{\noplus \noplus + -} - A^{- +}) \sqcup (A^{- +} -
        A^{+ -}) \sqcup (A^{- +} \cap A^{\noplus + -}) &  \nonumber\\
        & A^{\noplus \noplus + -} \assign \{ (a, b) \in \mathcal{I} | a - b
        \leqslant - \nu + q - 1 \} &  \nonumber
      \end{eqnarray}
      \item If $0 < \nu \leqslant \frac{p + q + 1}{2} - 2$, then $\mathcal{I}=
      (A^{\noplus \noplus + -} - A^{- +}) \sqcup (A^{- +} - A^{+ -}) \sqcup
      (A^{- +} \cap A^{\noplus + -})$ and
      \begin{center}
        \begin{tabular}{|c|c|c|}
          \hline
          & $\nu \in 2\mathbbm{Z}$ & $\nu \in 2\mathbbm{Z}+ 1$\\
          \hline
          $(/ / \cup \backslash\backslash)^c$ & $\tmop{full}$ & $A^{+ -}$\\
          \hline
          $\backslash\backslash - / /$ & $\tmop{full}$ & $A^{+ -} \cap A^{-
          +}$\\
          \hline
          $/ / \cap \backslash\backslash, k \geqslant l$ & $\tmop{full}$ &
          $\varnothing$\\
          \hline
        \end{tabular}
      \end{center}
      \item If $\nu = \frac{p + q + 1}{2} - 1$, then $A^{+ -} \cap A^{- +} =
      \varnothing$ and
      \begin{center}
        \begin{tabular}{|c|c|c|}
          \hline
          & $n \in 4\mathbbm{Z}+ 1 \Leftrightarrow \nu \in 2\mathbbm{Z}$ & $n
          \in 4\mathbbm{Z}+ 3 \Leftrightarrow \nu \in 2\mathbbm{Z}+ 1$\\
          \hline
          $(/ / \cup \backslash\backslash)^c$ & $\tmop{full}$ & $A^{+ -}$\\
          \hline
          $/ / \cap \backslash\backslash, k = l$ & $\tmop{full}$ &
          $\varnothing$\\
          \hline
        \end{tabular}
      \end{center}
      \item If $\frac{p + q + 1}{2} \leqslant \nu \leqslant (p + q + 1) - 4$,
      then $A^{+ -} \cap A^{- +} = \varnothing$ and
      
      \begin{center}
        \begin{center}
          \begin{tabular}{|c|c|c|}
            \hline
            & $\nu \in 2\mathbbm{Z}$ & $\nu \in 2\mathbbm{Z}+ 1$\\
            \hline
            $(/ / \cup \backslash\backslash)^c$ & $\tmop{full}$ & $A^{+ -}$\\
            \hline
            $/ / \cap \backslash\backslash, k < l$ & $\tmop{full}$ &
            $\varnothing$\\
            \hline
            $/ / -\backslash\backslash$ & $\tmop{full}$ & $\varnothing$\\
            \hline
          \end{tabular}
        \end{center}
      \end{center}
      
      \item If $\nu > (p + q + 1) - 4$, then $A^{+ -} \cap A^{- +} =
      \varnothing$ and for $A^{- -} \assign \{ (a, b) \in \mathcal{I} | a + b
      \geqslant \nu - n + 3 \}$
      
      \begin{center}
        \begin{center}
          \begin{center}
            \begin{center}
              \begin{tabular}{|c|c|c|}
                \hline
                & $\nu \in 2\mathbbm{Z}$ & $\nu \in 2\mathbbm{Z}+ 1$\\
                \hline
                $(/ / \cup \backslash\backslash)^c$ & $\tmop{full}$ & $A^{+
                -}$\\
                \hline
                $/ / \cap \backslash\backslash, k < l$ & $A^{- -}$ &
                $\varnothing$\\
                \hline
                $/ / -\backslash\backslash$ & $\tmop{full}$ & $\varnothing$\\
                \hline
              \end{tabular}
            \end{center}
          \end{center}
          
          \ 
        \end{center}
      \end{center}
    \end{enumerate}
    \item Suppose $p, q \in 2\mathbbm{Z}+ 1$ (note that this implies
    $\backslash\backslash \cap / / = \varnothing$). Then
    \begin{enumerate}
      \item If $\nu \leqslant 0$, then $A^{\noplus + +} \subset A^{+ -}$ and
      
      \begin{center}
        \begin{center}
          \begin{center}
            \begin{center}
              \begin{tabular}{|c|c|c|}
                \hline
                & $\nu \in 2\mathbbm{Z}$ & $\nu \in 2\mathbbm{Z}+ 1$\\
                \hline
                $(/ / \cup \backslash\backslash)^c$ & $A^{\noplus + +}$ &
                $\tmop{full}$\\
                \hline
                $\backslash\backslash - / /$ & $A^{\noplus + +}$ & $A^{- +}$\\
                \hline
                $/ / -\backslash\backslash$ & $\varnothing$ & $\tmop{full}$\\
                \hline
              \end{tabular}
            \end{center}
          \end{center}
        \end{center}
      \end{center}
      
      \item If $0 < \nu \leqslant p + q - 3$, then
      
      \begin{center}
        \begin{center}
          \begin{center}
            \begin{center}
              \begin{tabular}{|c|c|c|}
                \hline
                & $\nu \in 2\mathbbm{Z}$ & $\nu \in 2\mathbbm{Z}+ 1$\\
                \hline
                $(/ / \cup \backslash\backslash)^c$ & $A^{\noplus + -}$ &
                $\tmop{full}$\\
                \hline
                $\backslash\backslash - / /$ & $A^{\noplus + -}$ & $A^{- +}$\\
                \hline
                $/ / -\backslash\backslash$ & $\varnothing$ & $\tmop{full}$\\
                \hline
              \end{tabular}
            \end{center}
          \end{center}
        \end{center}
      \end{center}
      
      \
      
      \item If $\nu > p + q - 3$, then $A^{- -} \supset A^{- +}$ and
      
      \
      
      \begin{center}
        \begin{center}
          \begin{center}
            \begin{center}
              \begin{tabular}{|c|c|c|}
                \hline
                & $\nu \in 2\mathbbm{Z}$ & $\nu \in 2\mathbbm{Z}+ 1$\\
                \hline
                $(/ / \cup \backslash\backslash)^c$ & $A^{\noplus + -}$ &
                $\tmop{full}$\\
                \hline
                $\backslash\backslash - / /$ & $A^{+ -}$ & $A^{- +}$\\
                \hline
                $/ / -\backslash\backslash$ & $\varnothing$ & $\tmop{full}$\\
                \hline
              \end{tabular}
            \end{center}
          \end{center}
        \end{center}
      \end{center}
      
      \ 
    \end{enumerate}
    \item Suppose $p, q \in 2\mathbbm{Z}$ (note that this implies
    $\backslash\backslash \cap / / = \varnothing$). Then
    \begin{enumerate}
      \item If $\nu \leqslant 0$, then $A^{\noplus + +} \subset A^{- +}$ and
      
      \begin{center}
        \begin{center}
          \begin{center}
            \begin{center}
              \begin{tabular}{|c|c|c|}
                \hline
                & $\nu \in 2\mathbbm{Z}$ & $\nu \in 2\mathbbm{Z}+ 1$\\
                \hline
                $(/ / \cup \backslash\backslash)^c$ & $A^{\noplus + +}$ &
                $A^{+ -}$\\
                \hline
                $\backslash\backslash - / /$ & $A^{\noplus + +}$ & $A^{+ -}$\\
                \hline
                $/ / -\backslash\backslash$ & $\varnothing$ & $\varnothing$\\
                \hline
              \end{tabular}
            \end{center}
          \end{center}
        \end{center}
      \end{center}
      
      \item If $0 < \nu \leqslant p + q - 3$, then
      
      \begin{center}
        \begin{center}
          \begin{center}
            \begin{center}
              \begin{tabular}{|c|c|c|}
                \hline
                & $\nu \in 2\mathbbm{Z}$ & $\nu \in 2\mathbbm{Z}+ 1$\\
                \hline
                $(/ / \cup \backslash\backslash)^c$ & $\tmop{full}$ & $A^{+
                -}$\\
                \hline
                $\backslash\backslash - / /$ & $A^{- +}$ & $A^{+ -}$\\
                \hline
                $/ / -\backslash\backslash$ & $\tmop{full}$ & $\varnothing$\\
                \hline
              \end{tabular}
            \end{center}
          \end{center}
        \end{center}
      \end{center}
      
      \
      
      \item If $\nu > p + q - 3$, then $A^{+ -} \subset A^{- -}$ and
      
      \begin{center}
        \begin{center}
          \begin{center}
            \begin{center}
              \begin{tabular}{|c|c|c|}
                \hline
                & $\nu \in 2\mathbbm{Z}$ & $\nu \in 2\mathbbm{Z}+ 1$\\
                \hline
                $(/ / \cup \backslash\backslash)^c$ & $\tmop{full}$ & $A^{+
                -}$\\
                \hline
                $\backslash\backslash - / /$ & $A^{- +}$ & $A^{+ -}$\\
                \hline
                $/ / -\backslash\backslash$ & $\tmop{full}$ & $\varnothing$\\
                \hline
              \end{tabular}
            \end{center}
          \end{center}
        \end{center}
      \end{center}
      
      \ 
    \end{enumerate}
    \item Suppose $p \in 2\mathbbm{Z}, q \in 2\mathbbm{Z}+ 1$. Then
    \begin{enumerate}
      \item If $\nu \leqslant 0$, then $A^{+ +} \subset A^{+ -}, A^{- +}$ and
      (here and in subsequent we enclose into square brackets that part of the
      criterion that do not cause effect because for the range of $\nu$)
      
      \begin{center}
        \begin{center}
          \begin{tabular}{|c|c|c|c|}
            \hline
            & $\nu \in 2\mathbbm{Z}$ & $\nu \in 2\mathbbm{Z}+ 1$ &
            $\tmop{criterion}$\\
            \hline
            $(/ / \cup \backslash\backslash)^c$ & $A^{+ +}$ & $\tmop{full}$ &
            $\frac{\Gamma \left( \frac{a + b + \nu}{2} \right)}{\Gamma \left(
            \frac{\nu}{2} \right)} \times \frac{1}{\Gamma \left( \frac{1 - a +
            b - \nu + q}{2} \right)}$\\
            \hline
            $\backslash\backslash - / /$ & $A^{+ +}$ & $\tmop{full}$ &
            $\frac{\Gamma \left( \frac{a + b + \nu}{2} \right)}{\Gamma \left(
            \frac{\nu}{2} \right)} \times \frac{1}{\Gamma \left( \frac{1 - a +
            b - \nu + q}{2} \right)} \times \frac{1}{\left[ \Gamma \left(
            \frac{a + b + n - 1 - \nu}{2} \right) \right] \Gamma \left(
            \frac{a - b + p - \nu}{2} \right)}$\\
            \hline
            $/ / \cap \backslash\backslash, k > l$ & $\varnothing$ &
            $\tmop{full}$ & $\frac{\Gamma \left( \frac{a + b + \nu}{2}
            \right)}{\Gamma \left( \frac{\nu}{2} \right)} \times
            \frac{1}{\Gamma \left( \frac{1 - a + b - \nu + q}{2} \right)}
            \times \frac{1}{\Gamma \left( \frac{a + b + \nu}{2} \right) \Gamma
            \left( \frac{1 + a - b - q + \nu}{2} \right)}$\\
            \hline
          \end{tabular}
        \end{center}
      \end{center}
      
      \item If $0 < \nu \leqslant \frac{p + q + 1}{2} - 2$, then
      \begin{center}
        \begin{tabular}{|c|c|c|c|}
          \hline
          & $\nu \in 2\mathbbm{Z}$ & $\nu \in 2\mathbbm{Z}+ 1$ &
          $\tmop{criterion}$\\
          \hline
          $(/ / \cup \backslash\backslash)^c$ & $A^{+ -}$ & $\tmop{full}$ &
          $\Gamma^{- 1} \left( \frac{1 - a + b - \nu + q}{2} \right)$\\
          \hline
          $\backslash\backslash - / /$ & $A^{\noplus + -} \cap A^{- +}$ &
          $\tmop{full}$ & $\Gamma^{- 1} \left( \frac{1 - a + b - \nu + q}{2}
          \right) \left[ \Gamma^{- 1} \left( \frac{a + b + n - 1 - \nu}{2}
          \right) \right] \Gamma^{- 1} \left( \frac{a - b + p - \nu}{2}
          \right)$\\
          \hline
          $/ / \cap \backslash\backslash, k > l$ & $\varnothing$ &
          $\tmop{full}$ & $\Gamma^{- 1} \left( \frac{1 - a + b - \nu + q}{2}
          \right) \left[ \Gamma^{- 1} \left( \frac{a + b + \nu}{2} \right)
          \right] \Gamma^{- 1} \left( \frac{1 + a - b + \nu - q}{2} \right)$\\
          \hline
        \end{tabular}
      \end{center}
      \item If $\nu = \frac{p + q + 1}{2} - 1$, then $A^{+ -} \cap A^{- +} =
      \varnothing$ and
      \begin{center}
        \begin{tabular}{|c|c|c|c|}
          \hline
          & $\nu \in 2\mathbbm{Z}$ & $\nu \in 2\mathbbm{Z}+ 1$ &
          $\tmop{criterion}$\\
          \hline
          $(/ / \cup \backslash\backslash)^c$ & $A^{+ -}$ & $\tmop{full}$ &
          $\Gamma^{- 1} \left( \frac{1 - a + b - \nu + q}{2} \right)$\\
          \hline
          $/ / \cap \backslash\backslash, k = l$ & $\varnothing$ &
          $\tmop{full}$ & $\Gamma^{- 1} \left( \frac{1 - a + b - \nu + q}{2}
          \right) \left[ \Gamma^{- 1} \left( \frac{a + b + \nu}{2} \right)
          \right] \Gamma^{- 1} \left( \frac{1 + a - b + \nu - q}{2} \right)$\\
          \hline
        \end{tabular}
      \end{center}
      \item If $\frac{p + q + 1}{2} \leqslant \nu \leqslant (p + q + 1) - 4$,
      then $A^{+ -} \cap A^{- +} = \varnothing$ and
      
      \begin{center}
        \begin{center}
          \begin{tabular}{|c|c|c|c|}
            \hline
            & $\nu \in 2\mathbbm{Z}$ & $\nu \in 2\mathbbm{Z}+ 1$ &
            $\tmop{criterion}$\\
            \hline
            $(/ / \cup \backslash\backslash)^c$ & $A^{+ -}$ & $\tmop{full}$ &
            $\Gamma^{- 1} \left( \frac{1 - a + b - \nu + q}{2} \right)$\\
            \hline
            $/ / \cap \backslash\backslash, k < l$ & $\varnothing$ &
            $\tmop{full}$ & $\Gamma^{- 1} \left( \frac{1 - a + b - \nu + q}{2}
            \right) \left[ \Gamma^{- 1} \left( \frac{a + b + n - 1 - \nu}{2}
            \right) \right] \Gamma^{- 1} \left( \frac{a - b + p - \nu}{2}
            \right)$\\
            \hline
            $/ / -\backslash\backslash$ & $\varnothing$ & $\tmop{full}$ &
            $\Gamma^{- 1} \left( \frac{1 - a + b - \nu + q}{2} \right) \left[
            \Gamma^{- 1} \left( \frac{a + b + \nu}{2} \right) \right]
            \Gamma^{- 1} \left( \frac{1 + a - b + \nu - q}{2} \right)$\\
            \hline
          \end{tabular}
        \end{center}
      \end{center}
      
      \item If $\nu > (p + q + 1) - 4$, then $A^{+ -} \cap A^{- +} =
      \varnothing$ and
      
      \begin{center}
        \begin{center}
          \begin{center}
            \begin{center}
              \begin{tabular}{|c|c|c|c|}
                \hline
                & $\nu \in 2\mathbbm{Z}$ & $\nu \in 2\mathbbm{Z}+ 1$ &
                $\tmop{criterion}$\\
                \hline
                $(/ / \cup \backslash\backslash)^c$ & $A^{+ -}$ &
                $\tmop{full}$ & $\Gamma^{- 1} \left( \frac{1 - a + b - \nu +
                q}{2} \right)$\\
                \hline
                $/ / \cap \backslash\backslash, k < l$ & $\varnothing$ &
                $\tmop{full}$ & $\Gamma^{- 1} \left( \frac{1 - a + b - \nu +
                q}{2} \right) \Gamma^{- 1} \left( \frac{a + b + n - 1 -
                \nu}{2} \right) \Gamma^{- 1} \left( \frac{a - b + p - \nu}{2}
                \right)$\\
                \hline
                $/ / -\backslash\backslash$ & $\varnothing$ & $\tmop{full}$ &
                $\Gamma^{- 1} \left( \frac{1 - a + b - \nu + q}{2} \right)
                \left[ \Gamma^{- 1} \left( \frac{a + b + \nu}{2} \right)
                \right] \Gamma^{- 1} \left( \frac{1 + a - b + \nu - q}{2}
                \right)$\\
                \hline
              \end{tabular}
            \end{center}
          \end{center}
          
          \ 
        \end{center}
      \end{center}
      
      \ 
    \end{enumerate}
  \end{enumerate}
  for $p = 1$:
  \begin{center}
	  \hspace*{-3cm}\begin{tabular}{|c|c|c|c|c|c|}
      \hline
      $(\lambda, \nu) \in$ & $(/ / \cup \backslash\backslash)^c$ & $/ /
      -\backslash\backslash$ & $\backslash\backslash - / /$ & $/ / \cap
      \backslash\backslash, k < l$ & $/ / \cap \backslash\backslash, k
      \geqslant l$\\
      \hline
      $\nu \leqslant 0, \nu \in 2\mathbbm{Z}$ & $M$ & $\varnothing$ & $M$ &
      $\times$ & $\varnothing$\\
      \hline
      $0 < \nu < q, \nu \in 2\mathbbm{Z}$ & $\tmop{full}$ & $\left\{
      \begin{array}{ll}
        \nu - q \in 2\mathbbm{Z} \Rightarrow & \tmop{full}\\
        \nu - q \in 2\mathbbm{Z}+ 1 \Rightarrow & \varnothing
      \end{array} \right.$ & $\tmop{full}$ & $\tmop{full}$ & $\tmop{full}$\\
      \hline
      $q \leqslant \nu, \nu \in 2\mathbbm{Z}$ & $\left\{ \begin{array}{ll}
        \nu - q \in 2\mathbbm{Z} \Rightarrow & \tmop{full}\\
        \nu - q \in 2\mathbbm{Z}+ 1 \Rightarrow & M
      \end{array} \right.$ & $\left\{ \begin{array}{ll}
        \nu - q \in 2\mathbbm{Z} \Rightarrow & \tmop{full}\\
        \nu - q \in 2\mathbbm{Z}+ 1 \Rightarrow & \varnothing
      \end{array} \right.$ & $M$ & $M$ & $\times$\\
      \hline
      $\nu \leqslant 0, \nu \in 2\mathbbm{Z}+ 1$ & $\tmop{full}$ & $\left\{
      \begin{array}{ll}
        \nu - q \in 2\mathbbm{Z} \Rightarrow & \tmop{full}\\
        \nu - q \in 2\mathbbm{Z}+ 1 \Rightarrow & \varnothing
      \end{array} \right.$ & $\tmop{full}$ & $\times$ & $\varnothing$\\
      \hline
      $0 < \nu < q, \nu \in 2\mathbbm{Z}+ 1$ & $\tmop{full}$ & $\left\{
      \begin{array}{ll}
        \nu - q \in 2\mathbbm{Z} \Rightarrow & \varnothing\\
        \nu - q \in 2\mathbbm{Z}+ 1 \Rightarrow & \tmop{full}
      \end{array} \right.$ & $\varnothing$ & $\varnothing$ & $\tmop{full}$\\
      \hline
      $q \leqslant \nu, \nu \in 2\mathbbm{Z}+ 1$ & $\left\{ \begin{array}{ll}
        \nu - q \in 2\mathbbm{Z} \Rightarrow & \tmop{full}\\
        \nu - q \in 2\mathbbm{Z}+ 1 \Rightarrow & M
      \end{array} \right.$ & $\left\{ \begin{array}{ll}
        \nu - q \in 2\mathbbm{Z} \Rightarrow & \tmop{full}\\
        \nu - q \in 2\mathbbm{Z}+ 1 \Rightarrow & \varnothing
      \end{array} \right.$ & $\varnothing$ & $\varnothing$ & $\times$\\
      \hline
    \end{tabular}
    
  \end{center}
  here we assume $\backslash\backslash = \{ \lambda + \nu - n + 1 = - 2 k \in
  - 2\mathbbm{Z}_{\geqslant 0} \}$, $/ / = \{ \lambda - \nu = - 2 l \in -
  2\mathbbm{Z}_{\geqslant 0} \}$. Moreover, for $\nu \leqslant 0$ we let $M
  \assign \{ b \in \mathbbm{Z} | b \leqslant - \nu \}$, while for $\nu
  \geqslant q$ we let $M \assign \{ b \in \mathbbm{Z} | b \geqslant \nu - q +
  1 \}$.
    \end{theorem} 
\end{versiona}
\begin{remark}
  	上に記載された対称性破れ作用素$\OpR_{\lambda, \nu}^C, \OpR_{\lambda, \nu}^Y$に対して定理4、5、6、7と同様な命題も成り立つ。
\end{remark}
\bibliographystyle{alpha}
\bibliography{todai_master}
\end{document}
