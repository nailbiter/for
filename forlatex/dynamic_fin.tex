\documentclass[8pt]{article} % use larger type; default would be 10pt

%\usepackage[utf8]{inputenc} % set input encoding (not needed with XeLaTeX)
\usepackage[10pt]{type1ec}          % use only 10pt fonts
\usepackage[T1]{fontenc}
%\usepackage{CJK}
\usepackage{graphicx}
\usepackage{float}
\usepackage{CJKutf8}
\usepackage{subfig}
\usepackage{amsmath}
\usepackage{amsfonts}
\usepackage{hyperref}
\usepackage{enumerate}
\usepackage{enumitem}
%\usepackage{amsthm}

\newcommand{\norm}[1]{\|#1\|}

%theorem environments configuration
\newtheorem{theorem}{Theorem}
\newtheorem{lemma}[theorem]{Lemma}
\newtheorem{proposition}[theorem]{Proposition}
\newtheorem{corollary}[theorem]{Corollary}
\newenvironment{proof}[1][Proof]{\begin{trivlist}
\item[\hskip \labelsep {\bfseries #1}]}{\qed\end{trivlist}}
\newenvironment{definition}[1][Definition]{\begin{trivlist}
\item[\hskip \labelsep {\bfseries #1}]}{\end{trivlist}}
\newenvironment{example}[1][Example]{\begin{trivlist}
\item[\hskip \labelsep {\bfseries #1}]}{\end{trivlist}}
\newenvironment{remark}[1][Remark]{\begin{trivlist}
\item[\hskip \labelsep {\bfseries #1}]}{\end{trivlist}}
\newcommand{\qed}{\nobreak \ifvmode \relax \else
\ifdim\lastskip<1.5em \hskip-\lastskip
\hskip1.5em plus0em minus0.5em \fi \nobreak
  \vrule height0.75em width0.5em depth0.25em\fi}

\graphicspath{{./dynamic_fin_img/}}

\title{Special Topics in Dynamical Systems\\Final Report\\}
\author{歐立思\\9822058\\Department of Applied Mathematics\\National Chiao Tung University}
\begin{document}
\begin{CJK}{UTF8}{bsmi}
\maketitle
\end{CJK}
\begin{abstract}
In this report we shall give two proofs of the Reeb Sphere Theorem. Reeb Sphere Theorem states that given $M$ - connected, compact and oriented manifold and $f$ - smooth real-valued function on it, with
only two critical values, then $M$ is homeomorphic to the sphere. The first proof is made under assumption that $f$ is Morse function (that is, matrix of second derivatives is non-singular at critical point), second
without this assumption. Though all the proofs are taken from the references, some simplifications are done (sometimes, trading generality) in order to make proofs more accessible. While some technical lemmas remain unproven,
proofs are contained as references and are elementary. They were taken out of current work rather to save space any typing. Finally, some applications of Reeb Sphere Theorem are discussed, such as Smale Theorem and construction of
exotic 7-sphere.
\end{abstract}
\section{Motivation}
Morse theory is useful and full of fascinating results in the sense that it allows us to draw a conclusion about critical points of Morse function based only on the knowledge of its domain.
Recall that Morse function is real-valued smooth function with minor additional assumption that second derivatives should form nonsingular matrix at critical points. Since these additional assumptions are not very restricting,
the area of applicability of Morse theory is large. Besides, it often gives us quite tight lower bounds on number of critical points again, based only on knowledge of geometry of an underlying domain.\\
Nevertheless, one can in turn ask opposite question - given knowledge about critical points of particular smooth function on a manifold - what can be said about the geometry of a manifold? For example, can we calculate a homotopy type
of a manifold? As a concrete example, consider the manifold and a smooth function defined on it. Assume also that this function has only two critical points, as shown on the figure below\\
\begin{figure}[H]
\centering
\includegraphics[width=0.3\textwidth]{manifold.jpg}
\caption{}
\end{figure}
If our manifold is compact, these two points will be maximum and minimum.
Hence, the index of Morse at one of them is $n$ (dimension of manifold) and $0$ at another one. Morse theory then tells us upper bounds for homotopy numbers of manifold and if it would happen that equalities would be attained, we would
be able to conclude that manifold has homotopy type of a sphere.\\
What we achieved so far is not very impressive, since we still make nontrivial additional assumption that Morse inequalities turn to equalities. Hence, in the light of this it should be even more amazing to the reader, that under the
hypothesis above manifold can be shown to be not only homotopy sphere, but also to be homeomorphic to the sphere. We shall prove this result, which bears the name of a Reeb sphere theorem.
In subsequent Section 3 provides some Historical Background which is interesting, though irrelevant to the main theme.\\
Section 4 is the heart of this work. It describes the proof of the Reeb Sphere Theorem first under the assumption that function under the 
consideration is the Morse function, second with this assumption removed.\\
Chapter 5 outlines a few applications of the Reeb Sphere Theorem, namely outlines its usage in the course of the proof of Smale's Theorem. Secondly,
it is mentioned that Smale's Theorem can be used to construct exotic 7-sphere, that is 7-dimensional manifold, which is homeomorphic to the sphere,
but not diffeomorphic.\\
I would like to point out, that my contribution from the mathematical point of view is zero, since all the proofs are taken from references. What I have done, however, is I have read these proofs and understood them. Below I reproduce
proofs with some further explanations, which (I hope) can make them more accessible. In some cases argument is made more simple by tailoring it to our concrete goal, thus trading generality for simplicity. The proof of few lemmas
is omitted, with references given. Proof of lemmas was omitted where these lemmas matched two conditions: first, statement of lemma is intuitively correct. Second, proof is technically cumbersome (though elementary). In particular,
whenever possible I've been trying to reduce proofs to the statements from \cite{topoFrom}, which I treat as a basic knowledge. All figures have been taken from \cite{diffTopo}.
\section{Acknowledgements}
I would like to express my deep gratitude to Professor Kuo-Chang Chen, who not only pointed out this report topic to me, but also recommended great references \cite{morseTh} and \cite{diffTopo}. Without compact and accessible proofs
found in letter books, I wouldn't manage to write this. Second, I would like to express my gratitude to Professor Yi-Jen Lee (CUHK, Hong Kong) who suggested me to read \cite{topoFrom} around three months ago. Without this, I wouldn't
be able to access even simple proofs of \cite{morseTh} and \cite{diffTopo}.
\section{Mathematical Results}
In subsequent we shall prove the Reeb Sphere Theorem, which when stated formally looks like
\begin{theorem}\emph{(Reeb Sphere Theorem) }Let $M$ be a connected, compact and oriented manifold without boundary. Furthermore, let $f$ be a smooth real-valued function defined on $M$ which has only two critical points. Then $M$ is
	homeomorphic to a sphere.
\end{theorem}
We shall give two proofs of this result: first, based on the assumption that function $f$ is Morse function (that is, matrix of the second derivatives is nonsingular at critical points); second with this assumption removed.\\
For smooth function $f$ between manifolds I will denote its derivative (defined as in \cite[p.2]{diffTopo}) by $df$.
\subsection{Proof of Reeb Sphere Theorem for Morse functions}
Though
from the formal point of view the latter result would be sufficient (that is, with restriction removed), we will prove restricted version as well,
but not as a warm-up (in fact, proofs use quite different mechanism). We want to use an opportunity to re-introduce two well-known results
about Morse functions: Morse Lemma and, as we call it, Pulling Lemma (corresponding to Theorem 3.1 in \cite{morseTh}).\\
Let us fix here the manifold connected compact oriented
under the consideration, $M$ and write $n$ as its dimension. 
Morse Lemma then tells us that in a neighborhood of a critical point we can introduce local coordinates $u_1,u_2,\dots,u_n$ such that
$f=const\pm u_1^2\pm u_2^2\pm\dots\pm u_n^2$. If our manifold is compact, then two critical points of $f$ will be maximum ($f=b$) and minimum ($f=a$). Here, and till the end we name minimum and maximum values as $a$ and $b$ respectively. And we also will name preimages $f^{-1}(a)$ and
$f^{-1}(b)$ as $p$ and $q$ respectively till the end of this report.
In the neighborhood of minimum, under the suitable parametrization, we have
\[f=a+u_1^2+u_2^2+\dots+u_n^2\]
This implies that for some small $\epsilon>0$ the set points $x\in M$ with $f(x)\leq a+\epsilon$ will be diffeomorphic to the (closed) disk. Similarly, set of $x\in M$ with $f(x)\geq b-\epsilon$ is also diffeomorphic to the disk.
At this point, since set $\{x\in M|a+\epsilon\leq f(x) \leq b-\epsilon\}$ Pulling Lemma will assure us that set of points $x$ of $M$ with $f(x)\leq b-\epsilon$ is diffeomorphic to $\{x\in M|f(x)\leq a+\epsilon\}$ and this is
also diffeomorphic to the disk. This situation is depicted on a figure below. (dark "hats" diffeomorphic to the disk and arrows suggest the direction of "pulling")\\
\begin{figure}[H]
\centering
\includegraphics[width=0.3\textwidth]{figure1.jpg}
\caption{}
\end{figure}
Consequently, we see that $M$ is the union of two sets ($A$ and $B$), which both are diffeomorphic to the closed disk ($D^n$) and intersect along a common boundary ($G$). Let's take arbitrary homeomorphism $g:G\rightarrow S^{n-1}$.
The homeomorphism $h_A$ (from now on, forget about differentiable structure) between $A$ and $D^n$ in turn induces another homeomorphism between $G$ and $S^{n-1}$ and consequently (when inverted and
multiplied with $g$ from right) homeomorphism between
$S^{n-1}$ and itself. This homeomorphism can be extended to homeomorphism on $D^n$. This is so, since homeomorphism $h:S^{n-1}\rightarrow S^{n-1}$ can be extended to $H:D^n\rightarrow D^n$ by $H(tu)=th(u)$, the process known as
radial extension. Finally, $h_A\circ H$ is homeomorphism from $A$ to $D^n$, which agrees with $g$ on $G$. Doing same with $B$, we "glue" them to get sphere $S^n$, to which $M$ is homeomorphic. This concludes the proof.\\
\begin{remark}
	Since there is no differentiable analog of a radial extension, differentiable structure is inevitably destroyed during the "gluing". Hence, $M$ is \textit{homeomorphic} to a sphere, not \textit{diffeomorphic}.
\end{remark}
It therefore remains to show, that Morse Lemma and Pulling Lemma are true. We start with the former.
\begin{lemma}{(Lemma of Morse) }Let $p$ be non-degenerate critical point of real-valued $f$ (defined on $n$-dimensional manifold $M$). Then there is a local coordinated system for neighborhood $U$ of $p$ $y_1,y_2,\dots,y_n$ such that
	$f=f(p)-y_1^2-y_2^2-\dots-y_i^2+y_{i+1}^2+\dots+y_n^2$
\end{lemma}
\begin{proof}
	Without loss of generality, $p=0$, $f(p)=0$. Let us begin with some parametrization $x_1,x_2,\dots,x_n$ of neighborhood $U$ of $p$. Then notice that
	\[f(x_1,x_2,\dots,x_n)=\int_0^1 \frac{df(tx_1,tx_2,\dots,tx_n)}{dt}dt=\int_0^1\sum_{i=1}^n \frac{\partial f}{\partial x_i}(tx_1,tx_2,\dots,tx_n)\cdot x_i dt=\]
	\[=\sum_{i=1}^n x_i\int_0^1\frac{\partial f}{\partial x_i}(tx_1,tx_2,\dots,tx_n)=\sum_{i=1}^n x_ig_i(x_1,x_2,\dots,x_n)\]
	Now, since $g_i(0)=\frac{\partial f}{\partial x_i}(0)=0$ ($0$ is critical point) we may repeat operation above to each $g_i(x_1,x_2,\dots,x_n)$ and get
	\[g_i(x_1,x_2,\dots,x_n)=\sum_{j=1}^n x_j h_{ij}(x_1,x_2,\dots,x_n)\]
	hence
	\[f=\sum_{i,j=1}^n x_ix_jh_{ij}(x_1,x_2,\dots,x_n)\]
	Besides, by making a replacement $\bar{h}_{ij}=\bar{h}_{ji}=\frac{1}{2}(h_{ij}+h_{ji})$ when necessary we may assume $h_{ij}=h_{ji}$. Finally, matrix $\{h_{ij}(0)\}_{i,j}$ is the same as matrix of second partial derivatives of
	$f$ at $0$ and hence is non-singular. What happens before is just an imitation of the diagonalization process for quadratic forms (see for example, \url{http://planetmath.org/DiagonalizationOfQuadraticForm.html}).\\
	Suppose by induction that we found coordinates $u_1,u_2,\dots,u_n$ such that
	\[f=u_1^2\pm u_2^2\pm \dots\pm u_{r-1}^2+\sum_{i,j=r}^n u_iu_j H_{ij}(u_1,u_2,\dots,u_n)\]
	throughout $U_1$ neighborhood of 0
	and $\{H_{ij}\}_{i,j\geq r}$ form nonsingular symmetric matrix. If $H_{r,r}=0$, but $H_{j,j}\neq 0$ we may do change of variables $u_r\leftrightarrow u_j$ to get $H_{r,r}\neq 0$. If all $H_{j,j}=0$, (since all coefficients
	cannot be zero - matrix is nonsingular) locate $H_{i,j}u_iu_j\neq 0$ and do linear variable change $u_i=u_i^*+u_j^*,\;u_j=u_i^*-u_j^*$, which will generate $(u_i^*)^2$ and $(u_j^*)^2$ terms. In any case, we may freely
	assume that $H_{r,r}\neq 0$. Let moreover $g=\sqrt{|H_{r,r}|}$, $g$ will be smooth and nonzero in some neighborhood $U_2$ of $0$, smaller than $U_1$. 
	In that neighborhood we may introduce new coordinates
	\[v_i=u_i\;i\neq r\]
	\[v_r(u_1,u_2,\dots,u_n)=g(u_r+\sum_{i>r} u_iH_{ir}/H_{rr})\]
	Since 
	\[\frac{\partial v_r}{\partial u_r}(0)=g(0)\neq 0\]
	Inverse function theorem says us that $v_1,v_2,\dots,v_n$ will serve as new coordinates for some $U_3\subset U_2$ neighborhood of $0$. This finishes induction step and proof.
\end{proof}
To finish, it still remains to demonstrate the validity of Pulling Lemma, which we now will state formally
\begin{lemma}{(Pulling Lemma, p.13 of \cite{morseTh}) }
	Let $f$ be smooth real-valued function, defined on compact manifold $M$. Let $M^a:=\{x\in M|f(x)\leq a\}$. Assume also that set $f^{-1}([a,b])$ has no critical points of $f$. Then, $M^a$ is diffeomorphic to $M^b$.
\end{lemma}
The idea of proof is as trivial: define a flow on the manifold, that would flow perpendicularly to level curves of $h$ with suitable speed
(in fact, gradient vector provides us with such flow speed), then deform $M^a$ to $M^b$ according to this flow. One technical problem is
how to find flow, given its speed field. The solution is to do the same, what we usually do on manifolds: solve problem locally, pulling it to 
Euclidean space in small neighborhood of each point (and using standard theorem about solutions of initial value problems). Then (that's why
we need compactness!) since $M$ is covered by finitely many neighborhoods, represent flow for big values of time by sequence of small "jumps",
where each "jump" happens inside the neighborhood and is well-defined! More formally,
\begin{proof}
	First, let us define the $\nabla f$, which is the gradient of $f$. In case of $M$ embedded in Euclidean space, gradient is defined at point $p\in M$ as projection of $\nabla \tilde{f}$ on the tangent plane of $M$ at $p$, where
	$\tilde{f}$ is smooth extension of $f$ to neighborhood of $p$. The most important property of gradient, that will be of concern to us is that for any smooth curve $c(t)\in M$
	\[<\frac{dc}{dt},\nabla f>=\frac{d (f\circ c)}{dt}\]
	In the case of embedded $M$ this is obvious. At the same time, since right hand side depends only on values of $f$ on $M$, this shows that gradient is well-defined. This formula also holds in case of not-embedded manifold, 
	though I am not familiar with them.\\
	Now let us define smooth $\rho: M\rightarrow \mathbb{R}$, which will be equal to $1/|\nabla f|^2$ on $f^{-1}([a,b])$ and vanish outside $f^{-1}([a-\epsilon,b+\epsilon])$. Then, we may define smooth vector field on $M$ by
	$\rho\nabla f$.\\
	Assume that we have found 1-parameter group of isomorphisms, (that is $\phi_t(x):M\rightarrow M$, which is smooth in $t$ and $x$ and such that $phi_t$ is diffeomorphism and $\phi_x\circ\phi_y=\phi_{x+y}$) with property
	$\frac{d\phi}{dt}(x)=(\rho\nabla f)(x)$. We will show later how to find it. Then, for fixed $q\in f^{-1}([a,b])$
	\[\frac{df(\phi_t(q)}{dt}=<\frac{\phi_t(q)}{dt},\nabla f>=<\rho \nabla f,\nabla f>=1\]
	Hence, $t\rightarrow f(\phi_t(q))$ is linear in $t$ (with slope $+1$), as long as $a\leq f(q)\leq b$. (moreover, by similar logic it is nondecreasing for $q\in M$
	)Therefore, $\phi_{b-a}$ maps $M^a$ onto $M^b$ diffeomorphically, as desired.\\
	Finally, it remains to explain how to find $\phi_t$. For brevity we will denote $X=\rho \nabla f$ - smooth vector field on $M$, vanishing outside compact subset $K$ of $M$. For each $x\in K$ it has subset which is diffeomorphic
	to $\mathbb{R}^n$ (again, we assume $M$ is embedded into Euclidean space). On that subset we may locally find integral lines for vector field. Moreover, since vector field is smooth, resulting solutions will depend
	smoothly on initial conditions and time. Formally, there is a neighborhood $U\in x$ and $\epsilon>0$ such that ODE
	\[\frac{d\phi_t(q)}{dt}=X(\phi_t(q)),\;\phi_0(q)=q\]
	has unique solution for $q\in U$, $|t|<\epsilon$ which depends smoothly on $q$ and $t$. Since each $x\in K$ has such neighborhood and $K$ is compact, it is covered by finitely many such neighborhoods. Since number
	of neighborhoods is finite, we may take the smallest one from corresponding $\epsilon$ and call it simply $\epsilon$. We shall set $\phi_t(q)=q$ for $q\notin K$ and hence on $K$ $\phi_t$ is well-defined for $|t|<\epsilon/2$.
	Now, for $p=n\cdot\frac{\epsilon}{2}+r>0,\;0<r<\frac{\epsilon}{2}$ we define $\phi_p=\phi_{\epsilon/2}\circ\phi_{\epsilon/2}\circ\dots\circ\phi_{\epsilon/2}\circ\phi_r$, where $\phi_{\epsilon/2}$ is repeated $n$ times.
	For $p<0$ we define $\phi_p=\phi_{-p}^{-1}$
\end{proof}
This finishes the proof of Reeb Sphere Theorem when $f$ is assumed to be Morse function.
\subsection{Proof of Reeb Sphere Theorem for general functions}
The main difficulty is that while we can use Pulling Lemma, we cannot use Morse Lemma, since latter depends on non-singularity of matrix of second
derivatives. However, some neighborhood $M_1$ of $f^{-1}(a)$ is still diffeomorphic to open disk. Using diffeomorphism, that pulled $M^a$ to $M^b$
in Pulling Lemma, we may pull $M_1$ closer to $q:=f^{-1}(b)$ to get diffeomorphic $M_2\supset M_1$ (hence, also diffeomorphic to open disk $O$)
. This can be continued indefinitely
to get increasing sequence $M_i$ of embeddings of $O$ in $M$, that together cover $M\setminus\{q\}$.\\
The pulling we bear in mind is schematically depicted on the picture below.
\begin{figure}[H]
\centering
\includegraphics[width=0.4\textwidth]{manifold_pulling.jpg}
\caption{\label{manifoldPulling}}
\end{figure}
To pull we use diffeomorphism $\psi$ of $M$ (constructed in proof of Pulling Lemma) that:
\begin{enumerate}
	\item{Maps $f^{-1}(l)$ to $f^{-1}(L)$}
	\item{Fixes all the points $x\in M$ with $f(x)>L+\epsilon$ or $f(x)<l-\epsilon$ ($\epsilon>0$ small)}
	\item{$\forall x\in M\;f(x)\leq f(\psi(x))$}
\end{enumerate}
We set $l$ as smallest value of $f$ attained on boundary of $M_i$ (well defined, since boundary mentioned is closed subset of compact $M$ and
hence compact). Similarly, $\max_{x\in \partial M_i}f(x)$ is well-defined and $L$ is taken bigger than it.
In subsequent pullings, we take $L<b$ closer and closer to $b$, so that $M_i\to M$\\
What we really get out of all musing of previous paragraph, is following: we have increasing sequence of submanifolds $M_i\subset M_{i+1}\subset M$
whose union gives $M\setminus\{q\}$ and such that each of these submanifolds $M_i$ is diffeomorphic to the open disk $O$.
Moreover, disks can be thought as being of increasing radius and contained one in 
another. But monotone union of disks of increasing radius is nothing but the
Euclidean space. Hence, it seems to be worthwhile to try to show that $M\setminus\{q\}$ is diffeomorphic to $\mathbb{R}^n$.
This clearly will be sufficient to show that $M$ is homeomorphic to $S^n$. This intuition proves to be true.\\
\begin{remark}
Note, that while we will construct diffeomorphism between $M\setminus\{q\}$ and $\mathbb{R}^n$, we have no ideas
how it will behave near $q$. Therefore, we have no hopes for extending it to \textit{diffeomorphism} between $M$ and $S^n$. Again,
while hypothesis of Reeb Sphere Theorem involves differentiability, conclusion gives merely continuous function. This is weakness of Reeb Sphere
Theorem (while its strength being in easy hypothesis). Differentiable structure is required (in Pulling Lemma), but destroyed in course of proof.\\
\end{remark}
In fact, the only piece that we lacking is the ability to \textit{extend} diffeomoprhisms.
A priori each $M_i$ has its own diffeomorphism
to the $O_i=\{x\in\mathbb{R}^n||x|<i$. However, if we would be able to alter 
diffeomorphism $\psi_i$ between $M_i$ and $O_i$, so that it becomes extension of $\psi_{i-1}$ to $M_i$, we would finish the proof.\\
\begin{figure}[H]
\centering
\includegraphics[width=0.6\textwidth]{commute.PNG}
\caption{}
\end{figure}
The picture suggests that we have two diffeomorphisms from $O_{i-1}$ into $M_i$: one, resulted from original $\psi_{i-1}$
between $O_{i-1}$ and $M_{i-1}\subset M_i$. The second (call it $F_2$) obtained
from inclusion $O_{i-1}\subset O_i$ and subsequent projection using diffeomorphism $\psi_i$ (call it $\psi_i|_{O_{i-1}}$)
. Diagram suggests, that the problem would be resolved, if we would find diffeomorphism of $M_i$ $H$, such that $\psi_i|_{O_{i-1}}=H\circ 
\psi_{i-1}$.
Existence of such diffeomorphism is guaranteed by the following lemma, which corresponds to the Theorem B in \cite{palaisOuter}. Prove is taken
from there almost literally.
\begin{lemma}{Moving Lemma (Theorem B in \cite{palaisOuter})}
	\label{MovingLemma} Let $\phi$ and $\psi$ be smooth orientation-preserving embeddings of $O$ (open disk of radius $r$)
	into oriented n-manifold $M$. Furthermore, assume that for small $\delta>0$ $\phi$ and $\psi$ can be extended to embeddings
	of open disk $O^*\supset O$ of bigger radius $1+\delta$ (and same center as $O$).
	Then, there exists diffeomorphism $H$ on $M$ such that $\phi=H\circ\psi$
\end{lemma}
\begin{remark}
	\label{BogusHypothesis}
	Additional hypothesis about possibility of extension of $\psi$ and $\phi$ looks artificial and in fact absent in references. It can
	be removed, the price being requiring $\psi,\phi$ be embeddings of \textit{closed disks}. There are two reasons for introducing this highly
	strange assumption. First, while author of \cite{palaisInner} claims that embedding of closed disk can always be extended to embedding of
	open disk of slightly bigger radius, we do not see why this is obvious. Second, embeddings that we will apply Moving Lemma automatically
	satisfy this additional requirement. In reference to Figure~\ref{manifoldPulling} and accompanying discussion,
	we can just pull $M_i$ a to a bit bigger $L$ and then restrict to smaller circle. Note, that $\delta>0$ mentioned in this bogus hypothesis
	may not be common for all $\psi_i:O_i\to M$
\end{remark}
Let us take a look onto this situation more closely. More or less we want to move (submanifold diffeomorphic to the) open disk to another using
diffeomorphism of a big manifold, where both disks embedded. What makes situation slightly more complicated, is the we need to match these
disks according to preassigned point correspondence (given by $\phi\circ\psi^{-1}$).\\
We can look on a situation from a bit different angle.
$\phi\circ\psi^{-1}$ gives diffeomorphism between two submanifolds (embeddings of open disk into $M$)
, that we seek to \textit{extend to diffeomorphism} of $M$. Naturally, this leads us to studying the possibility of
extension of mappings between parts of $M$ to
diffeomorphisms on $M$.\\
As a warm-up, let's prove a weak result about extension to diffeomorphism
from \cite{palaisInner} (called Lemma 5.2 there). It roughly says that if we have smooth mapping $\phi$ of a neighborhood of point $m\in M$ to manifold
$M$, which behaves like identity near $m$, then values
of $f$ on small neighborhood of $m$ can be extended to get the diffeomorphism of $M$. Formally,
\begin{lemma}{(Lemma 5.2 in \cite{palaisInner}}\label{identExtension} Let $M$ be a connected manifold, $m\in M$, $U\ni m$ neighborhood of $m$ and smooth $\phi:U\rightarrow M$. Then if $\phi(m)=m,\;d\phi(m)=I$ (behaves like identity - 
	has same value and derivative at $m$), we can find
	neighborhood $K\subset U$ of $m$ and diffeomorphism $\psi:M\rightarrow M$ such that $\phi|U=\psi|K$
\end{lemma}
\begin{proof}
	Because of local nature of lemma, it is sufficient to consider the case $M=U$ is open subset of $\mathbb{R}^n$ and $m=0$. We shall denote $\phi=(\phi_1,\phi_2,\dots,\phi_n)$ and $\psi=(\psi_1,\psi_2,\dots,\psi_n)$.
	In proving this (and next) lemma we will need a function $\sigma_r:\mathbb{R}\rightarrow\mathbb{R}$
	depending on parameter $r>0$
with some special properties, which we now list:
\begin{enumerate}\label{sigma_r_prop}
	\item{$\sigma_r$ is smooth}
	\item{$0\leq \sigma_r(x)\leq 1$ on $\mathbb{R}$}
	\item{$\sigma_r(x)=1$ for $x\leq r^2$ and $\sigma_r(x)=0$ for $x\geq 4r^2$}
	\item{$|\sigma_r'(x)|\leq \alpha/r^2$ for $r^2\leq x\leq 4r^2$ and some $\alpha>0$ independent on $r$ and $x$}
\end{enumerate}
As in \cite{diffTopo}, the example of such a function can be constructed by taking $\sigma_r(x)=\sigma_1(x/r^2)$ and setting $\sigma_1(x)=
\frac{\lambda((4-x)/3)}{(\lambda((4-x)/3)+\lambda(1-(4-x)/3))}$,
where $\lambda(x)=\exp(-1/x^2)$ - smooth function, which
vanishes for negative $x$. Such setting obviously satisfies properties 1, 2 and 3. Property 4 is satisfied as well, since we may set $\alpha=\sup_{x\in\mathbb{R}} |sigma_1'(x)|$ and note 
that $|\sigma_r'(x)|=|sigma_1'(x/r^2)|/r^2\leq \alpha/r^2$\\
Now we shall take $r>0$ small, so ball of radius $4r$ will lay inside $U$ (so function $\phi$ is defined on a ball), and put
\[\begin{array}{l r}
	\psi(x)=x+\sigma_{r}(\norm{x}^2)(\phi(x)-x), & \mbox{for }\norm{x}\leq 2r,\\
	\psi(x)=x, & \mbox{for }\norm{x}\geq 2r\\
\end{array}\]
Now, if we will be able to show that $\psi$ so defined has no critical points on $\mathbb{R}^n$ (which we will do below), then $\psi$ is
indeed diffeomorpsism from $\mathbb{R}^n$, that agrees with $\phi$ on small neighborhood of origin ($\norm{x}\leq r$) and is identity outside bigger
(yet, small) neighborhood. This is because first, $\psi$ is either everywhere orientation-preserving, or everywhere orientation-reversing. This
is because if we would have two points with different signs of $\det (d\psi)$ at them, then connecting these points with line, we'd find (determinant
is continuous function) point with $\det (d\psi)=0$ - critical point, which does not exist by hypothesis. Second, since $\psi$ changes orientation
at the same way at all points, Theorem A from \cite[p. 28]{diffTopo} tells us that Brouwer degree (which in this case is equal to number
of points in preimage with plus or minus) is same for each point in image of $\psi$, since $\phi$ has no critical values. Finally, for big values
of $\psi(x)$ $\phi$ is identity, therefore size of preimage is 1. Hence, $\psi$ is 1-1, therefore is diffeomorphism (it is locally diffeomorphic
at each point by inverse function theorem).
Now we claim that $\psi(x)$ defined as above has no critical points on $\mathbb{R}^n$. This is non-obvious only for case $\norm{x}\leq 2r$. For this case
\[|\frac{\partial \psi_i}{\partial x_j}-\delta_{ij}|=
|2\sigma_r'(\|x\|^2)x_j\phi_i+\sigma_r(\norm{x}^2)(\frac{\partial \phi_i}{\partial x_j}-\delta_{ij})|\leq\]
\[\leq |2\sigma_r'(\|x\|^2)x_j\phi_i|+|\sigma_r(\norm{x}^2)(\frac{\partial \phi_i}{\partial x_j}-\delta_{ij})|\]
From Taylor's theorem (since $\phi_i(0)=0$ and $\frac{\partial\phi_i}{\partial x_j}-\delta_{ij}=0$) we can write
$|\phi_i(x)|=|\norm{x}^2B_i(x)|\leq \norm{x}^2N$ and $\frac{\partial\phi_i}{\partial x_j}-\delta_{ij}=|\norm{x}B_{ij}(x)|\leq \norm{x}N$ for
suitably chosen $N$ and small $r$. Hence
\[|\frac{\partial \psi_i}{\partial x_j}-\delta_{ij}|\leq \frac{2\alpha}{r^2}\norm{x}N\norm{x}^2+1\cdot N\norm{x}\leq 16r\alpha N+2Nr\]
Therefore, $|\frac{\partial \psi_i}{\partial x_j}-\delta_{ij}|$ can be made arbitrary small by taking small $r$ and we will thus have
that $\psi$ has no critical points, as claimed. This proves the claim and lemma.
\end{proof}
In fact, by using diffeomorphisms on $M$ we can do quite a lot. Not only we can find diffeomorphism, which maps any point to any other point
(Homogenity Lemma, \cite[p. 22]{diffTopo}), we also may find diffeomorphism $\psi$ with arbitrary $d\psi$ at given point (however, arbitrary
here means the one with positive determinant, since diffeomorphism cannot have another derivative matrix). Since matrices with positive
determinant form connected subset of $\mathbb{R}^{n\times n}$, it is enough to prove that we can find diffeomorphism with any matrix, which
differs from identity only a bit. More formally,
\begin{lemma}{(Theorem 5.4 in \cite{palaisInner}) }
	\label{rotationDiffm}
	Given $n$-manifold $M$, with point $m$, there exists $\epsilon>0$ such that for any given square matrix $T$ with $\det T>0$ and
	$\max_{i,j}|\delta_{ij}-T_{ij}|<\epsilon$ there is diffeomorphism $\psi:M\rightarrow M$ that has $m$ as fixed point and $d\psi|_m=T$
\end{lemma}
\begin{proof}
	As with previous result, because of the local nature of this lemma, it is enough to prove it for the case of diffeomorphism of
	$\mathbb{R}^n$, that fixes $0$ and everything outside some small circle around origin. That is, for any $r>0$ we shall construct
	$\psi:\mathbb{R}^n\rightarrow\mathbb{R}^n$ that will have properties:
	\begin{enumerate}
		\item{$\psi$ is diffeomorphism}
		\item{$\psi(x)=Tx,\;\mbox{if }\norm{x}\leq r$}
		\item{$\psi(x)=x,/;\mbox{if }\norm{x}\geq2r$}
	\end{enumerate}
	We shall exploit the function defined in proof of previous lemma, $\sigma_r$, whose properties of interest (and construction) were listed
	above in beginning of proof of previous lemma, \ref{sigma_r_prop}. We shall define
	\[\psi(x)=x+\sigma_r(\norm{x}^2)(Tx-x)\]
	All claimed properties of $\psi$ except of first one are obvious from construction and properties of $\sigma_r$. As in prove of
	lemma above, to show that $\psi$ is diffeomorphism it is enough to show that derivative of $\psi$ is nonsingular everywhere. And again,
	it is non obvious only for $\norm{x}\leq 2r$. For this case,
	\[\partial \psi_i/\partial x_j=\delta_{ij} +2x_i\sigma_r'(\norm{x}^2)(\sigma_{j=1}^n T_{ij}x_j-x_i)+\sigma_r(\norm{x}^2)(T_{ij}-\delta_{ij})
	\]
	Since $|\sigma_r(x)|\leq 1$ and $|\sigma_r'(x)|\leq \alpha/r^2|$, for $\norm{x}\leq 2r$ we have
	\[|\partial \psi_i/\partial x_j-\delta_{ij}|\leq \frac{4\alpha}{r}\norm{T-I}_2r+\epsilon\]
	Since $\norm{T-I}_2\leq n^2\epsilon^2$ we see that by requiring $\epsilon$ small we may ensure that $\det d\psi>0$ throughout $\norm{x}
	\leq 2r$. The proof is thus complete.
\end{proof}
Recall, that what we have is two orientation-preserving diffeomorphisms $\phi,\psi:O\rightarrow M$. And we want to extend diffeomorphism
$\phi\circ\psi^{-1}$ from $\psi(O)$ to $\phi(O)$ to diffeomorphism of $M$. This is what we call Moving Lemma (since we move $\psi(O)$ to
$\phi(O)$ using diffeomorphism of $M$).
Taken together, two lemmas proven above allow us to obtain weaker version of this, that in turn will be instrumental
in proof of Moving Lemma. Namely,
what we want to obtain is
\begin{lemma}{(Theorem 5.5 in \cite{palaisInner}) }
	\label{InnerLemma}
	Suppose that we have is two orientation-preserving diffeomorphisms $\phi,\psi:O\rightarrow M$
	, where $M$ is connected oriented compact
	$n$-manifold and $O$ is open disk. Then there is open disk $O'\subset O$ with same center as $O$, such that
	diffeomorphism $\phi\circ\psi^{-1}|_{\psi(O')}$ between $\psi(O')$ and $\phi(O')$ can be extended to diffeomorphism on $M$
\end{lemma}
\begin{proof}
	For brevity, let $m:=\psi(0)$ and
	$f:=\phi\circ\psi^{-1}:\psi(O)\rightarrow\phi(O)$. Since both $\phi$ and $\psi$ are orientation preserving diffeomorphisms,
	$\det df|_m>0$. By Lemma~\ref{rotationDiffm} (and Homogenity Lemma, \cite[p. 22]{diffTopo}) we can find diffeomorphism $\sigma:M\to M$
	such that $\sigma(f(m))=m$ and $d\sigma|_{f(m)}=df|_m^{-1}$. Then $\sigma\circ f:\psi(O):\phi(O)
	$ fulfills the hypothesis of Lemma~\ref{identExtension}
	and therefore there exist neighborhood $K\subset \psi(O)$ of $\psi(0)$
	and diffeomorphism $G:M\to M$ that agrees with $\sigma\circ f$ and $G$.
	We can assume (perhaps, by taking smaller $K$) that $K=\psi(O')$, with $O'\subset O$ - open disk concentric with $O$. From above,
	$\sigma^{-1}\circ G$ is diffeomorphism of $M$ that agrees with $f$ on $K=\psi(O')$
\end{proof}
Now, that we have all tools at our disposal, it is time to explain how we shall prove Moving Lemma. Once again, we have manifold $M$,
diffeomorphisms $\phi,\psi:O\rightarrow M$ and we want to extend $\phi\circ\psi^{-1}:\psi(O)\to\phi(O)$ to diffeomorphism $H$ of $M$. By result
just proven, we have open disk $O'\subset O$ and diffeomorphism $G:M\to M$ such that $\phi\circ\psi^{-1}|_{\psi(O')}=G|_{\psi(O')}$. First,
we extend $\phi,\psi$ to $O^*\supset O$ (see remark after the Moving Lemma, \ref{BogusHypothesis}).
Second,
we will find diffeomorphisms $A,B:M\to M$ that will shrink $\psi(O)$ and $\phi(O)$ to $\psi(O')$ and $\phi(O')$ respectively and at the same time
$\psi(O*\setminus O)$ and $\phi(O^*\setminus O)$ to $\psi(O^*\setminus O')$ and $\phi(O^*\setminus O')$ respectively. Then, composite diffeomorphism
$B^{-1}\circ G\circ A$ that first shrinks $\psi(O)$ to $\psi(O')$, then moves $\psi(O')$ to $\phi(O')$ and than expands $\phi(O')$ to $\phi(O)$
is obviously what we need. It will preserve mapping between $\phi\circ\psi^{-1}:\psi(O)\to\phi(O)$, since $G$ preserves its restriction.
\begin{proof}{(of Moving Lemma, \ref{MovingLemma}) }
	Using methods that were used to construct $\sigma_r:\mathbb{R}\to\mathbb{R}$, used in Lemma~\ref{identExtension}, \ref{sigma_r_prop},
	we may construct smooth $\lambda:\mathbb{R}\to\mathbb{R}$ such that $\lambda(t)=1$ if $t\leq 1$, $\lambda(t)=0$ if $t\geq 1+\delta/2$ and
	$0\leq \lambda(t) \leq 1$ for all $t$. Assuming $\phi,\psi:O^*\to M$ ($O^*$ being open disk of radius $r+\epsilon$ and same center as $O$)
	, we define two diffeomorphisms $M\to M$
\[\begin{array}{l r}
	A(\phi(t))=\phi((1-(1-\epsilon)\lambda(\norm{t}))t), & \mbox{for }\norm{t}<1+\delta\\
	A(x)=x, & \mbox{for }x\notin \phi(O^*)\\
	B(\psi(t))=\psi((1-(1-\epsilon)\lambda(\norm{t}))t), & \mbox{for }\norm{t}<1+\delta\\
	B(x)=x, & \mbox{for }x\notin \psi(O^*)\\
\end{array}\]
	Setting $H:=B^{-1}\circ G \circ A$, we see that $H$ is diffeomorphism on $M$, satisfying $\phi=H\circ\psi$.
\end{proof}
\section{Further Development}
\begin{theorem}{(Theorem of Smale) }If $M$ has a homotopy type of n-sphere and dimension $n\geq 5$, then it is possible to define on $M$ Morse function with only two critical points.
\end{theorem}
As a corollary, if $n$-manifold ($n\geq 5$) has homotopy type of a sphere, it is homeomorphic to the sphere. In the proof of this theorem Reeb Sphere Theorem is used. Roughly, the idea is to first construct Morse function on $M$ and
then to remove (by changing function) all critical points that are not essential to give $M$ its homotopy type. Since homotopy type of a sphere has only two nontrivial groups, there will be only two critical points essential. 
And the theorem follows.\\
In turn, the Theorem of Smale provides way to construct exotic 7-sphere (and exoctic spheres of higher dimension), as it is only necessary to check homotopy type for high dimensions. Accidentally, these rule out the possibility of "strengthening" Reeb Sphere Theorem, by replacing "homeomorphic"
with "diffeomorphic". Differential structure is lost \textit{inevitably} in the course of the proof. In some cases it \textit{cannot} be recovered.
Otherwise, exotic spheres constructed with help of Reeb Sphere Theorem would be diffeomorphic to $S^n$, and thus would not be exotic.
\begin{thebibliography}{9}
	\bibitem{diffTopo}J. Milnor, Differential Topology. In {\em Lectures on Modern Mathematics II}, edited by T.L. Saaty, 1964.
	\bibitem{morseTh} J. Milnor, {\em Morse theory}. Annals of Mathematical Studies, No. 51 Princeton University Press, Princeton 1963.
	\bibitem{topoFrom} J. Milnor, {\em Topology from the Differentiable Viewpoint}, The University Press of Virginia Charlottesville, 1965.
	\bibitem{palaisOuter} R. Palais, "Extending diffeomorphisms," {\em Proc. Amer. Math. Soc.} \textbf{11} (1960) 274-277
	\bibitem{palaisInner} R.S. Palais, {\em Natural operations on differential forms}, Trans. Amer. Math. Soc. vol. 92 (1959) pp. 125-141.
\end{thebibliography}
\end{document}
