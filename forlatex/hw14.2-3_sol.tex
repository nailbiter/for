\documentclass[8pt]{article} % use larger type; default would be 10pt

\usepackage[margin=1in]{geometry}
\usepackage{graphicx}
\usepackage{float}
\usepackage{subfig}
\usepackage{amsmath}
\usepackage{amsfonts}
\usepackage{hyperref}
\usepackage{enumitem}
\usepackage[neverdecrease]{paralist}

\usepackage{mystyle}

\title{Math 1540\\University Mathematics for Financial Studies\\2013-14 Term 1\\Suggested solutions for\\
Sec. 14.2-14.3}
\begin{document}
\maketitle
\section{Section 14.2}
\begin{description}
	\item[\# 6.]{{\it Find the limit}\[\lim_{(x,y)\to(0,0)}\cos\frac{x^2+y^3}{x+y+1}\]
		}
		As trigonometric function $\cos(x)$ is continuous, the required limit will be found, if we will be able to compute
		\[\lim_{(x,y)\to(0,0)}\frac{x^2+y^3}{x+y+1}\]
		We claim, that it's equal to zero and below is the formal proof. Assume $\epsilon>0$ is given. We want to find $
		\delta>0$, so that $\sqrt{x^2+y^2}<\delta$ will imply
		\[\myabs{\frac{x^2+y^3}{x+y+1}}\]
		Let's try to simplify the expression above. Note, that under the assumption $1-\myabs{x}-\myabs{y}>0$ we can write,
		based on standard inequality $\myabs{a-b}\geq\myabs{a}-\myabs{b}$
		\[\myabs{\frac{x^2+y^3}{x+y+1}}\leq\frac{\myabs{x}^2+\myabs{y}^3}{1-\myabs{x}-\myabs{y}}\]
		We really can assume that $1-\myabs{x}-\myabs{y}>0$, as for $\delta>0$ we have $\myabs{x},\myabs{y}\leq\sqrt{x^2+
		y^2}\leq\delta$. Thus, if $\delta\leq\myfrac{1}{4}$, we have $1-\myabs{x}-\myabs{y}\geq 1-\myfrac{1}{4}-\myfrac{1}{4}
		\geq\myfrac{1}{2}>0$, so the transformation above is valid. Thus, we have new target: prove
		\[\frac{\myabs{x}^2+\myabs{y}^3}{1-\myabs{x}-\myabs{y}}<\epsilon\]
		Now, if $\delta\leq\myfrac{1}{4}$, we have according $\myabs{x},\myabs{y}<\myfrac{1}{2}$ and hence
		\[\frac{\myabs{x}^2+\myabs{y}^3}{1-\myabs{x}-\myabs{y}}\leq
		\frac{\myabs{x}^2+\myabs{y}^3}{\myfrac{1}{2}}\leq2\cdot\mybra{\myabs{x}^2+1\cdot\myabs{y}^2}<2\delta^2\]
		Thus, we just need $2\delta^2\leq\epsilon^2$, or equivalently $\delta\leq\sqrt{\myfrac{\delta}{2}}$. Summarizing,
		we have two requirements to our delta: $\delta\leq\myfrac{1}{4}$ and $\sqrt{\myfrac{\delta}{2}}$. Hence, we may set
		\[\delta:=\min\mycbra{\frac{1}{4},\sqrt{\frac{\delta}{2}}}\]
		Thus, $\lim_{(x,y)\to(0,0)}\frac{x^2+y^3}{x+y+1}=0$ and hence
		\[\lim_{(x,y)\to(0,0)}\cos\frac{x^2+y^3}{x+y+1}=\cos(0)=1\]
	\item[\# 19.]{{\it Find the limit by rewriting fraction first.}
		\[\lim_{\begin{subarray}{c}(x,y)\to(2,0)\\2x-y\neq4\end{subarray}}\frac{\sqrt{2x-y}-2}{2x-y-4}\]
		Note that using the old trick $a^2-b^2=(a-b)(a+b)$ we have
		\[\frac{\sqrt{2x-y}-2}{2x-y-4}=\frac{1}{\sqrt{2x-y}+2}\]
		As square root and rational function are both continuous, it is enough to find \[\lim_{(x,y)\to(2,0)}(2x-y)\]
		However, as outlined in textbook, $\lim_{(x,y)\to(x_0,y_0)}x=x_0$ and $\lim_{(x,y)\to(x_0,y_0)}y=y_0$ and hence
		\[\lim_{(x,y)\to(2,0)}(2x-y)=2\lim_{(x,y)\to(2,0)}x-\lim_{(x,y)\to(2,0)}y=4\]
		and consequently
		\[\lim_{\begin{subarray}{c}(x,y)\to(2,0)\\2x-y\neq4\end{subarray}}\frac{\sqrt{2x-y}-2}{2x-y-4}
			=\lim_{\begin{subarray}{c}(x,y)\to(2,0)\\2x-y\neq4\end{subarray}}\frac{1}{\sqrt{2x-y}+2}=\frac{
				1}{\sqrt{4}+2}=\frac{1}{4}\]
		}
	\item[\# 21.]{{\it Find the limit by rewriting fraction first.}
		\[\lim_{(x,y)\to(0,0)}\frac{\sin(x^2+y^2)}{x^2+y^2}\]
		Let us introduce the function
		\[f(x)=\left\{\begin{array}{ll}\frac{\sin x}{x},&x\neq 0\\1,&x=0\end{array}\right.\]
		It is assumed to be known from the one-variable calculus that $f(x)$ is continuous on $\mathbb{R}$ (or, in other
		words, that $\lim_{x\to 0}\frac{\sin x}{x}=1$). Now, as both $f(x)$ and $g(x,y)=x^2+y^2$ are both continuous,
		their composition is also so, hence
		\[\lim_{(x,y)\to(0,0)}\frac{\sin(x^2+y^2)}{x^2+y^2}=\lim_{(x,y)\to(0,0)}f(g(x,y))=f(g(0,0))=f(0)=1\]
		}
	\item[\# 47.]{{\it By considering different paths of approach, show that the function has no limit as $(x,y)\to(0,0)$.
		\[h(x,y)=\frac{x^2+y}{y}\]}
		Note, that \[h(x,y)\bigg|_{x=0}=\frac{y}{y}=1\]
		and therefore
		\[\lim_{\begin{subarray}{c}(x,y)\to(0,0)\\\mbox{along }x=0\end{subarray}}\frac{x^2+y}{y}=
			\lim_{(x,y)\to(0,0)}\mysbra{\left.\frac{x^2+y}{y}\right|_{x=0}}=1\]
		However, \[\left.h(x,y)\right|_{y=x^2}=\frac{x^2+x^2}{x^2}=2\]
		and thus
		\[\lim_{\begin{subarray}{c}(x,y)\to(0,0)\\\mbox{along }y=x^2\end{subarray}}\frac{x^2+y}{y}=
			\lim_{(x,y)\to(0,0)}\mysbra{\left.\frac{x^2+y}{y}\right|_{y=x^2}}=2\neq 1\]
		hence limit does not exist by two-path test.
		}
	\item[\# 50.]{
		\newcommand{\f}{\frac{xy+1}{x^2-y^2}}
		{\it Show that the limit does not exist \[\lim_{(x,y)\to(1,-1)}\frac{xy+1}{x^2-y^2}\]
		}
		Let us use two-path test. Note, that 
		\[\left.\f\right|_{x=1}=\frac{y+1}{1-y^2}=\frac{1}{1-y}\]
		and therefore
		\[\lim_{\begin{subarray}{c}(x,y)\to(1,-1)\\\mbox{along }x=1\end{subarray}}\f=
			\lim_{(x,y)\to(1,-1)}\mysbra{\left.\f\right|_{x=1}}=\lim_{y\to-1}\frac{1}{1-y}=\frac{1}{2}\]
			However, \[\left.\f\right|_{y=-1}=\frac{-x+1}{x^2-1}=-\frac{1}{1+x}\]
		and thus
		\[\lim_{\begin{subarray}{c}(x,y)\to(1,-1)\\\mbox{along }y=-1\end{subarray}}\f=
			\lim_{(x,y)\to(1,-1)}\mysbra{\left.\f\right|_{y=-1}}=\lim_{x\to1}\mybra{-\frac{1}{1+x}}=-\frac{1}{2}\]
		hence limit does not exist by two-path test.
		}
	\item[\# 51.]{{\it Let}
		\[f(x,y)=\begin{cases}1,\mbox{ }y\geq x^4\\1,\mbox{ }y\leq 0\\0,\mbox{ othirwise.}\end{cases}\]
		{\it Find each of the following limits, or explain that the limit does not exist.}
		\begin{enumerate}[\bfseries a.]
			\item $\lim_{(x,y)\to(0,1)}f(x,y)$
			\item $\lim_{(x,y)\to(2,3)}f(x,y)$
			\item $\lim_{(x,y)\to(0,0)}f(x,y)$
		\end{enumerate}
		\begin{enumerate}[\bfseries a.]
			\item We claim that \[\lim_{(x,y)\to(0,1)}f(x,y)=1\]
				It is one of the rare cases when proving by definition might be reasonable. Given $\epsilon>0$ let
				us take $\delta=\myfrac{1}{2}$. Then, we have $0<\sqrt{x^2+(y-1)^2}<\delta=\myfrac{1}{2}$ and
				hence $\myabs{x},\myabs{y-1}<\myfrac{1}{2}$. Thus we have $x<\myfrac{1}{2}\implies
				x^4<\myfrac{1}{16}<\myfrac{1}{2}<y$, as $\myabs{y-1}\implies 1-y<\myfrac{1}{2}\implies y>\myfrac{1}
				{2}$ and thus $f(x,y)=1$, hence $\myabs{f(x,y)-1}=\myabs{1-1}=0<\epsilon$ and this proves the claim.
			\item We claim that \[\lim_{(x,y)\to(2,3)}f(x,y)=0\]
				and again, we shall prove this by definition. So assume $\epsilon>0$ given and let us take
				$\delta=\myfrac{1}{2}$, so we can assume in subsequent that 
				$0<\sqrt{(x-2)^2+(y-3)^2}<\delta=\myfrac{1}
				{2}$. Therefore $\myabs{x-2},\myabs{y-3}<\myfrac{1}{2}$. Now
				\[\myabs{x-2}<\myfrac{1}{2}\implies x-2>-\myfrac{1}{2}\implies x>\myfrac{3}{2}\]
				\[\myabs{y-3}<\myfrac{1}{2}\implies y-3<\myfrac{1}{2}\implies y<\myfrac{7}{2}\]
				\[\myabs{y-3}<\myfrac{1}{2}\implies y-3>-\myfrac{1}{2}\implies y>\myfrac{5}{2}>0\]
				hence \[x^4>\frac{81}{16}>\frac{7}{2}>y>0\]
				and thus $\myabs{f(x,y)-0}=\myabs{0-0}=0<\epsilon$.
			\item We claim that $\lim_{(x,y)\to(0,0)}f(x,y)$ does not exist by two-path test. Indeed, 
				$f(x,y)\bigg|_{y=x^4}=1$ and hence 
				\[\lim_{\begin{subarray}{c}(x,y)\to(0,0)\\\mbox{along }y=x^4\end{subarray}}f(x,y)=1\]
					while $f(x,y)\bigg|_{y=x^6}=0$, as for $0<\myabs{x}<1$ we have $0<x^6<x^4$ and hence
				\[\lim_{\begin{subarray}{c}(x,y)\to(0,0)\\\mbox{along }y=x^6\end{subarray}}f(x,y)=0\]
				hence limit does not exist by two-path test.
		\end{enumerate}
		}
	\item[\# 60.]{{\it Define $f(0,0)$ in a way that extends \[f(x,y)=xy\frac{x^2-y^2}{x^2+y^2}\] to be continuous at the origin.}
		\\Continuity at the origin means simply that \[\lim_{(x,y)\to(0,0)}f(x,y)=f(0,0)\] and thus we just need to compute
		the limit above and set $f(0,0)$ to that value. We claim that \[\lim_{(x,y)\to(0,0)}xy\frac{x^2-y^2}{x^2+y^2}=0\]
		Note, that
		\[0\leq
		\myabs{xy\frac{x^2-y^2}{x^2+y^2}}\leq\myabs{x}\cdot\myabs{y}\cdot\frac{\myabs{x^2}+\myabs{-y^2}}{x^2+y^2}=\myabs{x}
		\cdot\myabs{y}\]
		And as \[\lim_{(x,y)\to(0,0)}xy=0\]
		the claim is proven by squeezing theorem. Hence, $f(0,0)$ should be defined as $f(0,0)=0$.
	}
	\item[\# 73.]{{\it Given function $f(x,y)$ and a positive number $\epsilon$. Show that there exists $\delta>0$ such that
		for all $(x,y)$,\[\sqrt{x^2+y^2}<\delta\implies\myabs{f(x,y)-f(0,0)}<\epsilon\]}
		\[f(x,y)=\frac{xy^2}{x^2+y^2}\mbox{ and }f(0,0)=0,\quad\epsilon=0.04\]
		Note that
		\[\myabs{\frac{xy^2}{x^2+y^2}}=\myabs{x}\myabs{\frac{y^2}{x^2+y^2}}\leq\myabs{x}\]
		and hence it is enough to take $\delta=\epsilon=0.04$, as in this case
		\[\sqrt{x^2+y^2}<\delta\implies\myabs{x}<\delta\implies\myabs{\frac{xy^2}{x^2+y^2}}<\delta=\epsilon\]
		}
\section{Section 14.3}
	\item[\# 14.]{{\it Find $\partial f/\partial x$ and $\partial f/\partial y$.
		\[f(x,y)=e^{-x}\sin(x+y)\]}
		Probably, there is nothing much to comment in these exercises...
		\[\frac{\partial f}{\partial x}=-e^{-x}\sin(x+y)+e^{-x}\cos(x+y)\]
		\[\frac{\partial f}{\partial y}=e^{-x}\sin(x+y)\]
		}
	\item[\# 16.]{{\it Find $\partial f/\partial x$ and $\partial f/\partial y$.}
		\[f(x,y)=e^{xy}\ln y\]
		\[\frac{\partial f}{\partial x}=ye^{xy}\ln y\]
		\[\frac{\partial f}{\partial y}=xe^{xy}\ln y+e^{xy}\cdot\frac{1}{y}\]
		}
	\item[\# 25.]{{\it Find $f_x$, $f_y$ and $f_z$.}
		\[f(x,y,z)=x-\sqrt{y^2+z^2}\]
		\[f_x=1\]
		\[f_y=-\frac{y}{\sqrt{y^2+z^2}}\]
		\[f_z=-\frac{z}{\sqrt{y^2+z^2}}\]
		}
	\item[\# 27.]{{\it Find $f_x$, $f_y$ and $f_z$.}
		\[f(x,y,z)=\sin^{-1}(xyz)\]
		\[f_x=\frac{yz}{\sqrt{1-(xyz)^2}}\]
		\[f_y=\frac{xz}{\sqrt{1-(xyz)^2}}\]
		\[f_z=\frac{xy}{\sqrt{1-(xyz)^2}}\]
		}
	\item[\# 43.]{{\it Find all second-order partial derivatives of function \[g(x,y)=x^2y+\cos y+y\sin x\]}
		\[g_{xx}=2y-y\sin x\]
		\[g_{xy}=2x+\cos x\]
		\[g_{yx}=2x+\cos x\]
		\[g_{yy}=-\cos y\]
		}
	\item[\# 48.]{{\it Find all second-order partial derivatives of function \[w=ye^{x^2-y}\]}
		\[w_{xx}=2ye^{x^2-y}+4x^2ye^{x^2-y}\]
		\[w_{xy}=2xe^{x^2-y}-2xye^{x^2-y}\]
		\[w_{yx}=2xe^{x^2-y}-2xye^{x^2-y}\]
		\[w_{yy}=-e^{x^2-y}-e^{x^2-y}+ye^{x^2-y}=-2e^{x^2-y}+ye^{x^2-y}\]
		}
	\item[\# 60.]{{\it Use the limit definition of partial derivative to compute the partial derivatives of the function
		at the specified point.}
		\[f(x,y)=\left\{\begin{array}{ll}\frac{\sin(x^3+y^4)}{x^2+y^2}, &(x,y)\neq(0,0)\\
			0, &(x,y)=(0,0),
		\end{array}\right.\\
		\]
		{\it $\frac{\partial f}{\partial x}$ and $\frac{\partial f}{\partial y}$ at $(0,0)$}\\\\
		\[\frac{\partial f}{\partial x}\bigg|_{(0,0)}:=\lim_{x\to 0}\frac{f(x,0)-f(0,0)}{x}=\lim_{x\to 0}\frac{\sin(x^3)}
		{x^3}=1\]
		\[\frac{\partial f}{\partial y}\bigg|_{(0,0)}:=\lim_{x\to 0}\frac{f(0,y)-f(0,0)}{y}=\lim_{y\to 0}\frac{\sin(y^4)}
		{y^3}=\lim_{y\to0}\frac{\sin(y^4)}{y^4}\cdot y=1\cdot\lim_{y\to 0}y=0\]
		}
	\item[\# 62.]{{\it Let $f(x,y)=x^2+y^3$. Find the slope of the line tangent to this surface at the point $(-1,1)$ and lying in a plane}
		\begin{inparaenum}[\bfseries a.]\item \textit{plane} $x=-1$\quad\item \textit{plane} $y=1$.\end{inparaenum}
		\begin{enumerate}[\bfseries a.]
			\item \[x^2+y^3\bigg|_{x=-1}=1+y^3\]
				and therefore
				\[k=\frac{\partial}{\partial y}\mybra{x^2+y^3\bigg|_{x=-1}}=3y^2=3\]
				Note, that if one would trace the definition of $\frac{
				\partial f}{\partial y}(a,b)$ he would see that it is exactly {\it
				the slope of the line tangent to surface $z=f(x,y)$ at point $(a,b)$ lying in the plane $y=b$} and thus in this
				example $k$ could be also computed simply as
				\[k=\frac{\partial f}{\partial y}=3y^2=3\]
			\item Similarly,
				\[x^2+y^3\bigg|_{y=1}=x^2+1\]
				and therefore
				\[k=\frac{\partial}{\partial x}\mybra{x^2+y^3\bigg|_{y=1}}=2x=-2\]
				Note, that if one would trace the definition of $\frac{\partial f}{\partial x}
				(a,b)$ he would see that it is exactly {\it
				the slope of the line tangent to surface $z=f(x,y)$ at point $(a,b)$ lying in the plane $x=a$} and thus in this
				example $k$ could be also computed simply as
				\[k=\frac{\partial f}{\partial x}=2x=-2\]
		\end{enumerate}
		}
	\item[\# 71.]{{\it Let $f(x,y)=\left\{\begin{array}{ll}y^3,&y\geq 0\\-y^2,&y<0\end{array}\right.$\\Find $f_x,f_y,f_{xy},$ and $f_{yx},$ and
			state the domain of each partial derivative.}\\
			Note, that both $\mysetn{(x,y)\in\mathbb{R}^2}{y>0}$ and $\mysetn{(x,y)\in\mathbb{R}^2}{y<0}$ are open sets.
			Therefore, 
			\[\frac{\partial f}{\partial x}\bigg|_{y>0}=\frac{\partial (y^3)}{\partial x}=0\]
			\[\frac{\partial f}{\partial x}\bigg|_{y<0}=\frac{\partial (-y^2)}{\partial x}=0\]
			\[\frac{\partial f}{\partial y}\bigg|_{y>0}=\frac{\partial (y^3)}{\partial y}=3y^2\]
			\[\frac{\partial f}{\partial y}\bigg|_{y<0}=\frac{\partial (-y^2)}{\partial y}=-2y\]
			On the boundary $\mysetn{(x,y)\in\mathbb{R}^2}{y=0}$ partial derivative should be computed by definition
			\[\frac{\partial f}{\partial x}\bigg|_{y=0}=\lim_{h\to0}\frac{f(x+h,0)-f(x,0)}{h}=\lim_{h\to 0}\frac{0^3-0^3}
			{h}=0\]
			\[\frac{\partial f}{\partial y}\bigg|_{y=0}=\lim_{h\to0}\frac{f(x,h)-f(x,0)}{h}=\lim_{h\to 0}\frac{f(x,h)}{h
			}=0\]
			as both one-side limits are equal to zero
			\[\lim_{h\to 0-}\frac{f(x,h)}{h}=\lim_{h\to0-}\frac{-h^2}{h}=0\]
			\[\lim_{h\to 0+}\frac{f(x,h)}{h}\lim_{h\to0+}\frac{h^3}{h}=0\]
			Thus, we have determined that both $f_x$ and $f_y$ have the whole plane $\mathbb{R}^2$ as their domain and
			\[f_x=\left\{\begin{array}{ll}0,&y>0\\0,&y=0\\0,&y<0\end{array}\right.\]
			\[f_y=\left\{\begin{array}{ll}3y^2,&y>0\\0,&y=0\\-2y,&y<0\end{array}\right.\]
			Immediately, we see that $f_{xy}$ has the whole $\mathbb{R}^2$ as its domain as well, and
			\[f_{xy}\equiv0\]
			and 
			\[\frac{\partial f_y}{\partial x}\bigg|_{y>0}=\frac{\partial (3y^2)}{\partial x}=0\]
			\[\frac{\partial f_y}{\partial x}\bigg|_{y<0}=\frac{\partial (-2y)}{\partial x}=0\]
			\[\frac{\partial f_y}{\partial x}\bigg|_{y=0}=\lim_{h\to0}\frac{f_y(x+h,0)-f_y(x,0)}{h}=\lim_{h\to 0}
			\frac{0-0}{h}=0\]
			and $f_{yx}$ is defined for the whole $\mathbb{R}^2$ also.
		}
	\item[\# 92.]{{\it Let} $f(x,y)=\left\{\begin{array}{ll}0,&x^2<y<2x^2\\1,&\mbox{otherwise.}\end{array}\right.$\\
			{\it Show that $f_x(0,0)$ and $f_y(0,0)$ exist, but $f$ is not differentiable at $(0,0)$.}\\
			Indeed, by definition
			\[\frac{\partial f}{\partial x}(0,0)=\lim_{x\to0}\frac{f(x,0)-f(0,0)}{x}=\lim_{x\to0}\frac{1-1}{x}=0\]
			as for $y=0$ we have $x^2\geq y$. Next,
			\[\frac{\partial f}{\partial y}(0,0)=\lim_{y\to0}\frac{f(0,y)-f(0,0)}{x}=\lim_{y\to0}\frac{1-1}{x}=0\]
			as for $x=0$ we have $x^2=2x^2$ and hence $x^2<y<2x^2$ does not become true.\\
			Now, we will show that $f(x,y)$ is not even continuous at $(0,0)$, hence not differentiable. We shall use
		 	two-path test.
			\[\lim_{(x,0)\to(0,0)}f(x,y)=\lim_{x\to0}f(x,0)=1\]
			as shown above and
			\[\lim_{(x,1.5x^2)\to(0,0)}f(x,y)=\lim_{(x,1.5x^2)\to(0,0)} 0=0\]
		}
\end{description}
\end{document}
%Sec. 14.3 #14, 16, 25, 27, 43, 48, 60, 62, 71, 92
%ASK: maybe only answers?
