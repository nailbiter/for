\subsection{特殊化とhierarchy}
このように、
	命題\ref{prop:int-st-gg}の$\ell = m = 0$に対する積分公式
	はWarnaar、Varchenko、Tarasov などによるセルバーグ形の積分の一般化の特別な場合とも関係する。
	以上を図しにまとめる。
	{青}い{数字}は{公式}に{含}まれる連続パラメータの{個数}である。\footnote{I
	am not sure I have understood Your instructions correctly, so I might have mixed
up the order of sentences in this paragraph. Please, forgive me in advance.}
			\begin{figure*}[h]
				\centering
				\begin{tikzpicture}
				\draw[color=black] (0.0,0.0) rectangle (2.0,-0.5);
\node at (1.0,-0.25) {\color{black}{\scriptsize Warnaar: \color{blue}{4}}};
\draw[color=red] (2.1,0.0) rectangle (4.3,-0.5);
\node at (3.2,-0.25) {\color{red}{\scriptsize 命題 \ref{prop:exp-stz-gg}: \color{blue}{4}}};
\draw[color=black] (5.7,0.0) rectangle (8.9,-0.5);
\node at (7.300000000000001,-0.25) {\color{black}{\scriptsize Tarasov-Varchenko: \color{blue}{4}}};
\draw[color=black] (9.1,0.0) rectangle (12.1,-0.5);
\node at (10.6,-0.25) {\color{black}{\scriptsize Dotsenko-Fateev: \color{blue}{3}}};
\draw[color=red] (2.1,-2.0) rectangle (4.3,-2.5);
\node at (3.2,-2.25) {\color{red}{\scriptsize 命題 \ref{prop:int-st-gg}: \color{blue}{3}}};
\draw[color=black] (5.0,-2.0) rectangle (6.7,-2.5);
\node at (5.85,-2.25) {\color{black}{\scriptsize Selberg: \color{blue}{3}}};
\filldraw[color=red,pattern color=red,pattern=north east lines] (3.2,-4.0) circle(0.3);
\node at (3.85,-4.0) {\color{blue}{3}};
\fill[color=black] (1.5,-6.0) circle(0.1);
\node at (1.95,-6.0) {\color{blue}{2}};
\fill[color=black] (3.2,-6.0) circle(0.1);
\node at (3.6500000000000004,-6.0) {\color{blue}{2}};
\fill[color=black] (5.85,-6.0) circle(0.1);%TV
\node at (6.3,-6.0) {\color{blue}{2}};
\fill[color=black] (8.9,-6.0) circle(0.1);%DF
\node at (9.35,-6.0) {\color{blue}{2}};

\draw[->,>=angle 90,color=black] (2.0,-0.5) -- node {} (5.0,-2.0) ;
\draw[->,>=angle 90,color=red] (3.2,-0.5) -- node {} (3.2,-2.0) ;
\draw[->,>=angle 90,color=black] (6.18,-0.5) -- node {} (5.85,-2.0) ;
\draw[->,>=angle 90,color=black] (9.1,-0.5) -- node {} (6.615,-2.0) ;
\draw[->,>=angle 90,color=red] (3.2,-2.5) -- node {\color{black}{\scriptsize $\kern1.5cm\ell=m=0$}} (3.2000000000000006,-3.7) ;
\draw[->,>=angle 90,color=black] (1.0,-0.5) -- node {} (1.490946425395748,-5.90041067935323) ;
\draw[->,>=angle 90,color=black] (3.0057054739713385,-4.2285817953278375) -- node {} (1.5573628100792203,-5.93251434108327) ;
\draw[->,>=angle 90,color=black] (3.2,-4.3) -- node {} (3.2,-5.914999999999999) ;
\draw[->,>=angle 90,color=black] (3.4394567450999696,-4.180722071773562) -- node {} (5.777393604812446,-5.9452027206131675) ;
\draw[->,>=angle 90,color=black] (3.4830799946398643,-4.099326313908724) -- node {} (8.810326217188193,-5.968535514802874) ;
\draw[->,>=angle 90,color=black] (5.34,-2.5) -- node {} (3.252164719493017,-5.914683869988056) ;
\draw[->,>=angle 90,color=black] (10.3,-0.5) -- node {}  (8.92466,-5.90309);%DF->DF'
\draw[->,>=angle 90,color=black] (7.62,-0.5) -- node {} (5.88,-5.9048) ;%TV->TV'

				\end{tikzpicture}
				\caption{
					命題\ref{prop:int-st-gg}の$\ell=m=0$の特別の場合とその関連結果;
\mykana{青}{アオ}い\mykana{数字}{スウジ}は\mykana{公式}{コウシキ}に\mykana{含}{フク}まれる連続パラメータの\mykana{個数}{コスウ}である.
				}
				\label{fig:intdep}
			\end{figure*}
