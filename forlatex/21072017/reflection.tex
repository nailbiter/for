%japanese
% !TEX TS-program = pdflatex
% !TEX encoding = UTF-8 Unicode

% This is a simple template for a LaTeX document using the "article" class.
% See "book", "report", "letter" for other types of document.

\documentclass[12pt]{article} % use larger type; default would be 10pt

\usepackage[utf8]{inputenc} % set input encoding (not needed with XeLaTeX)

%%% Examples of Article customizations
% These packages are optional, depending whether you want the features they provide.
% See the LaTeX Companion or other references for full information.

%%% PAGE DIMENSIONS
\usepackage{geometry} % to change the page dimensions
\geometry{a4paper} % or letterpaper (US) or a5paper or....
% \geometry{margin=2in} % for example, change the margins to 2 inches all round
% \geometry{landscape} % set up the page for landscape
%   read geometry.pdf for detailed page layout information

\usepackage{graphicx} % support the \includegraphics command and options

% \usepackage[parfill]{parskip} % Activate to begin paragraphs with an empty line rather than an indent

%%% PACKAGES
\usepackage{ulem}
\usepackage{setspace}
\usepackage{booktabs} % for much better looking tables
\usepackage{array} % for better arrays (eg matrices) in maths
\usepackage{paralist} % very flexible & customisable lists (eg. enumerate/itemize, etc.)
\usepackage{verbatim} % adds environment for commenting out blocks of text & for better verbatim
\usepackage{subfig} % make it possible to include more than one captioned figure/table in a single float
\usepackage{amssymb}
\usepackage{amsfonts}
\usepackage{amsmath,amsthm}
\usepackage{xeCJK}
\usepackage{ruby}
%\renewcommand\rubysep{-5ex}
\newcommand{\kana}[2]{\ruby{#1}{#2}}

%CJK font
%\setCJKmainfont[AutoFakeBold=true]{MS PGothic}
\setCJKmainfont[AutoFakeBold=true]{Hiragino Mincho Pro}
% These packages are all incorporated in the memoir class to one degree or another...

%%% HEADERS & FOOTERS
\usepackage{fancyhdr} % This should be set AFTER setting up the page geometry
\pagestyle{fancy} % options: empty , plain , fancy
\renewcommand{\headrulewidth}{0pt} % customise the layout...
\lhead{}\chead{}\rhead{}
\lfoot{}\cfoot{\thepage}\rfoot{}

\newtheorem{question}{質問}
\newtheorem{answer}{私の答え}

%%% SECTION TITLE APPEARANCE
\usepackage{sectsty}
\allsectionsfont{\sffamily\mdseries\upshape} % (See the fntguide.pdf for font help)
% (This matches ConTeXt defaults)

%%% ToC (table of contents) APPEARANCE
\usepackage[nottoc,notlof,notlot]{tocbibind} % Put the bibliography in the ToC
\usepackage[titles,subfigure]{tocloft} % Alter the style of the Table of Contents
\renewcommand{\cftsecfont}{\rmfamily\mdseries\upshape}
\renewcommand{\cftsecpagefont}{\rmfamily\mdseries\upshape} % No bold!

%%%my commands
\newcounter{framecount}
\newcommand{\slide}{\noindent\stepcounter{framecount}$\backslash\backslash$[\arabic{framecount}]\\}
\newcommand{\mytime}[1]{\noindent #1\\}
\newcommand{\Sp}{\ensuremath \mathbb{S}^p}
\newcommand{\Sq}{\ensuremath \mathbb{S}^q}
%\newcommand{\kana}[2]{#1{\scriptsize (#2)}}
\newcommand{\J}{$J(\nu)$}
\newcommand{\I}{$I(\lambda)$}
\newcommand{\doubt}[1]{\fbox{#1}}
\newcommand{\mynum}{}
\newcommand{\pause}{$\bullet$}
\newcommand{\slowly}[1]{\dashuline{#1}}
\newcommand{\continuously}[1]{\underline{#1}}

%%% END Article customizations

%%% The "real" document content comes below...

\title{The reflection on my report on June 21st at RIMS}
\author{}
%\date % Activate to display a given date or no date (if empty),
         % otherwise the current date is printed 
\begin{document}
%\huge
\maketitle
\section{聞かれたご質問}
\begin{question}[\textbf{??}先生]
	命題$1,'1$と2でどうしてGegenbauer多項式が出て来ましたか?
	これの表現論的な意味は何ですか?
\end{question}
\begin{answer}
私は前に言っていましたように、コンパクト群$O(n)$のbranching rulesを具体的に書く時、
Kobayashi-Spehの2015年の論文に書いているように、Gegenbauer 多項式が出て来ます{\footnotesize (スライドのページ32を見せました)}。

元々の表現論プロジェクトで$(G,G')=(O(p+1,q+1),O(p,q+1))$でしたので、これの極大コンパクト群が
$(K,K')=(O(p+1)\times O(q+1),O(p)\times O(q+1))$なので、$O(n)$のブランチングに関する多項式(つまり、Gegenbauer多項式)も
出て来ます。
\end{answer}
\begin{question}[九大の落合先生]
	命題$1'$の積分公式を示す時、この積分は予めガンマ関数の積で表せると言うことが知っていましたか?
	もしかして、
	こんな積分はガンマ関数で表せるのは何か一般論から従いますか?
\end{question}
\begin{answer}
いいえ、これは予め知りませんでした。
一番最初はこの積分公式を色々な数学ソフト(Maple, Mathematicaなど)を計算してみて、失敗だって、
多分簡単な関数で書けないかと思っていましたが、特別な場合を計算した時($\lambda\in2\mathbb{L}$)綺麗な公式
になると言うことを見て、一般のConjectureをして、示しました。

実際には、Carlson定理によって、一般な場合は$\lambda\in 2\mathbb{Z}$から従いましたが、あの時Carlsonの定理
を知りませんでした、直接メソッド1に一般結果を示しました。

更に、私の知識によって、こんな積分がガンマ関数で表せると言う一般論はないです。
\end{answer}
\begin{question}[\textbf{??$'$}先生]
	対称性破れ作用素\footnote{この質問の意味を間違えた可能性が高いと思います。自分の返事もあまり良くなかったと
	思います}を分類した時、この対称性破れ作用素をどうして積分核で現わされているものしか出てこないですか?
\end{question}
\begin{answer}
	まず、自分の表現論プロジェクトの場合で、$G$と$G'$の表現$\pi$と$\tau$は球腿化主系列表現で、
	これは$G$と$G'$上$C^\infty$関数空間で表されている表現なので、田内さんの話見たい、
	対称性破れ作用素は積分核で表されています。
	これはKobayashi-Spehに示された一般論から従います。
\end{answer}
\section{発表の日の流れ}
特に悪いと思うポイントが\textbf{ボールド体}でマークされています:
\begin{center}
	\begin{tabular}[]{lp{0.6\textwidth}}
		23:20時、6月20日&先生からスピーチについての最後の指示をもらいました\\
		1時、6月21日&スピーチの新しいバーションを作りました。この段階で練習を始めました\\
		〜2:30時、6月21日&{\bf 寝てしまいました}\\
		9:50時、6月21日&起きてしまいました\\
		11:05&{\bf RIMSに来て、講演を始めました}\\
		11:40&{\bf 発表は短すぎて、35分で終わりました}\\
	\end{tabular}
\end{center}
\section{悪い発表の理由}
\begin{enumerate}
	\item 今回の自分の内容は足りませんでしたと思います。こんなに少ない(田内君、島本君などに比べて、非常に少ないと
		思います)内容で1時間の話するのは無理だと思います。
		私は最初からRIMSに参加しなかった方が良かったと思います;

	\item 発表の前、寝てしまいましたため、練習足りませんでした;
	\item 発表に遅れてしまいました;
\end{enumerate}
\section{可能な対策}
\begin{enumerate}
	\item スライドを作る時、$x$分の話に対して、$x
		\sim{2x}$枚スライドを作ります;
	\item 発表の場所より最も近いホテルを選ぶ;
\end{enumerate}
\end{document}
