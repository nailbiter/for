%japanese
% !TEX TS-program = pdflatex
% !TEX encoding = UTF-8 Unicode

% This is a simple template for a LaTeX document using the "article" class.
% See "book", "report", "letter" for other types of document.

\documentclass[12pt]{article} % use larger type; default would be 10pt

\usepackage[utf8]{inputenc} % set input encoding (not needed with XeLaTeX)

%%% Examples of Article customizations
% These packages are optional, depending whether you want the features they provide.
% See the LaTeX Companion or other references for full information.

%%% PAGE DIMENSIONS
\usepackage{geometry} % to change the page dimensions
\geometry{a4paper} % or letterpaper (US) or a5paper or....
% \geometry{margin=2in} % for example, change the margins to 2 inches all round
% \geometry{landscape} % set up the page for landscape
%   read geometry.pdf for detailed page layout information

\usepackage{graphicx} % support the \includegraphics command and options

% \usepackage[parfill]{parskip} % Activate to begin paragraphs with an empty line rather than an indent

%%% PACKAGES
\usepackage{ulem}
\usepackage{setspace}
\usepackage{booktabs} % for much better looking tables
\usepackage{array} % for better arrays (eg matrices) in maths
\usepackage{paralist} % very flexible & customisable lists (eg. enumerate/itemize, etc.)
\usepackage{verbatim} % adds environment for commenting out blocks of text & for better verbatim
\usepackage{subfig} % make it possible to include more than one captioned figure/table in a single float
\usepackage{amssymb}
\usepackage{amsfonts}
\usepackage{amsmath}
\usepackage{xeCJK}
\usepackage{ruby}
\renewcommand\rubysep{-5ex}
\newcommand{\kana}[2]{\ruby{#1}{#2}}

%CJK font
%\setCJKmainfont[AutoFakeBold=true]{MS PGothic}
\setCJKmainfont[AutoFakeBold=true]{Hiragino Mincho Pro}
% These packages are all incorporated in the memoir class to one degree or another...

%%% HEADERS & FOOTERS
\usepackage{fancyhdr} % This should be set AFTER setting up the page geometry
\pagestyle{fancy} % options: empty , plain , fancy
\renewcommand{\headrulewidth}{0pt} % customise the layout...
\lhead{}\chead{}\rhead{}
\lfoot{}\cfoot{\thepage}\rfoot{}

%%% SECTION TITLE APPEARANCE
\usepackage{sectsty}
\allsectionsfont{\sffamily\mdseries\upshape} % (See the fntguide.pdf for font help)
% (This matches ConTeXt defaults)

%%% ToC (table of contents) APPEARANCE
\usepackage[nottoc,notlof,notlot]{tocbibind} % Put the bibliography in the ToC
\usepackage[titles,subfigure]{tocloft} % Alter the style of the Table of Contents
\renewcommand{\cftsecfont}{\rmfamily\mdseries\upshape}
\renewcommand{\cftsecpagefont}{\rmfamily\mdseries\upshape} % No bold!

%%%my commands
\newcounter{framecount}
\newcommand{\slide}{\noindent\stepcounter{framecount}$\backslash\backslash$[\arabic{framecount}]\\}
\newcommand{\mytime}[1]{\noindent #1\\}
\newcommand{\Sp}{\ensuremath \mathbb{S}^p}
\newcommand{\Sq}{\ensuremath \mathbb{S}^q}
%\newcommand{\kana}[2]{#1{\scriptsize (#2)}}
\newcommand{\J}{$J(\nu)$}
\newcommand{\I}{$I(\lambda)$}
\newcommand{\doubt}[1]{\fbox{#1}}
\newcommand{\mynum}{}
\newcommand{\pause}{$\bullet$}
\newcommand{\slowly}[1]{\dashuline{#1}}
\newcommand{\continuously}[1]{\underline{#1}}

%%% END Article customizations

%%% The "real" document content comes below...

\title{math17talk}
\author{}
\date{} % Activate to display a given date or no date (if empty),
         % otherwise the current date is printed 

\begin{document}
\doublespacing
\slide
初めまして。東京大学、数理科学研究科のレオンチエフ・アレックスと申します。
今日は
2つのゲーゲンバウアー多項式に関連する積分公式について
というタイトルでお話ししたいと思います。
最初は、アウトライン\footnote{any japanese for アウトライン?}から始めます。\\
\mytime{0:00:11}

%%\slide
%%$p$次元球面$\Sp$と$q$次元球面$\Sq$の\kana{直積}{チョクセキ}
%%多様体を考えましょう。$\Sp$\kana{上}{{ ジョウ}}に普通のリーマン計量を入れ、一方、\kana{第}{ダイ}2\kana{成分}{セイブン}の球面$\Sq$\kana{上}{{ ジョウ}}にはネガティブなリーマン計量を入れることによって、
%%直積多様体に符号が$(p,q)$
%%の不定値計量を与えます。この擬リーマン多様体の共形変換群$G$は、{不定値直交群}
%%$O(p+1,q+1)$と同型になります。特に、$q$が0と等しい場合は、球面の
%%\kana{共形変換群}{キョウケイヘンカングン}が、Lorentz群と同型になるという\kana{古典的}{コテンテキナ}な\kana{結果}{ケッカ}になります。更に、直積多様体$\Sp$かける$\Sq$の
%%\kana{対跡点}{タイセキテン}を\kana{同ー視}{ドウイツシ}することによって得られる商多様体は、群$G$を極大放物型部分群$P$で割ることによって得られる\kana{実}{ジツ}\pause\continuously{\kana{旗}{ハタ}
%%\kern-0.15cm多様体}$G/P$と同型になっています。更に、このdiagramにおける$\mathbb{R}^{p,q}$は\slowly{open Bruhat cell}
%%ですが、\kana{共形幾何}{キョウケイキカ}の\kana{立場}{タチバ}{からは}
%%conformal compactificationの\kana{逆写像}{ギャクシャゾウ}は\kana{立体射影}{リッタイシャエイ}
%%の一般化とみなせます。さて、$G/P$を擬リーマン多様体とみたとき、\kana{共形幾何}{キョウケイキカ}
%%の\kana{一般論}{イッパンロン}
%%から複素数$\lambda$をパ
%%ラメータとするconformally equivariant line bundlesのfamilyを定義することができます。
%%これを$L_\lambda$と書くことにします。そうすると、
%%$L_\lambda$の$C$無限sectionのなす
%%Fr\'echet空間上に実現される$G$の表現は$G$の球退化主系列表現になります。
%%この退化主系列表現を$I(\lambda)$と書くことにします。同様に、直積球面$
%%\Sp\times \Sq$\slowly{のcodimension one}の部分多様体に対して同じことを考えますと\kana{複素}{フクソ}
%%数パラメーター$\nu$に対して同じように部分群$G'=O(p,q+1)$の球退化主系列表現が定まります。これを\slowly{\J}と書くことにします。このようにして、群$G$とその部分群$G'$の無限次元表現
%%$I(\lambda)$と$J(\nu)$がそれぞれ定義されました。\\
%%\mytime{{0:02:04}}
%%
%%\slide
%%さて、部分群$G'$の表現に\kana{注目}{チュウモク}して$G$の表現
%%\I から$G'$の表現\J への連続な$G'$-intertwining operatorはsymmetry breaking operator,対称性破れ作用素と呼ばれます。ここで\I は群$G$の表現ですが、
%%\kana{制限}{セイゲン}することによって部分群$G'$の表現とみなしているの{です}。スライドでは\slowly{symmetry breaking operator}の
%%\kana{頭文字}{カシラモジ}を{とって}SBOと\kana{略記}{リャクキ}することにします。
%%これに関して重要な問題を\kana{挙}{ア}げましょう:まずすべての複素数パラメーター$(\lambda,\nu)$に対して、対称性破れ作用素を\kana{構成}{コウセイ}し、完全な分類を与えることが大きな\kana{目標}{モクヒョウ}
%%となります。更に、対称性破れ作用素の\kana{間}{アイダ}の\kana{函数等式}{カンスウトウシキ}や\slowly{\kana{留数}{リュウスウ}}などの性質を調べることも重要な問題です。
%%これらの\kana{課題}{カダイ}は、\kana{無限次元}{ムゲンジゲン}表現の分岐則の研究を\kana{深化}{シンカ}させるものですが、一般には非常に難しいもので、今まで\kana{殆ど}{ホトンド}
%%研究が行われてきていませんでした。しかし、2015年にアメリカ数学会のannalsに\kana{出版}{シュッパン}されたKobayashiとSpehの本で$q$は0と等しい場合に対してこの2つ問題の完全
%%な答えが証明されました。その\kana{成功}{セイコウ}の重要なストラテジーは次の3つだと思います。\\
%%\mytime{0:01:18}
%%
%%\slide
%%1つ\kana{目}{メ}はまず\kana{良}{ヨ}い\kana{設定}{セッテイ}
%%を選ぶことです。$\lambda$と$\nu$を\kana{止}{ト}めたとき、\kana{一次独立}{イチジドクリツナ}
%%な対称性破れ作用素が\kana{高々}{タカダカ}\kana{有限個}{ユウゲンコ}
%%しかない設定は、この問題に対するwell-posedなcaseと考えられます。
%%このための\kana{幾何学}{キカガク}的な条件はKobayashi先生とOshima Toshio先生
%%によって\kana{予め}{アラカジメ}研究されていました。
%%これは小林先生の\kana{提唱}{テイショウ}したプログラム、いわゆる Kobayashi Program Aに\kana{相当}{ソウトウ}する部分です。
%%\footnote{I was not sure that I've read Your instructions correctly. Did I position this sentence in the right place?}
%%更に、$P$や$P'$が\kana{極小}{キョクショウ}
%%放物型部分群の場合にはKobayashi先生とMatsuki先生による分類があります。
%%2つ目は\kana{実}{ジツ}旗多様体$G/P$における$P'$-invariantなclosed subsetそれぞれに対し、
%%対称性破れ作用素のfamilyを構成し、更に、そのmeromorphic continuationを証明するというステップです。\slowly{構成をした\kana{後}{アト}}は分類になります。
%%これが3つ目のステップです。Closed setが小さいものから\kana{帰納的}{キノウテキ}に\kana{順}{ジュン}に\kana{分類}{ブンルイ}します。
%%そのstarting pointは、closed setが1点の\kana{場合}{バアイ}{で}、
%%このときは対称性破れ作用素は微分作用素で表せ\doubt{ます}。
%%このときは新しい手法であるF-methodが使えます。Strategy(2)と(3)については、
%%成功の重要な理由が次のファクトだと思います。\\
%%\mytime{{0:01:12}}
%%
%%\slide
%%この\kana{図}{ズ}はKobayashi-Spehの\kana{一般論}{イッパンロン}を\kana{今}{イマ}\kana{考}{カンガ}{え}ようとしている\kana{場合}{バアイ}に\kana{当}{ア}てはめたものです。
%%下の$S\mbox{ol}\left(
%%\mathbb{R}^{p,q}
%%\right)$という空間はある$(\lambda,\nu)$に依存する偏微分方程式を満たす超関数空間です。ここで、左から真ん中への同型写像はSchwartzの\kana{核}{カク}
%%定理を\kana{用}{モチ}います、真ん中から下へのrestという写像は、open Bruhat cellに\kana{積分核}{セキブンカク}
%%を制限することによって\kana{導}{ミチビ}かれます。この2つの写像は線形空間の同型写像になります。
%%なので、抽象的な対称性破れ作用素空間の代わりに具体的なEuclid空間上の\kana{連立}{レンリツ}偏微分方程式の解空間を研究に\kana{置}{オ}き\kana{換}{カ}えることができます。
%%ポイントはもう1つあります。真ん中の空間は$G/P$上の$P'$不変超関数空間なので、それぞれの元のサーポトが$P'$不変$G/P$閉部分集合になります。
%%つまり、真ん中のベクトル空間から$G$の\kana{両側剰余}{リョウガワジョウヨ}空間へのが\kana{定}{サダ}{まった}ということです。Target spaceであるこの\kana{両側剰余}
%%{リョウガワジョウヨ}空間は有限なので、$P'$不変$G/P$閉部分集合が大切なinvariantになります。\\
%%\mytime{{0:01:14}}
%%
%%\slide
%%\kana{今}{イマ}、\kana{述}{ノ}べた
%%ストラテジー
%%を$G$が不定値直交群の\kana{場合}{バアイ}に\kana{用}{モチ}いる
%%\kana{第一歩}{ダイイッポ}として、両側\kana{剰余類}{ジョウヨルイ}とその\slowly{closure relations}、\kana{閉包}{ヘイホウ}関係の\kana{記述}{キジュツ}を述べます。
%%今の設定では、両側\kana{剰余類}{ジョウヨルイ}の\kana{個数}{コスウ}は、$p$は1以上ならば5つあり、$p$が1に等しいならば、4つにな
%%ることが証明されます。図式では、$X$が\slowly{\kana{全体}{ゼンタイ}}、すなわち実旗多様体を表し、edgeのとなりの数字がgenericなcodimensionを表します。\\
%%\mytime{{0:00:36}}
%%
%%\slide
%%次の結果
%%は対称性破れ作用素の\kana{構成}{コウセイ}では、ここでは
%%\doubt{と}、複素数パラメーターに正則に\quad つまり、holomorphicに依存する対称性破れ作用素のファミリーを3つ構成します。ここで$R_{\lambda,\nu}^X$は、その
%%\kana{核関数}{カクカンスウ}の
%%サーポトが
%%genericには$X$\kana{全体}{ゼンタイ}と等しくなるので、regular対称性破れ作用素と呼ばれています。$\tilde{R}_{\lambda,\nu}^X$はregular対称性破れ作用素のrenormalizationです。
%%\kana{//}{ナナメノニジョウセン}$\qquad$と\kana{|||}{タテノサンブンセン}とはパラメータ集合で、affine subspaceの\kana{可算和}{カサンワ}です。
%%最後に、$R^{\{o\}}_{\lambda,\nu}$は微分作用素{となります。この微分対称性破れ作用素の具体的な公式はGegenbauer多項式から}{導かれる2変数多項式を用いて記述できます。}\\
%%\mytime{0:00:53}
%%
%%\slide
%%3つ目の定理は、対称性破れ作用素の分類です。
%%\kana{次元}{ジゲン}について\kana{述}{ノ}べるとすると
%%対称性破れ作用素の空間はgenericには1次元ですが、\kana{可算無限離散}{カサンムゲンリサン}集合では
%%\slowly{2次元}になることが証明されます。\\
%%\mytime{0:00:14}
%%
%%\slide
%%次の定理はregular対称性破れ作用素の
%%\slowly{\kana{留数}{リュウスウ}定理}です。ここに\kana{記}{シル}した2つの超関数はそれぞれ正則パラメータをもつ超関数ですが、そのwavefront setが重なりをもつので、
%%\kana{超関数}{チョウカンスウ}の
%%\slowly{\kana{積}{セキ}}はwell-defined
%%とは限りません。well-definedでない\kana{場所}{トコロ}にまた新しいpoleが\kana{生}{ショウ}じます。このpoleのresidueは微分対称性破れ作用素
%%になります。より\kana{精密}{セイミツ}に、比例\kana{定数}{テイスウ}を
%%\slowly{\kana{初等}{ショトウ}}\kana{関数}{カンスウ}の{\kana{積}{セキ}}として具体的に\kana{表}{アラワ}すことができました。\\
%%\mytime{0:00:28}
%%
%%\slide
%%さて、intertwining operatorと対称性破れ作用素の\kana{合成}{ゴウセイ}は\kana{再び}{フタタビ}対称性破れ作用素になります。
%%次の結果は、intertwining operatorとして\kana{古典}{コテン}的なKnapp-Stein作用素
%%を\kana{選}{エラ}んだ{とき}、それと
%%ここで\kana{求}{モト}めた
%%新しいregular対称性破れ作用素の\kana{合成}{ゴウセイ}がどのようになるかという\kana{問}{トイ}に答えるものです。以下のような関数等式を得ました。\\
%%\mytime{0:00:21}
%%
%%\slide
%%最後に\slowly{\kana{手法}{シュホウ}}について{すこし}\kana{触}{フ}れましょう。2つの無限次元表現の間の対称性破れ作用素を調べるために、特定のベクトルに対する作用を具体的に計算します。
%%\kana{異なる}{コトナル}空間の上での写像ですが、コンパクト群の作用によるある\kana{種}{{シュ}}の標準化を用いることによって、一般化した意味での\kana{固有値}{コユウチ}
%%を定義することができます。この固有値がいつゼロになるかどうかを決定することが\kana{鍵}{カギ}になります。{そのための}、いくつかの公式を証明しました。その1つは、次の定理です。
%%\mytime{0:00:38}
%%
%%\slide
%%この結果は、\I の$K$-finiteベクトルのイメージが\J の$K'$-finiteベクトルと比例にな
%%ることを、\kana{精密}{セイミツ}に\kana{表}{アラワ}したものです。
%%比例定数も計算できて、\kana{零点}{レイテン}が分かるように積公式で表せます。Bernstein-Reznikovは$p=q=1$、Kobayashi-Spehは$p$は一般で$q=0$の場合にそれぞれ
%%\kana{先行結果}{センコウケッカ}がありす。\\
%%\mytime{0:00:22}
%%
%%\slide
%%次の\kana{命題}{メイダイ}では、$s+t$の\kana{ベキ乗}{ベキジョウ}を2つのGegenbauer\kana{多項式}{タコウシキ}
%%で\kana{展開}{テンカイ}するという公式です。実はもう少し一般の公式も証明できているのですが、ここでの発表では記述を簡単にするため、\kana{少し}{スコシ}
%%\kana{特殊化}{トクシュカ}してで\kana{記述}{キジュツ}しました。
%%\kana{係数}{ケイスウ}は$\Gamma$\kana{関数}{カンスウ}の積として表せるので、その\kana{零点}{レイテン}が\kana{完全}{カンゼン}に
%%\kana{決定}{ケッテイ}できるの{が}\kana{応用上}{オウヨウジョウ}\kana{重要}{ジュウヨウ}なポイントとなります。
%%しかし、この\kana{特殊}{トクシュ}な\kana{場合}{バアイ}
%%でも、いろいろな文献を探したのですが、見つけることができませんでした
%%ので、Propositionという\kana{形}{\doubt{カタチ}}で述べさせていただきました。
%%もし、何かご\kana{存知}{ゾンジ}でしたら、\kana{後}{アト}で\kana{教えて}{オシエテ}いただけるとありがたいです。
%%\kana{今}{イマ}の
%%\kana{時点}{ジテン}で、私が\kana{理解}{リカイ}
%%している\kana{先行結果}{センコウケッカ}との関係をhierarchyの\kana{図式}{ズシキ}にしてみました。
%%\kana{青}{アオ}い\kana{数字}{スウジ}は\kana{公式}{コウシキ}に\kana{含}{フク}まれる連続パラメータの\kana{個数}{コスウ}です。
%%\\
%%\mytime{0:00:49}
%%
%%\slide
%%この\kana{講演}{コウエン}ではより一般の公式K-Lについてはお話していません。
%%$\ell=m=0$の場合は Warnaar, Varchenko, TarasovなどによるSelberg typeの積分の一般化とも関係しています。
%%この積分公式も、ここで求めた\kana{新}{アタラ}しい対称性破れ作用素の\kana{性質}{セイシツ}を調べるのに用いられます。

\noindent 以上です。どうも有り難うございます。\\
\mytime{0:00:19}

TOTAL: 0:10:39\\
\begin{tabular}[]{l|l}
	$\bullet$&pause\\
	\underline{one two}&speak continuously\\
	\dashuline{diffffficult}&speak slowly
\end{tabular}\\
NB:\begin{enumerate}
	\item pronounce ``Bruhat'' without ``r'';
	\item pronounce ``R'' like アール, not like エル;
	\item pronounce ``初等'' as ショトウ, not ショウトウ;
\end{enumerate}
\end{document}
