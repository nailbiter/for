%japanese
% !TEX TS-program = pdflatex
% !TEX encoding = UTF-8 Unicode

% This is a simple template for a LaTeX document using the "article" class.
% See "book", "report", "letter" for other types of document.

\documentclass[12pt]{article} % use larger type; default would be 10pt

\usepackage[utf8]{inputenc} % set input encoding (not needed with XeLaTeX)

%%% Examples of Article customizations
% These packages are optional, depending whether you want the features they provide.
% See the LaTeX Companion or other references for full information.

%%% PAGE DIMENSIONS
\usepackage{geometry} % to change the page dimensions
\geometry{a4paper} % or letterpaper (US) or a5paper or....
% \geometry{margin=2in} % for example, change the margins to 2 inches all round
% \geometry{landscape} % set up the page for landscape
%   read geometry.pdf for detailed page layout information

\usepackage{graphicx} % support the \includegraphics command and options

% \usepackage[parfill]{parskip} % Activate to begin paragraphs with an empty line rather than an indent

%%% PACKAGES
\usepackage{ulem}
\usepackage{setspace}
\usepackage{booktabs} % for much better looking tables
\usepackage{array} % for better arrays (eg matrices) in maths
\usepackage{paralist} % very flexible & customisable lists (eg. enumerate/itemize, etc.)
\usepackage{verbatim} % adds environment for commenting out blocks of text & for better verbatim
\usepackage{subfig} % make it possible to include more than one captioned figure/table in a single float
\usepackage{amssymb}
\usepackage{amsfonts}
\usepackage{amsmath}
\usepackage{xeCJK}
\usepackage{ruby}
%\renewcommand\rubysep{-5ex}
\newcommand{\kana}[2]{\ruby{#1}{#2}}

%CJK font
%\setCJKmainfont[AutoFakeBold=true]{MS PGothic}
\setCJKmainfont[AutoFakeBold=true]{Hiragino Mincho Pro}
% These packages are all incorporated in the memoir class to one degree or another...

%%% HEADERS & FOOTERS
\usepackage{fancyhdr} % This should be set AFTER setting up the page geometry
\pagestyle{fancy} % options: empty , plain , fancy
\renewcommand{\headrulewidth}{0pt} % customise the layout...
\lhead{}\chead{}\rhead{}
\lfoot{}\cfoot{\thepage}\rfoot{}

%%% SECTION TITLE APPEARANCE
\usepackage{sectsty}
\allsectionsfont{\sffamily\mdseries\upshape} % (See the fntguide.pdf for font help)
% (This matches ConTeXt defaults)

%%% ToC (table of contents) APPEARANCE
\usepackage[nottoc,notlof,notlot]{tocbibind} % Put the bibliography in the ToC
\usepackage[titles,subfigure]{tocloft} % Alter the style of the Table of Contents
\renewcommand{\cftsecfont}{\rmfamily\mdseries\upshape}
\renewcommand{\cftsecpagefont}{\rmfamily\mdseries\upshape} % No bold!

%%%my commands
\newcounter{framecount}
\newcommand{\slide}{\noindent\stepcounter{framecount}$\backslash\backslash$[\arabic{framecount}]\\}
\newcommand{\mytime}[1]{\noindent #1\\}
\newcommand{\Sp}{\ensuremath \mathbb{S}^p}
\newcommand{\Sq}{\ensuremath \mathbb{S}^q}
%\newcommand{\kana}[2]{#1{\scriptsize (#2)}}
\newcommand{\J}{$J(\nu)$}
\newcommand{\I}{$I(\lambda)$}
\newcommand{\doubt}[1]{\fbox{#1}}
\newcommand{\mynum}{}
\newcommand{\pause}{$\bullet$}
\newcommand{\slowly}[1]{\dashuline{#1}}
\newcommand{\continuously}[1]{\underline{#1}}

%%% END Article customizations

%%% The "real" document content comes below...

\title{math17talk}
\author{}
\date{} % Activate to display a given date or no date (if empty),
         % otherwise the current date is printed 

\setlength{\parskip}{2em}

\begin{document}
%\huge
\doublespacing
\slide
初めまして。東京大学、数理科学研究科のレオンチエフ・アレックスと申します。
今日は
2つのゲーゲンバウアー多項式に関連する積分公式について
というタイトルでお話ししたいと思います。
この研究は小林俊行先生と 
の\kana{共同}{キョウドウ}研究です。
%\mytime{0:00:11}

%%%\slide
%%まず\footnote{is this good word for formal speech?}、主定理のご紹介から始めたいと思います。
%%次は、
%%%主定理のモチベーションについて少しい言って、
%%主定理公式が関連すると様々な結果を少しい触ります。
%%最後は、お時間があれば、主定理の証明を述べます。\\
%%
\slide
\slide
最初は、主結果のご紹介ため、Gegenbauer多項式という多項式ファミリの定義と性質を復習したいと思います。
実部が十分大き複素数$\lambda$をパラメータとすると、
$[-1,1]$空間上で\kana{重み}{オモミ}関数 $(1-x^2)^{\alpha-1/2}$の直交多項式を定義することができます。
このファーミリは
Gegenbauer多項式(あるいは、超球多項式)と呼ばれています。
Gegenbauer多項式は様々な同値定義があり、例えば以下の微分方程式の特別な解として定義することもできます。
コンパクト群$O(n)$の分規則ルールの具体的な形を調べるとき、Gegenbauer多項式も出て来ます。

\slide
主結果は、$s+t$の\kana{ベキ乗}{ベキジョウ}を2つのGegenbauer\kana{多項式}{タコウシキ}
で\kana{展開}{テンカイ}するという公式です。
\kana{係数}{ケイスウ}は $\Gamma$関数の積として表せるので、その零点が完全に決定できるのが\kana{応用上}{オウヨウジョ}重要なポイントとなります.\\
この結果を、色々な文献を探したので
すが、見つけることができませんでしたので、Proposition という形で述べさせていただきました。もし、何かご存知でしたら、後で教えていただけるとあ
りがたいです。

\slide
実はもう少し一般の公式も証明できているのですが、ここでの発表では記述を簡単にするため、\kana{少し}{スコシ}
\kana{特殊化}{トクシュカ}してで\kana{記述}{キジュツ}しました。\\
一般な結果で
\kana{係数}{ケイスウ}は単純な$\Gamma$関数の積でなく、超幾何関数$_{2}F_1$を含むので、ご注意ください。

\slide
命題1は実際には色々な同型形があります。
例えば、命題$1'$に記述されたように、積分公式として述べられます。
\kana{換言}{カンゲン}すれば、命題$1'$の積分公式は命題$1$の展開公式と同値です。

\slide
今の 時点で、私が理解 している先行結果との関係を今から述べます。
最初の関連結果をご紹介する前、エルミート多項式というものを復習したいと思います。
エルミート多項式というのは、
$\mathbb{R}$空間上で\kana{重み}{オモミ}関数 $e^{-x^2/2}$の直交多項式ファミリです。\\
しかも、\kana{上記}{ジョウキ}でご紹介されたGegenbauer多項式の極限とみなせます。
%%発表では記述を簡単にするためここで述べませんが、
%%\kana{上記}{ジョウキ}でご紹介された命題2もう積分公式としての形があります。
%%この積分公式に{制限手続き}を応用すれば、
%%以下のHermite多項式に関する積分公式を得ます。因みに、この公式も文献にまだ見つけていません。
%%下の注2に述べたように、よく知られているMehta積分の特別な$k=2$の場合が
%%系1から従います。

\slide
前のスライドに述べたように、エルミート多項式がGegenbauer多項式の極限として見なせるので、
命題$1'$の積分公式に極限とれば、Gegenbauer多項式に関する積分公式から、
エルミート多項式に関する公式を得ます。\\
命題$1'$に対応する展開公式命題1があるように、この系に対応する展開公式も存在します。
しかし、ここでの発表で
は記述を簡単にするため、こ省略させていただきました。

\slide
そうだけでなく、
系1の $w = 1, \ell = m = 0$ の特殊化がMehta積分
の $k = 2$ の特別な場合になります。
Mehta積分というのは、物理で出てくるセルバーグ積分の系です。

\slide
$\ell = m = 0$ の
場合は Warnaar, Varchenko, Tarasov などによる セルバーグ type の積分の一般化 とも関係しています。
\kana{今}{イマ}の
\kana{時点}{ジテン}で、私が\kana{理解}{リカイ}
している\kana{他の先行結果}{センコウケッカ}との関係をhierarchyの\kana{図式}{ズシキ}にしてみました。
\kana{青}{アオ}い\kana{数字}{スウジ}は\kana{公式}{コウシキ}に\kana{含}{フク}まれる連続パラメータの\kana{個数}{コスウ}です。

\slide
今から、この関連結果を一個一個でご紹介したいと思います。
最初は、ここにいらっしゃる皆様がよくご ガイネン
存知セルバーグの1994年の結果からご紹介始めたいと思います。

\slide
図式に述べたように、
命題$1'$の $\ell = m = 0,\lambda=\mu$ の特殊化が
セルバーグ積分
の $k = 2,\alpha=\beta$ の特別な場合になります。

\slide%Warnaar
次は、Warnaarの2010年のセルバーグ積分の一般化との関係を明らかにしたいと思います。

\slide
図式に述べたように、
命題$1'$の $\ell = m = 0,\lambda+\mu+2\nu=-1$ の特殊化が
セルバーグ積分
の $k_1=k_2 = 2,\alpha_1=\beta_1,\alpha_2=\beta_2$ の特別な場合になります。

\slide%Tarasov-Varchenko
次は、TarasovとVarchenkoの2003年のセルバーグ積分の一般化との関係をご説明したいと思います。

\slide
図式に述べたように、
命題$1'$の $\ell = m = 0,\mu=\frac{1}{2}$ の特殊化が
セルバーグ積分
の $k = 2,\beta_2=1,\alpha=\beta_1$ の特別な場合になります。

\slide%Dotsenko-Fateev
最後は、DotsenkoとFateevの1985年のー般化との関係をご説明します。

\slide
図式に述べたように、
命題$1'$の $\ell = m = 0,\nu=-1$ の特殊化が
セルバーグ積分
の $m=n=1,\alpha'=\beta',\alpha=\beta$ の特別な場合になります。

\slide%HR
最後の関連結果として、系1の公式に積分変換すれば、2つHermite-Rodriguez関数の\kana{畳み込み}{タタミコミ}の
メリン変換の閉公式を得ます。
Hermite-Rodriguez関数というのは、\kana{機械学習}{キカイガクシュウ}のニューラルネットワークという分野で使われている関数です。
%%メリン逆変換応用すれば、Hermite-Rodriguez関数の\kana{畳み込み}{タタミコミ}の閉公式を得ます。

\slide
今から主結果命題1を示す方法を4つ述べたいと思います。しかし、
証明方法を述べる前、トリックをご紹介したいと思います。\\
今の補題に明らかにされているように、命題1(或いは、命題2)の一般な場合が
特別な場合$\ell=m=0$から従います。

\slide%M1
それで、命題1の最初の証明方法をご紹介します。
前のスライドに述べたように、$\ell=m=0$の特別な場合を示せば良いです。
{超幾何関数のオイラー積分表示}を用い、
次に級数展開を用いて、求める積分を${}_3F_2$で表せます。こうして、Whipple sumによって、得られた${}_3F_2$超幾何関数を$\Gamma$関数の積で表せます。

\slide%M2
よく知られているCarlson定理という強い結果を認めれば、もっと簡単な証明を\kana{提供}{テイキョウ}できます。
もし、あまりご存知ではない方がいらっしゃれば、Carlson定理というのは、exponetial typeという広い正則関数クラスに関する定理です。
みなさんはよくご存知通りに、普通の右半面上での正則関数は、整実数上でゼロであれば、\kana{恒等的}{コウトウテキ}に 0なります。
しかし、exponential typeであれば、更にもう一個条件を満たせば、同じ結果は整数上だけの情報から従うということがCarlsonの定理によって明らかにされています。

%%\newcommand{\slide}{\noindent\stepcounter{framecount}$\backslash\backslash$[\arabic{framecount}]\\}
%\newcounter{framecount}
\addtocounter{framecount}{-1}
\slide
この結果を認めれば、命題$1'$の積分等式の$\nu$は整数の特別な場合を示せば良いということが分かります。
この場合で、$s-t$のベキ乗が多項式になって、左辺の積分がBeta積分の有限和になります。\\
面白いですが、元々のセルベルグ積分のセルベルぐ最初の証明がこの方法を使っていました。

\slide
次は、3つ目のアプローチをご紹介したいと思います。
流れ的には前に説明されたメスッド1と似ていますが、示すのは
命題1じゃなくて、もっと一般的な命題2です。まず、メソッド1のように、
$\ell=m=0$の場合(つまり、一番上の積分公式)を示せば良いです。\\
オイラー積分表示を用いり、${}_2F_1$超幾何期関数の二次変換を用いると
一番下の公式を示せば良いということが明らかになります。
では、その公式は超幾何関数${}_2F_1$の係数展開とあるの超幾何関数${}_2F_1$に関数る等式から従います。

\slide
最後は、メソッド4をご紹介します。
この方法はそれなりに一番直接です。
アイディアは非常に簡単で、二つ知られている積分公式を使って、2変数の積分
順々に計算するということです。
具体的にいうと、まず、小林俊行先生とMano Gen先生の2011年の論文に
出てくる積分公式用いり、$s$に対しての積分をルジャンドル陪関数で表せます。
また、\kana{上記}{ジョウキ}で説明されたトリックのように、微分積分と
ロドリゲスの公式を用い、もっと簡単な場合に\kana{帰着}{キチャク}します。
最後は、同じい小林先生とMano先生の論文に出てくる公式によって、最後の
積分を$\Gamma$関数の席で表せます。

\slide
残りの時間でここで述べた結果と表現論を関係に軽く触れたいと思います。
まず、ノンコンパクト群$G$とその無限次元表現$\pi$と
ノンコンパクト部分群$G'$とその無限次元表現$\tau$を定めましょう。
$G'$は$G$の部分群なので、$\pi$を$G'$の表現としてみなせます。
そうすると、$\pi$から$\tau$までの$G'$-intertwining作用素を全て構成し、分類し、
性質を詳しく調べるという問題を考えましょう。

\slide
ある無限次元表現のクラスに対して(例えば、
主系列表現)、その作用素は超関数格を持つ積分変換で表せます。
異なる空間の上での写像ですが、コンパクト群の作用によるある種の標準
化を用いることによって、一般化した意味での固有値 を定義することができ
ます。

\slide
シューアの補題のように、ある意味での\kana{対角化}{タイカクカ}できます。
今の図式で表されているような条件がなります。
ここで、別様パターンが別様な$K'$-タイプを表します。

\slide
この固有値がいつゼロになるかどうかを決定することが鍵になります。
表現空間間の作用素は積分核で表されているので、
固有値は積分で表せます。
なので、積分を具体的に表せる公式を求めます。命題$1'$を示すの元々の
モティベーションはこんな感じでした。
以上です。どうも有り難うございます。

%%次は、同じい論文

%%TOTAL: 0:10:39\\
\begin{tabular}[]{l|l}
	$\bullet$&pause\\
	\underline{one two}&speak continuously\\
	\dashuline{diffffficult}&speak slowly\\
	\doubt{doubt}&I am in doubt (e.g. not sure that I read Your instructions correctly etc.)
\end{tabular}\\
NB:\begin{enumerate}
	\item pronounce ``Bruhat'' without ``r'';
	\item pronounce ``R'' like アール, not like エル;
	\item pronounce ``初等'' as ショトウ, not ショウトウ;
\end{enumerate}
TODO (speech): \begin{enumerate}
\item Do not use numbers like ``proposition 1'', ``proposition 3'' during the speech -- no one will remember them.
	Come up with some ``aliases'' (maybe, ``general proposition'', ``special case'' etc.).
	Or maybe, make one proposition to be theorem, another to be observation etc;
\end{enumerate}
\end{document}
