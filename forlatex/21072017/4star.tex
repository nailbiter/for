	\begin{cor}\label{cor:int-xzy-hh}
		$\nu\in\mathbb{C}$と$w\in\R$に対して、以下の積分公式が成り立つ:
		\begin{multline}
			\int_{- \infty}^{\infty} \int_{- \infty}^{\infty} | x - w y |^{2 \nu} e^{-
			x^2 - y^2} H_\ell (x) H_m (y) d x d y \\= (- \nu)_{\frac{\ell + m}{2}} (- 1)^{\frac{\ell
			- m}{2}} 2^{\ell + m} \pi^{\frac{1}{2}} \Gamma \left( \frac{1}{2} + \nu \right)
			(w^2 + 1)^{\nu - \frac{\ell + m}{2}} w^m .
		\end{multline}
	\end{cor}
	系\ref{cor:int-xzy-hh}の
	2通りの
	特殊値を
	考えると、それぞれ
	Mehta積分
	の特殊値、 Hermite--Rodriguez 関数の畳み込みの公式が再証明できる。
	\begin{fact}[Mehta積分, {\cite{mehta2004random}}]
			{
		\begin{equation*}
			\begin{array}[]{c}
			{(2\pi)^{-\frac{k}{2}}}\displaystyle\int_{\mathbb{R}^k}\prod_{1\le i<j\le k}\myabs{t_i-t_j}^{2\gamma}\exp\left( -\myabs{\mathbf{t}}^2/2 \right)d\mathbf{t}
			=\displaystyle\prod_{j=1}^k\frac{\Gamma\left( 
			1+j\gamma\right)}{\Gamma(1+\gamma)}.
			\end{array}
		\end{equation*}
	}
		\end{fact}

		系\ref{cor:int-xzy-hh}の $w = 1, \ell = m = 0$ の特殊化はMehta積分\cite{mehta2004random}
の $k = 2$ の場合になる:
\vspace{1em}
		\myrelationdiagram{系\ref{cor:int-xzy-hh}}{w=1\\\ell=m=0}{k=2}{Mehta積分}
		{$\int_{-\infty}^\infty\int_{-\infty}^\infty\myabs{x-y}^{2\nu}e^{-x^2-y^2}dxdy=\mypgf$}
