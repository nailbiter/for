\documentclass[12pt,a4paper,dvipdfmx]{jsarticle}
%%% PACKAGES
\usepackage{graphicx} % support the \includegraphics command and options
\usepackage{framed}
\setlength\FrameSep{0.5em}
\setlength\OuterFrameSep{\partopsep}
\usepackage{bigints}
\usepackage{textpos}
\usepackage{color,soul}
\usepackage{wrapfig}
\usepackage{mathtools}
\usepackage[all,cmtip]{xy}
\usepackage{mystyle}
\usepackage{lpic}
\usepackage{etoolbox}
\usepackage{comment}
\usepackage[normalem]{ulem}
\usepackage{setspace}
\usepackage{booktabs} % for much better looking tables
\usepackage{array} % for better arrays (eg matrices) in maths
\usepackage{paralist} % very flexible & customisable lists (eg. enumerate/itemize, etc.)
\usepackage{verbatim} % adds environment for commenting out blocks of text & for better verbatim
\usepackage{subfig} % make it possible to include more than one captioned figure/table in a single float
\usepackage{amssymb}
\usepackage{amsfonts}
\usepackage{amsmath}
\usepackage{amssymb}
\usepackage{amsfonts}
\usepackage{amsmath,amsthm}
\usepackage{amssymb,amssymb,amsthm,lmodern}
\usepackage[driver=dvipdfm,truedimen,top=3truecm,bottom=3truecm,left=2.5truecm,right=2.5truecm]{geometry}
\usepackage{hyperref}
\usepackage{tikz}
\usetikzlibrary{shapes,arrows,patterns}

\numberwithin{equation}{section}

\newcommand{\myre}[1]{\tmop{Re} #1}
\newcommand{\mysbo}{A:\pi\kern-0.1cm\mid_{G'}\xrightarrow{G'}\tau}
\newcommand{\myrelationdiagram}[5]{
	\vspace{1em}
		\centerline{
		\xymatrix{
			\framebox{\mbox{#1}}\ar@{=>}[rd]^{\begin{array}[]{c}
				#2
			\end{array}}&&\framebox{\mbox{#4}}\ar@{=>}[ld]_{\begin{array}[]{c}
			#3
	\end{array}}\\
			&
			\framebox{\mbox{#5}}
			&
		}}
	\vspace{1em}
	}
\newenvironment{proof*}[1]{\noindent\textbf{#1\ }}{\hspace*{\fill}\medskip}
\newcommand{\myrelationdiagramBigSix}[6]{
			\begin{samepage}
				\vspace{1em}
		\centerline{
		\xymatrixcolsep{#6cm}
		\xymatrix{
			\framebox{\mbox{#1}}\ar@{=>}[rd]^{\begin{array}[]{c}
				#2
			\end{array}}&&&&\framebox{\mbox{#4}}\ar@{=>}[ld]_{\kern-1.5cm\begin{array}[]{c}
			#3
	\end{array}}\\
			&&&&
		}}
		\nopagebreak
		\framebox{\mbox{#5}}
				\vspace{1em}
			\end{samepage}
	}
\newcommand{\myrelationdiagramBigCenter}[5]{
			\begin{samepage}
				\vspace{1em}
		\centerline{
		\xymatrixcolsep{2.5cm}
		\xymatrix{
			\framebox{\mbox{#1}}\ar@{=>}[rd]^{\begin{array}[]{c}
				#2
			\end{array}}&&&&\framebox{\mbox{#4}}\ar@{=>}[ld]_{\kern-1.5cm\begin{array}[]{c}
			#3
	\end{array}}\\
			&&&&
		}}
		\nopagebreak
		\centerline{
			\framebox{\mbox{#5}}}
				\vspace{1em}
			\end{samepage}
	}
\newcommand{\myrelationdiagramBig}[5]{
			\begin{samepage}
				\vspace{1em}
		\centerline{
		\xymatrixcolsep{2.5cm}
		\xymatrix{
			\framebox{\mbox{#1}}\ar@{=>}[rd]^{\begin{array}[]{c}
				#2
			\end{array}}&&&&\framebox{\mbox{#4}}\ar@{=>}[ld]_{\kern-1.5cm\begin{array}[]{c}
			#3
	\end{array}}\\
			&&&&
		}}
		\nopagebreak
		\framebox{\mbox{#5}}
				\vspace{1em}
			\end{samepage}
	}

\newcommand{\red}[1]{{\color[rgb]{0.6,0,0}#1}}
	\definecolor{Blue}{rgb}{0,0.0,1}
	\definecolor{Red}{rgb}{1,0.0,0}

\newcommand*\mytextcircled[1]{\tikz[baseline=(char.base)]{
	\node[shape=circle,draw,inner sep=1.8pt] (char) {\scalebox{0.8}{\kern-0.05cm #1}};}}
\newcommand{\mybigplus}{\scalebox{2.0}{$+$}}
\renewcommand{\implies}{\Rightarrow}
\newcommand{\mypgf}{{\mbox{ガンマ関数の積}}}
\newcommand{\Sol}{\mathcal{S}\mbox{ol}}
\newcommand{\Ind}{\mbox{\normalfont Ind}}
\newcommand{\D}{\mathcal{D}}
\newcommand{\A}{\mathcal{A}}
\newcommand{\Co}{\mathbb{C}}
\newcommand{\X}{\mathbb{X}}
\renewcommand{\setminus}{\backslash}
\newcommand{\nin}{\not\in}
\newcommand{\tmop}[1]{\ensuremath{\operatorname{#1}}}
\newcommand{\tmtextbf}[1]{{\bfseries{#1}}}
\newcommand{\tmtextit}[1]{{\itshape{#1}}}
\newcommand{\mss}{//}
\newcommand{\mbb}{\backslash\backslash}
\newcommand{\mmm}{\mid\mid}
\catcode`\<=\active \def<{
\fontencoding{T1}\selectfont\symbol{60}\fontencoding{\encodingdefault}}
\catcode`\>=\active \def>{
\fontencoding{T1}\selectfont\symbol{62}\fontencoding{\encodingdefault}}
\newcommand{\assign}{:=}
\newcommand{\comma}{{,}}
\newcommand{\um}{-}
\newcommand{\sol}{\mathcal{S}ol(\R^{p,q};\lambda,\nu)}
\newcommand{\Op}{\mbox{\normalfont Op}}
\newcommand{\Res}{\operatorname{Res}\displaylimits}
\newcommand{\OpR}{\mbox{\it R}}

%%%%%%%%%%%%%%%%%%%%%%%%%%%%%%%%%%%%%%%%%%%%%%%%%%%%
\makeatletter
\def\@author@list{}
\renewcommand{\author}[4]{%
  \ifx\@author@list\@empty\else\g@addto@macro\@author@list{\medskip}\fi
  \g@addto@macro\@author@list{\noindent #2\quad#1\par\noindent #3\par\noindent #4\par}%
}
\def\@maketitle{%
  \begin{center}
  {\LARGE \@title}\par
  \vskip 1.5em
  {\large\@author@list}%
  \end{center}
  \par\vskip 1.5em
}
\let\original@proof=\proof
\let\original@endproof=\endproof
\def\proof{\@ifnextchar[{\proof@}{\proof@[\proofname]}}
\def\proof@[#1]{%
  \labelsep=1zw
  \original@proof[{\headfont #1\inhibitglue}\nopunct]
}
\makeatother
\newtheoremstyle{jplain}{}{}{\normalfont}{}{\headfont}{}{1zw}{\thmname{#1}\thmnumber{\ #2}\thmnote{(#3)}}
\theoremstyle{jplain}
\renewcommand{\proofname}{証明}
% 概要の見出しを英語に
\renewcommand{\abstractname}{Abstract}
%%%%%%%%%%%%%%%%%%%%%%%%%%%%%%%%%%%%%%%%%%%%%%%%%%%%

% 定理環境の定義
\newcommand{\taggedpropsymb}{$\;1'$}
\newtheorem{thm}{定理}[section]
\newtheorem{lem}[thm]{補題}
\newtheorem*{lemma*}{補題}
\newtheorem{lemma}[thm]{補題}
% その他ご自由に……
\newtheorem{prop}[thm]{命題}
\newtheorem*{prop*}{命題}
\newtheorem{example}[thm]{例}
\newtheorem{cor}[thm]{系}
\newtheorem{method}{記明法}
\newtheorem{fact}[thm]{Fact}
\newtheorem*{fact*}{Fact}
\newtheorem*{example*}{例}
\newtheorem*{goal*}{目標}
\theoremstyle{remark}
\newtheorem*{remark*}{注意}
\newtheorem{remark}[thm]{注意}
\newtheorem*{setting*}{設定}
\newtheorem{setting}[thm]{設定}
\theoremstyle{definition}
\newtheorem{definition}[thm]{定義}

%proofreading symbols
\newcommand{\mykana}[2]{#1}
\newcommand{\doubt}[1]{\fbox{#1}}
\newcommand{\pause}{$\bullet$}
\newcommand{\slowly}[1]{\dashuline{#1}}
\newcommand{\continuously}[1]{\underline{#1}}
\newcommand{\badword}[1]{\uwave{#1}}

\newtheorem{taggedpropx}{命題}
\newenvironment{taggedprop}[1]
 {\renewcommand\thetaggedpropx{#1}\taggedpropx}
  {\endtaggedpropx}

% 文書情報
\title{2つのGegenbauer多項式を含むある積分公式}
\author{小林 俊行}{東京大学 大学院数理科学研究科、カブリ数物連携宇宙研究機構}{Toshiyuki Kobayashi}{
Graduate School of Mathematical Sciences, the University of Tokyo,\\
Kavli Institute for the Physics and Mathematics of the Universe}
\author{レオンチエフ アレックス}{東京大学 大学院数理科学研究科}{Alex Leontiev}{Graduate School of Mathematical Sciences,
The University of Tokyo}
%%\author{阿部 紀行}{北海道大学 理学研究院}{Noriyuki Abe}{Department of Mathematics, Hokkaido University}

\begin{document}
\maketitle
\begin{abstract}%TODO: write in japanese
Gegenbauer多項式
を2つ含んだある種の積分の具体的
公式を与える。
得られた公式の特殊値や極限値と、古典的な積分公式やWarnaar、Varchenko,Tarasov積分などによる種々のSelberg型積分の特殊値との関連について説明したい。
主結果の複数の証明方法を挙げる。
最後に、対称性破れ作用への応用について触れる。%. この研究は小林俊行先生との共同研究である。
\end{abstract}
\section{主結果}
	最初に、Gegenbauer多項式の定義と性質を復習する。
	Gegenbauer多項式$y=C^\lambda_n(x)$は以下の微分方程式
	\begin{equation*}
		(1-x^2)y''-(2\lambda+1)xy'+n(n+\lambda)y=0
	\end{equation*}
	を満たす多項式である。さらに、
	$\lambda$が実数で$\lambda>-\frac{1}{2}$をみたすとき、
	Gegenbauer多項式$\left\{ C_n^\lambda(x)\right\}_{n=1}^{\infty}$は
	$L^2\left( [-1,1],(1-x^2)^{\lambda-\frac{1}{2}}dx \right)$の直交多項式とな{る}。
	最初の数項を挙げよう。
%%		最初のいくつは
		\begin{eqnarray*}
			C_0^\lambda(x)&=&1,\\
			C_1^\lambda(x)&=&2\lambda x,\\
			C_2^\lambda(x)&=&-\lambda+2\lambda(1+\lambda)x^2,\\
			C_3^\lambda(x)&=&-2\lambda(1+\lambda)x+\frac{4}{3}\lambda(1+\lambda)(2+\lambda)x^3,\\
			C_4^\lambda(x)&=&\frac{1}{2}\lambda(1+\lambda)-2\lambda(1+\lambda)(2
			+\lambda)x^2+\frac{2}{3}\lambda(1+\lambda)(2+\lambda)(3+\lambda)x^4.
		\end{eqnarray*}
	まず、本稿の
	主結果を述べる。
	%は以下のようになる:
	$\lambda,\mu,\nu$の有理型関数$b(\lambda,\mu,\nu)$と、
		$\ell,m\in\N$に対して$\lambda,\mu,\nu$
		の正則関数$a^{\ell,m}_{\lambda,\mu,\nu}$
		を以下のように定義する。\footnote{Is there a particular reason we use subscript to indicate dependence on $\lambda,\mu,\nu$ in $a^{\ell,m}_{\lambda,\mu,\nu}$,
		but use brackets to indicate dependence on $\lambda,\mu,\nu$ in $b(\lambda,\mu,\nu)$? Maybe, it is better to write $a^{\ell,m}(\lambda,\mu,\nu)$?
		}
		\begin{equation*}
		\begin{array}{rcl}
			a_{\lambda,\mu,\nu}^{\ell,m}&:=&\frac{ (\lambda + \ell) (\mu + m)}{\Gamma \left( \lambda + \nu + \frac{\ell -
			  m}{2} + 1 \right)  \Gamma \left( \mu + \nu -
			  \frac{\ell - m}{2} + 1 \right)\Gamma \left( \lambda + \mu + \nu + \frac{\ell +
			  m}{2} + 1 \right)\Gamma\left(  \nu+1-\frac{\ell+m}{2}\right)},\\[0.4cm]
			  b(\lambda,\mu,\nu)&:=&2^{-2\nu}\Gamma (\lambda + \mu + 2 \nu + 1){\Gamma (\lambda)
			  \Gamma (\mu)\Gamma \left( 2\nu +
		  1 \right)}.
		\end{array}
		\end{equation*}
	\begin{thm}\label{prop:exp-st-gg}
		$\Re\lambda,\Re\mu>-\frac{1}{2},\Re\nu>0$ならば、
		\begin{equation}
			| s + t |^{2 \nu} =b(\lambda,\mu,\nu) \displaystyle\sum_{\scalebox{0.6}{ $\begin{array}[]{c}
			\ell, m = 0 \\ \ell + m : \tmop{even}
		\end{array}$}}^{\infty} a_{\lambda,\mu,\nu}^{\ell,m} C_\ell^{\lambda} (s) C_m^{\mu} (t)
			\label{eqn:exp-st-gg}
		\end{equation}
		が成り立つ。
	\end{thm}
	第\ref{sec:proof}節
	で、定理\ref{prop:exp-st-gg}の4通りの証明を紹介する。

	少しだけ一般的な結果も述べておこう。
	\begin{prop}\label{prop:exp-stz-gg}
		$\ell,m\in\N$に対して
		    \begin{equation*}
			    A^{\ell,m}_{\lambda,\mu,\nu} (x) \assign
			    \frac{\Gamma \left( \nu + \frac{1}{2} \right) \Gamma
				  (\lambda) \Gamma (\mu) (\lambda + l) x^m }{(1 + \nu)_{- \frac{\ell + m}{2}} \sqrt{\pi} \Gamma
				      (\mu + m) \Gamma \left( \lambda + \nu + \frac{\ell - m}{2} + 1 \right)}
				    {}_2 F_1 \left( \begin{array}{c}
				\frac{\ell + m}{2} - \nu, \frac{m - \ell}{2} - \nu - \lambda\\
				\mu + m + 1
			\end{array} ; x^2 \right)
		\end{equation*}
		とおくと、
		  $\tmop{Re} \lambda, \tmop{Re} \mu > - \frac{1}{2}$,
		    $\tmop{Re} \nu > 0$ かつ $-1 \leqslant x \leqslant 1$ のとき、
		\begin{equation*}
			       | s + t x |^{2 \nu}  = \sum_{\scalebox{0.6}{
			      $\begin{array}[]{c}
				  \ell,m=0\\\ell+ m\mbox{ :even}
				\end{array}
			$}}^{\infty} A^{\ell,m}_{\lambda,\mu,\nu}
				 (x) C_\ell^{\lambda} (s) C_m^{\mu} (t)
		    \end{equation*}
		    が成り立つ。
    \end{prop}
    ガウスの超幾何関数$_2F_1\left(\begin{array}[]{c}
	    a,b\\c
    \end{array};x\right)$の$x=1$における値
\begin{equation*}
		{}_2F_1\left( \begin{array}[]{c}
			a,b\\c
		\end{array};1\right)=\frac{\Gamma(c-a-b)\Gamma(c)}{\Gamma(c-a)\Gamma(c-b)},\quad \Re(c-a-b)>0
	\end{equation*}
	を用いると、\begin{equation*}
		A^{\ell,m}_{\lambda,\mu,\nu}(1)=b(\lambda,\mu,\nu)a^{
		\ell,m}_{\lambda,\mu,\nu}
	\end{equation*}が成り立つ。従って、
	定理\ref{prop:exp-st-gg}は命題\ref{prop:exp-stz-gg}において$x=1$を代入した特{殊}
	値であるということが分かる。

	定理\ref{prop:exp-st-gg}は次の
	積分公式と同値である。
	\begin{prop}
		\label{prop:int-st-gg}
		\begin{equation*}
			\int_{- 1}^1 \int_{- 1}^1 | s - t |^{2 \nu} (1 - s^2)^{\lambda - \frac{1}{2}}
			(1 - t^2)^{\mu - \frac{1}{2}} C_\ell^{\lambda} (s) C_m^{\mu} (t) d s d t
		\end{equation*}
		{
		\begin{equation}
			=\frac{(- \nu)_{\frac{\ell + m}{2}} (- 1)^{\frac{\ell - m}{2}} \pi^{\frac{1}{2}} (2
			\lambda)_\ell (2 \mu)_m \Gamma \left( \lambda + \frac{1}{2} \right) \Gamma \left(
			\mu + \frac{1}{2} \right) \Gamma \left( \nu + \frac{1}{2} \right) \Gamma
		(\lambda + \mu + 2 \nu + 1)}{\ell!m! \Gamma \left( \lambda + \nu + \frac{\ell -
		m}{2} + 1 \right) \Gamma \left( \mu + \nu - \frac{\ell - m}{2} + 1 \right) \Gamma
		\left( \lambda + \mu + \nu + \frac{\ell + m}{2} + 1 \right)}
			\label{eqn:int-st-gg}
		\end{equation}
		}
	\end{prop}
	ここでPochhammer記号$(x)_n$は以下のように定義される。\begin{equation*}
		(x)_n:=\frac{\Gamma(x+n)}{\Gamma(x)}=x(x+1)\cdots(x+n-1).
	\end{equation*}
%%	\begin{remark*}
%%		上記で記述された命題\ref{prop:exp-stz-gg}に対応する積分公式もあるが、
%%		ここで記述を簡単にするため、省略する。
%%	\end{remark*}
\section{主結果の様々な特殊化}
以下では定理\ref{prop:exp-st-gg}(あるいは同値なことであるが命題\ref{prop:int-st-gg})の
特殊値や極限値が、{既知}の積分公式とどのように関連しているかを説明する。
\subsection{Selberg積分の特殊値との比較}
\newcommand{\boldt}{\mathbf{t}}
$\boldt=\left( t_1,\cdots,t_k \right),\alpha,\beta,\gamma\in\C$に対して\begin{equation*}
	\Psi_k(\alpha,\beta,\gamma;\mathbf{t})\assign
\displaystyle\prod\displaylimits_{i=1}^k t_i^{\alpha-1}(1-t_i)^{\beta-1} 
		\displaystyle\prod\displaylimits_{1\le i<j\le k}\myabs{t_i-t_j}^{2\gamma}
\end{equation*}とおく。
		\normalfont
		\begin{fact}[{\cite[セルバーグ積分]{Selberg:411367}}]
				$\tmop{Re} (\alpha), \tmop{Re} (\beta) > 0, | \gamma | \ll 1$
				とすると\footnote{maybe, comma here?}
	 \begin{equation*}
		 \normalfont
		\begin{array}[]{c}
		\displaystyle\int_{t \in [0, 1]^k}  \Psi_k(\alpha,\beta,\gamma;\boldt)d\boldt=
		\displaystyle\prod_{i = 0}^{k - 1} \frac{\Gamma (\alpha + i \gamma) \Gamma (\beta + i
				\gamma) \Gamma (1 + (i + 1) \gamma)}{\Gamma (\alpha + \beta + (i + k - 1)
				\gamma) \Gamma (\gamma + 1)},
		\end{array}
			\end{equation*}
		\end{fact}
		命題\ref{prop:int-st-gg}の $\ell = m = 0,\lambda=\mu$ の特殊化は
		Selberg 積分
の $k = 2,\alpha=\beta$ の場合
と同じ結果を与える。
\\
			\begin{samepage}
				\vspace{1em}
		\centerline{
		\xymatrixcolsep{1.9cm}
		\xymatrix{
			\framebox{命題\ref{prop:int-st-gg}}
			\ar@{=>}[rd]^{\begin{array}[]{c}
				\ell=m=0,\\ \lambda=\mu
			\end{array}}&&&&\framebox{\mbox{\cite{Selberg:411367}}}
			\ar@{=>}[ld]_{\kern-1.5cm\begin{array}[]{c}
			k=2,\\
			\alpha=\beta
	\end{array}}\\
			&&&&
		}}
		\nopagebreak
		\framebox{\mbox{$
				\displaystyle\iint\displaylimits_{ [-1,1]^2} | s - t |^{2 \nu} (1 - s^2)^{\lambda - \frac{1}{2}}
			(1 - t^2)^{\lambda - \frac{1}{2}}d s d t
			=\prod_{i = 0}^{k - 1} \frac{\Gamma (\alpha + i \gamma) \Gamma (\beta + i
			\gamma) \Gamma (1 + (i + 1) \gamma)}{\Gamma (\alpha + \beta + (i + k - 1)
			\gamma) \Gamma (\gamma + 1)}.
		$}}
				\vspace{1em}
			\end{samepage}
		\subsection{Warnaarによる$\mathfrak{sl}_3$ Selberg積分の特殊値との比較}
		Selberg積分の1つの一般化であるWarnaarの$\mathfrak{sl}_3$ Selberg積分
		\cite{warnaar2010sl3}を復習しよう。
		これは5つのパラメータ$\alpha_1,\alpha_2,\beta_1,\beta_2,\gamma$を含むが、
		制\doubt{約}条件$\beta_1+\beta_2=\gamma+1$を課すので\footnote{maybe, comma here?}
		4つの自由パラメータを含む多重積分である。

		\begin{fact}[{\cite[(1.4)]{warnaar2010sl3}}]
				{
		\begin{equation*}
%%			
			\begin{array}{c}
				\displaystyle\int
				\displaylimits_{(t,s)\in C_{\beta_1,\gamma}^{k_1,k_2}}
				\Psi_{k_1}(\alpha_1,\beta_1,\gamma;\boldt)
				\Psi_{k_2}(\alpha_2,\beta_2,\gamma;\mathbf{s})
				\displaystyle\prod_{i,j=1}^{k_1,k_2}\myabs{t_i-s_j}^{-\gamma}
				d\boldt
				d\mathbf{s}
				      \end{array}\end{equation*}
		\begin{equation*}
			\begin{array}{c}
  = \displaystyle\prod_{i = 0}^{k_1 - 1} \frac{\Gamma (\alpha_1 + i \gamma) \Gamma (\beta_1
  + (i - k_2) \gamma) \Gamma ((i + 1) \gamma)}{\Gamma (\alpha_1 + \beta_1 + (i
  + k_1 - k_2 - 1) \gamma) \Gamma (\gamma)} \times\\
  \displaystyle\prod_{i = 0}^{k_2 - 1} \frac{\Gamma (\alpha_2 + i \gamma) \Gamma (\beta_2 +
  i \gamma) \Gamma ((i + 1) \gamma)}{\Gamma (\alpha_2 + \beta_2 + (i + k_2 -
  k_1 - 1) \gamma) \Gamma (\gamma)} \displaystyle\prod_{i = 0}^{k_1 - 1} \frac{\Gamma
  (\alpha_1 + \alpha_2 + (i - 1) \gamma)}{\Gamma (\alpha_1 + \alpha_2 + (i +
  k_2 - 1) \gamma)},\\
  \mbox{ここで  }
  C^{k_1,k_2}_{\beta_1,\gamma}\mbox{: ある特異チェイン, }
					  \beta_1 + \beta_2 = \gamma + 1,\\
					    \forall i:
					    \myre{\alpha_i}, \myre{\beta_i} > 0,
					    \myabs{\gamma} \ll 1 ,\\ \quad 1
						\le \forall i \le \displaystyle\min_j \{ k_j \} : \beta_1 + (i - k_2 - 1)
						  \gamma \nin \mathbb{Z}.
			\end{array}
			\end{equation*}
				}
			\end{fact}
命題\ref{prop:int-st-gg}の $\ell = m = 0,\lambda+\mu+2\nu=-1$ の特殊化は
\cite[(1.4)]{warnaar2010sl3}\footnote{
	  should I write the definition $C^{k_1,k_2}_{\beta_1,\gamma}$ (and $C^{k_1,k_2}_{\gamma}$ below) explicitly?
  }
  の $k_1=k_2 = 2,\alpha_1=\beta_1,\alpha_2=\beta_2$ の特別な場合と同じ結{果}を与える。\\
  \myrelationdiagramBig{命題\ref{prop:int-st-gg}}{\ell=m=0,\\
		\lambda+\mu+2\nu=-1}{k_1=k_2=1,\\\alpha_1=\beta_1,\\\alpha_2=\beta_2}{{\cite{warnaar2010sl3}}}{
		$
				\left(\displaystyle\iint\displaylimits_{\scalebox{0.7}{$\begin{array}[]{c}
					[0,1]^2\\t<s
				\end{array}$}}+\frac{\sin(\pi\alpha_1)}{\sin(\pi\alpha_2)}\displaystyle\iint\displaylimits_{\scalebox{0.7}{$\begin{array}[]{c}
					[0,1]^2\\t>s
				\end{array}$}} \right)
			(t(1-t))^{\mu - \frac{1}{2}}  (s(1-s))^{\lambda - \frac{1}{2}}  | t - s |^{2\nu} d s d t
			=\mypgf
			$
		}
		\subsection{TarasovとVarchenkoによるSelberg積分の一般化と定理\ref{prop:exp-st-gg}
	の比較}
	次にTarasovとVarchenkoによる$\mathfrak{sl}_3$に付随したSelberg積分の一般化との比較を行う。
	これは4つの自由パラメータ
	$\alpha,\beta_1,\beta_2,\gamma$を含む多重積分の公式である。
	\newcommand{\bolds}{\mathbf{s}}
			\begin{fact}[{\cite[(3.4)]{tarasov2003selberg}}]
			{
		\begin{equation*}
			\begin{array}{c}
				\displaystyle\int_{(t,s)\in C_\gamma^{k_1,k_2}}
				\Psi_{k_1}(\alpha,\beta_1,\gamma;\boldt)\Psi_{k_2}(1,\beta_2,\gamma
				;\bolds)
				\displaystyle\prod_{i,j=1}^{k_1,k_2}\myabs{t_i-s_j}^{-\gamma}
				d\boldt d\bolds
				      \end{array}\end{equation*}
			      \begin{equation*}
			\begin{array}{c}
  =\displaystyle\prod_{j = 0}^{k_1 - 1} \frac{\Gamma (\alpha + j \gamma) \Gamma (\gamma + j
  \gamma)}{\Gamma (\gamma)} \displaystyle\prod_{j = 0}^{k_1 - k_2 - 1} \frac{\Gamma
  (\beta_1 + j \gamma)}{\Gamma (\alpha + \beta_1 + (2 k_1 - k_2 - 2 - j)
  \gamma)} \times\\
  \displaystyle\prod_{j = 0}^{k_2 - 1} \frac{\Gamma (\beta_2 + j \gamma) \Gamma (\beta_1 +
  \beta_2 - \gamma + j \gamma) \Gamma (1 - k_1 \gamma + j \gamma) \Gamma
  (\gamma + j \gamma)}{\Gamma (\beta_2 + 1 + (2 k_2 - k_1 - 2 - j) \gamma)
  \Gamma (\alpha + \beta_1 + \beta_2 + (k_1 + k_2 - 3 - j) \gamma) \Gamma
  (\gamma)},\\
			  \mbox{ここで、  }	C^{k_1,k_2}_{\gamma}\mbox{: ある特異チェイン},\\
			  \myre{\alpha_i}, \myre{\beta_2} > 0, | \tmop{Re} \gamma | \ll 1.
			\end{array}
			\end{equation*}
				}
		\end{fact}
		命題\ref{prop:int-st-gg}の $\ell = m = 0,\mu=\frac{1}{2}$ の特殊化が
		\cite[(3.4)]{tarasov2003selberg}
の $k = 2,\beta_2=1,\alpha=\beta_1$ の特別な場合になる:\\
		{
			\myrelationdiagramBigSix{命題\ref{prop:int-st-gg}}{\ell=m=0,\\
			\mu=\frac{1}{2}}
			{
				k=2,\\
				\beta_2=1,\\
				\alpha=\beta_1
			}
		{{\cite{tarasov2003selberg}}}
		{\kern2cm$
			\displaystyle\iint\displaylimits_{\scalebox{0.7}{$\begin{array}[]{c}
				(s, t) \in [0, 1]^2\\t<s
			\end{array}$}} s^{\lambda - \frac{1}{2}} (1 - s)^{\lambda-\frac{1}{2}} \myabs{s-t}^{2\nu} d s d t
			=\mypgf
		$\kern2cm}{1.9}
		\subsection{DotsenkoとFateevによる積分公式と定理\ref{prop:exp-st-gg}との比較}
		DotsenkoとFateev \cite{dotsenko1985four}は
		3つの関係式\eqref{eqn:DF3}を満たす
		6つのパラメータ$\alpha,\alpha',\beta,
			\beta',\rho,\rho'$を含む以下の多重積分の具体的表示を与えた\footnote{
			maybe, we should put 。 here to indicate end of a sentence?}
		\begin{fact}[{\cite[(A.35)]{dotsenko1985four}}]
			{
		\begin{equation*}
			\begin{array}{c}
				\displaystyle\displaystyle\int_{(t,\tau)\in [0, 1]^{k_1+k_2}} 
				\Psi_{k_1}(\alpha'+1,\beta'+1,\rho';\boldt)\Psi_{k_2}(\alpha+1,\beta+1
				,\rho;\bolds)
				\displaystyle\prod_{i,j=1}^{k_1,k_2}\myabs{t_i-\tau_j}^{-2}
				dtd\tau\mbox{\footnotemark} \\
				      \end{array}\end{equation*}
			
			      \begin{equation*}\footnotetext{which variables should
		      we use in the integral: $(\boldt,\bolds)$ or $(t,\tau)$?}
			\begin{array}{c}
  =n!m!\rho^{2 n m} \displaystyle\prod_{i, j = 1}^{n, m} \frac{1}{j \rho - i} \displaystyle\prod_{i = 1}^n
  \frac{\Gamma (i \rho')}{\Gamma (\rho')} \displaystyle\prod_{j = 1}^m \frac{\Gamma (j
  \rho)}{\Gamma (\rho)} \times\\
  \displaystyle\prod_{i, j = 0}^{n, m} \frac{1}{(\alpha + j \rho - i) (\beta + j \rho - i)
  (\alpha + \beta + \rho (m - 1 + j) - (n - 1 + i))} \times\\
  \displaystyle\prod_{i = 0}^{n - 1} \frac{\Gamma (1 + \alpha' + i \rho') \Gamma (1 +
  \beta' + i \rho')}{\Gamma (2 - 2 m + \alpha' + \beta' + (n - 1 + i) \rho')}
  \displaystyle\prod_{j = 0}^{m - 1} \frac{\Gamma (1 + \alpha + j \rho) \Gamma (1 + \beta +
  j \rho)}{\Gamma (2 - 2 n + \alpha + \beta + (m - 1 + j) \rho)}\end{array}\end{equation*}
					  ここで、
					  \begin{equation}
	 \alpha' = - \rho' \alpha, \beta' = - \rho' \beta, \rho' \rho = 1,
						  \label{eqn:DF3}
					  \end{equation}\begin{equation*}
  \myre{\rho} < 0, \quad \myre{\alpha}, \myre{\beta} > (n - 1)
  + | \myre{\rho} | (m - 1).
					  \end{equation*}
				}
		\end{fact}

命題\ref{prop:int-st-gg}の $\ell = m = 0,\nu=-1$ の特殊化が
\cite[(A.35)]{dotsenko1985four}
の $m=n=1,\alpha'=\beta',\alpha=\beta$ の特別な場合になる:\\
\myrelationdiagramBigCenter{命題\ref{prop:int-st-gg}}{\ell=m=0,\\
		\nu=-1}
			{
				m=n=1,\\
				\alpha'=\beta',\\
				\alpha=\beta
			}
		{{\cite{dotsenko1985four}}}{
				$\displaystyle\iint\displaylimits_{(t, \tau) \in [0, 1]^2} t^{\alpha'} (1 - t)^{\alpha'} \tau^{\alpha} (1
				- \tau)^{\alpha} | t - \tau |^{- 2} dtd\tau
			=\mypgf$
		}

		\subsection{特殊化とhierarchy}
このように、
	命題\ref{prop:int-st-gg}の$\ell = m = 0$に対する積分公式
	はWarnaar、Varchenko、Tarasov などによるSelberg型の積分の一般化の特別な場合とも関係する。
	以上を図1にまとめる。
	{青}い{数字}は{公式}に{含}まれる連続パラメータの{個数}である。
			\begin{figure*}[h]
				\centering
				\begin{tikzpicture}
				\draw[color=black] (0.0,0.0) rectangle (2.0,-0.5);
\node at (1.0,-0.25) {\color{black}{\scriptsize Warnaar: \color{blue}{4}}};
\draw[color=red] (2.1,0.0) rectangle (4.3,-0.5);
\node at (3.2,-0.25) {\color{red}{\scriptsize 命題 \ref{prop:exp-stz-gg}: \color{blue}{4}}};
\draw[color=black] (5.7,0.0) rectangle (8.9,-0.5);
\node at (7.300000000000001,-0.25) {\color{black}{\scriptsize Tarasov-Varchenko: \color{blue}{4}}};
\draw[color=black] (9.1,0.0) rectangle (12.1,-0.5);
\node at (10.6,-0.25) {\color{black}{\scriptsize Dotsenko-Fateev: \color{blue}{3}}};
\draw[color=red] (2.1,-2.0) rectangle (4.3,-2.5);
\node at (3.2,-2.25) {\color{red}{\scriptsize 命題 \ref{prop:int-st-gg}: \color{blue}{3}}};
\draw[color=black] (5.0,-2.0) rectangle (6.7,-2.5);
\node at (5.85,-2.25) {\color{black}{\scriptsize Selberg: \color{blue}{3}}};
%\draw[color=black] (3.2,-4.0) circle(0.3);
\node at (3.85,-4.0) {\color{blue}{3}};
\fill[color=black] (1.5,-6.0) circle(0.1);
\node at (1.95,-6.0) {\color{blue}{2}};
\fill[color=black] (3.2,-6.0) circle(0.1);
\node at (3.6500000000000004,-6.0) {\color{blue}{2}};
\fill[color=black] (5.85,-6.0) circle(0.1);%TV
\node at (6.3,-6.0) {\color{blue}{2}};
\fill[color=black] (8.9,-6.0) circle(0.1);%DF
\node at (9.35,-6.0) {\color{blue}{2}};

\draw[->,>=angle 90,color=black] (2.0,-0.5) -- node {} (5.0,-2.0) ;
\draw[->,>=angle 90,color=red] (3.2,-0.5) -- node {} (3.2,-2.0) ;
\draw[->,>=angle 90,color=black] (6.18,-0.5) -- node {} (5.85,-2.0) ;
\draw[->,>=angle 90,color=black] (9.1,-0.5) -- node {} (6.615,-2.0) ;
\draw[->,>=angle 90,color=red] (3.2,-2.5) -- node {\color{black}{\scriptsize $\kern1.5cm\ell=m=0$}} (3.2,-3.6) ;
\draw[->,>=angle 90,color=black] (1.0,-0.5) -- node {} (1.490946425395748,-5.90041067935323) ;
\draw[->,>=angle 90,color=black] (3.00,-4.23) -- node {} (1.56,-5.93) ;%L->W'
\draw[->,>=angle 90,color=black] (3.2,-4.3) -- node {} (3.2,-5.915) ;%L->S'
\draw[->,>=angle 90,color=black] (3.44,-4.18) -- node {} (5.78,-5.95) ;
\draw[->,>=angle 90,color=black] (3.48,-4.1) -- node {} (8.8,-5.97) ;
\draw[->,>=angle 90,color=black] (5.34,-2.5) -- node {} (3.252164719493017,-5.914683869988056) ;
\draw[->,>=angle 90,color=black] (10.3,-0.5) -- node {}  (8.92466,-5.90309);%DF->DF'
\draw[->,>=angle 90,color=black] (7.62,-0.5) -- node {} (5.88,-5.9048) ;%TV->TV'

\node[draw,black,fill=white,circle,minimum size=0.3cm,inner sep=0pt] at (3.2,-4.0) {\eqref{eqn:lmzero}};

				\end{tikzpicture}
				\caption{
					命題\ref{prop:int-st-gg}の$\ell=m=0$の特別の場合とその関連結果;
\mykana{青}{アオ}い\mykana{数字}{スウジ}は\mykana{公式}{コウシキ}に\mykana{含}{フク}まれる連続パラメータの\mykana{個数}{コスウ}である.
				}
				\label{fig:intdep}
			\end{figure*}

		\subsection{定理\ref{prop:exp-st-gg}の極限値}
最後に、命題\ref{prop:int-st-gg}のある種の極限値と比較しよう。このために、
エルミート多項式を復習する。
エルミート多項式$\left\{ H_n(x) \right\}_{n=1}^\infty$は、
	$L^2\left( \mathbb{R},e^{-\frac{x^2}{2}}dx \right)$の直交多項式であり、
	Gegenbauer多項式の極限として再現できる:
	\begin{equation*}
			H_n (x) = n! \lim_{\lambda \rightarrow \infty} \lambda^{- \frac{n}{2}}
			C_n^{\lambda} \left( \frac{x}{\sqrt{\lambda}} \right).
	\end{equation*}
	最初の数項を挙げよう。
		\begin{eqnarray*}
		H_0(x)&=& 1,\\
		H_1(x)&=& 2x,\\
		H_2(x)&=& 
		4x^2-2,\\
		H_3(x)&=& 8x^3-12x,\\
		H_4(x)&=& 16x^4-48x^2+12.\\
		\end{eqnarray*}
	命題 \ref{prop:int-st-gg}で$\mu/\lambda$を一定にした上で$\lambda,\mu\to\infty$ 
	という極限をとれば、Gegenbauer多項式に関する積分公式(命題\ref{prop:int-st-gg})から
	エルミート多項式に関する以下の積分公式を得る。

			\begin{cor}\label{cor:int-xzy-hh}
		$\nu\in\mathbb{C}$と$w\in\R$に対して、以下の公式が成り立つ:
		\begin{multline}
			\int_{- \infty}^{\infty} \int_{- \infty}^{\infty} | x - w y |^{2 \nu} e^{-
			x^2 - y^2} H_\ell (x) H_m (y) d x d y \\= (- \nu)_{\frac{\ell + m}{2}} (- 1)^{\frac{\ell
			- m}{2}} 2^{\ell + m} \pi^{\frac{1}{2}} \Gamma \left( \frac{1}{2} + \nu \right)
			(w^2 + 1)^{\nu - \frac{\ell + m}{2}} w^m .
		\end{multline}
	\end{cor}
	系\ref{cor:int-xzy-hh}の特殊値をMehta積分と比較しよう。
	\begin{fact}[Mehta積分, {\cite{mehta2004random}}]
			{
		\begin{equation*}
			\begin{array}[]{c}
			{(2\pi)^{-\frac{k}{2}}}\displaystyle\int_{\mathbb{R}^k}\prod_{1\le i<j\le k}\myabs{t_i-t_j}^{2\gamma}\exp\left( -\myabs{\mathbf{t}}^2/2 \right)d\mathbf{t}
			=\displaystyle\prod_{j=1}^k\frac{\Gamma\left( 
			1+j\gamma\right)}{\Gamma(1+\gamma)}.
			\end{array}
		\end{equation*}
	}
		\end{fact}

		系\ref{cor:int-xzy-hh}の $w = 1, \ell = m = 0$ の特殊化がMehta積分
の $k = 2$ の特別な場合になる:
\vspace{1em}
		\myrelationdiagram{系\ref{cor:int-xzy-hh}}{w=1\\\ell=m=0}{k=2}{Mehta積分}
		{$\int_{-\infty}^\infty\int_{-\infty}^\infty\myabs{x-y}^{2\nu}e^{-x^2-y^2}dxdy=\mypgf$}

	更に、系\ref{cor:int-xzy-hh}の積分に$w=s-t$変数変化し、$z=1$をとれば、
	次の公式と同等な公式(正確には、そのMellin変換)が得られる.
	\begin{fact}[{\cite[(18)]{conte1994hermite}}]
		以下のような公式が成り立つ。
		\begin{equation}
			w_{\lambda,n}\ast w_{\lambda,k}(t)=\sqrt{\frac{(n+k)!}{2^{n+k}n!k!}}w_{\sqrt{2}\lambda,n+k}(t).
			\label{eqn:mellin-hh}
		\end{equation}
	\end{fact}
	ここで、
	記号$w_{\lambda,n}$は\underline{Hermite-Rodriguez}\footnote{
		should I use \underline{underline} here, is is \textbf{boldface} better
	(I personally prefer the boldface)?} 関数 \cite{yusoff2007application}
	\begin{equation*}
		w_{\lambda,n}:=\frac{1}{\sqrt{2^nn!}}H_n\left( \frac{t}{\lambda} \right)\frac{1}{\sqrt{\pi}\lambda}e^{-t^2/\lambda^2},\quad \lambda>0
	\end{equation*}
	を表す。
	\section{定理\ref{prop:exp-st-gg}の証明について}\label{sec:proof}
	第\ref{sec:proof}節では、定理\ref{prop:exp-st-gg}の4通りの証明をあげる.
	下に述べられている証明法1と証明法3は最初のステップとして以下の補題を
	共通に用いる。
	\begin{lemma}[$\ell=m=0$場合に帰着]
		\begin{equation*}
			\begin{array}[]{c}
				\mbox{$\ell=m=0$特殊な場合に命題\ref{prop:int-st-gg}が成り立つ}\\[0.4cm]
				\Downarrow\\[0.4cm]
				\mbox{命題\ref{prop:int-st-gg}の一般な場合が成り立つ.}
			\end{array}
		\end{equation*}
	\end{lemma}
	
	\begin{proof*}{証明のスケッチ}
		Gegenbauer多項式のロドリゲスの公式
		{\begin{equation*}
				(1-x^2)^{\alpha-\frac{1}{2}}C_n^\alpha(x)=\frac{(-1)^n}{2^nn!}\frac{\Gamma\left( \alpha+\frac{1}{2} \right)\Gamma\left( n+2\alpha \right)}{\Gamma(2\alpha)\Gamma\left(\alpha+n+\frac{1}{2}  \right)}
				\frac{d^n}{dx^n} (1-x^2)^{n+\alpha-\frac{1}{2}}
			\end{equation*}}
	と部分積分を用いる。
	\end{proof*}

	\begin{proof*}{注意}
	定理\ref{prop:exp-st-gg}の一般化である
	命題\ref{prop:exp-stz-gg}の証明においても同じアイディアを適用
	することができ
	{、}$\ell=m=0$の場合にその証明を帰着することができる。
	\end{proof*}
	\begin{method}[直接]
		\quad\\
	\begin{enumerate}
		\item $\ell=m=0$ 場合を扱えばよい.つまり、以下の等式を示せばよい:
		\begin{equation*}
			\displaystyle\iint_{[-1,1]^2}
			\myabs{s-t}^{2\nu}(1-s^2)^{\lambda-\frac{1}{2}}(1-t^2)^{\mu-\frac{1}{2}}=\mypgf;
		\end{equation*}
		\item

			超幾何関数のオイラー積分表示
			\begin{equation*}
				{}_2F_1\left(\begin{array}[]{c}
					a,b\\c
				\end{array};z  \right)=\frac{\Gamma(c)}{\Gamma(b)\Gamma(c-b)}
				\displaystyle\int_0^1x^{b-1}(1-x)^{c-b-1}(1-zx)^{-a}dx,
				\quad\myre{c}>\myre{b}>0
			\end{equation*}
			を用い、次の公式に帰着する:\begin{equation*}
			\displaystyle\int_{-1}^1 (1 + t)^{\mu-\frac{1}{2}} {}_2 F_1 \left( \begin{array}{c}
				\frac{1}{2}-\lambda, \lambda+\frac{1}{2}\\
					2 \nu+\lambda+\frac{3}{2}
				\end{array} ; \frac{1 - t}{2} \right)(1-t)^{2\nu+\lambda+\mu}={\mypgf};
			\end{equation*}
		\item
			{超幾何関数${}_2F_1$の級数展開}
			を用いて、\mbox{求める積分を ${}_3F_2$ で表す}.
			よって、以下の等式を示せばよい:
		\begin{equation*}
			{}_3F_2\left( \begin{array}[]{c}
				\frac{1}{2}-\lambda, \lambda+\frac{1}{2},2\nu+\lambda+\mu+1\\
				2 \nu+\lambda+\frac{3}{2},2\nu+\lambda+2\mu+\frac{1}{2}
			\end{array};1\right)=\mypgf;
		\end{equation*}
			\item 
				Whipple Sum(下に記載されているFact $\ref{fact:whipple}$)
				によって、超幾何関数
				${}_3F_2(;1)$をガンマ関数の積で表せる.
	\end{enumerate}
\end{method}
		\begin{fact}[Whipple sum]\label{fact:whipple}
			\begin{equation*}
			\begin{array}[]{c}
			\left\{  \begin{array}[]{l}
				a+b=1,\\2c+1=d+e
			\end{array}\right.
			\mbox{ならば、}\quad
			\implies{}_{3}F_2\left( \begin{array}[]{c}
					a,b,c\\d,e
				\end{array};1\right)={\mypgf}.
			\end{array}
		\end{equation*}
				\end{fact}
				次のアプローチを示す前、Carlsonの定理\cite{carlson1960classe}を復習する:

\begin{fact}[Carlsonの定理]
	$f$が$\left\{ z\in\mathbb{C}\mid \Re(z)\ge0 \right\}$右半平面上で連続で、
	内部$\left\{ z\in\mathbb{C}\mid\Re(z)>0 \right\}$上で
	正則であるとする.
	更に、$f$が以下の3条件をみたすならば、$f\equiv0$となる。
	\begin{enumerate}
		\item ある$C,\tau>0$に対して、右半平面上に$\myabs{f(z)}\le C e^{\tau\myabs{z}}$が成り立つ;
		\item ある$ C>0,c<\pi$が存在し、
			全ての$y\in\R$に対して、$\myabs{f(iy)}\le Ce^{c\myabs{y}}$が成り立つ;
		\item $f(n)=0$が任意の$n\in\N_+$に対して成り立つ。
	\end{enumerate}
\end{fact}
%%\begin{remark*}
%%	\begin{equation*}
%%		f(z)=\sin(\pi z)
%%	\end{equation*}
%%	という関数が条件2.を除いで、Carlsonの定理の全ての条件を満たす.特に、
%%	$f\big|_{\Z}=0$が成り立つ.しかし、 $f(z)$が恒等的に0ならない.
%%
%%	よって、Carlsonの定理の条件2.が必要である.
%%\end{remark*}
\begin{method}[cf. セルバーグ積分の証明]
	上に記載したCarlsonの定理を認めれば、
	命題\ref{prop:int-st-gg}
	の積分等式は$\nu$が整数である特別な場合を示せば良いということが分かる。
この場合は、$s+t$のベキ乗が多項式になって、左辺の積分がBeta積分の有限和になるので、
計算可能である。
\end{method}
\begin{remark*}
	Atle Selbergの元々のセルバーグ積分の証明\cite{Selberg:411367}はこのような手法が用いられ{た}。
\end{remark*}
\begin{method}[命題\ref{prop:exp-stz-gg}によって]
	命題\ref{prop:exp-stz-gg}を示すため、
	まず、$\ell=m=0$場合に帰着する。すなわち、
			\begin{equation}
				2\iint\displaylimits_{[-1,1]^2} \left( s-tz \right)_+^{2c-1}  (1 - s^2)^{a - 1} (1 -
				t^2)^{b - 1} d s d t
				=
				\frac{\sqrt{\pi} \Gamma (a) \Gamma (b) \Gamma
			(c)}{\Gamma (a + c) \Gamma \left( b + \frac{1}{2} \right)} {}_2 F_1 \left(
			\begin{array}{c}
				  - c + \frac{1}{2}, - a - c + 1\\
				    b + \frac{1}{2}
			    \end{array} ; z^2 \right)
				\label{eqn:prop21}
			\end{equation}
を示せば良い。
	この公式は以下の主張から従う:
	\begin{enumerate}
		\item オイラー積分表示;
		\item 二次変換:
			\begin{equation*}
			\quad {}_2 F_1 \left( \begin{array}{c}
				  1 - a, b\\
				    2 b
			    \end{array} ; z \right) = \left( 1 - \frac{z}{2} \right)^{a - 1} {}_2 F_1 \left(
			    \begin{array}{c}
				      \frac{1 - a}{2}, \frac{2 - a}{2}\\
					b + \frac{1}{2}
				\end{array} ; \left( \frac{z}{2 - z} \right)^2 \right);
			\end{equation*}
		\item 次の補題:
	\end{enumerate}
			\begin{lemma}
			\begin{equation*}
				\sum_{i = 0}^{\infty} \frac{(a)_i (1 - a)_i}{2^i i! (d)_i} {}_2 F_1 \left(
				\begin{array}{c}
					  \frac{1 - d - i}{2}, \frac{2 - d - i}{2}\\
					    b + \frac{1}{2}
				    \end{array} ; \zeta \right) = 
				    \overbrace{\sum_{j = 0}^{\infty} \frac{(1 - d)_{2 j} \zeta^j}{2^{2 j} j! \left( b +
				    \frac{1}{2} \right)_j}}^{\scalebox{1}{$={}_2F_1(;\zeta)$}} \overbrace{{}_2 F_1 \left( \begin{array}{c}
					      a, 1 - a\\
					        d - 2 j
					\end{array} ; \frac{1}{2} \right) }^{\mbox{ \begin{tabular}[]{c}
					ガンマ関数の\\積で表せる
				\end{tabular}}}.
				\end{equation*}
			\end{lemma}
\end{method}
\begin{method}[積分公式とルジャンドル関数$ P^\alpha_\beta$を用いる]
	直接命題\ref{prop:int-st-gg}を示す:
	\begin{enumerate}
		\item まず、\cite[7.4.11]{kobayashi2011schrodinger}の公式
			{
				
			\begin{equation*}
				\int_{-s}^1(s+t)^{2\nu} {C}_m^\mu(t)(1-t^2)^{\mu-\frac{1}{2}}dt=
				\frac{\sqrt{\pi}\Gamma(2\mu+m)\Gamma(2\nu+1)}{\Gamma(\mu)2^{\mu-\frac{1}{2}}m!}
				(1-s^2)^{\nu+\frac{\mu}{2}+\frac{1}{4}}P_{\mu+m-\frac{1}{2}}^{-2\nu-\mu-\frac{1}{2}}(-s).
			\end{equation*}
		}
		を用いる(ここで、$P_{\mu+m-\frac{1}{2}}^{-2\nu-\mu-\frac{1}{2}}(-s)$
		はルジャンドルの陪関数である);
		\item 次の段階は以下の積分に帰着できる
			\begin{equation}\label{eqn:m4-1}
				\int_{-1}^1(1-s^2)^{\nu+\frac{\mu}{2}+\lambda-\frac{1}{4}}P_{\mu+m-\frac{1}{2}}^{-2\nu-\mu-\frac{1}{2}}(-s)C^{\lambda-\frac{1}{2}}_\ell(s)ds;
			\end{equation}
			
		\item 部分積分と公式
		%Use integration by parts together with the formul\ae
			\begin{equation*}
				\left(C^\lambda_\ell(s)  \right)'=C^{\lambda+1}_{\ell-1}(s),\quad\left((1-s^2)^{-a/2}P^a_b(s) \right)'
			=-(1-s^2)^{-\frac{a+1}{2}}P_b^{a+1}(s)
			\end{equation*}
			を用いて、\eqref{eqn:m4-1}を以下の公式までに簡易化する:
%%			to reduce \eqref{eqn:m4-1} to the form
			\begin{equation}\label{eqn:m4-2}
				\int_{-1}^1(1-s^2)^aP^b_c(s)ds;
			\end{equation}
			
		\item \eqref{eqn:m4-2} は \cite[L2]{kobayashi2011schrodinger}の積分公式から従う。
	\end{enumerate}
\end{method}
\section{表現論における対称性破れ作用素への応用}
前述の積分公式の動機を手短に触れよう。詳しくは\cite{kobayashi2015program,kobayashi2015symmetry}
を参照されたい。
	\begin{setting}\label{set:1}
%%ノンコンパクト群$G$とその無限次元表現$\pi$と
%%ノンコンパクト部分群$G'$とその無限次元表現$\tau$を定める:\\
		$G$を群、$G'$をその部分群とし、$\pi$、$\tau$をそれぞれ群$G$および$G'$
		の(連続な)表現とする。
	\end{setting}
	設定\ref{set:1}の下で
	$G'$は$G$の部分群なので、$\pi$を$G'$の表現としてみなせる。
	\begin{definition}
		連続な線型写像$A:\pi\to\tau$が$G'$絡作用素であるとき$A$を
		対称性破れ作用素(symmetry breaking operator)と呼ぶ。
	\end{definition}
	\centerline{
		\xymatrixcolsep{0.5pc}
		\xymatrixrowsep{1pc}
		\xymatrix{G\ar@/_1pc/[d]&&\supset&&G'\ar@/^1pc/[d]&\\
			\pi&\ar[rr]^{A}&&\tau\kern-1cm&&{\begin{array}[]{l}
		\end{array}}%
	}}
	\begin{remark}
		以下では$\pi$、$\tau$はそれぞれ群$G$、$G'$の既約表現、あるいは
		やや{緩}めて長さが有限の表現とする。また$G$および$G'$は
		(非コンパクトな)簡約リー群、$\pi$および$\tau$は無限次元の表現で
		あることを想定している。また、表現とその表現空間は同じ記号で表すことにする。
	\end{remark}
$\pi$から$\tau$への対称性破れ作用素に関して以下の問題を考えよう。
\begin{goal*}(\cite{kobayashi2015program})
		対称性破れ作用素${A:\pi\kern-0.1cm\mid_{G'}\to\tau}$
		を全て構成し、分類し、さらにその性質を詳しく調べる.
	\end{goal*}
	この問題は$(G,G')=(O(n+1,1),O(n,1))$で$\pi$、$\tau$がそれぞれ群$G$、$G'$
	の球主系列表現のとき、Kobayashi--Speh \cite{kobayashi2015symmetry}によって
	完全に解決された。その時に用いられた手法を分析する。
	\begin{enumerate}
		\item $\pi$と$\tau$は異なる
			空間に定義された
			表現であるが、例えば$(K,K')$-重複度が1の場合に(一般化した意味での)
			固有値を定義することができる;
		\item
それぞれの表現を極大コンパクト群に制限したときに
無重複であり、
$K$から$K'$への制限も無重複であるという仮定のもとでは、
Schurの補題を用いることで、
対称性破れ作用素の\mykana{対角化}{タイカクカ}を行うことができる。

図\ref{fig:1}は、上の段では、それぞれの$K$-typeが$K'$の表現として
無重複に分解されている様子を表している。
図では異なる色は互{い}に同型でない
$K'$-タイプを表す。
対称性破れ作用素
は$G'$準同型なので、特に$K'$準
同型であり{、}従って
、
上段と下段の
同じ色の間に写像が誘導される;
\begin{figure}[h]
	\centering
	\begin{tikzpicture}[rotate=-90,scale=2]
	\draw[color=black] (-9.0,0.75) rectangle (-5.0,-6.05);
\draw[color=black] (-8.75,0.25) rectangle (-5.25,-1.25);
\draw[color=black] (-8.5,0.0) rectangle (-5.5,-0.5);
\draw[color=black] (-8.75,-1.75) rectangle (-5.25,-3.75);
\draw[color=black] (-8.5,-1.825) rectangle (-5.5,-2.325);
\draw[color=black] (-8.5,-2.4) rectangle (-5.5,-2.9);
\draw[color=black] (-8.5,-2.975) rectangle (-5.5,-3.475);
\draw[color=black] (-8.75,-4.5) rectangle (-5.25,-6.0);
\node at (-7.25,-4.0) {\color{black}{\Huge \dots}};
\node at (-7.25,-4.7) {\color{black}{\Huge \dots}};
\node at (-0.25,-3.0) {\color{black}{\Huge \dots}};
\draw[color=black] (-8.5,-4.85) rectangle (-5.5,-5.35);
\draw[color=black] (-8.5,-5.475) rectangle (-5.5,-5.975);
\draw[color=black] (-2.0,0.75) rectangle (2.0,-6.05);
\draw[color=black] (-1.75,0.25) rectangle (1.75,-1.25);
\draw[color=black] (-1.75,-4.5) rectangle (1.75,-6.0);

\node at (-7.0,1.0) {\color{black}{$I(\lambda)$}};
\node at (0.0,1.0) {\color{black}{$J(\nu)$}};

\draw[-open triangle 90] (-5.0,-2.65) to node {$R_{\lambda,\nu}^X$} (-2.0,-2.65);
\draw[-open triangle 90] (-5.5,-0.25) to node {$c^{p,q}_{0,0,0}I$} (-1.75,-0.5);
\draw[-open triangle 90] (-5.5,-2.075) to node {$c^{p,q}_{0,2,0}I$} (-1.75,-0.5);
\draw[-open triangle 90] (-5.5,-5.1) to node {$c^{p,q}_{m',m,n}I$} (-1.75,-5.25);

	\end{tikzpicture}
	\caption{対称性破れ作用素と$(K,K')$タイプ}
	\label{fig:1}
\end{figure}
		\item $(K,K')$固有値がゼロになるかどうかを判定したい;
		\item $\mysbo$の積分核が別の手法で決定されたとする;
		\item 積分核を用いて、$(K,K')$固有値を積分の形で表示できる;
	\end{enumerate}
	群$(G,G')=(O(p+1,q),O(p,q))$に対する対称性破れ作用素の研究
	で命題\ref{prop:int-st-gg}
	の積分が{現}れた。命題\ref{prop:int-st-gg}では積分値
	がガンマ関数の積公式として与えられるのでゼロかどうかが完全に決定できる。

	\nocite{Selberg:411367}
	\nocite{warnaar2010sl3}
	\nocite{dotsenko1985four}
	\nocite{tarasov2003selberg}

%%	\bibliographystyle{alpha}
%%	\bibliography{intdep}
	\begin{thebibliography}{CMS94}

\bibitem[Car14]{carlson1960classe}
F.~Carlson.
\newblock Sur une classe de s\'eries de {T}aylor.
\newblock 1914.
\newblock Dissertation.

\bibitem[CMS94]{conte1994hermite}
L.~R.~L. Conte, R.~Merletti, and G.~V. Sandri.
\newblock Hermite expansions of compact support waveforms: applications to
  myoelectric signals.
\newblock {\em IEEE Trans. Biomed. Eng.}, 41(12):1147--1159, Dec 1994.
\newblock Available online at \url{http://dx.doi.org/10.1109/10.335863}.

\bibitem[DF85]{dotsenko1985four}
V.~S. Dotsenko and V.~A. Fateev.
\newblock Four-point correlation functions and the operator algebra in 2{D}
  conformal invariant theories with central charge $c\leq 1$.
\newblock {\em {N}uclear {P}hys. B}, 251:691--734, 1985.

\bibitem[Kob15]{kobayashi2015program}
T.~Kobayashi.
\newblock A program for branching problems in the representation theory of real
  reductive groups.
\newblock In {\em {\normalfont Special issue in honor of Vogan's 60th years
  birthday}}, Progr. Math. vol. 312, pages 277--322. Birkh{\"a}user, 2015.
\newblock Available online at
  \url{http://dx.doi.org/10.1007/978-3-319-23443-4_10}.

\bibitem[KM11]{kobayashi2011schrodinger}
T.~Kobayashi and G.~Mano.
\newblock {\em The Schr{\"o}dinger {M}odel for the {M}inimal {R}epresentation
  of the {I}ndefinite {O}rthogonal {G}roup $O (p, q)$}.
\newblock Memoirs of Amer. Math. Soc., vol. 213, no. 1000, 2011.
\newblock Available online at
  \url{http://dx.doi.org/10.1090/S0065-9266-2011-00592-7}.

\bibitem[KS15]{kobayashi2015symmetry}
T.~Kobayashi and B.~Speh.
\newblock {\em Symmetry {B}reaking for {R}epresentations of {R}ank {O}ne
  {O}rthogonal {G}roups}, 
\newblock {\normalfont Memoirs of the Amer. Math. Soc}, vol. \textbf{238}, 2015.
\newblock Available online at \url{http://dx.doi.org/10.1090/memo/1126}.

\bibitem[Meh04]{mehta2004random}
M.~L. Mehta.
\newblock {\em Random {M}atrices}, {\em {\normalfont Pure Appl.
  Math. vol. 142 ({A}msterdam})}.
\newblock Elsevier/{A}cademic {P}ress, {A}msterdam, 2004.

\bibitem[Sel44]{Selberg:411367}
A.~Selberg.
\newblock {Remarks on a multiple integral}.
\newblock {\em Norsk Mat. Tidsskr.}, 26:71--78, 1944.

\bibitem[TV03]{tarasov2003selberg}
V.~Tarasov and A.~Varchenko.
\newblock Selberg-type integrals associated with $\mathfrak{sl}_3$.
\newblock {\em Lett. Math. Phys.}, 65(3):173--185, 2003.
\newblock Available online at
  \url{http://dx.doi.org/10.1023/B:MATH.0000010712.67685.9d}.

\bibitem[War10]{warnaar2010sl3}
S.~O. Warnaar.
\newblock The {${\mathfrak{sl}}_3$} {S}elberg integral.
\newblock {\em Adv. Math.}, 224(2):499--524, 2010.
\newblock Available online at
  \url{http://dx.doi.org/10.1016/j.aim.2009.11.011}.

\bibitem[Yus07]{yusoff2007application}
M.~A. Yusoff.
\newblock Application of {H}ermite--{R}odriguez functions to pulse shaping
  analog filter design.
\newblock {\em World Acad. Sci., Eng. Technol.}, 36:180--183, 2007.
\newblock Available online at
  \url{http://waset.org/publications/2098/application-of-hermite-rodriguez-functions-to-pulse-shaping-analog-filter-design}.

\end{thebibliography}


\end{document}
