\documentclass[pdf,notes]{beamer}
\mode<presentation>{\usetheme[secheader]{Boadilla}}

\usepackage{mystyle}
\includecomment{versiona}
\usepackage{xeCJK}
\usepackage{lpic}
\usefonttheme{professionalfonts}
\usepackage{etoolbox}

\newcommand{\red}[1]{{\color[rgb]{0.6,0,0}#1}}
\setbeamertemplate{theorems}[numbered] 

\newcommand{\Sol}{\mathcal{S}\mbox{ol}}
\newcommand{\Ind}{\mbox{\normalfont Ind}}
\newcommand{\Hom}{\mbox{\normalfont Hom}}
\newcommand{\D}{\mathcal{D}}
\newcommand{\A}{\mathcal{A}}
\newcommand{\Co}{\mathbb{C}}
\newcommand{\X}{\mathbb{X}}
\renewcommand{\setminus}{\backslash}
\newcommand{\nin}{\not\in}
\newcommand{\tmop}[1]{\ensuremath{\operatorname{#1}}}
\newcommand{\tmtextbf}[1]{{\bfseries{#1}}}
\newcommand{\tmtextit}[1]{{\itshape{#1}}}
\newcommand{\mss}{//}
\newcommand{\mbb}{\backslash\backslash}
\newcommand{\mmm}{\mid\mid}
\catcode`\<=\active \def<{
\fontencoding{T1}\selectfont\symbol{60}\fontencoding{\encodingdefault}}
\catcode`\>=\active \def>{
\fontencoding{T1}\selectfont\symbol{62}\fontencoding{\encodingdefault}}
\newcommand{\assign}{:=}
\newcommand{\comma}{{,}}
\newcommand{\um}{-}
\newcommand{\sol}{\mathcal{S}ol(\R^{p,q};\lambda,\nu)}
\newcommand{\Op}{\mbox{\normalfont Op}}
\newcommand{\Res}{\operatorname{Res}\displaylimits}
\newcommand{\OpR}{\mbox{\it R}}

%%\makeatletter
%%\newenvironment<>{proofs}[1][\proofname]{\par\def\insertproofname{#1\@addpunct{.}}\usebeamertemplate{proof begin}#2}
%%{\usebeamertemplate{proof end}}
%%\makeatother

\newtheorem{prop}{命題}
\newtheorem{remark}{注}

\AtBeginEnvironment{remark}{%
	    \setbeamercolor{block title}{bg=orange,fg=white}
		\setbeamercolor{block body}{bg=orange!20,fg=black}
}

\setCJKmainfont[AutoFakeBold=true]{Hiragino Mincho Pro} %my Mac

\title{2つのゲーゲンバウアー多項式に関連する積分公式について\footnote{60 min talk}}

% A subtitle is optional and this may be deleted

\author[レオンチエフ・アレックス]{小林俊行\inst{1} \and \underline{レオンチエフ・アレックス}\inst{2}}

\institute[東大] % (optional, but mostly needed)
{
  \inst{1}%
  大学院数理科学研究科、カブリ数物連携宇宙研究機構\\
  東京大学
  \and
  \inst{2}%
  大学院数理科学研究科\\
  東京大学
  }

  \date[表現論とその周辺分野の広がり]{RIMS共同研究(公開型)\\「表現論とその周辺分野の広がり」\\京都大学数理解析研究所}
% - Either use conference name or its abbreviation.
% - Not really informative to the audience, more for people (including
%   yourself) who are reading the slides online

\subject{表現論}

\begin{document}
\begin{frame}\titlepage\end{frame}

\begin{frame}{Outline}
	\tableofcontents
\end{frame}
\section{The integral formula related to two Gegenbauer polynomials}
\begin{frame}
	\begin{figure}[h]
		\centering
\begin{lpic}[]{wanted(0.75)}
	\lbl[bl]{3,15; \large$\begin{array}[]{c}
		\mbox{\it Expansion of $| s + t |^{2 \nu}$}\\\\\mbox{ in the form} \\\\
	| s + t |^{2 \nu} =	\\\\
	\sum_{\mbox{\scriptsize $\begin{array}[]{c}
\ell, m = 0 \\	 l + m\in2\N
\end{array}$}}^{\infty}
		a_{\lambda, \mu, \nu}^{\ell, m} C_{\ell}^{\lambda} (s) C_m^{\mu} (t)
	\end{array}$} %dot 
		
	\end{lpic}
		\label{fig:wanted}
	\vspace{-0.5cm}
		\caption*{Can we find the \underline{closed expression} for $a_{\lambda,\mu,\nu}^{\ell,m}$???}
	\end{figure}
\end{frame}
\begin{frame}
	\begin{prop}
		\begin{equation}
			| s + t |^{2 \nu} = \sum_{\mbox{\scriptsize $\begin{array}[]{c}
			\ell, m = 0 \\ \ell + m : \tmop{even}
		\end{array}$}}^{\infty} a_{\lambda,\mu,\nu}^{\ell,m} C_\ell^{\lambda} (s) C_m^{\mu} (t),
			\label{eqn:exp-st-gg}
		\end{equation}
		{\scriptsize
		\begin{equation*}
	a_{\lambda,\mu,\nu}^{\ell,m}= \frac{ 2^{-2\nu}(\lambda + \ell) (\mu + m)  \Gamma (\lambda + \mu + 2 \nu + 1) \Gamma (\lambda)
  \Gamma (\mu)\Gamma \left( 2\nu +
1 \right)}{\Gamma \left( \lambda + \nu + \frac{\ell -
  m}{2} + 1 \right)  \Gamma \left( \mu + \nu -
  \frac{\ell - m}{2} + 1 \right) \Gamma \left( \lambda + \mu + \nu + \frac{\ell +
  m}{2} + 1 \right)\Gamma\left(  \nu+1-\frac{\ell+m}{2}\right)}
		\end{equation*}
	}
	\end{prop}
	Here, $C_{\ell}^\lambda(t)$ ($\lambda\in\mathbb{C},\ell\in\N$) is the Gegenbauer polynomial.
	\begin{remark}
		We could not find this formula in the literature.
	\end{remark}
\end{frame}
\begin{frame}
	In fact, we can do more:
	\begin{prop}
		
	\end{prop}<++>
\end{frame}<++>
\section{Original motivation}
\begin{frame}
	
\end{frame}
\section{What does it have to do with?}
\end{document}


