\documentclass[pdf,notes]{beamer}
\mode<presentation>{\usetheme[secheader]{Boadilla}}

\usepackage{tcolorbox}
\tcbuselibrary{theorems}
\usepackage{bigints}
%\usepackage[showboxes]{textpos}
\usepackage{textpos}
\usepackage{color,soul}
\usepackage{wrapfig}
\usepackage{mathtools}
\usepackage{xypic}
\usepackage[all,cmtip]{xy}
\usepackage{mystyle}
\excludecomment{versiona}
\usepackage{xeCJK}
\usepackage{lpic}
\usefonttheme{professionalfonts}
\usepackage{etoolbox}
\usepackage{comment}

    \setul{0.5ex}{0.3ex}
	\definecolor{Blue}{rgb}{0,0.0,1}
	\definecolor{Red}{rgb}{1,0.0,0}
    \setulcolor{Blue}

\newcommand{\red}[1]{{\color[rgb]{0.6,0,0}#1}}
\setbeamertemplate{theorems}[numbered] 

\usepackage{tikz}
\usetikzlibrary{shapes,arrows,patterns}
\newcommand*\mytextcircled[1]{\tikz[baseline=(char.base)]{
	\node[shape=circle,draw,inner sep=1.8pt] (char) {\scalebox{0.8}{\kern-0.05cm #1}};}}
\newcommand{\mybigplus}{\scalebox{2.0}{$+$}}
\renewcommand{\implies}{\Rightarrow}
\newcommand{\mypgf}{{\mbox{\textbf{ガンマ関数の積}}}}
\newcommand{\Sol}{\mathcal{S}\mbox{ol}}
\newcommand{\Ind}{\mbox{\normalfont Ind}}
\newcommand{\Hom}{\mbox{\normalfont Hom}}
\newcommand{\D}{\mathcal{D}}
\newcommand{\A}{\mathcal{A}}
\newcommand{\Co}{\mathbb{C}}
\newcommand{\X}{\mathbb{X}}
\renewcommand{\setminus}{\backslash}
\newcommand{\nin}{\not\in}
\newcommand{\tmop}[1]{\ensuremath{\operatorname{#1}}}
\newcommand{\tmtextbf}[1]{{\bfseries{#1}}}
\newcommand{\tmtextit}[1]{{\itshape{#1}}}
\newcommand{\mss}{//}
\newcommand{\mbb}{\backslash\backslash}
\newcommand{\mmm}{\mid\mid}
\catcode`\<=\active \def<{
\fontencoding{T1}\selectfont\symbol{60}\fontencoding{\encodingdefault}}
\catcode`\>=\active \def>{
\fontencoding{T1}\selectfont\symbol{62}\fontencoding{\encodingdefault}}
\newcommand{\assign}{:=}
\newcommand{\comma}{{,}}
\newcommand{\um}{-}
\newcommand{\sol}{\mathcal{S}ol(\R^{p,q};\lambda,\nu)}
\newcommand{\Op}{\mbox{\normalfont Op}}
\newcommand{\Res}{\operatorname{Res}\displaylimits}
\newcommand{\OpR}{\mbox{\it R}}

%%\makeatletter
%%\newenvironment<>{proofs}[1][\proofname]{\par\def\insertproofname{#1\@addpunct{.}}\usebeamertemplate{proof begin}#2}
%%{\usebeamertemplate{proof end}}
%%\makeatother

\newtheorem*{fact*}{Fact}
\newtheorem{prop}{命題}
\newtheorem{taggedpropx}{命題}
\newtheorem{remark}{注}
\newtheorem{cor}{系}
%%\newtheorem*{proof}{証明}
\newtheorem*{example*}{例}
\newtheorem*{lemma*}{補題}
\newenvironment{taggedprop}[1]
 {\renewcommand\thetaggedpropx{#1}\taggedpropx}
  {\endtaggedpropx}

\AtBeginEnvironment{remark}{%
	    \setbeamercolor{block title}{bg=orange,fg=white}
		\setbeamercolor{block body}{bg=orange!20,fg=black}
}
\AtBeginEnvironment{fact}{%
	    \setbeamercolor{block title}{bg=green,fg=black}
		\setbeamercolor{block body}{bg=orange!20,fg=black}
}
\AtBeginEnvironment{cor}{%
	    \setbeamercolor{block title}{bg=blue,fg=white}
		\setbeamercolor{block body}{bg=orange!20,fg=black}
}

\setCJKmainfont[AutoFakeBold=true]{Hiragino Mincho Pro} %my Mac

\title[2つのゲーゲンバウアー多項式に\dots]{2つのゲーゲンバウアー多項式に関連する積分公式について}

% A subtitle is optional and this may be deleted

\author[レオンチエフ・アレックス]{小林俊行\inst{1} \and \underline{レオンチエフ・アレックス}\inst{2}}

\institute[東大] % (optional, but mostly needed)
{
  \inst{1}%
  大学院数理科学研究科、カブリ数物連携宇宙研究機構\\
  東京大学
  \and
  \inst{2}%
  大学院数理科学研究科\\
  東京大学
  }

  \date[表現論とその周辺分野\dots]{RIMS共同研究(公開型)\\「表現論とその周辺分野の広がり」\\京都大学数理解析研究所}
% - Either use conference name or its abbreviation.
% - Not really informative to the audience, more for people (including
%   yourself) who are reading the slides online

\subject{表現論}

\begin{document}
\begin{frame}\titlepage\end{frame}

\begin{frame}{Outline}
	\tableofcontents
\end{frame}
\section{主結果}
\begin{frame}{Gegenbauer多項式}%gegenbauer
	Gegenbauer多項式$\left\{ C_n^\lambda \right\}_{n=1}^{\infty}$は、
	$\Re(\lambda)>-\frac{1}{2}$のとき$L^2\left( [-1,1],(1-x^2)^{\lambda-\frac{1}{2}}dx \right)$の直交多項式である.
	\vspace{-0.2cm}
	\begin{example*}
		最初のいくつは
	\vspace{-0.3cm}
		\begin{eqnarray*}
	\vspace{-0.2cm}
			C_0^\lambda(x)&=&1,\\
			C_1^\lambda(x)&=&2\lambda x,\\
			C_2^\lambda(x)&=&-\lambda+2\lambda(1+\lambda)x^2,\\
			C_3^\lambda(x)&=&-2\lambda(1+\lambda)x+\frac{4}{3}\lambda(1+\lambda)(2+\lambda)x^3,\\
			C_4^\lambda(x)&=&\left(\frac{4+\lambda^3}{3}+4\lambda^2+\frac{8\lambda}{3}  \right)x^3+(-2\lambda^2-2\lambda)x.
		\end{eqnarray*}
	\end{example*}
	Gegenbauer多項式は以下のような微分方程式を満たす:
		\vspace{-0.2cm}
	\begin{equation*}
		(1-x^2)y''-(2\lambda+1)xy'+n(n+\lambda)y=0.
	\end{equation*}
\end{frame}
\begin{frame}
	\begin{prop}\label{prop:exp-st-gg}
		$\Re\lambda,\Re\mu>-\frac{1}{2},\Re\nu>0$のとき、
		\begin{equation}
			| s + t |^{2 \nu} =b(\lambda,\mu,\nu) \sum_{\mbox{\scriptsize $\begin{array}[]{c}
			\ell, m = 0 \\ \ell + m : \tmop{even}
		\end{array}$}}^{\infty} a_{\lambda,\mu,\nu}^{\ell,m} C_\ell^{\lambda} (s) C_m^{\mu} (t),
			\label{eqn:exp-st-gg}
		\end{equation}
		{\scriptsize
			\begin{equation*}
				\begin{array}{rcl}
	a_{\lambda,\mu,\nu}^{\ell,m}&:=&\frac{ (\lambda + \ell) (\mu + m)   }{\Gamma \left( \lambda + \nu + \frac{\ell -
	  m}{2} + 1 \right)  \Gamma \left( \mu + \nu -
	  \frac{\ell - m}{2} + 1 \right)\Gamma \left( \lambda + \mu + \nu + \frac{\ell +
	  m}{2} + 1 \right)\Gamma\left(  \nu+1-\frac{\ell+m}{2}\right)},\\[0.4cm]
	  b(\lambda,\mu,\nu)&:=&2^{-2\nu}\Gamma (\lambda + \mu + 2 \nu + 1){\Gamma (\lambda)
	  \Gamma (\mu)\Gamma \left( 2\nu +
	1 \right)}
			\end{array}
			\end{equation*}
	}
	\end{prop}
	{\bf Plan:} これの4通りの証明をあげる.
%%	Here, $C_{\ell}^\lambda(t)$ ($\lambda\in\mathbb{C},\ell\in\N$) is the Gegenbauer polynomial.
%%	\begin{remark}
%%		We could not find this formula in the literature.
%%	\end{remark}
\end{frame}
%%\begin{frame}
%%	\begin{figure}[h]
%%		\centering
%%\begin{lpic}[]{wanted(0.75)}
%%	\lbl[bl]{3,15; \large$\begin{array}[]{c}
%%		\mbox{\it Expansion of $| s + t |^{2 \nu}$}\\\\\mbox{ in the form} \\\\
%%	| s + t |^{2 \nu} =	\\\\
%%	\sum_{\mbox{\scriptsize $\begin{array}[]{c}
%%\ell, m = 0 \\	 l + m\in2\N
%%\end{array}$}}^{\infty}
%%		a_{\lambda, \mu, \nu}^{\ell, m} C_{\ell}^{\lambda} (s) C_m^{\mu} (t)
%%	\end{array}$} %dot 
%%		
%%	\end{lpic}
%%		\label{fig:wanted}
%%	\vspace{-0.5cm}
%%	\caption*{Can we find the {closed expression} for $a_{\lambda,\mu,\nu}^{\ell,m}$ \underline{in literature}???}
%%	\end{figure}
%%\end{frame}
\begin{frame}
	実は、もっと一般的な主張が成り立つ:
	\begin{prop}\label{prop:exp-stz-gg}
		  \label{thm:4}$\tmop{Re} \lambda, \tmop{Re} \mu > - \frac{1}{2}$,
		    $\tmop{Re} \nu > 0$ と $-1 \leqslant z \leqslant 1$ のとき、
		      \begin{eqnarray}
			      & | s + t z |^{2 \nu}  = \sum_{\scalebox{0.6}{
				      $\begin{array}[]{c}
						  \ell,m=0\\\ell+ m\mbox{ :even}
					  \end{array}
				  $}}^{\infty} a_{\ell, m}
					          (z) C_\ell^{\lambda} (s) C_m^{\mu} (t), &  \nonumber\\
						      & a_{\ell, m} (z) = \frac{\Gamma \left( \nu + \frac{1}{2} \right) \Gamma
						      (\lambda) \Gamma (\mu) (\lambda + l) z^m }{(1 + \nu)_{- \frac{\ell + m}{2}} \sqrt{\pi} \Gamma
										      (\mu + m) \Gamma \left( \lambda + \nu + \frac{\ell - m}{2} + 1 \right)}&\nonumber
										      \\&\times{}_2 F_1 \left( \begin{array}{c}
								        \frac{\ell + m}{2} - \nu, \frac{m - \ell}{2} - \nu - \lambda\\
									      \mu + m + 1
									          \end{array} ; z^2 \right). & 
										          \nonumber
											    \end{eqnarray}
    \end{prop}
    命題\ref{prop:exp-st-gg}は命題\ref{prop:exp-stz-gg}において$z=1$を代入したときの結果である.
\end{frame}
\section{主結果の様々な特殊化}
\begin{frame}
	命題\ref{prop:exp-st-gg}は次の
	積分公式と同値である.
	\begin{taggedprop}{$\;1'$}
		\label{prop:int-st-gg}
		\begin{equation*}
			\int_{- 1}^1 \int_{- 1}^1 | s - t |^{2 \nu} (1 - s^2)^{\lambda - \frac{1}{2}}
			(1 - t^2)^{\mu - \frac{1}{2}} C_\ell^{\lambda} (s) C_m^{\mu} (t) d s d t
		\end{equation*}
		{\scriptsize
		\begin{equation}
			=\frac{(- \nu)_{\frac{\ell + m}{2}} (- 1)^{\frac{\ell - m}{2}} \pi^{\frac{1}{2}} (2
			\lambda)_\ell (2 \mu)_m \Gamma \left( \lambda + \frac{1}{2} \right) \Gamma \left(
			\mu + \frac{1}{2} \right) \Gamma \left( \nu + \frac{1}{2} \right) \Gamma
		(\lambda + \mu + 2 \nu + 1)}{\ell!m! \Gamma \left( \lambda + \nu + \frac{\ell -
		m}{2} + 1 \right) \Gamma \left( \mu + \nu - \frac{\ell - m}{2} + 1 \right) \Gamma
		\left( \lambda + \mu + \nu + \frac{\ell + m}{2} + 1 \right)}
			\label{eqn:int-st-gg}
			\tag{1$'$}
		\end{equation}
		}
	\end{taggedprop}
	ここでポッホハマー記号$(x)_n$は以下のように定義される:\begin{equation*}
		(x)_n:=\frac{\Gamma(x+n)}{\Gamma(x)}=x(x+1)\cdots(x+n-1).
	\end{equation*}
\end{frame}
\begin{frame}{エルミート多項式}
%%	Gegenbauer polynomials are the family of polynomials $\left\{ C_n^\alpha \right\}_{n=1}^{\infty}$ that form an orthogonal basis of $L^2\left( [-1,1],(1-x^2)^{\lambda-\frac{1}{2}}dx \right)$.
%%	\begin{example*}
%%	\begin{equation*}
%%		\begin{array}[]{c}
%%			C_0^\lambda(x)=1;\\
%%			C_1^\lambda(x)=2\lambda x;\\
%%			C_2^\lambda(x)=-\lambda+2\lambda(1+\lambda)x^2;\\
%%			C_3^\lambda(x)=-2\lambda(1+\lambda)x+\frac{4}{3}\lambda(1+\lambda)(2+\lambda)x^3;\\
%%			C_4^\lambda(x)=\left(\frac{4+\lambda^3}{3}+4\lambda^2+\frac{8\lambda}{3}  \right)x^3+(-2\lambda^2-2\lambda)x;\\
%%			\cdots
%%		\end{array}
%%	\end{equation*}
%%	\end{example*}
%%	They also satisfy the second-order differential equation\begin{equation*}
%%		(1-x^2)y''-(2\lambda+1)xy'+n(n+\lambda)y=0.
%%	\end{equation*}
エルミート多項式$\left\{ H_n \right\}_{n=1}^\infty$は、
	$L^2\left( \mathbb{R},e^{-\frac{x^2}{2}}dx \right)$の直交多項式である.
	\begin{example*}
		最初のいくつは
	\vspace{-0.3cm}
		\begin{eqnarray*}
		H_0(x)&=& 1,\\
		H_1(x)&=& 2x,\\
		H_2(x)&=& 
		4x^2-2,\\
		H_3(x)&=& 8x^3-12x,\\
		H_4(x)&=& 16x^4-48x^2+12.\\
		\end{eqnarray*}
%%	\begin{equation*}
%%		\begin{array}[]{c}
%%		H_0(x)=1;\\
%%		H_1(x)=2x;\\
%%		H_2(x)=4x^2-2;\\
%%		H_3(x)=8x^3-12x;\\
%%		H_4(x)=16x^4-48x^2+12;\\
%%		\cdots
%%		\end{array}
%%	\end{equation*}
	\end{example*}
	エルミート多項式はGegenbauer多項式の極限として再現できる:
	%%\vspace{-0.3cm}
	\begin{equation*}
			H_n (x) = n! \lim_{\lambda \rightarrow \infty} \lambda^{- \frac{n}{2}}
			C_n^{\lambda} \left( \frac{x}{\sqrt{\lambda}} \right).
	\end{equation*}
\end{frame}
\begin{frame}
	命題 \ref{prop:int-st-gg}で $\lambda,\mu\to\infty$ 
	という極限をとれば(ただし、$\mu/\lambda=w\in\mathbb{R}$を一定にして極限をとる)、Gegenbauer多項式に関する積分公式(命題\ref{prop:int-st-gg})から
	エルミート多項式に関する積分公式を得る.
	\begin{cor}\label{cor:int-xzy-hh}
		$\nu\in\mathbb{C}$と$w\in\R$に対して、以下の公式が成り立つ:
		\begin{equation*}
			\begin{array}[]{c}
			\int_{- \infty}^{\infty} \int_{- \infty}^{\infty} | x - w y |^{2 \nu} e^{-
			x^2 - y^2} H_\ell (x) H_m (y) d x d y \\= (- \nu)_{\frac{\ell + m}{2}} (- 1)^{\frac{\ell
			- m}{2}} 2^{\ell + m} \pi^{\frac{1}{2}} \Gamma \left( \frac{1}{2} + \nu \right)
			(w^2 + 1)^{\nu - \frac{\ell + m}{2}} w^m .
			\end{array}
		\end{equation*}
	\end{cor}
\end{frame}
\begin{frame}
		\scriptsize
		\setcounter{cor}{0}
	\begin{textblock*}{0.5\textwidth}(-0.2cm,-3cm)
%%	命題 \ref{prop:exp-st-gg}で $\lambda,\mu\to\infty$ のように
%%	極限をとれば($\mu/\lambda=w\in\mathbb{R}$を保つように)、Gegenbauer多項式に関する積分公式(命題\ref{prop:int-st-gg})から
%%	エルミート多項式に関する積分公式を得る.
	\begin{cor}\label{cor:int-xzy-hh}
		\scalebox{0.9}{$\nu\in\mathbb{C}$と$w\in\R$
%%		複素数$\nu$と実素数$w$
	に対して、以下の公式が成り立つ:}
		{\tiny
		\begin{equation*}
			\kern-0.2cm
			\begin{array}[]{c}
				\pi^{-\frac{1}{2}}\int_{- \infty}^{\infty} \int_{- \infty}^{\infty} | x + w y |^{2 \nu} e^{-
			x^2 - y^2} H_\ell (x) H_m (y) d x d y \\= (- \nu)_{\frac{\ell + m}{2}}  (2i)^{\ell + m} \Gamma \left( \frac{1}{2} + \nu \right)
			(w^2 + 1)^{\nu - \frac{\ell + m}{2}} w^m .
			\end{array}
		\end{equation*}
	}
	\end{cor}
	\end{textblock*}
	\begin{textblock*}{0.5\textwidth}(0.52\textwidth,-3.1cm)
		\begin{fact*}[Mehta積分]
			{\tiny
		\begin{equation*}
			\begin{array}[]{c}
			{(2\pi)^{-\frac{k}{2}}}\displaystyle\int_{\mathbb{R}^k}\prod_{1\le i<j\le k}\myabs{t_i-t_j}^{2\gamma}\exp\left( -\myabs{\mathbf{t}}^2/2 \right)d\mathbf{t}\\
			=\displaystyle\prod_{j=1}^k\frac{\Gamma\left( 
			1+j\gamma\right)}{\Gamma(1+\gamma)}.
			\end{array}
		\end{equation*}
	}
		\end{fact*}
	\end{textblock*}
	\begin{textblock*}{0.5\textwidth}(0.3\textwidth,0.8cm)
		\tiny
		\xymatrixcolsep{5pc}
		\xymatrixrowsep{3pc}
		\xymatrix{
			\ar@{=>}[rd]_{\begin{array}[]{c}
			w=-1,\\\ell=m=0
		\end{array}}&&\ar@{=>}[ld]^{\begin{array}[]{c}
		k=2
	\end{array}}\\
			&&
		}
	\end{textblock*}
	\begin{textblock*}{0.5\textwidth}(0.26\textwidth,2.4cm)
		\begin{equation*}
			\int_{-\infty}^\infty\int_{-\infty}^\infty\myabs{x-y}^{2\nu}e^{-x^2-y^2}dxdy=\mypgf
		\end{equation*}
	\end{textblock*}
%%	\vspace{-0.2cm}
%%	\begin{remark}
%%		系\ref{cor:int-xzy-hh}の$w=1,\ell=m=0$の特殊化が
%%		\underline{Mehta積分}
%%	\vspace{-0.3cm}
%%		\begin{equation*}
%%			\frac{1}{(2\pi)^{k/2}}\int_{\mathbb{R}^k}\prod_{1\le i\le j\le k}\myabs{t_i-t_j}^{2\gamma}\exp\left( -\myabs{\mathbf{t}}^2/2 \right)d\mathbf{t}=\prod_{j=1}^k\frac{\Gamma\left( 
%%			1+j\gamma\right)}{\Gamma(1+\gamma)}
%%	\vspace{-0.4cm}
%%		\end{equation*}
%%		の$k=2$の特別な場合になる.
%%	\end{remark}
\end{frame}
\begin{frame}[fragile]{ハイアラーキー}
\begin{tikzpicture}
\draw[color=black] (0.0,0.0) rectangle (2.0,-0.5);
\node at (1.0,-0.25) {\color{black}{\scriptsize Warnaar'10: \color{blue}{4}}};
\draw[color=red] (2.1,0.0) rectangle (4.3,-0.5);
\node at (3.2,-0.25) {\color{red}{\scriptsize 命題 2: \color{blue}{4}}};
\draw[color=black] (5.7,0.0) rectangle (8.9,-0.5);
\node at (7.300000000000001,-0.25) {\color{black}{\scriptsize Tarasov-Varchenko'03: \color{blue}{3}}};
\draw[color=black] (9.1,0.0) rectangle (12.1,-0.5);
\node at (10.6,-0.25) {\color{black}{\scriptsize Dotsenko-Fateev'85: \color{blue}{4}}};
\draw[color=red] (2.1,-2.0) rectangle (4.3,-2.5);
\node at (3.2,-2.25) {\color{red}{\scriptsize 命題 $1'$: \color{blue}{3}}};
\draw[color=black] (5.0,-2.0) rectangle (6.7,-2.5);
\node at (5.85,-2.25) {\color{black}{\scriptsize Selberg'44: \color{blue}{3}}};
\filldraw[color=red,pattern color=red,pattern=north east lines] (3.2,-4.0) circle(0.3);
\node at (3.85,-4.0) {\color{blue}{3}};
\fill[color=black] (1.5,-6.0) circle(0.1);
\node at (1.95,-6.0) {\color{blue}{2}};
\fill[color=black] (3.2,-6.0) circle(0.1);
\node at (3.6500000000000004,-6.0) {\color{blue}{2}};
\fill[color=black] (5.85,-6.0) circle(0.1);%TV
\node at (6.3,-6.0) {\color{blue}{2}};
\fill[color=black] (8.9,-6.0) circle(0.1);%DF
\node at (9.35,-6.0) {\color{blue}{2}};

\draw[->,>=angle 90,color=black] (2.0,-0.5) -- node {} (5.0,-2.0) ;
\draw[->,>=angle 90,color=red] (3.2,-0.5) -- node {} (3.2,-2.0) ;
\draw[->,>=angle 90,color=black] (6.18,-0.5) -- node {} (5.85,-2.0) ;
\draw[->,>=angle 90,color=black] (9.1,-0.5) -- node {} (6.615,-2.0) ;
\draw[->,>=angle 90,color=red] (3.2,-2.5) -- node {\color{black}{\scriptsize $\kern1.5cm\ell=m=0$}} (3.2000000000000006,-3.7) ;
\draw[->,>=angle 90,color=black] (1.0,-0.5) -- node {} (1.490946425395748,-5.90041067935323) ;
\draw[->,>=angle 90,color=black] (3.0057054739713385,-4.2285817953278375) -- node {} (1.5573628100792203,-5.93251434108327) ;
\draw[->,>=angle 90,color=black] (3.2,-4.3) -- node {} (3.2,-5.914999999999999) ;
\draw[->,>=angle 90,color=black] (3.4394567450999696,-4.180722071773562) -- node {} (5.777393604812446,-5.9452027206131675) ;
\draw[->,>=angle 90,color=black] (3.4830799946398643,-4.099326313908724) -- node {} (8.810326217188193,-5.968535514802874) ;
\draw[->,>=angle 90,color=black] (5.34,-2.5) -- node {} (3.252164719493017,-5.914683869988056) ;
\draw[->,>=angle 90,color=black] (10.3,-0.5) -- node {}  (8.92466,-5.90309);%DF->DF'
\draw[->,>=angle 90,color=black] (7.62,-0.5) -- node {} (5.88,-5.9048) ;%TV->TV'

\end{tikzpicture}
ここで{\color{blue}{青い数字}}は連続パラメータの個数を表す.
\end{frame}
\begin{frame}[fragile]
\begin{tikzpicture}
\draw[color=black] (-0.2,0.0) rectangle (2.2,-0.5);
\node at (1.0,-0.25) {\color{black}{\scriptsize Warnaar'10: \color{blue}{4}}};

\draw[color=red] (2.4,0.0) rectangle (4.4,-0.5);
\node at (3.45,-0.25) {\color{red}{\scriptsize 命題 2: \color{blue}{4}}};

\draw[color=black] (4.5,0.0) rectangle (7.7,-0.5);
\node at (6.1,-0.25) {\color{black}{\scriptsize Tarasov-Varchenko'03: \color{blue}{3}}};

\draw[color=black] (8.4,0.0) rectangle (11.35,-0.5);
\node at (9.85,-0.25) {\color{black}{\scriptsize Dotsenko-Fateev'85: \color{blue}{4}}};

\draw[color=red] (2.1,-2.0) rectangle (4.3,-2.5);
\node at (3.2,-2.25) {\color{red}{\scriptsize 命題 1$'$: \color{blue}{3}}};

\filldraw[color=red,pattern color=red,pattern=north east lines] (3.2,-4.0) circle(0.3);
\node at (3.85,-4.0) {\color{blue}{3}};

\fill[color=black] (1.5,-6.0) circle(0.1);
\node at (1.95,-6.0) {\color{blue}{2}};

\fill[color=green] (3.2,-6.0) circle(0.3);
\node at (3.6500000000000004,-6.0) {\color{blue}{2}};

\fill[color=black] (5.85,-6.0) circle(0.1);
\node at (6.3,-6.0) {\color{blue}{2}};

\fill[color=black] (8.9,-6.0) circle(0.1);
\node at (9.35,-6.0) {\color{blue}{2}};


\draw[->,>=angle 90,color=black] (2.0,-0.5) -- node {} (5.0,-2.0) ;%War -> S
\draw[->,>=angle 90,style=very thick,color=red] (3.2,-0.5) -- node {} (3.2,-2.0) ; %P2 -> P3
\draw[->,>=angle 90,color=black] (5.1,-0.5) -- node {} (5.85,-2.0) ; %TV -> S
\draw[->,>=angle 90,color=black] (8.5,-0.5) -- node {} (6.605,-2.0) ; % DF -> S
\draw[->,>=angle 90,style=very thick,color=red] (3.2,-2.5) -- node {} (3.2000000000000006,-3.7) ; % P3 -> P3'
\draw[->,>=angle 90,color=black] (1.0,-0.5) -- node {} (1.490946425395748,-5.90041067935323) ; %W -> W'
\draw[->,>=angle 90,color=black] (3.0057054739713385,-4.2285817953278375) -- node {} (1.5573628100792203,-5.93251434108327) ;%P3' -> W'
\draw[->,>=angle 90,style=very thick,color=red] (3.2,-4.3) -- node {} (3.2,-5.714999999999999) ;%P3' -> S'
\draw[->,>=angle 90,color=black] (3.4394567450999696,-4.180722071773562) -- node {} (5.777393604812446,-5.9452027206131675) ;%P3' -> DF'
\draw[->,>=angle 90,color=black] (3.4830799946398643,-4.099326313908724) -- node {} (8.810326217188193,-5.968535514802874) ;%P3' -> TV'
\draw[->,>=angle 90,style=very thick,color=blue] (5.34,-2.5) -- node {} (3.35649415847905,-5.74405160996417) ;%S -> S'
\draw[->,>=angle 90,color=black] (9.6,-0.5) -- node {}  (8.92466,-5.90309);%DF -> DF'
\draw[->,>=angle 90,color=black] (6.5,-0.5) -- node {} (5.88,-5.9048) ;%TV -> TV'

\draw[color=blue,fill=white] (4.8,-2.0) rectangle (6.9,-2.5);
\node at (5.85,-2.25) {\color{blue}{\scriptsize Selberg'44: \color{blue}{3}}};

\end{tikzpicture}
ここで{\color{blue}{青い数字}}は連続パラメータの個数を表す.
\end{frame}
\begin{frame}[fragile]%frame Selberg
	\scriptsize
	\begin{textblock*}{\textwidth}(1cm,0.8cm)
	  	\begin{tikzpicture}
			\draw[->,>=angle 90,color=red,style=very thick] (2.0,4.8) -- node{} (2.0,1.95);
			\node at (2.8,3.4) {\color{red}{\textnormal{ \tiny$\begin{array}[]{c}\ell=m=0,\\ \lambda=\mu \end{array}$}}};
		\end{tikzpicture}
	\end{textblock*}
	\begin{textblock*}{\textwidth}(3.99cm,1.7cm)
	  	\begin{tikzpicture}
			\draw[->,>=angle 90,color=blue,style=very thick] (1.0,1.0) -- node{} (-0.5,-1.0);
			\node at (-0.25,0.0) {\color{blue}{\normalfont{\tiny$\begin{array}[]{c}
				k=2,\\
			\alpha=\beta
		\end{array}$}}};
		\end{tikzpicture}
	\end{textblock*}
	\begin{textblock*}{0.5\textwidth}(-0.4cm,3.7cm)
		 {\tiny
			\begin{tcolorbox}[colback=green!10!white,colframe=green]
				\vspace{-0.5cm}
		\begin{equation*}
			\kern-0.4cm
			\begin{array}[]{c}
				\displaystyle\iint\displaylimits_{\kern0.2cm [-1,1]^2}\kern-0.3cm | s - t |^{2 \nu}\kern-0.05cm (1 - s^2)^{\lambda - \frac{1}{2}}
			\kern-0.05cm(1 - t^2)^{\lambda - \frac{1}{2}}d s d t\\
			=\mypgf
			\end{array}
		\end{equation*}
			     \end{tcolorbox}
		}
	\end{textblock*}
	\begin{textblock*}{0.5\textwidth}(0.5\textwidth,1.3cm)
    			\setulcolor{Blue}
		\begin{fact*}[{\ul{Selberg'44}}]
			{\tiny
	 \begin{equation*}
		\begin{array}[]{c}
		\displaystyle\int_{t \in [0, 1]^k} \displaystyle\prod\displaylimits_{i=1}^k t_i^{\alpha-1}(1-t_i)^{\beta-1} 
		\kern-0.1cm\displaystyle\prod\displaylimits_{1\le i<j\le k}\myabs{t_i-t_j}^{2\gamma} d
				t\\=\mypgf,\\
				\tmop{Re} (\alpha), \tmop{Re} (\beta) > 0, | \gamma | \ll 1.
		\end{array}
			\end{equation*}
		}
		\end{fact*}
	\end{textblock*}
	\begin{textblock*}{0.5\textwidth}(0.635\textwidth,-2.8cm)
			  \begin{tikzpicture}[scale=0.4]
				\draw[color=black] (-0.2,0.0) rectangle (2.2,-0.5);
\node at (1.0,-0.25) {\color{black}{\tiny W'10:\color{blue}{4}}};

\draw[color=red] (2.4,0.0) rectangle (4.4,-0.5);
\node at (3.45,-0.25) {\color{red}{\tiny 命題 2:\color{blue}{4}}};

\draw[color=black] (4.6,0.0) rectangle (7.6,-0.5);
\node at (6.1,-0.25) {\color{black}{\tiny T-V'03:\color{blue}{3}}};

\draw[color=black] (8.5,0.0) rectangle (11.25,-0.5);
\node at (9.85,-0.25) {\color{black}{\tiny D-F'85:\color{blue}{4}}};

\draw[color=red] (2.1,-2.0) rectangle (4.3,-2.5);
\node at (3.2,-2.25) {\color{red}{\tiny 命題 1$'$:\color{blue}{3}}};

\filldraw[color=red,pattern color=red,pattern=north east lines] (3.2,-4.0) circle(0.3);
\node at (3.85,-4.0) {\color{blue}{3}};

\fill[color=black] (1.5,-6.0) circle(0.1);
\node at (1.95,-6.0) {\color{blue}{2}};

\fill[color=green] (3.2,-6.0) circle(0.3);
\node at (3.6500000000000004,-6.0) {\color{blue}{2}};

\fill[color=black] (5.85,-6.0) circle(0.1);
\node at (6.3,-6.0) {\color{blue}{2}};

\fill[color=black] (8.9,-6.0) circle(0.1);
\node at (9.35,-6.0) {\color{blue}{2}};


\draw[->,>=angle 90,color=black] (2.0,-0.5) -- node {} (5.0,-2.0) ;%War -> S
\draw[->,>=angle 90,style=very thick,color=red] (3.2,-0.5) -- node {} (3.2,-2.0) ; %P2 -> P3
\draw[->,>=angle 90,color=black] (5.1,-0.5) -- node {} (5.85,-2.0) ; %TV -> S
\draw[->,>=angle 90,color=black] (8.5,-0.5) -- node {} (6.605,-2.0) ; % DF -> S
\draw[->,>=angle 90,style=very thick,color=red] (3.2,-2.5) -- node {} (3.2000000000000006,-3.7) ; % P3 -> P3'
\draw[->,>=angle 90,color=black] (1.0,-0.5) -- node {} (1.490946425395748,-5.90041067935323) ; %W -> W'
\draw[->,>=angle 90,color=black] (3.0057054739713385,-4.2285817953278375) -- node {} (1.5573628100792203,-5.93251434108327) ;%P3' -> W'
\draw[->,>=angle 90,style=very thick,color=red] (3.2,-4.3) -- node {} (3.2,-5.714999999999999) ;%P3' -> S'
\draw[->,>=angle 90,color=black] (3.4394567450999696,-4.180722071773562) -- node {} (5.777393604812446,-5.9452027206131675) ;%P3' -> DF'
\draw[->,>=angle 90,color=black] (3.4830799946398643,-4.099326313908724) -- node {} (8.810326217188193,-5.968535514802874) ;%P3' -> TV'
\draw[->,>=angle 90,style=very thick,color=blue] (5.34,-2.5) -- node {} (3.35649415847905,-5.74405160996417) ;%S -> S'
\draw[->,>=angle 90,color=black] (9.6,-0.5) -- node {}  (8.92466,-5.90309);%DF -> DF'
\draw[->,>=angle 90,color=black] (6.5,-0.5) -- node {} (5.88,-5.9048) ;%TV -> TV'

\draw[color=blue,fill=white] (4.8,-2.0) rectangle (6.9,-2.5);
\node at (5.85,-2.25) {\color{blue}{\tiny S'44:\color{blue}{3}}};

				\end{tikzpicture}
				%\caption{A gull}
	\end{textblock*}
	{
	\begin{textblock*}{0.5\textwidth}(-0.2cm,-2.4cm)
    			\setulcolor{Red}
			\begin{block}{{\ul{{\mbox{命題}} $1'$}}}
		{\tiny
		\begin{equation*}
			\kern-0.4cm
			\begin{array}[]{c}
				\displaystyle\iint\displaylimits_{\kern0.2cm [-1,1]^2}\kern-0.3cm | s - t |^{2 \nu}\kern-0.05cm (1 - s^2)^{\lambda - \frac{1}{2}}
			\kern-0.05cm(1 - t^2)^{\mu - \frac{1}{2}} C_\ell^{\lambda} (s) C_m^{\mu} (t) d s d t\\
			=\mypgf
			\end{array}
		\end{equation*}
		}
	\end{block}
	\end{textblock*}
	}
\end{frame}
\begin{frame}[fragile]
\begin{tikzpicture}
\draw[color=blue] (-0.2,0.0) rectangle (2.2,-0.5);
\node at (1.0,-0.25) {\color{blue}{\scriptsize Warnaar'10: \color{blue}{4}}};

\draw[color=red] (2.4,0.0) rectangle (4.4,-0.5);
\node at (3.45,-0.25) {\color{red}{\scriptsize 命題 2: \color{blue}{4}}};

\draw[color=black] (4.6,0.0) rectangle (7.6,-0.5);
\node at (6.1,-0.25) {\color{black}{\scriptsize Tarasov-Varchenko'03: \color{blue}{3}}};

\draw[color=black] (8.5,0.0) rectangle (11.25,-0.5);
\node at (9.85,-0.25) {\color{black}{\scriptsize Dotsenko-Fateev'85: \color{blue}{4}}};

\draw[color=red] (2.1,-2.0) rectangle (4.3,-2.5);
\node at (3.2,-2.25) {\color{red}{\scriptsize 命題 1$'$: \color{blue}{3}}};

\filldraw[color=red,pattern color=red,pattern=north east lines] (3.2,-4.0) circle(0.3);
\node at (3.85,-4.0) {\color{blue}{3}};

\fill[color=green] (1.5,-6.0) circle(0.3);
\node at (1.95,-6.0) {\color{blue}{2}};

\fill[color=black] (3.2,-6.0) circle(0.1);
\node at (3.6500000000000004,-6.0) {\color{blue}{2}};

\fill[color=black] (5.85,-6.0) circle(0.1);
\node at (6.3,-6.0) {\color{blue}{2}};

\fill[color=black] (8.9,-6.0) circle(0.1);
\node at (9.35,-6.0) {\color{blue}{2}};


\draw[->,>=angle 90,color=black] (2.0,-0.5) -- node {} (5.0,-2.0) ;%War -> S
\draw[->,>=angle 90,color=red,style=very thick] (3.2,-0.5) -- node {} (3.2,-2.0) ; %P2 -> P3
\draw[->,>=angle 90,color=black] (5.1,-0.5) -- node {} (5.85,-2.0) ; %TV -> S
\draw[->,>=angle 90,color=black] (8.5,-0.5) -- node {} (6.605,-2.0) ; % DF -> S
\draw[->,>=angle 90,color=red,style=very thick] (3.2,-2.5) -- node {} (3.2000000000000006,-3.7) ; % P3 -> P3'
\draw[->,>=angle 90,color=blue,style=very thick] (1.0,-0.5) -- node {} (1.4719,-5.7) ; %W -> W'
\draw[->,>=angle 90,color=red,style=very thick] (3.00,-4.23) -- node {} (1.68,-5.77) ;%P3' -> W'
\draw[->,>=angle 90,color=black] (3.2,-4.3) -- node {} (3.2,-5.914999999999999) ;%P3' -> S'
\draw[->,>=angle 90,color=black] (3.4394567450999696,-4.180722071773562) -- node {} (5.777393604812446,-5.9452027206131675) ;%P3' -> DF'
\draw[->,>=angle 90,color=black] (3.4830799946398643,-4.099326313908724) -- node {} (8.810326217188193,-5.968535514802874) ;%P3' -> TV'
\draw[->,>=angle 90,color=black] (5.34,-2.5) -- node {} (3.252164719493017,-5.914683869988056) ;%S -> S'
\draw[->,>=angle 90,color=black] (9.6,-0.5) -- node {}  (8.92466,-5.90309);%DF -> DF'
\draw[->,>=angle 90,color=black] (6.5,-0.5) -- node {} (5.88,-5.9048) ;%TV -> TV'

\draw[color=black,fill=white] (4.8,-2.0) rectangle (6.9,-2.5);
\node at (5.85,-2.25) {\color{black}{\scriptsize Selberg'44: \color{black}{3}}};

\end{tikzpicture}
ここで{\color{blue}{青い数字}}は連続パラメータの個数を表す.
\end{frame}
\begin{frame}[fragile]%frame Warnaar
	\scriptsize
	\begin{textblock*}{0.5\textwidth}(-0.2cm,-2.65cm)
    			\setulcolor{Blue}
			\begin{fact*}[\ul{Warnaar'10} (1.4)]
				{\tiny
		\begin{equation*}
%%			\kern-2.15cm
			\begin{array}{c}
				\displaystyle\int\displaylimits_{(t,s)\in C_{\beta_1,\gamma}^{k_1,k_2}}
				\displaystyle\prod\displaylimits_{i=1}^{k_1}t_i^{\alpha_1-1}(1-t_i)^{\beta_1-1}\\
				\times\displaystyle\prod\displaylimits_{j=1}^{k_2}s_j^{\alpha_2-1}(1-s_j)^{\beta_2-1}
				\displaystyle\prod_{1\le i<j\le k_1}\myabs{t_i-t_j}^{2\gamma}\\
				\times\displaystyle\prod_{1\le i<j\le k_2}\myabs{s_i-s_j}^{2\gamma}
				\displaystyle\prod_{i,j=1}^{k_1,k_2}\myabs{t_i-s_j}^{-\gamma}dsdt
%%				\displaystyle\bigintssss_{ \mbox{\tiny$\kern1.9cm (t,s)\kern-0.1cm\in\kern-0.1cm C_{\beta_1, \gamma}^{k_1, k_2} [0, 1]$}} \kern-1.95cm\textstyle\prod^{\alpha_1 - 1,
%%				\beta_1 - 1}\kern-0.05cm (t) \kern-0.09cm\textstyle\prod^{\alpha_2 - 1, \beta_2 - 1}\kern-0.05cm (s)
%%%%					    \\ \times
%%					    \Delta^{2 \gamma}\kern-0.05cm(t)
%%					      \Delta^{2 \gamma}\kern-0.05cm (s) \Delta^{- \gamma}\kern-0.05cm (t, s)\\
				      \end{array}\end{equation*}
%%			\vspace{-0.5cm}
		\begin{equation*}
			\kern-0.29cm
			\begin{array}{c}
					        = \mypgf,\\
%%					  \textstyle\prod^{\alpha_1 - 1,
%%					  \beta_1 - 1} (t):=\displaystyle\prod_{i=1}^{k_1}t_i^{\alpha_1-1}(1-t_i)^{\beta_1-1},\\
%%					  \Delta^{2\gamma}(t):=\displaystyle\prod_{1\le i<j\le k_1 }\kern-0.2cm\myabs{t_i}^{2\gamma}, \Delta^\gamma(t,s):=\displaystyle\prod\displaylimits_{i,j=1}^{k_1,k_2}
%%					  \myabs{t_i-s_j}^{\gamma}\kern-0.1cm,\\
%%					  \\
					  \mbox{ここで  }C^{k_1,k_2}_{\beta_1,\gamma}\mbox{: ある特異チェイン, }
					  \beta_1 + \beta_2 = \gamma + 1,\\
					    \tmop{Re} (\alpha_i), \tmop{Re} (\beta_i) > 0,  \myabs{\gamma} \ll 1 ,\\ \quad 1
						\le \forall i \le \displaystyle\min_i \{ k_i \} : \beta_1 + (i - k_2 - 1)
						  \gamma \nin \mathbb{Z}.
			\end{array}
			\end{equation*}
			\vspace{-0.29cm}
				}
			\end{fact*}
	\end{textblock*}
	\begin{textblock*}{\textwidth}(7.99cm,2.75cm)
	  	\begin{tikzpicture}
			\draw[->,>=angle 90,color=blue,style=very thick] (1.0,1.0) -- node{} (-2.0,-1.0);
			\node at (0.5,-0.3) {\color{blue}{\tiny$
				\begin{array}[]{c}
					\ell=m=0,\\
				\lambda+\mu+2\nu=-1
				\end{array}
		$}};
		\end{tikzpicture}
	\end{textblock*}
	\begin{textblock*}{0.7\textwidth}(-0.4cm,3.7cm)
		 {\tiny
			\begin{tcolorbox}[colback=green!10!white,colframe=green]
				\vspace{-0.7cm}
		\begin{equation*}
			\kern-0.4cm
			\begin{array}[]{c}
				\left(\displaystyle\iint\displaylimits_{\scalebox{0.7}{$\begin{array}[]{c}
					[0,1]^2\\t<s
				\end{array}$}}+\frac{\sin(\pi\alpha_1)}{\sin(\pi\alpha_2)}\displaystyle\iint\displaylimits_{\scalebox{0.7}{$\begin{array}[]{c}
					[0,1]^2\\t>s
				\end{array}$}} \right)(t(1-t))^{\mu - \frac{1}{2}}  (s(1-s))^{\lambda - \frac{1}{2}}  | t - s |^{2\nu} d s d t
			\\=\mypgf
			\end{array}
		\end{equation*}
				\vspace{-0.7cm}
			     \end{tcolorbox}
		}
	\end{textblock*}
	\begin{textblock*}{\textwidth}(0.7cm,3.1cm)
	  	\begin{tikzpicture}
			\draw[->,>=angle 90,color=red,style=very thick] (2.0,2.8) -- node{} (2.0,2.0);
			\node at (4.0,2.3) {\color{red}{\tiny$k_1=k_2=1,\alpha_1=\beta_1,\alpha_2=\beta_2$}};
		\end{tikzpicture}
	\end{textblock*}
	\begin{textblock*}{0.5\textwidth}(0.635\textwidth,-2.8cm)
			  \begin{tikzpicture}[scale=0.4]
				\draw[color=blue] (-0.2,0.0) rectangle (2.2,-0.5);
\node at (1.0,-0.25) {\color{blue}{\tiny W'10:\color{blue}{4}}};

\draw[color=red] (2.4,0.0) rectangle (4.4,-0.5);
\node at (3.45,-0.25) {\color{red}{\tiny 命題 2:\color{blue}{4}}};

\draw[color=black] (4.6,0.0) rectangle (7.6,-0.5);
\node at (6.1,-0.25) {\color{black}{\tiny T-V'03:\color{blue}{3}}};

\draw[color=black] (8.5,0.0) rectangle (11.25,-0.5);
\node at (9.85,-0.25) {\color{black}{\tiny D-F'85:\color{blue}{4}}};

\draw[color=red] (2.1,-2.0) rectangle (4.3,-2.5);
\node at (3.2,-2.25) {\color{red}{\tiny 命題 1$'$:\color{blue}{3}}};

\filldraw[color=red,pattern color=red,pattern=north east lines] (3.2,-4.0) circle(0.3);
\node at (3.85,-4.0) {\color{blue}{3}};

\fill[color=green] (1.5,-6.0) circle(0.3);
\node at (1.95,-6.0) {\color{blue}{2}};

\fill[color=black] (3.2,-6.0) circle(0.1);
\node at (3.6500000000000004,-6.0) {\color{blue}{2}};

\fill[color=black] (5.85,-6.0) circle(0.1);
\node at (6.3,-6.0) {\color{blue}{2}};

\fill[color=black] (8.9,-6.0) circle(0.1);
\node at (9.35,-6.0) {\color{blue}{2}};


\draw[->,>=angle 90,color=black] (2.0,-0.5) -- node {} (5.0,-2.0) ;%War -> S
\draw[->,>=angle 90,color=red,style=very thick] (3.2,-0.5) -- node {} (3.2,-2.0) ; %P2 -> P3
\draw[->,>=angle 90,color=black] (5.1,-0.5) -- node {} (5.85,-2.0) ; %TV -> S
\draw[->,>=angle 90,color=black] (8.5,-0.5) -- node {} (6.605,-2.0) ; % DF -> S
\draw[->,>=angle 90,color=red,style=very thick] (3.2,-2.5) -- node {} (3.2000000000000006,-3.7) ; % P3 -> P3'
\draw[->,>=angle 90,color=blue,style=very thick] (1.0,-0.5) -- node {} (1.4719,-5.7) ; %W -> W'
\draw[->,>=angle 90,color=red,style=very thick] (3.00,-4.23) -- node {} (1.68,-5.77) ;%P3' -> W'
\draw[->,>=angle 90,color=black] (3.2,-4.3) -- node {} (3.2,-5.914999999999999) ;%P3' -> S'
\draw[->,>=angle 90,color=black] (3.4394567450999696,-4.180722071773562) -- node {} (5.777393604812446,-5.9452027206131675) ;%P3' -> DF'
\draw[->,>=angle 90,color=black] (3.4830799946398643,-4.099326313908724) -- node {} (8.810326217188193,-5.968535514802874) ;%P3' -> TV'
\draw[->,>=angle 90,color=black] (5.34,-2.5) -- node {} (3.252164719493017,-5.914683869988056) ;%S -> S'
\draw[->,>=angle 90,color=black] (9.6,-0.5) -- node {}  (8.92466,-5.90309);%DF -> DF'
\draw[->,>=angle 90,color=black] (6.5,-0.5) -- node {} (5.88,-5.9048) ;%TV -> TV'

\draw[color=black,fill=white] (4.8,-2.0) rectangle (6.9,-2.5);
\node at (5.85,-2.25) {\color{black}{\tiny S'44:\color{black}{3}}};

				\end{tikzpicture}
	\end{textblock*}
	\begin{textblock*}{0.5\textwidth}(0.52\textwidth,0.75cm)
    			\setulcolor{Red}
			\begin{block}{\ul{{\mbox{命題}} $1'$}}
		{\tiny
		\begin{equation*}
			\kern-1.1cm
			\begin{array}[]{c}
				\displaystyle\iint\displaylimits_{\kern0.2cm [-1,1]^2}\kern-0.3cm | s - t |^{2 \nu}\kern-0.05cm (1 - s^2)^{\lambda - \frac{1}{2}}
			\kern-0.05cm(1 - t^2)^{\mu - \frac{1}{2}} C_\ell^{\lambda} (s) C_m^{\mu} (t) d s d t\kern-1cm\\
			=\mypgf
			\end{array}
		\end{equation*}
		}
	\end{block}
	\end{textblock*}
\end{frame}
\begin{frame}[fragile]
\begin{tikzpicture}
\draw[color=black] (-0.2,0.0) rectangle (2.2,-0.5);
\node at (1.0,-0.25) {\color{black}{\scriptsize Warnaar'10: \color{blue}{4}}};

\draw[color=red] (2.4,0.0) rectangle (4.4,-0.5);
\node at (3.45,-0.25) {\color{red}{\scriptsize 命題 2: \color{blue}{4}}};

\draw[color=blue] (4.5,0.0) rectangle (7.7,-0.5);
\node at (6.1,-0.25) {\color{blue}{\scriptsize Tarasov-Varchenko'03: \color{blue}{3}}};

\draw[color=black] (8.4,0.0) rectangle (11.35,-0.5);
\node at (9.85,-0.25) {\color{black}{\scriptsize Dotsenko-Fateev'85: \color{blue}{4}}};

\draw[color=red] (2.1,-2.0) rectangle (4.3,-2.5);
\node at (3.2,-2.25) {\color{red}{\scriptsize 命題 1$'$: \color{blue}{3}}};

\filldraw[color=red,pattern color=red,pattern=north east lines] (3.2,-4.0) circle(0.3);
\node at (3.85,-4.0) {\color{blue}{3}};

\fill[color=black] (1.5,-6.0) circle(0.1);
\node at (1.95,-6.0) {\color{blue}{2}};

\fill[color=black] (3.2,-6.0) circle(0.1);
\node at (3.6500000000000004,-6.0) {\color{blue}{2}};

\fill[color=green] (5.85,-6.0) circle(0.3);
\node at (6.3,-6.0) {\color{blue}{2}};

\fill[color=black] (8.9,-6.0) circle(0.1);
\node at (9.35,-6.0) {\color{blue}{2}};


\draw[->,>=angle 90,color=black] (2.0,-0.5) -- node {} (5.0,-2.0) ;%War -> S
\draw[->,>=angle 90,style=very thick,color=red] (3.2,-0.5) -- node {} (3.2,-2.0) ; %P2 -> P3
\draw[->,>=angle 90,color=black] (5.1,-0.5) -- node {} (5.85,-2.0) ; %TV -> S
\draw[->,>=angle 90,color=black] (8.5,-0.5) -- node {} (6.605,-2.0) ; % DF -> S
\draw[->,>=angle 90,style=very thick,color=red] (3.2,-2.5) -- node {} (3.2000000000000006,-3.7) ; % P3 -> P3'
\draw[->,>=angle 90,color=black] (1.0,-0.5) -- node {} (1.490946425395748,-5.90041067935323) ; %W -> W'
\draw[->,>=angle 90,color=black] (3.0057054739713385,-4.2285817953278375) -- node {} (1.5573628100792203,-5.93251434108327) ;%P3' -> W'
\draw[->,>=angle 90,color=black] (3.2,-4.3) -- node {} (3.2,-5.914999999999999) ;%P3' -> S'
\draw[->,>=angle 90,color=red,style=very thick] (3.44,-4.18) -- node {} (5.6174,-5.82446) ;%P3' -> TV'
\draw[->,>=angle 90,color=black,color=black] (3.48,-4.099) -- node {} (8.81,-5.9685) ;%P3' -> DF'
\draw[->,>=angle 90,color=black] (5.34,-2.5) -- node {} (3.252164719493017,-5.914683869988056) ;%S -> S'
\draw[->,>=angle 90,color=black] (5.34,-2.5) -- node {} (3.252164719493017,-5.914683869988056) ;%S -> S'
\draw[->,>=angle 90,color=black] (9.6,-0.5) -- node {}  (8.92466,-5.90309);%DF -> DF'
\draw[->,>=angle 90,color=blue,style=very thick] (6.5,-0.5) -- node {} (5.90279,-5.70610) ;%TV -> TV'

\draw[color=black,fill=white] (4.8,-2.0) rectangle (6.9,-2.5);
\node at (5.85,-2.25) {\color{black}{\scriptsize Selberg'44: \color{black}{3}}};

\end{tikzpicture}
ここで{\color{blue}{青い数字}}は連続パラメータの個数を表す.
\end{frame}
\begin{frame}[fragile]%frame TV
	\scriptsize
	\begin{textblock*}{\textwidth}(1cm,0.8cm)
	  	\begin{tikzpicture}
			\draw[->,>=angle 90,color=red,style=very thick] (2.0,4.8) -- node{} (2.0,1.95);
			\node at (2.8,3.4) {\color{red}{\textnormal{ \tiny$\begin{array}[]{c}\ell=m=0,\\ \mu=\frac{1}{2} \end{array}$}}};
		\end{tikzpicture}
	\end{textblock*}
	\begin{textblock*}{\textwidth}(3.89cm,1.7cm)
	  	\begin{tikzpicture}
			\draw[->,>=angle 90,color=blue,style=very thick] (1.0,1.0) -- node{} (-0.5,-1.0);
			\node at (-0.45,0.0) {\color{blue}{\normalfont{\tiny$\begin{array}[]{c}
				k=2,\\
				\beta_2=1,\\
				\alpha=\beta_1
		\end{array}$}}};
		\end{tikzpicture}
	\end{textblock*}
	\begin{textblock*}{0.5\textwidth}(-0.4cm,3.7cm)
		 {\tiny
			\begin{tcolorbox}[colback=green!10!white,colframe=green]
				\vspace{-0.5cm}
			\begin{equation*}
				\kern-0.6cm
				\begin{array}[]{c}
					
				\displaystyle\iint\displaylimits_{\begin{array}[]{c}
				(s, t) \in [0, 1]^2\\t<s
			\end{array}} s^{\lambda - \frac{1}{2}} (1 - s)^{\lambda-\frac{1}{2}} \myabs{s-t}^{2\nu} d s d t\\
			=\mypgf
				\end{array}
				\vspace{-0.3cm}
			\end{equation*}

			     \end{tcolorbox}
		}
	\end{textblock*}
	\begin{textblock*}{0.5\textwidth}(0.515\textwidth,0.1cm)
    			\setulcolor{Blue}
		\begin{fact*}[{\ul{Tarasov-Varchenko'03}}]
			{\tiny
		\begin{equation*}
%%			\kern-1.45cm
			\begin{array}{c}
				\displaystyle\int_{(t,s)\in C_\gamma^{k_1,k_2}}
				\displaystyle\prod\displaylimits_{i=1}^{k_1}t_i^{\alpha-1}(1-t_i)^{\beta_1-1}\\
				\times\displaystyle\prod\displaylimits_{j=1}^{k_2}(1-s_j)^{\beta_2-1}
				\displaystyle\prod_{1\le i<j\le k_1}\myabs{t_i-t_j}^{2\gamma}\\
				\times\displaystyle\prod_{1\le i<j\le k_2}\myabs{s_i-s_j}^{2\gamma}
				\displaystyle\prod_{i,j=1}^{k_1,k_2}\myabs{t_i-s_j}^{-\gamma}dsdt
%%				\displaystyle\bigintssss_{ \mbox{\tiny$\kern1.4cm (t,s)\kern-0.1cm\in\kern-0.1cm C_{\gamma}^{k_1, k_2} [0, 1]$}} \kern-1.65cm\textstyle\prod^{\alpha - 1,
%%				\beta_1 - 1}\kern-0.05cm (t) \kern-0.09cm\textstyle\prod^{0, \beta_2 - 1}\kern-0.05cm (s)
%%%%					    \\ \times
%%					    \Delta^{2 \gamma}\kern-0.05cm(t)
%%					      \Delta^{2 \gamma}\kern-0.05cm (s) \Delta^{- \gamma}\kern-0.05cm (t, s)\\
				      \end{array}\end{equation*}
%%			\vspace{-0.5cm}
%%			\kern-0.29cm
			      \begin{equation*}
			\begin{array}{c}
					        = \mypgf\\
					  \mbox{ここで、  }	C^{k_1,k_2}_{\gamma}\mbox{: ある特異チェイン}.\\
%%					  \textstyle\prod^{\alpha - 1,
%%					  \beta_1 - 1} (t):=\displaystyle\prod_{i=1}^{k_1}t_i^{\alpha-1}(1-t_i)^{\beta_1-1},\\
%%					  \Delta^{2\gamma}(t):=\displaystyle\prod_{1\le i<j\le k_1 }\kern-0.2cm\myabs{t_i}^{2\gamma}, \Delta^\gamma(t,s):=\displaystyle\prod\displaylimits_{i,j=1}^{k_1,k_2}
%%					  \myabs{t_i-s_j}^{\gamma}\kern-0.1cm.\\
%%					  \\
%%					\tmop{Re} (\alpha_i), \tmop{Re} (\beta_2) > 0, | \tmop{Re} \gamma | \ll 1.
			\end{array}
			\end{equation*}
			\vspace{-0.29cm}
				}
		\end{fact*}
	\end{textblock*}
	\begin{textblock*}{0.5\textwidth}(0.635\textwidth,-2.8cm)
			  \begin{tikzpicture}[scale=0.4]
				\draw[color=black] (-0.2,0.0) rectangle (2.2,-0.5);
\node at (1.0,-0.25) {\color{black}{\tiny W'10:\color{blue}{4}}};

\draw[color=red] (2.4,0.0) rectangle (4.4,-0.5);
\node at (3.45,-0.25) {\color{red}{\tiny 命題 2:\color{blue}{4}}};

\draw[color=blue] (4.6,0.0) rectangle (7.6,-0.5);
\node at (6.1,-0.25) {\color{blue}{\tiny T-V'03:\color{blue}{3}}};

\draw[color=black] (8.5,0.0) rectangle (11.25,-0.5);
\node at (9.85,-0.25) {\color{black}{\tiny D-F'85:\color{blue}{4}}};

\draw[color=red] (2.1,-2.0) rectangle (4.3,-2.5);
\node at (3.2,-2.25) {\color{red}{\tiny 命題 1$'$:\color{blue}{3}}};

\filldraw[color=red,pattern color=red,pattern=north east lines] (3.2,-4.0) circle(0.3);
\node at (3.85,-4.0) {\color{blue}{3}};

\fill[color=black] (1.5,-6.0) circle(0.1);
\node at (1.95,-6.0) {\color{blue}{2}};

\fill[color=black] (3.2,-6.0) circle(0.1);
\node at (3.6500000000000004,-6.0) {\color{blue}{2}};

\fill[color=green] (5.85,-6.0) circle(0.3);
\node at (6.3,-6.0) {\color{blue}{2}};

\fill[color=black] (8.9,-6.0) circle(0.1);
\node at (9.35,-6.0) {\color{blue}{2}};


\draw[->,>=angle 90,color=black] (2.0,-0.5) -- node {} (5.0,-2.0) ;%War -> S
\draw[->,>=angle 90,style=very thick,color=red] (3.2,-0.5) -- node {} (3.2,-2.0) ; %P2 -> P3
\draw[->,>=angle 90,color=black] (5.1,-0.5) -- node {} (5.85,-2.0) ; %TV -> S
\draw[->,>=angle 90,color=black] (8.5,-0.5) -- node {} (6.605,-2.0) ; % DF -> S
\draw[->,>=angle 90,style=very thick,color=red] (3.2,-2.5) -- node {} (3.2000000000000006,-3.7) ; % P3 -> P3'
\draw[->,>=angle 90,color=black] (1.0,-0.5) -- node {} (1.490946425395748,-5.90041067935323) ; %W -> W'
\draw[->,>=angle 90,color=black] (3.0057054739713385,-4.2285817953278375) -- node {} (1.5573628100792203,-5.93251434108327) ;%P3' -> W'
\draw[->,>=angle 90,color=black] (3.2,-4.3) -- node {} (3.2,-5.914999999999999) ;%P3' -> S'
\draw[->,>=angle 90,color=red,style=very thick] (3.44,-4.18) -- node {} (5.6174,-5.82446) ;%P3' -> TV'
\draw[->,>=angle 90,color=black,color=black] (3.48,-4.099) -- node {} (8.81,-5.9685) ;%P3' -> DF'
\draw[->,>=angle 90,color=black] (5.34,-2.5) -- node {} (3.252164719493017,-5.914683869988056) ;%S -> S'
\draw[->,>=angle 90,color=black] (5.34,-2.5) -- node {} (3.252164719493017,-5.914683869988056) ;%S -> S'
\draw[->,>=angle 90,color=black] (9.6,-0.5) -- node {}  (8.92466,-5.90309);%DF -> DF'
\draw[->,>=angle 90,color=blue,style=very thick] (6.5,-0.5) -- node {} (5.90279,-5.70610) ;%TV -> TV'

\draw[color=black,fill=white] (4.8,-2.0) rectangle (6.9,-2.5);
\node at (5.85,-2.25) {\color{black}{\tiny S'44:\color{black}{3}}};

				\end{tikzpicture}
				%\caption{A gull}
	\end{textblock*}
	{
	\begin{textblock*}{0.5\textwidth}(-0.2cm,-2.4cm)
    			\setulcolor{Red}
			\begin{block}{{\ul{{\mbox{命題}} $1'$}}}
		{\tiny
		\begin{equation*}
			\kern-0.4cm
			\begin{array}[]{c}
				\displaystyle\iint\displaylimits_{\kern0.2cm [-1,1]^2}\kern-0.3cm | s - t |^{2 \nu}\kern-0.05cm (1 - s^2)^{\lambda - \frac{1}{2}}
			\kern-0.05cm(1 - t^2)^{\mu - \frac{1}{2}} C_\ell^{\lambda} (s) C_m^{\mu} (t) d s d t\\
			=\mypgf
			\end{array}
		\end{equation*}
		}
	\end{block}
	\end{textblock*}
	}
\end{frame}
\begin{frame}[fragile]
\begin{tikzpicture}
\draw[color=black] (-0.2,0.0) rectangle (2.2,-0.5);
\node at (1.0,-0.25) {\color{black}{\scriptsize Warnaar'10: \color{blue}{4}}};

\draw[color=red] (2.4,0.0) rectangle (4.4,-0.5);
\node at (3.45,-0.25) {\color{red}{\scriptsize 命題 2: \color{blue}{4}}};

\draw[color=black] (4.5,0.0) rectangle (7.7,-0.5);
\node at (6.1,-0.25) {\color{black}{\scriptsize Tarasov-Varchenko'03: \color{black}{3}}};

\draw[color=blue] (8.4,0.0) rectangle (11.35,-0.5);
\node at (9.85,-0.25) {\color{blue}{\scriptsize Dotsenko-Fateev'85: \color{blue}{4}}};

\draw[color=red] (2.1,-2.0) rectangle (4.3,-2.5);
\node at (3.2,-2.25) {\color{red}{\scriptsize 命題 1$'$: \color{blue}{3}}};

\filldraw[color=red,pattern color=red,pattern=north east lines] (3.2,-4.0) circle(0.3);
\node at (3.85,-4.0) {\color{blue}{3}};

\fill[color=black] (1.5,-6.0) circle(0.1);
\node at (1.95,-6.0) {\color{blue}{2}};

\fill[color=black] (3.2,-6.0) circle(0.1);
\node at (3.6500000000000004,-6.0) {\color{blue}{2}};

\fill[color=black] (5.85,-6.0) circle(0.1);
\node at (6.3,-6.0) {\color{blue}{2}};

\fill[color=green] (8.9,-6.0) circle(0.3);
\node at (9.35,-6.0) {\color{blue}{2}};


\draw[->,>=angle 90,color=black] (2.0,-0.5) -- node {} (5.0,-2.0) ;%War -> S
\draw[->,>=angle 90,style=very thick,color=red] (3.2,-0.5) -- node {} (3.2,-2.0) ; %P2 -> P3
\draw[->,>=angle 90,color=black] (5.1,-0.5) -- node {} (5.85,-2.0) ; %TV -> S
\draw[->,>=angle 90,color=black] (8.5,-0.5) -- node {} (6.605,-2.0) ; % DF -> S
\draw[->,>=angle 90,style=very thick,color=red] (3.2,-2.5) -- node {} (3.2000000000000006,-3.7) ; % P3 -> P3'
\draw[->,>=angle 90,color=black] (1.0,-0.5) -- node {} (1.490946425395748,-5.90041067935323) ; %W -> W'
\draw[->,>=angle 90,color=black] (3.0057054739713385,-4.2285817953278375) -- node {} (1.5573628100792203,-5.93251434108327) ;%P3' -> W'
\draw[->,>=angle 90,color=black] (3.2,-4.3) -- node {} (3.2,-5.914999999999999) ;%P3' -> S'
\draw[->,>=angle 90,color=black] (3.44,-4.18) -- node {} (5.777393604812446,-5.9452027206131675) ;%P3' -> TV'
\draw[->,>=angle 90,color=red,style=very thick] (3.48,-4.099) -- node {} (8.621,-5.9) ;%P3' -> DF'
\draw[->,>=angle 90,color=black] (5.34,-2.5) -- node {} (3.252164719493017,-5.914683869988056) ;%S -> S'
\draw[->,>=angle 90,color=blue,style=very thick] (9.6,-0.5) -- node {}  (8.9494,-5.70463);%DF -> DF'
\draw[->,>=angle 90,color=black] (6.5,-0.5) -- node {} (5.88,-5.9048) ;%TV -> TV'

\draw[color=black,fill=white] (4.8,-2.0) rectangle (6.9,-2.5);
\node at (5.85,-2.25) {\color{black}{\scriptsize Selberg'44: \color{black}{3}}};

\end{tikzpicture}
ここで{\color{blue}{青い数字}}は連続パラメータの個数を表す.
\end{frame}
\begin{frame}[fragile]%frame DF
	\scriptsize
	\begin{textblock*}{\textwidth}(1cm,0.8cm)
	  	\begin{tikzpicture}
			\draw[->,>=angle 90,color=red,style=very thick] (2.0,4.8) -- node{} (2.0,1.95);
			\node at (2.8,3.4) {\color{red}{\textnormal{ \tiny$\begin{array}[]{c}\ell=m=0,\\ \nu=-1\end{array}$}}};
		\end{tikzpicture}
	\end{textblock*}
	\begin{textblock*}{\textwidth}(3.89cm,1.7cm)
	  	\begin{tikzpicture}
			\draw[->,>=angle 90,color=blue,style=very thick] (1.0,1.0) -- node{} (-0.5,-1.0);
			\node at (-0.45,0.0) {\color{blue}{\normalfont{\tiny$\begin{array}[]{c}
				m=n=1,\\
				\alpha'=\beta',\\
				\alpha=\beta
		\end{array}$}}};
		\end{tikzpicture}
	\end{textblock*}
	\begin{textblock*}{0.5\textwidth}(-0.4cm,3.7cm)
		 {\tiny
			\begin{tcolorbox}[colback=green!10!white,colframe=green]
				\vspace{-0.5cm}
			\begin{equation*}
				\kern-0.6cm
				\begin{array}{c}
				\displaystyle\iint\displaylimits_{(t, \tau) \in [0, 1]^2} t^{\alpha'} (1 - t)^{\alpha'} \tau^{\alpha} (1
				- \tau)^{\alpha} | t - \tau |^{- 2} dtd\tau\\
			=\mypgf
				\end{array}
				\vspace{-0.3cm}
			\end{equation*}

			     \end{tcolorbox}
		}
	\end{textblock*}
	\begin{textblock*}{0.5\textwidth}(0.515\textwidth,-0.1cm)
    			\setulcolor{Blue}
		\begin{fact*}[{\ul{Dotsenko-Fateev'85}}]
			{\tiny
		\begin{equation*}
			\kern-0.25cm
			\begin{array}{c}
				\frac{1}{n!m!}
				\kern-1.45cm
				\displaystyle\bigintssss_{\mbox{\tiny$\kern1.0cm(t,\tau)\kern-0.1cm\in\kern-0.1cm [0, 1]^{k_1+k_2}$}} \kern-1.35cm
   \Pi^{\alpha', \beta'} (t)
  \Pi^{\alpha, \beta} (\tau) \Delta^{2 \rho'}(t) \Delta^{2 \rho}(\tau)
  \Delta^{- 2} (t, \tau) dtd\tau \\
				      \end{array}\end{equation*}
			\vspace{-0.5cm}
		\begin{equation*}
			\kern-0.29cm
			\begin{array}{c}
					        = \mypgf,\\
					  \mbox{ここで  }\textstyle\prod^{\alpha - 1,
					  \beta_1 - 1} (t):=\displaystyle\prod_{i=1}^{k_1}t_i^{\alpha-1}(1-t_i)^{\beta_1-1},\\
					  \Delta^{2\gamma}(t):=\displaystyle\prod_{1\le i<j\le k_1 }\kern-0.2cm\myabs{t_i}^{2\gamma}, \Delta^\gamma(t,s):=\displaystyle\prod\displaylimits_{i,j=1}^{k_1,k_2}
					  \myabs{t_i-s_j}^{\gamma}\kern-0.1cm,\\
					  \\
  \alpha' = - \rho' \alpha, \beta' = - \rho' \beta, \rho' \rho = 1,\\
  \tmop{Re} (\rho) < 0, \quad \tmop{Re} (\alpha), \tmop{Re} (\beta) > (n - 1)
  + | \tmop{Re} (\rho) | (m - 1).
			\end{array}
			\end{equation*}
			\vspace{-0.29cm}
				}
		\end{fact*}
	\end{textblock*}
	\begin{textblock*}{0.5\textwidth}(0.635\textwidth,-2.8cm)
			  \begin{tikzpicture}[scale=0.4]
				\draw[color=black] (-0.2,0.0) rectangle (2.2,-0.5);
\node at (1.0,-0.25) {\color{black}{\tiny W'10:\color{blue}{4}}};

\draw[color=red] (2.4,0.0) rectangle (4.4,-0.5);
\node at (3.45,-0.25) {\color{red}{\tiny 命題 2:\color{blue}{4}}};

\draw[color=black] (4.6,0.0) rectangle (7.6,-0.5);
\node at (6.1,-0.25) {\color{black}{\tiny T-V'03:\color{black}{3}}};

\draw[color=blue] (8.5,0.0) rectangle (11.25,-0.5);
\node at (9.85,-0.25) {\color{blue}{\tiny D-F'85:\color{blue}{4}}};

\draw[color=red] (2.1,-2.0) rectangle (4.3,-2.5);
\node at (3.2,-2.25) {\color{red}{\tiny 命題 1$'$:\color{blue}{3}}};

\filldraw[color=red,pattern color=red,pattern=north east lines] (3.2,-4.0) circle(0.3);
\node at (3.85,-4.0) {\color{blue}{3}};

\fill[color=black] (1.5,-6.0) circle(0.1);
\node at (1.95,-6.0) {\color{blue}{2}};

\fill[color=black] (3.2,-6.0) circle(0.1);
\node at (3.6500000000000004,-6.0) {\color{blue}{2}};

\fill[color=black] (5.85,-6.0) circle(0.1);
\node at (6.3,-6.0) {\color{blue}{2}};

\fill[color=green] (8.9,-6.0) circle(0.3);
\node at (9.35,-6.0) {\color{blue}{2}};


\draw[->,>=angle 90,color=black] (2.0,-0.5) -- node {} (5.0,-2.0) ;%War -> S
\draw[->,>=angle 90,style=very thick,color=red] (3.2,-0.5) -- node {} (3.2,-2.0) ; %P2 -> P3
\draw[->,>=angle 90,color=black] (5.1,-0.5) -- node {} (5.85,-2.0) ; %TV -> S
\draw[->,>=angle 90,color=black] (8.5,-0.5) -- node {} (6.605,-2.0) ; % DF -> S
\draw[->,>=angle 90,style=very thick,color=red] (3.2,-2.5) -- node {} (3.2000000000000006,-3.7) ; % P3 -> P3'
\draw[->,>=angle 90,color=black] (1.0,-0.5) -- node {} (1.490946425395748,-5.90041067935323) ; %W -> W'
\draw[->,>=angle 90,color=black] (3.0057054739713385,-4.2285817953278375) -- node {} (1.5573628100792203,-5.93251434108327) ;%P3' -> W'
\draw[->,>=angle 90,color=black] (3.2,-4.3) -- node {} (3.2,-5.914999999999999) ;%P3' -> S'
\draw[->,>=angle 90,color=black] (3.44,-4.18) -- node {} (5.777393604812446,-5.9452027206131675) ;%P3' -> TV'
\draw[->,>=angle 90,color=red,style=very thick] (3.48,-4.099) -- node {} (8.621,-5.9) ;%P3' -> DF'
\draw[->,>=angle 90,color=black] (5.34,-2.5) -- node {} (3.252164719493017,-5.914683869988056) ;%S -> S'
\draw[->,>=angle 90,color=blue,style=very thick] (9.6,-0.5) -- node {}  (8.9494,-5.70463);%DF -> DF'
\draw[->,>=angle 90,color=black] (6.5,-0.5) -- node {} (5.88,-5.9048) ;%TV -> TV'

\draw[color=black,fill=white] (4.8,-2.0) rectangle (6.9,-2.5);
\node at (5.85,-2.25) {\color{black}{\tiny S'44:\color{black}{3}}};

				\end{tikzpicture}
				%\caption{A gull}
	\end{textblock*}
	{
	\begin{textblock*}{0.5\textwidth}(-0.2cm,-2.4cm)
    			\setulcolor{Red}
			\begin{block}{{\ul{{\mbox{命題}} $1'$}}}
		{\tiny
		\begin{equation*}
			\kern-0.4cm
			\begin{array}[]{c}
				\displaystyle\iint\displaylimits_{\kern0.2cm [-1,1]^2}\kern-0.3cm | s - t |^{2 \nu}\kern-0.05cm (1 - s^2)^{\lambda - \frac{1}{2}}
			\kern-0.05cm(1 - t^2)^{\mu - \frac{1}{2}} C_\ell^{\lambda} (s) C_m^{\mu} (t) d s d t\\
			=\mypgf
			\end{array}
		\end{equation*}
		}
	\end{block}
	\end{textblock*}
	}
\end{frame}
\begin{frame}
	更に、系\ref{cor:int-xzy-hh}の積分に$w=s-t$変数変化し、$z=1$をとれば、
	次の公式と同等な公式(正確には、そのMellin変換)が得られる.
	\begin{fact*}[{\cite[(18)]{conte1994hermite}}]
		以下のような公式が成り立つ:
		\begin{equation}
			w_{\lambda,n}\ast w_{\lambda,k}(t)=\sqrt{\frac{(n+k)!}{2^{n+k}n!k!}}w_{\sqrt{2}\lambda,n+k}(t).
			\label{eqn:mellin-hh}
		\end{equation}
	\end{fact*}
	ここで、
	記号$w_{\lambda,n}$は\underline{Hermite-Rodriguez} 関数 \cite{yusoff2007application}
	\begin{equation*}
		w_{\lambda,n}:=\frac{1}{\sqrt{2^nn!}}H_n\left( \frac{t}{\lambda} \right)\frac{1}{\sqrt{\pi}\lambda}e^{-t^2/\lambda^2},\quad \lambda>0
	\end{equation*}
	を表す.
	%%Applying inverse Mellin transform, we arrive at the well-known formula for convolution of Hermite-Rodriguez functions.
\end{frame}
\section{証明}
\begin{frame}
	以下の述べられているメソッド1とメソッド3は最初のステップとして以下のような補題を使う:
	\begin{lemma*}[$\ell=m=0$場合に帰着]
		\begin{equation*}
			\begin{array}[]{c}
				\mbox{$\ell=m=0$特殊な場合に命題\ref{prop:int-st-gg}が成り立つ}\\[0.4cm]
				\Downarrow\\[0.4cm]
				\mbox{命題\ref{prop:int-st-gg}の一般な場合が成り立つ.}
			\end{array}
		\end{equation*}
	\end{lemma*}
	\vspace{-0.2cm}
	\begin{proof}\renewcommand{\qedsymbol}{}
		Gegenbauer多項式のロドリゲスの公式
		{\scriptsize\vspace{-0.2cm}\begin{equation*}
				(1-x^2)^{\alpha-\frac{1}{2}}C_n^\alpha(x)=a(\alpha,n)
				\frac{d^n}{dx^n} (1-x^2)^{n+\alpha-\frac{1}{2}},\quad a(\alpha,n)\mbox{は定数}
			\vspace{-0.2cm}\end{equation*}}
		と部分積分(integration by parts)
%%		によって、
%%				{\scriptsize\begin{equation}
%%			\vspace{-0.2cm}
%%					\begin{array}[]{c}
%%						\mbox{左辺}\eqref{eqn:int-st-gg}/a(\lambda,l)/a(\mu,m)\\
%%					=\iint\displaylimits_{[-1,1]^2}
%%					| s - t |^{2 \nu} \frac{\partial^l}{\partial s^l}(1-s^2)^{\lambda+l-\frac{1}{2}}\frac{\partial^m}{\partial t^m}(1-t^2)^{\mu+m-\frac{1}{2}}
%%					\\=(-1)^m(-2\nu)_{l+m}\iint\displaylimits_{[-1,1]^2}\myabs{s-t}^{2\nu-l-m}(1-s^2)^{\lambda+l-\frac{1}{2}}(1-t^2)^{\mu+m-\frac{1}{2}}.
%%					\end{array}
%%					\label{eqn:m1-1}
%%				\end{equation}}
%%				\vspace{-0.4cm}
		を用いる.
	\end{proof}
	命題\ref{prop:exp-stz-gg}の証明においても同じアイディアが適用できる.
%%	命題\ref{prop:exp-stz-gg}の積分公式の形に対して、このよう補題も成り立つ.
\end{frame}
\begin{frame}{メソッド 1 (直接)}
	\scriptsize
	\begin{enumerate}
		\item $\ell=m=0$ 場合を扱えばよい;
		\item 
			\begin{equation*}
				\kern-1cm
				\begin{array}[]{c}
					\mbox{
%%						\begin{tcolorbox}[colback=green!10!white,colframe=green]
							超幾何関数のオイラー積分表示
%%						\end{tcolorbox}
					}\\
					{}_2F_1\left(\begin{array}[]{c}
						a,b\\c
					\end{array};z  \right)=\frac{\Gamma(c)}{\Gamma(b)\Gamma(c-b)}\textstyle\int_0^1x^{b-1}(1-x)^{c-b-1}(1-zx)^{-a}dx,\quad\Re(c)>\Re(b)>0\\[0.2cm]
					\mybigplus\\[0.2cm]
					\mbox{超幾何関数${}_2F_1$の係数展開}\\[0.2cm]
					\scalebox{2.0}{$=$}\quad\mbox{求める積分を ${}_3F_2$ で表す}
				\end{array}
			\end{equation*}
%%			\begin{equation*}
%%				\begin{array}[]{c}
%%					\mbox{オイラー積分表示}\\[0.2cm]
%%					\mybigplus\\[0.2cm]
%%					\mbox{二次変換:\footnote{is 二次変換 correct for "quadratic transformation"?}}\quad {}_2 F_1 \left( \begin{array}{c}
%%						  1 - a, b\\
%%						    2 b
%%					    \end{array} ; u \right) = \left( 1 - \frac{z}{2} \right)^{a - 1} {}_2 F_1 \left(
%%					    \begin{array}{c}
%%						      \frac{1 - a}{2}, \frac{2 - a}{2}\\
%%						        b + \frac{1}{2}
%%						\end{array} ; \left( \frac{u}{2 - u} \right)^2 \right)\\[0.2cm]
%%				\mybigplus\\
%%				\end{array}
%%			\end{equation*}
			%%
%%			によって、
%%			{\tiny
%%			\begin{equation*}
%%			\vspace{-0.2cm}
%%				{\mbox{右辺(補題)}}/{(-1)^m/(-2\nu)_{l+m}}/2^{\lambda+l-\frac{1}{2}}
%%			\end{equation*}
%%			\vspace{-0.2cm}
%%			\begin{equation}
%%				\kern-1.2cm
%%				 =B \left( 2\nu-l-m+1, \lambda+l-\frac{1}{2} \right)
%%				\kern-0.2cm\int_{-1}^1 (1 - t)^{2 \nu+\lambda-m+\frac{3}{2}} {}_2 F_1 \left( \kern-0.2cm\begin{array}{c}
%%					\frac{1}{2}-l-\lambda, 2\nu-l-m+1\\
%%					2 \nu+\lambda-m+\frac{3}{2}
%%				\end{array}\kern-0.2cm ; \frac{1 - t}{2} \right)(1-t^2)^{\mu+m-\frac{1}{2}}dt
%%				\label{eqn:m1-2}
%%			\end{equation}
%%		}
%%
%%%%			via \cite[ET II 186(9)]{gradshteinryzhik}:
%%%%		\begin{equation*}
%%%%			\kern-1cm\int_0^u x^{\nu - 1} (x + \alpha)^{\lambda} (u - x)^{\mu - 1} d x =
%%%%			\alpha^{\lambda} u^{\mu + \nu - 1} B (\mu, \nu)_2 F_1 \left( - \lambda, \nu ;
%%%%			\mu + \nu ; - \frac{u}{\alpha} \right);
%%%%		\end{equation*}
%%			\vspace{-0.5cm}
%%	\item  ${}_2 F_1 \left( \kern-0.2cm\begin{array}{c}
%%					\frac{1}{2}-l-\lambda, 2\nu-l-m+1\\
%%					2 \nu+\lambda-m+\frac{3}{2}
%%				\end{array}\kern-0.2cm ; \frac{1 - t}{2} \right)$ 
%%				の展開とベータ積分$\int_0^1t^{x-1}(1-t)^{y-1}dt=\Gamma(x)\Gamma(y)/\Gamma(x+y)$によって、
%%				$\mbox{右辺}\eqref{eqn:m1-2}$と
%%			を用い、次に級数展開を用いて、求める積分を${}_3F_2$で表す;
%%				${}_3F_2(;1)$超幾何関数比例であるということが分かる;
		\item 
{
				\begin{fact*}[Whipple sum]
			\begin{equation*}
			\begin{array}[]{c}
			\left\{  \begin{array}[]{l}
				a+b=1,\\2c+1=d+e
			\end{array}\right.
			\mbox{ならば、 }\\[0.4cm]
				_{3}F_2\left( \begin{array}[]{c}
					a,b,c\\d,e
				\end{array};1\right)={\mypgf}.
			\end{array}
		\end{equation*}
				\end{fact*}
			}
		Whipple Sumによって、超幾何関数
				${}_3F_2(;1)$をガンマ関数の積で表せる.
	\end{enumerate}
	\begin{textblock*}{0.6\textwidth}(0.4\textwidth,-7.8cm)
		\tiny
		\begin{tcolorbox}[colback=green!10!white,colframe=green]
			\vspace{-0.4cm}
		\begin{equation*}
			\kern-0.4cm
			\displaystyle\iint_{[-1,1]^2}\myabs{s-t}^{2\nu}(1-s^2)^{\lambda-\frac{1}{2}}(1-t^2)^{\mu-\frac{1}{2}}=\mypgf
			\vspace{-0.2cm}
		\end{equation*}
		\end{tcolorbox}
	\end{textblock*}
	\begin{textblock*}{0.55\textwidth}(0.51\textwidth,-5.6cm)
		\tiny
		\begin{tcolorbox}[colback=green!10!white,colframe=green]
			\vspace{-0.7cm}
		\begin{equation*}
			\kern-0.45cm
			\textstyle\int_{-1}^1 (1 + t)^{\mu-\frac{1}{2}} {}_2 F_1 \kern-0.1cm\left( \kern-0.2cm\begin{array}{c}
				\frac{1}{2}-\lambda, \lambda+\frac{1}{2}\\
					2 \nu+\lambda+\frac{3}{2}
				\end{array}\kern-0.2cm ; \frac{1 - t}{2} \right)(1-t)^{2\nu+\lambda+\mu}
			\vspace{-0.35cm}
		\end{equation*}
		\end{tcolorbox}
	\end{textblock*}
	\begin{textblock*}{0.58\textwidth}(0.45\textwidth,-3.6cm)
		\tiny
		\begin{tcolorbox}[colback=green!10!white,colframe=green]
			\vspace{-0.7cm}
		\begin{equation*}
			\kern-0.4cm
			{}_3F_2\left( \begin{array}[]{c}
				\frac{1}{2}-\lambda, \lambda+\frac{1}{2},2\nu+\lambda+\mu+1\\
				2 \nu+\lambda+\frac{3}{2},2\nu+\lambda+2\mu+\frac{1}{2}
			\end{array};1\right)=\mypgf
			\vspace{-0.2cm}
		\end{equation*}
		\end{tcolorbox}
	\end{textblock*}
\end{frame}
%%\begin{frame}{メソッド 1 (direct) (2/2)}
%%	\scriptsize
%%	$\cdots$
%%	\begin{enumerate}
%%			\setcounter{enumi}{2}
%%	\item Expanding the ${}_2 F_1 \left( \kern-0.2cm\begin{array}{c}
%%					\frac{1}{2}-l-\lambda, 2\nu-l-m+1\\
%%					2 \nu+\lambda-m+\frac{3}{2}
%%				\end{array}\kern-0.2cm ; \frac{1 - t}{2} \right)$ in series and using the beta integral, we see that $\mbox{RHS}\eqref{eqn:m1-2}$
%%				is proportional to hypergeometric ${}_3F_2(;1)$;
%%			\item Finally, hypergeometric ${}_3F_2(;1)$ is expressed as ratio of Gamma functions via the Whipple's sum
%%			\vspace{-0.2cm}
%%			\begin{equation*}
%%		\kern-1cm{}_{3}F_{2}\left({a,1-a,c\atop d,2c-d+1};1\right)=\frac{\pi\Gamma\left(d%
%%				\right)\Gamma\left(2c-d+1\right)2^{1-2c}}{\Gamma\left(c+\frac{1}{2}(a-d+1)%
%%				\right)\Gamma\left(c+1-\frac{1}{2}(a+d)\right)\Gamma\left(\frac{1}{2}(a+d)%
%%			\right)\Gamma\left(\frac{1}{2}(d-a+1)\right)},
%%		\end{equation*}
%%	\end{enumerate}
%%	\begin{remark}
%%		Note that Whipple's sum requires the parameters of hypergeometric ${}_3F_2(;1)$ to satisfy certain relations (in fact, out of 5 parameters only 3 are {\bf free}).
%%		The general expression \begin{equation*}
%%			{}_3F_2\left( \begin{array}[]{c}
%%				a,b,c\\d,e
%%			\end{array};1 \right)
%%		\end{equation*}
%%		with 5 free parameters {\bf cannot} be expressed by Gamma functions \cite{ebisu2016three}.
%%	\end{remark}
%%\end{frame}
\begin{frame}{メソッド 2 (\only<1>{1}\only<2>{2}/2) cf. セルバーグ積分の証明}
			\vspace{-0.2cm}
	\begin{fact*}[Carlsonの定理]
		$f$が$\left\{ z\in\mathbb{C}\mid \Re(z)\ge0 \right\}$右半平面上で連続で、
		内部$\left\{ z\in\mathbb{C}\mid\Re(z)>0 \right\}$上に\footnote{should I change に to でhere as well?}正則であるとする.
		更に、
		\begin{enumerate}
			\item ある$C,\tau>0$に対して、右半平面上に$\myabs{f(z)}\le C e^{\tau\myabs{z}}$が成り立つ;
			\item ある$ C>0,\exists c<\pi$と全ての$y\in\R$に対して、$\myabs{f(iy)}\le Ce^{c\myabs{y}}$が成り立つ;
			\item $f$が正の整数上で、ゼロ ($f\mid_{\N_+}=0$)にならば、
		\end{enumerate}
		$f\equiv0$である.
	\end{fact*}
	\only<1>{
			\vspace{-0.2cm}
	\begin{remark}
		\begin{equation*}
			\vspace{-0.2cm}
			f(z)=\sin(\pi z)
		\end{equation*}
		という関数が条件\mytextcircled{2}を除いで、Carlsonの定理の全ての条件を満たす.特に、
		$f\big|_{\Z}=0$が成り立つ.しかし、 $f(z)$が恒等的に0ならない。

		よって、Carlsonの定理の条件\mytextcircled{2}が必要である.
	\end{remark}
	}
	\only<2>{
	これを用いて、公式\eqref{eqn:int-st-gg} 
			\vspace{-0.2cm}
			{\scriptsize
		\begin{equation*}
			\vspace{-0.2cm}
			\int_{- 1}^1 \int_{- 1}^1 | s - t |^{2 \nu} (1 - s^2)^{\lambda - \frac{1}{2}}
			(1 - t^2)^{\mu - \frac{1}{2}} C_\ell^{\lambda} (s) C_m^{\mu} (t) d s d t=\mypgf
		\end{equation*}
	}
	で$\nu\in\N_{+}$
	のときは$\myabs{s-t}^{2\nu}$が多項式になるので、命題\ref{prop:int-st-gg}
	は計算可能.
	}
\end{frame}
\begin{frame}{メソッド 3 (命題\ref{prop:exp-stz-gg}によって)}
	\scriptsize
	命題\ref{prop:exp-stz-gg}を示すため、まず、$\ell=m=0$場合に帰着する
			{\scriptsize \begin{equation}
				\kern-0.2cm 2\kern-0.2cm\iint\displaylimits_{[-1,1]^2} \left( s-tz \right)_+^{2c-1}  (1 - s^2)^{a - 1} (1 -
				t^2)^{b - 1} d s d t = \frac{\sqrt{\pi} \Gamma (a) \Gamma (b) \Gamma
			(c)}{\Gamma (a + c) \Gamma \left( b + \frac{1}{2} \right)} {}_2 F_1 \left(
			\begin{array}{c}
				  - c + \frac{1}{2}, - a - c + 1\\
				    b + \frac{1}{2}
			    \end{array} ; z^2 \right) .
				\label{eqn:prop21}
			\end{equation}}
			この公式が以下のものから従う:
			\vspace{-0.4cm}
			\begin{equation*}
				\begin{array}[]{c}
					\mbox{
					\begin{tcolorbox}[colback=red!10!white,colframe=red,width=0.28\textwidth,height=3em]
						\vspace{-0.15cm}
							オイラー積分表示
					\end{tcolorbox}
						\vspace{-0.15cm}
				}\\[0.2cm]
					\mybigplus\\[0.2cm]
					\mbox{
					\begin{tcolorbox}[colback=green!10!white,colframe=green,width=0.9\textwidth,height=2cm]
						\vspace{-0.2cm}
						\begin{equation*}
						\vspace{-0.2cm}
					\mbox{二次変換:}\quad {}_2 F_1 \left( \begin{array}{c}
						  1 - a, b\\
						    2 b
					    \end{array} ; u \right) = \left( 1 - \frac{z}{2} \right)^{a - 1} {}_2 F_1 \left(
					    \begin{array}{c}
						      \frac{1 - a}{2}, \frac{2 - a}{2}\\
						        b + \frac{1}{2}
						\end{array} ; \left( \frac{u}{2 - u} \right)^2 \right)
						\end{equation*}
					\end{tcolorbox}}\\[0.2cm]
				\mybigplus\\
				\end{array}
			\end{equation*}
%%	\begin{enumerate}
%%		\item Via the variable change $s=(1-tz)(1-w)+tz$ and Euler's integral representation of ${}_2F_1$, 
%%				The inner integral (with respect to $s$) is done as:
%%				\begin{equation*}
%%				\int_{- 1}^1 (s - tz)_+^{2 c - 1} (1 - s^2)^{a - 1} d s = 2^{a - 1} B (2 c, a)
%%				(1 - z)^{2 c + a} {}_2 F_1 \left( \begin{array}{c}
%%					  1 - a, 2 c\\
%%					    2 c + a
%%				    \end{array} ; \frac{1 - tz}{2} \right)
%%			\end{equation*}
%%		\item \label{enum:m4-1}We then expand ${}_2F_1(;\frac{1-tz}{2})$ in series and take the integral with respect to $t$ by the formula\begin{equation*}
%%				\int_{- 1}^1 (1 - t z)^{a - 1} (1 - t^2)^{b - 1} d t = B \left( \frac{1}{2},
%%				b \right) {}_2 F_1 \left( \begin{array}{c}
%%					  \frac{1 - a}{2}, \frac{2 - a}{2}\\
%%					    b + \frac{1}{2}
%%				    \end{array} ; z^2 \right)
%%			\end{equation*}
%%		\item Finally, apply the summation formula
%%				\vspace{-0.3cm}
%%			\begin{equation*}
%%				\kern-1cm\sum_{i = 0}^{\infty} \frac{(a)_i (1 - a)_i}{2^i i! (d)_i} {}_2 F_1 \left(
%%				\begin{array}{c}
%%					  \frac{1 - d - i}{2}, \frac{2 - d - i}{2}\\
%%					    b + \frac{1}{2}
%%				    \end{array} ; \zeta \right) = 
%%				    \overbrace{\sum_{j = 0}^{\infty} \frac{(1 - d)_{2 j} \zeta^j}{2^{2 j} j! \left( b +
%%				    \frac{1}{2} \right)_j}}^{\kern-0.2cm\scalebox{1}{$={}_2F_1(;\zeta)$}} \overbrace{{}_2 F_1 \left( \begin{array}{c}
%%					      a, 1 - a\\
%%					        d - 2 j
%%					\end{array} ; \frac{1}{2} \right) }^{\mbox{\scriptsize expressed as ratio of $\Gamma$-functions}}
%%%%				    =\frac{2^{1 - d} \sqrt{\pi} \Gamma (d)}{\Gamma
%%%%					    \left( \frac{a + d}{2} \right) \Gamma \left( \frac{1 - a + d}{2} \right)} {}_2
%%%%					    F_1 \left( \begin{array}{c}
%%%%						      1 - \frac{a + d}{2}, \frac{1 + a - d}{2}\\
%%%%						        b + \frac{1}{2}
%%%%						\end{array} ; \zeta \right) .
%%			\end{equation*}
%%%%		\item In turn, \eqref{eqn:prop21} is proven via the equality {\scriptsize \begin{equation}
%%%%				\kern-1.5cm \int_{-1}^1(1-t z)^{a-1}(1-t^2)^{b-1}d t=B\left(\frac{1}{2},b  \right)\ {}_2F_1\left( 
%%%%				\begin{array}[]{c}
%%%%					\frac{1-a}{2},\frac{2-a}{2}\\b+\frac{1}{2}
%%%%				\end{array}
%%%%				;z^2 \right),
%%%%				\label{eqn:lem31}
%%%%			\end{equation}}
%%%%			Euler's integral and the transformation formula for $_2F_1$ hypergeometric which we derived.
%%%%		\item \begin{equation}
%%%%				\sum_{i = 0}^{\infty} \frac{(a)_i (1 - a)_i}{2^i i! (d)_i} _2 F_1 \left(
%%%%				\begin{array}{c}
%%%%					  \frac{1 - d - i}{2}, \frac{2 - d - i}{2}\\
%%%%					    b + \frac{1}{2}
%%%%				    \end{array} ; \zeta \right) = \frac{2^{1 - d} \sqrt{\pi} \Gamma (d)}{\Gamma
%%%%					    \left( \frac{a + d}{2} \right) \Gamma \left( \frac{1 - a + d}{2} \right)} _2
%%%%					    F_1 \left( \begin{array}{c}
%%%%						      1 - \frac{a + d}{2}, \frac{1 + a - d}{2}\\
%%%%						        b + \frac{1}{2}
%%%%						\end{array} ; \zeta \right) .
%%%%						\label{eqn:lem32}
%%%%			\end{equation}
%%	\end{enumerate}
			\vspace{-0.4cm}
			\begin{lemma*}
			\begin{equation*}
				\sum_{i = 0}^{\infty} \frac{(a)_i (1 - a)_i}{2^i i! (d)_i} {}_2 F_1 \left(
				\begin{array}{c}
					  \frac{1 - d - i}{2}, \frac{2 - d - i}{2}\\
					    b + \frac{1}{2}
				    \end{array} ; \zeta \right) = 
				    \overbrace{\sum_{j = 0}^{\infty} \frac{(1 - d)_{2 j} \zeta^j}{2^{2 j} j! \left( b +
				    \frac{1}{2} \right)_j}}^{\kern-0.2cm\scalebox{1}{$={}_2F_1(;\zeta)$}} \overbrace{{}_2 F_1 \left( \begin{array}{c}
					      a, 1 - a\\
					        d - 2 j
					\end{array} ; \frac{1}{2} \right) }^{\mbox{\scriptsize \begin{tabular}[]{c}
					ガンマ関数の\\積公式で表せる
				\end{tabular}}}
				\end{equation*}
			\end{lemma*}
\end{frame}
\begin{frame}{メソッド 4 (積分公式とルジャンドル関数$\kern-0.1cm P^\alpha_\beta\kern-0.1cm$を用いる}
	\scriptsize
	直接命題\ref{prop:int-st-gg}を示す:
	\begin{enumerate}
		\item まず、\cite[7.4.11]{kobayashi2011schrodinger}の公式
			{
				\scriptsize
			\begin{equation*}
				\kern-1cm\int_{-s}^1(s+t)^{2\nu} {C}_m^\mu(t)(1-t^2)^{\mu-\frac{1}{2}}dt=
				\frac{\sqrt{\pi}\Gamma(2\mu+m)\Gamma(2\nu+1)}{\Gamma(\mu)2^{\mu-\frac{1}{2}}m!}
				(1-s^2)^{\nu+\frac{\mu}{2}+\frac{1}{4}}P_{\mu+m-\frac{1}{2}}^{-2\nu-\mu-\frac{1}{2}}(-s).
			\end{equation*}
		}
		を用いる(ここで、$P_{\mu+m-\frac{1}{2}}^{-2\nu-\mu-\frac{1}{2}}(-s)$がルジャンドル陪関数である);
		\item 次の段階は以下の積分に帰着できる
			\begin{equation}\label{eqn:m4-1}
				\int_{-1}^1(1-s^2)^{\nu+\frac{\mu}{2}+\lambda-\frac{1}{4}}P_{\mu+m-\frac{1}{2}}^{-2\nu-\mu-\frac{1}{2}}(-s)C^{\lambda-\frac{1}{2}}_\ell(s)ds;
			\end{equation}
			\vspace{-0.4cm}
		\item 部分積分と公式
		%Use integration by parts together with the formul\ae
			\begin{equation*}
				\left(C^\lambda_\ell(s)  \right)'=C^{\lambda+1}_{\ell-1}(s),\quad\left((1-s^2)^{-a/2}P^a_b(s) \right)'
			=-(1-s^2)^{-\frac{a+1}{2}}P_b^{a+1}(s)
			\end{equation*}
			を用いて、\eqref{eqn:m4-1}を以下の公式までに簡易化する.
%%			to reduce \eqref{eqn:m4-1} to the form
			\begin{equation}\label{eqn:m4-2}
				\int_{-1}^1(1-s^2)^aP^b_c(s)ds;
			\end{equation}
			\vspace{-0.4cm}
		\item \eqref{eqn:m4-2} が \cite[L2]{kobayashi2011schrodinger}積分公式から従う.
	\end{enumerate}
\end{frame}
\section{表現論の対称性破れ作用素に対して応用}
\newcommand{\mysbo}{A:\pi\kern-0.1cm\mid_{G'}\xrightarrow{G'}\tau}
{
	\begin{frame}{表現論の対称性破れ作用素に対して応用}
	\begin{block}{設定}
		
	\centerline{
		\xymatrixcolsep{0.5pc}
		\xymatrixrowsep{1pc}
		\xymatrix{G\ar@/_1pc/[d]&&\supset&&G'\ar@/^1pc/[d]&\\
			\pi&\ar[rr]^{A}&&\tau&&{\begin{array}[]{l}
				\\
		\mbox{ \underline{ノンコンパクト}群$G,G'$の}\\
		\mbox{ :\underline{無限次元}表現.}\\
		\end{array}}%
	}}
	\end{block}
	\setbeamercolor{block title}{bg=red!30,fg=black}
	\hspace{1cm}
	\vspace{-0.5cm}
	\begin{block}{目標}
%%		\[\begin{array}{cccl}
%%				G:=O(p+1,q+1)&\curvearrowright &\pi\mbox{ :infinitely-dimensional rep.}\\
%%		\bigcup&&&\\
%%		G':=O(p,q+1)&\curvearrowright &\tau\mbox{ :infinitely-dimensional rep.}
%%		\end{array}\]
		$G'$-intertwining作用素$\mysbo$を全て構成し、分類し、さらにその性質を詳しく調べる.
	\end{block}
\end{frame}
\begin{frame}{アイディア}
	\begin{itemize}%[<+(1)->]
		\item $\pi$と$\tau$は異なる表現であるが、例えば$(K,K')$-重複度が1の場合に一般化した意味での固有値を定義することができる;
		%\item Although $\pi$ and $\tau$ are realized on different spaces, we can define ``generalized eigenvalues'';
		\item $\mysbo$ はある意味で``対角化''できる(Schurの補題の考え方);
		\item $\implies$ 固有値を具体的に計算したい.
	\end{itemize}
\end{frame}
}
\begin{frame}{$(G,G')=\left( O(p,q),O(p-1,q) \right)$であるとする}
\begin{center}
	\begin{tikzpicture}[rotate=-90]
	\draw[color=black] (-9.0,0.75) rectangle (-6.0,-6.05);%upper
\draw[color=black] (-8.75,0.25) rectangle (-6.25,-1.25);%right in upper
\draw[color=black] (-8.5,0.0) rectangle (-6.5,-0.5);%in ``right in upper''
\draw[color=black] (-8.75,-1.75) rectangle (-6.25,-3.75);%middle in upper
\draw[color=black] (-8.5,-1.825) rectangle (-6.5,-2.325);%right in ``middle in upper''
\draw[color=black] (-8.5,-2.4) rectangle (-6.5,-2.9);%middle in ``middle in upper''
\draw[color=black] (-8.5,-2.975) rectangle (-6.5,-3.475);%left in ``middle in upper''
\draw[color=black] (-8.75,-4.5) rectangle (-6.25,-6.0);%left in upper
\draw[color=black] (-8.5,-4.85) rectangle (-6.5,-5.35);%right in ``left in upper''
\draw[color=black] (-8.5,-5.475) rectangle (-6.5,-5.975); %left in ``left in upper''

\draw[color=black] (-5.0,0.75) rectangle (-2.0,-6.05);%lower
\draw[color=black] (-4.75,0.25) rectangle (-2.25,-1.25);%right in lower
\draw[color=black] (-4.75,-4.5) rectangle (-2.25,-6.0);%left in lower

\node at (-7.25,-4.0) {\color{black}{\Huge \dots}};
\node at (-7.25,-4.7) {\color{black}{\Huge \dots}};
\node at (-2.25,-3.0) {\color{black}{\Huge \dots}};

%%\node at (-7.0,1.0) {\color{black}{$I(\lambda)$}};
%%\node at (-2.0,1.0) {\color{black}{$J(\nu)$}};

\draw[-open triangle 90] (-6.0,-2.65) to node {$R_{\lambda,\nu}^X$} (-5.0,-2.65);
\draw[-open triangle 90] (-5.5,-0.25) to node {$c^{p,q}_{0,0,0}I$} (-1.75,-0.5);
\draw[-open triangle 90] (-5.5,-2.075) to node {$c^{p,q}_{0,2,0}I$} (-1.75,-0.5);
\draw[-open triangle 90] (-6.5,-5.725) to node {$c^{p,q}_{m',m,n}I$} (-4.75,-5.25);

	\end{tikzpicture}
\end{center}
\end{frame}
\begin{frame}{アイディア}
	\begin{itemize}%[<+(1)->]
		\item \ldots
		\item $\implies$ 固有値を具体的に計算してほしい
		\item $\mysbo$が積分核に決まる;
		\item $\implies$ 固有値が積分公式で表示できる;
		\item $\implies$ 命題\kern-0.1cm\ref{prop:int-st-gg}.
	\end{itemize}
\end{frame}
\appendix
\begin{frame}[allowframebreaks]{References}
	\bibliographystyle{apalike}
	\nocite{Selberg:411367}
	\nocite{warnaar2010sl3}
	\nocite{dotsenko1985four}
	\nocite{tarasov2003selberg}
%%	\nocite{kobayashi2015symmetry}
%%	\nocite{kobayashi2015differential2}
%%	\nocite{kobayashi2016differential1}
%%	\nocite{kobayashi2014classification}
%%	\nocite{kobayashi2013finite}
%%	\nocite{kobayashi2015program}
\bibliography{intdep}
\end{frame}
\addtocounter{framenumber}{-3}
\begin{frame}[noframenumbering]{Gegenbauer多項式のロドリゲスの公式}
			\begin{equation*}
				(1-x^2)^{\alpha-\frac{1}{2}}C_n^\alpha(x)=a(\alpha,n)
				\frac{d^n}{dx^n} (1-x^2)^{n+\alpha-\frac{1}{2}},
			\end{equation*}
			\begin{equation*}
				a(\alpha,n)=\frac{(-1)^n}{2^nn!}\frac{\Gamma\left( \alpha+\frac{1}{2} \right)\Gamma\left( n+2\alpha \right)}{\Gamma(2\alpha)\Gamma\left(\alpha+n+\frac{1}{2}  \right)}
			\end{equation*}
\end{frame}
\begin{frame}[noframenumbering]{ンパクト群 O(n) の分規則 ルール}
	\begin{equation*}
		\mathcal{H}^L(\mathbb{S}^n)=\bigoplus_{N=0}^L \mathcal{H}^N(\mathbb{S}^{n-1}),
	\end{equation*}
	\begin{equation*}
		I_{N\to L}:\mathcal{H}^N(\mathbb{S}^{n-1})\to \mathcal{H}^L(\mathbb{S}^n),
	\end{equation*}
	\begin{equation*}
		\left(I_{N\to L}\varphi  \right)(\eta_0,\eta)=\myabs{\eta}^N\varphi\left( \frac{\eta}{\myabs{\eta}} \right)\tilde{\tilde{C}}^{\frac{n-1}{2}+N}_{L-N}(\eta_0).
	\end{equation*}
\end{frame}
\begin{frame}[noframenumbering]{命題\ref{prop:exp-stz-gg}$\implies$命題\ref{prop:exp-st-gg}}
\begin{equation*}
		{}_2F_1\left( \begin{array}[]{c}
			a,b\\c
		\end{array};1\right)=\frac{\Gamma(c-a-b)\Gamma(c)}{\Gamma(c-a)\Gamma(c-b)},\quad \Re(c-a-b)>0
	\end{equation*}
\end{frame}
\begin{frame}[noframenumbering]{超幾何関数に関数る等式}
	\[F\left(\begin{array}[]{c}
		a,1-a\\b
	\end{array}\tfrac{1}{2}\right)=\frac{2^{1-b}\sqrt{\pi}\Gamma\left(b\right)%
}{\Gamma\left(\tfrac{1}{2}a+\tfrac{1}{2}b\right)\Gamma\left(\tfrac{1}{2}b-%
\tfrac{1}{2}a+\tfrac{1}{2}\right)}.\]
\end{frame}
\end{document}
