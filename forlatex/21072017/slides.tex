\documentclass[pdf,notes]{beamer}
\mode<presentation>{\usetheme[secheader]{Boadilla}}

\usepackage{mathtools}
\usepackage{xypic}
\usepackage[all,cmtip]{xy}
\usepackage{mystyle}
\includecomment{versiona}
\usepackage{xeCJK}
\usepackage{lpic}
\usefonttheme{professionalfonts}
\usepackage{etoolbox}

\newcommand{\red}[1]{{\color[rgb]{0.6,0,0}#1}}
\setbeamertemplate{theorems}[numbered] 

\usepackage{tikz}
\usetikzlibrary{shapes,arrows,patterns}
\renewcommand{\implies}{\Rightarrow}
\newcommand{\Sol}{\mathcal{S}\mbox{ol}}
\newcommand{\Ind}{\mbox{\normalfont Ind}}
\newcommand{\Hom}{\mbox{\normalfont Hom}}
\newcommand{\D}{\mathcal{D}}
\newcommand{\A}{\mathcal{A}}
\newcommand{\Co}{\mathbb{C}}
\newcommand{\X}{\mathbb{X}}
\renewcommand{\setminus}{\backslash}
\newcommand{\nin}{\not\in}
\newcommand{\tmop}[1]{\ensuremath{\operatorname{#1}}}
\newcommand{\tmtextbf}[1]{{\bfseries{#1}}}
\newcommand{\tmtextit}[1]{{\itshape{#1}}}
\newcommand{\mss}{//}
\newcommand{\mbb}{\backslash\backslash}
\newcommand{\mmm}{\mid\mid}
\catcode`\<=\active \def<{
\fontencoding{T1}\selectfont\symbol{60}\fontencoding{\encodingdefault}}
\catcode`\>=\active \def>{
\fontencoding{T1}\selectfont\symbol{62}\fontencoding{\encodingdefault}}
\newcommand{\assign}{:=}
\newcommand{\comma}{{,}}
\newcommand{\um}{-}
\newcommand{\sol}{\mathcal{S}ol(\R^{p,q};\lambda,\nu)}
\newcommand{\Op}{\mbox{\normalfont Op}}
\newcommand{\Res}{\operatorname{Res}\displaylimits}
\newcommand{\OpR}{\mbox{\it R}}

%%\makeatletter
%%\newenvironment<>{proofs}[1][\proofname]{\par\def\insertproofname{#1\@addpunct{.}}\usebeamertemplate{proof begin}#2}
%%{\usebeamertemplate{proof end}}
%%\makeatother

\newtheorem{prop}{命題}
\newtheorem*{prop*}{命題}
\newtheorem{remark}{注}
\newtheorem{cor}{系}

\AtBeginEnvironment{remark}{%
	    \setbeamercolor{block title}{bg=orange,fg=white}
		\setbeamercolor{block body}{bg=orange!20,fg=black}
}
\AtBeginEnvironment{fact}{%
	    \setbeamercolor{block title}{bg=green,fg=black}
		\setbeamercolor{block body}{bg=orange!20,fg=black}
}
\AtBeginEnvironment{cor}{%
	    \setbeamercolor{block title}{bg=blue,fg=white}
		\setbeamercolor{block body}{bg=orange!20,fg=black}
}

\setCJKmainfont[AutoFakeBold=true]{Hiragino Mincho Pro} %my Mac

\title[2つのゲーゲンバウアー多項式に\dots]{2つのゲーゲンバウアー多項式に関連する積分公式について\footnote{60 min talk}}

% A subtitle is optional and this may be deleted

\author[レオンチエフ・アレックス]{小林俊行\inst{1} \and \underline{レオンチエフ・アレックス}\inst{2}}

\institute[東大] % (optional, but mostly needed)
{
  \inst{1}%
  大学院数理科学研究科、カブリ数物連携宇宙研究機構\\
  東京大学
  \and
  \inst{2}%
  大学院数理科学研究科\\
  東京大学
  }

  \date[表現論とその周辺分野\dots]{RIMS共同研究(公開型)\\「表現論とその周辺分野の広がり」\\京都大学数理解析研究所}
% - Either use conference name or its abbreviation.
% - Not really informative to the audience, more for people (including
%   yourself) who are reading the slides online

\subject{表現論}

\begin{document}
\begin{frame}\titlepage\end{frame}

\begin{frame}{Outline}
	\tableofcontents
\end{frame}
\section{The Main Result}
\begin{frame}
	\begin{prop}\label{prop:exp-st-gg}
		\begin{equation}
			| s + t |^{2 \nu} = \sum_{\mbox{\scriptsize $\begin{array}[]{c}
			\ell, m = 0 \\ \ell + m : \tmop{even}
		\end{array}$}}^{\infty} a_{\lambda,\mu,\nu}^{\ell,m} C_\ell^{\lambda} (s) C_m^{\mu} (t),
			\label{eqn:exp-st-gg}
		\end{equation}
		{\scriptsize
		\begin{equation*}
	a_{\lambda,\mu,\nu}^{\ell,m}= \frac{ 2^{-2\nu}(\lambda + \ell) (\mu + m)  \Gamma (\lambda + \mu + 2 \nu + 1) \Gamma (\lambda)
  \Gamma (\mu)\Gamma \left( 2\nu +
1 \right)}{\Gamma \left( \lambda + \nu + \frac{\ell -
  m}{2} + 1 \right)  \Gamma \left( \mu + \nu -
  \frac{\ell - m}{2} + 1 \right) \Gamma \left( \lambda + \mu + \nu + \frac{\ell +
  m}{2} + 1 \right)\Gamma\left(  \nu+1-\frac{\ell+m}{2}\right)}
		\end{equation*}
	}
	\end{prop}
	Here, $C_{\ell}^\lambda(t)$ ($\lambda\in\mathbb{C},\ell\in\N$) is the Gegenbauer polynomial.
	\begin{remark}
		We could not find this formula in the literature.
	\end{remark}
\end{frame}
\note{TODO (speech):\begin{enumerate}
\item Do not use numbers like ``proposition 1'', ``proposition 3'' during the speech -- no one will remember them.
	Come up with some ``aliases'' (maybe, ``general proposition'', ``special case'' etc.).
	Or maybe, make one proposition to be theorem, another to be observation etc;
\end{enumerate}
}
%%\begin{frame}
%%	\begin{figure}[h]
%%		\centering
%%\begin{lpic}[]{wanted(0.75)}
%%	\lbl[bl]{3,15; \large$\begin{array}[]{c}
%%		\mbox{\it Expansion of $| s + t |^{2 \nu}$}\\\\\mbox{ in the form} \\\\
%%	| s + t |^{2 \nu} =	\\\\
%%	\sum_{\mbox{\scriptsize $\begin{array}[]{c}
%%\ell, m = 0 \\	 l + m\in2\N
%%\end{array}$}}^{\infty}
%%		a_{\lambda, \mu, \nu}^{\ell, m} C_{\ell}^{\lambda} (s) C_m^{\mu} (t)
%%	\end{array}$} %dot 
%%		
%%	\end{lpic}
%%		\label{fig:wanted}
%%	\vspace{-0.5cm}
%%	\caption*{Can we find the {closed expression} for $a_{\lambda,\mu,\nu}^{\ell,m}$ \underline{in literature}???}
%%	\end{figure}
%%\end{frame}
\begin{frame}
	In fact, we can do more:
	\begin{prop}\label{prop:exp-stz-gg}
		  \label{thm:4}For $\tmop{Re} \lambda, \tmop{Re} \mu > - \frac{1}{2}$,
		    $\tmop{Re} (\nu) > 0$ and $-1 \leqslant z \leqslant 1$ the following holds:
		      \begin{eqnarray}
			          & | s + t z |^{2 \nu} \tmop{sgn}^{\frac{1 \pm 1}{2}} (s + t z) = \sum_{l,
					      m = 0 \mid l + m \equiv \frac{1 \pm 1}{2} \tmop{mod} 2}^{\infty} a_{l, m}
					          (z) C_l^{\lambda} (s) C_m^{\mu} (t), &  \nonumber\\
						      & a_{l, m} (z) = \frac{\Gamma \left( \nu + \frac{1}{2} \right) \Gamma
						      (\lambda) \Gamma (\mu) (\lambda + l) z^m }{(1 + \nu)_{- \frac{l + m}{2}} \sqrt{\pi} \Gamma
										      (\mu + m) \Gamma \left( \lambda + \nu + \frac{l - m}{2} + 1 \right)}&\nonumber
										      \\&\times{}_2 F_1 \left( \begin{array}{c}
								        \frac{l + m}{2} - \nu, \frac{m - l}{2} - \nu - \lambda\\
									      \mu + m + 1
									          \end{array} ; z^2 \right). & 
										          \nonumber
											    \end{eqnarray}
										    \end{prop}
\end{frame}
\note{TODO (slides):\begin{enumerate}
\item investigate why LHS in Prop. 1, 2 and Cor. 1  are non-analytic, while RHS are seemingly analytic
\end{enumerate}
TODO (speech):\begin{enumerate}
	\item keep in mind the formula\begin{equation*}
			{}_2F_1\left( \begin{array}[]{c}
				a,b\\c
			\end{array};1\right)=\frac{\Gamma(c-a-b)\Gamma(c)}{\Gamma(c-a)\Gamma(c-b)},\quad \Re(c-a-b)>0
		\end{equation*}
		during the talk;
\end{enumerate}}
\section{Related Results}
\begin{frame}
	Furthermore,
	Letting $\lambda,\mu\to\infty$ (while keeping $\mu/\lambda=w\in\mathbb{R}$) in Proposition \ref{prop:exp-st-gg}, we arrive at:
	\begin{cor}\label{cor:int-xzy-hh}
		For $\nu\in\mathbb{C}$, $w\in\R$ the following holds:
		\begin{equation*}
			\begin{array}[]{c}
			\int_{- \infty}^{\infty} \int_{- \infty}^{\infty} | x - w y |^{2 \nu} e^{-
			x^2 - y^2} H_\ell (x) H_m (y) d x d y \\= (- \nu)_{\frac{\ell + m}{2}} (- 1)^{\frac{\ell
			- m}{2}} 2^{\ell + m} \pi^{\frac{1}{2}} \Gamma \left( \frac{1}{2} + \nu \right)
			(w^2 + 1)^{\nu - \frac{\ell + m}{2}} w^m .
			\end{array}
		\end{equation*}
	\end{cor}
	\begin{remark}
		The particular case $k=2$ of \underline{Mehta integral} follows from this (take $w=1,\ell=m=0$).
	\end{remark}
	Even more,
\end{frame}
\note{TODO (speech):\begin{enumerate}
	\item keep in mind the formula \begin{equation*}
			H_n (x) = n! \lim_{\lambda \rightarrow \infty} \lambda^{- \frac{n}{2}}
			C_n^{\lambda} \left( \frac{x}{\sqrt{\lambda}} \right)
		\end{equation*}
		during the talk;
\end{enumerate}
}
\begin{frame}[fragile]{Hierarchy}
\begin{tikzpicture}
\draw[color=black] (0.0,0.0) rectangle (2.0,-0.5);
\node at (1.0,-0.25) {\color{black}{\scriptsize Warnaar'10: \color{blue}{4}}};
\draw[color=red] (2.1,0.0) rectangle (4.3,-0.5);
\node at (3.2,-0.25) {\color{red}{\scriptsize Proposition 2: \color{blue}{4}}};
\draw[color=black] (5.7,0.0) rectangle (8.9,-0.5);
\node at (7.300000000000001,-0.25) {\color{black}{\scriptsize Tarasov-Varchenko'03: \color{blue}{3}}};
\draw[color=black] (9.1,0.0) rectangle (12.1,-0.5);
\node at (10.6,-0.25) {\color{black}{\scriptsize Dotsenko-Fateev'85: \color{blue}{4}}};
\draw[color=red] (2.1,-2.0) rectangle (4.3,-2.5);
\node at (3.2,-2.25) {\color{red}{\scriptsize Proposition 3: \color{blue}{3}}};
\draw[color=black] (5.0,-2.0) rectangle (6.7,-2.5);
\node at (5.85,-2.25) {\color{black}{\scriptsize Selberg'44: \color{blue}{3}}};
\filldraw[color=red,pattern color=red,pattern=north east lines] (3.2,-4.0) circle(0.3);
\node at (3.85,-4.0) {\color{blue}{3}};
\fill[color=black] (1.5,-6.0) circle(0.1);
\node at (1.95,-6.0) {\color{blue}{2}};
\fill[color=black] (3.2,-6.0) circle(0.1);
\node at (3.6500000000000004,-6.0) {\color{blue}{2}};
\fill[color=black] (5.85,-6.0) circle(0.1);
\node at (6.3,-6.0) {\color{blue}{2}};
\fill[color=black] (8.9,-6.0) circle(0.1);
\node at (9.35,-6.0) {\color{blue}{2}};
\draw[->,>=angle 90,color=black] (2.0,-0.5) -- node {} (5.0,-2.0) ;
\draw[->,>=angle 90,color=red] (3.2,-0.5) -- node {} (3.2,-2.0) ;
\draw[->,>=angle 90,color=black] (6.18,-0.5) -- node {} (5.85,-2.0) ;
\draw[->,>=angle 90,color=black] (9.1,-0.5) -- node {} (6.615,-2.0) ;
\draw[->,>=angle 90,color=red] (3.2,-2.5) -- node {\color{black}{\scriptsize $\kern1.5cm\ell=m=0$}} (3.2000000000000006,-3.7) ;
\draw[->,>=angle 90,color=black] (1.0,-0.5) -- node {} (1.490946425395748,-5.90041067935323) ;
\draw[->,>=angle 90,color=black] (3.0057054739713385,-4.2285817953278375) -- node {} (1.5573628100792203,-5.93251434108327) ;
\draw[->,>=angle 90,color=black] (3.2,-4.3) -- node {} (3.2,-5.914999999999999) ;
\draw[->,>=angle 90,color=black] (3.4394567450999696,-4.180722071773562) -- node {} (5.777393604812446,-5.9452027206131675) ;
\draw[->,>=angle 90,color=black] (3.4830799946398643,-4.099326313908724) -- node {} (8.810326217188193,-5.968535514802874) ;
\draw[->,>=angle 90,color=black] (5.34,-2.5) -- node {} (3.252164719493017,-5.914683869988056) ;
\draw[->,>=angle 90,color=black] (10.3,-0.5) -- node {} (5.912899489957162,-5.922259057356316) ;
\draw[->,>=angle 90,color=black] (7.62,-0.5) -- node {} (8.877333022939228,-5.902602832941987) ;

\end{tikzpicture}
\end{frame}
\begin{frame}
	Furthermore, making variable change $w=s-t$ in Corollary \ref{cor:int-xzy-hh} and taking $z=1$ we arrive at\footnote[frame]{FIXME: check correctness}:
	\begin{cor}
		The following holds:
		\begin{equation}
			\mathcal{M}(hr_n\ast hr_m)(s)=
			\left(  \frac{1-s}{2}\right)_{\frac{l + m}{2}} (- 1)^{\frac{l
			+ m}{2}}  \pi^{\frac{1}{2}} \Gamma \left( \frac{s}{2} \right)
			2^{\frac{s-3}{2} + \frac{l + m}{2}}.
%%			\begin{array}[]{c}
%%			\int_{- \infty}^{\infty} \int_{- \infty}^{\infty} | x - y |^{2 \nu} e^{-
%%			x^2 - y^2} H_l (x) H_m (y) d x d y \\= (- \nu)_{\frac{l + m}{2}} (- 1)^{\frac{l
%%			- m}{2}} 2^{l + m} \pi^{\frac{1}{2}} \Gamma \left( \frac{1}{2} + \nu \right)
%%			2^{\nu - \frac{l + m}{2}} .
%%			\end{array}
			\label{eqn:mellin-hh}
		\end{equation}
	\end{cor}
	where $\mathcal{M}(\cdot)$ is the Mellin transform\begin{equation*}
		\left\{ \mathcal{M}(f) \right\}(s)=\int_0^\infty x^{s-1}f(x)\;dx,
	\end{equation*}
	and $hr_n$ are \underline{Hermite-Rodriguez} functions \cite{mackenzie2003hermite}
	\begin{equation*}
		hr_n(x):=H_n(x)e^{-x^2}.
	\end{equation*}
	Applying inverse Mellin transform, we arrive at the well-known formula for convolution of Hermite-Rodriguez functions.
\end{frame}
\note{TODO (slides):\begin{enumerate}
\item make the Corollary better;
\end{enumerate}}
\section{Proof}
\begin{frame}{Method 1 (direct)}
	\scriptsize
	We show Proposition \ref{prop:int-st-gg} as:
	\begin{enumerate}
		\item Using the property $
				(1-x^2)^{\alpha-\frac{1}{2}}C_n^\alpha(x)=a(\alpha,n)
				\frac{d^n}{dx^n} (1-x^2)^{n+\alpha-\frac{1}{2}}, a(\alpha,n):=\frac{(-1)^n}{2^nn!}\frac{\Gamma\left( \alpha+\frac{1}{2} \right)\Gamma\left( n+2\alpha \right)}{\Gamma(2\alpha)\Gamma\left(\alpha+n+\frac{1}{2}  \right)}$
				of Gegenbauer polynomials and integration by parts we have ${\mbox{LHS}\eqref{eqn:int-st-gg}}/{a(\lambda,l)/a(\mu,m)}=$\begin{equation}
					\kern-1.2cm
					\iint\displaylimits_{[-1,1]^2}
					\kern-0.2cm| s - t |^{2 \nu} \frac{\partial^l}{\partial s^l}(1-s^2)^{\lambda+l-\frac{1}{2}}\frac{\partial^m}{\partial t^m}(1-t^2)^{\mu+m-\frac{1}{2}}
					\kern-0.1cm=(-1)^m(-2\nu)_{l+m}\kern-0.2cm\iint\displaylimits_{[-1,1]^2}\kern-0.2cm\myabs{s-t}^{2\nu-l-m}(1-s^2)^{\lambda+l-\frac{1}{2}}(1-t^2)^{\mu+m-\frac{1}{2}};
					\label{eqn:m1-1}
				\end{equation}
			\item Upon change of variables $s=(1-t)(1-w)+t$ and using Euler's integral representation of ${}_2F_1$
			we have ${\mbox{RHS}\eqref{eqn:m1-1}}/{(-1)^m/(-2\nu)_{l+m}}=$
			\begin{equation*}
				\kern-1.2cm
				2^{\lambda+l-\frac{1}{2}} B \left( 2\nu-l-m+1, \lambda+l-\frac{1}{2} \right)
				\int_{-1}^1 (1 - t)^{2 \nu+\lambda-m+\frac{3}{2}} {}_2 F_1 \left( \kern-0.3cm\begin{array}{c}
					\frac{1}{2}-l-\lambda, 2\nu-l-m+1\\
					2 \nu+\lambda-m+\frac{3}{2}
				\end{array} ; \frac{1 - t}{2} \right)(1-t^2)^{\mu+m-\frac{1}{2}}dt
			\end{equation*}

%%			via \cite[ET II 186(9)]{gradshteinryzhik}:
%%		\begin{equation*}
%%			\kern-1cm\int_0^u x^{\nu - 1} (x + \alpha)^{\lambda} (u - x)^{\mu - 1} d x =
%%			\alpha^{\lambda} u^{\mu + \nu - 1} B (\mu, \nu)_2 F_1 \left( - \lambda, \nu ;
%%			\mu + \nu ; - \frac{u}{\alpha} \right);
%%		\end{equation*}
	\item Taking the outer integral, we arrive at hypergeometric $_3F_2(;1)$ function which simplifies via Whipple's sum:\begin{equation*}
		\kern-1cm{}_{3}F_{2}\left({a,1-a,c\atop d,2c-d+1};1\right)=\frac{\pi\Gamma\left(d%
				\right)\Gamma\left(2c-d+1\right)2^{1-2c}}{\Gamma\left(c+\frac{1}{2}(a-d+1)%
				\right)\Gamma\left(c+1-\frac{1}{2}(a+d)\right)\Gamma\left(\frac{1}{2}(a+d)%
			\right)\Gamma\left(\frac{1}{2}(d-a+1)\right)},
		\end{equation*}
	\end{enumerate}
\end{frame}
\begin{frame}{Method 1 (direct)}
	\scriptsize
	We show Proposition \ref{prop:int-st-gg} as:
	\begin{enumerate}
		\item In fact, \eqref{eqn:int-st-gg} follows from the special case $\ell=m=0$:
			\begin{equation*}
				\kern-1cm\int_{- 1}^1 \int_{- 1}^1 | s - t |^{2 \nu} (1 - s^2)^{\lambda - \frac{1}{2}}
				(1 - t^2)^{\mu - \frac{1}{2}} d s d t = \frac{\pi^{\frac{1}{2}} \Gamma \left(
				\lambda + \frac{1}{2} \right) \Gamma \left( \mu + \frac{1}{2} \right) \Gamma
				\left( \nu + \frac{1}{2} \right) \Gamma (\lambda + \mu + 2 \nu + 1)}{\Gamma
				(\lambda + \nu + 1) \Gamma (\mu + \nu + 1) \Gamma (\lambda + \mu + \nu + 1)}
			\end{equation*}
			via integration by parts, due to the property of Gegenbauer polynomials:\begin{equation*}
				(1-x^2)^{\alpha-\frac{1}{2}}C_n^\alpha(x)=\frac{(-1)^n}{2^nn!}\frac{\Gamma\left( \alpha+\frac{1}{2} \right)\Gamma\left( n+2\alpha \right)}{\Gamma(2\alpha)\Gamma\left(\alpha+n+\frac{1}{2}  \right)}
				\frac{d^n}{dx^n}\left[ (1-x^2)^{n+\alpha-\frac{1}{2}} \right]
			\end{equation*}
		\item Upon change of variables $w=s-t$, the inner integral (in $w$) can be evaluated via Euler's integral representation of hypergeometric function;
%%			via \cite[ET II 186(9)]{gradshteinryzhik}:
%%		\begin{equation*}
%%			\kern-1cm\int_0^u x^{\nu - 1} (x + \alpha)^{\lambda} (u - x)^{\mu - 1} d x =
%%			\alpha^{\lambda} u^{\mu + \nu - 1} B (\mu, \nu)_2 F_1 \left( - \lambda, \nu ;
%%			\mu + \nu ; - \frac{u}{\alpha} \right);
%%		\end{equation*}
	\item Taking the outer integral, we arrive at hypergeometric $_3F_2(;1)$ function which simplifies via Whipple's sum:\begin{equation*}
		\kern-1cm{}_{3}F_{2}\left({a,1-a,c\atop d,2c-d+1};1\right)=\frac{\pi\Gamma\left(d%
				\right)\Gamma\left(2c-d+1\right)2^{1-2c}}{\Gamma\left(c+\frac{1}{2}(a-d+1)%
				\right)\Gamma\left(c+1-\frac{1}{2}(a+d)\right)\Gamma\left(\frac{1}{2}(a+d)%
			\right)\Gamma\left(\frac{1}{2}(d-a+1)\right)},
		\end{equation*}
	\end{enumerate}
\end{frame}
\note{TODO (slides):\begin{enumerate}
\item 
\end{enumerate}
TODO (speech):\begin{enumerate}
\item elaborate that Whipple's sum works only when there are special constraints on arguments and not true in general;
\end{enumerate}}
\begin{frame}{Method 2 (using heavy machinery)}
	\begin{fact}[Carlson's Theorem]
		Suppose $f$ is a continuous function defined on the right half-plane $\left\{ z\in\mathbb{C}\mid \Re(z)\ge0 \right\}$, which
		is analytic on its interior $\left\{ z\in\mathbb{C}\mid\Re(z)>0 \right\}$ and such that:\begin{enumerate}
			\item $\exists C,\tau>0$, such that $\myabs{f(z)}\le C e^{\tau\myabs{z}}$ for all $z$ in right half-plane;
			\item $\exists C>0,\exists c<\pi$, such that $\myabs{f(iy)}\le Ce^{c\myabs{y}}$ for all $y\in\R$;
			\item $f$ vanishes on positive integers $\N_+$.
		\end{enumerate}
		Then $f\equiv0$.
	\end{fact}
	Using this, \eqref{eqn:int-st-gg} can be deduced from the case $\nu\in\N_{+}$, which in turn follows after simple computations (using beta integrals).
\end{frame}
\note{TODO (slides):\begin{enumerate}
\item write down the proof that the relation in Prop. 1,2 satisfies the hypothesis of Carlson's Thm.
\item add more explanation
\end{enumerate}}
\begin{frame}{Method 3 (proving Prop. \ref{prop:exp-stz-gg})}
	\scriptsize
	\begin{enumerate}
		\item %\label{enum:m4-1}
			Prove (the integral form of) Proposition \ref{prop:exp-stz-gg}
			using the equality {\scriptsize \begin{equation}
				\kern-1cm 2\kern-0.2cm\iint\displaylimits_{[-1,1]^2} \left( s-tz \right)_+^{2c-1}  (1 - s^2)^{a - 1} (1 -
				t^2)^{b - 1} d s d t = \frac{\sqrt{\pi} \Gamma (a) \Gamma (b) \Gamma
			(c)}{\Gamma (a + c) \Gamma \left( b + \frac{1}{2} \right)} {}_2 F_1 \left(
			\begin{array}{c}
				  - c + \frac{1}{2}, - a - c + 1\\
				    b + \frac{1}{2}
			    \end{array} ; z^2 \right) .
				\label{eqn:prop21}
			\end{equation}}
		\item Via the variable change $s=(1-tz)(1-w)+tz$ and Euler's integral representation of ${}_2F_1$, 
				The inner integral (with respect to $s$) is done as:
				\begin{equation*}
				\int_{- 1}^1 (s - tz)_+^{2 c - 1} (1 - s^2)^{a - 1} d s = 2^{a - 1} B (2 c, a)
				(1 - z)^{2 c + a} {}_2 F_1 \left( \begin{array}{c}
					  1 - a, 2 c\\
					    2 c + a
				    \end{array} ; \frac{1 - tz}{2} \right)
			\end{equation*}
		\item \label{enum:m4-1}We then expand ${}_2F_1(;\frac{1-tz}{2})$ in series and take the integral with respect to $t$ by the formula\begin{equation*}
				\int_{- 1}^1 (1 - t z)^{a - 1} (1 - t^2)^{b - 1} d t = B \left( \frac{1}{2},
				b \right) {}_2 F_1 \left( \begin{array}{c}
					  \frac{1 - a}{2}, \frac{2 - a}{2}\\
					    b + \frac{1}{2}
				    \end{array} ; z^2 \right)
			\end{equation*}
		\item Finally, apply the summation formula\begin{equation*}
				\kern-1cm\sum_{i = 0}^{\infty} \frac{(a)_i (1 - a)_i}{2^i i! (d)_i} {}_2 F_1 \left(
				\begin{array}{c}
					  \frac{1 - d - i}{2}, \frac{2 - d - i}{2}\\
					    b + \frac{1}{2}
				    \end{array} ; \zeta \right) = 
				    \overbrace{\sum_{j = 0}^{\infty} \frac{(1 - d)_{2 j} \zeta^j}{2^{2 j} j! \left( b +
				    \frac{1}{2} \right)_j}}^{\kern-0.2cm\scalebox{1}{$={}_2F_1(;\zeta)$}} \overbrace{{}_2 F_1 \left( \begin{array}{c}
					      a, 1 - a\\
					        d - 2 j
					\end{array} ; \frac{1}{2} \right) }^{\mbox{\scriptsize expressed as ratio of $\Gamma$-functions}}
%%				    =\frac{2^{1 - d} \sqrt{\pi} \Gamma (d)}{\Gamma
%%					    \left( \frac{a + d}{2} \right) \Gamma \left( \frac{1 - a + d}{2} \right)} {}_2
%%					    F_1 \left( \begin{array}{c}
%%						      1 - \frac{a + d}{2}, \frac{1 + a - d}{2}\\
%%						        b + \frac{1}{2}
%%						\end{array} ; \zeta \right) .
			\end{equation*}
%%		\item In turn, \eqref{eqn:prop21} is proven via the equality {\scriptsize \begin{equation}
%%				\kern-1.5cm \int_{-1}^1(1-t z)^{a-1}(1-t^2)^{b-1}d t=B\left(\frac{1}{2},b  \right)\ {}_2F_1\left( 
%%				\begin{array}[]{c}
%%					\frac{1-a}{2},\frac{2-a}{2}\\b+\frac{1}{2}
%%				\end{array}
%%				;z^2 \right),
%%				\label{eqn:lem31}
%%			\end{equation}}
%%			Euler's integral and the transformation formula for $_2F_1$ hypergeometric which we derived.
%%		\item \begin{equation}
%%				\sum_{i = 0}^{\infty} \frac{(a)_i (1 - a)_i}{2^i i! (d)_i} _2 F_1 \left(
%%				\begin{array}{c}
%%					  \frac{1 - d - i}{2}, \frac{2 - d - i}{2}\\
%%					    b + \frac{1}{2}
%%				    \end{array} ; \zeta \right) = \frac{2^{1 - d} \sqrt{\pi} \Gamma (d)}{\Gamma
%%					    \left( \frac{a + d}{2} \right) \Gamma \left( \frac{1 - a + d}{2} \right)} _2
%%					    F_1 \left( \begin{array}{c}
%%						      1 - \frac{a + d}{2}, \frac{1 + a - d}{2}\\
%%						        b + \frac{1}{2}
%%						\end{array} ; \zeta \right) .
%%						\label{eqn:lem32}
%%			\end{equation}
	\end{enumerate}
\end{frame}
\note{TODO (slides):\begin{enumerate}
\item 
%%\item elaborate about the transformation formula we used
%%\item make this the fourth step;
%%\item put the emphasize on the fourth step, do not spend so much space on third step;
\end{enumerate}
TODO (speech): \begin{enumerate}
	\item Mention that \eqref{eqn:prop21} is just case $m=\ell=0$ of integral form of Prop. \ref{prop:exp-stz-gg}
	\item explain that formula in step \ref{enum:m4-1} is the Euler's integral plus quadratic transformation
\end{enumerate}}
\begin{frame}{Method 4 (using non-trivial integral formul\ae)}
	\scriptsize
	We prove the Proposition \ref{prop:int-st-gg} (with $\myabs{s-t}$ replaced with $(s+t)_+$) directly:
	\begin{enumerate}
		\item We first take the inner integral, using the formula \cite[7.4.11]{kobayashi2011schrodinger}
			{
				\scriptsize
			\begin{equation*}
				\kern-1cm\int_{-s}^1(s+t)^{2\nu} {C}_m^\mu(t)(1-t^2)^{\mu-\frac{1}{2}}dt=
				\frac{\sqrt{\pi}\Gamma(2\mu+m)\Gamma(2\nu+1)}{\Gamma(\mu)2^{\mu-\frac{1}{2}}m!}
				(1-s^2)^{\nu+\frac{\mu}{2}+\frac{1}{4}}P_{\mu+m-\frac{1}{2}}^{-2\nu-\mu-\frac{1}{2}}(-s).
			\end{equation*}
		}
			where $P_{\mu+m-\frac{1}{2}}^{-2\nu-\mu-\frac{1}{2}}(-s)$ is the associated Legendre function. 
		\item The remaining integral then is of the form\begin{equation}\label{eqn:m4-1}
				\int_{-1}^1(1-s^2)^{\nu+\frac{\mu}{2}+\lambda-\frac{1}{4}}P_{\mu+m-\frac{1}{2}}^{-2\nu-\mu-\frac{1}{2}}(-s)C^{\lambda-\frac{1}{2}}_\ell(s)ds
			\end{equation}
		\item Use integration by parts together with the formul\ae
			\begin{equation*}
				\left(C^\lambda_\ell(s)  \right)'=C^{\lambda+1}_{\ell-1}(s),\quad\left((1-s^2)^{-a/2}P^a_b(s) \right)'
			=-(1-s^2)^{-\frac{a+1}{2}}P_b^{a+1}(s)
			\end{equation*}
			to reduce \eqref{eqn:m4-1} to the form
			\begin{equation}\label{eqn:m4-2}
				\int_{-1}^1(1-s^2)^aP^b_c(s)ds
			\end{equation}
		\item \eqref{eqn:m4-2} is readily handled by \cite[7.132.1]{gradshteinryzhik}.
	\end{enumerate}
\end{frame}
\note{TODO (slides):\begin{enumerate}
	\item check that this Methods 1 and 4 are {\it really} different;
\end{enumerate}
}
\section{Relation to Representation Theory}
\newcommand{\mysbo}{A:\pi\kern-0.1cm\mid_{G'}\xrightarrow{G'}\tau}
{
\begin{frame}{Original Goal}
	\begin{block}
		
	\centerline{
		\xymatrixcolsep{0.5pc}
		\xymatrixrowsep{1pc}
		\xymatrix{G\ar@/_1pc/[d]&&\supset&&G'\ar@/^1pc/[d]&\\
			\pi&\ar[rr]^{A}&&\tau&&{\begin{array}[]{l}
				\\
		\mbox{ :\underline{infinitely-dimensional} representations}\\
		\mbox{ of \underline{noncompact} groups $G,G'$.}
		\end{array}}%
	}}
	\end{block}
	\setbeamercolor{block title}{bg=red!30,fg=black}
	\hspace{1cm}
	\vspace{-0.5cm}
	\begin{block}{Goal:}
%%		\[\begin{array}{cccl}
%%				G:=O(p+1,q+1)&\curvearrowright &\pi\mbox{ :infinitely-dimensional rep.}\\
%%		\bigcup&&&\\
%%		G':=O(p,q+1)&\curvearrowright &\tau\mbox{ :infinitely-dimensional rep.}
%%		\end{array}\]
		Classify all $G'$-intertwining operators $\mysbo$.
	\end{block}
\end{frame}
\begin{frame}{Idea}
	\begin{itemize}%[<+(1)->]
		\item Although $\pi$ and $\tau$ are realized on different spaces, we can define ``generalized eigenvalues'';
		\item it turns out that any $\mysbo$ is ``diagonal'';
		\item $\implies$ it would help to compute eigenvalues explicitly;
	\end{itemize}
\end{frame}
}
\begin{frame}
\begin{center}
	\begin{tikzpicture}
	\draw[color=black] (-9.0,0.75) rectangle (-5.0,-6.05);
\draw[color=black] (-8.75,0.25) rectangle (-5.25,-1.25);
\draw[color=black] (-8.5,0.0) rectangle (-5.5,-0.5);
\draw[color=black] (-8.75,-1.75) rectangle (-5.25,-3.75);
\draw[color=black] (-8.5,-1.825) rectangle (-5.5,-2.325);
\draw[color=black] (-8.5,-2.4) rectangle (-5.5,-2.9);
\draw[color=black] (-8.5,-2.975) rectangle (-5.5,-3.475);
\draw[color=black] (-8.75,-4.5) rectangle (-5.25,-6.0);
\node at (-7.25,-4.0) {\color{black}{\Huge \dots}};
\node at (-7.25,-4.7) {\color{black}{\Huge \dots}};
\node at (-0.25,-3.0) {\color{black}{\Huge \dots}};
\draw[color=black] (-8.5,-4.85) rectangle (-5.5,-5.35);
\draw[color=black] (-8.5,-5.475) rectangle (-5.5,-5.975);
\draw[color=black] (-2.0,0.75) rectangle (2.0,-6.05);
\draw[color=black] (-1.75,0.25) rectangle (1.75,-1.25);
\draw[color=black] (-1.75,-4.5) rectangle (1.75,-6.0);

\node at (-7.0,1.0) {\color{black}{$I(\lambda)$}};
\node at (0.0,1.0) {\color{black}{$J(\nu)$}};

\draw[-open triangle 90] (-5.0,-2.65) to node {$R_{\lambda,\nu}^X$} (-2.0,-2.65);
\draw[-open triangle 90] (-5.5,-0.25) to node {$c^{p,q}_{0,0,0}I$} (-1.75,-0.5);
\draw[-open triangle 90] (-5.5,-2.075) to node {$c^{p,q}_{0,2,0}I$} (-1.75,-0.5);
\draw[-open triangle 90] (-5.5,-5.1) to node {$c^{p,q}_{m',m,n}I$} (-1.75,-5.25);

	\end{tikzpicture}
\end{center}
\end{frame}
\begin{frame}{Idea}
	\begin{itemize}%[<+(1)->]
		\item \ldots
		\item $\implies$ it would help to compute eigenvalues explicitly;
		\item $\mysbo$ are given by integral kernels 
		\item $\implies$ eigenvalues are given by integrals.
	\end{itemize}
\end{frame}
\begin{frame}
	Proposition \ref{prop:exp-st-gg} is equivalent to the integral equality
	\begin{prop}
		\label{prop:int-st-gg}
		\begin{equation*}
			\int_{- 1}^1 \int_{- 1}^1 | s - t |^{2 \nu} (1 - s^2)^{\lambda - \frac{1}{2}}
			(1 - t^2)^{\mu - \frac{1}{2}} C_l^{\lambda} (s) C_m^{\mu} (t) d s d t
		\end{equation*}
		{\scriptsize
		\begin{equation}
			=\frac{(- \nu)_{\frac{l + m}{2}} (- 1)^{\frac{l - m}{2}} \pi^{\frac{1}{2}} (2
			\lambda)_l (2 \mu)_m \Gamma \left( \lambda + \frac{1}{2} \right) \Gamma \left(
			\mu + \frac{1}{2} \right) \Gamma \left( \nu + \frac{1}{2} \right) \Gamma
		(\lambda + \mu + 2 \nu + 1)}{l!m! \Gamma \left( \lambda + \nu + \frac{l -
		m}{2} + 1 \right) \Gamma \left( \mu + \nu - \frac{l - m}{2} + 1 \right) \Gamma
		\left( \lambda + \mu + \nu + \frac{l + m}{2} + 1 \right)}
			\label{eqn:int-st-gg}
			\tag{1$'$}
		\end{equation}
		}
	\end{prop}
\end{frame}
\begin{frame}[allowframebreaks]{References}
	\bibliographystyle{apalike}
	\nocite{Selberg:411367}
	\nocite{warnaar2010sl3}
	\nocite{dotsenko1985four}
	\nocite{tarasov2003selberg}
%%	\nocite{kobayashi2015symmetry}
%%	\nocite{kobayashi2015differential2}
%%	\nocite{kobayashi2016differential1}
%%	\nocite{kobayashi2014classification}
%%	\nocite{kobayashi2013finite}
%%	\nocite{kobayashi2015program}
\bibliography{intdep}
\end{frame}
\end{document}
