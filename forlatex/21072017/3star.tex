\subsection{定理\ref{prop:exp-st-gg}の極限値}
最後に、
定理\ref{prop:exp-st-gg}\doubt{、}あるいはこれと本質的に同\doubt{等}である
命題\ref{prop:int-st-gg}
に関して、パラメータが無限になったときの
極限値を考える。このために、
エルミート多項式を復習する。
エルミート多項式$\left\{ H_n(x) \right\}_{n=1}^\infty$は、
	$L^2\left( \mathbb{R},e^{-\frac{x^2}{2}}dx \right)$の直交多項式であり、
	Gegenbauer多項式の極限として再現できる:
	\begin{equation*}
			H_n (x) = n! \lim_{\lambda \rightarrow \infty} \lambda^{- \frac{n}{2}}
			C_n^{\lambda} \left( \frac{x}{\sqrt{\lambda}} \right).
	\end{equation*}
	最初の数項を挙げよう。
		\begin{eqnarray*}
		H_0(x)&=& 1,\\
		H_1(x)&=& 2x,\\
		H_2(x)&=& 
		4x^2-2,\\
		H_3(x)&=& 8x^3-12x,\\
		H_4(x)&=& 16x^4-48x^2+12.\\
		\end{eqnarray*}
	命題 \ref{prop:int-st-gg}で$\mu/\lambda$を一定にした上で$\lambda,\mu\to\infty$ 
	という極限をとれば、Gegenbauer多項式に関する積分公式から
	エルミート多項式に関する以下の積分公式を得る。
