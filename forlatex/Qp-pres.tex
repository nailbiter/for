%\documentclass[pdf]{beamer}
\documentclass[pdf,notes]{beamer}
\mode<presentation>{\usetheme[secheader]{Boadilla}}
\usepackage{mystyle}
\includecomment{versiona}

\newcommand{\red}[1]{{\color[rgb]{0.6,0,0}#1}}
\newcommand{\DEBUG}{}
\newcommand{\condproof}[1]{\ifdefined\DEBUG\begin{proof}#1\end{proof}\fi}
\newcommand{\trueproof}[1]{\begin{proof}#1\end{proof}}
\newcommand{\falseproof}[1]{}
\newcommand{\tr}{\mbox{tr}}

\makeatletter
\newenvironment<>{proofs}[1][\proofname]{
	\par
	\def\insertproofname{#1\@addpunct{.}}\usebeamertemplate{proof begin}#2}
	{\usebeamertemplate{proof end}}
\makeatother

\makeatletter
\def\th@mystyle{%
    \normalfont % body font
    \setbeamercolor{block title example}{bg=orange,fg=white}
    \setbeamercolor{block body example}{bg=orange!20,fg=black}
    \def\inserttheoremblockenv{exampleblock}
  }
\makeatother

\theoremstyle{mystyle}
\newtheorem{prop}{Proposition}

\title{Admissible and smooth reps of $GL_2(\Q_p)$.}
\subtitle{Definition and basic properties}
\author{Alex Leontiev}
%\setbeamertemplate{background canvas}{\includegraphics[width=\paperwidth,height=\paperheight]{clouds.jpg}}

\begin{document}
\begin{frame}\titlepage\end{frame}
\begin{frame}{Outline}
\tableofcontents
\end{frame}
\section{Basic properties and definition of $F:=\Q_p$}
\begin{frame}
	Fix positive prime number $p$.
	\begin{definition}
		We introduce norm on $\Q$ defined by absolute value
		$\myabs{0}_p:=0$ and $\myabs{p^n(a/b)}_p:=p^{-n}$
		for $n\in\Z$ and $a,\;b$ pairwise simple with $p$.
	\end{definition}
		This is a metric. Moreover, $\forall a,b\in\Q$:
		$\myabs{ab}_p=\myabs{a}_p\myabs{b}_p$, $\myabs{a+b}_p\le\max\left\{ \myabs{a}_p,\myabs{b}_p \right\}$.
		Moreover, if $\myabs{a}_p\neq\myabs{b}_p$, we have $\myabs{a+b}_p=\max\left\{ \myabs{a}_p,\myabs{b}_p \right\}$.
\end{frame}
\begin{frame}
	\begin{definition}
		We let $\Q_p$ to be the completion of $\Q$ with respect to the above metric.
	\end{definition}
	\begin{prop}
		$Q_p$ is locally compact topological field. Metric on $\Q$ extends to be a metric on $\Q_p$ with all distances
		lying in set $\left\{ 0 \right\}\cup\left\{ p^n \right\}_{n\in\Z}$ (hence, $\Q_p$ is totally disconnected). Also in
		$Q_p$ the only compact sets are closed and bounded ones and every bounded infinite set has a limit point.
	\end{prop}
\end{frame}
\begin{frame}
	\begin{definition}
		We let $R_p:=\mysetn{x\in\Q_p}{\myabs{x}_p\le1}$, $\mathfrak{P}:=\mysetn{x\in\Q_p}{\myabs{x}_p<1}$,
		$U:=\mysetn{x\in\Q_p}{\myabs{x}_p=1}$ and $\mathfrak{P}^n:=\mysetn{x\in\Q_p}{\myabs{x}<p^{-n}}$ for $n\in\Z$.
	\end{definition}
	\begin{prop}
		$R_p$ is maximal compact subring of $\Q_p$, $\mathfrak{P}$ is unique maximal ideal of $R_p$. We have $\{
			\mathfrak{P}^n\}_{n\in\Z}$
			forming the local base of $0\in\Q_p$ consisting of open compact sets and $\left[ \mathfrak{P}^m:\mathfrak{P}^n
			\right]=p^{m-n}$ for $m>n\in\Z$.
	\end{prop}
\end{frame}
\begin{frame}
	\begin{prop}
		$\Z\subset R_p$ is dense.
		\label{}
	\end{prop}
	\begin{prop}
		$R_p$ can be written as a disjoint union $R_p=\bigsqcup_{k=0}^{p-1}\left( k+\mathfrak{P} \right)$. More generally,
		every open set in $\Q_p$ can be written as a countable disjoint union of sets $a+\mathfrak{P}^n$ for $a\in\Q$
		and $n\in\Z$.
	\end{prop}
\end{frame}
\begin{frame}
	\begin{prop}
		We can introduce translation-invariant Haar measure $dx$ on $\Q_p$ normalized such that $\myabs{R_p}=1$.
		Then, $d^{\times}x:=\left( 1-p^{-1} \right)\myabs{x}_p^{-1}dx$ forms a Haar measure on a multiplicative group
		$Q_p^{\times}$ normalized such that $\myabs{R_p\setminus\left\{ 0 \right\}}_p=1$.
	\end{prop}
\end{frame}
\section{Basic properties of $GL_n(F)$}
\begin{frame}
	From now on, $G$ will denote a closed subgroup of $GL_n(F)$ for $F=\Q_p$ for some prime $p\in\Z_{>0}$.
	\begin{prop}
		If $U\ni1$ is open in $G$, we can find $K\subset U$ -- compact open subgroup of $G$. Moreover, if $G$ is compact,
		$K$ may in addition be taken to be normal in $G$.
	\end{prop}
	\note{
		One possible choice is $G_i:=\mysetn{g\in G}{g-I_n\in M_n(\mathfrak{P}^i)}$. These $G_i$ form a
		local base of cpt open neighborhoods of $1$.
		
		If $G$ is compact, we may assume $G\subset GL_n(R_p)$ and the same choice of $G_i$ then yields normal
		subgroups.
	}
	Hence, $G$ in particular is locally compact and clearly Hausdorff. Hence, we may talk of Haar measure and unimodularity.
	\begin{prop}
		If $G$ is compact, then it is unimodular.
		If $G=GL_n(\Q_p)$, then the measure $d^{GL}x:=\myabs{\det(g)}^{-n}_pdx$ is unimodular, where $dx$ is the translation
		invariant Haar measure on $M_n(\Q_p)$.
	\end{prop}
\end{frame}
\section{Basic properties of admissible and smooth reps}
\begin{frame}
	\begin{definition}
		Let $(\pi,V)$ be a representation of $G$. We say $\pi$ is {\it smooth} if $\forall v\in V$ we have
		$G_v:=\mysetn{g\in G}{g\cdot v=v}$ is open. If, in addition, for every open subgroup $U\subset G$ we have $V^U:=
		\mysetn{v\in V}{\forall g\in U,\;g\cdot v=v}$ we have $V^U$ being finite-dimensional, we say
		$\pi$ is {\it admissible}.
	\end{definition}
	\begin{example}
		Let $C^\infty(G)$ be the set of locally constant $\C$-valued functions on $G$, $C^\infty_c(G)$ be the elements
		of $C^\infty(G)$ with constant support and $C^\infty_u(G)$ be $f\in C^\infty(G)$ such that $\exists K\subset G$
		compact open such that $f$ is $K$-bi-invariant. Then $C^\infty_u(G)$ and $C^\infty_c(G)$ are smooth representations
		of $G$ with action given by multiplication from left or right.
	\end{example}
\end{frame}
\begin{frame}
	For $\Gamma$ topological group, let $\hat{\Gamma}$ denote the set of equivalence classes of finite dimensional
	irreps $\pi$ of $\Gamma$
	such that $\Ker(\pi)\subset \Gamma$ is open. If $V$ is $\Gamma$-module (with no continuity assumptions) and
	$\rho\in\hat{\Gamma}$,
	let $V(\rho)$ denote the sum of invariant subspaces of $V$ that are of class $\rho$.
	\begin{prop}
		If $\Gamma$ is finite, $V=\bigoplus_{\rho\in\hat{\Gamma}}V(\rho)$.
		\label{}
	\end{prop}
	\begin{theorem}
		If $V$ is smooth representation of $G$ and $K\subset G$ is compact open,
		then $V=\bigoplus_{\rho\in\hat{K}}V(\rho)$. Moreover,
		$V$ is admissible iff all $V(\rho)$ are finite-dimensional.
		\label{}
	\end{theorem}
\end{frame}
\begin{frame}
	\begin{proofs}
		By prop. above, for $v\in V$ we can choose $K_0\triangleleft K$, so that finite group $\Gamma:=K/K_0$ ($\implies
		\hat{\Gamma}\subset\hat{K}$) acts
		on $V^{K_0}\ni v$\footnote{$k_0k\cdot v=k\cdot Ad(k^{-1})k_0\cdot v=k\cdot v$}
		and hence \[v\in V^{K_0}=\bigoplus_{\rho\in\hat{\Gamma}}V^{K_0}(\rho)\subset\sum_{\rho\in\hat{K}}
		V^{K_0}(\rho)\subset\sum_{\rho\in\hat{K}}V(\rho)\]

		Decomposition is direct, as if $\sum_{i=1}^nc_{\rho_i}v_{\rho_i}=0$ and $K_0:=\bigcup_{i=1}^n\Ker(\rho_i)$, then
		directness of decomposition of $\left\{ v_{\rho_i} \right\}_i\subset V^{K_0}\curvearrowleft\Gamma=K/K_0$ implies 
		$c_{\rho_i}\equiv0$.
	\end{proofs}
\end{frame}
\begin{frame}
	\begin{proof}[(cont.)]
		If $\pi$ is admissible, then $\forall V(\rho)\subset V^{\ker(\rho)}$:finite dimensional (since $\ker(\rho)$:open).

		Conversely, if $\pi$ is not admissible, for some compact open $K_0\triangleleft K$ we have $V^{K_0}$: infinite
		dimensional and as for $\Gamma:=K/K_0$ $V^{K_0}$ decomposes as direct sum, we should have for some $\rho_0\in
		\hat{\Gamma}$ that $V^{K_0}(\rho_0)$: infinite dim. Hence, so should be $V(\rho_0)\supset V^{K_0}(\rho_0)$.
	\end{proof}
\end{frame}
\begin{frame}
	\begin{definition}
		For $V$ being $G$-representation, and $\hat{v}$ -- $\C$-linear functional on $V$. We say $\hat{v}$ is {\it smooth}
		if for all $g$ in some open set $1\in U\subset G$ we have $\forall v\in V,\;\hat{v}(g\cdot v)=\hat{v}(v)$.
	\end{definition}
	\begin{prop}
		For $V$ being smooth $G$-representations, let $\hat{V}$ be set of smooth linear functionals on $V$. Then
		$\hat{V}$ (with obvious action) is smooth $G$-module. Moreover, $\hat{V}\simeq\bigoplus_{\rho\in\hat{K}}V(\rho)
		\,{\hat{}}$. Moreover, if $V$ was admissible, so will be $\hat{V}$ and we have self-duality.
		\label{}
	\end{prop}
	This $\hat{V}$ is called contragredient representation.
\end{frame}
\begin{frame}
	\begin{prop}
		$C^{\infty}_c(G)$ is closed under convolution, which makes it an algebra without a unit.
		We will call it {\it Hecke algebra} and denote by $\mathcal{H}$.
		\label{}
	\end{prop}
	\begin{prop}
		For $K\subset G$ being open compact subgroup, let $C^{\infty}_K(G)\subset C^{\infty}_c(G)$ be the elements that
		are $K$-bi-invariant. Then $C^{\infty}_K$ is also closed under convolution, which makes it an algebra $\mathcal{H}_K$
		with identity given by $\varepsilon_K:=1_K/\myabs{K}$.
		\label{}
	\end{prop}
\end{frame}
\begin{frame}
	\begin{definition}
		For smooth representation $(\pi,V)$ of $G$ and $\varphi\in\mathcal{H}$ we define an endomorphism
		$\pi(\varphi)$ on $V$ by $\pi(\varphi)v:=\int_G\varphi(g)\pi(g)v\;dg$.
	\end{definition}
	\begin{theorem}
		For $(\pi,V)$: smooth rep of $G$ TFAE:
		\begin{enumerate}[(1)]
			\item $\pi$ is irreducible;
			\item $V$ is simple as an $\mathcal{H}$-module (the structure of $\mathcal{H}$-module given by previous
				definition;
			\item for every $K\subset G$ compact open\footnote{typo in Prop. 4.2.3 in Bump} subgroup, $V^{K}=0$ or $V^K$ is simple as $\mathcal{H}_K$-module
				\footnote{$V^K$ is invariant under $\mathcal{H}_K$}.
		\end{enumerate}
		\label{}
	\end{theorem}
\end{frame}
\begin{frame}
	\begin{proof}
		$(2)\implies(1)$: true, as $G$-invariant subspace is $\mathcal{H}$-invariant.

		$(1)\implies(2)$. Let $W\subset V$ be $\mathcal{H}$-invariant, $v\in W$ and $g\in G$. Then $v$ is fixed
		by compact open $N\ni1$ and hence for $\mathcal{H}\ni\varphi:=1_{gN}/\mbox{vol}(N)$ we have 
		$\mathcal{H}\ni\pi(\varphi)v=\pi(g)v$.

		$(3)\implies(2)$. Let $0\subsetneq W\subsetneq V$ be $\mathcal{H}$-submodule. Take $w\in W$, $v\in V\setminus W$
		and choose $K\ni1$ cpt open subgroup, such that it stabilizes both. Then, $w\in W^{K}$ and $v\in V^K$, hence $V^K$
		is a proper $\mathcal{H}_K$ submodule of $V^K$.

		$(2)\implies(3)$. Let $W\subset V^K$ be proper nonzero. It suffices to show $\pi(\mathcal{H})W\cap V^K=W$, as then 
		$\pi(\mathcal{H})W\subset V$ is proper nonzero. Only $\subseteq$ is nontrivial. So let $\sum_i\pi(\varphi_i)w_i\in
		V^K$. Then (as $\pi(\varepsilon_K)w_i=w_i$),
		$\sum_i\pi(\varphi_i)w_i=\pi(\varepsilon_K)\sum_i\pi(\varphi_i)w_i=\sum_i\pi(\varepsilon_K
		\star\varphi_i\star\varepsilon_K)w_i$. But as $\varepsilon_K\star\varphi_i\star\varepsilon_K\in\mathcal{H}_K$ and
		$w_i\in W$, the latter sum is in $W$.
	\end{proof}
\end{frame}
\section{Further properties of admissible and smooth reps}
\begin{frame}
	\begin{theorem}[Schur lemma]
		If $(\pi,V)$ is admissible $G$-irrep and $A:V\to V$ is $G$-intertwiner, then $A=c\cdot I$ for some $c\in\C$.
		\label{}
	\end{theorem}
	\begin{proof}
		Take $K$ so that $V^K\neq0$. Then $T\big|_{V^K}:V^K\to V^K$ and since $\dim(V^K)<\infty$, $T\big|_{V^K}$ has
		eigenvalue $c$, hence $\ker(T-cI)\neq0\implies T-cI=0$.
	\end{proof}
	\begin{theorem}
		If $(\pi,V)$ is admissible, $K\subset G$ is open compact subgroup
		and $(\hat{\pi},\hat{V})$ is contragredient, then restriction of natural pairing
		on $V\times\hat{V}$ to $V^K\times\hat{V}^{K}$ is non-degenerate.
		\label{}
	\end{theorem}
\end{frame}
\begin{frame}
	\begin{proof}
		Let $0\neq v\in V^K$. By the symmetry $\hat{\hat{V}}$, it suffices to find $\hat{v}\in\hat{V}^K$
		such that $\hat{v}(v)\neq0$. Find $\hat{v}'\in\hat{V}$ such that $\hat{v}'(v)\neq0$. Then
		\[0\neq\hat{v}'(v)=\hat{v}'(\pi(\varepsilon_K)v)=\left( \hat{\pi}(\varepsilon_K)v' \right)(v)\]
		and since $\hat{\pi}(\varepsilon_K)v'\in\hat{V}^K$, we are done.
	\end{proof}
\end{frame}
\begin{frame}
	\begin{definition}
		Let $\mathcal{D}(G)$ denote the space of linear $\C$-valued functionals on $C^\infty_c(G)$. We call elements
		of $\mathcal{D}(G)$ {\it distributions}. They form $G$-module.
	\end{definition}
	\begin{definition}
		For $V$: $\C$-linear space and $A:V\to V$ be linear operator of finite rank, we define $\tr(A)$ as $\tr(A):=\tr(A
		\bigg|_U:U\to U)$ for any finite dimensional $U$ containing image of $A$. Such $\tr(A)$ is well-defined.
	\end{definition}
	\begin{definition}
		Let $(\pi,V)$ be admissible $G$-rep. We will call distribution $\chi_\pi:\varphi\to
		\tr(\pi(\varphi))$ a {\it character} of $\pi$.
	\end{definition}
\end{frame}
\begin{frame}
	\begin{theorem}[from ``Algebra'' by S. Lang]
		Let $R$ be a $\C$-algebra and $E_i$ simple finite-dimensional $R$-modules for $i=1,2$. Then every element
		$\varphi\in R$ denote homomorphism $\lambda_i(\varphi):E_i\to E_i$ given by multiplication with $\varphi$. If
		$\forall\varphi\in R,\;\tr(\lambda_1(\varphi))=\tr(\lambda_2(\varphi))$, then $E_1\simeq E_2$.
	\end{theorem}
	\begin{fact}[Thm 3.2 from ``Algebra'' by S. Lang]
		Let $E$ be semisimple $R$-module, $R':=\End_R(E)$ and $f\in\End_{R'}(E)$. Then if $\left\{ x_i \right\}_{i=1}^n\subset
		E$ there exists $\alpha\in R$ such that $\alpha x_i=f(x_i)$.
	\end{fact}
	\note{
		For completeness, let's prove the fact taken from ``Algebra''. We will prove only the case when $E$ decomposes
		as {\it finite} direct sum of simples, since this suffices for our purposes.

		First, let's consider the case $n=1$, that is
		for $x\in E$ and $f\in\End_{R'}(E)$ we want to show that $\exists\alpha\in R$ with $\alpha\cdot x=f(x)$. Indeed,
		the semisimplicity allows us to write $E=Rx\oplus F$ and consider $\pi:E\to Rx$ projection. Then, as $\pi\in R'$,
		we have $f(x)=f(\pi(x))=\pi(f(x))$, hence $f(x)\in Rx$.

		Next, assume $n>1$. Indeed, setting $\End(E^n)\ni f^{(n)}(y_1,\hdots,y_n):=(f(y_1),\hdots,f(y_n))
		$ and letting $R_n':=\End_R(E^n)$ (one may view element of $R_n'$ as $n\times n$ matrix with entries in $R'$)
		) we have $f^{(n)}\in\End_{R_n'}(E^n)$ and application of the previous paragraph ends
		the proof in this case as well.
	}
	\begin{proofs}
		Suppose $E_1\not\simeq E_2$, $E:=E_1\oplus E_2$ and $\pi_1\in\End(E)$ being projection on $E_1$.
		Easy to see that for any $\varphi\in\End_{R}(E)$ we have $\varphi(E_i)\subset E_i$, hence $\pi_1\in\End_{R'}(E)$.
		By the fact above we have $\alpha$ such that multiplication with $\lambda_1(\alpha)=id$ and $\lambda_2(\alpha)=0$.
	\end{proofs}
\end{frame}
\begin{frame}
	\begin{theorem}
		If $(\pi_i,V_i)$ for $i=1,2$ are admissible $G$-irreps and $\forall K\subset G$ cpt open subgroup we have
		$V_1^K\simeq V_2^K$ as $\mathcal{H}_K$ modules, then $\pi_1\simeq\pi_2$.
		\label{}
	\end{theorem}
	We first prove the next theorem, and then the previous one.
	\begin{theorem}
		If two admissible $G$-irreps have identical characters, they are isomorphic.
		\label{}
	\end{theorem}
	\begin{proof}
		Take Two such irreps $(\pi_i,V_i)$ for $i=1,2$. For $K\subset G$ compact open subgroup, $E_i=V_i^K$ are isomorphic
		as $\mathcal{H}_K$-modules by theorem from ``Algebra'' above, hence by previous theorem we are done.
	\end{proof}
\end{frame}
\begin{frame}
	\begin{proofs}
		Take $K\subset G$ compact open small enough so that $V_i^K$ are both non-zero. By proposition above, these are simple 
		f-d $\mathcal{H}_K$ modules. By hypothesis, we should have $\mathcal{H}_K$-intertwiner $\sigma_K:V_1^K\diffsm V_2^K$.
		By Schur lemma (for f-d simple modules) it is determined up to const.

		Now, for $K_0\subset K$ open compact subgroup, we have $V_i^{K_0}\supset V_i^K$.
		Similarly, hypothesis implies the existence of $\sigma_{K_0}:V_1^{K_0}\diffsm
		V_2^{K_0}$. Then, as $\varepsilon_K\in\mathcal{H}_K\subset \mathcal{H}_{K_0}$,
		$\pi_i(\varepsilon_K)V_i^{K_0}=V_i^{K}$ and $\sigma_{K_0}$ is an $\mathcal{H}_{K_0}$-intertwiner, we have 
		$\sigma_{K_0}V^K_1=\sigma_{K_0}
		\pi_1(\varepsilon_K)V^{K_0}_1=\pi_2(\varepsilon_K)\sigma_{K_0}V^{K_0}_1=\pi_2(\varepsilon_K)
		V_2^{K_0}=V_2^K$. Thus, $\sigma_{K_0}\big|_{V_1^K}$
		maps $V_1^K$ into $V_2^K$, hence induces $\mathcal{H}_K$ isomorphism
		between $V_i^K$, which is then proportional to $\sigma_K$. By adjusting $\sigma_{K_0}$ we may then ensure 
		$\sigma_{K_0}\big|_{K}=\sigma_K$.

		Hence, by shrinking $K$ to ${1}$ we can obtain $\mathcal{H}$-intertwiner $\sigma:V_1\diffsm V_2$ and it remains
		to show that it is in fact $G$-intertwiner.
	\end{proofs}
\end{frame}
\begin{frame}
	\begin{proof}
		But the latter follows, as for $v_0\in V_1$, and $g_0\in G$ we can find compact open subgroup $N\subset G$ 
		such that $N\subset G_{v_0}$ and $N\subset G_{\sigma(v_0)}$. 
		Then by taking $\varphi_0:=1_{g_0N}/\myabs{N}$ we have $\pi_1(\varphi_0)v_0=\pi_1(g_0)v_0$ and $\pi_2(\varphi_0)
		\sigma(v_0)=\pi_2(g_0)\sigma(v_0)$. 
		
		And as $\sigma$ intertwines $\pi_i(\varphi_0)$, it will also intertwine $\pi_i(g_0)$.
	\end{proof}
\end{frame}
\begin{frame}
	\begin{fact}
		For $G=GL_2(\Q_p)$ and $D\in\mathcal{D}$ being invariant under $Ad(G)$, we have $D$ being also invariant under 
		transposition on $G$.\footnote{proven as at p.449 of Bump's book}
		\label{}
	\end{fact}
	\begin{theorem}
		Let $G=GL_2(\Q_p)$ and $(\pi,V)$ be an admissible $G$-irrep. Then,
		if $(\pi_1,V)$ defined as $\pi_1(g):=\pi(g^{-T})$, $\pi_1\simeq\hat{\pi}$.
	\end{theorem}
	We first prove the next theorem, and then the previous one.
	\begin{theorem}
		If $\pi$ is an admissible rep of $G=GL_n(\Q_p)$, then $\pi$ is irrep iff $\hat{\pi}$ is so.
		\label{}
	\end{theorem}
\end{frame}
\begin{frame}
	\begin{proof}
		As by previous theorem $\hat{\pi}\simeq\pi_1$ and $G$ is clearly closed under $g\mapsto g^{-T}$,
		we see that subspace of rep would be $\pi$-invariant iff it would be $\pi_1$ invariant and we're done.
	\end{proof}
	\begin{proof}
		As $\chi_\pi$ is easily seen to be conjugation-invariant\footnote{if $g_0\in G$ and 
			$\varphi\in\mathcal{H}_K\subset\mathcal{H}$, we may WLOG assume $Ad(g_0)K=K$ (by taking $K\leftarrow K\cap
			Ad(g_0)K$) and hence $Ad(g_0)\mathcal{H}_K=\mathcal{H}_K$ and $Ad(g_0)V^K=V^K$, hence $\mbox{im}(\pi(\varphi
			)), \mbox{im}(\pi(Ad(g_0)\varphi))\subset V^K$ and $\tr(\pi(\varphi)\big|_{V^K})=\tr(\pi(Ad(g_0)\varphi)
		\big|_{V^K})$},
		it is also transpose-invariant by the result above. Hence, if for $\varphi\in\mathcal{H}$ we define elements of $
		\mathcal{H}$ by $\varphi^{-1}(g):=\varphi(g^{-1})$ and $\varphi^{-T}(g):=\varphi(g^{-T})$, we have
		$\chi_{\pi_1}(\varphi)=\chi_{\pi}(\varphi^{-T})=\chi_\pi(\varphi^{-1})$.

		Finally, $\pi(\varphi^{-1})$ and $\hat{\pi}(\varphi)$ are adjoints of each other, hence have equal trace,
		hence $\chi_{\pi_1}(\varphi)=\chi_\pi(\varphi^{-1})=\chi_{\hat{\pi}}(\varphi)$ and hence $\hat{\pi}\simeq\pi_1$.
	\end{proof}
\end{frame}
\end{document}
