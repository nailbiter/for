\documentclass[12pt]{article} % use larger type; default would be 10pt

\usepackage{mathtext}                 % підключення кирилиці у математичних формулах
                                          % (mathtext.sty входить в пакет t2).
\usepackage[T1,T2A]{fontenc}         % внутрішнє кодування шрифтів (може бути декілька);
                                          % вказане останнім діє по замовчуванню;
                                          % кириличне має співпадати з заданим в ukrhyph.tex.
\usepackage[utf8]{inputenc}       % кодування документа; замість cp866nav
                                          % може бути cp1251, koi8-u, macukr, iso88595, utf8.
\usepackage[english,russian,ukrainian]{babel} % національна локалізація; може бути декілька
                                          % мов; остання з переліку діє по замовчуванню. 
\usepackage{mystyle}

\newtheorem{prob}{Завдання}
\newcommand{\ds}{\;ds}
\newcommand{\dt}{\;dt}
\newcommand{\dx}{\;dx}
\newcommand{\dta}{\;d\tau}
\newcommand{\extr}{\mbox{\normalfont extr}}

\newtheorem{myulem}[mythm]{Лема}

\renewenvironment{myproof}[1][Доведення]{\begin{trivlist}
\item[\hskip \labelsep {\bfseries #1}]}{\myqed\end{trivlist}}

\title{Рівняння математичної фізики (10 семестр)}
\author{}

\begin{document}
\def\dx{\Delta x}
\def\dt{\Delta t}
\def\dX{\frac{\partial}{\partial x}}
\maketitle
\begin{prob}Звести до канонічного вигляду запропоноване рівняння
	\[u_{x_1x_1}+2\sum_{k=2}^nu_{x_kx_k}-2\sum_{k=1}^{n-1}u_{x_kx_{k+1}}=0\]
\end{prob}
Помітимо, що
\[u_{x_1x_1}+2\sum_{k=2}^nu_{x_kx_k}-2\sum_{k=1}^{n-1}u_{x_kx_{k+1}}=\]\[=\mybra{\uwave{
\frac{\partial^2}{\partial x_1^2}+\frac{\partial^2}{\partial x_2^2}-
2\frac{\partial^2}{\partial x_1\partial x_2}}+\frac{\partial}{\partial x_2^2}
+2\sum_{k=3}^n\frac{\partial^2}{\partial x_k^2}-2\sum_{k=2}^{n-1}\frac{\partial^2}{\partial x_k\partial x_{k+1}}}u=\]
\[=\mybra{\uwave{\mybra{\frac{\partial}{\partial x_1}-\frac{\partial}{\partial x_2}}}^2+\frac{\partial}{\partial x_2^2}
+2\sum_{k=3}^n\frac{\partial^2}{\partial x_k^2}-2\sum_{k=2}^{n-1}\frac{\partial^2}{\partial x_k\partial x_{k+1}}}u=\dots=\]
\[=\mybra{\mybra{\frac{\partial}{\partial x_1}-\frac{\partial}{\partial x_2}}^2+
\mybra{\frac{\partial}{\partial x_2}-\frac{\partial}{\partial x_3}}^2+\hdots+
\mybra{\frac{\partial}{\partial x_{n-1}}-\frac{\partial}{\partial x_n}}^2+\frac{\partial^2}{\partial x_n^2}
}u\]
і рівняння перетворюється на
\[u{y_1y_1}+\hdots+u_{y_{n-1}y_{n-1}}+u_{x_nx_n}=0\]
Це рівняння еліптичного типу.
\begin{prob}Звести рівняння до канонічного вигляду в кожній з областей, де зберігається його тип
	\[u_{xx}-6u_{xy}+10u_{yy}+u_x-3u_y=0\]
\end{prob}
Це рівняння еліптичного типу, оскільки в заданій області 
\[\det\begin{vmatrix}1&-3\\-3&10\end{vmatrix}=1>0\]
Оскільки коефіцієнти в старших членах сталі, можна просто вгадати заміну змінних, як і в попередньому прикладі:
\[\mybra{\frac{\partial^2}{\partial x^2}-6\frac{\partial^2}{\partial x\partial y}+10\frac{\partial^2}{\partial y^2}}u=\]
\[=\mybra{\mybra{\frac{\partial}{\partial x}-3\frac{\partial}{\partial y}}^2+\frac{\partial^2}{\partial y^2}}u\]
Таким чином, маємо зробити лінійну заміну $x=a\alpha+b\beta,\;y=c\alpha+d\beta$ таку, щоб виконувалось
\[u_\alpha=au_x+cu_y=u_x-3u_y\implies a=1,\;c=-3\]
\[u_\beta=bu_x+du_y=u_y\implies b=0,\;d=1\]
Таким чином, роблячи заміну $x=\alpha,\;y=-3\alpha+\beta$ рівняння перетворюється на рівняння еліптичного типу
\[u_{\alpha\alpha}+u_{\beta\beta}=-u_\alpha\]
\end{document}
