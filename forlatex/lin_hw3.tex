\documentclass[8pt]{article} % use larger type; default would be 10pt

\usepackage{mystyle}

\title{MATH1030 C,D\\Suggested solutions to HW1}
\author{Alex Leontiev}
\begin{document}
\maketitle
	\renewcommand{\v}{\mathbf{v}}
	\renewcommand{\u}{\mathbf{u}}
	\renewcommand{\i}{\mathbf{i}}
	\renewcommand{\j}{\mathbf{j}}
	\renewcommand{\k}{\mathbf{k}}
	\newcommand{\w}{\mathbf{w}}
\textbf{Section 1.5}
\begin{enumerate}[1]
	\setcounter{enumi}{2}
	\item This sort of problem is basically just done by inspection. Since problem asks us to find "an" elementary matrix,
		it means that there should be only one. Hence, in each subproblem $A$ can be brought to $B$ via just one elementary
		row operation. We just need to find out which one, and write it as an elementary matrix (recall, that to do so
		one just needs to apply elementary row operation to the identity matrix and write down the result).
		\begin{enumerate}[\bf \bf(a)]
			\item {\bf Answer: }operation performed is the multiplication of a first row by $-2$, which corresponds to
				\[E=\mybramatii{-2}{0}{0}{1}\]
			\item {\bf Answer: }operation performed is interchange of second and third row, which corresponds to
				\[E=\mybramatiii{1}{0}{0}{0}{0}{1}{0}{1}{0}\]
			\item {\bf Answer: }operation performed is adding the second row to the third two times, which corresponds to
				\[E=\mybramatiii{1}{0}{0}{0}{1}{0}{0}{2}{1}\]
		\end{enumerate}
	\setcounter{enumi}{5}
	\item
		\begin{enumerate}[\bf(a)]
			\item {\bf Answer: }basically, we just perform Gaussian elimination to bring $A$ to its row echelon form
				(which will be upper triangular by definition), noting the operations we've done. Question, however,
				suggests that we should use only three elementary operations, so as textbook does in examples, we
				shall use only row addition in the reduction process. Thus, what we'll get isn't strictly speaking
				"row echelon form" (it pivot elements won't be 1), but it will still be upper triangular. Once
				we are done with reduction, first second and third operations will correspond to $E_1$, $E_2$ and $E_3
				$ respectively.
				
				It's important to notice, that textbook writes matrices as $E_3E_2E_1A$,
				not as $E_1E_2E_3A$. The latter is incorrect, while the former is correct -- operations are applied
				"right-to-left" when in matrix notation. This is due to the way how multiplication and associativity
				work: $E_3E_2E_1A:=E_3(E_2(E_1A))$, so operations are applied in the order they supposed to, on
				contrary to $E_1E_2E_3A:=E_1(E_2(E_3A))$, which reverses the order. Anyway, here's how reduction goes
				\[A=\mybramatiii{2}{1}{1}{6}{4}{5}{4}{1}{3}\xrightarrow{\text{\textcircled{2}-3*\textcircled{1}}}
				\mybramatiii{2}{1}{1}{0}{1}{2}{4}{1}{3}\xrightarrow{\text{\textcircled{3}-2*\textcircled{1}}}
				\mybramatiii{2}{1}{1}{0}{1}{2}{0}{-1}{1}\xrightarrow{\text{\textcircled{3}+1*\textcircled{2}}}
				\mybramatiii{2}{1}{1}{0}{1}{2}{0}{0}{3}=:U
				\]
				and three operations done are written as elementary matrices as:
				\[{\text{\textcircled{2}-3*\textcircled{1}}}\implies E_1=\mybramatiii{1}{0}{0}{-3}{1}{0}{0}{0}{1}\]
				\[{\text{\textcircled{3}-2*\textcircled{1}}}\implies E_2=\mybramatiii{1}{0}{0}{0}{1}{0}{-2}{0}{1}\]
				\[{\text{\textcircled{3}+1*\textcircled{2}}}\implies E_3=\mybramatiii{1}{0}{0}{0}{1}{0}{0}{1}{1}\]
			\item {\bf Answer: }
				inverses of elementary matrices are found directly, as the inverses of corresponding elementary row
				operations, that is
				\[\mybra{\text{\textcircled{2}-$3*$\textcircled{1}}}^{-1}={\text{\textcircled{2}+3*\textcircled{1}}}
				\implies E_1^{-1}=\mybramatiii{1}{0}{0}{3}{1}{0}{0}{0}{1}\]
				\[\mybra{\text{\textcircled{3}-$2*$\textcircled{1}}}^{-1}={\text{\textcircled{3}+2*\textcircled{1}}}
				\implies E_2^{-1}=\mybramatiii{1}{0}{0}{0}{1}{0}{2}{0}{1}\]
				\[\mybra{\text{\textcircled{3}+$1*$\textcircled{2}}}^{-1}={\textcircled{3}-1*\textcircled{2}}
				\implies E_3^{-1}=\mybramatiii{1}{0}{0}{0}{1}{0}{0}{-1}{1}\]
				this gives us
				\[L=E_1^{-1}E_2^{-1}E_3^{-1}=\left(\begin{array}{rrr}1&0&0\\3&1&0\\2&-1&1\\\end{array}\right)\]
				which is lower-triangular, and indeed we have
				\[LU=\mybramatiii{2}{1}{1}{6}{4}{5}{4}{1}{3}=A\]
		\end{enumerate}
	\item 
		\begin{enumerate}[\bf(a)]
			\item{\bf Answer: }we should note first that such expansion is not unique. One way to obtain it, is to perform
				Gaussian elimination on $E$, bringing it to reduced row echelon form, which will be identity as $A$
				is non-singular (recall, that only non-singular matrices can be written
				as product of elementary ones). Writing operations that we
				have done as elementary matrices $E_1$, $E_2$ up to $E_n$ (for some $n$), we will get thus
				(note the remark above about reversing the order)
				\[E_nE_{n-1}\hdots E_1A=I\implies A=E_1^{-1}E_2^{-1}\hdots E_n^{-1}\]
				this will give $A$ written as product of elementary matrices, since $E_i^{-1}$ are elementary,
				as $E_i$ are elementary. Let's perform the elimination:
				\[A=\mybramatii{2}{1}{6}{4}\xrightarrow{\text{\textcircled{1}$/2$}}
				\mybramatii{1}{\mysfrac{1}{2}}{6}{4}
				\xrightarrow{\text{\textcircled{2}-$6*$\textcircled{1}}}\mybramatii{1}{\mysfrac{1}{2}}{0}{1}
				\xrightarrow{\text{\textcircled{1}-\textcircled{2}$/2$}}\mybramatii{1}{0}{0}{1}
				\]
				Those operations can be written as elementary matrices as
				\[{\text{\textcircled{1}$/2$}}\implies E_1=\mybramatii{\mysfrac{1}{2}}{0}{0}{1}\]
				\[{\text{\textcircled{2}-$6*$\textcircled{1}}}\implies E_2=\mybramatii{1}{0}{-6}{1}\]
				\[{\text{\textcircled{1}-\textcircled{2}$/2$}}\implies E_3=\mybramatii{1}{-\mysfrac{1}{2}}{0}{1}\]
				inverses are correspondingly
				\[\mybra{\text{\textcircled{1}$/2$}}^{-1}={\text{2*\textcircled{1}}}
				\implies E_1^{-1}=\mybramatii{2}{0}{0}{1}\]
				\[\mybra{\text{\textcircled{2}-$6*$\textcircled{1}}}^{-1}={\text{\textcircled{2}+$6*$\textcircled{1}}}
				\implies E_2^{-1}=\mybramatii{1}{0}{6}{1}\]
				\[{\mybra{\text{\textcircled{1}+\textcircled{2}$/2$}}}^{-1}\implies 
					E_3^{-1}=\mybramatii{1}{\mysfrac{1}{2}}{0}{1}\]
				thus, after all
				\[A=E_1^{-1}E_2^{-1}E_3^{-1}=\mybramatii{2}{0}{0}{1}\cdot\mybramatii{1}{0}{6}{1}\cdot
				\mybramatii{1}{\mysfrac{1}{2}}{0}{1}.\]
			\item {\bf Answer: }as seen from computations above, $E_nE_{n-1}\hdots E_1A=I$, hence
				\[A^{-1}=E_3E_2E_1=\mybramatii{\mysfrac{1}{2}}{0}{0}{1}\cdot\mybramatii{1}{0}{-6}{1}\cdot
				\mybramatii{1}{-\mysfrac{1}{2}}{0}{1}\]
		\end{enumerate}
	\setcounter{enumi}{9}
		\begin{enumerate}[\bf(a)]
			\setcounter{enumi}{3}
			\item{\bf Answer: }this is done by Gaussian elimination, as follows
				\[\left(\begin{array}{rr|rr}
					3&0&1&0\\
					9&3&0&1\\
				\end{array}\right)
				\xrightarrow{\text{\textcircled{2}$-3*$\textcircled{1}}}\left(\begin{array}{rr|rr}
					3&0&1&0\\
					0&3&-3&1\\
				\end{array}\right)
				\]\[\left(\begin{array}{rr|rr}
					3&0&1&0\\
					0&3&-3&1\\
				\end{array}\right)
				\xrightarrow{\text{$\mysfrac{1}{3}*$\textcircled{1}}}\left(\begin{array}{rr|rr}
					1&0&\mysfrac{1}{3}&0\\
					0&3&-3&1\\
				\end{array}\right)
				\]\[\left(\begin{array}{rr|rr}
					1&0&\mysfrac{1}{3}&0\\
					0&3&-3&1\\
				\end{array}\right)
				\xrightarrow{\text{$\mysfrac{1}{3}*$\textcircled{2}}}\left(\begin{array}{rr|rr}
					1&0&\mysfrac{1}{3}&0\\
					0&1&-1&\mysfrac{1}{3}\\
				\end{array}\right)
				\]
				Hence the inverse is \[\mybramatii{\mysfrac{1}{3}}{0}{-1}{\mysfrac{1}{3}}\]
			\setcounter{enumi}{5}
		\item {\bf Answer: }similarly to previous example,
\[\left(\begin{array}{rrr|rrr}
2&0&5&1&0&0\\
0&3&0&0&1&0\\
1&0&3&0&0&1\\
\end{array}\right)
\xrightarrow{\text{$\mysfrac{1}{2}*$\textcircled{1}}}\left(\begin{array}{rrr|rrr}
1&0&\mysfrac{5}{2}&\mysfrac{1}{2}&0&0\\
0&3&0&0&1&0\\
1&0&3&0&0&1\\
\end{array}\right)
\]\[\left(\begin{array}{rrr|rrr}
1&0&\mysfrac{5}{2}&\mysfrac{1}{2}&0&0\\
0&3&0&0&1&0\\
1&0&3&0&0&1\\
\end{array}\right)
\xrightarrow{\text{\textcircled{3}$-1*$\textcircled{1}}}\left(\begin{array}{rrr|rrr}
1&0&\mysfrac{5}{2}&\mysfrac{1}{2}&0&0\\
0&3&0&0&1&0\\
0&0&\mysfrac{1}{2}&-\mysfrac{1}{2}&0&1\\
\end{array}\right)
\]\[\left(\begin{array}{rrr|rrr}
1&0&\mysfrac{5}{2}&\mysfrac{1}{2}&0&0\\
0&3&0&0&1&0\\
0&0&\mysfrac{1}{2}&-\mysfrac{1}{2}&0&1\\
\end{array}\right)
\xrightarrow{\text{$\mysfrac{1}{3}*$\textcircled{2}}}\left(\begin{array}{rrr|rrr}
1&0&\mysfrac{5}{2}&\mysfrac{1}{2}&0&0\\
0&1&0&0&\mysfrac{1}{3}&0\\
0&0&\mysfrac{1}{2}&-\mysfrac{1}{2}&0&1\\
\end{array}\right)
\]\[\left(\begin{array}{rrr|rrr}
1&0&\mysfrac{5}{2}&\mysfrac{1}{2}&0&0\\
0&1&0&0&\mysfrac{1}{3}&0\\
0&0&\mysfrac{1}{2}&-\mysfrac{1}{2}&0&1\\
\end{array}\right)
\xrightarrow{\text{$2*$\textcircled{3}}}\left(\begin{array}{rrr|rrr}
1&0&\mysfrac{5}{2}&\mysfrac{1}{2}&0&0\\
0&1&0&0&\mysfrac{1}{3}&0\\
0&0&1&-1&0&2\\
\end{array}\right)
\]\[\left(\begin{array}{rrr|rrr}
1&0&\mysfrac{5}{2}&\mysfrac{1}{2}&0&0\\
0&1&0&0&\mysfrac{1}{3}&0\\
0&0&1&-1&0&2\\
\end{array}\right)
\xrightarrow{\text{\textcircled{1}$-\mysfrac{5}{2}*$\textcircled{3}}}\left(\begin{array}{rrr|rrr}
1&0&0&3&0&-5\\
0&1&0&0&\mysfrac{1}{3}&0\\
0&0&1&-1&0&2\\
\end{array}\right)
\]
Hence, the inverse is
\[\left(\begin{array}{rrr}
3&0&-5\\
0&\mysfrac{1}{3}&0\\
-1&0&2\\
\end{array}\right)
\]
			\setcounter{enumi}{7}
		\item {\bf Answer: }similarly to previous example,
\[\left(\begin{array}{rrr|rrr}
1&0&1&1&0&0\\
-1&1&1&0&1&0\\
-1&-2&-3&0&0&1\\
\end{array}\right)
\xrightarrow{\text{\textcircled{2}$+1*$\textcircled{1}}}\left(\begin{array}{rrr|rrr}
1&0&1&1&0&0\\
0&1&2&1&1&0\\
-1&-2&-3&0&0&1\\
\end{array}\right)
\]\[\left(\begin{array}{rrr|rrr}
1&0&1&1&0&0\\
0&1&2&1&1&0\\
-1&-2&-3&0&0&1\\
\end{array}\right)
\xrightarrow{\text{\textcircled{3}$+1*$\textcircled{1}}}\left(\begin{array}{rrr|rrr}
1&0&1&1&0&0\\
0&1&2&1&1&0\\
0&-2&-2&1&0&1\\
\end{array}\right)
\]\[\left(\begin{array}{rrr|rrr}
1&0&1&1&0&0\\
0&1&2&1&1&0\\
0&-2&-2&1&0&1\\
\end{array}\right)
\xrightarrow{\text{\textcircled{3}$+2*$\textcircled{2}}}\left(\begin{array}{rrr|rrr}
1&0&1&1&0&0\\
0&1&2&1&1&0\\
0&0&2&3&2&1\\
\end{array}\right)
\]\[\left(\begin{array}{rrr|rrr}
1&0&1&1&0&0\\
0&1&2&1&1&0\\
0&0&2&3&2&1\\
\end{array}\right)
\xrightarrow{\text{$\mysfrac{1}{2}*$\textcircled{3}}}\left(\begin{array}{rrr|rrr}
1&0&1&1&0&0\\
0&1&2&1&1&0\\
0&0&1&\mysfrac{3}{2}&1&\mysfrac{1}{2}\\
\end{array}\right)
\]\[\left(\begin{array}{rrr|rrr}
1&0&1&1&0&0\\
0&1&2&1&1&0\\
0&0&1&\mysfrac{3}{2}&1&\mysfrac{1}{2}\\
\end{array}\right)
\xrightarrow{\text{\textcircled{2}$-2*$\textcircled{3}}}\left(\begin{array}{rrr|rrr}
1&0&1&1&0&0\\
0&1&0&-2&-1&-1\\
0&0&1&\mysfrac{3}{2}&1&\mysfrac{1}{2}\\
\end{array}\right)
\]
Hence the inverse matrix is
\[\left(\begin{array}{rrr}
1&0&0\\
-2&-1&-1\\
\mysfrac{3}{2}&1&\mysfrac{1}{2}\\
\end{array}\right)
\]
	\end{enumerate}
\end{enumerate}
\textbf{Section 1.6}
\begin{enumerate}[1]
	\setcounter{enumi}{3}
	\item 
		\begin{enumerate}[\bf(a)]
			\item {\bf Answer: }by direct computations,
				\[\mybramatii{O}{I}{I}{O}\mybramatii{B_{11}}{B_{12}}{B_{21}}{B_{22}}=
				\mybramatii{B_{21}}{B_{22}}{B_{11}}{B_{12}}=
\left(\begin{array}{rrrr}
3&1&1&1\\
3&2&1&2\\
1&1&1&1\\
1&2&1&1\\
\end{array}\right)
				\]
			\item 
				\[\mybramatii{C}{O}{O}{C}\mybramatii{B_{11}}{B_{12}}{B_{21}}{B_{22}}=
				\mybramatii{CB_{11}}{CB_{12}}{CB_{21}}{CB_{22}}=
\left(\begin{array}{rrrr}
1&1&1&1\\
0&1&0&0\\
3&1&1&1\\
0&1&0&1\\
\end{array}\right)
				\]
			\item 
				\[\mybramatii{D}{O}{O}{I}\mybramatii{B_{11}}{B_{12}}{B_{21}}{B_{22}}=
				\mybramatii{DB_{11}}{DB_{12}}{B_{21}}{B_{22}}=
\left(\begin{array}{rrrr}2&2&2&2\\2&4&2&2\\3&1&1&1\\3&2&1&2\end{array}\right)
				\]
			\item 
				\[\mybramatii{E}{O}{O}{E}\mybramatii{B_{11}}{B_{12}}{B_{21}}{B_{22}}=
				\mybramatii{EB_{11}}{EB_{12}}{EB_{21}}{EB_{22}}=
\left(\begin{array}{rrrr}1&2&1&1\\1&1&1&1\\3&2&1&2\\3&1&1&1\end{array}\right)
				\]
		\end{enumerate}
\end{enumerate}
\textbf{Section 2.1}
\begin{enumerate}[1]
	\setcounter{enumi}{2}
	\item \begin{enumerate}[\bf(a)]
	\setcounter{enumi}{3}
\item {\bf Answer: }by direct computations
\[\left|\begin{array}{rrr}
4&3&0\\
3&1&2\\
5&-1&-4\\
\end{array}\right|=
4*\left|\begin{array}{rr}
1&2\\
-1&-4\\
\end{array}\right|-3*
\left|\begin{array}{rr}
3&2\\
5&-4\\
\end{array}\right|=4*(-2)-3*(-22)=58
\]
	\setcounter{enumi}{5}
\item {\bf Answer: }by direct computations
\[\left|\begin{array}{rrr}
2&-1&2\\
1&3&2\\
5&1&6\\
\end{array}\right|=
2*
\left|\begin{array}{rr}
3&2\\
1&6\\
\end{array}\right|+1*
\left|\begin{array}{rr}
1&2\\
5&6\\
\end{array}\right|
+2*
\left|\begin{array}{rr}
1&3\\
5&1\\
\end{array}\right|=2*16-4-28=0\]
\item
	\[
\left|\begin{array}{rrrr}
2&0&0&1\\
0&1&0&0\\
1&6&2&0\\
1&1&-2&3\\
\end{array}\right|=1*
\left|\begin{array}{rrr}
2&0&1\\
1&2&0\\
1&-2&3\\
\end{array}\right|=
2*\left|\begin{array}{rr}
2&0\\
-2&3\\
\end{array}\right|
+1*
\left|\begin{array}{rr}
1&2\\
1&-2\\
\end{array}\right|=12-4=8
	\]
\item
	\[
\left|\begin{array}{rrrr}
2&1&2&1\\
3&0&1&1\\
-1&2&-2&1\\
-3&2&3&1\\
\end{array}\right|=-3*
\left|\begin{array}{rrr}
1&2&1\\
2&-2&1\\
2&3&1\\
\end{array}\right|-1*
\left|\begin{array}{rrr}
2&1&1\\
-1&2&1\\
-3&2&1\\
\end{array}\right|+1*
\left|\begin{array}{rrr}
2&1&2\\
-1&2&-2\\
-3&2&3\\
\end{array}\right|=\]\[=
-3*\mybra{1*
\left|\begin{array}{rr}
-2&1\\
3&1\\
\end{array}\right|-2*
\left|\begin{array}{rr}
2&1\\
2&1\\
\end{array}\right|
+1*
\left|\begin{array}{rr}
2&-2\\
2&3\\
\end{array}\right|}-
	\]
\[-1*\mybra{+2\left|\begin{array}{rr}
2&1\\
2&1\\
\end{array}\right|
-1\left|\begin{array}{rr}
-1&1\\
-3&1\\
\end{array}\right|
+1\left|\begin{array}{rr}
-1&2\\
-3&2\\
\end{array}\right|
}+\]
\[+1*\mybra{+2\left|\begin{array}{rr}
2&-2\\
2&3\\
\end{array}\right|
-1\left|\begin{array}{rr}
-1&-2\\
-3&3\\
\end{array}\right|
+2\left|\begin{array}{rr}
-1&2\\
-3&2\\
\end{array}\right|
}=\]
\[=-3(-5+10)-(-2+4)+1(20+9+8)=-15-2+37=20\]
	\end{enumerate}
\end{enumerate}
\textbf{Section 3.6}
\begin{enumerate}[1]
	%\setcounter{enumi}{2}
	\item \begin{enumerate}[\bf(a)]
	\item {\bf Answer: }the row echelon form of matrix is
			\[\left(\begin{array}{rrr}
			1&3&2\\
			0&1&0\\
			0&0&0\\
			\end{array}\right)
			\]
			Recall that both the row space and the null space are unaffected by elementary row operations.
			Hence the row space is spanned by $(1,3,2)$ and $(0,1,0)$, while the null space is spanned by $(-2,0,1)$ (as solutions
			to homogeneous system corresponding to row echelon form would be $y=0,\;x=-2z,\;z\in\mathbb{R}$). 
			
			Now, rank of matrix is 2, hence to find basis for column space, it suffices to find two linearly independent columns
			of original matrix. As textbook says, we may select these that contain leading ones in row echelon form.
			Hence, first two columns $(1,2,4)^T$ and $(3,1,7)^T$ would do.
	\item {\bf Answer: }one possible row echelon form would be
\[\left(\begin{array}{rrrr}
-3&1&3&4\\
1&2&-1&-2\\
-3&8&4&2\\
\end{array}\right)
\xrightarrow{\text{\textcircled{1}$\leftrightarrow$\textcircled{2}}}\left(\begin{array}{rrrr}
1&2&-1&-2\\
-3&1&3&4\\
-3&8&4&2\\
\end{array}\right)
\]\[\left(\begin{array}{rrrr}
1&2&-1&-2\\
-3&1&3&4\\
-3&8&4&2\\
\end{array}\right)
\xrightarrow{\text{\textcircled{2}$+3*$\textcircled{1}}}\left(\begin{array}{rrrr}
1&2&-1&-2\\
0&7&0&-2\\
-3&8&4&2\\
\end{array}\right)
\]\[\left(\begin{array}{rrrr}
1&2&-1&-2\\
0&7&0&-2\\
-3&8&4&2\\
\end{array}\right)
\xrightarrow{\text{\textcircled{3}$+3*$\textcircled{1}}}\left(\begin{array}{rrrr}
1&2&-1&-2\\
0&7&0&-2\\
0&14&1&-4\\
\end{array}\right)
\]\[\left(\begin{array}{rrrr}
1&2&-1&-2\\
0&7&0&-2\\
0&14&1&-4\\
\end{array}\right)
\xrightarrow{\text{$\mysfrac{1}{7}*$\textcircled{2}}}\left(\begin{array}{rrrr}
1&2&-1&-2\\
0&1&0&-\mysfrac{2}{7}\\
0&14&1&-4\\
\end{array}\right)
\]\[\left(\begin{array}{rrrr}
1&2&-1&-2\\
0&1&0&-\mysfrac{2}{7}\\
0&14&1&-4\\
\end{array}\right)
\xrightarrow{\text{\textcircled{3}$-14*$\textcircled{2}}}\left(\begin{array}{rrrr}
1&2&-1&-2\\
0&1&0&-\mysfrac{2}{7}\\
0&0&1&0\\
\end{array}\right)
\]
\[\left(\begin{array}{rrrr}
1&2&-1&-2\\
0&1&0&-\mysfrac{2}{7}\\
0&0&1&0\\
\end{array}\right)
\]
We see that row space is spanned by $(1,2,-1,-2)$, $(0,1,0,-\mysfrac{2}{7})$ and $(0,0,1,0)$, null space is spanned by $(\mysfrac{10}{7}
,\mysfrac{2}{7},0,1)$. As rank is 3, it suffices to find three linearly independent colums, first three would do, thus basis for
column space is $(-3,1,-3)^T$, $(1,2,8)^T$ and $(3,-1,4)^T$.
	\item {\bf Answer: }one possible row echelon form (with unpivoted rows) would be
			\[\left(\begin{array}{rrrr}
			1&3&-2&1\\
			0&-5&7&0\\
			0&0&4&3\\
			\end{array}\right)
			\]
			Hence, we see that row space is generated by $(1,3,-2,1)$, $(0,-5,7,0)$ and $(0,0,4,3)$, while null space is spanned by
			$(-3.7,-2.1,-3,4)$. As rank is 3, it suffices to find 3 linearly independent columns, first three would do,
			hence basis for column space is $(1,2,3)^T$, $(3,1,4)^T$ and $(-2,3,5)^T$.
	\end{enumerate}
	\setcounter{enumi}{13}
	\item {\bf Answer: }as the null space is the same for $A$ and its reduced row echelon form, we see that the vectors
		$(-1,-4,0,1)$ and $(-2,-1,1,0)$ are in the null space of $A$, hence for columns $\mathbf{a_i}$ the following system holds
		\[\begin{cases}-\mathbf{a_1}-4\mathbf{a_2}+\mathbf{a_4}=\mathbf{0}\\-2\mathbf{a_1}-\mathbf{a_2}+\mathbf{a_3}=\mathbf{0}\end{cases}\]
		solving it one gets
		\[\mathbf{a_3}=2\mathbf{a_1}+\mathbf{a_2}=(-2,7,11,1)^T\]
		\[\mathbf{a_4}=\mathbf{a_1}+4\mathbf{a_2}=(13,-7,30,-3)^T\]
	\item \begin{enumerate}[\bf(a)]
	\item {\bf Answer: }null space of $A$ should coincide with that of $U$, hence its basis is $(1,2,0,-5,1)^T$ and $(-2,-3,1,0,0)^T$.
	\item \begin{enumerate}[\bf(i)]
		\item{\bf Answer: }as null space of $A$ has basis we've found above, and difference of any two solutions to $A\mathbf{x}=\mathbf{b}$
				belong to null space of $A$ (and conversely, sum of solution and element of null space is again solution),
				we have that all solutions are given by
				\[\mathbf{x_0}+N(A)=\mysetn{\mathbf{x_0}+\alpha(1,2,0,-5,1)^T+\beta(-2,-3,1,0,0)^T}{\alpha,\beta\in\mathbb{R}}\]
		\item{\bf Answer: }as in previous problem, since $(1,2,0,-5,1)^T$ and $(-2,-3,1,0,0)^T$ are in the nullspace of $A$
			and from the information given in this subproblem about $A\mathbf{x_0}=\mathbf{b}$, we have
			\[\begin{cases}\mathbf{a_1}+2\mathbf{a_2}-5\mathbf{a_4}+\mathbf{a_5}=\mathbf{0}\\-2\mathbf{a_1}-3\mathbf{a_2}+\mathbf{a_3}=
				\mathbf{0}\\3\mathbf{a_1}+2\mathbf{a_2}+2\mathbf{a_4}=\mathbf{b}\end{cases}\]
			hence
			\[\mathbf{a_3}=2\mathbf{a_1}+3\mathbf{a_2}=(1,8,3,-1)^T\]
			\[\mathbf{a_4}=\mathbf{b}-3\mathbf{a_1}-2\mathbf{a_2}=(-4,-2,6,8)^T\]
			\[\mathbf{a_5}=-\mathbf{a_1}-2\mathbf{a_2}+5\mathbf{a_4}=(-20,-15,27,40)^T\]
		\end{enumerate}
	\end{enumerate}
		\setcounter{enumi}{20}
	\item {\bf Answer: }indeed, $\mbox{rank}(A+B)$ is the dimension of the space, spanned by columns of $A+B$, which
		cannot be bigger than the dimension of space $R$ spanned by columns of $A$ \textit{and} columns of $B$ (since $R$
	contains column space of $A+B$, as $\forall\alpha_i,\;\sum_i\alpha_i(\mathbf{a_i}+\mathbf{b_i})=\sum_i\alpha_i\mathbf{a_i}+
		\sum_i\alpha_i\mathbf{b_i}\subset\mbox{span}\mycbra{\mycbra{\mathbf{a_i}}_i,\mycbra{\mathbf{b_i}}_i}$. And
		in turn, the dimension of $R$ cannot exceed
		$\mbox{rank}(A)+\mbox{rank}(B)$, as if we let $a:=\mbox{rank}(A)$, and $b:=\mbox{rank}(B)$, we have that column
		space of $A$ (hence, in particular, all of $\mathbf{a_i}$), can be spanned by some $a$ vectors and column space
		of $B$ (hence, in particular, all of $\mathbf{b_i}$), can be spanned by some $b$ vectors, hence $R$ can be spanned
		by $a+b$ vectors, and its dimension cannot exceed this number. Thus, we get, as required
		\[\mbox{rank}(A+B)\leq\dim R\leq\mbox{rank}(A)+\mbox{rank}(B)\]
	\item \begin{enumerate}[\bf(a)]
			\item {\bf Answer: }first, we have
				\[BAx=0\iff B(Ax)=0\iff Ax=0\implies N(BA)=N(A)\]
				and therefore, as $BA$ and $A$ have the same number of columns and sum of rank and dimension
				of null space equals to number of columns, we have that rank of $A$ and $BA$ are the same.
			\item {\bf Answer: }we will show directly that column space of two matrices is the same, that is
				$AC(\mathbb{R}^m)=A(\mathbb{R}^m)$ (here we view matrices as linear transformations)
				\[y\in AC(\mathbb{R}^m)\iff y=ACx\iff y=Az\iff y\in A(\mathbb{R}^m)\]
				where $y=ACx\iff y=Az$ is because $C$ invertible, and we have $Cx=z\iff x=C^{-1}z$.
		\end{enumerate}
		\setcounter{enumi}{24}
	\item \begin{enumerate}[\bf(a)]
			\item {\bf Answer: }If $AB=O$, then this means that $\forall 1\leq i\leq n$ we have $A(\mathbf{b_i})=\mathbf{0
				}$, hence $\mathbf{b_i}\in N(A)$ and thus the whole column space of $B$ is inside the null space of $A
				$. Conversely, if column space of $B$ is contained in null space of $A$ this means that for all $i$,
				$A\mathbf{b_i}=\mathbf{0}$, hence $AB=O$.
			\item {\bf Answer: }as show in previous item, if $AB=O$, we have that null-space of $A$ contains
				column space of $B$, hence nullity $a$ of $A$ is no less than the rank $b$ of $B$. Now, if we let
				$r$ to be the rank of $A$, we have
				$a+r=n,\;a\geq b\implies b+r\leq n$ and we are done.
		\end{enumerate}
\end{enumerate}
\end{document}
