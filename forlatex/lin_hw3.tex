\documentclass[8pt]{article} % use larger type; default would be 10pt

\usepackage{mystyle}

\title{MATH1030 C,D\\Suggested solutions to HW1}
\author{Alex Leontiev}
\begin{document}
\maketitle
	\renewcommand{\v}{\mathbf{v}}
	\renewcommand{\u}{\mathbf{u}}
	\renewcommand{\i}{\mathbf{i}}
	\renewcommand{\j}{\mathbf{j}}
	\renewcommand{\k}{\mathbf{k}}
	\newcommand{\w}{\mathbf{w}}
\textbf{Section 1.5}
\begin{enumerate}[1]
	\setcounter{enumi}{2}
	\item This sort of problem is basically just done by inspection. Since problem asks us to find "an" elementary matrix,
		it means that there should be only one. Hence, in each subproblem $A$ can be brought to $B$ via just one elementary
		row operation. We just need to find out which one, and write it as an elementary matrix (recall, that to do so
		one just needs to apply elementary row operation to the identity matrix and write down the result).
		\begin{enumerate}[(a)]
			\item {\bf Answer: }operation performed is the multiplication of a first row by $-2$, which corresponds to
				\[E=\mybramatii{-2}{0}{0}{1}\]
			\item {\bf Answer: }operation performed is interchange of second and third row, which corresponds to
				\[E=\mybramatiii{1}{0}{0}{0}{0}{1}{0}{1}{0}\]
			\item {\bf Answer: }operation performed is adding the second row to the third two times, which corresponds to
				\[E=\mybramatiii{1}{0}{0}{0}{1}{0}{0}{2}{1}\]
		\end{enumerate}
	\setcounter{enumi}{5}
	\item
		\begin{enumerate}[(a)]
			\item {\bf Answer: }basically, we just perform Gaussian elimination to bring $A$ to its row echelon form
				(which will be upper triangular by definition), noting the operations we've done. Question, however,
				suggests that we should use only three elementary operations, so as textbook does in examples, we
				shall use only row addition in the reduction process. Thus, what we'll get isn't strictly speaking
				"row echelon form" (it pivot elements won't be 1), but it will still be upper triangular. Once
				we are done with reduction, first second and third operations will correspond to $E_1$, $E_2$ and $E_3
				$ respectively.
				
				It's important to notice, that textbook writes matrices as $E_3E_2E_1A$,
				not as $E_1E_2E_3A$. The latter is incorrect, while the former is correct -- operations are applied
				"right-to-left" when in matrix notation. This is due to the way how multiplication and associativity
				work: $E_3E_2E_1A:=E_3(E_2(E_1A))$, so operations are applied in the order they supposed to, on
				contrary to $E_1E_2E_3A:=E_1(E_2(E_3A))$, which reverses the order. Anyway, here's how reduction goes
				\[A=\mybramatiii{2}{1}{1}{6}{4}{5}{4}{1}{3}\xrightarrow{\text{\textcircled{2}-3*\textcircled{1}}}
				\mybramatiii{2}{1}{1}{0}{1}{2}{4}{1}{3}\xrightarrow{\text{\textcircled{3}-2*\textcircled{1}}}
				\mybramatiii{2}{1}{1}{0}{1}{2}{0}{-1}{1}\xrightarrow{\text{\textcircled{3}+1*\textcircled{2}}}
				\mybramatiii{2}{1}{1}{0}{1}{2}{0}{0}{3}=:U
				\]
				and three operations done are written as elementary matrices as:
				\[{\text{\textcircled{2}-3*\textcircled{1}}}\implies E_1=\mybramatiii{1}{0}{0}{-3}{1}{0}{0}{0}{1}\]
				\[{\text{\textcircled{3}-2*\textcircled{1}}}\implies E_2=\mybramatiii{1}{0}{0}{0}{1}{0}{-2}{0}{1}\]
				\[{\text{\textcircled{3}+1*\textcircled{2}}}\implies E_3=\mybramatiii{1}{0}{0}{0}{1}{0}{0}{1}{1}\]
			\item {\bf Answer: }
				inverses of elementary matrices are found directly, as the inverses of corresponding elementary row
				operations, that is
				\[\mybra{\text{\textcircled{2}-$3*$\textcircled{1}}}^{-1}={\text{\textcircled{2}+3*\textcircled{1}}}
				\implies E_1^{-1}=\mybramatiii{1}{0}{0}{3}{1}{0}{0}{0}{1}\]
				\[\mybra{\text{\textcircled{3}-$2*$\textcircled{1}}}^{-1}={\text{\textcircled{3}+2*\textcircled{1}}}
				\implies E_2^{-1}=\mybramatiii{1}{0}{0}{0}{1}{0}{2}{0}{1}\]
				\[\mybra{\text{\textcircled{3}+$1*$\textcircled{2}}}^{-1}={\textcircled{3}-1*\textcircled{2}}
				\implies E_3^{-1}=\mybramatiii{1}{0}{0}{0}{1}{0}{0}{-1}{1}\]
				this gives us
				\[L=E_1^{-1}E_2^{-1}E_3^{-1}=\left(\begin{array}{rrr}1&0&0\\3&1&0\\2&-1&1\\\end{array}\right)\]
				which is lower-triangular, and indeed we have
				\[LU=\mybramatiii{2}{1}{1}{6}{4}{5}{4}{1}{3}=A\]
		\end{enumerate}
	\item 
		\begin{enumerate}[(a)]
			\item{\bf Answer: }we should note first that such expansion is not unique. One way to obtain it, is to perform
				Gaussian elimination on $E$, bringing it to reduced row echelon form, which will be identity as $A$
				is non-singular (recall, that only non-singular matrices can be written
				as product of elementary ones). Writing operations that we
				have done as elementary matrices $E_1$, $E_2$ up to $E_n$ (for some $n$), we will get thus
				(note the remark above about reversing the order)
				\[E_nE_{n-1}\hdots E_1A=I\implies A=E_1^{-1}E_2^{-1}\hdots E_n^{-1}\]
				this will give $A$ written as product of elementary matrices, since $E_i^{-1}$ are elementary,
				as $E_i$ are elementary. Let's perform the elimination:
				\[A=\mybramatii{2}{1}{6}{4}\xrightarrow{\text{\textcircled{1}$/2$}}
				\mybramatii{1}{\mysfrac{1}{2}}{6}{4}
				\xrightarrow{\text{\textcircled{2}-$6*$\textcircled{1}}}\mybramatii{1}{\mysfrac{1}{2}}{0}{1}
				\xrightarrow{\text{\textcircled{1}-\textcircled{2}$/2$}}\mybramatii{1}{0}{0}{1}
				\]
				Those operations can be written as elementary matrices as
				\[{\text{\textcircled{1}$/2$}}\implies E_1=\mybramatii{\mysfrac{1}{2}}{0}{0}{1}\]
				\[{\text{\textcircled{2}-$6*$\textcircled{1}}}\implies E_2=\mybramatii{1}{0}{-6}{1}\]
				\[{\text{\textcircled{1}-\textcircled{2}$/2$}}\implies E_3=\mybramatii{1}{-\mysfrac{1}{2}}{0}{1}\]
				inverses are correspondingly
				\[\mybra{\text{\textcircled{1}$/2$}}^{-1}={\text{2*\textcircled{1}}}
				\implies E_1^{-1}=\mybramatii{2}{0}{0}{1}\]
				\[\mybra{\text{\textcircled{2}-$6*$\textcircled{1}}}^{-1}={\text{\textcircled{2}+$6*$\textcircled{1}}}
				\implies E_2^{-1}=\mybramatii{1}{0}{6}{1}\]
				\[{\mybra{\text{\textcircled{1}+\textcircled{2}$/2$}}}^{-1}\implies 
					E_3^{-1}=\mybramatii{1}{\mysfrac{1}{2}}{0}{1}\]
				thus, after all
				\[A=E_1^{-1}E_2^{-1}E_3^{-1}=\mybramatii{2}{0}{0}{1}\cdot\mybramatii{1}{0}{6}{1}\cdot
				\mybramatii{1}{\mysfrac{1}{2}}{0}{1}.\]
			\item {\bf Answer: }as seen from computations above, $E_nE_{n-1}\hdots E_1A=I$, hence
				\[A^{-1}=E_3E_2E_1=\mybramatii{\mysfrac{1}{2}}{0}{0}{1}\cdot\mybramatii{1}{0}{-6}{1}\cdot
				\mybramatii{1}{-\mysfrac{1}{2}}{0}{1}\]
		\end{enumerate}
	\setcounter{enumi}{9}
		\begin{enumerate}[(a)]
			\setcounter{enumi}{3}
			\item{\bf Answer: }this is done by Gaussian elimination, as follows
				\[\left(\begin{array}{rr|rr}
					3&0&1&0\\
					9&3&0&1\\
				\end{array}\right)
				\xrightarrow{\text{\textcircled{2}$-3*$\textcircled{1}}}\left(\begin{array}{rr|rr}
					3&0&1&0\\
					0&3&-3&1\\
				\end{array}\right)
				\]\[\left(\begin{array}{rr|rr}
					3&0&1&0\\
					0&3&-3&1\\
				\end{array}\right)
				\xrightarrow{\text{$\mysfrac{1}{3}*$\textcircled{1}}}\left(\begin{array}{rr|rr}
					1&0&\mysfrac{1}{3}&0\\
					0&3&-3&1\\
				\end{array}\right)
				\]\[\left(\begin{array}{rr|rr}
					1&0&\mysfrac{1}{3}&0\\
					0&3&-3&1\\
				\end{array}\right)
				\xrightarrow{\text{$\mysfrac{1}{3}*$\textcircled{2}}}\left(\begin{array}{rr|rr}
					1&0&\mysfrac{1}{3}&0\\
					0&1&-1&\mysfrac{1}{3}\\
				\end{array}\right)
				\]
				Hence the inverse is \[\mybramatii{\mysfrac{1}{3}}{0}{-1}{\mysfrac{1}{3}}\]
			\setcounter{enumi}{5}
			\setcounter{enumi}{7}
	\end{enumerate}
	\item tesi me
\end{enumerate}
\end{document}
