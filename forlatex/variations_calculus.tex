\documentclass[12pt]{article} % use larger type; default would be 10pt

\usepackage{mystyle}

\newtheorem{prob}{Завдання}
\newcommand{\ds}{\;ds}
\newcommand{\dt}{\;dt}
\newcommand{\dx}{\;dx}
\newcommand{\dta}{\;d\tau}
\newcommand{\extr}{\mbox{\normalfont extr}}

\newtheorem{myulem}[mythm]{Лема}

\renewenvironment{myproof}[1][Доведення]{\begin{trivlist}
\item[\hskip \labelsep {\bfseries #1}]}{\myqed\end{trivlist}}

\title{Варіаційне числення (9 семестр)}
\author{Олексій Леонтьєв}

\begin{document}
\maketitle
\begin{prob}{\normalfont 5.3}\; $\int_0^1((x')^2+x^2)\dt\to\extr,\quad x(0)=0,\;x(1)=1$\end{prob}
Як і в усіх проблемах нижче, ми припустимо, що пошук екстремума ведеться на множині неперервно диференційованих функцій $C^1([0,1],\mathbb{R})$,
що задовольняють граничні умови.

За Теоремою 5.1 з \cite{tb} (оскільки $L(x,x',t)=(x')^2+x^2$ має неперервні похідні по всіх аргументах)
маємо, що екстремум може досягатися лише функцією, що задовольняє рівняння Ейлера, яке в даному разі пишеться як
\[x-x''=0\iff x(t)=ae^t+be^{-t}\]
Граничні умови приводять до
\[\begin{cases}a+b=0\\ae+be^{-1}=1\end{cases}\]
\[a=\frac{e}{e^2-1},\;b=-\frac{e}{e^2-1}\]
\[x(t)=\frac{e}{e^2-1}(e^t-e^{-t})\]

Помітимо, що максимума в задачі бути не може, адже за даних граничних умов навіть $\int_0^1x^2\dt$ може бути зробленим нескінченно
великим, підставляючи параболи в ролі $x(t)$. Щодо мінімуму, то він існує, і знайдена вище функція $x(t)$ якраз ним і є. Дійсно, нехай
існує інша функція $x^*(t)$, що задовольняє ті ж граничні умови і дає менше значення функціоналу. Тоді $x^*=x+h$, де $h(0)=h(1)=0$ і ми маємо
\[\int_0^1\mybra{\frac{e}{e^2-1}(e^t-e^{-t})+h(t)}^2+\mybra{\frac{e}{e^2-1}(e^t+e^{-t})+h'(t)}^2\dt\leq\]
\[\leq\int_0^1\mybra{\frac{e}{e^2-1}(e^t-e^{-t})}^2+\mybra{\frac{e}{e^2-1}(e^t+e^{-t})}^2\dt\]
\[\int_0^1(h^2(t)+h'^2(t))\dt+2\frac{e}{e^2-1}\int_0^1(e^t-e^{-t})h(t)+(e^t+e^{-t})h'(t)\dt\leq0\]
Помітимо, що враховуючи $h(0)=h(1)=0$ і виконуючи інтегрування частинами, отримуєм
\[\int_0^1e^th'(t)\dt=\cancel{e^th(t)\bigg|_0^1}-\int_0^1e^th(t)\dt\]
\[\int_0^1e^{-t}h'(t)\dt=\cancel{e^{-t}h(t)\bigg|_0^1}+\int_0^1e^{-t}h(t)\dt\]
і таким чином другий доданок в нерівності вище знищується, ми маємо
\[\int_0^1h^2(t)+h'^2(t)\dt\leq0\implies h(t)\equiv0\implies x=x^*\]
\begin{prob}{\normalfont 5.19}\; $\int_0^1(x'_1x'_2+6tx_1+12t^2x_2)\dt\to\extr,\quad x_1(0)=x_2(0)=0,\;x_1(1)=x_2(1)=1$
\end{prob}
Як і попередня задача, ця не має максимуму, адже для $(x_1(t),x_2(t))=(f(t),f(t))$, де $f(t)$ -- (єдина) парабола, що задовольняє 
$f(0)=0$, $f(1)=1$, $f(1/2)=A>1$, маємо $f(t)\geq0$ на $[0,1]$, $x'_1x'_2=(f'(t))^2\geq0$, і значення функціонала прямує до нескінченності при
$A\to\infty$. Щодо мінімуму, його також не існує, адже для $x_1(t):=t-A\sin\pi t$ та $x_2(t):=t+A\sin\pi t$, що задовольняють граничним
умовам, маємо
\[\int_0^1(x'_1x'_2+6tx_1+12t^2x_2)\dt=\]
\[=\int_0^1(1-A\pi\cos\pi t)(1+A\pi\sin\pi t)+6t(t-A\pi\cos\pi t)+12t^2(t+A\pi\sin\pi t)\dt=\]
\[=1-A^2\int_0^1\pi^2\cos^2\pi t\dt-A\int_0^16t\pi\sin\pi t\dt+A\int_0^112t^2\pi\sin\pi t\dt+\int_0^1(6t^2+12t^3)\dt\]
і останній вираз прямує до $-\infty$ при $A\to+\infty$, тому мінімуму також не існує.
\begin{prob}{\normalfont 5.36}\; $\int_1^2t^2(x')^2\dt+2x(1)+x^2(2)\to\extr$
\end{prob}
Помітимо, що це функціонал не має максимуму. Тому ми лише можемо шукати мінімум.
Запишемо рівняння Ейлера та умови трансверсальності
\[\begin{cases}
	0=4tx'+2t^2x''\iff 2x'+tx''=0\iff x+tx'=C\iff tx(t)=Ct+D\\
	2x'(1)=2\implies D=-1\\
	8x'(2)=-2x(2)\implies C=-\frac{1}{2}
\end{cases}\]
Щоб показати, що знайдене дійсно є мінімумом. Це так, адже для $-1/2-1/t+h(t)$ і з припущення, що це не збільшує значення функціонала, маємо
\[\int_1^2t^2\mybra{\frac{2h'}{t^2}+(h')^2}\dt+2h(1)-2h(2)+h^2(2)\leq0\]
\[\cancel{2\int_1^2h'(t)\dt}+\int_1^2t^2\mybra{(h')^2}\dt+\cancel{2h(1)}-\cancel{2h(2)}+h^2(2)\leq0\implies h'(t)\equiv h(2)=0\implies h(t)\equiv0\]
\begin{prob}{\normalfont 9.18}\; $\int_0^1x^2(x')^2\dt\to\extr,\quad x(0)=1,\;x(1)=\sqrt{2}$\end{prob}
	Записуючи рівняння Ейлера, маємо
	\[2x(x')^2=4x(x')^2+2x^2x''\iff (x')^2+xx''=0\iff xx'=C\iff x=\sqrt{At+B}\]
	застосовуючи граничні умови, бачимо, що єдиний можливий екстремум -- це $x(t)=\sqrt{t+1}$. Зауважимо також, що задача
	максимуму немає (знову ж таки, застосовуючи параболи, що проходять через $(0,0)$, $(1,\sqrt{2})$ та $(0.5,A)$ для великих $A$). Залишається
	лише перевірити достатні умови, щоб показати, що знайдено локальний мінімум.
	\begin{enumerate}
		\item Рівняння Ейлера -- виконується за побудовою.
		\item граничні умови -- виконуються за побудовою.
		\item посилену умову Лежандра -- виконується, адже
			\[L_{x'x'}=2x^2=2(t+1)>0\mbox{ на }[0,1]\]
		\item посилену умову Якобі. Рівняння Якобі має вигляд
			\[2x^2h''+2xx'h'+(xx''-(x')^2)h=0\]
			\[2x^2h''+2xx'h'+(xx''-(x')^2)h=0\]
			\[h''+\frac{1}{2(t+1)}h'-\frac{1}{4(t+1)^2}h=0\]
			\[4(t+1)^2h''+2(t+1)h'-h=0\]
			роблячи заміну $s:=t+1$ ($d^nh/ds^n=d^nh/dt^n$) маємо 
			\[4s^2h''+2sh'-h=0\]
			є рівнянням Ейлера. Як викладено в підручнику \cite[с. 290]{samoilenko}, ми знаходимо характеристичне
			рівняння (\cite[(3.28)]{samoilenko})
			\[4s^2-2s-1=0\iff s=\frac{1\pm\sqrt{5}}{4}\]
			Оскільки їх різниця не є цілим числом, загальний розв’язок записується як
			\[h=c_1(t+1)^{\frac{1-\sqrt{5}}{4}}\sum_{i=0}^\infty c^{(1)}_i(t+1)^i+
			c_1(t+1)^{\frac{1+\sqrt{5}}{4}}\sum_{i=0}^\infty c^{(2)}_i(t+1)^i\]
			і підстановка показує, що загальний розв’язок є рівним
			\[h=c_1(t+1)^{\frac{1-\sqrt{5}}{4}}+c_2(t+1)^{\frac{1+\sqrt{5}}{4}}\]
			Підставляючи початкові умови $h(0)=0,\;h'(0)=1$ маємо
			\[h=\frac{2}{\sqrt{5}}\mybra{(t+1)^{\frac{1-\sqrt{5}}{4}}-(t+1)^{\frac{1+\sqrt{5}}{4}}}\]
			і оскільки ця функція не має нулів на $(0,1]$, посилена умова Якобі також виконується.
	\end{enumerate}
	Помітимо, що це все лише показує, що $\sqrt{t+1}$ є \textbf{локальним} мінімумом. Доведення, що мінімум є глобальним
	все ж вимагає прямого аргументу. Нехай $h$ -- неперервна, $h(0)=h(1)=0$, маємо
	\[\int_0^1(x+h)^2(x'+h')^2\dt\leq\int_0^1x^2(x')^2\dt\]
	\[\int_0^1(2x^2x'h'+x^2h'^2+2hxx'^2+h^2x'^2+4xhx'h'+2h^2x'h'+2xhh'^2+h^2h'^2)\dt\leq0\]
	Оскільки $x=\sqrt{t+1}$, з попереднього маємо
	\[\int_0^1(\bcancel{xh'}+x^2h'^2+\bcancel{hx'}+h^2x'^2+\cancel{2hh'}+{2h^2x'h'}+2xhh'^2+h^2h'^2)\dt\leq0\]
	оскільки
	\[\int_0^12hh'=h^2\bigg|_0^1=0\]
	\[\int_0^1(xh'+hx')\dt=xh\bigg|_0^1=0\]
	залишається
	\[\int_0^1(x^2h'^2+h^2x'^2+{2h^2x'h'}+2xhh'^2+h^2h'^2)\dt\leq0\]
	оскільки 
	\[\int_0^12xx'hh'\dt=\int_0^1hh'\dt=\frac{h^2}{2}\bigg|_0^1=0\]
	це еквівалентне
	\[\int_0^1(2xx'hh'+x^2h'^2+h^2x'^2+{2h^2x'h'}+2xhh'^2+h^2h'^2)\dt\leq0\]
	\[\int_0^1((xh'+hx')^2+2hh'(xh'+hx')+h^2h'^2)\dt\leq0\]
	\[\int_0^1(xh'+hx'+hh')^2\dt\leq0\]
	\[hx'+hx'+hh'\equiv0\]
\begin{prob}{\normalfont 9.37}\; $\int_0^{\frac{\pi}{4}}(4x^2-(x')^2)\dt\to\extr,\quad x(0)=1,\;x(\pi/4)=0$\end{prob}
	Рівняння Ейлера
	\[8x+2x''=0\iff x=\alpha\cos2t+\beta\sin2t\]
	Підставляючи граничні умови, єдина можливий екстремаль це
	\[x=\cos2t\]
	Перевіримо його за достатніми умовами
	\begin{enumerate}
		\item Рівняння Ейлера -- виконується
		\item граничні умови -- виконуються
		\item посилену умову Лежандра.
			\[L_{x'x'}=-2<0\]
			\textbf{не} виконується
	\end{enumerate}
	Помітимо, що оскільки не лише сильна, а і звичайна умова Лежандра не виконується, теорема про умову Лежандра каже нам,
	що знайдена функція \textbf{не є} екстремаллю, а отже задача не має мінімумів чи максимумів в принципі, адже єдина функція, як могла б бути
	екстремаллю, не є такою.
\begin{prob}{\normalfont 10.17}\; $\int_0^1(x')^2\dt\to\extr,\quad\int_0^1x^2\dt=2,\;x(0)=x(1)=0$\end{prob}
	Запишемо Лагранжіан
	\[L=\lambda_1x'^2+\lambda_2x^2\]
	і рівняння Ейлера
	\[2\lambda_2x=2\lambda_1x''\]
	Якщо $\lambda_1=0$, то $\lambda_2x=0$ і або $x\equiv0$ (що не є допустимим через умову $\int_0^1x^2\dt=2$)
	або $\lambda_2=0$ і всі множники рівні нулю. Будь-який випадок не є припустимим, і тому $\lambda_1\neq0$,
	і без втрати загальності $\lambda_1=1$, тобто ми маємо
	\[x''=\lambda x\]
	Розв’язки в загальному випадку
	\[x(t)=Ae^{\sqrt{\lambda}t}+Be^{-\sqrt{\lambda}t}\]
	Підставляючи умову $x(0)=0$, маємо
	\[x(t)=A(e^{\sqrt{\lambda}t}-e^{-\sqrt{\lambda}t}\]
	а підставляючи $x(1)=0$ маємо додаткову умову $\exp(2\sqrt{\lambda})=1\iff \sqrt{\lambda}=n\pi i,\;n\in\mathbb{Z}$ що в свою чергу дає
	\[x(t)=A(e^{n\pi it}-e^{-n\pi it})=A(2i\sin\pi nt\]
	а оскільки $x(t)$ має бути дійсно-значним, маємо
	\[x(t)=A\sin(n\pi t),\;A\in\mathbb{R}\]
	а умова $\int_0^1x^2\dt=2$ дає
	\[x(t)={2}\sin(n\pi t),\;n\neq0\]
	Які дають значення функціонала
	\[\int_0^1(x')^2\dt=2n^2\pi^2\]
	і таким чином, задача не має максимуму. Щодо мінімуму, помітимо, що для кожної тригонометричної суми виду
	\[f(t)=\sum_{0<\myabs{n}<N}a_n\sin n\pi t\]
	маємо
	\[\int_0^1f^2(t)\dt=\frac{1}{2}\sum a_n^2\]
	\[\int_0^1f'^2(t)\dt=\frac{1}{2}\sum{a_n^2n^2\pi^2}\geq\pi^2\int_0^1f^2(t)\]
	і оскільки кожна неперервна функція на $[0,1]$ рівна нулю в кінцевих точках, може
	бути розширена до непарної на $[-1,1]$. Остання, в свою чергу, може бути рівномірно наближена вищеописаними сумами, а тому
	нерівність вищу справджується також і для неперервних функцій, що задовольняють граничні умови задачі, тому для них
	\[\int_0^1f'^2(t)\dt\geq 2\pi^2\]
	і ми бачимо, що задача має мінімум, який досягається лише функціями $\sin(\pm\pi t)$.
\begin{thebibliography}{9}
\bibitem{tb}
Моклячук М. П. \emph{Варіаційне числення. Екстремальні задачі.} --
Київ: 2003. 380 с.
\bibitem{samoilenko}
Самойленко А.М., Перестюк М.О., Парасюк I.О. \emph{Диференцiальнi рiвняння: Підручник.} -- К.: Либiдь, 2003р. -- 600с.
\end{thebibliography}
\end{document}
