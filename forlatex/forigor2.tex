
\documentclass[10pt]{article} % use larger type; default would be 10pt

%%\usepackage[T1,T2A]{fontenc}
%%\usepackage[utf8]{inputenc}
%%\usepackage[english,ukrainian]{babel} % може бути декілька мов; остання з переліку діє по замовчуванню. 
\usepackage{enumerate}
\usepackage{CJKutf8}
\usepackage{mystyle}

\title{}
\author{}
\begin{document}
\maketitle
Indeed, let $\omega=fdx_1\wedge dx_2\wedge\hdots\wedge dx_n$. Then,
\[df=\frac{\partial f}{\partial x_1}dx_1\wedge dx_1\wedge dx_2\wedge\hdots\wedge dx_n+
\frac{\partial f}{\partial x_2}dx_2\wedge dx_1\wedge dx_2\wedge\hdots\wedge dx_n+\dots+
\frac{\partial f}{\partial x_n}dx_n\wedge dx_1\wedge dx_2\wedge\hdots\wedge dx_n+\dots=0\]
as every summand contains $d$'s for the same variable twice and we know that $dx_i\wedge dx_i=0$.
%%\begin{thebibliography}{9}
%%\bibitem{gelbaum}Gelbaum, B.R. and Olmsted, J.M.H.. Counterexamples in Analysis. Dover Publications. 2003
%%\end{thebibliography}
\end{document}
