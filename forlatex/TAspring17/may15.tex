\documentclass[12pt]{article} % use larger type; default would be 10pt

\usepackage{enumerate}
\usepackage{xeCJK}
\usepackage{mystyle}
\usepackage{ruby}
\usepackage{longtable}
\usepackage{hyperref}
\usepackage{amsthm,amssymb}

\theoremstyle{definition}
\newtheorem{question}{問}

\setCJKmainfont[AutoFakeBold=true]{Hiragino Mincho Pro} %my Mac
%\setCJKmainfont{MS PGothic} %AJP windows
%\renewcommand\rubysep{-5ex}
\newcommand{\kana}[2]{\ruby{#1}{#2}}
\newcommand{\mytabra}[1]{$\myabra{\mbox{#1}}$}

\title{2017年度S1数理科学基礎演習・微積分(理二三21−24)第3回}
\begin{document}
\maketitle
\begin{question}
	\begin{enumerate}[(1)]
		\item \begin{equation*}
				\begin{array}[]{c}
					f_x=y^2+2xy^3\\
					f_y=2xy+3x^2y^2
				\end{array}
			\end{equation*}
		\item \begin{equation*}
				\begin{array}[]{c}
					f_x=y\cos(xy)\\
					f_y=x\cos(xy).
				\end{array}
			\end{equation*}
	\end{enumerate}
\end{question}
\begin{question}
	\begin{enumerate}[(1)]
		\item \begin{equation*}
				\begin{array}[]{c}
					f_{xx}=2ax\\
					f_{xy}=f_{yx}=b\\
					f_{yy}=2c
				\end{array}
			\end{equation*}
		\item\begin{equation*}
				\begin{array}[]{c}
					f_{xx}=\left( 2a+4a^2x^2 \right)\exp\left( ax^2+by^2 \right)\\
					f_{xy}=f_{yx}= 4abxy\exp\left( ax^2+by^2 \right)\\
					f_{yy}=\left( 2b+4b^2y^2 \right)\exp\left(  ax^2+by^2\right).
				\end{array}
			\end{equation*}
	\end{enumerate}
\end{question}
\begin{question}
	\begin{enumerate}[(1)]
		\item 原点が最大になる(もちろん、極大にもなる):\begin{equation*}
				f(x,y)=-x^2+2xy-3y^2=-(x-y)^2-2y^2
			\end{equation*}
			になるので、yは0と等しくなかったら、$f(x,y)\le-2y^2<0=f(0,0)$になる。更に、$y$が0と等しくて、xは0でなかったら、
			$f(x,y)=-x^2<0$になる。
		\item 原点が極大にならない(もちろん、最大にもならない):\begin{equation*}
				f(x,y)=-x^2+3xy-2y^2=-\left( x-\frac{3}{2}y \right)^2+\frac{1}{4}y^2
			\end{equation*}
			になるので、$(x_n,y_n):=\left( \frac{3}{2n},\frac{1}{2n} \right)$ が原点に近づくが、$f(x_n,y_n)>0=f(0,0)$である。
	\end{enumerate}
\end{question}
\begin{question}
	\begin{enumerate}[(1)]
		\item 原点が極小にならない(もちろん、最小にもならない):
			$(x_n,y_n):=\left( -\frac{1}{n},0 \right)$ が原点に近づくが、$f(x_n,y_n)<0=f(0,0)$である。
		\item 原点が最小になる(もちろん、極小にもなる):
			yは0と等しくなかったら、$f(x,y)\ge y^4>0=f(0,0)$になる。更に、$y$が0と等しくて、xは0でなかったら、
			$f(x,y)=x^2>0$になっる。
	\end{enumerate}
\end{question}
\begin{question}
	\begin{enumerate}[(1)]
		\item 
			まず、勾配ベクトルを求めよ
			\begin{equation*}
				\begin{array}[]{c}
					f(x,y)=xy^2\\
					f_x(x,y)=y^2\\f_x(2,1)=1\\
					f_y(x,y)=2xy\\f_y(2,1)=4.
				\end{array}
			\end{equation*}
				なので、接平面が以下のようになる:\begin{equation*}
					\begin{array}[]{c}
						z-2=1\cdot(x-2)+4\cdot(y-1)\\
						x+4y-z=4.
					\end{array}
				\end{equation*}
		\item 
			まず、勾配ベクトルを求めよ
			\begin{equation*}
				\begin{array}[]{c}
					f(x,y)=2x^2-y^2\\
					f_x(x,y)=4x\\f_x(2,1)=8\\
					f_y(x,y)=-2y\\f_y(2,1)=-2.
				\end{array}
			\end{equation*}
				なので、接平面が以下のようになる:\begin{equation*}
					\begin{array}[]{c}
						z-7=8\cdot(x-2)-2\cdot(y-1)\\
						8x-2y-z=7.
					\end{array}
				\end{equation*}
	\end{enumerate}
\end{question}
\begin{question}
	\begin{enumerate}[(1)]
		\item まず、偏微分を求めよ:
			\begin{equation*}
				\begin{array}[]{c}
					f(x,y)=(x^2-y^2)^3\\
					f_x(x,y)=6x(x^2-y^2)^2\\
					f_y(x,y)=6y(x^2-y^2)^2.\\
				\end{array}
			\end{equation*}
			次、停留点を求めよ:\begin{equation*}
				\begin{array}[]{c}
					\left\{\begin{array}[]{l}
						6x(x^2-y^2)^2=0\\
						6y(x^2-y^2)^2=0\\
					\end{array}\right.\\[2em]
					\left\{\begin{array}[]{l}
						\left[\begin{array}[]{l}
							x=0\\
							x^2-y^2=0
						\end{array}\right.\\
						\left[\begin{array}[]{l}
							y=0\\
							x^2-y^2=0
						\end{array}\right.\\
					\end{array}\right.\\[2em]
					\left[\begin{array}[]{c}
						\left\{\begin{array}[]{l}
							x=0\\y=0
						\end{array}\right.\\
						x^2-y^2=0
					\end{array}\right.\\[2em]
					\left[
						\begin{array}[]{l}
x=y=0\\(x-y)(x+y)=0
						\end{array}
						\right.\\[2em]
						\left[\begin{array}[]{l}
							x=y=0\\x=y\\x=-y
						\end{array}\right.
				\end{array}
			\end{equation*}
			なので、停留点が:\begin{equation*}
				(0,0),\quad\left\{ (t,t)\mid t\in\R \right\},\quad \left\{ (t,-t)\mid t\in\R \right\}.
			\end{equation*}
		\item まず、偏微分を求めよ:
			\begin{equation*}
				\begin{array}[]{c}
					f(x,y)=(x^2+y^2)\exp\left(-x^2-y^2  \right)\\
					f_x(x,y)=2x(1-x^2-y^2)\exp\left( -x^2-y^2 \right)\\
					f_x(x,y)=2y(1-x^2-y^2)\exp\left( -x^2-y^2 \right).\\
				\end{array}
			\end{equation*}
			次、停留点を求めよ:\begin{equation*}
				\begin{array}[]{c}
					\left\{\begin{array}[]{l}
					2x(1-x^2-y^2)\exp\left( -x^2-y^2 \right)=0\\
					2y(1-x^2-y^2)\exp\left( -x^2-y^2 \right)=0\\
					\end{array}\right.\\[2em]
					\left\{\begin{array}[]{l}
						\left[\begin{array}[]{l}
							x=0\\
							1-x^2-y^2=0
						\end{array}\right.\\
						\left[\begin{array}[]{l}
							y=0\\
							1-x^2-y^2=0
						\end{array}\right.\\
					\end{array}\right.\\[2em]
					\left[\begin{array}[]{c}
						\left\{\begin{array}[]{l}
							x=0\\y=0
						\end{array}\right.\\
						x^2+y^2=1.
					\end{array}\right.
				\end{array}
			\end{equation*}
			なので、停留点が:\begin{equation*}
				(0,0),\quad\left\{ (\cos(t),\sin(t))\mid 0\le t<2\pi \right\}.
			\end{equation*}
	\end{enumerate}
\end{question}
\begin{question}
	$(x_n,y_n):=\left( 1/n,1/n \right)$とする。そうしたら、\begin{equation*}
		\begin{array}[]{c}
			(x_n,y_n)\to(0,0)\\
			f(x_n,y_n)
			=1/2.
		\end{array}
	\end{equation*}
		なので、$(x_n,y_n)$が原点に収束するが、$f(x_n,y_n)$が$f(0,0)=0$に収束しない。
\end{question}
\begin{question}
	まず、$f_x.f_y,f_{xy},f_{yx}$を求める:
	\begin{equation*}
		\begin{array}[]{c}
			(x,y)\neq0\implies f_x(x,y)=\frac{\partial}{\partial x}\left( xy\frac{x^2-y^2}{x^2+y^2} \right)=\frac{x^4y-y^5-4x^2y^3}{\left( x^2+y^2 \right)^2}\\
			f_x(0,0)=\lim_{h\to0}\frac{f(h,0)}{h}=\lim_{h\to0}\frac{1}{1}
		\end{array}
	\end{equation*}
\end{question}
\end{document}




