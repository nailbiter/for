\documentclass[12pt]{article} % use larger type; default would be 10pt

\usepackage{enumerate}
\usepackage{xeCJK}
\usepackage{mystyle}
\usepackage{ruby}
\usepackage{longtable}
\usepackage{hyperref}
\usepackage{amsthm,amssymb}

\theoremstyle{definition}
\newtheorem{question}{問}

\setCJKmainfont[AutoFakeBold=true]{Hiragino Mincho Pro} %my Mac
%\setCJKmainfont{MS PGothic} %AJP windows
%\renewcommand\rubysep{-5ex}
\newcommand{\kana}[2]{\ruby{#1}{#2}}
\newcommand{\mytabra}[1]{$\myabra{\mbox{#1}}$}

\title{2017年度S1数理科学基礎演習・微積分(理二三21−24)第3回}
\begin{document}
\maketitle
\begin{question}
	\begin{enumerate}[(1)]
		\item \begin{equation*}
				\begin{array}[]{c}
					f_x=y^2+2xy^3\\
					f_y=2xy+3x^2y^2
				\end{array}
			\end{equation*}
		\item \begin{equation*}
				\begin{array}[]{c}
					f_x=y\cos(xy)\\
					f_y=x\cos(xy).
				\end{array}
			\end{equation*}
	\end{enumerate}
\end{question}
\begin{question}
	\begin{enumerate}[(1)]
		\item \begin{equation*}
				\begin{array}[]{c}
					f_{xx}=2ax\\
					f_{xy}=f_{yx}=b\\
					f_{yy}=2c
				\end{array}
			\end{equation*}
		\item\begin{equation*}
				\begin{array}[]{c}
					f_{xx}=\left( 2a+4a^2x^2 \right)\exp\left( ax^2+by^2 \right)\\
					f_{xy}=f_{yx}= 4abxy\exp\left( ax^2+by^2 \right)\\
					f_{yy}=\left( 2b+4b^2y^2 \right)\exp\left(  ax^2+by^2\right).
				\end{array}
			\end{equation*}
	\end{enumerate}
\end{question}
\begin{question}
	\begin{enumerate}[(1)]
		\item 原点が最大になる(もちろん、極大にもなる)
	\end{enumerate}<++>
\end{question}<++>
\end{document}




