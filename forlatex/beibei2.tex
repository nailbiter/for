\documentclass[12pt]{article}

\usepackage{amsmath}
\usepackage{amsfonts}

\newenvironment{proof}[1][Proof]{\begin{trivlist}
\item[\hskip \labelsep {\bfseries #1}]}{\end{trivlist}}

\newcommand{\qed}{\nobreak \ifvmode \relax \else
      \ifdim\lastskip<1.5em \hskip-\lastskip
      \hskip1.5em plus0em minus0.5em \fi \nobreak
      \vrule height0.75em width0.5em depth0.25em\fi}

\oddsidemargin 0 truemm \evensidemargin 0 truemm \marginparsep 0pt
\topmargin -50pt \textheight 240 truemm \textwidth 160 truemm
\parindent 0em \parskip 1ex
\newcommand{\ds}{\displaystyle}
\newcommand{\rto}{\rightarrow}
\newcommand{\defeq}{\stackrel{\tiny{\mathbf{def}}}{=\!\!=}}
%custom commands to save typing
\newcommand{\mynorm}[1]{\left|\left|#1\right|\right|}
\newcommand{\myabs}[1]{\left|#1\right|}
\newcommand{\myset}[1]{\left\{#1\right\}}
\newcommand{\mysca}[2]{\left<#1,#2\right>}
\newcommand{\mysetn}[2]{\left\{#1\big| #2\right\}}
\newcommand{\myfincol}[2]{\left\{#1\right\}_{i=1}^{#2}}
\newcommand{\myfrac}[2]{{^{#1}/_{#2}}}
\newcommand{\mycol}[2]{\left\{#1\right\}_{#2}}
\newcommand{\mysetg}[1]{\left<\left\{#1\right\}\right>}
%\newcommand{\myvec}[1]{\overrightarrow{#1}}
\newcommand{\myvec}[1]{\mbox{\overrightharp{$#1$}}}
\newcommand{\myvecprodexpanded}[6]{\begin{vmatrix}\mathbf{i}&\mathbf{j}&\mathbf{k}\\#1&#2&#3\\#4&#5&#6\end{vmatrix}=
	\begin{vmatrix}#2&#3\\#5&#6\end{vmatrix}\mathbf{i}
	-\begin{vmatrix}#1&#3\\#4&#6\end{vmatrix}\mathbf{j} 
	+\begin{vmatrix}#1&#2\\#4&#5\end{vmatrix}\mathbf{k}}
\newcommand{\mybra}[1]{\left(#1\right)}
\newcommand{\mycbra}[1]{\left\{#1\right\}}
\newcommand{\mysbra}[1]{\left[#1\right]}
\newcommand{\myabra}[1]{\left<#1\right>}
\newcommand{\mypic}[2]{\begin{figure}[H]\centering\includegraphics[width=#1\textwidth]{#2}\end{figure}}
\newcommand{\myexplain}[1]{\mbox{ (\textit{#1})}}
\newcommand{\myr}[1]{\mathbb{R}^{#1}}
\newcommand{\mys}[1]{\mathbb{S}^{#1}}

%put subscript under lim and others
\let\oldlim\lim
\renewcommand{\lim}{\displaystyle\oldlim}
\let\oldmin\min
\renewcommand{\min}{\displaystyle\oldmin}
\let\oldmax\max
\renewcommand{\max}{\displaystyle\oldmax}
\let\oldsum\sum
\renewcommand{\sum}{\displaystyle\oldsum}
\let\oldsup\sup
\renewcommand{\sup}{\displaystyle\oldsup}

%normal Re and Im
\renewcommand\Re{\operatorname{Re}}
\renewcommand\Im{\operatorname{Im}}

\begin{document}           % End of preamble and beginning of text.
\pagestyle{myheadings} \thispagestyle{empty} \markright{}

\begin{center}
{\bf THE CHINESE UNIVERSITY OF HONG KONG}\\
{\bf Department of Mathematics}\\
{\bf Course Code: MATH1010F}\\
{\bf Liu Beibei}\\
\end{center}

Let $T$ be a linear transformation of a vector space $V$ into itself. Suppose $x\in V$ is such that $T^mx=0$, $T^{m-1}x\neq0$ for
some positive integer $m$. Show that $x,\;Tx,\hdots,T^{m-1}x$ are linearly independent.

\textbf{Solution:}

Indeed, let us argue by contradiction. Assume aforementioned vectors are in fact linearly dependent, so for some
real numbers $\mycbra{a_i}_{i=0}^{m-1}$, not simultaneously equal to zero, we have
\[a_0x+a_1T^1x+\hdots+a_{m-1}T^{m-1}x=0\]
As $a_i$ are not simultaneously equal to zero by assumption, let $1\leq j\leq m-1$ be the index, so that $a_j\neq 0$ and
$a_i=0$ for all $1\leq i<j$. Then, we have
\[0=T^{m-j-1}0=T^{m-j-1}(a_0x+a_1T^1x+\hdots+a_{m-1}T^{m-1}x)=T^{m-j-1}(a_jT^jx+\hdots+a_{m-1}T^{m-1}x)=a_jT^{m-1}x\]
as for $k>m$ we have $T^kx=T^{k-m}(T^mx)=T^{k-m}0=0$ and since by assumption $T^{m-1}x\neq0$, we get $a_j=0$, contradicting
our choice of $j$. Obtained contradiction finishes the proof.
\end{document}
