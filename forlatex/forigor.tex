
\documentclass[10pt]{article} % use larger type; default would be 10pt

%%\usepackage[T1,T2A]{fontenc}
%%\usepackage[utf8]{inputenc}
%%\usepackage[english,ukrainian]{babel} % може бути декілька мов; остання з переліку діє по замовчуванню. 
\usepackage{enumerate}
\usepackage{mystyle}
\usepackage{CJKutf8}

\title{Advanced Calculus Homework 2}
\author{}
\begin{document}
\maketitle
\begin{enumerate}[(1)]
	\item So, let $[a,b]$ be the compact interval of positive length (which implies $a<b$). Take $\epsilon=1/4>0$ and assume
		(in order to get a contradiction) that family $f_n$ is equicontinuous, hence that $\exists\delta>0$ such that 
		\begin{equation}\forall n\forall y:\myabs{y-(a+b)/2}<\delta\implies
			\myabs{\sin(ny)-\sin(n(a+b)/2)}<\epsilon\label{Prob1}\end{equation}
		
		Then, take $n$ big enough, so that $\pi/n<\delta$ and $\pi/n<(b-a)/2$. Hence, inside $(a,b)$ and $((a+b)/2-\delta,
		(a+b)/2+\delta)$ (whatever smaller) there should be a pair of points $\pi (m-^1/_2)/n$ and $\pi(m+^1/_2)/n$. Hence, 
		\ref{Prob1} implies $\myabs{f_n(\pi(m+1)/n)-f_n(\pi m/n)}<2\epsilon=1/2$, hence
		$1/2>\myabs{\sin(\pi(m+^1/_2))-\sin((m-^1/_2)\pi)}=2$. Contradiction. Thus, family $f_n$ is not equicontinuous.
	\setcounter{enumi}{3}
\item\begin{enumerate}[(a)]
		\item Yes, it is equicontinuous. For indeed, we have every $f_n$ being differentiable (only right-differentiable at 
			$x=0$, as it's not defined for $x<0$ by the problem statement) and the derivative is
			\[f_n'(x)=\cos\sqrt{x+4n^2\pi^2}\frac{1}{2\sqrt{x+4n^2\pi^2}}\]
			hence (assuming count starts from $n=1$) we have $\myabs{f_n'(x)}\leq1/4\pi$ (if count starts from $n=0$
			our conclusion will still remain true, as taking in or out finitely many continuous
			terms of family does not change its equicontinuity). Henceforth, given $\epsilon>0$ if we let
		$\delta=4\pi\epsilon$ we would have for every $x,y\in[0,\infty)$ the following
			$\myabs{x-y}<\delta=4\pi\epsilon\implies\myabs{f_n(x)-f_n(y)}=\myabs{f_n'(c)}\myabs{x-y}<4\pi\epsilon(1/4\pi)
			=\epsilon$, which is even stronger than equicontinuity.
		\item In fact, \[f_n(x)=\sin\sqrt{x+4\pi^2n^2}=\sin\mybra{\sqrt{x+4\pi^2n^2}-2\pi n}=
			\sin\frac{x}{\sqrt{x+4\pi^2n^2}+2\pi n}\to0\]
			so $f_n$ do converge pointwise to $f\equiv0$.
		\item Convergence is not uniform, as if it would be so, for say $\epsilon=1/2$ there would be $N$ such that
		\begin{equation}
			\forall x>0,\;\forall n>N,\quad \mynorm{\sin\sqrt{x+4n^2\pi^2}}<\epsilon=1/2\label{Prob4}\end{equation}
			However, for $n=N+1$ we can take $m$ positive integer big enough, so that
			\[\pi^2/4+2\pi^2m+4\pi^2m^2-4n^2\pi^2>0\]
			then we may put $x:=\pi^2/4+2\pi^2m+4\pi^2m^2-4n^2\pi^2>0$
			then we will have
			\[\sin\sqrt{x+4\pi^2n^2}=\sin(\pi/2+2\pi m)=1\]
			in contradiction with \ref{Prob4}. The contradiction obtained proves the initial claim that convergence
			was not uniform.
	\end{enumerate}
\item Note, that by Arzel\`a-Ascoli for every compact subinterval of $(a,b)$ there is a subsequence of $f_n$, that converges
	uniformly on it. Indeed, as on every compact subinterval $I$ functions $f_n$ are bounded by continuous $F$ and $F$ in turn is
	bounded by its maximum on $I$, we see uniform boundedness. Equicontinuity follows, as by mean-value theorem
	$\myabs{f_n(x)-f_n(y)}=\myabs{x-y}\myabs{f_n'(c)}\leq\myabs{x-y}\myabs{G(c)}$ and $G$ is uniformly bounded on compact $I$, being continuous.\\

	Now, let $(r_k,p_k)$ be the enumeration of pairs of rational points in $(a,b)$, such that $r_k<p_k$. Based on previous
	result, for $[r_1,p_1]$ we can find subsequence ${f^{(1)}_n}_n$ of $f_n$, such that it uniformly converges on $[r_1,p_1]$. Then,
	we can take subsequence $\mycbra{
	f^{(2)}_n}_n$ of $\mycbra{f^{(1)}_n}_n$ of $f_n$, such that it uniformly converges on $[r_2,p_2]$.
	Proceeding in this way, may find sequence $\mycbra{f^{(k)}_n}_n$ of subsequences of $f_n$ such that $\mycbra{f^{(k+1)}_n}_n$ 
	is a subsequence of $\mycbra{f^{(k)}_n}_n$ and $\mycbra{f^{(k)}_n}_n$ uniformly converges on 
	$[r_j,p_j]$ for all $1\leq j\leq k$. Take subsequence $\mycbra{f^{(n)}_n}_n$. For any given subsequence 
	$\mycbra{f^{(k)}_n}_n$, $\mycbra{f^{(n)}_n}_n$ becomes a subsequence of it for big $n$. Hence
	$\mycbra{f^{(n)}_n}_n$ uniformly converges on every fixed $[r_k,p_k]$ and we claim that it is the subsequence we've been looking for.\\

	Indeed, for $[r,p]$ be compact subinterval of $a_b$ there are rational $r', p'$ such that $[r',p']\supset[r,p]$ and
	since $r',p'$ are rational, they should be in our enumeration above, hence 
	$\mycbra{f^{(n)}_n}_n$ uniformly converges on $[r',p']$, hence on $[r,p]$.
\end{enumerate}
%%\begin{thebibliography}{9}
%%\bibitem{gelbaum}Gelbaum, B.R. and Olmsted, J.M.H.. Counterexamples in Analysis. Dover Publications. 2003
%%\end{thebibliography}
\end{document}
