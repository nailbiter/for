\documentclass[10pt]{article}
\usepackage{fontspec}
\usepackage{array, xcolor, lipsum, bibentry}
\usepackage[margin=3cm]{geometry}
\usepackage{sectsty} % Allows changing the font options for sections in a document

%font configuration
\defaultfontfeatures{Mapping=tex-text}
\setromanfont[Ligatures={Common}, Numbers={OldStyle}, Variant=01]{Linux Libertine O} % Main text font
\sectionfont{\mdseries\upshape\Large} % Set font options for sections
\subsectionfont{\mdseries\scshape\normalsize} % Set font options for subsections
\subsubsectionfont{\mdseries\upshape\large} % Set font options for subsubsections
\chardef\&="E050 % Custom ampersand character

\title{Fluctuation and coarse-graining away from thermal equilibrium\\
Project Description\\(potential supervisor: Adrian Baule)
}
\author{Oleksii Leontiev\\Application No.:120729274}
\begin{document}
\maketitle
Seemingly complicated interactions in many-particle systems can often be described by much simpler
(approximate) ones by eliminating irrelevant degrees of freedom. This coarse-graining of the
microscopic dynamics is at the heart of equilibrium statistical mechanics and particularly important,
e.g., in the study of critical phenomena. Coarse-graining in equilibrium systems is simplified due to
our fundamental understanding of the role of thermal fluctuations, as exemplified by the principle of
detailed balance, which is satisfied by transitions between microscopic and coarse-grained states alike.
Much less is known for systems away from thermal equilibrium. This project aims to investigate the properties
of nonequilibrium fluctuations under coarse-graining using analytical methods and computer simulations.
More specific aspects concern the validity of fluctuation theorems, the behaviour of large deviation functions,
effective dynamics from constrained path-ensembles, and statistical forces arising in the Mori-Zwanzig formalism.

%why I am interested
The description of this project, as suggested to me by Professor Baule attracted my attention because it involves both Applied Mathematics
and analytic work. I always prefer to see results of what I am doing in the form of concrete applications. On the other hand, I would
like to leverage my Computer Science (Double Degree) background and this project gives me perfect ability to do so, in the form of simulations.
Not only I am motivated to do this project, but I also possess background in analysis and probability. I have already to start to work on 
relevant literature, as Professor Baule has suggested. The book I am working on right now is {\em "The Fokker-Planck Equation"} by H. Risken.
As during my bachelor studies (and now) I was interested in Stochastic Process and Probability, I believe that I have necessary background
in order to tackle this project.
%why am I good
%how will I roll it
%what I already did
\end{document}
%About GSoC 2013 and whether PCL will participate in it.
%
%Dear Mr/Mrs!
%
%Let me introduce myself. My name is Alex, I am studying in Taiwan. I am a senior now, double major in Applied Math and CS. I am working in the lab of one CS Department professor and I'm assisting in doing some research, involving reconstruction and registration. And we often use PCL.
%
%While doing this I became very interested in internals of PCL as I had to trace the source code several times. Currently I'm using PCL 1.7.
%
%I would like to participate in this incoming Google Summer of Code 2013 and do something with you, guys, that is adding some functionality to PCL. In connection with this I'd have a several (two) questions:
%1. Will PCL participate in GSoC 2013?
%2. If I would like to join to GSoC with PCL, should I come up with project topic completely by myself or I should ask you guys for assistance?
%
%Best Regards,
%Alex
