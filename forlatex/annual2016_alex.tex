%platex
%%数理科学研究科 M2以上(研究生・学振含む)の院生                  
%%・学振(PD)・特任研究員の皆様
%%
%%前略
%%
%%2016年度年間研究成果報告書の作成に取りかかります。 Annual Report用
%%スタイルファイルをお送りいたします。 
%%今年度もLaTeXでA4サイズ・ニ段組で作成致しますので
%%よろしくお願い致します。
%%必ず今年のスタイルファイルで作成して下さい。また、ハードコピーを
%%提出して下さい。
%%
%%実績報告書等へのデータ使用の為、2016年1月1日〜12月31日までの掲載論文は
%%すべて記入して下さい。
%%
%%A.研究概要は、英文も含めて1.5ページを上限としてお書き下さい。
%%
%%メールで原稿をお送りいただける場合は、(主任室I 福井)
%%                   annual@ms.u-tokyo.ac.jp
%%
%%にお送り下さい。
%%
%%締め切りは、
%%
%%      平成29年2月28日(火)(期限厳守!)
%%
%%とします。期限厳守でお願い致します。
%%
%%また海外出張などで留守にされる方は出発前にご提出下さい。
%%
%%ご協力よろしくお願い致します。
%%
%%                    平成29年1月31日
%%                    専攻長 二木 昭人
%%                   (編集担当 福井 伸江)




%%--------------- Text starts from here ----------- %%

%%%%%%%%%%%%%%%2016年度Annual Report用スタイルファイル%%%%%%%%%%%%%%%%%%%%%%%%
%このFormatはpLaTex を使用しています。
%以下に報告書の基本形が示してありますので、参考にしてお書き下さい。 
%数字は2桁以上は全て半角で書いて下さい。 
%文末の空白は必ず半角でお願いします。全角の空白は TeX では特殊文字と 
%判断して問題を起こすことがあります。
%数式はかならずmath mode でお願いいたします。 
%事務局では校正をせずにprint out したものをそのまま印刷に回しますので、 
%一度コンパイルして、スペルチェック、校正は必ず行なって下さい。 
%まとめの編集の都合上、\newcommand, \renewcommand, \def の追加等はさけて下さいま%すようお願いいたします。
%%%%%%%%%%%%%%%%%%%%%%%%%%%%%%%%%%%%%%%%%%%%%%%%%%%%%%%%%%%%%%%%%%%%%%%%%%%%%%%%


\documentclass[a4j,twocolumn]{jarticle}

\usepackage{amssymb,amsmath}
\textheight=25cm
\textwidth=15cm
\parskip=0mm
\parindent=0mm
\topmargin=-1cm
\oddsidemargin=5mm

\begin{document}

%%%%%%%%%%%%%%%%%%%%%%%%%%%%%%%%%%%%%%%%%%%%%%%%%%%%%%%%%%%%%%%%%%%%%%%%%%% 
% 新しい連絡先
% 2017年度から所属や連絡先が変わる予定の方は連絡先(住所・メールアドレス)
%をご記入下さい。
%%%%%%%%%%%%%%%%%%%%%%%%%%%%%%%%%%%%%%%%%%%%%%%%%%%%%%%%%%%%%%%%%%%%%%%%%%% 


%%%%%%%%%%%%%%%%%%%%%%%%%%%%%%%%%%%%%%%%%%%%%%%%%%%%%%%%%%%%%%%%%%%%%%%%%%% 
% 2016年度 東大数理における該当する身分を選択し、それ以外の項目は
% 削除して下さい。
%%%%%%%%%%%%%%%%%%%%%%%%%%%%%%%%%%%%%%%%%%%%%%%%%%%%%%%%%%%%%%%%%%%%%%%%%%% 
%%教授 (Professor) 
%%准教授 (Associate Professor)
%%助教 (Reseach Associates)
%%特任教授 (Project Professor)
%%特任准教授 (Project Associate Professor)
%%特任助教 (Project Research Associate) 
%%教育支援員 (Teaching Support Staffs)
%%外国人客員教授・准教授 (Foreign Visiting (Associate) Professor)
%%連携併任講座 (Special Visiting Chairs) 
%%客員教授 (Visiting Professors) 
%%学振特別研究員 (JSPS Fellow)
%%特任研究員 (Project Researcher) 
%%協力研究員 (Associate Fellows)
博士課程学生 (Doctoral Course Student)
%%修士課程学生 (Master's Course Student)
%%研究生 (Research Student)
%%%%%%%%%%%%%%%%%%%%%%%%%%%%%%%%%%%%%%%%%%%%%%%%%%%%%%%%%%%%%%%%%%%%%%%%%%% 


%%%%%%%%%%%%%%%%%%%%%%%%%%%%%%%%%%%%%%%%%%%%%%%%%%%%%%%%%%%%%%%%%%%%%%%%%%%% 
% 氏名(ローマ字綴りで名字は全て大文字,名前は最初の字だけ大文字) 
% を書いて下さい.
%{\bf 数理 太郎 (SURI Taro)}
%学生で学振DC1・学振DC2に該当する者は記入をして下さい。
%学生でFMSPコース生に該当する者は記入して下さい。
%\hspace{5cm}(学振DC1または学振DC2)
%\hspace{5cm}(FMSPコース生)
%というように
%%%%%%%%%%%%%%%%%%%%%%%%%%%%%%%%%%%%%%%%%%%%%%%%%%%%%%%%%%%%%%%%%%%%%%%%%%% 

{\bf レオンチエフ・アレックス (LEONTIEV Oleksii)}
\hspace{3.85cm}(学振DC2)\\
\hspace{4.5cm}(FMSPコース生)

%%%%%%%%%%%%%%%%%%%%%%%%%%%%%%%%%%%%%%%%%%%%%%%%%%%%%%%%%%%%%%%%%%%%%%%%
% 索引用データ
% 一人あたりAlphabet順索引・五十音順索引用に
% 2行必要です。
%
% 日本人 
% \index{アルファベット表記 (日本語表記)}
% \index{かな読み@日本語表記}
% 例
% \index{TSUBOI Takashi (坪井 俊)}
% \index{つぼいたかし@坪井 俊}
%
% 外国人(漢字表記なし)
% \index{アルファベット表記}
% \index{かな読み@カタカナ表記}
% 例
% \index{SUTHICHITRANONT Noppakhun}
% \index{すってぃちとらのん@スッティチトラノン ノッパクン}
\index{Leontiev Oleksii}
\index{れおんちえふ あれっくす@レオンチエフ アレックス}
%
% 外国人(漢字表記あり)
% \index{アルファベット表記 (漢字表記)}
% \index{かな読み@カタカナ表記}
% 例
% \index{LI Xiaolong (李 曉龍)}
% \index{りしゃおろん@リ シャオロン}
%
%%%%%%%%%%%%%%%%%%%%%%%%%%%%%%%%%%%%%%%%%%%%%%%%%%%%%%%%%%%%%%%%%%%%%%%%



\vspace{0.2cm}
\noindent
A. 研究概要

\vspace{0.1cm}
%%%%%%%%%%%%%%%%%%%%%%%%%%%%%%%%%%%%%%%%%%%%%%%%%%%%%%%%%%%%%%%%%%%%%%%%%% 
% 研究の要約を日本語で,その下に英訳をつけて書いて下さい.
%%%%%%%%%%%%%%%%%%%%%%%%%%%%%%%%%%%%%%%%%%%%%%%%%%%%%%%%%%%%%%%%%%%%%%%%%% 

%和文%
$G$ を Lie 群、$G'$ を $G$ の閉部分群とする。さらに、$(\pi,V)$と$(\tau,W)$を$G$と$G'$の表現とする。
その時、$V$から$W$へ$G'$-線形作用素は対称性破れ作用素と呼ばれる。特に、$\pi$が無
限次元で、$G'$ が非コンパクトの時、対称性破れ作用素の空間$\mbox{Hom}_{G'}(\pi\big|_{G'},\tau)$を
具体的に求めるという問題はかなり難しい。しかし最近、$O(n + 1, 1) \supset O(n, 1)$ という特別な場合
に、すべての対称性の破れ作用素が2014、2015年に小林俊行氏と B. Speh 氏によって
完全に分類された。これはその問題の完全な答えとして、一番最初である。\\
私の目的は小林俊行氏と B. Speh 氏によって発展された一般な手法によって、$(G, G') = (O(p+1,q),O(p,q))$
の場合の対称性の破れ作用素を研究するということであった。
具体的には、以下の問題を考えた:\\
{\noindent}\textbf{問\textbf{1}.}{与えられた $( \lambda, \nu) \in
\mathbb{C}^2$
に対して、対称性の破れ作用素の空間 $\operatorname{Hom}_{G'}(I(\lambda),J(\nu))$ を具体的に求めよ。
特に、この空間の基底を具体的に求めよ。ここで、$I(\lambda):=C^{\infty}\left(  G\times_P\mathbb{C}_\lambda\right)$
と$J(\nu):=C^{\infty}\left( G'\times_{P'}\mathbb{C}_{\nu} \right)$は$G$と
$G'$の退化主系列である。}\\
私は本年度の研究において、この問題に完全な答えを与えた。さらに、
具体的に作られた基底の元の特徴も調べた。



\vspace{0.5cm}
%英文%
Let $G$ be a Lie group and $G'$ be its closed subgroup. Moreover, let $(\pi,V)$ and $(\tau,W)$ be representations of $G$ and $G'$
respectively. Then, the $G'$-intertwining operator from $V$ to $W$ is called symmetry breaking operator. In particular, when 
$\pi$ is infinitely-dimensional and $G'$ is non-compact, the problem of explicit description of space
$\mbox{Hom}_{G'}(\pi\big|_{G'},\tau)$ of symmetry breaking operators becomes highly nontrivial. However, in their recent work 
T. Kobayashi and B. Speh (2014,2015) were able to obtain the complete classification of symmetry breaking operators between
the principal series in the setting $(G, G') = (O(n+1,1),O(n,1))$. To my knowledge, this is the first example of complete
description of symmetry breaking operators.\\
My goal was to classify symmetry breaking operators between the degenerate principal series representations for the setting
$(G, G') = (O(p+1,q),O(p,q))$. More precisely, the following question was posed:\\
{\noindent}\textbf{Question \textbf{1}.}{\itshape{For every pair $( \lambda, \nu) \in
\mathbb{C}^2$, explicitly describe the space
$\operatorname{Hom}_{G'}(I(\lambda),J(\nu))$ of symmetry breaking operators. In particular, find the explicit basis.
Here $I(\lambda):=C^{\infty}\left(  G\times_P\mathbb{C}_\lambda\right)$
and $J(\nu):=C^{\infty}\left( G'\times_{P'}\mathbb{C}_{\nu} \right)$ are the degenerate principal series representations of $G$ and
$G'$respectively.}}\\
During this year I was able to give a complete answer to the latter question. Besides, I've explicitly constructed basis
elements and investigated their properties.




%\\%
\vspace{0.2cm}


\noindent
B. 発表論文

\vspace{0.1cm}
%%%%%%%%%%%%%%%%%%%%%%%%%%%%%%%%%%%%%%%%%%%%%%%%%%%%%%%%%%%%%%%%%%%%%%%%%%%%%% 
% 5年以内(2012〜2016年度)10篇まで書いて下さい。但し、2016年1月1日〜 
% 2016年12月31日に出版されたものは、10篇を超えてもすべて含めて下さい。
% 様式は以下の例のように
% 著者・共著者名・ \lq\lq 題名・ジャーナル名・巻・年・ページの順に書いて下さい.
% タイトルの前に著者・共著者名を入れる形です。
% 共著の場合 T. Katsura and #.####などと書きwith 共著者名とはしない様に
% お願い致します。
%%%%%%%%%%%%%%%%%%%%%%%%%%%%%%%%%%%%%%%%%%%%%%%%%%%%%%%%%%%%%%%%%%%%%%%%%%%%% 

%\begin{enumerate}
%\item G. van der Geer and T. Katsura:\lq\lq On a stratification of 
%the moduli of K3 surfaces",
%J.\ Eur.\ Math.\ Soc. {\bf 2} (2000) 259--290.
%\end{enumerate}

\vspace{0.2cm}
\noindent
C. 口頭発表

\vspace{0.1cm}
%%%%%%%%%%%%%%%%%%%%%%%%%%%%%%%%%%%%%%%%%%%%%%%%%%%%%%%%%%%%%%%%%%%%%%%%%%%%%%%
% シンポジュームや学外セミナーでの発表で5年以内(2012〜2016年度)10項目まで。 
% タイトル・シンポジューム(またはセミナー等)名・場所・月・年を 
% 書いて下さい.国際会議の場合は国名をお願いします.タイトルは原題で。 
%%%%%%%%%%%%%%%%%%%%%%%%%%%%%%%%%%%%%%%%%%%%%%%%%%%%%%%%%%%%%%%%%%%%%%%%%%%%%%% 
\begin{enumerate}
	\item[(1)] 
2017年3月26日, 日本数学会 2017年度年会, 共形変換群 $O( p;q )$ に関する対称性破れ作用素 (日本語), 都大学東京
	\item[(2)] 
2016年11月30日, Symposium on Representation Theory 2016, 不定値直交群 $O ( p; q )$ の対称性破れ作用素 (日本語), Grand Mer Resort, 沖縄
	\item[(3)] 
2016年11月19日, 日本数学会 異分野・異業種研究交流会 2016, Symmetry breaking operators of indefinite orthogonal groups $O(p,q)$ (poster, eng), 明治大学
	\item[(4)] 
2016年10月7日, 広島幾何学研究集会2016, Symmetry breaking operators of indefinite orthogonal groups $O(p,q)$ (日本語), 広島大学
	\item[(5)] 
2016年9月18日, 日本数学会2016年度秋季総合分科会, , 関西大学
	\item[(6)] 
2016年8月11日, Workshop on “Actions of Reductive Groups and Global Analysis”, ”Discrete decomposability of the restriction of $A_q(\lambda)$ with respect to reductive subgroups and its applications (T. Kobayashi, Invent Math) の紹介(日本語), 東京大学 玉原国際セミナーハウス
	\item[(7)] 
2016年7月19日, 広島大学幾何セミナー, 不定値直交群 O(p,q) の対称性破れ作用素 (日本語), 広島大学
%%(1) 曲面の写像類群とは, (2) 写像類群をめぐるこれまでの結果と夢,
%%Encounter with Mathematics 第11回, 中央大学理工学部,
%%1999年4,5月.
\end{enumerate}

\vspace{0.2cm}
\noindent
D. 講義 (学生さんは記入されなくてもよい。)

\vspace{0.1cm}
%%%%%%%%%%%%%%%%%%%%%%%%%%%%%%%%%%%%%%%%%%%%%%%%%%%%%%%%%%%%%%%%%%%%%%%%%%%%%% 
% 講義名, 講義の種類,簡単な内容説明を1,2行でお願いします。 
% 講義の種類は, 数理大学院・4年生共通講義, 理学部2年生(後期)・3年生向け講義, 
% 教養学部前期課程講義, 教養学部基礎科学科講義, 
% 集中講義のいずれかでお願いします。
% 集中講義の場合は,場所と時期もお願いします。
%%%%%%%%%%%%%%%%%%%%%%%%%%%%%%%%%%%%%%%%%%%%%%%%%%%%%%%%%%%%%%%%%%%%%%%%%%%%%%

%\begin{enumerate}
%<例>\item 代数幾何学・代数学XG : 代数幾何の入門講義,代数多様体の定義 
% などのほか, 代数多様体の変形理論を扱った.(数理大学院・4年生共通講義) 

%\end{enumerate}

\vspace{0.2cm}




\noindent
E. 修士・博士論文 (学生さんは記入されなくてもよい。)

\vspace{0.1cm}
%%%%%%%%%%%%%%%%%%%%%%%%%%%%%%%%%%%%%%%%%%%%%%%%%%%%%%%%%%%%%%%%%%%%%%%%%%%%%% 
% 今年度に学位を取得した人の論文の指導教官または主査の方はお願いします。 
% 学位を取得した人の名前を日本語・ローマ字名で,論文名は原題でお願いします。 
% 論文博士か課程博士か修士かは氏名の前に書いて下さい。 
%%%%%%%%%%%%%%%%%%%%%%%%%%%%%%%%%%%%%%%%%%%%%%%%%%%%%%%%%%%%%%%%%%%%%%%%%%%%%%%

%\begin{enumerate}
%<例>\item (論文博士)中山 昇(NAKAYAMA Noboru): On smooth exceptional 
% curves in threefolds.
%<例>\item (修士)高田 聖治(TAKADA Seiji): 正標数の代数曲面の cotangent 
% bundle のstability と Bogomolov の不等式. 

%\end{enumerate}

\vspace{0.2cm}
\noindent
F. 対外研究サービス

\vspace{0.1cm}
%%%%%%%%%%%%%%%%%%%%%%%%%%%%%%%%%%%%%%%%%%%%%%%%%%%%%%%%%%%%%%%%%%%%%%%%%%%% 
% 学会役員,雑誌のエディター,学外セミナーやシンポジュームのオーガナイザー等 
%%%%%%%%%%%%%%%%%%%%%%%%%%%%%%%%%%%%%%%%%%%%%%%%%%%%%%%%%%%%%%%%%%%%%%%%%%%% 
%\begin{enumerate}
%\item


%\end{enumerate}

\vspace{0.2cm}
\noindent
G. 受賞

\vspace{0.1cm}
%%%%%%%%%%%%%%%%%%%%%%%%%%%%%%%%%%%%%%%%%%%%%%%%%%%%%%%%%%%%%%%%%%%%%%%%%% 
% 過去5年の間にありましたら書いて下さい。 
%%%%%%%%%%%%%%%%%%%%%%%%%%%%%%%%%%%%%%%%%%%%%%%%%%%%%%%%%%%%%%%%%%%%%%%%
\begin{enumerate}
	\item 平成27年度学生表彰「数理科学研究科長賞」;
	\item 文部科学省奨学金 (2016--2017);
	\item 学振DC2 (2017--2019).
\end{enumerate}




\vspace{0.2cm}
\noindent

H. 海外からのビジター

%\vspace{0.1cm}
%%%%%%%%%%%%%%%%%%%%%%%%%%%%%%%%%%%%%%%%%%%%%%%%%%%%%%%%%%%%%%%%%%%%%%%%%%% 
%% JSPS等で海外からのビジターのホストになった方は、
%% 研究内容,講演のスケジュール、内容などの簡単な紹介を英語で書いて下さい。 
%% または、A〜同じように書いてもらってください。
%% 人数が多い場合は、主なものを5件までとします。
%%%%%%%%%%%%%%%%%%%%%%%%%%%%%%%%%%%%%%%%%%%%%%%%%%%%%%%%%%%%%%%%%%%%%%%%%% 

%%\vspace{0.2cm}
%%\noindent
%%連携併任講座
%%
%%\vspace{0.1cm}

%%%%%%%%%%%%%%%%%%%%%%%%%%%%%%%%%%%%%%%%%%%%%%%%%%%%%%%%%%%%%%%%%%%%%%%%%%%
%連携併任講座の教官の世話人になった方は, その教官についての成果
%(A)から(G)までをご報告ください.
%(別頁に掲載致します)または書いていただくよう依頼をお願い致します
%%%%%%%%%%%%%%%%%%%%%%%%%%%%%%%%%%%%%%%%%%%%%%%%%%%%%%%%%%%%%%%%%%%%%%%%%%%%% 

\end{document} 





%%%%%%%%%%%%%%%%%%%%%%%%%%%%%%%%%%%%%%%%%%%%%%%%%%%%%%%%%%%%%%%%%%%%%%%%%%%%%% 
% ☆その他
% 研究会などのプログラムについては、下記のTex形式で作成をし、別のファイルとして
%お送り下さい。
%%%%%%%%%%%%%%%%%%%%%%%%%%%%%%%%%%%%%%%%%%%%%%%%%%%%%%%%%%%%%%%%%%%%%%%%%%%%%%%

%\documentclass[a4paper,10pt]{article}
%\usepackage{amssymb,amsmath}
%\begin{document}

%\begin{Large}

%%%%%%%%%%%%%%%%%%%%%%%%%%%%%%%%%%%%%%%%%%%%%%%%%%%%%%%%%%
%\noindent
%{\bf The 4th East Asian Conference on Algebraic Topology}
%研究集会のタイトルをご記入下さい。
%%%%%%%%%%%%%%%%%%%%%%%%%%%%%%%%%%%%%%%%%%%%%%%%%%%%%%%%%%

%\noindent
%{ }

%\end{Large}

%\vspace{2mm}

%\begin{large}

%%%%%%%%%%%%%%%%%%%%%%%%%%%%%%%%%%%%%%%%%%%%%%%%%%%%%%%%%%%
%\noindent
%{December 5 -- December 9, 2011}
%
%\noindent
%{Lecture Hall, Graduate School of Mathematical Sciences, 
%The University of Tokyo}
%研究集会の開催日、開催場所をご記入下さい。
%%%%%%%%%%%%%%%%%%%%%%%%%%%%%%%%%%%%%%%%%%%%%%%%%%%%%%%%%%%

%\noindent
%{ }

%\noindent
%{ }

%\vspace{2mm}

%\noindent
%{\bf Program}
%\end{large}

%\vspace{2mm}

%%%%%%%%%%%%%%%%%%%%%%%%%%%%%%%%%%%%%%%%%%%%%%%%%%%%%%%%%
%\noindent
%{\bf Monday, December 5}
%
%\vspace{2mm}
%
%\noindent
%9:30 -- 10:15   {\bf Kazuo Habiro} (RIMS, Kyoto University)
%
%Quantum fundamental groups of 3-manifolds
%上の例のように以下にプログラムを記入して下さい。
%
%
%\end{document}
%
%%%%%%%%%%%%%%%%%%%%%%%%%%%%%%%%%%%%%%%%%%%%%%%%%%%%%%%%%%%%%%%%%%%%%%%%%%%%%%%%
%TODO: changeEnglishSummary; changeJapSummary
