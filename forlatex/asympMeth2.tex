\documentclass[12pt]{article} % use larger type; default would be 10pt

\usepackage{textcomp} %for copyleft symbol
\usepackage{mathtext}                 % підключення кирилиці у математичних формулах
                                          % (mathtext.sty входить в пакет t2).
\usepackage[T1,T2A]{fontenc}         % внутрішнє кодування шрифтів (може бути декілька);
                                          % вказане останнім діє по замовчуванню;
                                          % кириличне має співпадати з заданим в ukrhyph.tex.
\usepackage[utf8]{inputenc}       % кодування документа; замість cp866nav
                                          % може бути cp1251, koi8-u, macukr, iso88595, utf8.
\usepackage[english,ukrainian]{babel} % національна локалізація; може бути декілька
                                          % мов; остання з переліку діє по замовчуванню. 

\usepackage{sectsty}   %in order to make chapter headings and title centered
\chapterfont{\centering}

\usepackage{mystyle}

\newtheorem{prob}{Вправа}
%вправи 1-6 на стор.19-20
%вправи 1-4 на стор. 25-26
%вправа 1 на стор. 33

\title{
Якісні аналітичні методи диференціальних рівнянь\\
Контрольна робота (10 семестр)\\}
\author{Олексій Леонтьєв}
\begin{document}
\maketitle
\section{стор. 19-20}
\begin{prob}Довести, що будь-який розв’язок рівняння \[\ddot{x}+\mybra{1+\frac{1}{1+t^2}}x=0\]є обмеженою функцією на $(-\infty,\infty)$
разом зі своєю похідною.\end{prob}
Введенням нової змінної $y:=\dot{x}$ рівняння зводиться до системи
\[\begin{cases}\dot{x}=y\\\dot{y}=-\mybra{1+\frac{1}{1+t^2}}x\end{cases}\]
За теоремою 1 з \cite[\S7]{demidovich} достатньо показати, що тривіальний розв’язок системи є стійким. Це покаже те, що всі розв’язки $(x,y)$
її є обмеженими на $\mathbb{R}$, а оскільки ми ввели $y$ як $y=x'$, з цього випливатиме і бажана обмеженість першої похідної. Далі, за теоремою
1 з \cite[\S12]{demidovich} і того факту, що $\int_{-\infty}^\infty\frac{1}{1+t^2}\;dt=\arcsin t\big|_{-\infty}^\infty=\pi<+\infty$ бачимо, що
необхідно показати стійкість системи
\[\begin{bmatrix}x'\\y'\end{bmatrix}=\begin{bmatrix}0&1\\-1&0\end{bmatrix}\begin{bmatrix}x\\y\end{bmatrix}\]
стійкість цієї системи, в свою чергу, випливає з теореми 1 з \cite[\S8]{demidovich}, адже власний поліном матриці вище рівний $\lambda^2+1$
і має корені $\pm i$ із недодатніми дійсними частинами.
\begin{prob}Нехай $a>0,\;\int_0^\infty\mynorm{b(t)}\;dt<\infty$. Довести, що будь-який розв’язок рівняння 
	\[\ddot{x}+(a+b(t))x=0\] є обмеженою функцією на $[0,+\infty)$ разом зі своєю похідною\end{prob}
Повністю аналогічно до попередньої задачі. 
Введенням нової змінної $y:=\dot{x}$ рівняння зводиться до системи
\[\begin{cases}\dot{x}=y\\\dot{y}=-\mybra{a+b(t)}x\end{cases}\]
За теоремою 1 з \cite[\S7]{demidovich} достатньо показати, що тривіальний розв’язок системи є стійким. Це покаже те, що всі розв’язки $(x,y)$
її є обмеженими на $\mathbb{R}$, а оскільки ми ввели $y$ як $y=x'$, з цього випливатиме і бажана обмеженість першої похідної. Далі, за теоремою
1 з \cite[\S12]{demidovich} і того факту, що $\int_{0}^\infty\mynorm{b(t)}\;dt<+\infty$ за умовою бачимо, що
необхідно показати стійкість системи
\[\begin{bmatrix}x'\\y'\end{bmatrix}=\begin{bmatrix}0&1\\-a&0\end{bmatrix}\begin{bmatrix}x\\y\end{bmatrix}\]
стійкість цієї системи, в свою чергу, випливає з теореми 1 з \cite[\S8]{demidovich}, адже власний поліном матриці вище рівний $\lambda^2+a$
і має корені $\pm \sqrt{a}i$ із недодатніми дійсними частинами.
\begin{prob}Довести, що пряма $x=1$ є горизонтальною асимптотою при $t\to\infty$ для будь-якої інтегральної кривої рівняння
	\[\ddot{x}+\dot{x}+\mybra{1+\frac{1}{1+\sqrt{t}}}x=1+\frac{1}{1+\sqrt{t}}.\]\end{prob}
Заміною змінних $(y,z)=(x-1,\dot{x})$ рівняння зводиться до системи
\[\begin{bmatrix}y'\\z'\end{bmatrix}=\begin{bmatrix}0&1\\-1-\frac{1}{1+\sqrt{t}}&-1\end{bmatrix}\begin{bmatrix}y\\z\end{bmatrix}\]
для якої розв’язку $x=1$ рівняння відповідає тривіальний розв’язок, і отже достатньо довести, що система є асимптотично стійкою. Оскільки
$1/(1+\sqrt{t})\to0$ при $t\to\infty$, теорема 2 з \cite[\S12]{demidovich} каже нам, що достатньо показати асимптотичну стійкість системи
\[\begin{bmatrix}y'\\z'\end{bmatrix}=\begin{bmatrix}0&1\\-1&-1\end{bmatrix}\begin{bmatrix}y\\z\end{bmatrix}\]
Оскільки характеристичний поліном матриці останньої системи $\lambda^2+\lambda+1$ має корені $-1/2\pm i\sqrt{3}/2$ з від’ємними
дійсними частинами, теорема 2 з \cite[\S8]{demidovich} показує асимптотичну стійкість.
\begin{prob}Довести, що будь-який розв’язок системи 
\[\begin{cases}\dot{x}=(-3+\frac{1}{t^2})x+2y\\\dot{y}=x-(4+\frac{1}{t^3})y-e^{-3t-\frac{1}{t}}\end{cases}\]
прямує до нуля при $t\to\infty$.\end{prob}
За допомогою заміни $(a,b)=(x-e^{-3t-\frac{1}{t}},y)$ приходимо до системи
\[\begin{cases}\dot{a}=(-3+\frac{1}{t^2})a+2b\\\dot{y}=x-(4+\frac{1}{t^3})y\end{cases}\]
причому оскільки $(a,b)-(x,y)=(-e^{-3t-\frac{1}{t}},0)\to0$ при $t\to+\infty$, достатньо показати, що остання записана
система асимптотично стійка. Оскільки
$1/t^2,\;1/t^3\to0$ при $t\to+\infty$, за теоремою 2 з \cite[\S12]{demidovich}, достатньо показати асимптотичну стійкість
\[\begin{bmatrix}a'\\b'\end{bmatrix}=\begin{bmatrix}-3&2\\1&-4\end{bmatrix}\begin{bmatrix}a\\b\end{bmatrix}\]
останнє в свою чергу випливає з того, що власний поліном матриці останньої записаної системи $\lambda^2+7\lambda+10$ має корені
$-2$ та $-5$ з від’ємними дійсними частинами, і теореми 2 з \cite[\S8]{demidovich}.
\begin{prob}Нехай система зі сталою матрицею $\dot{x}=Ax$ асимптотично стійка, $\int_{t_0}^\infty\mynorm{B(t)}\;dt<\infty,\;\mynorm{f(t)}\to0,\;
t\to+\infty$. Довести, що будь-який розв’язок неоднорідної системи\[\frac{dy}{dt}=(A+B(t))y+f(t)\]прямує до нуля при $t\to+\infty$.\end{prob}
Без втрати загальності, $t_0=0$. По-перше, позначимо за $U(t)$ матриціант однорідної системи
\[y'(t)=(A+B(t))y\]
Із загальної формули для розв’язку неоднорідного рівняння випливає, що для досліджуваного рівняння маємо
\[y(t)=U(t)y(0)+\int_0^tU(t-s)f(s)\;ds\]
Припустимо, що ми зможемо довести, що для певних $c,\alpha>0$ маємо $\mynorm{U(t)}\leq ce^{-\alpha t}$. Тоді перший доданок точно прямує до нуля.
Стосовно ж другого, зафіксуємо довільне $\epsilon>0$. Оскільки $\mynorm{f(t)}\to0$, $\exists A>0\;\forall t>A,\;\mynorm{f(t)}<\epsilon$ і, до того ж,
оскільки $f(t)$ неперервна, $\exists B\;\forall t,\;\mynorm{f(t)}<B$. Таким чином, для великих $t$
\[\mynorm{\int_0^tU(t-s)f(s)\;ds}\leq c\int_0^te^{-\alpha (t-s)}\mynorm{f(s)}\;ds\leq c\mysbra{\int_0^ae^{-\alpha(t-s)}B\;ds+\int_a^te^{-
\alpha(t-s)}\epsilon\;ds}\leq\]
\[\leq c\mysbra{\tilde{B}e^{-\alpha(t-a)}+\epsilon\cdot1}\to c\epsilon\]
і таким чином, для великих $t$ другий доданок стає меншим за $2c\epsilon$ для довільного $\epsilon>0$, а отже прямує до нуля. Залишається довести
$\mynorm{U(t)}\leq ce^{-\alpha t}$.

Знову ж таки, запишемо загальну формулу розв’язку неоднорідного рівняння, на цей раз для системи $y'=(A+B)y$, на цей раз трактуючи $B(t)y(t)$ як
неоднорідність. Матимемо
Оскільки всі власні значення $A$ мають від’ємну дійсну частину, існують $\alpha,c>0$ такі, що $\mynorm{e^{At}}\leq ce^{-\alpha t}$ і відповідно
\[\mynorm{y(t)}\leq ce^{-\alpha t}\mynorm{y(0)}+\int_0^te^{-\alpha(t-s)}\mynorm{B(s)}\mynorm{y(s)}\;ds\]
еквівалентно,
\[e^{\alpha t}\mynorm{y(t)}\leq c\mynorm{y(0)}+\int_0^te^{\alpha s}\mynorm{y(s}\mynorm{B(s)}\;ds\]
і за лемою Гронуолла-Беллмана для $e^{\alpha t}\mynorm{y(t)}$ матимемо
\[e^{\alpha t}\mynorm{y(t)}\leq c\mynorm{y(0)}\exp\mycbra{\int_0^t\mynorm{B(s)}\;ds}\leq c\mynorm{y_0}\exp\mycbra{\int_0^{\infty}\mynorm{B(s)}\;ds}
:=\mynorm{y_0}D\]
Відповідно,
\[\mynorm{U(t)y(0)}=\mynorm{y(t)}\leq De^{-\alpha t}\mynorm{y_0}\implies\mynorm{U(t)}\leq De^{-\alpha t}\]
що і завершує доведення.
\begin{prob}Довести, що система \[\begin{cases}\dot{x}=-x+\cos(t)y\\\dot{y}=\cos(t)x-y\end{cases}\]є асимптотично стійкою.\end{prob}
Покажемо виконання умов теореми з \cite[\S13]{demidovich}. Треба перевірити існування границі
\[A=\lim_{t\to\infty}\frac{1}{t}\int_{t_0}^t\begin{bmatrix}-1&\cos(s)\\\cos(s)&-1\end{bmatrix}\;ds\]
Він існує, адже
\[\lim_{t\to\infty}\frac{1}{t}\int_{t_0}^t\begin{bmatrix}-1&\cos(s)\\\cos(s)&-1\end{bmatrix}\;ds=\lim_{t\to\infty}\begin{bmatrix}
	\frac{t_0-t}{t}&\frac{\sin(t)-\sin(t_0)}{t}\\\frac{\sin(t)-\sin(t_0)}{t}&\frac{t_0-t}{t}\end{bmatrix}=\begin{bmatrix}-1&0\\0&-1\end{bmatrix}
		\]
і має всі власні значення із від’ємними дійсними частинами. Залишається перевірити критерій Лаппо-Данілієвського, тобто, що для довільних
\section{стор. 25-26}
\begin{prob}Встановити асимптотичну еквівалентність між рівняннями $\dot{x_1}=-a_1x_1$ та $\dot{x_2}=-a_2x_2$, де $a_1>0$, $a_2>0$.
\end{prob}
Розв’язки рівнянь мають вигляд відповідно $x(t)=e^{-a_1t}x(0)$ та $x(t)=e^{-a_2t}x(0)$. У відповідність розв’язку $x(t)$ першого
рівняння ми ставитемо розв'язок $e^{-a_2t}x(0)$ другого. Через взаємно-однозначну відповідність розв’язків та початкових умов, ця
відповідність між розв’язками також взаємно-однозначна. Далі, для розв’язку першого рівняння $x(t)=e^{-a_1t}x(0)$
та відповідного йому розв’язку другого $x(0)e^{-a_2t}$ матимемо
\[\lim_{t\to\infty}\myabs{e^{-a_1t}x(0)-e^{-a_2t}x(0)}=\myabs{x(0)}\lim_{t\to\infty}\myabs{e^{-a_1t}-e^{-a_2t}}=0\]
що і є означенням асимптотичної еквівалентності.
\begin{prob}Довести, що за умови $\int_0^\infty\mybra{\myabs{\alpha(t)}+\myabs{\beta(t)}}<\infty$ для довільного розв’язку системи
$\begin{cases}\dot{x}=\alpha(t)y\\\dot{y}=\beta(t)x\end{cases}$ існують числа $c_1,\,c_2$ такі, що $x(t)\to c_1$, $y(t)\to c_2$.
\end{prob}
За теоремою Левінсона, дана система асимптотично еквівалентна стійкій лінійній системі
\[\begin{bmatrix}\dot{x}\\\dot{y}\end{bmatrix}=0\begin{bmatrix}x\\y\end{bmatrix}\]
і оскільки розв’язки останньої мають вигляд $(x(t),y(t))=(c_1,c_2)$ твердження випливає із означення асимптотичної еквівалентності.
\begin{prob}Довести, що система
\[\begin{cases}\dot{x}=(1+e^{-t})x-y\\\dot{y}=2x-2\mybra{1+e^{-t^2}}y\end{cases}\]
має нетривіальний розв’язок, що прямує до нуля при $t\to\infty$.
\end{prob}
За теоремою Левінсона, ця система асимптотично еквівалентна стійкій лінійній
\[\begin{bmatrix}\dot{x}\\\dot{y}\end{bmatrix}=\begin{bmatrix}1&-1\\2&-2\end{bmatrix}\begin{bmatrix}x\\y\end{bmatrix}\]
(остання є стійкою, адже власні значення матриці в правій частині, це -1 та 0) і відповідно
\section{стор. 33}
\begin{thebibliography}{9}
\bibitem{demidovich}
Демидович Б. П. \emph{Лекции по математической теории устойчивости} --
Москва, 1967 г., 472 стр. с илл.
\end{thebibliography}
\end{document}
