\documentclass[10pt]{article} % use larger type; default would be 10pt

\usepackage[utf8]{inputenc}       % кодування документа; замість cp866nav
\usepackage[margin=0.5in]{geometry}
\usepackage[russian,english]{babel} % національна локалізація; може бути декілька
\usepackage{setspace}
\usepackage{CJKutf8}
\usepackage{mdframed}

\title{Divine Liturgy\\Reading from Epistles}
\date{Week 17 after the Pentecost\vspace{-3ex}}
\begin{document}
\pagenumbering{gobble}
\begin{otherlanguage*}{russian}
\maketitle
\end{otherlanguage*}
\large\singlespacing
\framebox[\textwidth]{
\begin{minipage}[t]{0.45\textwidth}
\begin{otherlanguage*}{russian}
\textbf{2 Кор., 194 зач., XI, 31 - XII, 9.}\\
Бог и Отец Господа нашего Иисуса Христа, благословенный во веки, знает, что я не лгу.\\
В Дамаске областной правитель царя Ареты стерег город Дамаск, чтобы схватить меня; и я в корзине был спущен из окна по стене и избежал его рук. \\
Не полезно хвалиться мне, ибо я приду к видениям и откровениям Господним.\\
Знаю человека во Христе, который назад тому четырнадцать лет (в теле ли - не знаю, вне ли тела - не знаю: Бог знает) восхищен был до третьего неба.\\
И знаю о таком человеке (только не знаю - в теле, или вне тела: Бог знает),\\
что он был восхищен в рай и слышал неизреченные слова, которых человеку нельзя пересказать.\\
Таким человеком могу хвалиться; собою же не похвалюсь, разве только немощами моими.\\
Впрочем, если захочу хвалиться, не буду неразумен, потому что скажу истину; но я удерживаюсь, чтобы кто не подумал о мне более, нежели сколько во мне видит или слышит от меня.\\
И чтобы я не превозносился чрезвычайностью откровений, дано мне жало в плоть, ангел сатаны, удручать меня, чтобы я не превозносился.\\
Трижды молил я Господа о том, чтобы удалил его от меня.\\
Но Господь сказал мне: "довольно для тебя благодати Моей, ибо сила Моя совершается в немощи". И потому я гораздо охотнее буду хвалиться своими немощами, чтобы обитала во мне сила Христова\\
\end{otherlanguage*}
\end{minipage}
\hfill
\begin{minipage}[t]{0.45\textwidth}
\textbf{2 Corinthians 11:31 -- 12:9.}\\
The God and Father of our Lord Jesus Christ, which is blessed for evermore, knoweth that I lie not.\\
In Damascus the governor under Aretas the king kept the city of the Damascenes with a garrison, desirous to apprehend me:\\
And through a window in a basket was I let down by the wall, and escaped his hands.\\
It is not expedient for me doubtless to glory. I will come to visions and revelations of the Lord.\\
I knew a man in Christ above fourteen years ago, (whether in the body, I cannot tell; or whether out of the body, I cannot tell: God knoweth;) such an one caught up to the third heaven.\\
And I knew such a man, (whether in the body, or out of the body, I cannot tell: God knoweth;)\\
How that he was caught up into paradise, and heard unspeakable words, which it is not lawful for a man to utter.\\
Of such an one will I glory: yet of myself I will not glory, but in mine infirmities.\\
For though I would desire to glory, I shall not be a fool; for I will say the truth: but now I forbear, lest any man should think of me above that which he seeth me to be, or that he heareth of me.\\
And lest I should be exalted above measure through the abundance of the revelations, there was given to me a thorn in the flesh, the messenger of Satan to buffet me, lest I should be exalted above measure.\\
For this thing I besought the Lord thrice, that it might depart from me.\\
And he said unto me, My grace is sufficient for thee: for my strength is made perfect in weakness. Most gladly therefore will I rather glory in my infirmities, that the power of Christ may rest upon me.\\
\end{minipage}}
\newpage\huge\singlespacing
\framebox[\textwidth]{
\begin{minipage}[t]{\textwidth}
\begin{CJK}{UTF8}{bsmi}
\textbf{哥林多後書 11:31 -- 12:9}\\
那永遠可稱頌之主耶穌的父上帝知道我不說謊。\\
在 大馬士革的 亞哩達王手下的提督把守 大馬士革城,要捉拿我,\\
我被人用筐子從城牆上的窗口縋下,逃脫了他的手。 \\
雖然自誇無益,我還是不得不誇。我現在要提到主的異象和啟示。\\
我認識一個在基督裏的人,他在十四年前被提到第三層天上去;或在身內,我不知道,或在身外,我也不知道,只有上帝知道。\\
我認識的這樣的一個人—或在身內,或在身外,我都不知道,只有上帝知道—\\
他被提到樂園裏,聽見隱祕的言語,是人不可說的。\\
為這人,我要誇口;但是為我自己,除了我的軟弱以外,我並不誇口。\\
就是我願意誇口也不算狂,因為我會說實話;只是我絕口不談,恐怕有人把我看得太高了,過於他在我身上所看見所聽見的;\\
又恐怕我因所得的啟示太高深,就過於高抬自己,所以(註)有一根刺加在我身上,就是撒但的差役來折磨我,免得我過於高抬自己。\\
為了這事,我曾三次求主使這根刺離開我。\\
他對我說:「我的恩典是夠你用的,因為我的能力是在人的軟弱上顯得完全。」所以,我更喜歡誇耀自己的軟弱,好使基督的能力覆庇我。 \\
\end{CJK}
\end{minipage}}
\end{document}
%\singlespacing
