\documentclass[8pt]{article} % use larger type; default would be 10pt

\usepackage[margin=1in]{geometry}
\usepackage{graphicx}
\usepackage{float}
\usepackage{subfig}
\usepackage{amsmath}
\usepackage{amsfonts}
\usepackage{hyperref}
\usepackage{enumerate}
\usepackage{enumitem}
\usepackage{harpoon}

\usepackage{mystyle}

\title{Problem Set 2, MAT 5070}
\author{Alex Leontiev, 1155040702, CUHK}
\begin{document}
\maketitle
\begin{enumerate}[label=\bfseries \arabic*.]
	\item{Let us make a few preliminary observations first. To begin with, for fixed $1\leq i\leq \min\{m,n\}$, the set $M\subset \mathbb{R}^{n\times m}$
		of matrices of rank $i$ forms a submanifold of $\mathbb{R}^{n\times m}$. The proof of this fact is given as Example 8.14 in
		\cite{lee} and will not be repeated here.

		Now, given smooth $f:\mathbb{R}^m\ni x\mapsto f(x)\in\mathbb{R}^n$ we will associate with it map $Df:\mathbb{R}^m\ni x\mapsto
		Df(x):=df(x)\in\mathbb{R}^{n\times m}$ -- it's Jacobian. Note, that Jacobian is also a smooth map and for linear map $\alpha:\mathbb{R}^m\ni x\mapsto
		\alpha(x)=Ax\in\mathbb{R}^n$ we have $D(f+\alpha)=Df+A$. Now the statement of a problem can be reformulated in terms of finding such $A$ with
		norm (by "norm" here we mean the norm of corresponding linear map $\alpha$ and the particular norm used is not important for the course of a proof,
		as long as it is used consistently)
		smaller than some arbitrary prescribed number $\epsilon>0$, such that $D(f+Ax)^{-1}(M)\subset\mathbb{R}^m$ is a submanifold. From
		\cite{gp} we may recall that the sufficient condition for $D(f+Ax)^{-1}(M)$ to be a manifold is the {\it transversality} of the smooth map $D(f+Ax)=
		Df+A$
		and the submanifold $M\subset\mathbb{R}^{n\times m}$. Therefore, it is sufficient to show transversality mentioned above for some suitable $A$.
		
		In subsequent we shall use the Transversality Theorem (stated and proven in \cite[Chapter 2,\S 3]{gp}), which states that for $X$ --
		smooth manifold with boundary, $S$ and $Z\subset Y$ boundaryless smooth manifolds ($Z$ is submanifold of $Y$, which in turn
		may have boundary) and smooth map $F:X\times S\mapsto Y$ transversality of $F$ and $\partial F:=F\mid_{\partial(X
		\times S)}:\partial(X\times S)\mapsto Y$ to $Z$ implies that for almost every $s$ maps $f_s(:=F(\cdot,s):X\mapsto Y)$ and 
		$\partial f_s(:=f_s\mid_{\partial X}:\partial X\mapsto Y)$ are both transversal to $Z$.

		For our purpose, we shall take $\epsilon>0$ arbitrary fixed, let $S$ be equal to {\it open} set $\mysetn{A\in\mathbb{R}^{n\times m}}{
		\mynorm{A}<\epsilon}$ (by $\mynorm{A}$ here we mean matrix norm induced by the norm of corresponding linear map $\alpha(x)=Ax$)
		. As it is open, it is a boundaryless submanifold of $\mathbb{R}^{n\times m}$, so this requirement is satisfied. Next, we take
		$Z\subset Y$ pair to be equal to $M\subset\mathbb{R}^{n\times m}$. And finally, we take $X=\mathbb{R}^n$ and $F(x,s)=Df(x)+s$. Note, that as
		$X$ is boundaryless, we have nothing to check regarding $\partial F$. Now, as
		for every particular $x$ fixed, map $F(x,\cdot)$ is the translation of $S$, it is submersion, and hence $F(x,s)$ with both parameters free
		is also submersion. Hence, $F$ is transversal to any submanifold of $\mathbb{R}^{n\times m}$, in particular to $M$.
		Thus by Transversality Theorem for almost every $s\in S$, map $Df(x)+s$ will be transversal to $M$ and thus $D(f+sx)^{-1}(M)\subset\mathbb{R}^m$ will
		be a submanifold.
		}
	\item{\renewcommand{\dim}{\mbox{dim }}
		Having $n$ and $k$ fixed as in the statement of a problem, we shall denote by $S\subset\mathbb{R}^{k\times k}$ the set of all symmetric $k$-by-$k$
		matrices. As it is a vector subspace of $\mathbb{R}^{k\times k}$, it is naturally a submanifold as well,
		and it is important to note that it is "flat" in the sense that for $\forall s\in S$ every derivation in $T_s S$ can be represented as
		$X_{s'}:C^{\infty}(S)\ni f\mapsto X_{s'}(f):=\frac{df(s+s't)}{dt}\mid_{t=0}\in\mathbb{R}$ for some unique $s'\in S$. It is also important to keep in
		mind that dimension of $S$ is $\frac{k(k+1)}{2}$, which we also take as a known fact.

		Now, let us set up the smooth mapping $F:\mathbb{R}{kn}\ni A\mapsto F(A):=A^TA\in S$. Here we interpret members of $\mathbb{R}^{kn}$ as $n$-by-$k$
		matrices, so in particular they may be seen as matrices $A=\begin{bmatrix}v_1&v_2&\dots&v_k\end{bmatrix}$ with $k$ columns being the vectors in 
		$\mathbb{R}^n$. Hence, in particular we may treat $V_n^k$ as a subset of $\mathbb{R}^{kn}$. As $\forall A\in\mathbb{R}^{kn},\;(A^TA)^T=A^TA$, image of
		$F$ indeed lies in $S$. As from the definition we see that $V_n^k=F^{-1}(\left\{I\right\})$ we see that if we will be able to show that identity matrix
		$I\in S$ is a regular value of $F$, then the statement that $V_n^k\subset\mathbb{R}^{nk}$ is a submanifold will follow from the preimage theorem.
		Thus, we now have to show that for arbitrary $A\in V_n^k$, $dF:T_A\mathbb{R}^{kn}\mapsto T_IS$ is surjective. As $\mathbb{R}^{kn}$ is also flat (as
		it is also a vector space), every member of $T_A\mathbb{R}^{kn}$ can be represented as $X_B:C^{\infty}(\mathbb{R}^{kn})\ni f\mapsto
		X_B(f):=\frac{df(A+Bt)}{dt}\mid_{t=0}\in\mathbb{R}$ for some unique $B\in\mathbb{R}^{kn}$. In this formulation,
		\[dF(X_B)(f)=\lim_{t\to 0}\frac{f(F(A+Bt))-f(F(A))}{t}=\lim_{t\to 0}\frac{f(I+(A^TB+B^TA)t+B^TBt^2)-f(I)}{t}\]applying chain rule we see that
		\[dF(X_B)(f)=\lim_{t\to 0}\frac{f(I+(A^TB+B^TA)t+B^TBt^2)-f(I)}{t}=\nabla f\cdot\lim_{t\to 0}\frac{(A^TB+B^TA)t+B^TBt^2}{t}=\]
		\[=\nabla f\cdot(A^TB+B^TA)=\lim_{t\to 0}\frac{f(I+(A^TB+B^TA)t)-f(I)}{t}=X_{A^TB+B^TA}(f)\]
		In the light of the discussion in the first paragraph, as every derivation in $T_I S$ can be represented as $X_{s'}$ for some $s'\in S$, to
		establish surjectivity we need to show that $\forall s\in S$ can be represented as $s=A^TB+B^TA$ for suitably chosen $B\in\mathbb{R}^{kn}$.
		Let us choose $B=\frac{1}{2}(A^T)^{-1}s$. Then $A^TB+B^TA=\frac{1}{2}A^T(A^T)^{-1}s+\frac{1}{2}s^TA^{-1}A=\frac{1}{2}s+\frac{1}{2}s^T=s$,
		as $s$ is symmetric, and thus
		surjectivity is proven, $I$ is indeed regular value of $F$, so preimage theorem yields that $V_n^k\subset\mathbb{R}^{kn}$ is a submanifold.
		Besides, by the very same preimage theorem, $\dim V_n^k=\dim \mathbb{R}^{nk}-\dim S=nk-\frac{k(k+1)}{2}$.

		Finally, $V_n^k\subset\mathbb{R}^{kn}$ is compact as it is the subset of euclidean space which is both closed and bounded. It is closed, as
		$V_n^k=F^{-1}(\left\{I\right\})$ with $F$ being smooth (and hence continuous) and $\left\{I\right\}\subset\mathbb{R}^{k\times k}$ being closed
		as a singleton, and hence $V_n^k$ is the preimage of a closed set under smooth map, hence closed. To see that it is bounded, we consider the standard
		2-norm on $\mathbb{R}^{kn}$ (when treated as set of vectors, not matrices)
		and note that for arbitrary $A\in V_n^k$ we have $\mynorm{A}^2=\sum_{i=1}^k\mynorm{v_i}_2^2=k$ and hence 2-norm of all members of $V_n^k$ is uniformly
		bounded.
		}
		\item{Following the scheme of proof outlined in Exercise 11 of Chapter 2, $\S1$ of \cite{gp}, let us first establish two auxiliary statements.
			\begin{myprop}Let $x\in\partial X$, then there exists a neighborhood $U_x\ni x$
				and $f_x:U_x\mapsto\mathbb{R}_{\geq 0}$ 
				$f_x(z)=0\iff z\in\partial X$ and besides $df_x(\myvec{n_z})>0$
				for every $z\in\partial X$, where $\myvec{n_z}\in T_z X$.
		\end{myprop}
		\begin{myproof}
			Given $x\in\partial X$, it has neighborhood $U_x$ that is diffeomorphic to the open half-space
			$H_n:={x_n\geq 0}\subset\mathbb{R}^n$ (where $n$ is assumed to be the dimension of $X$), thus we have
			diffeomorphism $\psi: U_x\mapsto H_n$ and we define $f_x(z):=\left(\psi(x)\right)_n:
			U_x\mapsto\mathbb{R}_{\geq 0}$ -- the last coordinate
			of $\psi(z)$. By definition of $\partial X$ we see that $_x(z)=0\iff z\in\psi^{-1}(\mysetn{z\in H_n}
			{z_n=0})=\partial X\cap U_x$.

			Now, given $z\in U_x\cap\partial X$
			let us select $n_z:=d\psi^{-1}(\frac{\partial}{\partial x_n})\in T_x X$ and then (introducing the function
			$\pi_n:H_n\ni z=(z_1,z_2,\dots,z_n)\mapsto \pi_n(z):=z_n\in\mathbb{R_{\geq 0}}$)
			$df_x(n_z)=d\pi_n d\psi (n_z)=d\pi_n(\frac{\partial}{\partial x_n})=1>0$ 
		\myqed\end{myproof}
		Now, let $\left\{\phi_x\right\}_{x\in\partial X}\cup\left\{\phi_0\right\}$ be the partition of unity subordinate
		to the open cover $\left\{U_x\right\}_{x\in\partial X}\cup\left\{U_0:=X\setminus\partial X\right\}$ with
		$U_x$ defined as in Proposition above for every $x\in\partial X$. Let us define a smooth function $f:M\ni z\mapsto
		f(x):=\sum_{x\in\partial X}\phi_x(z) f_x(z)+\phi_0(z)\in\mathbb{R}_{\geq 0}$
		with $\phi_x$ defined as in Proposition above. Now, we claim that $f(x)=0\iff
		x\in\partial X$. If $z\in X\setminus\partial X$
		then there should be some member of partition of unity that does not vanish on neighborhood of $z$. If such
		member can be chosen to be $\phi_0$, then $f(z)\geq \phi_0(z)>0$. If $\phi_x(z)>0$, then
		$z\in\mbox{supp }\phi_x\subset U_x$, hence $f(z)\geq \phi_x(z) f_x(z)>0$, as $\phi_x(z)>0$ by assumption
		and $f_x(z)>0$ by Proposition. Conversely, if $z\in\partial X$, then $z\notin U_0\supset
		\mbox{supp }\phi_0\implies \phi_0(z)=0$ and if $\phi_x(z)>0$, then $\phi_x(z)f_x(z)=0$, as $f_x(z)=0$ by Proposition,
		thus $f(z)=0$.

		Finally, let us show that for arbitrary $z\in\partial X$, $z$ is a regular point of $f$. Given such $z$, as
		$z\notin U_0\supset\mbox{supp }\phi_0\implies \phi_0(z)=0$, there are only finitely many $\left\{
		f_{x_n}\right\}_{n=1}^N$, that do not vanish on some suitably chosen neighborhood $U\ni z$. Then,
		denoting $f_n:=f_{x_n}$, we have that
		$f(z')=\sum_{i=1}^N f_i(z')\phi_i(z')$ for $z'\in U$.
		Let us also introduce $\myvec{n}\in T_z X$ corresponding to Proposition above applied to $z$ and $U_{x_1}\ni z$. 
		We will now show that $d(f_i\phi_i)(\myvec{n})>0$ for all $i$, which will
		imply that $df(\myvec{n})>0$ and thus $z$ is a regular value of $f$. Recalling the way we defined $\myvec{n}$ we
		have that $d(f_i\phi_i)(\myvec{n})=f_i(z)d\phi_i(\myvec{n})+df_i(\myvec{n})\phi_i(z)=df_i(\myvec{n})d\phi_i(z)$
		and is positive iff $df_i(\myvec{n})$ is positive (as $\phi_i(z)>0$ by assumption).
		Continuing, $df_i(\myvec{n})=d\pi_n d\psi_i(\frac{\partial}{\partial x_n})$, where $\psi_i:V\mapsto W$
		 is a diffeomorphism
		 between the open subsets of $V,W\subset H_n$ containing the image of $z$ under $\psi_{x_1}$ and
		 $\psi_{x_i}$ respectively. Therefore, by shifting $\psi_{x_1}$ and $\psi_{x_i}$ (this won't affect
		 their pushforwards and hence $d\psi_i$) we may assume that both images of $z$ correspond to $0\in V,W$. 
		 As $d\pi_n$ is just the projection on the last component of the vector, it is enough to show that $d\psi_i(\frac{\partial}{\partial x_n})\in H_n
		 \setminus\partial H_n$. Now, as $\psi_i$ is the diffeomorphism between the open subsets of $H_n$, both $\psi_i$ and its
		 inverse should map points of $\partial H_n$ to points of $\partial H_n$. Hence, the restriction of $\psi_i$ to $\partial H_n$ induces
		 diffeomorphism between $\partial H_n\cap V$ and $\partial H_n\cap W$ and hence $\myvec{v}\in\partial H_n\iff \psi_i(\myvec{v})\in\partial H_n$ and
		 thus $d\psi_i(\frac{\partial}{\partial x_n})\notin\partial H_n$. Finally, $d\psi_i(\frac{\partial}{\partial x_n})\notin \mathbb{R}^n
		 \setminus H_n$, as if we would have $d\psi_i(\frac{\partial}{\partial x_n})\in\mathbb{R}^n\setminus H_n$, then curve $\gamma:
		 [0,+\infty)\ni x\mapsto\gamma(x):=\psi_i((0,0,\dots,0,x))$ would have $\gamma'(0)=d\psi_i(\frac{\partial}{\partial x_n})\in\mathbb{R}^n\setminus H_n$
		 and hence for small $\epsilon>0$ we would have $\gamma(\epsilon)\notin H_n$, however for all 
		 small $\epsilon>0$ we have $(0,0,\dots,0,\epsilon)\in V
		 \subset H_n$, thus
		 $\psi_i$ would map points of $H_n$ outside $H_n$, yielding contradiction. After all this we see that $df_i(\myvec{n})>0$, hence
		 $z$ is the regular value of $f$ and we are done.
			}
\end{enumerate}
\begin{thebibliography}{9}
	\bibitem{gp} {\em Introduction to Smooth Manifolds} John M. Lee, 2006 Springer
	\bibitem{lee} {\em Differential Topology}, Victor Guillemin , Alan Pollack
\end{thebibliography}
\end{document}
