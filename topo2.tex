\documentclass[8pt]{article} % use larger type; default would be 10pt

\usepackage[margin=1in]{geometry}
\usepackage{graphicx}
\usepackage{float}
\usepackage{subfig}
\usepackage{amsmath}
\usepackage{amsfonts}
\usepackage{hyperref}
\usepackage{enumerate}
\usepackage{enumitem}

\usepackage{mystyle}

\title{Problem Set 2, MAT 5070}
\author{Alex Leontiev, 1155040702, CUHK}
\begin{document}
\maketitle
\begin{enumerate}[label=\bfseries \arabic*.]
	\item{Let us make a few preliminary observations first. To begin with, for fixed $1\leq i\leq \min\{m,n\}$, the set $M\subset \mathbb{R}^{n\times m}$
		of matrices of rank $i$ forms a submanifold of $\mathbb{R}^{n\times m}$. The proof of this fact is given as Example 8.14 in
		\cite{lee} and will not be repeated here.

		Now, given smooth $f:\mathbb{R}^m\ni x\mapsto f(x)\in\mathbb{R}^n$ we will associate with it map $Df:\mathbb{R}^m\ni x\mapsto
		Df(x):=df(x)\in\mathbb{R}^{n\times m}$ -- it's Jacobian. Note, that Jacobian is also a smooth map and for linear map $\alpha:\mathbb{R}^m\ni x\mapsto
		\alpha(x)=Ax\in\mathbb{R}^n$ we have $D(f+\alpha)=Df+A$. Now the statement of a problem can be reformulated in terms of finding such $A$ with
		norm (by "norm" here we mean the norm of corresponding linear map $\alpha$ and the particular norm used is not important for the course of a proof,
		as long as it is used consistently)
		smaller than some arbitrary prescribed number $\epsilon>0$, such that $D(f+Ax)^{-1}(M)\subset\mathbb{R}^m$ is a submanifold. From
		\cite{gp} we may recall that the sufficient condition for $D(f+Ax)^{-1}(M)$ to be a manifold is the {\it transversality} of the smooth map $D(f+Ax)=
		Df+A$
		and the submanifold $M\subset\mathbb{R}^{n\times m}$. Therefore, it is sufficient to show transversality mentioned above for some suitable $A$.
		
		In subsequent we shall use the Transversality Theorem (stated and proven in \cite[Chapter 2,\S 3]{gp}), which states that for $X$ --
		smooth manifold with boundary, $S$ and $Z\subset Y$ boundaryless smooth manifolds ($Z$ is submanifold of $Y$, which in turn
		may have boundary) and smooth map $F:X\times S\mapsto Y$ transversality of $F$ and $\partial F:=F\mid_{\partial(X
		\times S)}:\partial(X\times S)\mapsto Y$ to $Z$ implies that for almost every $s$ maps $f_s(:=F(\cdot,s):X\mapsto Y)$ and 
		$\partial f_s(:=f_s\mid_{\partial X}:\partial X\mapsto Y)$ are both transversal to $Z$.

		For our purpose, we shall take $\epsilon>0$ arbitrary fixed, let $S$ be equal to {\it open} set $\mysetn{A\in\mathbb{R}^{n\times m}}{
		\mynorm{A}<\epsilon}$ (by $\mynorm{A}$ here we mean matrix norm induced by the norm of corresponding linear map $\alpha(x)=Ax$)
		. As it is open, it is a boundaryless submanifold of $\mathbb{R}^{n\times m}$, so this requirement is satisfied. Next, we take
		$Z\subset Y$ pair to be equal to $M\subset\mathbb{R}^{n\times m}$. And finally, we take $X=\mathbb{R}^n$ and $F(x,s)=Df(x)+s$. Note, that as
		$X$ is boundaryless, we have nothing to check regarding $\partial F$. Now, as
		for every particular $x$ fixed, map $F(x,\cdot)$ is the translation of $S$, it is submersion, and hence $F(x,s)$ with both parameters free
		is also submersion. Hence, $F$ is transversal to any submanifold of $\mathbb{R}^{n\times m}$, in particular to $M$.
		Thus by Transversality Theorem for almost every $s\in S$, map $Df(x)+s$ will be transversal to $M$ and thus $D(f+sx)^{-1}(M)\subset\mathbb{R}^m$ will
		be a submanifold.
		}
	\item{\renewcommand{\dim}{\mbox{dim }}
		Having $n$ and $k$ fixed as in the statement of a problem, we shall denote by $S\subset\mathbb{R}^{k\times k}$ the set of all symmetric $k$-by-$k$
		matrices. As it is a vector subspace of $\mathbb{R}^{k\times k}$, it is naturally a submanifold as well,
		and it is important to note that it is "flat" in the sense that for $\forall s\in S$ every derivation in $T_s S$ can be represented as
		$X_{s'}:C^{\infty}(S)\ni f\mapsto X_{s'}(f):=\frac{df(s+s't)}{dt}\mid_{t=0}\in\mathbb{R}$ for some unique $s'\in S$. It is also important to keep in
		mind that dimension of $S$ is $\frac{k(k+1)}{2}$, which we also take as a known fact.

		Now, let us set up the smooth mapping $F:\mathbb{R}{kn}\ni A\mapsto F(A):=A^TA\in S$. Here we interpret members of $\mathbb{R}^{kn}$ as $n$-by-$k$
		matrices, so in particular they may be seen as matrices $A=\begin{bmatrix}v_1&v_2&\dots&v_k\end{bmatrix}$ with $k$ columns being the vectors in 
		$\mathbb{R}^n$. Hence, in particular we may treat $V_n^k$ as a subset of $\mathbb{R}^{kn}$. As $\forall A\in\mathbb{R}^{kn},\;(A^TA)^T=A^TA$, image of
		$F$ indeed lies in $S$. As from the definition we see that $V_n^k=F^{-1}(\left\{I\right\})$ we see that if we will be able to show that identity matrix
		$I\in S$ is a regular value of $F$, then the statement that $V_n^k\subset\mathbb{R}^{nk}$ is a submanifold will follow from the preimage theorem.
		Thus, we now have to show that for arbitrary $A\in V_n^k$, $dF:T_A\mathbb{R}^{kn}\mapsto T_IS$ is surjective. As $\mathbb{R}^{kn}$ is also flat (as
		it is also a vector space), every member of $T_A\mathbb{R}^{kn}$ can be represented as $X_B:C^{\infty}(\mathbb{R}^{kn})\ni f\mapsto
		X_B(f):=\frac{df(A+Bt)}{dt}\mid_{t=0}\in\mathbb{R}$ for some unique $B\in\mathbb{R}^{kn}$. In this formulation,
		\[dF(X_B)(f)=\lim_{t\to 0}\frac{f(F(A+Bt))-f(F(A))}{t}=\lim_{t\to 0}\frac{f(I+(A^TB+B^TA)t+B^TBt^2)-f(I)}{t}\]applying chain rule we see that
		\[dF(X_B)(f)=\lim_{t\to 0}\frac{f(I+(A^TB+B^TA)t+B^TBt^2)-f(I)}{t}=\nabla f\cdot\lim_{t\to 0}\frac{(A^TB+B^TA)t+B^TBt^2}{t}=\]
		\[=\nabla f\cdot(A^TB+B^TA)=\lim_{t\to 0}\frac{f(I+(A^TB+B^TA)t)-f(I)}{t}=X_{A^TB+B^TA}(f)\]
		In the light of the discussion in the first paragraph, as every derivation in $T_I S$ can be represented as $X_{s'}$ for some $s'\in S$, to
		establish surjectivity we need to show that $\forall s\in S$ can be represented as $s=A^TB+B^TA$ for suitably chosen $B\in\mathbb{R}^{kn}$.
		Let us choose $B=\frac{1}{2}(A^T)^{-1}s$. Then $A^TB+B^TA=\frac{1}{2}A^T(A^T)^{-1}s+\frac{1}{2}s^TA^{-1}A=\frac{1}{2}s+\frac{1}{2}s^T=s$,
		as $s$ is symmetric, and thus
		surjectivity is proven, $I$ is indeed regular value of $F$, so preimage theorem yields that $V_n^k\subset\mathbb{R}^{kn}$ is a submanifold.
		Besides, by the very same preimage theorem, $\dim V_n^k=\dim \mathbb{R}^{nk}-\dim S=nk-\frac{k(k+1)}{2}$.

		Finally, $V_n^k\subset\mathbb{R}^{kn}$ is compact as it is the subset of euclidean space which is both closed and bounded. It is closed, as
		$V_n^k=F^{-1}(\left\{I\right\})$ with $F$ being smooth (and hence continuous) and $\left\{I\right\}\subset\mathbb{R}^{k\times k}$ being closed
		as a singleton, and hence $V_n^k$ is the preimage of a closed set under smooth map, hence closed. To see that it is bounded, we consider the standard
		2-norm on $\mathbb{R}^{kn}$ (when treated as set of vectors, not matrices)
		and note that for arbitrary $A\in V_n^k$ we have $\mynorm{A}^2=\sum_{i=1}^k\mynorm{v_i}_2^2=k$ and hence 2-norm of all members of $V_n^k$ is uniformly
		bounded.
		}
		\item{Following the scheme of proof outlined in Exercise 11 of Chapter 2, $\S1$ of \cite{gp}, let us first establish two auxiliary statements.
			\begin{proposition}Let $x\in\partial X$, there exists a neighborhood $U\ni x$ and $f:U\mapsto\mathbb{R}$ smooth, such that

		\end{proposition}
		\begin{proof}
			On the one hand, assume $z\in\partial U$ 
		\qed\end{proof}
			}
\end{enumerate}
\begin{thebibliography}{9}
	\bibitem{gp} {\em Introduction to Smooth Manifolds} John M. Lee, 2006 Springer
	\bibitem{lee} {\em Differential Topology}, Victor Guillemin , Alan Pollack
\end{thebibliography}
\end{document}
