\documentclass[8pt]{article} % use larger type; default would be 10pt

\usepackage[margin=1in]{geometry}
\usepackage{graphicx}
\usepackage{float}
\usepackage{subfig}
\usepackage{amsmath}
\usepackage{amsfonts}
\usepackage{hyperref}
\usepackage{enumitem}
\usepackage[neverdecrease]{paralist}

\usepackage{mystyle}

\title{Math 1540\\University Mathematics for Financial Studies\\2013-14 Term 1\\Suggested solutions for\\
Sec. 14.9, 15.1--15.2}
\begin{document}
\maketitle
\section{Section 14.9}
\begin{description}
	\item[\# 8.]{{\it Use Taylor's formula for $f(x,y)$ at the origin to find quadratic and cubic approximations of $f$ near the origin.
		\[f(x,y)=\ln(2x+y+1)\]}
		First, let's compute the partial derivatives at the origin
		\[f(0,0)=0\]
		\[f_x(0,0)=\frac{2}{2x+y+1}\bigg|_{(0,0)}=2\]
		\[f_y(0,0)=\frac{1}{2x+y+1}\bigg|_{(0,0)}=1\]
		\[f_{xx}(0,0)=\frac{-4}{(2x+y+1)^2}\bigg|_{(0,0)}=-4\]
		\[f_{xy}(0,0)=f_{yx}(0,0)=\frac{-2}{(2x+y+1)^2}\bigg|_{(0,0)}=-2\]
		\[f_{yy}(0,0)=\frac{-1}{(2x+y+1)^2}\bigg|_{(0,0)}=-1\]
		\[f_{xxx}(0,0)=\frac{16}{(2x+y+1)^3}\bigg|_{(0,0)}=16\]
		\[f_{xxy}(0,0)=f_{xyx}(0,0)=f_{yxx}(0,0)=\frac{8}{(2x+y+1)^3}\bigg|_{(0,0)}=8\]
		\[f_{xyy}(0,0)=f_{yxy}(0,0)=f_{yyx}(0,0)=\frac{4}{(2x+y+1)^3}\bigg|_{(0,0)}=4\]
		\[f_{yyy}(0,0)=\frac{2}{(2x+y+1)^3}\bigg|_{(0,0)}=2\]
		Having these date, the rest is just a matter of substitution into formulas. The quadratic approximation goes as
		\[P_2(x,y)=f(0,0)+f_x(0,0)x+f_y(0,0)y+\frac{1}{2!}\mybra{x^2f_{xx}(0,0)+2xyf_{xy}(0,0)+y^2f_{yy}(0,0)}=2x+y-2x^2-2xy-\frac{1}{2}y^2\]
		while the cubic is
		\[P_3(x,y)=f(0,0)+f_x(0,0)x+f_y(0,0)y+\frac{1}{2!}\mybra{x^2f_{xx}(0,0)+2xyf_{xy}(0,0)+y^2f_{yy}(0,0)}+\]\[+\frac{1}{3!}\mybra{
		x^3f_{xxx}(0,0)+3x^2yf_{xxy}(0,0)+3xy^2f_{xyy}(0,0)+y^3f_{yyy}(0,0)}=\]
		\[=2x+y-2x^2-2xy-\frac{1}{2}y^2+\frac{8}{3}x^3+4x^2y+2xy^2+\frac{1}{3}y^3\]
		}
	\item[\# 9.]{{\it Use Taylor's formula for $f(x,y)$ at the origin to find quadratic and cubic approximations of $f$ near the origin.
		\[f(x,y)=\frac{1}{1-x-y}\]}
		Again, first let us compute the partial derivatives at the origin
		\[f(0,0)=1\]
		\[f_x(0,0)=\frac{1}{(1-x-y)^2}\bigg|_{(0,0)}=1\]
		\[f_y(0,0)=\frac{1}{(1-x-y)^2}\bigg|_{(0,0)}=1\]
		\[f_{xx}(0,0)=\frac{2}{(1-x-y)^3}\bigg|_{(0,0)}=2\]
		\[f_{xy}(0,0)=f_{yx}(0,0)=\frac{2}{(1-x-y)^3}\bigg|_{(0,0)}=2\]
		\[f_{yy}(0,0)=\frac{2}{(1-x-y)^3}\bigg|_{(0,0)}=2\]
		\[f_{xxx}(0,0)=\frac{6}{(1-x-y)^4}\bigg|_{(0,0)}=6\]
		\[f_{xxy}(0,0)=f_{xyx}(0,0)=f_{yxx}(0,0)=\frac{6}{(1-x-y)^4}\bigg|_{(0,0)}=6\]
		\[f_{xyy}(0,0)=f_{yxy}(0,0)=f_{yyx}(0,0)=\frac{6}{(1-x-y)^4}\bigg|_{(0,0)}=6\]
		\[f_{yyy}(0,0)=\frac{6}{(1-x-y)^4}\bigg|_{(0,0)}=6\]
		Having these date, the rest is just a matter of substitution into formulas. The quadratic approximation goes as
		\[P_2(x,y)=f(0,0)+f_x(0,0)x+f_y(0,0)y+\frac{1}{2!}\mybra{x^2f_{xx}(0,0)+2xyf_{xy}(0,0)+y^2f_{yy}(0,0)}=1+x+y+x^2+2xy+y^2\]
		while the cubic is
		\[P_3(x,y)=f(0,0)+f_x(0,0)x+f_y(0,0)y+\frac{1}{2!}\mybra{x^2f_{xx}(0,0)+2xyf_{xy}(0,0)+y^2f_{yy}(0,0)}+\]\[+
		\frac{1}{3!}\mybra{x^3f_{xxx}(0,0)+3x^2yf_{xxy}(0,0)+3xy^2f_{xyy}(0,0)+y^3f_{yyy}(0,0)}=\]
		\[=1+x+y+x^2+2xy+y^2+y^3+3x^2y+3xy^2+y^3\]
		}
	\item[\# 12.]{{\it Use Taylor's formula to find a quadratic approximation of $e^x\sin y$ at the origin. Estimate the error in the 
		approximation if $\myabs{x}\leq 0.1$ and $\myabs{y}\leq 0.1$.}\\
		Again, we start with computing partial derivatives
		\[f(0,0)=0\]
		\[f_x(0,0)=e^x\sin y\bigg|_{(0,0)}=0\]
		\[f_y(0,0)=e^x\cos y\bigg|_{(0,0)}=1\]
		\[f_{xx}(0,0)=e^x\sin y\bigg|_{(0,0)}=0\]
		\[f_{xy}(0,0)=f_{yx}(0,0)=e^x\cos y\bigg|_{(0,0)}=1\]
		\[f_{yy}(0,0)=-e^x\sin y\bigg|_{(0,0)}=0\]
		}
		Thus quadratic approximation together with error term goes as
		\[f(x,y)=\underbrace{f(0,0)+f_x(0,0)x+f_y(0,0)y+\frac{1}{2!}\mybra{x^2f_{xx}(0,0)+2xyf_{xy}(0,0)+y^2f_{yy}(0,0)}}_{P_2(x,y)}+\]
		\[+\underbrace{\frac{1}{3!}\mybra{x^3f_{xxx}(x',y')+3x^2yf_{xxy}(x',y')+3xy^2f_{xyy}(x',y')+y^3f_{yyy}(x',y')}
		}_{\mbox{error term, where $\myabs{x'-0}\leq\myabs{x-0}\leq 0.1$, $\myabs{y'-0}\leq\myabs{y-0}\leq 0.1$}}=\]
		so \[P_2(x,y)=y+xy\]
		and \[\myabs{\frac{1}{3!}\mybra{x^3f_{xxx}(x',y')+3x^2yf_{xxy}(x',y')+3xy^2f_{xyy}(x',y')+y^3f_{yyy}(x',y')}}\leq\]
		\[\leq \frac{1}{6}\mybra{
		\myabs{x^3f_{xxx}(x',y')}+\myabs{3x^2yf_{xxy}(x',y')}+\myabs{3xy^2f_{xyy}(x',y')}+\myabs{y^3f_{yyy}(x',y')}}=\]
		\[=\frac{1}{6}\mybra{
		\myabs{x}^3\cdot\myabs{f_{xxx}(x',y')}+
		3\cdot\myabs{x}^2\cdot\myabs{y}\cdot\myabs{f_{xxy}(x',y')}+
		3\cdot\myabs{x}\cdot\myabs{y}^2\cdot\myabs{f_{xyy}(x',y')}+
		\myabs{y}^3\cdot\myabs{f_{yyy}(x',y')}}=
		\]
		\[=\frac{1}{6}\mybra{\myabs{x}^3\cdot\myabs{e^x\sin y}+
		3\cdot\myabs{x}^2\cdot\myabs{y}\cdot\myabs{e^x\cos y}+
		3\cdot\myabs{x}\cdot\myabs{y}^2\cdot\myabs{-e^x\sin y}+
		\myabs{y}^3\cdot\myabs{-e^x\cos y}}=\]
		\[=\frac{1}{6}\mybra{\myabs{x}^3\cdot\myabs{e^x}\cdot\myabs{\sin y}+
		3\cdot\myabs{x}^2\cdot\myabs{y}\cdot\myabs{e^x}\cdot\myabs{\cos y}+
		3\cdot\myabs{x}\cdot\myabs{y}^2\cdot\myabs{e^x}\cdot\myabs{\sin y}+
		\myabs{y}^3\cdot\myabs{e^x}\cdot\myabs{\cos y}}\leq\]
		\[\frac{1}{6}\mybra{0.1^3\cdot e^{0.1}\cdot\sin(0.1)+3\cdot 0.1^2\cdot 0.1\cdot e^{0.1}\cdot1+
		3\cdot 0.1\cdot0.1^2\cdot e^{0.1}\cdot\sin(0.1)+0.1^3\cdot e^{0.1}\cdot 1}=\frac{0.004\cdot e^{0.1}}{3}(2+2\sin(0.1))=
		\]\[=0.00324134375148227022\]
		\textbf{Note. } Answer to this sort of problems (estimation) is \textit{not} unique and, in principle, any answer
		bigger than $0.00324134375148227022$ (and, perhaps, some smaller) will be valid.
\section{Section 15.1}
	\item[\# 19.]{{\it Evaluate the double integral over the given region $R$.}
		\[\iint\limits_R \frac{xy^3}{x^1+1}dA,\quad R:\;0\leq x\leq1,\;0\leq y\leq2\]
		Computations are straightforward
		\[\iint\limits_R \frac{xy^3}{x^1+1}dA=\iint\limits_R \frac{xy^3}{x^1+1}dxdy=\int_0^1\frac{x}{x^2+1}dx\int_0^2y^3dy=\]
		\[=\frac{1}{2}\ln(x^2+1)\bigg|_0^1\cdot \frac{y^4}{4}\bigg|_0^2=2\ln2\]
		}
\section{Section 15.2}
\newcommand{\dx}{\;dx}
\newcommand{\dy}{\;dy}
\newcommand{\dz}{\;dz}
	\item[\# 24.]{{\it Sketch the region of integration and evaluate the integral.}
		\[\int_1^4\int_0^{\sqrt{x}}\frac{3}{2}e^{y/\sqrt{x}}dydx\]
		The sketch of the region looks like\\
		\mypic{0.6}{fin_hw_graph1.jpg}
		Computations are straightforward
		\[\begin{array}{rr}
		\int_1^4\int_0^{\sqrt{x}}\frac{3}{2}e^{y/\sqrt{x}}dydx=&\myexplain{variable change $t=y/\sqrt{x}\implies dt=dy/\sqrt{x}$}\\
		\end{array}\]
		\[=\int_1^4\int_0^1\frac{3}{2}e^{t}\sqrt{x}dtdx=\int_1^4\frac{3}{2}(e-1)\sqrt{x}dx=\frac{3(e-1)}{2}\frac{2}{3}x^{\frac{3}{2}}\bigg|_0
		^4=8(e-1)\]
		}
	\item[\# 42.]{{\it Sketch the region of integration and write an equivalent double integral with the order of integration reversed.}
		\[\int_0^2\int_{-\sqrt{4-x^2}}^{4-x^2}6x\dy\dx\]
		The sketch of the region looks like\\
		\mypic{0.6}{fin_hw_graph2.jpg}
		And looking on the picture, it is easy to see how to reverse the order of integration
		\[\int_0^2\int_{-\sqrt{4-x^2}}^{4-x^2}6x\dy\dx=\int_{-2}^2\int_0^{\sqrt{4-y^2}}6x\dx\dy\]
		}
	\item[\# 54.]{{\it Sketch the region of integration, reverse the order of integration and evaluate the integral.}
		\[\int_0^8\int_{\sqrt[3]{x}}^2\frac{\dy\dx}{y^4+1}\]
		The region of integration can be sketched like this.
		\mypic{0.6}{fin_hw_graph3.jpg}
		By the way, it is interesting to note that near the origin, the slope of $f(x)=\sqrt[3]{x}$ should become vertical, corresponding 
		to the fact that $f'(0)=+\infty$. From the picture one sees that the order should be reversed and computations should be done as
		\[\int_0^8\int_{\sqrt[3]{x}}^2\frac{\dy\dx}{y^4+1}=\int_0^2\int_0^{y^3}\frac{1}{y^4+1}\dx\dy=\int_0^2\frac{1}{y^4+1}y^3dy=
		\frac{1}{4}\ln (y^4+1)\bigg|_0^2=\frac{1}{4}\ln17\]
		}
	\item[\# 60.]{{\it Find the volume of the solid in the first octant bounded by the coordinate planes, the cylinder $x^2+y^2=4$, and the plane
		$z+y=3$.}\\
		The solid we are interested in can be described by inequalities
		\[x,y,z\geq 0,\;x^2+y^2\leq4,\;z+y\leq 3\footnote{
		both later inequalities should be "$\leq$" indeed, as only first of them involves $x$ and only second involves $z$, hence if
		first (second) of them is "$\geq$", the $x$ (or $y$) is allowed to take infinite values, hence the solid becomes unbounded
		}\]
		The last inequality can be written as $0\leq z\leq 3-y$ (note that, $x^2+y^2\leq 4\implies y\leq 2\implies 3-y>0$)
		Thus we can try to compute volume as
		\[\def\arraystretch{3}
		\begin{array}{rr}
	V=\displaystyle\iint\limits_{\mycbra{x,y\geq0,\;x^2+y^2\leq4}}\mybra{\int_0^{3-y}1\dz}\dx\dy=&\myexplain{passing to polar coordinates}\\
	\end{array}\]
	\[=\displaystyle\int_0^{\pi/2}\int_0^2 (3-r\cos\theta)r\;dr\;d\theta=\int_0^{\pi/2}(6-\frac{8}{3}\cos\theta)\;d\theta=3\pi-\frac{8}{3}\]
		}
	\item[\# 63.]{{\it Find the volume of the wedge cut from the first octant by the cylinder $z=12-3y^2$ and the plane $x+y=2$.}
		Similarly to previous problem, the solid can be described by inequalities
		\[x,y,z\geq0,\;z+3y^2\leq 12,\;x+y\leq2\]
		Hence, the volume can be computed as (again, $z\geq 0$ and $z+3y^2\leq12$ implies that $3y^2\leq z+3y^2\leq12\implies
		y^2\leq4\implies y\leq2$)
		\[V=\displaystyle\iint\limits_{\mycbra{y,z\geq0,\;z+3y^2\leq12}}\mybra{\int_0^{2-y}1\dx}\dz\dy=\int_{-2}^2\int_0^{12-3y^2}(2-y)\dz\dy
		=\int_{-2}^2(12-3y^2)(2-y)dy=64\]
		}
\end{description}
\end{document}
%42, 52, 54, 60, 63
