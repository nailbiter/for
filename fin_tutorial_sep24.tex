\documentclass[8pt]{article} % use larger type; default would be 10pt

%\usepackage[utf8]{inputenc} % set input encoding (not needed with XeLaTeX)
%\usepackage{CJK}
\usepackage{graphicx}
\usepackage{float}
\usepackage{subfig}
\usepackage{amsmath}
\usepackage{amsfonts}
\usepackage{hyperref}
\usepackage{enumerate}
\usepackage{enumitem}

\usepackage{mystyle}

\title{Math 1540\\University Mathematics for Financial Studies\\2013-14 Term 1\\Suggested problems for tutorial on\\September 24, 2013}
\begin{document}
\maketitle
The problems in this tutorial will deal solely with solving systems of linear equations using augmented matrices and Gaussian elimination.
For simplicity, we won't deal with more than 3 variables or equations.
\begin{enumerate}
	\newcommand{\myexplain}[3]{#1\xrightarrow{\text{#2}}#3}
	\newcommand{\myexplainf}[4]{#1\xrightarrow{\begin{subarray}{c}\text{#2}\\\text{#3}\end{subarray}}#4}
	\newcommand{\myfrac}[2]{^#1/_#2}
	\item{Let's start with simple
		\[\begin{cases}-2x_1+3x_2=8\\3x_1-x_2=-5\end{cases}\]
		Rewriting this in the form of augmented matrix we have
		\[\left(\begin{array}{cc|c} -2 & 3 & 8\\ 3&-1&-5\end{array}\right)\]
		First, we want to make a pivot element in a first row, for this we need to multiply first row with $-1/2$, or
		writing more formally
		\[\myexplain{\left(\begin{array}{cc|c} -2 & 3 & 8\\ 3&-1&-5\end{array}\right)}
			{\textcircled{1}$*\left(-\myfrac{1}{2}\right)$}
			{\left(\begin{array}{cc|c} 1 & -\myfrac{3}{2} & -4\\ 3&-1&-5\end{array}\right)}
		\]
		Next, we want to make all the elements under the pivot to be zero. For this we subtract the first row from the second
		three times. Writing formally again, we get
		\[\myexplain
			{\left(\begin{array}{cc|c} 1 & -\myfrac{3}{2} & -4\\ 3&-1&-5\end{array}\right)}
			{\textcircled{2}-3*\textcircled{3}}
			{\left(\begin{array}{cc|c} 1 & -\myfrac{3}{2} & -4\\ 0&\myfrac{7}{2}&7\end{array}\right)}
		\]
		The last operation is to make the pivot in the second row
		\[\myexplain
			{\left(\begin{array}{cc|c} 1 & -\myfrac{3}{2} & -4\\ 0&\myfrac{7}{2}&7\end{array}\right)}
				{\textcircled{2}$*\frac{2}{7}$}
			{\left(\begin{array}{cc|c} 1 & -\myfrac{3}{2} & -4\\ 0&1&2\end{array}\right)}
		\]
		Recovering from the matrix notation, we get
		\[\begin{cases}x_1-\frac{3}{2}x_2=-4\\x_2=2\end{cases}\]
		From this answer is readily
		\[x_1=-1,\;x_2=2\]
		}
	\item{Let's increase the number of variables and equations
		\[\begin{cases}2x_1+x_2+3x_3=1\\2x_1+6x_2+8x_3=3\\6x_1+8x_2+18x_3=5\end{cases}\]
		Rewriting this in the form of augmented matrix we have
		\[\left(\begin{array}{ccc|c}2&1&3&1\\2&6&8&3\\6&8&18&5\end{array}\right)\]
		Again, we begin with making a pivot in a first row
		\[\myexplain
			{\left(\begin{array}{ccc|c}2&1&3&1\\2&6&8&3\\6&8&18&5\end{array}\right)}
			{\textcircled{1}$*\frac{1}{2}$}
			{\left(\begin{array}{ccc|c}1&\myfrac{1}{2}&\myfrac{3}{2}&\myfrac{1}{2}\\2&6&8&3\\6&8&18&5\end{array}\right)}
		\]
		and making everything under it equal to zero
		\[\myexplainf
			{\left(\begin{array}{ccc|c}1&\myfrac{1}{2}&\myfrac{3}{2}&\myfrac{1}{2}\\2&6&8&3\\6&8&18&5\end{array}\right)}
				{\textcircled{2}-2*\textcircled{1}}{\textcircled{3}-6*\textcircled{1}}
			{\left(\begin{array}{ccc|c}1&\myfrac{1}{2}&\myfrac{3}{2}&\myfrac{1}{2}\\0&5&5&2\\0&5&9&2\end{array}\right)}
		\]
		Next we make a second pivot
		\[\myexplain
			{\left(\begin{array}{ccc|c}1&\myfrac{1}{2}&\myfrac{3}{2}&\myfrac{1}{2}\\0&5&5&2\\0&5&9&2\end{array}\right)}
				{\textcircled{2}$/5$}
		{\left(\begin{array}{ccc|c}1&\myfrac{1}{2}&\myfrac{3}{2}&\myfrac{1}{2}\\0&1&1&\myfrac{2}{5}\\0&5&9&2\end{array}\right)}
		\]
		and make everything under it vanish
		\[\myexplain
		{\left(\begin{array}{ccc|c}1&\myfrac{1}{2}&\myfrac{3}{2}&\myfrac{1}{2}\\0&1&1&\myfrac{2}{5}\\0&5&9&2\end{array}\right)}
			{\textcircled{3}-5*\textcircled{2}}
		{\left(\begin{array}{ccc|c}1&\myfrac{1}{2}&\myfrac{3}{2}&\myfrac{1}{2}\\0&1&1&\myfrac{2}{5}\\0&0&4&0\end{array}\right)}
		\]
		Recovering from matrix notation we get
		\[\begin{cases}x_1+\frac{x_2}{2}+\frac{3x_3}{2}=\frac{1}{2}\\x_2+x_3=\frac{2}{5}\\4x_3=0\end{cases}\]
		And the answer is readily
		\[x_1=\frac{3}{10},\;x_2=\frac{2}{5},\;x_3=0\]
		}
	\item{Let's try with underdetermined system next
		\[\begin{cases}3x_1+x_2-6x_3=-10\\2x_1+x_2-5x_3=-8\\6x_1-3x_2+3x_3=0\end{cases}\]
		Rewriting this in the form of augmented matrix we have
		\[
		\left(\begin{array}{ccc|c}3&1&-6&-10\\2&1&-5&-8\\6&-3&3&0\end{array}\right)
			\]
		Again, we begin with making a pivot in a first row
		\[\myexplain
			{\left(\begin{array}{ccc|c}3&1&-6&-10\\2&1&-5&-8\\6&-3&3&0\end{array}\right)}
			{\textcircled{1}$*\frac{1}{3}$}
			{\left(\begin{array}{ccc|c}1&\frac{1}{3}&-2&-\frac{10}{3}\\2&1&-5&-8\\6&-3&3&0\end{array}\right)}
		\]
		make everything under it vanish
		\[\myexplainf
			{\left(\begin{array}{ccc|c}1&\frac{1}{3}&-2&-\frac{10}{3}\\2&1&-5&-8\\6&-3&3&0\end{array}\right)}
			{\textcircled{2}-2*\textcircled{1}}{\textcircled{3}-6*\textcircled{1}}
			{\left(\begin{array}{ccc|c}1&\frac{1}{3}&-2&-\frac{10}{3}\\0&\myfrac{1}{3}&-1&-\myfrac{4}{3}
			\\0&-5&15&20\end{array}\right)}
		\]
		Next, we make pivot in the second row
		\[\myexplain
			{\left(\begin{array}{ccc|c}1&\frac{1}{3}&-2&-\frac{10}{3}\\0&\myfrac{1}{3}&-1&-\myfrac{4}{3}
			\\0&-5&15&20\end{array}\right)}
			{\textcircled{2}*3}
			{\left(\begin{array}{ccc|c}1&\frac{1}{3}&-2&-\frac{10}{3}\\0&1&-3&-4\\0&-5&15&20\end{array}\right)}
		\]
		and see that the third row vanish entirely
		\[\myexplain
			{\left(\begin{array}{ccc|c}1&\frac{1}{3}&-2&-\frac{10}{3}\\0&1&-3&-4\\0&-5&15&20\end{array}\right)}
			{\textcircled{2}*3}
			{\left(\begin{array}{ccc|c}1&\frac{1}{3}&-2&-\frac{10}{3}\\0&1&-3&-4\\0&0&0&0\end{array}\right)}
		\]
		we have arrived at underdetermined system, where we have more unknowns than {\it independent} equations.
		The third variable $x_3$ can thus be taken as {\it free variable} and can be thought as a parameter. Thus, we have
		\[x_2=-4+3x_3\]\[x_1=x_3-2\]
		}
\end{enumerate}
\end{document}
