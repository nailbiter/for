\documentclass{article}
\usepackage[english]{babel}
\usepackage{amsmath,amssymb,bbm}

%%%%%%%%%% Start TeXmacs macros
\newcommand{\assign}{:=}
\newcommand{\comma}{{,}}
\newcommand{\tmop}[1]{\ensuremath{\operatorname{#1}}}
\newcommand{\tmtextbf}[1]{{\bfseries{#1}}}
\newcommand{\tmtextit}[1]{{\itshape{#1}}}
\newcommand{\tmtextup}[1]{{\upshape{#1}}}
\newtheorem{definition}{Definition}
%%%%%%%%%% End TeXmacs macros

\begin{document}

\

\

\

\section{Symmetry breaking for full orthogonal groups $O (N)$}{}

\subsection{Irreps of $\tmop{SO} (n)$ and $O (n)$}{}

Recall that (see {\cite{knapp2013lie}})
\begin{eqnarray}
  & \begin{array}{lll}
    \tmop{HV} (N) \assign \left\{ \begin{array}{ll}
      \lambda_1 \geqslant \lambda_2 \geqslant \ldots \geqslant \lambda_{k - 1}
      \geqslant | \lambda_k | \geqslant 0, & N = 2 k\\
      \lambda_1 \geqslant \lambda_2 \geqslant \ldots \geqslant \lambda_k
      \geqslant 0, & N = 2 k + 1
    \end{array} \right. & \xrightarrow{\sim} & \widehat{\tmop{SO}} (N)\\
    \lambda = (\lambda_i) & \mapsto & [\lambda]
  \end{array} &  \nonumber
\end{eqnarray}
{\noindent}\tmtextbf{Fact \tmtextup{3}. }\tmtextit{{\cite[Sec.
5.5.5]{goodman2000representations}} (``irreps of $O (N)$'')
\begin{enumerate}
  \item We have
  \[ \hat{O} (2 k + 1) = \{ (\pi^{\lambda, \varepsilon}, V^{\lambda,
     \varepsilon}) \}_{\lambda \in \tmop{HV} (2 k + 1), \varepsilon = \pm 1}
  \]
  where $\pi^{\lambda, \varepsilon}$ is characterized by $\pi^{\lambda,
  \varepsilon} \mid_{\tmop{SO} (2 k + 1)} = [\lambda]$ and $\pi^{\lambda,
  \varepsilon} (- \tmop{id}_{2 k + 1}) = \varepsilon$.
  
  \item We have
  \[ \hat{O} (2 k) = \{ (\pi^{\lambda, \varepsilon}, V^{\lambda \comma
     \varepsilon}) \}_{\lambda \in \tmop{HV} (2 k) \mid \lambda_k = 0,
     \varepsilon = \pm 1} \sqcup \{ (\pi^{\lambda}, V^{\lambda}) \}_{\lambda
     \in \tmop{HV} (2 k) \mid \lambda_k > 0} \]
  where $\pi^{\lambda, \varepsilon}$ is characterized by $\pi^{\lambda,
  \varepsilon} \mid_{\tmop{SO} (2 k)} = [\lambda]$ and $\pi^{\lambda,
  \varepsilon} (g_0) = \varepsilon$, where
  \[ g_0 \assign \tmop{diag} \left( 1_{k - 1}, \left[ \begin{array}{ll}
       0 & 1\\
       1 & 0
     \end{array} \right], 1_{k - 1} \right), \]
  and $\pi^{\lambda} \assign \tmop{Ind}_{\tmop{SO} (2 k)}^{O (2 k)}
  [\lambda]$.
\end{enumerate}}{\hspace*{\fill}}{\medskip}

{\noindent}\tmtextbf{Fact \tmtextup{4}.
}\tmtextit{{\cite{fulton2013representation}}}{\hspace*{\fill}}{\medskip}

{\noindent}\tmtextbf{Fact \tmtextup{5}. }\tmtextit{{\cite[Sec 10.2.4,
10.2.5]{goodman2000representations}} (``explicit construction of irreps of $O
(N)$'')

TODO}{\hspace*{\fill}}{\medskip}

\subsection{Characters of irreps of $\tmop{SO} (N)$}{}

\begin{definition}
  Let $G = \tmop{SO} (N)$ and $\tau$ be irrep of $G$. We then let $D_{\tau}
  (g) \assign \tmop{tr} (\tau (g))$ and $D_{\tau}$ is then continuous function
  on $G$, which is invariant under $\tmop{Ad} (G)$.
  
  $S T_N$ defined as below is the Cartan subgroup of $G$. In particular, $G =
  \tmop{Ad} (G) S T_N$.
  
  Hence, $D_{\tau} (g)$ is determind by its values on
  \begin{eqnarray}
    & S T_N \assign \left\{ \begin{array}{ll}
      \left[ \begin{array}{ll}
        \cos \theta_1 & \sin \theta_1\\
        - \sin \theta_1 & \cos \theta_1
      \end{array} \right] \oplus \left[ \begin{array}{ll}
        \cos \theta_1 & \sin \theta_1\\
        - \sin \theta_1 & \cos \theta_1
      \end{array} \right] \oplus \ldots \oplus \left[ \begin{array}{ll}
        \cos \theta_k & \sin \theta_k\\
        - \sin \theta_k & \cos \theta_k
      \end{array} \right], & N = 2 k\\
      \left[ \begin{array}{ll}
        \cos \theta_1 & \sin \theta_1\\
        - \sin \theta_1 & \cos \theta_1
      \end{array} \right] \oplus \left[ \begin{array}{ll}
        \cos \theta_1 & \sin \theta_1\\
        - \sin \theta_1 & \cos \theta_1
      \end{array} \right] \oplus \ldots \oplus \left[ \begin{array}{ll}
        \cos \theta_k & \sin \theta_k\\
        - \sin \theta_k & \cos \theta_k
      \end{array} \right] \oplus 1, & N = 2 k + 1
    \end{array} \right. &  \nonumber
  \end{eqnarray}
  Schur orthogonality relations imply that $D_{\tau}$ characterizes $\tau$ up
  to equivalence class.
\end{definition}

\footnote{TODO: explain how characters are characterized (as functions of
\tmtextit{what})}

{\noindent}\tmtextbf{Fact \tmtextup{6}.
}\tmtextit{{\cite[{\textsection}VII.10]{boerner1963representations}} Let
\begin{eqnarray}
  & s (\alpha) = 2 i \sin (2 \pi \alpha) = e^{2 \pi i \alpha} - e^{- 2 \pi i
  \alpha}, &  \nonumber\\
  & c (\alpha) = 2 \cos (2 \pi \alpha) = e^{2 \pi i \alpha} + e^{- 2 \pi i
  \alpha} . &  \nonumber
\end{eqnarray}
Furthermore, for $\{ \tau_j \in \mathbbm{C}: | \tau_j | = 1 \}_{j = 1}^p, \{
l_j \in \mathbbm{Z} \}_{j = 1}^p$ we let
\begin{eqnarray}
  & S \left( \begin{array}{c}
    \tau_1, \ldots, \tau_p\\
    l_1, \ldots, l_p
  \end{array} \right) \assign \left| \begin{array}{cccc}
    s (l_1 \tau_1) & s (l_2 \tau_1) & \ldots & s (l_p \tau_1)\\
    s (l_1 \tau_2) & s (l_2 \tau_2) & \ldots & s (l_p \tau_2)\\
    \ldots & \ldots & \ddots & \ldots\\
    s (l_1 \tau_p) & s (l_2 \tau_p) & \ldots & s (l_p \tau_p)
  \end{array} \right| &  \nonumber\\
  & C \left( \begin{array}{c}
    \tau_1, \ldots, \tau_p\\
    l_1, \ldots, l_p
  \end{array} \right) \assign \left| \begin{array}{cccc}
    c (l_1 \tau_1) & c (l_2 \tau_1) & \ldots & c (l_p \tau_1)\\
    c (l_1 \tau_2) & c (l_2 \tau_2) & \ldots & c (l_p \tau_2)\\
    \ldots & \ldots & \ddots & \ldots\\
    c (l_1 \tau_p) & c (l_2 \tau_p) & \ldots & c (l_p \tau_p)
  \end{array} \right| &  \nonumber
\end{eqnarray}
\begin{description}
  \item[Case a] $O (n) = O (2 p)$. Let
  \[ r_1 = p - 1, r_2 = p - 2, \ldots, r_p = 0. \]
  Then the character of the irrep corresponding to $(a_i)_{i = 1}^p \in
  \tmop{HV} (2 p)$ equals to:
  \[ S \left( \begin{array}{c}
       \tau_1, \ldots, \tau_p\\
       a_1, \ldots, a_p
     \end{array} \right) / S \left( \begin{array}{c}
       \tau_1, \ldots, \tau_p\\
       r_1, \ldots, r_p
     \end{array} \right) \]
  \item[Case b] $O (n) = O (2 p + 1)$. Let
  \[ r_1 = p - \frac{1}{2}, r_2 = p - \frac{3}{2}, \ldots, r_p = \frac{1}{2} .
  \]
  Then the character of the irrep corresponding to $(a_i)_{i = 1}^p \in
  \tmop{HV} (2 p + 1)$ equals to:
  \[ \left\{ C \left( \begin{array}{c}
       \tau_1, \ldots, \tau_p\\
       a_1, \ldots, a_p
     \end{array} \right) + S \left( \begin{array}{c}
       \tau_1, \ldots, \tau_p\\
       a_1, \ldots, a_p
     \end{array} \right) \right\} / C \left( \begin{array}{c}
       \tau_1, \ldots, \tau_p\\
       r_1, \ldots, r_p
     \end{array} \right) \]
\end{description}}{\hspace*{\fill}}{\medskip}

\

\begin{thebibliography}{1}
  \bibitem[1]{boerner1963representations}Hermann Boerner, PG Murphy, J
  Mayer-Kalkschmidt , and  P Carr.{\newblock} Representations of groups: with
  special consideration for the needs of modern physics.{\newblock}
  1963.{\newblock}
  
  \bibitem[2]{goodman2000representations}Roe Goodman  and  Nolan~R
  Wallach.{\newblock} \tmtextit{Representations and invariants of the
  classical groups},  volume~68.{\newblock} Cambridge University Press,
  2000.{\newblock}
  
  \bibitem[3]{howe2005stable}Roger Howe, Eng-Chye Tan , and  Jeb
  Willenbring.{\newblock} Stable branching rules for classical symmetric
  pairs.{\newblock} \tmtextit{Transactions of the American mathematical
  society}, 357(4):1601--1626, 2005.{\newblock}
  
  \bibitem[4]{kobayashi2016classification}Toshiyuki Kobayashi, Toshihisa Kubo
  , and  Michael Pevzner.{\newblock} Classification of differential symmetry
  breaking operators for differential forms.{\newblock} \tmtextit{Comptes
  Rendus Mathematique}, 354(7):671--676, 2016.{\newblock}
  
  \bibitem[5]{knapp2013lie}Anthony~W Knapp.{\newblock} \tmtextit{Lie groups
  beyond an introduction},  volume  140.{\newblock} Springer Science \&
  Business Media, 2013.{\newblock}
\end{thebibliography}

\

\end{document}
