\documentclass{svjour3}
\usepackage[english]{babel}
\usepackage{amsmath,amssymb,graphicx,bbm,latexsym,theorem}

%%%%%%%%%% Start TeXmacs macros
\catcode`\<=\active \def<{
\fontencoding{T1}\selectfont\symbol{60}\fontencoding{\encodingdefault}}
\newcommand{\assign}{:=}
\newcommand{\comma}{{,}}
\newcommand{\tmdummy}{$\mbox{}$}
\newcommand{\tmop}[1]{\ensuremath{\operatorname{#1}}}
\newcommand{\tmscript}[1]{\text{\scriptsize{$#1$}}}
\newcommand{\tmtextbf}[1]{{\bfseries{#1}}}
\newcommand{\tmtextit}[1]{{\itshape{#1}}}
\newenvironment{proof*}[1]{\noindent\textbf{#1\ }}{\hspace*{\fill}$\Box$\medskip}
\newtheorem{corollary}{Corollary}
{\theorembodyfont{\rmfamily}\newtheorem{example}{Example}}
\newtheorem{lemma}{Lemma}
\newtheorem{proposition}{Proposition}
{\theorembodyfont{\rmfamily}\newtheorem{remark}{Remark}}
\newtheorem{theorem}{Theorem}
%%%%%%%%%% End TeXmacs macros

%

\newcommand{\sectionsep}{{\sectionalsep}}

\begin{document}

April 27, 2017

\section{Main results}

\begin{theorem}
  For $\lambda, \mu, \nu \in \mathbbm{C}$ with $\tmop{Re} \lambda, \tmop{Re}
  \mu, \tmop{Re} \nu > - 1 / 2$\footnote{I think we need to make the
  assumption $\tmop{Re} \nu > 0$. Note that the conditions for convergence of
  Selberg integral $\int_{t \in [0, 1]^n} \Pi^{\alpha - 1, \beta - 1} (t)
  \Delta^{2 \gamma} (t) d t$ are $\tmop{Re} \alpha, \tmop{Re} \beta > 0$ and
  $\tmop{Re} \gamma > - \min \left\{ \frac{1}{n}, \frac{\tmop{Re} \alpha}{n -
  1}, \frac{\tmop{Re} \beta}{n - 1} \right\}$, as noted in
  {\cite{forrester2008importance}}.}, \ the following expansion holds:
  \begin{eqnarray}
    & | s + t |^{2 \nu} = \sum_{\ell, m = 0 \mid l + m :
    \tmop{even}}^{\infty} a_{\lambda, \mu, \nu}^{\ell, m} C_{\ell}^{\lambda}
    (s) C_m^{\mu} (t), &  \nonumber\\
    & a_{\lambda, \mu, \nu}^{\ell, m} = \frac{2^{- 2 \nu} (\lambda + \ell)
    (\mu + m) \Gamma (\lambda + \mu + 2 \nu + 1) \Gamma (\lambda) \Gamma (\mu)
    \Gamma (2 \nu + 1)}{\Gamma \left( \lambda + \nu + \frac{\ell - m}{2} + 1
    \right) \Gamma \left( \mu + \nu - \frac{\ell - m}{2} + 1 \right) \Gamma
    \left( \lambda + \mu + \nu + \frac{\ell + m}{2} + 1 \right) \Gamma \left(
    \nu + 1 - \frac{\ell + m}{2} \right)} &  \nonumber
  \end{eqnarray}
\end{theorem}

Let $C_l^{\lambda} (s)$ be the Gegenbauer polynomial of degree $l$, and we set
\[ u_l^{\lambda} (s) \assign \frac{2^{2 \lambda - 1} l! \Gamma
   (\lambda)}{\Gamma (2 \lambda + l)} (1 - s^2)^{\lambda - \frac{1}{2}}
   C_l^{\lambda} (s) . \]
\begin{theorem}
  \label{main-thm}Suppose $l, m \in \mathbbm{N}$ such that $l + m$ is even.
  For $\lambda, \mu, \nu \in \mathbbm{C}$ with $\tmop{Re} \lambda, \tmop{Re}
  \mu, \tmop{Re} \nu > - 1 / 2$\footnote{I think we need to make the
  assumption $\tmop{Re} \nu > 0$. Note that the conditions for convergence of
  Selberg integral $\int_{t \in [0, 1]^n} \Pi^{\alpha - 1, \beta - 1} (t)
  \Delta^{2 \gamma} (t) d t$ are $\tmop{Re} \alpha, \tmop{Re} \beta > 0$ and
  $\tmop{Re} \gamma > - \min \left\{ \frac{1}{n}, \frac{\tmop{Re} \alpha}{n -
  1}, \frac{\tmop{Re} \beta}{n - 1} \right\}$, as noted in
  {\cite{forrester2008importance}}.}, and for $0 \leqslant z \leqslant
  1$\footnote{Check the proof when $z = 1$}, \ the following integral
  converges:
  \begin{eqnarray}
    & \int_{- 1}^1 \int_{- 1}^1 | s - t z |^{2 \nu} u_l^{\lambda} (s)
    u_m^{\mu} (t) d s d t &  \nonumber\\
    & = \frac{(- \nu)_{\frac{l + m}{2}} (- 1)^{\frac{l - m}{2}}
    \pi^{\frac{3}{2}} \Gamma \left( \nu + \frac{1}{2} \right) z^m _2 F_1
    \left( \begin{array}{c}
      \frac{l + m}{2} - \nu, \frac{m - l}{2} - \nu - \lambda\\
      \mu + m + 1
    \end{array} ; z^2 \right)}{\Gamma (\mu + m + 1) \Gamma \left( \lambda +
    \nu + \frac{l - m}{2} + 1 \right)},  \label{eqn:main} & 
  \end{eqnarray}
  where $(y)_j \assign y (y + 1) (y + 2) \cdots (y + j - 1)$ for $j \in
  \mathbbm{N}$.
\end{theorem}

Note that as the Gegenbauer polynomial $g (s) \assign C_l^{\lambda} (s)$
satisfies the second-order differential equation
\begin{eqnarray}
  & (1 - s^2) g'' - (2 \lambda + 1) s g' + n (n + 2 \lambda) g = 0, & 
  \nonumber
\end{eqnarray}
$f (s) \assign u_l^{\lambda} (s)$ satisfies
\begin{eqnarray}
  & (1 - s^2)^2 f'' + (1 - s^2) (1 - 2 \lambda - (2 \lambda + 1) s) f' + ((2
  s + 1) (\lambda^2 - 1 / 4) + l (l + 2 \lambda) (1 - s^2)) f = 0. & 
  \nonumber
\end{eqnarray}
\begin{remark}
  Since $u_l^{\lambda} (- s) = (- 1)^l u_l^{\lambda} (s)$, Theorem
  \ref{main-thm} can be extended easily to the case $- 1 \leqslant z \leqslant
  0$.
\end{remark}

The substitution of $z = 1$ in Theorem \ref{main-thm} yields:

\begin{corollary}
  \label{cor:1}{\tmdummy}
  
  \begin{eqnarray}
    & \int_{- 1}^1 \int_{- 1}^1 | s - t |^{2 \nu} (1 - s^2)^{\lambda -
    \frac{1}{2}} (1 - t^2)^{\mu - \frac{1}{2}} C_l^{\lambda} (s) C_m^{\mu} (t)
    d s d t &  \nonumber\\
    & = \frac{(- \nu)_{\frac{l + m}{2}} (- 1)^{\frac{l - m}{2}}
    \pi^{\frac{1}{2}} (2 \lambda)_l (2 \mu)_m \Gamma \left( \lambda +
    \frac{1}{2} \right) \Gamma \left( \mu + \frac{1}{2} \right) \Gamma \left(
    \nu + \frac{1}{2} \right) \Gamma (\lambda + \mu + 2 \nu + 1)}{l!m! \Gamma
    \left( \lambda + \nu + \frac{l - m}{2} + 1 \right) \Gamma \left( \mu + \nu
    - \frac{l - m}{2} + 1 \right) \Gamma \left( \lambda + \mu + \nu + \frac{l
    + m}{2} + 1 \right)}  \label{eqn:cor:1} . & 
  \end{eqnarray}
\end{corollary}

Taking the limit in $(\ref{eqn:cor:1})$ as both $\lambda$ and
$\nu$\footnote{maybe, ``and $\mu$''?} tends\footnote{maybe, ``tend''?} to be
zero, we obtain

\begin{corollary}
  \label{cor:170599}For $\rho \in \mathbbm{C}$ with $\tmop{Re} \rho > 0$ and
  $r \in \{ 0, 1 \}$\footnote{Note that using the relation of Gegenbauer
  polynomials},
  \begin{eqnarray}
    & | \cos \varphi + \cos \psi |^{\rho} \tmop{sgn}^r (\cos \varphi + \cos
    \psi) &  \nonumber\\
    & = \sum_{\tmscript{\begin{array}{c}
      l, m = 0\\
      l \equiv m + r \tmop{mod} 2
    \end{array}}}^{\infty} \frac{H (l) H (m) 2^{2 - \rho} \Gamma (\rho +
    1)^2}{\prod_{\delta, \varepsilon \in \{ \pm 1 \}} \Gamma \left( 1 +
    \frac{1}{2} (\rho + \delta l + \varepsilon m) \right)} \cos l \varphi \cos
    m \psi . &  \nonumber
  \end{eqnarray}
  where Heaviside step function $H (x)$ is defined as:
  \[ H (x) \assign \left\{ \begin{array}{ll}
       0, & x < 0,\\
       1 / 2, & x = 0,\\
       1, & x > 0.
     \end{array} \right. \]
\end{corollary}

Selberg-type integrals are related to (finite-dimensional) representation
theory of semisimple Lie algebras, see {\cite{forrester2008importance}},
{\cite{tarasov2003selberg}} and references therein. On the other hand, Theorem
\ref{main-thm} and Corollary \ref{cor:1} will be used in the study of symmetry
breaking operators for infinite-dimensional representations when we extend the
work {\cite{kobayashi2015symmetry}} to indefinite orthogonal groups $O (p,
q)$. This will be done in a separate paper.

Special cases and the limit case of Theorem \ref{main-thm} will be discussed
in Section \ref{sec:4}.

\section{Proof of main theorem}\label{sec:2}

In this section we prove Theorem \ref{main-thm} by assuming the following
integral formula $(\ref{eqn:stz})$, which will be proved in Section
\ref{sec:3}.

\begin{proposition}
  \label{prop:2}For $a, b, c \in \mathbbm{C}$ such that $\tmop{Re} a,
  \tmop{Re} b, \tmop{Re} c > 0$ and $0 \leqslant z \leqslant 1$ we have
  \begin{eqnarray}
    & \int_{- 1}^1 \int_{- 1}^1 | s - t z |^{2 c - 1} (1 - s^2)^{a - 1} (1 -
    t^2)^{b - 1} d s d t = \frac{\sqrt{\pi} \Gamma (a) \Gamma (b) \Gamma
    (c)}{\Gamma (a + c) \Gamma \left( b + \frac{1}{2} \right)} _2 F_1 \left(
    \begin{array}{c}
      - c + \frac{1}{2}, - a - c + 1\\
      b + \frac{1}{2}
    \end{array} ; z^2 \right) .  \label{eqn:stz} & \\
    &  &  \nonumber
  \end{eqnarray}
\end{proposition}

\begin{proof*}{Proof of Theorem \ref{main-thm}}
  By the Rodrigues formula for the Gegenbauer polynomial:
  \[ u_l^{\lambda} (t) = \frac{(- 1)^l 2^{- l} \sqrt{\pi}}{\Gamma \left(
     \lambda + l + \frac{1}{2} \right)} \cdot \frac{d^l}{d t^l} (1 -
     t^2)^{\lambda + l - \frac{1}{2}}, \eq-number \label{eqn:Rod} \]
  the left-hand side of $(\ref{eqn:main})$ amounts to
  \begin{eqnarray}
    & \frac{2^{- l - m} \pi}{\Gamma \left( \lambda + l + \frac{1}{2} \right)
    \Gamma \left( \mu + m + \frac{1}{2} \right)} \int_{- 1}^1 \int_{- 1}^1 | s
    - t z |^{2 \nu} \frac{\partial^l}{\partial s^l} (1 - s^2)^{\lambda + l -
    \frac{1}{2}} \frac{\partial^m}{\partial t^m} (1 - t^2)^{\mu + m -
    \frac{1}{2}} d s d t. &  \nonumber
  \end{eqnarray}
  Suppose that $\tmop{Re} \nu > \frac{l + m}{2}$, $\tmop{Re} \lambda >
  \frac{1}{2}$ and $\tmop{Re} \mu > \frac{1}{2}$. Then $| s - t z |^{2 \nu}$
  is of $C^{l + m}$ class and the above integral is equal to
  \begin{eqnarray}
    & \int_{- 1}^1 \int_{- 1}^1 (1 - s^2)^{\lambda + l - \frac{1}{2}} (1 -
    t^2)^{\mu + m - \frac{1}{2}} \frac{\partial^{l + m}}{\partial s^l \partial
    t^m} | s - t z |^{2 \nu} d s d t  \label{eqn:derst} & 
  \end{eqnarray}
  by integration by parts. Since $l + m \in 2\mathbbm{N}$ we have
  \[ \frac{\partial^{l + m}}{\partial s^l \partial t^m} | s - t z |^{2 \nu} =
     (- 2 \nu)_{l + m} (- z)^m | s - t z |^{2 \nu - l - m}, \]
  hence $(\ref{eqn:derst})$ equals
  \begin{eqnarray}
    & (- 2 \nu)_{l + m} (- z)^m \int_{- 1}^1 \int_{- 1}^1 | s - t z |^{2 \nu
    - l - m} (1 - s^2)^{\lambda + l - \frac{1}{2}} (1 - t^2)^{\mu + m -
    \frac{1}{2}} d s d t. &  \nonumber
  \end{eqnarray}
  Then Theorem \ref{main-thm} follows from Proposition \ref{prop:2}.
\end{proof*}

\section{Proof of Proposition \ref{prop:2}}\label{sec:3}

In this section we show Proposition \ref{prop:2}, and thus complete the proof
of Theorem \ref{main-thm}. We use the following two lemmas.

\begin{lemma}
  \label{lem4}For $a, b, z \in \mathbbm{C}$ with $\tmop{Re} a, \tmop{Re} b >
  0$ and $| z | < 1$ we have
  \begin{eqnarray}
    & \int_{- 1}^1 (1 - t z)^{a - 1} (1 - t^2)^{b - 1} d t = B \left(
    \frac{1}{2}, b \right) _2 F_1 \left( \begin{array}{c}
      \frac{1 - a}{2}, \frac{2 - a}{2}\\
      b + \frac{1}{2}
    \end{array} ; z^2 \right) . &  \nonumber
  \end{eqnarray}
\end{lemma}

\begin{lemma}
  \label{lem:Fisum}Suppose $| \zeta | < 1$.
  \begin{eqnarray}
    & \sum_{i = 0}^{\infty} \frac{(a)_i (1 - a)_i}{2^i i! (d)_i} _2 F_1
    \left( \begin{array}{c}
      \frac{1 - d - i}{2}, \frac{2 - d - i}{2}\\
      b + \frac{1}{2}
    \end{array} ; \zeta \right) = \frac{2^{1 - d} \sqrt{\pi} \Gamma
    (d)}{\Gamma \left( \frac{a + d}{2} \right) \Gamma \left( \frac{1 - a +
    d}{2} \right)} _2 F_1 \left( \begin{array}{c}
      1 - \frac{a + d}{2}, \frac{1 + a - d}{2}\\
      b + \frac{1}{2}
    \end{array} ; \zeta \right) .  \label{eqn:iF} & 
  \end{eqnarray}
\end{lemma}

Postponing the verification of Lemmas \ref{lem4} and \ref{lem:Fisum}, we first
show Proposition \ref{prop:2}.

\begin{proof*}{Proof of Proposition \ref{prop:2}}
  By the change of variables $s = (1 - z) (1 - t) + z$, we have
  \begin{eqnarray}
    & \int_{- 1}^1 (s - z)_+^{2 c - 1} (1 - s^2)^{a - 1} d s = 2^{a - 1} B (2
    c, a) (1 - z)^{2 c + a} _2 F_1 \left( \begin{array}{c}
      1 - a, 2 c\\
      2 c + a
    \end{array} ; \frac{1 - z}{2} \right) &  \nonumber
  \end{eqnarray}
  from Euler's integral representation of the hypergeometric function $_2
  F_1$, because $- 1 \leqslant s \leqslant 1$ and $s - z \geqslant 0$ if and
  only if $0 \leqslant t \leqslant 1$. Therefore the left-hand side of
  $(\ref{eqn:stz})$ equals:
  \begin{eqnarray}
    & 2 \int_{- 1}^1 \int_{- 1}^1 (s - t z)_+^{2 c - 1} (1 - s^2)^{a - 1} (1
    - t^2)^{b - 1} d s d t &  \nonumber\\
    & = 2^a B (2 c, a) \int_{- 1}^1 (1 - t z)^{2 c + a - 1} _2 F_1 \left(
    \begin{array}{c}
      1 - a, a\\
      2 c + a
    \end{array} ; \frac{1 - t z}{2} \right) (1 - t^2)^{b - 1} d t. & 
    \nonumber
  \end{eqnarray}
  Fix $\varepsilon > 0$. Assume $| z | \leqslant 1 - 2 \varepsilon$. Then
  $\left| \frac{1 - t z}{2} \right| \leqslant 1 - \varepsilon$. Expanding the
  hypergeometric function as a uniformly convergent power series, we can
  rewrite the integral in the right-hand side as
  \begin{eqnarray}
    & \sum_{i = 0}^{\infty} \frac{(1 - a)_i (a)_i}{2^i i! (2 c + a)_i}
    \int_{- 1}^1 (1 - t z)^{2 c + a - 1 + i} (1 - t^2)^{b - 1} d t. & 
    \nonumber
  \end{eqnarray}
  Owing to Lemma \ref{lem4}, this is equal to
  \begin{eqnarray}
    & B \left( \frac{1}{2}, b \right) \sum_{i = 0}^{\infty} \frac{(1 - a)_i
    (a)_i}{2^i i! (2 c + a)_i} _2 F_1 \left( \begin{array}{c}
      \frac{1 - 2 c - a - i}{2}, \frac{2 - 2 c - a - i}{2}\\
      b + \frac{1}{2}
    \end{array} ; z^2 \right) . &  \nonumber
  \end{eqnarray}
  Now (\ref{eqn:stz}) follows from Lemma \ref{lem:Fisum} with $\zeta = z^2$.
\end{proof*}

\begin{proof*}{Proof of Lemma \ref{lem4}}
  By Euler's integral representation of $_2 F_1$ again, we have
  \begin{eqnarray}
    & \int_{- 1}^1 (1 - t z)^{a - 1} (1 - t^2)^{b - 1} d t = 2^{2 b - 1} (1 +
    z)^{a - 1} B (b, b)_2 F_1 \left( \begin{array}{c}
      1 - a, b\\
      2 b
    \end{array} ; \frac{2 z}{1 + z} \right) .  \label{eqn:quad} & 
  \end{eqnarray}
  Applying the following quadratic transformation of $_2 F_1$ (cf. {\cite[Thm.
  3.13]{andrews1999special}}):
  \begin{eqnarray}
    & \;_2 F_1 \left( \begin{array}{c}
      1 - a, b\\
      2 b
    \end{array} ; u \right) = \left( 1 - \frac{z}{2} \right)^{a - 1} _2 F_1
    \left( \begin{array}{c}
      \frac{1 - a}{2}, \frac{2 - a}{2}\\
      b + \frac{1}{2}
    \end{array} ; \left( \frac{u}{2 - u} \right)^2 \right), &  \nonumber
  \end{eqnarray}
  with $u = \frac{2 z}{1 + z}$, we get the desired result after a small
  computation using the duplication formula of the $\Gamma$-function.
\end{proof*}

\begin{proof*}{Proof of Lemma \ref{lem:Fisum}}
  The Pochammer symbol $(y)_j = \frac{\Gamma (y + j)}{\Gamma (y)}$ satisfies
  \[ \begin{array}{lll}
       y_j (1 - y)_{- j} & = \; (- 1)^j, & \eq-number \label{eqn:p1}\\
       \left( \frac{y}{2} \right)_j \left( \frac{1 + y}{2} \right)_j & = \;
       2^{- 2 j} (y)_{2 j}, & \eq-number \label{eqn:p2}\\
       (y)_i (1 - y)_{2 j} & = \; (1 - y - i)_{2 j} (y - 2 j)_i . & \eq-number
       \label{eqn:p3}
     \end{array} \]
  We claim that the left-hand side of $(\ref{eqn:iF})$ equals
  \begin{eqnarray}
    & \sum_{j = 0}^{\infty} \frac{(1 - d)_{2 j} \zeta^j}{2^{2 j} j! \left( b
    + \frac{1}{2} \right)_j} _2 F_1 \left( \begin{array}{c}
      a, 1 - a\\
      d - 2 j
    \end{array} ; \frac{1}{2} \right) .  \label{eqn:Fijsum} & 
  \end{eqnarray}
  Indeed, by expanding the hypergeometric function as a power series and by
  using $(\ref{eqn:p2})$ with $y = d$, the left-hand side of
  $(\ref{eqn:Fijsum})$ amounts to
  \begin{eqnarray}
    & \sum_{i = 0}^{\infty} \sum_{j = 0}^{\infty} \frac{(a)_i (1 - a)_i}{2^{i
    + 2 j} i!j! (d)_i}  \frac{(1 - d - i)_{2 j}}{\left( b + \frac{1}{2}
    \right)_j} \zeta^j, &  \nonumber
  \end{eqnarray}
  which is equal to the right-hand side of $(\ref{eqn:Fijsum})$ by
  $(\ref{eqn:p3})$. Hence we have verified the claim $(\ref{eqn:iF}) =
  (\ref{eqn:Fijsum})$.
  
  By $\;_2 F_1 \left( \begin{array}{c}
    a, 1 - a\\
    b
  \end{array} ; \frac{1}{2} \right) = \frac{2^{1 - b} \sqrt{\pi} \Gamma
  (b)}{\Gamma \left( \frac{a + b}{2} \right) \Gamma \left( \frac{b - a + 1}{2}
  \right)}$ (see {\cite[Thm. 5.4]{andrews1999special}} for instance),
  \begin{eqnarray}
    & \begin{array}{ll}
      (\ref{eqn:Fijsum}) & = \sum_{j = 0}^{\infty} \frac{(1 - d)_{2 j}
      \zeta^j}{2^{2 j} j! \left( b + \frac{1}{2} \right)_j} \cdot \frac{2^{1 -
      d + 2 j} \sqrt{\pi} \Gamma (d - 2 j)}{\Gamma \left( \frac{a + d}{2} - j
      \right) \Gamma \left( \frac{1 - a + d}{2} - j \right)}\\
      & = \frac{2^{1 - d} \sqrt{\pi} \Gamma (d)}{\Gamma \left( \frac{a +
      d}{2} \right) \Gamma \left( \frac{1 - a + d}{2} \right)} \sum_{j =
      0}^{\infty} \frac{\left( 1 - \frac{a + d}{2} \right)_j \left( \frac{1 +
      a - d}{2} \right)_j}{j! \left( b + \frac{1}{2} \right)_j} \zeta^j
    \end{array} &  \nonumber
  \end{eqnarray}
  where we have used $(\ref{eqn:p1})$ in the second equality. Hence Lemma
  \ref{lem:Fisum} is proved.
\end{proof*}

\section{Limit case and special values}\label{sec:4}

In this section we examine our main result (Theorem \ref{main-thm}) by taking
the ``limit'' or by evaluating at special values of parameters. We also
compare them with special values of the existing integral formul{\ae} such as
the Selberg-type integrals.

Since the Hermite polynomial $H_n (x)$ is obtained as a limit of the
Gegenbauer polynomial:
\begin{eqnarray}
  & H_n (x) = n! \lim_{\lambda \rightarrow \infty} \lambda^{- \frac{n}{2}}
  C_n^{\lambda} \left( \frac{x}{\sqrt[]{\lambda}} \right), &  \nonumber
\end{eqnarray}
we can deduce the following integral formula of the Hermite polynomial from
Corollaly \ref{cor:1}:

\begin{corollary}
  \label{cor:Hermite}Suppose $l, m \in \mathbbm{N}$ with $l + m$ even.
  \begin{eqnarray}
    & \int_{- \infty}^{\infty} \int_{- \infty}^{\infty} | x - z y |^{2 \nu}
    e^{- x^2 - y^2} H_l (x) H_m (y) d x d y = (- \nu)_{\frac{l + m}{2}} (-
    1)^{\frac{l - m}{2}} 2^{l + m} \pi^{\frac{1}{2}} \Gamma \left( \frac{1}{2}
    + \nu \right) (z^2 + 1)^{\nu - \frac{l + m}{2}} z^m . &  \nonumber
  \end{eqnarray}
\end{corollary}

\begin{example}
  (Mehta integral {\cite{mehta2004random}}) The Mehta integral
  \begin{eqnarray}
    & \frac{1}{(2 \pi)^{n / 2}} \int_{- \infty}^{\infty} \int_{-
    \infty}^{\infty} \ldots \int_{- \infty}^{\infty} \prod_{i = 1}^n e^{-
    t_i^2 / 2}_{} \prod_{1 \leqslant i < j \leqslant n} | t_i - t_j |^{2 \nu}
    d t_1 \cdots d t_n &  \nonumber\\
    & = \prod_{j = 1}^n \frac{\Gamma (1 + j \nu)}{\Gamma (1 + \nu)} & 
    \nonumber
  \end{eqnarray}
  in special case $n = 2$ implies the following equation
  \begin{eqnarray}
    & \frac{1}{2 \pi} \int_{- \infty}^{\infty} \int_{- \infty}^{\infty} e^{-
    \frac{x^2 + y^2}{2}} | x - y |^{2 \nu} d x d y = \frac{\Gamma (1 + 2
    \nu)}{\Gamma (1 + \nu)} . &  \nonumber\\
    &  &  \nonumber
  \end{eqnarray}
  This coincides with the special case of Corollary \ref{cor:Hermite} with
  $(l, m, z) = (0, 0, 1)$.
\end{example}

In what follows, we shall examine the relationship between Theorem
\ref{main-thm} and some known integral formul{\ae} by Selberg, Dotsenko,
Fateev, Tarasov Varchenko, Warnaar among others when the parameters take
special values. The hierarchy of the formul{\ae} treated here is summarized in
Table \ref{table}.

For this, we limit ourselves to the special case of Theorem \ref{main-thm}
with $(l, m, z) = (0, 0, 1)$, or equivalently, of Corollary \ref{cor:1} with
$(l, m) = (0, 0)$:
\begin{eqnarray}
  & \int_{- 1}^1 \int_{- 1}^1 | s - t |^{2 \nu} (1 - s^2)^{\lambda -
  \frac{1}{2}} (1 - t^2)^{\mu - \frac{1}{2}} d s d t = \frac{\pi^{\frac{1}{2}}
  \Gamma \left( \lambda + \frac{1}{2} \right) \Gamma \left( \mu + \frac{1}{2}
  \right) \Gamma \left( \nu + \frac{1}{2} \right) \Gamma (\lambda + \mu + 2
  \nu + 1)}{\Gamma (\lambda + \nu + 1) \Gamma (\mu + \nu + 1) \Gamma (\lambda
  + \mu + \nu + 1)} .  \label{eqn:lm0} & 
\end{eqnarray}


\begin{example}
  \label{ex:1}(Selberg integral {\cite{Selberg:411367}}) The Selberg integral
  \begin{eqnarray}
    & \int_0^1 \ldots \int_0^1 \prod_{i = 1}^n t_i^{\alpha - 1} (1 -
    t_i)^{\beta - 1} \left| \prod_{1 \leqslant i < j \leqslant n} (t_i - t_j)
    \right|^{2 \nu} d t_1 \cdots d t_n  \label{eqn:selberg} & \\
    & = \prod_{j = 1}^n \frac{\Gamma (\alpha + (j - 1) \nu) \Gamma (\beta +
    (j - 1) \nu) \Gamma (1 + j \nu)}{\Gamma (\alpha + \beta + (n + j - 2) \nu)
    \Gamma (1 + \nu)} &  \nonumber
  \end{eqnarray}
  is a generalization of the Euler beta integral. A special case of
  $(\ref{eqn:selberg})$ with $(n, \alpha, \beta) = \left( 2, \lambda +
  \frac{1}{2}, \lambda + \frac{1}{2} \right)$ says
  \begin{eqnarray}
    & \frac{1}{2^{4 \lambda + 2 \nu}} \int_{- 1}^1 \int_{- 1}^1 (1 -
    s^2)^{\lambda - \frac{1}{2}} (1 - t^2)^{\lambda - \frac{1}{2}} | s - t
    |^{2 \nu} d s d t = &  \nonumber\\
    & = \frac{\Gamma \left( \lambda + \frac{1}{2} \right)^2}{\Gamma (2
    \lambda + 1 + \nu)} \cdot \frac{\Gamma \left( \lambda + \nu + \frac{1}{2}
    \right)^2 \Gamma (1 + 2 \nu)}{\Gamma (2 \lambda + 2 \nu + 1) \Gamma (1 +
    \nu)}, &  \nonumber
  \end{eqnarray}
  after a change of variables $(t_1, t_2) = \left( \frac{1 + s}{2}, \frac{1 +
  t}{2} \right)$. This coincides with the special case of Theorem
  \ref{main-thm} with $l = m = 0$, $z = 1$ and $\lambda = \mu$. 
\end{example}

\begin{example}
  \label{ex:2}(Warnaar integral) The integral formula (1.4) in
  {\cite{warnaar2010sl3}} in the special case $(k_1, k_2, \alpha_1, \beta_1,
  \alpha_2 \comma \beta_2, \gamma) = \left( 1, 1, \lambda + \frac{1}{2},
  \lambda + \frac{1}{2}, \mu + \frac{1}{2}, \mu + \frac{1}{2} \comma \lambda +
  \mu \right)$ implies the following equation
  \begin{eqnarray}
    & \frac{1}{2^{\lambda + \mu}} \left( \int \int_{0 \leqslant s < t
    \leqslant 1} + \frac{\cos (\pi \lambda)}{\cos (\pi \mu)} \int \int_{0
    \leqslant t < s \leqslant 1} \right) (1 - s^2)^{\lambda - \frac{1}{2}} (1
    - t^2)^{\mu - \frac{1}{2}} | s - t |^{- \lambda - \mu} d s d t & 
    \nonumber\\
    & = \frac{\Gamma \left( \lambda + \frac{1}{2} \right) \Gamma \left(
    \frac{1}{2} - \mu \right) \Gamma \left( \mu + \frac{1}{2}
    \right)^2}{\Gamma (\lambda + 1 - \mu) \Gamma (\mu + 1 - \lambda) \Gamma
    (\lambda + \mu + 1)} . &  \nonumber
  \end{eqnarray}
  This coincides with the special case of Theorem \ref{main-thm} with $(l, m,
  z, \nu) = \left( 0, 0, 1, - \frac{\lambda + \mu}{2} \right)$.
\end{example}

\begin{example}
  \label{ex:3}($\mathfrak{s}\mathfrak{l}_3$ Selberg integral of Tarasov and
  Varchenko) The integral formula (3.4) in {\cite{tarasov2003selberg}} in the
  special case $(k_1, k_2, \alpha, \beta_1, \beta_2, \gamma) = \left( 1, 1,
  \lambda + \frac{1}{2}, \lambda + \frac{1}{2}, 1, - 2 \nu \right)$ reduces to
  the following equation
  \begin{eqnarray}
    & \frac{1}{2^{2 \lambda + 2 \nu + 1}} \int_{- 1}^1 \int_{- 1}^1 (1 -
    s^2)^{\lambda - \frac{1}{2}} (t - s)_+^{2 \nu} d s d t = \frac{\Gamma
    \left( \lambda + \frac{1}{2} \right) \Gamma \left( \frac{3}{2} + \lambda +
    2 \nu \right)}{(1 + 2 \nu) \Gamma (2 + 2 \lambda + 2 \nu)} . &  \nonumber
  \end{eqnarray}
  This coincides with the special case of Theorem \ref{main-thm} with $(l, m,
  z, \mu) = \left( 0, 0, 1, \frac{1}{2} \right)$.
\end{example}

\begin{example}
  \label{ex:4}(Dotsenko-Fateev integral) The integral formula (A1)$=$(A35) in
  {\cite{dotsenko1985four}} in the special case $(n, m, \alpha, \beta, \rho) =
  \left( 1, 1, \mu - \frac{1}{2}, \mu - \frac{1}{2}, - \frac{\mu -
  \frac{1}{2}}{\lambda - \frac{1}{2}} \right)$ reduces to the following
  equation
  \begin{eqnarray}
    & 2^{2 - 2 \lambda - 2 \mu} \int_{- 1}^1 \int_{- 1}^1 (1 - s^2)^{\lambda
    - \frac{1}{2}} (1 - t^2)^{\mu - \frac{1}{2}} | s - t |^{- 2} d s d t & 
    \nonumber\\
    & = \frac{- 2 \Gamma \left( \lambda + \frac{1}{2} \right)^2 \Gamma \left(
    \mu + \frac{1}{2} \right)^2}{(\lambda + \mu - 1) \Gamma (2 \lambda) \Gamma
    (2 \mu)} . &  \nonumber
  \end{eqnarray}
  This coincides with the special case of Theorem \ref{main-thm} with $(l, m,
  z, \nu) = (0, 0, 1, - 1)$.
\end{example}

The hierarchy of the integral formul{\ae} in Examples \ref{ex:1}-\ref{ex:4}
and Theorem \ref{main-thm} is summarized as follows:\footnote{Should I include
the following in the diagram: Corollary \ref{cor:Hermite} (and its relation
with the Mehta integral); relation with the results of
{\cite{kobayashi2011schrodinger}}; relations with the expansion of $x^n$ into
Gegenbauer polynomials in {\cite{rainville1960special}}?}

\

\begin{table}[h]
  \begin{tabular}{l}
    \resizebox{788px}{264px}{\includegraphics{intdep.png}}
  \end{tabular}
  \caption{Hierarchy of various integral formul{\ae}\label{table}}
\end{table}

\begin{thebibliography}{10}
  \bibitem[1]{andrews1999special}George~E Andrews, Richard Askey , and  Ranjan
  Roy.{\newblock} \tmtextit{Special Functions},  volume~\tmtextbf{71} of
  \tmtextit{Encyclopedia of Mathematics and its Applications}.{\newblock}
  Cambridge University Press, Cambridge, 1999.{\newblock}
  
  \bibitem[2]{dotsenko1985four}Vl~S Dotsenko  and  Vladimir~A
  Fateev.{\newblock} Four-point correlation functions and the operator algebra
  in 2d conformal invariant theories with central charge $c \leq
  1$.{\newblock} \tmtextit{Nuclear Physics B}, 251:691--734, 1985.{\newblock}
  
  \bibitem[3]{forrester2008importance}Peter Forrester  and  SVEN
  Warnaar.{\newblock} The importance of the selberg integral.{\newblock}
  \tmtextit{Bulletin of the American Mathematical Society}, 45(4):489--534,
  2008.{\newblock} \tmtextbf{I've read this paper completely}.{\newblock}
  
  \bibitem[4]{kobayashi2015symmetry}T.~Kobayashi  and  B.~Speh.{\newblock}
  \tmtextit{Symmetry Breaking for Representations of Rank One Orthogonal
  Groups}.{\newblock} Memoirs of the Amer. Math. Soc. 2015.{\newblock}
  
  \bibitem[5]{kobayashi2011schrodinger}Toshiyuki Kobayashi  and  Gen
  Mano.{\newblock} \tmtextit{The Schr{\"o}dinger model for the minimal
  representation of the indefinite orthogonal group $O (p, q)$},  volume 
  213.{\newblock} American Mathematical Society, 2011.{\newblock}
  
  \bibitem[6]{mehta2004random}Madan~Lal Mehta.{\newblock} \tmtextit{Random
  matrices},  volume  142.{\newblock} Academic press, 2004.{\newblock}
  
  \bibitem[7]{rainville1960special}Earl~David Rainville.{\newblock}
  \tmtextit{Special functions},  volume~8.{\newblock} Macmillan New York,
  1960.{\newblock}
  
  \bibitem[8]{Selberg:411367}A Selberg.{\newblock} Remarks on a multiple
  integral.{\newblock} \tmtextit{Norsk Mat. Tidsskr.}, 26:71--78,
  1944.{\newblock}
  
  \bibitem[9]{tarasov2003selberg}V Tarasov  and  Alexander
  Varchenko.{\newblock} Selberg-type integrals associated with
  $\mathfrak{sl}_3$.{\newblock} \tmtextit{Letters in Mathematical Physics},
  65(3):173--185, 2003.{\newblock}
  
  \bibitem[10]{warnaar2010sl3}S~Ole Warnaar.{\newblock} The $\mathfrak{sl}_3$
  Selberg integral.{\newblock} \tmtextit{Advances in Mathematics},
  224(2):499--524, 2010.{\newblock}
\end{thebibliography}

\end{document}
