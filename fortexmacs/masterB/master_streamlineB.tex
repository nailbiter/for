\documentclass[12pt]{article}
\usepackage{rotating}
\usepackage{afterpage}
\usepackage[english]{babel}
\usepackage{geometry,amsmath,amssymb,graphicx,bbm,latexsym,amsthm}
\usepackage{geometry,amsmath,amssymb,graphicx,bbm,latexsym,mathtools}
\geometry{letterpaper}
\usepackage[all]{xy}
\usepackage{xypic}
\usepackage{catchfilebetweentags}
\usepackage[toc,page]{appendix}
\usepackage{dot2texi}
\usepackage{tikz}
\usepackage[colorlinks]{hyperref}

%custom theorems
%\newtheorem{theorem}{Theorem}[section]

%%%%%%%%%% Start TeXmacs macros
\newcommand{\ontop}[2]{\genfrac{}{}{0pt}{}{#1}{#2}}
\newcommand*{\longhookrightarrow}{\ensuremath{\lhook\joinrel\relbar\joinrel\rightarrow}} 
\newcommand{\nobracket}{}
\newcommand{\nni}{\not\ni}
\newcommand{\nosymbol}{}
\newenvironment{itemizedot}{\begin{itemize} \renewcommand{\labelitemi}{$\bullet$}\renewcommand{\labelitemii}{$\bullet$}\renewcommand{\labelitemiii}{$\bullet$}\renewcommand{\labelitemiv}{$\bullet$}}{\end{itemize}}
\catcode`\<=\active \def<{
\fontencoding{T1}\selectfont\symbol{60}\fontencoding{\encodingdefault}}
\catcode`\>=\active \def>{
\fontencoding{T1}\selectfont\symbol{62}\fontencoding{\encodingdefault}}
\newcommand{\assign}{:=}
\newcommand{\comma}{{,}}
\newcommand{\nin}{\not\in}
\newcommand{\nocomma}{}
\newcommand{\tmop}[1]{\ensuremath{\operatorname{#1}}}
\newcommand{\tmrsub}[1]{\ensuremath{_{\textrm{#1}}}}
\newcommand{\tmrsup}[1]{\textsuperscript{#1}}
\newcommand{\tmtextbf}[1]{{\bfseries{#1}}}
\newcommand{\tmtextit}[1]{{\itshape{#1}}}
\newcommand{\tmtextup}[1]{{\upshape{#1}}}
\newcommand{\um}{-}
\renewenvironment{proof}{\noindent\textbf{Proof\ }}{\hspace*{\fill}$\Box$\medskip}
\newtheorem{proposition}{Proposition}[section]
\newtheorem{definition}[proposition]{Definition}
\newtheorem{lemma}[proposition]{Lemma}
\newtheorem{fact}[proposition]{Fact}
\theoremstyle{remark}
\newtheorem{remark}[proposition]{Remark}
%%%%%%%%%% End TeXmacs macros

\newcommand{\D}{\mathcal{D}}
% 

\newcommand{\supp}{\tmop{supp}}
% 

\newcommand{\proofexplanation}[1]{(#1)}
% 

\newcommand{\C}{\ensuremath{\mathbbm{C}}}
\newcommand{\Z}{\mathbbm{Z}}
% 

\newcommand{\Sp}{\mathbbm{S}}
% 

\newcommand{\R}{\mathbbm{R}}
% 

\newcommand{\mybra}[1]{(#1)}
% 

\newcommand{\mysbra}[1]{\left[ #1 \right]}
% 

\newcommand{\mycbra}[1]{\left\{#1\right\}}
\newcommand{\sone}{\ensuremath{\mybra{\D' (G \times_P \C_{\lambda - n})
\otimes \C_{\nu}}^{\Delta (P')}}}
\newcommand{\Upp}{{\mysetn{(x,y){\in}{\R}\tmrsup{p,q}}{x{\neq}0,{\hspace{0.75em}}y{\neq}0}}}
\newcommand{\Stab}{O(p,q)\tmrsub{e\tmrsub{p}}}
%\newcommand{\sol*}[1]{S ol(#1; {\lambda}, {\nu})}
\newcommand{\sol}{{\sol*{{\R}\tmrsup{p,q}}}}
\newcommand{\solXi}{{\sol*{{\Xi}}}}
\title{Symmetry breaking operators of indefinite orthogonal groups $O(p,q)$}
\author{Kobayashi T., Leontiev O.}
\begin{document}
\maketitle
\def\Dp{\mathcal{D}'}
\def\R{\mathbb{R}}


\setcounter{section}{8}
\section{$\mathcal{S} \tmop{ol} (U ; \lambda, \nu)$ and related notions (B8)}\label{sec:sol}


\subsection{Main results}

\begin{definition}
  \label{def-n-nots:def-n+invar}For $F \in \D' (U)$, where $U \subset
  \mathbbm{R}^{p, q}$ is an open set, we say that $F$ is
  \tmtextbf{$N_+'$-invariant on $U$} if $\forall b \in \mathbbm{R}^{p, q}$
  with $b_p = 0$ and $x_0 \in U$ such that $\frac{x_0 - Q (x_0) b}{1 - 2 Q
  (x_0, b) + Q (x_0) Q (b)} \in U$ and the expression makes sense (i.e. the
  denominator is non-zero) we have
  \begin{equation}
    \label{eq-Nequiv} | 1 - 2 Q (b, x) + Q (x) Q (b) |^{\lambda - n} F \left(
    \frac{x - Q (x) b}{1 - 2 Q (x, b) + Q (x) Q (b)} \right) = F (x)
  \end{equation}
  equality holding for $x$ near $x_0$.
\end{definition}

\begin{definition}
  \label{sol:def-sol}For $F \in \D' (U)$, where $U \subset \mathbbm{R}^{p, q}$
  is the open set, we say that $F \in \mathcal{S} \tmop{ol} (U ; \lambda,
  \nu)$ if the following holds:
  \begin{enumerate}
    \item if $x_0 \in U$ and $- x_0 \in U$, then $F (x) = F (- x)$ for $x$
    near $x_0$;
    
    \item if $(m, x_0, m \cdot x_0) \in O (p, q)_{e_p} \times U \times U$,
    then $F (x) = F (m \cdot x)$ for $x$ near $x_0$, where $O (p, q)_{e_p}
    \assign \{g \in O (p, q) |g \cdot e_p = e_p \}$;
    
    \item if $(\alpha, x_0, \alpha x_0) \in \mathbbm{R}_{> 0} \times U \times
    U$, then $\alpha^{\lambda - \nu - n} F (x) = F (\alpha x)$ for $x$ near
    $x_0$;
    
    \item $F$ is $N_+'$-invariant on $U$.{
    
    }{
    
    }For a closed set $S \subset U$ we will also use the notation $\mathcal{S}
    \tmop{ol}_S (U ; \lambda, \nu) \assign \{ u \in \mathcal{S} \tmop{ol} (U ;
    \lambda, \nu) | \tmop{supp} (u) \subset S \}$
  \end{enumerate}
\end{definition}

\begin{remark}
  It is easy to observe the following:
  \begin{enumerate}
    \item more precisely, item 1. means that for $(x_0, - x_0) \in U^2$ there
    is an open sets $U \supset V \ni x_0$ such that for $\psi : x \mapsto - x$
    a diffeomorphism of $\mathbbm{R}^{p, q}$ we have $u \mid_V =
    \psi^{\ast} (u |_{\psi (V)})$. Similarly, items 2, 3 and $N_+'$-invariance
    are explained. Note that all condinitions of definition \ref{sol:def-sol}
    are particular instances of situation in lemma \ref{sol:lem-holodep} (say,
    for item 4 one takes $\psi = \psi_b$ and $f = | c_b |$);
    
    \item that if $U = \bigcup_{i \in \Lambda} U_i$ is an open cover, then for
    $u \in \mathcal{D}' (U)$ and $(\lambda, \nu) \in \mathbbm{C}^2$ we have $u
    \in \mathcal{S} \tmop{ol} (U ; \lambda, \nu) \Leftrightarrow \forall i \in
    \Lambda, \; u \mid_{U_i} \in \mathcal{S} \tmop{ol} (U_i ; \lambda,
    \nu)$;
    
    \item when $U$ is an open cone, item 3 is equivalent to the usual
    definition of homogeneity of order $\lambda - \nu - n$;
  \end{enumerate}
\end{remark}

\begin{proposition}
  \label{sol:prop-sol}For every $(\lambda, \nu) \in \mathbbm{C}^2$ we have
  $\mathcal{S} \tmop{ol} (\mathbbm{R}^{p, q} ; \lambda, \nu) \simeq
  \tmop{Hom}_{G'} (I (\lambda), J (\nu))$.
\end{proposition}

\begin{proposition}
  \label{sol:prop-holocont}Suppose that $\Omega' \subseteq \Omega \subset
  \mathbbm{C}^k$ are open sets and $\Omega \ni \mu \mapsto (\lambda (\mu), \nu
  (\mu)) \in \mathbbm{C}^2$ is holomorphic open map. Suppose further that
  $K_{\mu} \in \mathcal{D}' (U)$ with ($U \subset \mathbbm{R}^{p, q}$:open)
  holomorphically depends on $\mu \in \Omega$ and for $\mu \in \Omega'$ we
  have $K_{\mu} \in \mathcal{S} \tmop{ol} (U ; \lambda (\mu), \nu (\mu))$.
  Then $K_{\mu} \in \mathcal{S} \tmop{ol} (U ; \lambda (\mu), \nu (\mu))$ for
  $\mu \in \Omega$ as well.
\end{proposition}

\begin{remark}
  The last two propositions are \tmtextbf{not} new. The statement and proof
  below are just rephrasing and somehow elaboration of those given in
  {\cite[thm 3.16]{kobayashi2015symmetry}} and {\cite[prop.
  3.18]{kobayashi2015symmetry}} respectively.
\end{remark}

\subsection{Auxiliary lemmas}

\begin{lemma}
  \label{sol:lem-unfold}For $G \assign O (p + 1, q + 1)$ and $\lambda \in
  \mathfrak{a}_{\mathbbm{C}}^{\ast}$ let $\lambda : P \rightarrow
  \mathbbm{C}^{\times}$ be a homomorphism defined as in proposition
  \ref{def-n-nots:prop-degseries}. Then for $C^{\infty} (G)_{\lambda} \assign
  \left\{ f \in C^{\infty} (G) \mid \forall p \in P, \; f (\cdot p) =
  \lambda (p^{- 1}) f (\cdot) \right\}$ we have the map $C^{\infty} \ni f
  (\cdot) \mapsto [g P \mapsto [g, f (g)]] \in C^{\infty} (G \times_P
  \mathbbm{C}_{\lambda})$ being an isomorphism of vector spaces.
\end{lemma}

\begin{proof}
  The well-definedness of this map is straightforward to verify. Conversely,
  given $f' \in C^{\infty} (G \times_P \mathbbm{C}_{\lambda})$, we can
  construct the corresponding element of $C^{\infty} (G)_{\lambda}$ as
  follows: for $g \in G$ we let $f (g) \assign x \in \mathbbm{C}$ such that
  $(g, x) \in f' (g H)$. It is easy to see that such $x$ exists and is unique.
  The obtained mapping $g \mapsto f (g)$ forms a smooth map of $C^{\infty}
  (G)$, which is in fact an element of $C^{\infty} (G)_{\lambda}$, as we
  should have $g p$ for $p \in P$ being mapped to $\lambda (p^{- 1}) f (g)$,
  since $[g, x] = [g p, \lambda (p^{- 1}) x]$.
\end{proof}

\begin{lemma}
  \label{sol:lem-commdiag}Recall the $G \assign O (p + 1, q + 1)
  \curvearrowright \Xi$ action. Fix $U \subset N_- \simeq \mathfrak{n}_-
  \simeq \mathbbm{R}^{p, q}$ an open set. The following diagram then commutes:
  
\ExecuteMetaData[../masterdiags/b1.tex]{diagram}
  
  Here for $\lambda \in \mathbbm{C}$, and $V \subset \Xi$ invariant under
  multiplication by $\mathbbm{R}^{\times}$ we let $C^{\infty}_{- \lambda} (V)
  \assign \left\{ f \in C^{\infty} (V) \mid \forall \alpha \in
  \mathbbm{R}^{\times}, \; f (\alpha \cdot) = | \alpha |^{- \lambda} f (\cdot)
  \right\}$. $G \times_P \mathbbm{C}_{\lambda} \mid_{\psi (U)}$ denotes
  the portion of vector bundle $G \times_P \mathbbm{C}_{\lambda}$ above $\psi
  (U)$. $\psi (\cdot)$ denotes embedding of $N_- \simeq \mathfrak{n}_- \simeq
  \mathbbm{R}^{p, q}$ into $G / P$ (on the left) or into $\Xi$ (on right) The
  top map is a continuous $G$-equivariant isomorphism, defined for $f (\cdot)
  \in C^{\infty} (G \times_P \mathbbm{C}_{\lambda}) \simeq C^{\infty}
  (G)_{\lambda}$ as $\tilde{f} \in C^{\infty}_{- \lambda} (\Xi)$, $\tilde{f}
  (g p_+) \assign f (g)$ (here $p_+ \assign (1, 0_{p + q}, 1) \in \Xi$). The
  left vertical arrow is a restriction to an open subset $\psi (U) P \subset G
  / P$. The right-vertical arrow is restriction to a submanifold $U \cdot (1,
  0_{p + q}, 1) \subset \Xi$.
\end{lemma}

\begin{proof}
  We assume $U =\mathfrak{n}_-$ for simplicity, the proof for general $U$
  proceeds in precisely the same manner. First, we show that for smooth
  functions the top map is well-defined. Indeed, let $f \in C^{\infty}
  (G)_{\lambda} $ as in lemma \ref{sol:lem-unfold}. Then the map $\tilde{f} :
  g (1, 0_{p + q}, 1) \mapsto f (g)$ on $G p_+$ is well-defined, as
  centralizer of $p_+$ equals to $M^0 N_+$ by proposition
  \ref{def-n-nots:prop-ximodel}, and for $\lambda : P \rightarrow
  \mathbbm{C}^{\times}$ is in lemma \ref{sol:lem-unfold}, we have $\lambda
  (M^0 N_+)$. Moreover, as action $G \curvearrowright \Xi$ is transitive by an
  argument similar to that of lemma \ref{doublePGP:lem-Gp-act-Xi}, we have $g
  p_+ \mapsto f (g)$ being well-defined map $\tilde{f} \in C^{\infty} (\Xi)$.
  
  We next show that $\tilde{f}$ is also homogeneous of degree $- \lambda$, as
  we have for $\tilde{f} (- g p_+) = \tilde{f} (g m_0 p_+)$, for $m_0 \assign
  \tmop{diag} (- 1, 1_{p + q}, 1) \in M$ and hence $\tilde{f} (g m_0 p_+)
  \assign f (g m_0) = f (g) = \tilde{f} (g p_+)$, since $f \in C^{\infty}
  (G)_{\lambda}$. Similarly, for $\alpha \in \mathbbm{R}_{> 0}$ we have
  $\tilde{f} (\alpha g p_+) = \tilde{f} (g a (t) p_+)$ with $a (t)$ is in
  (\ref{def-n-nots:eq-A}) for $t = \ln (\alpha)$, and hence the chain of
  equalities continues as $= \tilde{f} (g a (t) p_+) \assign f (g a (t)) =
  e^{- \lambda t} f (g) = \alpha^{- \lambda} f (g p_+)$. This shows
  homogeneity.
  
  The fact that top map is isomorphic is implied by the fact that we can
  construct inverse map $C^{\infty}_{- \lambda} (\Xi) \rightarrow C^{\infty}
  (G)_{\lambda}$ by $\tilde{f} \mapsto f$, where we set $f (g) \assign
  \tilde{f} (g p_+)$. It can be similarly to above shown that this map is
  well-defined.
  
  Finally, the diagram commutes by direct inspection and proposition
  \ref{def-n-nots:prop-ximodel}.
\end{proof}

\begin{lemma}
  \label{sol:lem-holodep}Suppose $\Omega' \subset \Omega \subset
  \mathbbm{C}^m$ are an open set and $\kappa : \Omega \rightarrow \mathbbm{C}$
  is holomorphic. Suppose also that $U, V \subset \mathbbm{R}^k$ are open with
  $\psi : U \rightarrow V$ diffeomorphism, and $K_{\mu}^U \in \mathcal{D}'
  (U), K^V_{\mu} \in \mathcal{D}' (V)$ are holomorphic in $\mu \in \Omega$.
  Then if for $f \in C^{\infty} (U \rightarrow \mathbbm{R}_{> 0})$ we have
  $\forall \mu \in \Omega', \; f^{\kappa (\mu)} K^U_{\mu} = \psi^{\ast}
  K^V_{\mu}$, then this also holds for every $\mu \in \Omega$.
\end{lemma}

\begin{proof}
  In the light of holomorphic rigidity, it suffices to show that $\psi^{\ast}
  K^V_{\mu}, f^{\kappa (\mu)} K^U_{\mu} \in \mathcal{D}' (U)$ are holomorphic
  in $\mu \in \Omega$. For $\psi^{\ast} K_{\mu}^V$ this is readily given by
  proposition \ref{holomorphicity-preserving:prop-pullback-holo}. Regarding
  $f^{\kappa (\mu)} K^U_{\mu}$, lemma
  \ref{KR-normalization-recur:lem-mult-smth} implies that the latter can be
  seen as product of distributions and hence in the light of propositions
  \ref{holomorphicity-preserving:prop-tensor-holo} and
  \ref{holomorphicity-preserving:prop-pullback-holo}, it suffices to show that
  $f^{\kappa (\mu)}$ is holomorphic when seen as an element of $\mathcal{D}'
  (U)$. This can be proven directly.
\end{proof}

\subsection{Proofs}

\begin{definition}
  \label{sol:def-localaciton}For $\lambda \in \mathbbm{C}$ let $P'
  \curvearrowright G / P$ by left multiplication. Introduce also an embedding
  $\mathbbm{R}^{p, q} \ni w \mapsto \psi (w) \assign n_- (w) P \in G / P$.
  Suppose now that $(U, V)$ is a pair of open subsets of $\mathbbm{R}^{p, q}$
  and $p' \in P$ such that $\psi (U) = p' \psi (V)$. Then straightforward
  generalization of lemma \ref{KR-normalization-recur:lem-pull-comm-restr}
  tells us that the following diagram commutes\\
  \ExecuteMetaData[../masterdiags/b2.tex]{diagram}
  here $G \times_P \mathbbm{C}_{\lambda} \mid_{\psi (U)}$ denotes portion
  of the bundle above $\psi (U)$ and similarly for $\psi (V)$. The top and
  bottom maps are isomorphisms. Now, pullback by $\psi (\cdot)$ induces the
  isomorphism $\mathcal{D}' (U) \rightarrow \mathcal{D}' (V)$ so that the
  following map commutes:\\
  \ExecuteMetaData[../masterdiags/b3.tex]{diagram}
  
  all maps of it are isomorphisms and we will call the one at the bottom by $c
  (\lambda ; p' ; U, V) : \mathcal{D}' (U) \rightarrow \mathcal{D}' (V)$.
\end{definition}

\begin{definition}
  \label{sol:def-D'n}For $(\lambda, \nu) \in \mathbbm{C}^2$ we let
  $\mathcal{D}' (\mathfrak{n}_-, \mathbbm{C}_{n - \lambda} \otimes
  \mathbbm{C}_{\nu})^{M' A', \mathfrak{n}'_+}$ to denote the space of $u \in
  \mathcal{D}' (\mathbbm{R}^{p, q})$ such that for all $U, V, p'$ as in
  definition \ref{sol:def-localaciton} we have $u \mid_V = \nu^{} ((p')^{-
  1}) c (n - \lambda ; p' ; U, V) u \mid_V$ with homomorphism $\nu : P'
  \rightarrow \mathbbm{C}^{\times}$ defined as in proposition
  \ref{def-n-nots:prop-degseries}.
\end{definition}

\begin{proof}
  (of prop. \ref{sol:prop-sol}) Proposition \ref{doublePGP:prop-pnp} implies
  that we can apply {\cite[thm. 3.16]{kobayashi2015symmetry}} and it
  immediately gives
  \[ \tmop{Hom}_{G'} (I (\lambda), J (\nu)) \simeq \mathcal{D}'
     (\mathfrak{n}_-, \mathbbm{C}_{n - \lambda} \otimes \mathbbm{C}_{\nu})^{M'
     A', \mathfrak{n}'_+} \]
  Hence it suffices to show that $\mathcal{D}' (\mathfrak{n}_-, \mathbbm{C}_{n
  - \lambda} \otimes \mathbbm{C}_{\nu})^{M' A', \mathfrak{n}'_+} =\mathcal{S}
  \tmop{ol} (\mathbbm{R}^{p, q} ; \lambda, \nu)$ as in definition
  \ref{sol:def-sol}. Hence, we just need to concretely write down the $c
  (\lambda ; p' ; U, V)$ map of \ref{sol:def-localaciton}. Moreover, as we
  have all arrows in commutative diagrams of definitions
  \ref{sol:def-localaciton} being continuous and mapping smooth to smooth,
  using approximation by smooth sections, we can assume that we have
  $\mathcal{D}'$ replaced by $C^{\infty}$ in all diagrams of definitions
  \ref{sol:def-localaciton}. Now, lemma \ref{sol:lem-commdiag} allows us to
  rewrite second commutative diagram of definition \ref{sol:def-localaciton}
  (with $\lambda$ replaced by $n - \lambda$, according to definition
  \ref{sol:def-D'n})
  
  \ExecuteMetaData[../masterdiags/b4.tex]{diagram}
  
  with an embedding $\psi : \mathbbm{R}^{p, q} \ni w \mapsto n_- (w) \cdot (1,
  0_{p + q}, 1) = (1 - Q (w), 2 w, 1 + Q (w)) \in \Xi$.
  
  Now, let $(p, U, V)$ be as in definition \ref{sol:def-localaciton}. If $p$
  is of the form
  \[ p \assign \left[ \begin{array}{ccc}
       1 & 0 & 0\\
       0 & m & 0\\
       0 & 0 & 1
     \end{array} \right] \in M' \]
  with $m \in O (p, q)$ such that $m \cdot e_p = e_p$, then for $w \in
  \mathbbm{R}^{p, q}$ we have $p \cdot \psi (w) = \psi (m \cdot w)$.
  
  This implies that for $p, U, V$ as in definition \ref{sol:def-localaciton}
  with $p$ as above we should have $U = w V$ (and conversely all such triples
  satisfy definition \ref{sol:def-localaciton}) and $c (p ; \lambda ; U, V)
  (F) (\cdot) = F (w \cdot)$.
  
  Next ,for an element $p \assign \tmop{diag} (- 1, 1_{p + q}, - 1) \in M'$ we
  have $p \cdot \psi (w) = \psi (- w)$. This implies that for $p, U, V$ as in
  definition \ref{sol:def-localaciton} with $p$ as above we should have $U = -
  V$ (and conversely all such triples satisfy definition
  \ref{sol:def-localaciton}) and $c (p ; \lambda ; U, V) (F) (\cdot) = F ((-
  1) \cdot)$.
  
  Next, for an element $a (t) \in A'$ as in (\ref{def-n-nots:eq-A}), direct
  calculations show that we have we have $a (t) \cdot \psi (w) = e^t \psi
  (e^{- t} w)$ and hence for an element $f \in C^{\infty}_{\lambda - n}$ we
  would have $f (a (t) \psi (w)) = e^{(\lambda - n) t} f (\psi (e^{- t} w))$,
  which implies that for $p, U, V$ as in definition \ref{sol:def-localaciton}
  with $p = a (t)$ as above we should have $U = e^{- t} V$ (and conversely all
  such triples satisfy definition \ref{sol:def-localaciton}) and $c (a (t) ;
  \lambda ; U, V) (F) (\cdot) = e^{(\lambda - n) t} F (e^{- t} \cdot)$.
  
  Finally, for $w \in \mathbbm{R}^{p, q}$ with $w_p = 0$ and an element $n_+
  (w) \in N'_+$ as in (\ref{def-n-nots:eq-N+}), we observe that for $v \in
  \mathbbm{R}^{p, q}$ we have for $\psi_w (\cdot)$ and $c_w (\cdot)$ as in
  definition \ref{def-n-nots:def-n+invar}
  \[ n_+ (w) n_- (v) = c_w (v) \cdot n_- (\psi_w (v)) \]
  whenever both sides make sense. Hence, for $p, U, V$ as in definition
  \ref{sol:def-localaciton} with $p = n_+ (w)$ as above we should have $U =
  \psi_w (V)$ (and conversely all such triples satisfy definition
  \ref{sol:def-localaciton}) and $c (n_+ (w) ; \lambda ; U, V) (F) (\cdot) = |
  c_w (\cdot) |^{\lambda - n} F (\psi_w (\cdot))$.
  
  As the reasoning above tells us how $c (\lambda ; p' ; U, V)$ are written
  for $p' \in M', A', N_+'$, we see that these are precisely equivalent to
  items of definition \ref{sol:def-sol} and thus Langlands decomposition of
  $P'$ tells us now that $\mathcal{D}' (\mathfrak{n}_-, \mathbbm{C}_{n -
  \lambda} \otimes \mathbbm{C}_{\nu})^{M' A', \mathfrak{n}'_+} =\mathcal{S}
  \tmop{ol} (\mathbbm{R}^{p, q} ; U, V)$.
\end{proof}

\begin{proof}
  (of prop. \ref{sol:prop-holocont}) In the light of the first item of remark
  after the definition \ref{sol:def-sol}, the conclusion is implied by lemma
  \ref{sol:lem-holodep}.
\end{proof}

\section{Determination of $\mathcal{S} \tmop{ol} (\{ Q \neq 0 \} ; \lambda,\nu)$ (B9)}\label{sec:lem67}


\subsection{Main results}

\begin{proposition}
  \label{lem67:prop-67}For $(\lambda, \nu) \in \mathbbm{C}^2$ we have the
  following\footnote{Note that when restricted to $\{ Q \neq 0 \}$, $| Q |^{-
  \nu}$ is smooth, so this is just product of distribution and smooth
  function}:
  \[ \mathcal{S} \tmop{ol} (\{ Q \neq 0 \} ; \lambda, \nu) =\mathbbm{C} | Q
     |^{- \nu} \frac{| x_p |^{\lambda + \nu - n}}{\Gamma \left( \frac{\lambda
     + \nu - n + 1}{2} \right)} \]
\end{proposition}

\subsection{Auxiliary results}

\begin{definition}
	\label{lem67:def-SCSCC}
  Fix $n \in \mathbbm{Z}_{\geqslant 1}$ and define the following:
  \begin{enumerate}
    \item Let $\mathcal{S}\mathcal{C} \subset 2^{\mathbbm{R}^n}$ be the family
    of all open subsets $U \subset \mathbbm{R}^n$ such that $\forall x_0 \in
    \mathbbm{R}^{n - 1}$ we have sets $\{ t \geqslant 0 | (x_0, t) \in U \}$
    and $\{ t \leqslant 0 | (x_0, t) \in U \}$ being connected (for $n = 1$ we
    let $\mathcal{S}\mathcal{C}$ denote all open subsets of $\mathbbm{R}^1$);
    
    \item Let $\mathcal{S}\mathcal{S}\mathcal{C} \subset 2^{\mathbbm{R}^n}$ to
    denote the family of all open subsets $U \subset \mathbbm{R}^n$ such that
    we have $\pi^i (U) \in \mathcal{S}\mathcal{C}$ for $i = 0 \ldots n - 1$,
    where $\pi : \mathbbm{R}^n \rightarrow \mathbbm{R}^{n - 1}$ is projection
    along the last coordinate.
  \end{enumerate}
\end{definition}

\begin{lemma}
  \label{lem67:lem-geom-aux}For $Q$ being quadratic form on $\mathbbm{R}^{p,
  q}$ with $p, q \geqslant 1$, $n \assign p + q$, $\tilde{Q}$ quadratic form
  on $\mathbbm{R}^{p, q - 1}$ and $\pi : \mathbbm{R}^n \rightarrow
  \mathbbm{R}^{n - 1}$ projection along the last coordinate, we have
  \begin{eqnarray*}
    \pi (\{ Q > 0 \}) = \{ \tilde{Q} > 0 \} &  & \\
    \pi (\{ Q < 0 \}) =\mathbbm{R}^{n - 1} &  & 
  \end{eqnarray*}
\end{lemma}

\begin{proof}
  The $\supseteq$ is clear for both equalities and this immediately proves the
  second one. For the first, $\subseteq$ is also clear, as $Q (x, x_n) =
  \tilde{Q} (x) - x_n^2 \leqslant \tilde{Q} (x)$.
\end{proof}

\begin{lemma}
  \label{lem67:lem-geom}If $Q$ is a quadratic form on $\mathbbm{R}^{p, q}$
  with $p, q \geqslant 1$, then $\{ \pm Q > 0 \} \in
  \mathcal{S}\mathcal{C}\mathcal{C}$.
\end{lemma}

\begin{proof}
  In the light of lemma \ref{lem67:lem-geom-aux} (and as for $P$ quadratic on
  $\mathbbm{R}^{k, 0}$, $\{ P > 0 \} =\mathbbm{R}^{n - 1} \backslash \{ 0 \}
  \in \mathcal{S}\mathcal{C}\mathcal{C}$ clearly), it suffices to show that
  $\{ \pm Q > 0 \} \in \mathcal{S}\mathcal{C}$. But the latter follows by
  direct check, as for $x_0 \in \mathbbm{R}^{n - 1}$ and $\varepsilon_{1, 2} =
  \pm 1$, $\{ t | \varepsilon_1 t \geqslant 0, \varepsilon_2 Q (x_0, t) > 0 \}
  = \{ t | \varepsilon_1 t \geqslant 0, \varepsilon_2 t^2 < \tmop{const} \}$
  with $\tmop{const} : = \varepsilon_2 Q (x_0, 0)$ and is clearly connected.
\end{proof}

\begin{lemma}
	\label{lem67:lem-tensor-aux}With notation as in Definition \ref{lem67:def-SCSCC}, let $\mathcal{S}\mathcal{C} \ni X \subset
  \mathbbm{R}^n$ be stable under change of sign of $x_n$ and $u \in
  \mathcal{D}' (X)$ with $\partial_n u = 0$ and even in the $x_n$. Then, there
  exists unique $g \in \mathcal{D}' (\pi (X))$ such that $u = (g \otimes 1)
  \mid_X$. Moreover,
  \begin{enumerate}
    \item If $\psi$ and $\psi'$ are diffemorophisms of $X$ and $\pi (X)$
    respectively, such that $\pi \circ \psi = \psi' \circ \pi$ and $u (\psi
    (\cdot)) = u (\cdot)$ on $X$, then $g (\psi' (\cdot)) = g (\cdot)$ on $\pi
    (X)$;
    
    \item If $X$ was an open cone and for Euler operator $E$ we have $E u =
    \lambda u$ on $X$ for some $\lambda \in \mathbbm{C}$, then $\tilde{E} g =
    \lambda g$, where $\tilde{E}$ is Euler operator in $\mathbbm{R}^{n - 1}$
    (note that $\pi (X)$ is also an open cone);
    
    \item If for some $i < n$ we have $\partial_i u = 0$, then $\partial_i g =
    0$.
  \end{enumerate}
\end{lemma}

\begin{fact}
\label{fact:localization}{\cite[Thm 2.2.4]{hormander1983analysis}}
Let $X_i$ be the family of open subsets of $\R^n$ and $f_i \in \D' (X_i)$.
Suppose further that $\forall i, j, \hspace{0.75em} f_i \mid_{X_i \cap
X_j} = f_j \mid_{X_i \cap X_j}$. Then there exists unique $f \in \D'
(\bigcup_i X_i)$ such that $f \mid_{X_i} =
f_i$.
\end{fact}

\begin{fact}
{\proofexplanation{{\cite[Thm. 3.1.4']{hormander1983analysis}}}}
\label{fact:sing-q-4}Let $u \in \D' (Y \times I)$ with $Y \subset \R^n$ open
and $I \subset \R$ interval. Assume further that $\partial_n u = 0$. Then,
there exists a distribution $u_0 \in \D' (Y)$ such that $u (\varphi) = \int_I
u_0  (x \mapsto \varphi (x, t))  \hspace{0.75em}
dt$.
\end{fact}

\begin{fact}
\label{lem67:fact-pullback}{\cite[thm.
6.1.2]{hormander1983analysis}} For $X_j \subset \mathbbm{R}^{n_j}$ for $j = 1,
2$ being open sets and $f : X_1 \rightarrow X_2$ such that $\forall x \in X_1,
\; (D f) (x) : \mathbbm{R}^{n_1} \rightarrow \mathbbm{R}^{n_2}$ is surjective,
there is unique continuous map $f^{\ast} : \mathcal{D}' (X_2) \rightarrow
\mathcal{D}' (X_1)$ such that for $u \in C^{\infty} (X_2)$ we have $f^{\ast} u
= u \circ f$. Moreover, the following formulae hold:
\begin{eqnarray}
  & \partial_j f^{\ast} u = \sum_{k = 1}^{n_2} \partial_j f_k \cdot f^{\ast}
  (\partial_k u) &  \nonumber\\
  & (f^{\ast} a) (f^{\ast} u) = f^{\ast} (a u), \quad a \in C^{\infty} (X_2)
  &  \nonumber
\end{eqnarray}
\end{fact}

\begin{proof}
  We first prove the uniqueness of $g$. Suppose $u = (g \otimes 1) \mid_X
  = (g' \otimes 1) \mid_X$ for $g, g' \in \mathcal{D}' (X)$. Take $y_0 \in
  \pi (X)$, then there exists $x_0 \in X$ such that $\pi (x_0) = y_0$ and we
  can take a small open neighborhood $y_0 \in U \subset \pi (X)$ such that for
  some open interval $I \subset \mathbbm{R}$ we have $x_0 \in U \times I
  \subset X$. Uniqueness part of fact
  \ref{holomorphicity-preserving:fact-tensor} then implies that $u \mid_{U
  \times I} = g \mid_U \otimes 1 = g' \mid_U \otimes 1$ and then again
  uniqueness of fact \ref{holomorphicity-preserving:fact-tensor} implies that
  $g \mid_U = g' \mid_U$. As $y_0 \in X$ was arbitrary, application of
  fact \ref{fact:localization} shows that $g = g'$. This shows the uniqueness.
  
  Now, we prove the existence. We take arbitrary $y_0 \in \pi (X)$ and
  corresponding $x_0 \ni U$ as in previous paragraph, so that $x_0 \in U
  \times I \subset X$. Fact \ref{fact:sing-q-4} then tells us that $u
  \mid_{U \times I} = u \mid_U \otimes g_U$ for some $g_U \in
  \mathcal{D}' (U)$. In the light of fact \ref{fact:localization}, it suffices
  then to show that for $U, U' \subset \pi (X)$ such that $U \cap U' \neq
  \varnothing$ and $I, I' \subset \mathbbm{R}$ connected we have $g_U =
  g_{U'}$ on $U \cap U'$. Again, fact \ref{fact:localization} allows us to
  work locally, so we take $y_0 \in U \cap U'$. We take corresponding $x_0 \in
  U \times I$ and $x_0' \in U \times I'$ such that $\pi (x_0) = \pi (x_0') =
  y_0$.
  
  If $x_0$ and $x_0'$ lie on the same side of $\{ x_n = 0 \}$ plane (say, both
  have $n$-th corrdinate non-negative), then assumption $X \in
  \mathcal{S}\mathcal{C}$ tells us that $x_0$ and $x_0'$ can be connected by
  line segment $J$ orthogonal to $\{ x_n = 0 \}$ and lying within $U$. Due to
  the openness of $X$ we then can take small $y_0 \in V \subset \pi (X)$ and
  $J \subset J' \subset \mathbbm{R}$ such that $V \times J' \subset X$. Fact
  \ref{fact:sing-q-4} then implies that $u \mid_{V \times J'} = g_V
  \otimes 1$ and thus $g_U \mid_V = g_{U'} \mid_V$.
  
  On the other hand, if $x_0$ and $x_0'$ lie on different sides, invariance
  of $X$ and $u$ with respect to flip in the $n$-th coordinate allows us to
  bring this situation to the one discussed in the previous paragraph and thus
  to get the same conclusion. This shows existence.
  
  Finally, we show items 1, 2 and 3. Item 1 readily follows by uniqueness
  we've shown, as we have $u = \psi^{\ast} u = \psi^{\ast} \pi_n^{\ast} g =
  (\pi_n \circ \psi)^{\ast} g = (\psi' \circ \pi_n)^{\ast} g = \pi_n^{\ast}
  (\psi')^{\ast} g = (\psi')^{\ast} g \otimes 1$. Item 3 holds true by, as we
  have $\partial_i (g \otimes 1) = (\partial_i g) \otimes 1$, this in turn
  holding by uniqueness part of fact
  \ref{holomorphicity-preserving:fact-tensor}. And finally, item 2 holds true
  because of the formulae in fact \ref{lem67:fact-pullback}.
\end{proof}

\begin{lemma}
	\label{lem67:lem-tensor}With notation as in Definition \ref{lem67:def-SCSCC}, let $\mathcal{S}\mathcal{S}\mathcal{C} \ni X \subset
  \mathbbm{R}^n$ be stable under $(x_i)_{i = 1}^n \mapsto (\varepsilon_i
  x_i)_{i = 1}^n$ for $\varepsilon_i = \pm 1$ and $u \in \mathcal{D}' (X)$
  with $\partial_i u = 0$ for $i = 2, 3, \ldots, n$ and even in every $x_i$.
  Then, there exists unique $g \in \mathcal{D}' (\pi^{n - 1} (X))$ which is
  even (note that $\pi^{n - 1} (X) \subset \mathbbm{R}$). Moreover, if $X$ was
  an open cone and $E u = \lambda u$ for $\lambda \in \mathbbm{C}$ and $E$
  Euler operator, then $\pi^{n - 1} (X)$ is an open cone as well with $x
  \frac{\partial}{\partial x} g = \lambda g$.
\end{lemma}

\begin{proof}
  Let's do the induction on $n$, the case $n = 1$ being trivial, as hypothesis
  and conclusion coincide. Now, assuming statement holding for lower
  dimensions, let's prove it for $X \subset \mathbbm{R}^n$. Lemma
  \ref{lem67:lem-tensor-aux} readily grants us $g \in \mathcal{D}' (\pi (X))$,
  so that $u = (g \otimes 1) \mid_X$. Now, $g$ is even in every $x_i$ with
  $i \geqslant 2$ and for $i \geqslant 2$, $\partial_i g = 0$ by first and
  third item of lemma \ref{lem67:lem-tensor-aux} respectively and this allows
  us to use induction assumption and finish the proof. The resulting $g$ is
  even, as $u$ was even in all $x_i$ by hypothesis. The ``moreover'' part of
  lemma follows similarly by induction, employing item 2. of lemma
  \ref{lem67:lem-tensor-aux} on induction step.
\end{proof}

\begin{lemma}
  \label{lem67:lem-homogR}For $\lambda \in \mathbbm{C}$ and $E \assign x
  (\partial / \partial x)$ we have
  \[ \begin{array}{cc}
       \begin{array}{c}
         
       \end{array} u \in \mathcal{D}' (\mathbbm{R}\backslash \{ 0 \}), \; u (-
       x) = u (x), \; E u = \lambda u \Leftrightarrow & u (x) \in \mathbbm{R}
       | x |^{\lambda}\\
       u \in \mathcal{D}' (\mathbbm{R}), u (- x) = u (x), \; E u = \lambda u
       \Leftrightarrow & u (x) \in \mathbbm{R} | x |^{\lambda} / \Gamma \left(
       \frac{\lambda + 1}{2} \right)
     \end{array} \]
\end{lemma}

\begin{fact}
\label{fact:homog-tempered}{\cite[thm.
7.1.18]{hormander1983analysis}} If $u \in \mathcal{D}' (\mathbbm{R}^n)$ and $u
\mid_{\mathbbm{R}^n \backslash \{ 0 \}}$ is homogeneous, then $u \in
\mathcal{S}' \subset \mathcal{D}'$.
\end{fact}

\begin{proof}
  Indeed, the first statement follows, as condition $E u = \lambda u$ implies
  that $(\partial / \partial x) (| x |^{- \lambda} u)$=0 on $\mathbbm{R}^{}
  \backslash \{ 0 \}$ and hence fact \ref{fact:sing-q-4} implies that $u = c_+
  x_+^{\lambda} + c_- x_-^{\lambda}$ and evenness assumption implies that $c_+
  = c_-$ which gives the final answer. And conversely, multiples of $| x
  |^{\lambda}$ clearly satisfy the requirements.
  
  Regarding the second statement, multiples of $| x |^{\lambda} / \Gamma
  \left( \frac{\lambda + 1}{2} \right)$ clearly satisfy the requirements, as
  requirements are analytic in $\lambda$, so is $| x |^{\lambda} / \Gamma
  \left( \frac{\lambda + 1}{2} \right)$ and the latter satisfies requirements
  for $\tmop{Re} (\lambda) \gg 0$. Conversely, given such a function $u$, the
  reverse implication is readily granted by fact
  \ref{holomorphicity-preserving:fact-homog} in case $\lambda \nin
  -\mathbbm{Z}_{\geqslant 0}$.
  
  In turn, $\lambda \in - 2\mathbbm{Z}_{\geqslant 0}$, we have (according to
  the result above and as $| x |^{- 2 n}$ for $n \in \mathbbm{Z}_{\geqslant
  0}$ is well-defined generalized function on $\mathbbm{R}$) that for some $c
  \in \mathbbm{R}$, $u - c | x |^{\lambda}$ is supported at $0$, hence should
  be a finite sum of derivatives of delta function. Now, as $E (u - c | x
  |^{\lambda}) = \lambda (u - c | x |^{\lambda})$ and derivatives of delta
  functions are linearly independent (this can be seen by repeatedly applying
  $E$ to the sum of them), we should have $u - c | x |^{\lambda} = a
  \delta^{(- 1 - \lambda)} (x)$ (as $\delta^{(k)}$ is homogeneous of degree $-
  k - 1$), but the right hand side of latter equality is odd, while left-hand
  side is even, hence $u = c | x |^{\lambda}$.
  
  Finally, suppose that $\lambda \in - 2\mathbbm{Z}_{\geqslant 0} - 1$. Now,
  fact \ref{fact:homog-tempered} implies that $u$ under the assumptions taken
  is in fact tempered distribution (note that the first statement tells us
  directly that $u \mid_{\mathbbm{R}\backslash \{ 0 \}}$ is homogeneous),
  hence we may consider Fourier transform $\hat{u} \in \mathcal{S}'
  (\mathbbm{R}) \subset \mathcal{D}' (\mathbbm{R})$ of it. Properties of
  Fourier transform imply that $\hat{u}$ is even real-valued on $\mathcal{S}
  (\mathbbm{R})$ and $E \hat{u} = - (\lambda + 1) \hat{u}$, hence $\hat{u}$ is
  a multiple of $x^{- \lambda - 1}$ and then inverse Fourier transform gives
  the desired.
\end{proof}

\begin{lemma}
  \label{lem67:lem-homogImpliesE}Let $U \subset \mathbbm{R}^m$ be an open cone
  and $F \in \mathcal{D}' (U)$. Then $F$ is homogeneous of degree $a \in
  \mathbbm{C}$ (that is, $\forall \alpha > 0$, $F (\alpha \cdot) = \alpha^a F
  (\cdot)$) iff $E F = a F$ on $U$.
\end{lemma}

\begin{proof}
  We do proof for $U =\mathbbm{R}^m$, the proof of general case following the
  same pattern. We first prove the ``$\Rightarrow$'' direction. We recall that
  what $F (\alpha \cdot) = \alpha^a F (\cdot)$ means for $F \in \mathcal{D}'
  (\mathbbm{R}^m)$ is that for every $\varphi \in C^{\infty}_0
  (\mathbbm{R}^m)$ we have $\langle F, \varphi \rangle = \alpha^{a + m}
  \langle F, \varphi_{\alpha} \rangle$, where $\varphi_{\alpha} (\cdot)
  \assign \varphi (\alpha \cdot)$. We now fix arbitrary $1 \in V \subset
  \mathbbm{R}_{> 0}$ open and such that $\bar{V}$ is a compact subset of
  $\mathbbm{R}_{> 0}$, and let $\Phi : V \ni \alpha \mapsto \langle F,
  \varphi_{\alpha} \rangle$. The assumptions we put on $V$ imply that
  $S_{\alpha} \assign \left\{ x \in \mathbbm{R}^m \mid \exists \alpha \in
  \bar{V}, \; \alpha^{- 1} x \in \tmop{supp} (\varphi) \right\} \subset
  \mathbbm{R}^m$ is compact and since $\forall \alpha \in V$ we have
  $\varphi_{\alpha} = 0$ outside $S_{\alpha}$, we can apply fact
  \ref{holomorphicity-preserving:fact-basic} which tells us that $\Phi \in
  C^{\infty} (V)$ and $\partial \Phi / \partial \alpha = \langle F, (\partial
  / \partial \alpha) \varphi_{\alpha} \rangle = \langle F, (E \varphi) (\alpha
  \cdot) \rangle$.
  
  Now, assuming that $\forall \alpha > 0$, $F (\alpha \cdot) = \alpha^a F
  (\cdot)$ holds, the definitions made imply that $\forall \alpha > 0, \;
  \alpha^{a + m} \Phi (\alpha) = \Phi (1)$ and taking the derivative of both
  sides with respect to $\alpha$ at $\alpha = 1$, one arrives at $0 = (a + m)
  \langle F, \varphi \rangle + \langle F, E \varphi \rangle = (a + m) \langle
  F, \varphi \rangle - \langle E F, \varphi \rangle - m \langle F \comma
  \varphi \rangle$ and \ this implies that $E F = a F$ on $\mathbbm{R}^m$
  (since $\varphi$ was arbitrary). This proves the ``$\Rightarrow$''.
  
  Now, assume $E F = a F$ holds on $\mathbbm{R}^m$ and let $V$ be as above and
  $\Phi : V \ni \alpha \mapsto \alpha^{a + m} \langle F, \varphi_{\alpha}
  \rangle \in \mathbbm{C}$ be as above. It suffices to show now that
  $(\partial / \partial \alpha) \Phi = 0$ on $V$. We first show that it holds
  for $\alpha = 1$. This is so, as $(\partial / \partial \alpha)
  \mid_{\alpha = 1} \Phi = a \langle F, \varphi \rangle - \langle E F,
  \varphi \rangle$ as shown above, and the latter is equal to 0 by hypothesis.
  
  Next, take arbitrary $\alpha_0 \in V$ and $V'$ neighborhood of $1$ small
  enough so that $V' \cdot \alpha_0 \subset V$. Then for $\Phi' : V' \ni \beta
  \mapsto \beta^{a + m} \langle F, (\varphi_{\alpha_0})_{\beta} \rangle$ we
  have $\Phi' (\beta) = \alpha_0^{- m - a} \Phi (\beta \alpha_0)$ and the
  previous paragraph now implies that $(\partial / \partial \beta)
  \mid_{\beta = 1} \Phi' = 0$ and hence that $(\partial / \partial \alpha)
  \mid_{\alpha = \alpha_0} \Phi = 0$ and this ends the proof.
\end{proof}

\begin{lemma}
  \label{lem67:lem-eveninall}eSuppose $X \subset \mathbbm{R}^{p, q}$ is stable
  under $(x_i)_{i = 1}^n \mapsto (\varepsilon_i x_i)_{i = 1}^n$ with $n
  \assign p + q$ and $u \in \mathcal{S} \tmop{ol} (X ; \lambda, \nu)$. Then
  $u$ is even in all coordinates.
\end{lemma}

\begin{proof}
  Evenness in $x_i$ with $i \neq p$ is readily given by item 2 of definition
  \ref{sol:def-sol}. As item 1 also implies $u$ being even in $X$, it should
  be also even in $x_p$ as well.
\end{proof}

\begin{lemma}
  \label{lem67:lem-Qpm}For $(\lambda, \nu) \in \mathbbm{C}^2$ and $p, q
  \geqslant 1$ we have:
  \[ \begin{array}{c}
       \mathcal{S} \tmop{ol} (\{ Q > 0 \} ; \lambda, \nu) =\mathbbm{C} \left\{
       \begin{array}{ll}
         Q_+^{- \nu} | x_p |^{\lambda + \nu - n}, & p = 1\\
         Q_+^{- \nu} \frac{| x_p |^{\lambda + \nu - n}}{\Gamma \left(
         \frac{\lambda + \nu - n + 1}{2} \right)}, & p \geqslant 2
       \end{array} \right.\\
       \mathcal{S} \tmop{ol} (\{ Q < 0 \} ; \lambda, \nu) =\mathbbm{C}Q_-^{-
       \nu} \frac{| x_p |^{\lambda + \nu - n}}{\Gamma \left( \frac{\lambda +
       \nu - n + 1}{2} \right)}
     \end{array} \]
\end{lemma}

\begin{remark}
  Note that when $p = 1$, $| x_p |^{\lambda + \nu - n} \in C^{\infty}$ on $\{
  Q > 0 \}$, as all points of latter set have $x_p > 0$.
\end{remark}

\begin{proof}
  $\supseteq$ is easy to see. Indeed, the direct verification of can be done
  in case $\tmop{Re} (\lambda + \nu) \gg 0$, when $| x_p |^{\lambda + \nu - n}
  \in C^1$ and then it can be seen for all $(\lambda, \nu) \in \mathbbm{C}^2$
  by proposition \ref{sol:prop-holocont} (distributions in the right-hand
  sides of formulae in statement depend on $(\lambda, \nu) \in \mathbbm{C}^2$
  holomorphically due to propositions
  \ref{holomorphicity-preserving:prop-tensor-holo} and
  \ref{holomorphicity-preserving:prop-pullback-holo}).
  
  Thus, it remain to show $\subseteq$. Note first that taking derivative of
  (\ref{eq-Nequiv}) one arrives at equations (\ref{Ndiff}) and these in turn
  imply that for $u \in \mathcal{S} \tmop{ol} (U ; \lambda, \nu)$ with $U
  \subset \{ Q \neq 0 \}$ (note that then $| Q |^{- \nu}$ is smooth nonzero on
  $U$) one has $\partial_i (| Q |^{\nu} u) = 0$ for $i \in \{ 1, 2, \ldots, n
  \} \backslash \{ p \}$. Now, lemmas \ref{lem67:lem-geom} and
  \ref{lem67:lem-eveninall} implies that lemma \ref{lem67:lem-tensor} is
  applicable (strictly speaking, we have to re-order coordinates $x_1, x_2,
  \ldots, x_p$ though, to make $x_p$ be the first one), and the latter tells
  us that $| Q |^{\nu} u \in \mathbbm{C}1_{\mathbbm{R}^{p - 1}} \otimes u_0
  (x_p) \otimes 1_{\mathbbm{R}^q}$ (where $1_U$ denotes constant 1
  distribution on $U$) with $u_0 \in \mathcal{D}' (\mathbbm{R}\backslash \{ 0
  \})$ (in case of $\{ Q > 0 \}$ and $p = 1$) or $u_0 \in \mathcal{D}'
  (\mathbbm{R})$ (otherwise) being even. Moreover, lemma
  \ref{lem67:lem-homogImpliesE} and ``moreover'' part of lemma
  \ref{lem67:lem-tensor} imply that $E u_0 = (\lambda - \nu - n) u_0$. Then,
  application of lemma \ref{lem67:lem-homogR} ends the proof.
\end{proof}

\begin{lemma}
  \label{lem67:lem-flip}For $p, q \in \mathbbm{Z}_{\geqslant 1}$ let $Q$:
  quadratic form on $\mathbbm{R}^{p, q}$. For $x, b \in \mathbbm{R}^{p, q}$ we
  let $c_b (x) \assign 1 - 2 Q (x, b) + Q (x) Q (b)$ and $\psi_b (x) \assign
  (x - Q (x) b) / c_b (x)$.
  
  We then have the following:
  \begin{enumerate}
    \item For $p \geqslant 1$ there exist $x^{(0)}, b^{(0)} \in
    \mathbbm{R}^{p, q}$ with $b^{(0)}_p = 0$, such that $Q (\psi_{b^{(0)}}
    (x^{(0)})) > 0$ and $Q (x^{(0)}) < 0$;
    
    \item For $p \geqslant 2$ we can in addition make $\psi_{b^{(0)}}
    (x^{(0)})$ having it's $p$-th coordinate vanish.
  \end{enumerate}
\end{lemma}

\begin{proof}
  For the first item, it suffices to find $x^{(0)}$ and $b^{(0)}$ such that $Q
  (x^{(0)}) < 0$ and $Q (x^{(0)} - Q (x^{(0)}) b^{(0)}) > 0$ (note that this
  will automatically grant $c_{b^{(0)}} (x^{(0)}) \neq 0$, as $Q (x - Q (x) b)
  = Q (x) (1 - 2 Q (x, b) + Q (x) Q (b))$). Now, by assuming $x_i^{(0)} =
  b^{(0)}_i = 0$ for $i \neq p, p + 1$ (while still requiring $b_p^{(0)} =
  0$), and thus we have $Q (x^{(0)}) = (x_p^{(0)})^2 - (x_{p + 1}^{(0)})^2$
  and $Q (x^{(0)} - Q (x^{(0)}) b^{(0)}) = (x_p^{(0)})^2 - (x_{p + 1}^{(0)} -
  ((x_p^{(0)})^2 - (x_{p + 1}^{(0)})^2) b^{(0)}_{p + 1})^2$. Hence, by taking
  arbitrary $x^{(0)}$ with $Q (x^{(0)}) < 0$ and $b_{p + 1}^{(0)} \assign x_{p
  + 1}^{(0)} / ((x_p^{(0)})^2 - (x_{p + 1}^{(0)})^2)$, we get the required
  pair of elements of $\mathbbm{R}^{p, q}$. This shows the first item.
  
  Regarding the second item, we proceed similarly, except that this time we
  require $x_i^{(0)} = 0$ for $i \neq p - 1, p + 1$, $b^{(0)}_i = 0$ for $i
  \neq p$ and replace $x_p^{(0)}$ by $x_{p - 1}^{(0)}$ everywhere in
  computations of previous paragraph. As $\psi_{b^{(0)}} (x^{(0)})$ and
  $x^{(0)}$ have their $p$-th component being the same, this construction
  suffices.
\end{proof}

\subsection{Proofs}

\begin{proof}
  (of prop. \ref{lem67:prop-67}) We note that $| Q |^{- \nu} \frac{| x_p
  |^{\lambda + \nu - n}}{\Gamma \left( \frac{\lambda + \nu - n + 1}{2}
  \right)} \in \mathcal{S} \tmop{ol} (\{ Q \neq 0 \} ; \lambda, \nu)$, as it
  is clearly so for $\tmop{Re} (\lambda + \nu) \gg 0$ (when it becomes $C^1$)
  by direct check, and it holomorphically depends on $(\lambda, \nu) \in
  \mathbbm{C}^2$ (we use propositions
  \ref{holomorphicity-preserving:prop-tensor-holo} and
  \ref{holomorphicity-preserving:prop-pullback-holo} to see this and then
  proposition \ref{sol:prop-holocont}). Now, in the light of lemma
  \ref{lem67:lem-Qpm} it suffices to show that if $u \in \mathcal{S} \tmop{ol}
  (\{ Q \neq 0 \} ; \lambda, \nu)$ and $u \mid_{\{ Q < 0 \}} = 0$, then $u
  = 0$.
  
  First, assume $p = 1$. Lemma \ref{lem67:lem-Qpm} tells us that if $u
  \mid_{\{ Q > 0 \}} \neq 0$, then it is supported on the whole $\{ Q > 0
  \}$. But then the first item of lemma \ref{lem67:lem-flip}, equation
  (\ref{eq-Nequiv}) and assumption $u \mid_{\{ Q < 0 \}}$ imply that near
  some $y \in \{ Q > 0 \}$ we have $u = 0$ and this contradiction gives the
  desired conclusion in case $p = 1$.
  
  Similarly, for $p \geqslant 2$ the second item of lemma \ref{lem67:lem-flip}
  equation (\ref{eq-Nequiv}) and assumption $u \mid_{\{ Q < 0 \}}$ imply
  that near some $y \in \{ Q > 0 \} \cap \{ x_p = 0 \}$ we have $u = 0$.
  But then lemma \ref{lem67:lem-Qpm} tells us that if $u \mid_{\{ Q > 0
  \}} \neq 0$, it has to be supported at least on $\{ x_p = 0 \}$. This
  contradiction gives the desired conclusion in case $p \geqslant 2$ as well.
\end{proof}

\section{The construction of regular symmetry breaking operator (B10)}\label{sec:supp-R}


In this section we will construct the distribution $\tilde{K}_{\lambda, \nu}^X
\in \mathcal{S} \tmop{ol} (\mathbbm{R}^n ; \lambda, \nu)$ which is holomorphic
in $(\lambda, \nu) \in \mathbbm{C}^2$ and vanishes on a discrete subset of
$\mathbbm{C}^2$.

Here, we let
\begin{eqnarray}
  & \mid \mid \assign \{ (\lambda, \nu) \in \mathbbm{C}^2 \mid \nu \in 1 +
  2\mathbbm{Z}_{\geqslant 0} \}, &  \nonumber\\
  & / / \assign \{ (\lambda, \nu) \in \mathbbm{C}^2 \mid \lambda - \nu \in -
  2\mathbbm{Z}_{\geqslant 0} \}, &  \nonumber\\
  & \backslash\backslash \assign \{ (\lambda, \nu) \in \mathbbm{C}^2 \mid
  \lambda + \nu - n + 1 \in - 2\mathbbm{Z}_{\geqslant 0} \}, &  \nonumber\\
  & \mid \mid \mid \assign \{ (\lambda, \nu) \in \mathbbm{C}^2 \mid \nu \in -
  2\mathbbm{Z}_{\geqslant 0} \cup (q + 1 + 2\mathbbm{Z}) \}, &  \nonumber
\end{eqnarray}
and the closed subsets of $\mathbbm{R}^n$:
\begin{eqnarray}
  & X \assign \mathbbm{R}^{p, q}, &  \nonumber\\
  & Y \assign \{ x \in \mathbbm{R}^{p, q} \mid x_p = 0 \}, &  \nonumber\\
  & C \assign \{ x \in \mathbbm{R}^{p, q} \mid Q (x) = 0 \} . &  \nonumber
\end{eqnarray}

\subsection{Main results}

\begin{proposition}
  \label{supp-R:prop-main}The generalized function
  \[ \tilde{K}_{\lambda, \nu}^X \assign \frac{| x_p |^{\lambda + \nu - n} | Q
     |^{- \nu}}{\Gamma \left( \frac{\lambda + \nu - n + 1}{2} \right) \Gamma
     \left( \frac{1 - \nu}{2} \right) \Gamma \left( \frac{\lambda - \nu}{2}
     \right)} \in \mathcal{D}' (\mathbbm{R}^{p, q}) \]
  defined initially by corresponding continuous function for $\tmop{Re} (-
  \nu), \tmop{Re} (\lambda + \nu - n) > 0$ holomorphically extends to
  $(\lambda, \nu) \in \mathbbm{C}^2$. It is a member of $\mathcal{S} \tmop{ol}
  (\mathbbm{R}^{p, q} ; \lambda, \nu)$ and we will denote the corresponding
  member of $\tmop{Hom}_{G'} (I (\lambda), J (\nu))$ by $R_{\lambda, \nu}^X$.
  We furthermore have
  \[ \tmop{supp} (\tilde{K}_{\lambda, \nu}^X) = \left\{ \begin{array}{ll}
       X, & (\lambda, \nu) \nin \mid \mid \cup \backslash\backslash \cup / /\\
       C, & (\lambda, \nu) \in \mid \mid -\backslash\backslash - / /\\
       Y, & (\lambda, \nu) \in \backslash\backslash - \mid \mid - / /\\
       \varnothing, & p = 1, \; (\lambda, \nu) \in \mid \mid \cap
       \backslash\backslash - / /\\
       Y \cap C, & p > 1, \; (\lambda, \nu) \in \mid \mid \cap
       \backslash\backslash - / /\\
       \varnothing, & (\lambda, \nu) \in / / \cap \mid \mid \mid\\
       \{ 0 \}, & (\lambda, \nu) \in / / - \mid \mid \mid .
     \end{array} \right. \]
\end{proposition}

\subsection{Auxiliary results}

\begin{lemma}
  \label{supp-R:lem-gelfand}Let $\Omega \subset \mathbbm{C}$ be an open set
  and $a : \Omega \rightarrow \mathbbm{C}$ be holo with nonvanishing
  derivative. Suppose that for $D \subset \Omega$ discrete we have
  $F_{\lambda} \in \mathcal{D}' (\mathbbm{R}^k)$ be holomorphic in $\lambda
  \in \Omega \backslash D$ and homogeneous of degree $a (\lambda)$ and $D =
  a^{- 1} (- k -\mathbbm{Z}_{\geqslant 0})$. Suppose further that $F_{\lambda}
  \mid_{\mathbbm{R}^k - \{ 0 \}}$ extends to holo in $\lambda \in \Omega$.
  
  Then, $\tilde{F}_{\lambda} : = \tilde{\rho} (F_{\lambda} \mid_{\mathbbm{R}^k
  - \{ 0 \}}) \assign F_{\lambda} / \Gamma (a (\lambda) + k)$ extends to
  holomorphic in $\lambda \in \Omega$. Moreover, for $\lambda \in D$ we
  have
  \begin{eqnarray}
    & \tilde{F}_{\lambda} = \sum_{| \alpha | = - a (\lambda) - k}
    c_{\alpha} \delta^{(\alpha)} &  \nonumber\\
    & c_{\alpha} \assign \langle F_{\lambda} |_{\mathbbm{S}^k}, x^{\alpha}
    \rangle &  \nonumber
  \end{eqnarray}
\end{lemma}

\begin{remark}
  This basically is a restatement of a discussion in
  {\cite[III.{\textsection}3.5]{gelfand1980distribution}}.
\end{remark}

\begin{proof}
  We first claim that $F_{\lambda} \mid_{\mathbbm{S}^k}$ is well-defined
  and holomorphic in $\lambda \in \Omega$. Indeed, under the assumption that
  $F_{\lambda} \mid_{\mathbbm{R}^k - \{ 0 \}}$ extends to holomorphic
  distribution in $\lambda \in \Omega$, we have similarly to the proof of
  lemma \ref{k-finite:lem-holo-easy} that $F_{\lambda}$ is holomorphic in
  $\mathcal{D}'_{\Gamma} (\mathbbm{R}^k - \{ 0 \})$ with $\Gamma$ being the
  cone $\{ (x, \xi) \in \mathbbm{R}^k \backslash \{ 0 \} \times \mathbbm{R}^k
  | x \perp \xi \}$ and thus proposition
  \ref{holomorphicity-preserving:prop-pullback-holo} implies the claim.
  
  Discreteness of $D$ allows us to consider matters locally near every point
  of $D$, thus we may assume $D = \{ \lambda_0 \}$. Moreover, as $a (\lambda)$
  has novanishing derivative at $\lambda_0$, we can biholomorphically change
  coordinates in parameter space and thus assume that $a (\lambda) = \lambda$
  and $\lambda_0 = - k - l \in - k -\mathbbm{Z}_{\geqslant 0}$. We next claim
  that near (but not equal to) $\lambda = \lambda_0$ we have
  \begin{eqnarray}
    & \langle \tilde{F}_{\lambda}, \varphi \rangle = \frac{1}{\Gamma (\lambda
    + k)} \langle r^{\lambda + k - 1}, u_{\lambda} \rangle, 
    \label{KR-normalization-even:eq-lemeq} & \\
    & u_{\lambda} (r) \assign \langle F_{\lambda} |_{\mathbbm{S}^{k - 1}},
    \varphi (r \cdot) \rangle . &  \nonumber
  \end{eqnarray}
  We note that $u \in C^{\infty}_0 (\mathbbm{R}_{\geqslant 0})$, as is easily
  seen by direct check. Now, both sides
  $(\ref{KR-normalization-even:eq-lemeq})$ are clearly homogeneous of degree
  $\lambda$, so it suffices (by fact
  \ref{holomorphicity-preserving:fact-homog}) to show that they coincide on
  $\mathbbm{R}^k \backslash \{ 0 \}$, where the equality is clear (passing to
  polar coordinates $F_{\lambda}$ becomes tensor product).
  
  We next show that $\tilde{F}_{\lambda}$ is holomorphic at $\lambda =
  \lambda_0$. We fix $\varphi \in C^{\infty}_0 (\mathbbm{R}^k)$ and it
  suffices to show the continuity of right-hand side of
  $(\ref{KR-normalization-even:eq-lemeq})$ in $\lambda$. So suppose that
  $\lambda_n \rightarrow \lambda_0 = - k - l$. Now, lemma
  \ref{supp-n-waves:lem-weakened-conv} implies that we just need to show that
  $u_{\lambda_n} \assign u_n \rightarrow u_0 \assign u_{\lambda_0}$ pointwise
  with all derivatives in $\mathbbm{R}_{\geqslant 0}$ and that $u_{\lambda_n}$
  have their derivatives are uniformly bounded in $n$. Now, fact
  \ref{holomorphicity-preserving:fact-basic} applied to $\psi :
  \mathbbm{R}_{\geqslant 0} \times \mathbbm{S}^{k - 1} \ni (r, \omega) \mapsto
  \varphi (r \omega)$ implies that (we let $f_{\lambda} \assign F_{\lambda}
  \mid_{\mathbbm{S}^{n - 1}}$ and $\varphi^{(\alpha)}$ denotes partial
  derivative)
  \[ \frac{d^m}{d r^m} u_{\lambda} (r) = \left\langle f_{\lambda},
     \frac{d^m}{d r^m} \varphi (r \cdot) \right\rangle = \left\langle
     f_{\lambda}, \sum_{| \alpha | = m} \omega^{\alpha} \varphi^{(\alpha)} (r
     \cdot) \right\rangle \]
  which readily settles the issue about the pointwise convergence of
  derivatives, so it only remains to show that these are uniformly bounded in
  $n$ and $x \in \mathbbm{R}_{\geqslant 0}$. The latter equality implies that
  it suffices to show this for the 0-th derivative. That is, we just need to
  show that $u_n$ are uniformly bounded in $r \in \mathbbm{R}_{\geqslant 0}$
  and $n$.
  
  Now, the third item of fact \ref{holomorphicity-preserving:fact-basic} tells
  us that
  \[ | \langle f_n, \varphi (r \cdot) \rangle | \leqslant C \sum_{i = 0}^k
     \sup | \partial^i \varphi (r \cdot) | = \sum_{i = 0}^k \sup \left(
     \sum_{| \alpha | = i} r^i \varphi^{(\alpha)} (r \cdot) \right) \]
  and as the right-hand side is uniformly bounded in $n$ and $r$, we are done
  with showing the holomorphicity of $\tilde{F}_{\lambda}$.
  
  Finally, we need to show what $\tilde{F}_{\lambda}$ becomes for $\lambda =
  \lambda_0$. As by hypothesis $F_{\lambda} \mid_{\{ x \neq 0 \}}$ is
  holomorphic at $\lambda_0$, we see that $F_{\lambda} / \Gamma (\lambda + k)
  \mid_{\{ x \neq 0 \}}$ necessary vanishes at $\lambda = \lambda_0$, thus
  $\tilde{F}_{\lambda_0}$ is supported only at $\{ 0 \}$ and therefore should
  be a finite sum of derivatives of delta functions. Moreover, as
  $\tilde{F}_{\lambda}$ is holomorphic in $\lambda$, it has to be homogeneous
  of degree $\lambda_0$ at $\lambda_0$ and therefore
  \[ \tilde{F}_{\lambda_0} = \sum_{| \alpha | = - \lambda_0 - k} c_{\alpha}
     \delta^{(\alpha)} . \]
  and
  \begin{eqnarray}
    & c_{\alpha} = (- 1)^{| \alpha |} \frac{\langle \tilde{F}_{\lambda_0},
    x^{\alpha} \rangle}{| \alpha | !} = (- 1)^{| \alpha |} \frac{\langle
    \delta^{- k - \lambda_0}, u \rangle}{(- k - \lambda_0) !} = &  \nonumber
  \end{eqnarray}
  where $u = \langle f_{\lambda_0}, x^{\alpha} \rangle = \langle
  f_{\lambda_0}, r^{- \lambda_0 - k} \omega^{\alpha} \rangle = r^{- \lambda_0
  - k} \langle f_{\lambda_0}, \omega^{\alpha} \rangle$ (where $\omega \assign
  x \mid_{\mathbbm{S}^{n - 1}}$) and thus we can continue
  \[ = \langle f_{\lambda_0}, \omega^{\alpha} \rangle (- 1)^{| \alpha |}
     \frac{\langle \delta^{- k - \lambda_0}, r^{- \lambda_0 - k} \rangle}{(- k
     - \lambda_0) !} = \langle f_{\lambda_0}, \omega^{\alpha} \rangle, \]
  and we are done.
\end{proof}

\begin{lemma}
  \label{supp-R:lem-supp-of-K}The distribution $K_{\lambda, \nu}^X \in
  \mathcal{D}' (\mathbbm{R}^n - \{ 0 \})$, defined as an proof of proposition
  \ref{supp-R:prop-main}, has the support being equal to
  \[ \tmop{supp} (K_{\lambda, \nu}^X) = (\mathbbm{R}^n - \{ 0 \}) \cap \left\{
     \begin{array}{ll}
       \mathbbm{R}^n, & (\lambda, \nu) \in (\backslash\backslash \cup \mid
       \mid)^c,\\
       \{ x_p = 0 \}, & (\lambda, \nu) \in \backslash\backslash - \mid \mid,\\
       \{ Q = 0 \}, & (\lambda, \nu) \in \mid \mid -\backslash\backslash,\\
       \{ x_p = 0 \} \cap \{ Q = 0 \}, & p > 1, (\lambda, \nu) \in
       \backslash\backslash \cap \mid \mid,\\
       \varnothing, & p = 1, (\lambda, \nu) \in \backslash\backslash \cap \mid
       \mid .
     \end{array} \right. \]
\end{lemma}

\begin{proof}
  First of all, for $(\lambda, \nu) \in (\backslash\backslash \cup \mid
  \mid)^c$ we have that both $| x_p |^{\lambda + \nu - n} / \Gamma \left(
  \frac{\lambda + \nu - n + 1}{2} \right), | Q |^{- \nu} / \Gamma \left(
  \frac{1 - \nu}{2} \right)$ are smooth nonzero when restricted to $\{ Q \neq
  0, x_p \neq 0 \}$ and hence by lemma
  \ref{KR-normalization-recur:lem-mult-smth}, so is the restriction of
  $K_{\lambda, \nu}^X$ to $\{ Q \neq 0, x_p \neq 0 \}$, hence in this case
  $\tmop{supp} (K_{\lambda, \nu}^X) \supset \{ Q \neq 0, x_p \neq 0 \}$, hence
  the statement.
  
  Next, for $(\lambda, \nu) \in \mid \mid -\backslash\backslash$ we have $| Q
  |^{- \nu} / \Gamma \left( \frac{1 - \nu}{2} \right)$ being zero on $\{ Q
  \neq 0 \}$, hence again by lemma \ref{KR-normalization-recur:lem-mult-smth},
  we have $\tmop{supp} (K_{\lambda, \nu}^X) \subset \{ Q = 0 \}$. On the other
  hand, we have $| x_p |^{\lambda + \nu - n} / \Gamma \left( \frac{\lambda +
  \nu - n + 1}{2} \right)$ being smooth nonzero when restricted to $\{ x_p
  \neq 0 \}$, hence by lemma \ref{KR-normalization-recur:lem-mult-smth} we
  conclude that $K_{\lambda, \nu}^X$ is nonzero at the points of $\{ Q = 0,
  x_p \neq 0 \}$, hence $\{ Q = 0 \} \subset \tmop{supp} (K_{\lambda, \nu}^X)
  \subset \{ Q = 0, x_p \neq 0 \}$, hence the conclusion. Similarly, the case
  $(\lambda, \nu) \in \backslash\backslash - \mid \mid$ is handled.
  
  So, finally, we assume $(\lambda, \nu) \in \mid \mid \cap
  \backslash\backslash$. When $p = 1$ we know that $\{ x_p = 0 \} \cap \{ Q =
  0 \} = \{ 0 \}$, hence (as $K_{\lambda, \nu}^X \in \mathcal{D}'
  (\mathbbm{R}^n - \{ 0 \})$), we conclude by lemma
  \ref{KR-normalization-recur:lem-mult-smth} that $K_{\lambda, \nu}^X$ is zero
  on $\{ x_p \neq 0 \}$ and on $\{ Q \neq 0 \}$, hence the conclusion.
  
  Finally, we proceed to $p > 1$ case. Logic same as that of previous
  paragraph implies that $\tmop{supp} (K_{\lambda, \nu}^X) \subset \{ Q = 0,
  x_p = 0 \}$, so we only need to show the converse inclusion. Let's take an
  arbitrary point $x_0 \in \{ Q = 0 \} \cap \{ x_p = 0 \}$ (this set is
  nonempty under the $p \neq 1$ condition). It is seen by the direct check,
  that we have normals of $\{ Q = 0 \}$ and $\{ x_p = 0 \}$ being independent
  at $x_0$, hence we may introduce local coordinate system $y (x) = (y_1 (x),
  y_2 (x), \ldots, y_n (x)) \in \Pi_i U_i \subset \Pi_i \mathbbm{R}$ near
  $x_0$, so that $y_1 (x) = x_p$ and $y_2 (x) = Q (x)$.
  
  Then, recalling the way pullback was defined in
  {\cite{hormander1983analysis}}, we see that $| x_p |^{\lambda + \nu - n} /
  \Gamma ((\lambda + \nu - n + 1) / 2)$ and $| Q |^{- \nu} / \Gamma ((1 - \nu)
  / 2)$ have expressions (proportional to) $\delta^a (y_1) \otimes 1_{\Pi_{i
  \geqslant 2} U_i}$ and $1_{U_1} \otimes \delta^b (y_2) \otimes 1_{\Pi_{i
  \geqslant 3} U_i}$ respectively with $a \assign - (\lambda + \nu - n + 1) /
  2$ and $b \assign - (- \nu + 1) / 2$. Then, applying lemma
  \ref{KR-normalization-recur:lem-mult-distrib-tensor} with $(X, Y) : = U_1,
  \Pi_{i \geqslant 2} U_i$ and $\Gamma_2^1 = \Gamma_1^2 = \varnothing$;
  $\Gamma^1_1$ and $\Gamma_2^2$ full in $U_1$ and $\Pi_{i \geqslant 2} U_i$
  respectively, we have that $K_{\lambda, \nu}^{\mathbbm{R}^n}$ has in $y (x)$
  expression $(\delta^a (y_1) \cdot 1_{U_1}) \otimes (1_{\Pi_{i \geqslant 2}
  U_i} \cdot \delta^b (y_2) \otimes 1_{\Pi_{i \geqslant 3} U_i})$. Finally,
  lemma \ref{KR-normalization-recur:lem-mult-smth} tells us that latter equals
  to $\delta^a (y_1) \otimes (\delta^b (y_2) \otimes 1_{\Pi_{i \geqslant 3}
  U_i})$. But the latter is tensor product of nonzero distributions, hence
  nonzero. This shows that under the hypothesis taken $K_{\lambda,
  \nu}^{\mathbbm{R}^n}$ is supported at least on $\{ Q = 0 \} \cap \{ x_p = 0
  \}$.
\end{proof}

\begin{lemma}
  \label{supp-R:lem-Kzero}Suppose $(\lambda, \nu) \nin / /$ and
  $\tilde{K}_{\lambda, \nu}^X$ is defined as in the proof of proposition
  \ref{supp-R:prop-main}. Then $\tilde{K}_{\lambda, \nu}^X = 0$ iff $(\lambda,
  \nu) \in \mid \mid \mid$.
\end{lemma}

\begin{proof}
  We assume $\lambda - \nu = - 2 k \in - 2\mathbbm{Z}_{\geqslant 0}$. The way
  $\tilde{K}_{\lambda, \nu}^X$ is defined together with the lemma
  \ref{supp-R:lem-gelfand} imply that $\tilde{K}_{\lambda, \nu}^X = 0$ iff
  \[ \forall \gamma \in \mathbbm{Z}_{\geqslant 0}^n, | \gamma | = 2 k, \quad
     \left\langle \frac{| Q |^{- \nu}}{\Gamma \left( \frac{1 - \nu}{2}
     \right)} \cdot \frac{| x_p |^{\lambda + \nu - n}}{\Gamma \left(
     \frac{\lambda + \nu - n + 1}{2} \right)} \mid_{\mathbbm{S}^{n - 1}},
     \omega^{\gamma} \right\rangle = 0. \]
  Thus, explicit formula for $\langle | Q |^{- \nu} | x_p |^{\lambda + \nu -
  n}, \omega^{\gamma} \rangle_{\mathbbm{S}^{n - 1}}$ would come handy and we
  now proceed with giving such a formula.
  
  One notes that for $\tmop{Re} (- \nu), \tmop{Re} (\lambda + \nu) \gg 0$ one
  has by passing to bipolar coordinates $x = (r \omega, s \omega') \in
  \mathbbm{R}^p \times \mathbbm{R}^q$ (and splitting multiindex $\gamma \in
  \mathbbm{Z}_{\geqslant 0}^n$ as $\gamma = : (\alpha, \beta)$ accordingly)
  that
  \begin{eqnarray}
    & \langle | Q |^{- \nu} | x_p |^{\lambda + \nu - n}, x^{\gamma}
    \rangle_{\mathbbm{S}^{n - 1}} = \int_{\mathbbm{S}^{p - 1}} | \omega_p
    |^{\lambda + \nu - n} \omega^{\alpha_p}_p \tilde{\omega}^{\tilde{\alpha}}
    d \omega \times \int_{\mathbbm{S}^{q - 1}} (\omega')^{\beta} d \omega'
    \times &  \nonumber\\
    & \int_{r^2 + s^2 = 1 ; r, s > 0} | r^2 - s^2 |^{- \nu} r^{\lambda + \nu
    - n} r^{p + a - 1} s^{q + b - 1} d r d s, &  \nonumber
  \end{eqnarray}
  where we split multiindex $\alpha \in \mathbbm{Z}^p_{\geqslant 0}$ as
  $\alpha = : (\tilde{\alpha}, \alpha_p)$ and split $\omega \in \mathbbm{S}^{p
  - 1}$ as $\omega = : (\tilde{\omega}, \omega_p)$. One further notes that
  $\int_{\mathbbm{S}^{q - 1}} (\omega')^{\beta} d \omega' = 0$ unless $| \beta
  | \assign \sum_j \beta_j \in 2\mathbbm{Z}$ and thus we may in subsequent
  assume that this is so. Hence, (as $| \gamma | = | \alpha | + | \beta | = 2
  k$) we can restrict ourselves to situation $| \alpha | = : a \in
  2\mathbbm{Z}_{\geqslant 0}$ and $| \beta | = : b \in 2\mathbbm{Z}_{\geqslant
  0}$. Then, we may continue as
  \begin{eqnarray}
    & \simeq \int_{\mathbbm{S}^{p - 1}} | \omega_p |^{\lambda + \nu - n}
    \omega^{\alpha_p}_p \tilde{\omega}^{\tilde{\alpha}} d \omega \int_{r^2 +
    s^2 = 1 ; r, s > 0} | r^2 - s^2 |^{- \nu} r^{\lambda + \nu - n} r^{p + a -
    1} s^{q + b - 1} d r d s \simeq &  \nonumber
  \end{eqnarray}
  Now, inspecting the expression for the volume element of $\mathbbm{S}^{p -
  1}$ sphere, one concludes that for $p > 1$
  \[ \int_{\mathbbm{S}^{p - 1}} | \omega_p |^{\lambda + \nu - n}
     \omega^{\alpha_p}_p \tilde{\omega}^{\tilde{\alpha}} d \omega =
     \int_0^{\pi} | \cos \varphi |^{\lambda + \nu - n} \cos^{\alpha_p}
     \varphi^{} \cdot \sin^{p - 2} \varphi \left[ \int_{\mathbbm{S}^{p - 2}}
     \tilde{\omega}^{\tilde{\alpha}} d \tilde{\omega} \right] d \varphi \]
  and as $\int_{\mathbbm{S}^{p - 2}} \tilde{\omega}^{\widetilde{\alpha}_p}
  d \tilde{\omega} = 0$ unless $| \tilde{\alpha}_p | \in 2\mathbbm{Z}$, we can
  assume in subsequent that this is so, hence $\alpha_p \in 2\mathbbm{Z}$.
  Whereas for $p = 1$ we have
  \[ \int_{\mathbbm{S}^{p - 1}} | \omega_p |^{\lambda + \nu - n}
     \omega^{\alpha_p}_p \tilde{\omega}^{\tilde{\alpha}} d \omega = \left\{
     \begin{array}{ll}
       0, & \alpha_p \nin 2\mathbbm{Z}\\
       \tmop{const} \neq 0, & \alpha_p \in 2\mathbbm{Z}
     \end{array} \right. \]
  Moreover, using the formula for integrating along the curve $y = \sqrt{1 -
  t^2}$ in $\mathbbm{R}^2$, one sees that
  \[ \int_{r^2 + s^2 = 1 ; r, s > 0} | r^2 - s^2 |^{- \nu} r^{\lambda + \nu -
     n} r^{p + a - 1} s^{q + b - 1} d r d s = \int_0^1 | 1 - 2 t^2 |^{- \nu}
     \frac{1}{\sqrt{1 - t^2}} t^{\lambda + \nu - n + p + a - 1} \sqrt{1 -
     t^2}^{q + b - 1} d t \]
  and therefore we can continue the chain of equalities above in case $p > 1$
  (case $p = 1$ is handled similarly) as
  \begin{eqnarray}
    & \simeq \int_0^{\pi} | \cos \varphi |^{\lambda + \nu - n}
    \cos^{\alpha_p} \varphi^{} \cdot \sin^{p - 2} \varphi d \varphi \int_0^1 |
    1 - 2 t^2 |^{- \nu} \frac{1}{\sqrt{1 - t^2}} t^{\lambda + \nu + a - n + p
    - 1} \sqrt{1 - t^2}^{q + b - 1} d t = &  \nonumber\\
    & = \int_{- 1}^1 | t |^{\lambda + \nu - n} t^{\alpha_p} (1 - t^2)^{(p -
    3) / 2} d t \int_0^1 | 1 - 2 t |^{- \nu} (1 - t)^{(q + b - 2) / 2}
    t^{(\lambda + a + \nu - q) / 2 - 1} d t \simeq &  \nonumber
  \end{eqnarray}
  and as we assume that $\alpha_p \in 2\mathbbm{Z}$, we have that
  \[ \int_{- 1}^1 | t |^{\lambda + \nu - n} t^{\alpha_p} (1 - t^2)^{(p - 3) /
     2} d t = 2 \int_0^1 t^{\lambda + \nu - n + \alpha_p} (1 - t^2)^{(p - 3) /
     2} d t = 2 \int_0^1 t^{\frac{\lambda + \nu - n + \alpha_p - 1}{2}} (1 -
     t)^{\frac{p - 3}{2}} d t \]
  thus we can continue as (using the expression for beta function)
  \begin{eqnarray}
    & \simeq \int_0^1 t^{\frac{\lambda + \nu - n + \alpha_p - 1}{2}} (1 -
    t)^{\frac{p - 3}{2}} d t \int_0^1 | 1 - 2 t |^{- \nu} (1 - t)^{(q + a - 2)
    / 2} t^{(\lambda + b + \nu - q) / 2 - 1} d t \simeq &  \nonumber\\
    & \simeq \frac{\Gamma \left( \frac{\lambda + \nu - n + \alpha_p + 1}{2}
    \right)}{\Gamma \left( \frac{\lambda + \nu - q + \alpha_p}{2} \right)}
    \int_{- 1}^1 | w |^{- \nu} (1 - w)^{(\lambda + a + \nu - q) / 2 - 1} (1 +
    w)^{(q + b - 2) / 2} d w \simeq &  \nonumber
  \end{eqnarray}
  and accidentally the latter expression holds also for $p = 1$. Now, using
  the (valid for regular enough values of parameters) formula $\int_0^1 (1 +
  t)^{- a} (1 - t)^{c - 1} t^{b - 1} d t =_2 F_1 (a, b ; b + c ; - 1) B (b,
  c)$ we arrive at the expression
  \begin{eqnarray}
    & \int_{- 1}^1 | w |^{- \nu} (1 - w)^{(\lambda + a + \nu - q) / 2 - 1} (1
    + w)^{(q + b - 2) / 2} d w = &  \nonumber\\
    & = \int_0^1 w^{- \nu} (1 - w)^{(\lambda + a + \nu - q) / 2 - 1} (1 +
    w)^{(q + b - 2) / 2} d w + \int_0^1 w^{- \nu} (1 + w)^{(\lambda + \nu + a
    - q) / 2 - 1} (1 - w)^{(q + b - 2) / 2} d w = &  \nonumber\\
    & =_2 F_1 \left( 1 - \frac{q + b}{2}, 1 - \nu ; \frac{\lambda - \nu - q +
    a}{2} + 1 ; - 1 \right) B \left( 1 \um \nu, \frac{\lambda + \nu - q +
    a}{2} \right) + &  \nonumber\\
    & _2 F_1 \left( 1 - \frac{\lambda + \nu - q + a}{2}, 1 - \nu ; \frac{q +
    b}{2} - \nu + 1 ; - 1 \right) B \left( 1 - \nu, \frac{q + b}{2} \right) & 
    \nonumber
  \end{eqnarray}
  
  
  Now, as we assume that $\lambda - \nu = - 2 k = - (a + b)$, we conclude that
  both hypergeometric functions in the previous expression have their first
  and third arguments being equal, and as we have an equality (seen for
  example by recalling power series expansion of $_2 F_1$), $_2 F_1 (a, b ; a,
  - 1) = 2^{- b}$, we can continue the above chain of equalities as
  \begin{eqnarray}
    & \simeq \frac{\Gamma \left( \frac{\lambda + \nu - n + \alpha_p + 1}{2}
    \right)}{\Gamma \left( \frac{\lambda + \nu - q + \alpha_p}{2} \right)}
    \left( B \left( 1 \um \nu, \frac{\lambda + \nu - q + a}{2} \right) + B
    \left( 1 - \nu, \frac{q + b}{2} \right) \right) = &  \nonumber\\
    & = \frac{\Gamma \left( \frac{\lambda + \nu - n + \alpha_p + 1}{2}
    \right)}{\Gamma \left( \frac{\lambda + \nu - q + \alpha_p}{2} \right)}
    \Gamma (1 - \nu) \left\{ \frac{\Gamma \left( \frac{\lambda + \nu - q +
    a}{2} \right)}{\Gamma \left( 1 + \frac{\lambda - \nu - q + a}{2} \right)}
    + \frac{\Gamma \left( \frac{q + b}{2} \right)}{\Gamma \left( 1 - \nu +
    \frac{q + b}{2} \right)} \right\} = &  \nonumber
  \end{eqnarray}
  substituting $\lambda = \nu - 2 k$ and $b = 2 k - a$ the above gets
  rewritten as\quad
  \[ \frac{\Gamma \left( \frac{2 \nu - 2 k - n + \alpha_p + 1}{2}
     \right)}{\Gamma \left( \frac{2 \nu - 2 k - q + \alpha_p}{2} \right)}
     \Gamma (1 - \nu) \left\{ \frac{\Gamma \left( \frac{2 \nu - 2 k - q +
     a}{2} \right)}{\Gamma \left( 1 + \frac{- 2 k - q + a}{2} \right)} +
     \frac{\Gamma \left( \frac{q + 2 k - a}{2} \right)}{\Gamma \left( 1 - \nu
     + \frac{q + 2 k - a}{2} \right)} \right\} . \]
  The reasoning at the beginning of the proof now suggests that it suffices to
  find
  \begin{eqnarray}
    & \max \left\{ \mathfrak{P} \left( \frac{\Gamma \left( \frac{2 \nu - 2
    k - n + \alpha_p + 1}{2} \right)}{\Gamma \left( \frac{2 \nu - 2 k - q +
    \alpha_p}{2} \right)} \times \right. \right. &  \nonumber\\
    & \times \left. \left. \frac{\Gamma (1 - \nu)}{\Gamma \left( \frac{1 -
    \nu}{2} \right) \Gamma \left( \frac{\lambda + \nu - n + 1}{2} \right)}
    \left\{ \frac{\Gamma \left( \frac{2 \nu - 2 k - q + a}{2} \right)}{\Gamma
    \left( 1 + \frac{- 2 k - q + a}{2} \right)} + \frac{\Gamma \left( \frac{q
    + 2 k - a}{2} \right)}{\Gamma \left( 1 - \nu + \frac{q + 2 k - a}{2}
    \right)} \right\} \right) \right\}_{(\alpha_p, a) \in \mathfrak{I}} & 
    \nonumber
  \end{eqnarray}
  where $\mathfrak{I} \assign \{ (\alpha_p, a) \in (2\mathbbm{Z}_{\geqslant
  0})^2 | \alpha_p \leqslant a \leqslant 2 k \}$. Now, using the formulae
  $\Gamma (1 - z) \Gamma (z) = \pi / \sin (\pi z)$ and $\sin (\alpha) + \sin
  (\beta) = 2 \sin ((\alpha + \beta) / 2) \cos ((\alpha - \beta) / 2)$, we
  obtain
  \begin{eqnarray}
    & \frac{\Gamma \left( \frac{2 \nu - 2 k - q + a}{2} \right)}{\Gamma
    \left( 1 + \frac{- 2 k - q + a}{2} \right)} + \frac{\Gamma \left( \frac{q
    + 2 k - a}{2} \right)}{\Gamma \left( 1 - \nu + \frac{q + 2 k - a}{2}
    \right)} \simeq \frac{\sin \frac{\pi \nu}{2} \cos \left[ \pi \left( x -
    \frac{\nu}{2} \right) \right]}{\Gamma (1 - x) \Gamma (1 + x - \nu) \sin
    [\pi (\nu - x)] \sin (x)} \simeq &  \nonumber\\
    & \simeq \frac{\sin \frac{\pi \nu}{2} \cos \left[ \pi \left( x -
    \frac{\nu}{2} \right) \right]}{\Gamma (1 + x - \nu) \sin [\pi (\nu - x)]}
    \simeq \sin \frac{\pi \nu}{2} \cos \left[ \pi \left( x - \frac{\nu}{2}
    \right) \right] \Gamma (\nu - x) \quad, \; x \assign \frac{q}{2} + k -
    \frac{a}{2} > 0. &  \nonumber
  \end{eqnarray}
  Thus we see that it is sufficient to compute
  \begin{eqnarray}
    & \max \{ A_{\alpha_p, a} + B_{\alpha_p, a} + C_{\alpha_p, a} +
    D_{\alpha_p, a} \}_{(\alpha_p, a) \in \mathfrak{I}} &  \nonumber\\
    & A_{\alpha_p, a} \assign \mathfrak{P} \left( \frac{\Gamma (1 -
    \nu)}{\Gamma \left( \frac{1 - \nu}{2} \right)} \right) = \{ \nu \in
    2\mathbbm{Z}_{> 0} \} &  \nonumber\\
    & B_{\alpha_p, a} \assign \mathfrak{P} \left( \frac{\Gamma \left( \frac{2
    \nu - 2 k - n + \alpha_p + 1}{2} \right)}{\Gamma \left( \frac{2 \nu - 2 k
    - n + 1}{2} \right)} \right) = - \left\{ \nu - k + \frac{1 - n}{2} = 1 -
    \frac{\alpha_p}{2}, 2 - \frac{\alpha_p}{2}, \ldots, 0 \right\} & 
    \nonumber\\
    & C_{\alpha_p, a} \assign \mathfrak{P} \left( \frac{\Gamma \left( \nu -
    \frac{q}{2} - k + \frac{a}{2} \right)}{\Gamma \left( \nu - \frac{q}{2} - k
    + \frac{\alpha_p}{2} \right)} \right) = - \left\{ \nu - \frac{q}{2} - k +
    \frac{\alpha_p}{2} = 1 - \frac{a - \alpha_p}{2}, 2 - \frac{a -
    \alpha_p}{2}, \ldots, 0 \right\} &  \nonumber\\
    & D_{\alpha_p, a} \assign \mathfrak{P} \left( \sin \frac{\pi \nu}{2} \cos
    \left[ \pi \left( \frac{q + 2 k - a}{2} - \frac{\nu}{2} \right) \right]
    \right) = - \{ \nu \in 2\mathbbm{Z} \} - \{ \nu \in 1 + q + 2 k - a +
    2\mathbbm{Z} \} &  \nonumber
  \end{eqnarray}
  hence
  \[ \max \{ A_{\alpha_p, a} + B_{\alpha_p, a} + C_{\alpha_p, a} +
     D_{\alpha_p, a} \}_{(\alpha_p, a) \in \mathfrak{I}} = - \{ \nu \in
     2\mathbbm{Z}_{< 0} \} - \{ \nu \in 1 + q + 2\mathbbm{Z} \} \]
  and this directly implies the desired result.
\end{proof}

\subsection{Proofs}

\begin{proof}
  (of prop. \ref{supp-R:prop-main}) Using fact
  \ref{holomorphicity-preserving:fact-pullback}, we define distributions $|
  x_p |^{\lambda + \nu - n} / \Gamma \left( \frac{\lambda + \nu - n + 1}{2}
  \right), | Q |^{- \nu} / \Gamma \left( \frac{1 - \nu}{2} \right) \in
  \mathcal{D}' (\mathbbm{R}^n - \{ 0 \})$ as a pullback of $| x |^{\mu} /
  \Gamma \left( \frac{\mu + 1}{2} \right) \in \mathcal{D}' (\mathbbm{R})$ via
  smooth $\mathbbm{R}^n - \{ 0 \} \rightarrow \mathbbm{R}$ submersions given
  by $x \mapsto x_p, x \rightarrow Q (x)$ respectively. Now, proposition
  \ref{holomorphicity-preserving:prop-pullback-holo} and lemma
  \ref{supp-n-waves:lem-|x|-holo-in} tell us that $| x_p |^{\lambda + \nu - n}
  / \Gamma \left( \frac{\lambda + \nu - n + 1}{2} \right), | Q |^{- \nu} /
  \Gamma \left( \frac{1 - \nu}{2} \right) \in \mathcal{D}' (\mathbbm{R}^n - \{
  0 \})$ are holomorphic in $\mathcal{D}'_{\Gamma_1} (\mathbbm{R}^n - \{ 0
  \}), \mathcal{D}'_{\Gamma_2} (\mathbbm{R}^n - \{ 0 \})$ respectively where
  $\Gamma_1, \Gamma_2$ are normal bundles of $\{ x_p = 0 \}, \{ Q = 0 \}$
  respectively. Now, propositions
  \ref{holomorphicity-preserving:prop-tensor-holo},
  \ref{holomorphicity-preserving:prop-pullback-holo} imply that the product
  (see {\cite{hormander1983analysis}})
  \[ K_{\lambda, \nu}^X \assign \frac{| Q |^{- \nu}}{\Gamma \left( \frac{1 -
     \nu}{2} \right)} \cdot \frac{| x_p |^{\lambda + \nu - n}}{\Gamma \left(
     \frac{\lambda + \nu - n + 1}{2} \right)} \in \mathcal{D}' (\mathbbm{R}^n
     - \{ 0 \}) \]
  is holomorphic in $(\lambda, \nu) \in \mathbbm{C}^2$. Furthermore,
  propositions \ref{holomorphicity-preserving:prop-pullback-cts} and
  \ref{holomorphicity-preserving:prop-tensor-cts} imply that for the open
  region
  \[ \Omega \assign \{ (\lambda, \nu) \in \mathbbm{C}^2 \mid \tmop{Re} (-
     \nu), \tmop{Re} (\lambda + \nu - n) > 0 \} \]
  the latter distribution coincides with the corresponding continuous
  function. As the latter is homogeneous of degree $\lambda - \nu - n$, so
  should be $K_{\lambda, \nu}^X$ and as the latter is holomorphic, it is
  homogeneous not only on $\Omega$, but for the whole $\mathbbm{C}^2$. Next,
  fact \ref{holomorphicity-preserving:fact-homog} implies that for $(\lambda,
  \nu) \nin / /$ $K_{\lambda, \nu}^X$ extends to the homogeneous distribution
  $\rho (K_{\lambda, \nu}^X) \in \mathcal{D}' (\mathbbm{R}^n)$, which is also
  holomorphic in $(\lambda, \nu) \nin / /$ by proposition
  \ref{holomorphicity-preserving:prop-homog-holo} and coincides with the
  corresponding conituous function by proposition
  \ref{holomorphicity-preserving:prop-homog-cts}.
  
  Now, as it is clear (by direct check) that for $(\lambda, \nu) \in \Omega$
  the continuous function $| Q |^{- \nu} | x_p |^{\lambda + \nu - n}$ is a
  member of $\mathcal{S} \tmop{ol} (\mathbbm{R}^n ; \lambda, \nu)$, as $| Q
  |^{- \nu} | x_p |^{\lambda + \nu - n}$ is proportional to $\rho (K_{\lambda,
  \nu}^X)$ for $(\lambda, \nu) \in \Omega$ by previous paragraph, we have
  $\forall (\lambda, \nu) \in \Omega, \rho (K_{\lambda, \nu}^X) \in
  \mathcal{S} \tmop{ol} (\mathbbm{R}^n ; \lambda, \nu)$. Hence, by proposition
  \ref{sol:prop-holocont}, we have $\forall (\lambda, \nu) \nin / /, \rho
  (K_{\lambda, \nu}^X) \in \mathcal{S} \tmop{ol} (\mathbbm{R}^n ; \lambda,
  \nu)$. Finally, by lemma \ref{supp-R:lem-gelfand}, $\tilde{K}_{\lambda,
  \nu}^X \assign \rho (K_{\lambda, \nu}^X) / \Gamma \left( \frac{\lambda -
  \nu}{2} \right)$ (equal to that of the proposition statement by the
  uniqueness of holomorphic continuation) is holomorphic in $(\lambda, \nu)
  \in \mathbbm{C}^2$, and hence is a member of $\mathcal{S} \tmop{ol}
  (\mathbbm{R}^n ; \lambda, \nu)$, again by proposition
  \ref{sol:prop-holocont}. This proves all of the proposition, except for the
  last information on $\tmop{supp} (\tilde{K}_{\lambda, \nu}^X)$.
  
  We first suppose $(\lambda, \nu) \nin / /$, so that $\tilde{K}_{\lambda,
  \nu}^X$ is proportional to $\rho (K_{\lambda, \nu}^X)$ and as (by uniqueness
  part of fact \ref{holomorphicity-preserving:fact-homog} and closedness of
  support) we have
  \[ \tmop{supp} (\rho (F)) = \left\{ \begin{array}{ll}
       \varnothing, & F = 0\\
       \tmop{supp} (F) \cup \{ 0 \} & F \neq 0
     \end{array} \right., \]
  it suffices to compute the support of $K_{\lambda, \nu}^X \in \mathcal{D}'
  (\mathbbm{R}^n - \{ 0 \})$, which is in turn given by lemma
  \ref{supp-R:lem-supp-of-K}.
  
  Finally, we assume $(\lambda, \nu) \in / /$, where lemma
  \ref{supp-R:lem-gelfand} implies that $\tmop{supp} (\tilde{K}_{\lambda,
  \nu}^X) \in \{ 0 \}$, so it suffices to know whether $\tilde{K}_{\lambda,
  \nu}^X$ is zero or not, the information provided by lemma
  \ref{supp-R:lem-Kzero}. This ends the proof.
\end{proof}

\section{Construction of singular symmetry breaking operators (B11)}\label{sec:supp-sing}


In this section we construct three more families of SBO that are
holomorphically dependent on parameters $(\lambda, \nu)$ lying $\mid \mid
\mid, \backslash\backslash, \mid \mid$.

\subsection{Main results}

\begin{proposition}
  \label{supp-sing:prop-supp-mmm}Suppose $(\lambda, \nu) \in \mid \mid \mid$
  and define $\widetilde{\tilde{K}}_{\lambda, \nu}^X$ as
  \[ \widetilde{\tilde{K}}_{\lambda, \nu}^X \assign \Gamma \left(
     \frac{\lambda - \nu}{2} \right) \tilde{K}_{\lambda, \nu}^X . \]
  We then have $\widetilde{\tilde{K}}_{\lambda, \nu}^X \in
  \mathcal{S} \tmop{ol} (\mathbbm{R}^n ; \lambda, \nu)$ being holomorphic in $
  (\lambda, \nu) \in \mid \mid \mid$ and moreover
  \[ \tmop{supp} (\widetilde{\tilde{K}}_{\lambda, \nu}^Y) = \left\{
     \begin{array}{lll}
       X, &  & (\lambda, \nu) \nin \mid \mid \cup \backslash\backslash\\
       C, &  & (\lambda, \nu) \in \mid \mid -\backslash\backslash\\
       Y, &  & (\lambda, \nu) \in \backslash\backslash - \mid \mid\\
       \varnothing, & p = 1, \; & (\lambda, \nu) \in \mid \mid \cap
       \backslash\backslash - / /\\
       \{ 0 \}, & p = 1, & (\lambda, \nu) \in \mid \mid \cap
       \backslash\backslash \cap / /\\
       Y \cap C, & p > 1, & (\lambda, \nu) \in \mid \mid \cap
       \backslash\backslash .
     \end{array} \right. \]
\end{proposition}

\begin{proposition}
  \label{supp-sing:prop-supp-Y}Suppose $(\lambda, \nu) \in
  \backslash\backslash$ and define $\tilde{K}_{\lambda, \nu}^Y$ as
  \[ \tilde{K}^Y_{\lambda, \nu} \assign \tilde{K}_{\lambda, \nu}^X \times
     \left\{ \begin{array}{lll}
       \Gamma \left( \frac{1 - \nu}{2} \right), & p = 1, q \in 2\mathbbm{Z}+
       1, & \\
       \Gamma \left( \frac{1 - \nu}{2} \right), & p = 1, q \in 2\mathbbm{Z}, &
       \lambda + \nu > 0,\\
       \Gamma \left( \frac{\lambda - \nu}{2} \right) / \Gamma \left( -
       \frac{\nu}{2} \right), & p = 1, q \in 2\mathbbm{Z}, & \lambda + \nu
       \leqslant 0,\\
       1, & n \in 2\mathbbm{Z}, p > 1, & \\
       \Gamma \left( \frac{\lambda - \nu + 2}{4} \right), & q \in
       2\mathbbm{Z}+ 1, p \in 2\mathbbm{Z}_{\geqslant 1}, & \frac{\lambda +
       \nu}{2} : \tmop{odd},\\
       \Gamma \left( \frac{\lambda - \nu}{4} \right), & q \in 2\mathbbm{Z}+ 1,
       p \in 2\mathbbm{Z}_{\geqslant 1}, & \frac{\lambda + \nu}{2} :
       \tmop{even},\\
       \Gamma \left( \frac{\lambda - \nu}{2} \right) / \Gamma \left( \frac{2 -
       \nu}{2} \right), & q \in 2\mathbbm{Z}, p \in 2\mathbbm{Z}_{\geqslant 1}
       + 1, & \lambda + \nu \leqslant 0,\\
       \Gamma \left( \frac{\lambda - \nu + 2}{4} \right), & q \in
       2\mathbbm{Z}, p \in 2\mathbbm{Z}_{\geqslant 1} + 1 & \lambda + \nu > 0,
       \tmop{even},\\
       \Gamma \left( \frac{\lambda - \nu}{4} \right), & q \in 2\mathbbm{Z}, p
       \in 2\mathbbm{Z}_{\geqslant 1} + 1 & \lambda + \nu > 0, \tmop{odd} .
     \end{array} \right. \]
  We then have $\tilde{K}^Y_{\lambda, \nu} \in \mathcal{S} \tmop{ol}_Y
  (\mathbbm{R}^n ; \lambda, \nu)$ being nonzelo holomorphic in $ (\lambda,
  \nu) \in \backslash\backslash$ and moreover
  \begin{eqnarray}
    & \tmop{supp} (\tilde{K}_{\lambda, \nu}^Y) = \left\{ \begin{array}{ll}
      \left\{ \begin{array}{ll}
        Y, & (\lambda, \nu) \nin / /,\\
        \{ 0 \}, & (\lambda, \nu) \in / /,
      \end{array} \right. & \begin{array}{c}
        p = 1, q : \tmop{odd} \quad \tmop{or}\\
        p = 1, q : \tmop{even}, (\lambda, \nu) \in \backslash\backslash^+,
      \end{array}\\
      \left\{ \begin{array}{ll}
        Y, & \nu \nin \mathbbm{Z}_{\geqslant 0},\\
        \{ 0 \}, & \nu \in \mathbbm{Z}_{\geqslant 0},
      \end{array} \right. & p = 1, q : \tmop{even}, (\lambda, \nu) \in
      \backslash\backslash^-,\\
      \left\{ \begin{array}{ll}
        Y, & (\lambda, \nu) \in \mid \mid \sqcup / /,\\
        Y \cap C, & (\lambda, \nu) \in \mid \mid,\\
        \{ 0 \}, & (\lambda, \nu) \in / /,
      \end{array} \right. & p > 1, n \in 2\mathbbm{Z},\\
      \left\{ \begin{array}{ll}
        Y, & (\lambda, \nu) \nin \mid \mid \cup / /^-,\\
        \{ 0 \}, & (\lambda, \nu) \in / /^-,\\
        Y \cap C, & (\lambda, \nu) \in \mid \mid - / /^-,
      \end{array} \right. & q \in 2\mathbbm{Z}+ 1, p \in
      2\mathbbm{Z}_{\geqslant 1},\\
      \left\{ \begin{array}{ll}
        Y, & (\lambda, \nu) \in \mid \mid \sqcup / /^+,\\
        \{ 0 \}, & (\lambda, \nu) \in / /^+\\
        Y \cap C, & (\lambda, \nu) \in \mid \mid
      \end{array} \right. & q \in 2\mathbbm{Z}, p \in 1 +
      2\mathbbm{Z}_{\geqslant 1}, (\lambda, \nu) \in \backslash\backslash^+,\\
      \left\{ \begin{array}{ll}
        Y, & \nu \nin \mathbbm{Z}_{\geqslant 0},\\
        Y \cap C, & (\lambda, \nu) \in \mid \mid,\\
        \{ 0 \}, & \nu \in 2\mathbbm{Z}_{\geqslant 0},
      \end{array} \right. & q \in 2\mathbbm{Z}, p \in 1 +
      2\mathbbm{Z}_{\geqslant 1}, (\lambda, \nu) \in \backslash\backslash^-,
    \end{array} \right. &  \nonumber\\
    & / /^{\pm} \assign \left\{ (\lambda, \nu) \in / / \mid \nu \equiv
    \frac{1 - (\pm 1)}{2} \; \tmop{mod} \; 2 \right\}, \quad
    \backslash\backslash^+ \assign \{ (\lambda, \nu) \in \backslash\backslash
    \mid \lambda + \nu > 0 \}, \quad \backslash\backslash^- \assign
    \backslash\backslash -\backslash\backslash^+ . &  \nonumber
  \end{eqnarray}
\end{proposition}

\begin{proposition}
  \label{supp-sing:prop-supp-C}Suppose $(\lambda, \nu) \in \mid \mid$ and
  define $\tilde{K}_{\lambda, \nu}^C$ as
  \[ \tilde{K}^C_{\lambda, \nu} \assign \tilde{K}_{\lambda, \nu}^X \times
     \left\{ \begin{array}{ll}
       1, & q \in 2\mathbbm{Z}+ 1, p > 1\\
       \Gamma^{} \left( \frac{\lambda + \nu - n + 1}{2} \right), & q \in
       2\mathbbm{Z}, p = 1 \comma \nu \leqslant q / 2\\
       \Gamma^{} \left( \frac{\lambda - \nu}{2} \right), & q \in 2\mathbbm{Z},
       p = 1 \comma \nu > q / 2\\
       \Gamma \left( \frac{\lambda - \nu}{2} \right), & q \in 2\mathbbm{Z}, p
       > 1\\
       \Gamma \left( \frac{\lambda + \nu - n + 1}{2} \right), & q \in
       2\mathbbm{Z}+ 1, p = 1,
     \end{array} \right. \]
  We then have $\tilde{K}^C_{\lambda, \nu} \in \mathcal{S} \tmop{ol}_C
  (\mathbbm{R}^n ; \lambda, \nu)$ being nonzero holomorphic in $ (\lambda,
  \nu) \in \mid \mid$ and moreover
  \[ \tmop{supp} (\tilde{K}_{\lambda, \nu}^C) = \left\{ \begin{array}{lll}
       \{ [0] \}, & p = 1, q \in 2\mathbbm{Z}+ 1, & (\lambda, \nu) \in / /\\
       C, & p = 1, q \in 2\mathbbm{Z}+ 1, & (\lambda, \nu) \nin / /\\
       \{ [0] \}, & p = 1, q \in 2\mathbbm{Z}, & (\lambda, \nu) \in / / \cap
       \backslash\backslash\\
       C, & p = 1, q \in 2\mathbbm{Z}, & (\lambda, \nu) \nin / / \cap
       \backslash\backslash\\
       \{ [0] \}, & p > 1, q \in 2\mathbbm{Z}+ 1, & (\lambda, \nu) \in / /\\
       C, & p > 1, q \in 2\mathbbm{Z} + 1, & (\lambda, \nu) \nin / / \cup
       \backslash\backslash\\
       C \cap Y, & p > 1, q \in 2\mathbbm{Z} + 1, & (\lambda, \nu) \in
       \backslash\backslash - / /\\
       C, & p > 1, q \in 2\mathbbm{Z}, & (\lambda, \nu) \nin
       \backslash\backslash\\
       C \cap Y, & p > 1, q \in 2\mathbbm{Z}, & (\lambda, \nu) \in
       \backslash\backslash,
     \end{array} \right. \]
\end{proposition}

\subsection{Auxiliary lemmas}

\begin{lemma}
  \label{supp-sing:lem-strangelove}For $p = 1, q \in 2\mathbbm{Z}$ we have
  $\widetilde{\tilde{K}}_{\lambda, \nu}^X \neq 0$ when $(\lambda, \nu) \in
  \mid \mid \cap \backslash\backslash \cap / /$.
\end{lemma}

\begin{proof}
  Fix first $\nu_0$ such that $\mathbbm{C} \times \{ \nu_0 \} \subset \mid
  \mid$. Now, for generic $\lambda,$ $\widetilde{\tilde{K}}_{\lambda,
  \nu_0}^X$ is the nonzero finite multiple of $\tilde{K}_{\lambda, \nu_0}^X$,
  and hence (proving this for $\tilde{K}_{\lambda, \nu}^X$ by analytic
  ridigity and reduction to case $\tmop{Re} (- \nu), \tmop{Re} (\lambda + \nu
  - n) \gg 0$, when $\tilde{K}_{\lambda, \nu}^X$ gets continuous), we have
  (for generic $\lambda$ and by holomorphicity of
  $\widetilde{\tilde{K}}_{\lambda, \nu}^X$ expanding it to all $\lambda$)
  \[ \langle \widetilde{\tilde{K}}_{\lambda, \nu_0}^X, x^{\gamma} \rangle
     \simeq \Gamma (\lambda - \nu_0 + | \gamma |) \left\langle \frac{| Q |^{-
     \nu_0}}{\Gamma \left( \frac{1 - \nu_0}{2} \right)} \cdot \frac{| \omega_p
     |^{\lambda + \nu_0 - n}}{\Gamma \left( \frac{\lambda + \nu_0 - n + 1}{2}
     \right)}, \omega^{\gamma} \right\rangle_{\mathbbm{S}^{n - 1}} \]
  and proof of proposition \ref{supp-R:prop-main} implies that (from now on,
  we take $\omega^{2 \gamma} = y^{\beta} \in \mathbbm{C} [\mathbbm{S}^{q -
  1}]$, where $| \beta | = 2 k \in 2\mathbbm{Z}_{\geqslant 0}$)
  \[ \left\langle \frac{| Q |^{- \nu_0}}{\Gamma \left( \frac{1 - \nu_0}{2}
     \right)} \cdot \frac{| \omega_p |^{\lambda + \nu_0 - n}}{\Gamma \left(
     \frac{\lambda + \nu_0 - n + 1}{2} \right)}, \omega^{\gamma}
     \right\rangle_{\mathbbm{S}^{n - 1}} = f_1 (\lambda) \frac{1}{\Gamma
     \left( 1 + \frac{\lambda - \nu_0 - q}{2} \right)} + f_2 (\lambda)
     \frac{\Gamma \left( \frac{q + 2 k}{2} \right)}{\Gamma \left(
     \frac{\lambda + \nu_0 - q}{2} \right) \Gamma \left( 1 - \nu_0 + \frac{q +
     2 k}{2} \right)}, \]
  where $f_{1, 2}$ are holomorphic near $\lambda = \lambda_0$, such that
  $(\lambda_0, \nu_0) \in \mid \mid \cap \backslash\backslash \cap / /$ and
  $\lambda_0 - \nu_0 = - 2 k$; and such that $f_1 (\lambda_0, \nu_0) = f_2
  (\lambda_0, \nu_0) \neq 0$. Hence,
  \[ \langle \widetilde{\tilde{K}}_{\lambda, \nu_0}^X, x^{\gamma} \rangle
     \simeq \frac{\Gamma \left( \frac{\lambda - \lambda_0}{2} \right)}{\Gamma
     \left( \frac{\lambda - \lambda_0}{2} + \frac{\lambda_0 - \nu_0 - q +
     2}{2} \right)} + f (\lambda) \frac{\Gamma \left( \frac{\lambda -
     \lambda_0}{2} \right)}{\Gamma \left( \frac{\lambda - \lambda_0}{2} +
     \frac{\lambda_0 + \nu_0 - q}{2} \right)} \times \frac{\Gamma \left(
     \frac{q + \nu_0 - \lambda_0}{2} \right)}{\Gamma \left( 1 -
     \frac{\lambda_0 + \nu_0 - q}{2} \right)} \simeq \]
  Now, as we see that all three fractions are holomorphic when $\lambda =
  \lambda_0$ (first two have their denominators and enumerators' poles cancel;
  while the third one's denominator and enumerator are finite nonzero for
  $\lambda = \lambda_0$), hence we can simply substitute $\lambda = \lambda_0$
  (note that $\Gamma (x) / \Gamma (x - n) \rightarrow (- 1)^n n!$ as $x
  \rightarrow 0$):
  \begin{eqnarray}
    & \simeq \left( - \frac{\lambda_0 - \nu_0 - q + 2}{2} \right) ! + \left(
    - \frac{\lambda_0 + \nu_0 - q}{2} \right) ! \frac{\Gamma \left( \frac{q +
    \nu_0 - \lambda_0}{2} \right)}{\Gamma \left( 1 - \frac{\lambda_0 + \nu_0 -
    q}{2} \right)} > 0 \Rightarrow \tmop{nonzero} . &  \nonumber
  \end{eqnarray}
\end{proof}

\begin{lemma}
  \label{supp-sing:lem-strangelove-Y}For $p = 1, q \in 2\mathbbm{Z}, (\lambda,
  \nu) \in \backslash\backslash \cap \mid \mid \cap / /$ we have
  $\tilde{K}_{\lambda, \nu}^Y \neq 0$.
\end{lemma}

\begin{proof}
  We take $(\lambda_0, \nu_0) \in \backslash\backslash \cap \mid \mid \cap /
  /$ with $k \assign (\nu_0 - \lambda_0) / 2$ and consider $(\lambda, \nu)$
  changing in the set $\{ (\lambda, \nu) \mid \lambda + \nu = \lambda_0 +
  \nu_0 \}$. Now, for generic $(\lambda, \nu)$, $\tilde{K}_{\lambda, \nu}^Y$
  is the nonzero finite multiple of $\tilde{K}_{\lambda, \nu}^X$, and hence
  (proving this for $\tilde{K}_{\lambda, \nu}^X$ by analytic ridigity and
  reduction to case $\tmop{Re} (- \nu), \tmop{Re} (\lambda + \nu - n) \gg 0$,
  when $\tilde{K}_{\lambda, \nu}^X$ gets continuous), we have (for generic
  $\lambda$ and by holomorphicity of $\widetilde{\tilde{K}}_{\lambda, \nu}^X$
  expanding it to all $\lambda$)
  \[ \langle \tilde{K}_{\lambda, \nu}^Y, x^{\gamma} \rangle \simeq \Gamma
     (\lambda - \nu + | \gamma |) \left\langle \frac{| Q |^{- \nu}}{\Gamma
     \left( \frac{1 - \nu}{2} \right)} \cdot \frac{| \omega_p |^{\lambda + \nu
     - n}}{\Gamma \left( \frac{\lambda + \nu - n + 1}{2} \right)},
     \omega^{\gamma} \right\rangle_{\mathbbm{S}^{n - 1}} \]
  and proof of proposition \ref{supp-R:prop-main} implies that (from now on,
  we take $\omega^{2 \gamma} = y^{\beta} \in \mathbbm{C} [\mathbbm{S}^{q -
  1}]$, where $| \beta | = 2 k \in 2\mathbbm{Z}_{\geqslant 0}$)
  \[ \left\langle \frac{| Q |^{- \nu}}{\Gamma \left( \frac{1 - \nu}{2}
     \right)} \cdot \frac{| \omega_p |^{\lambda + \nu - n}}{\Gamma \left(
     \frac{\lambda + \nu - n + 1}{2} \right)}, \omega^{\gamma}
     \right\rangle_{\mathbbm{S}^{n - 1}} = \frac{1}{\Gamma \left( 1 +
     \frac{\lambda - \nu - q}{2} \right)} + f (\lambda) \frac{\Gamma \left(
     \frac{q + 2 k}{2} \right)}{\Gamma \left( \frac{\lambda + \nu - q}{2}
     \right) \Gamma \left( 1 - \nu + \frac{q + 2 k}{2} \right)}, \]
  where $f$ is holomorphic near $(\lambda_0, \nu_0)$, such that $f (\lambda_0,
  \nu_0) = 1$, and as $\lambda + \nu - q = \lambda_0 + \nu_0 - q \in -
  2\mathbbm{Z}_{\geqslant 0}$, the second addend vanishes. Hence,
  \[ \langle \tilde{K}_{\lambda, \nu}^Y, x^{\gamma} \rangle \simeq
     \frac{\Gamma \left( \frac{\lambda - \lambda_0}{2} \right)}{\Gamma \left(
     (\lambda - \lambda_0) + \frac{\lambda_0 - \nu_0 - q + 2}{2} \right)} . \]
  and as $(\lambda, \nu) \rightarrow (\lambda_0, \nu_0)$, the poles cancel and
  the fraction tends to a nonzero number.
\end{proof}

\begin{lemma}
  \label{supp-sing:lem-strangeelement}If $p = 1$ for $(\lambda, \nu) \in
  \backslash\backslash$ and $(\lambda, \nu) \in \mid \mid$ respectively we
  have distributions
  \[ \frac{| x_p |^{\lambda + \nu - n}}{\Gamma \left( \frac{\lambda + \nu - n
     + 1}{2} \right)} \cdot | Q |^{- \nu}, | x_p |^{\lambda + \nu - n} \cdot
     \frac{| Q |^{- \nu}}{\Gamma \left( \frac{1 - \nu}{2} \right)} \in
     \mathcal{D}' (\mathbbm{R}^n - \{ 0 \}) \]
  being well-defined (ie. the generalized function product used to define them
  is well-defined) members of $\mathcal{S} \tmop{ol} (\mathbbm{R}^n - \{ 0 \}
  ; \lambda, \nu)$ holomorphic in $(\lambda, \nu) \in \backslash\backslash,
  \mid \mid$ respectively. Moreover
  \begin{eqnarray}
    & \tmop{supp} \left( \frac{| x_p |^{\lambda + \nu - n}}{\Gamma \left(
    \frac{\lambda + \nu - n + 1}{2} \right)} \cdot | Q |^{- \nu} \right) = C
    \cap \{ \mathbbm{R}^n - \{ 0 \} \} &  \nonumber\\
    & \tmop{supp} \left( | x_p |^{\lambda + \nu - n} \cdot \frac{| Q |^{-
    \nu}}{\Gamma \left( \frac{1 - \nu}{2} \right)} \right) = Y \cap \{
    \mathbbm{R}^n - \{ 0 \} \} &  \nonumber
  \end{eqnarray}
\end{lemma}

\begin{proof}
  We do the proof only for $\frac{| x_p |^{\lambda + \nu - n}}{\Gamma \left(
  \frac{\lambda + \nu - n + 1}{2} \right)} \cdot | Q |^{- \nu}$, as the other
  half is handled in the same way. We note that uniqueness of holomorphic
  continuation implies that $\frac{| x_p |^{\lambda + \nu - n}}{\Gamma \left(
  \frac{\lambda + \nu - n + 1}{2} \right)} \cdot | Q |^{- \nu} = \Gamma \left(
  \frac{1 - \nu}{2} \right) K_{\lambda, \nu}^X$ (with $K_{\lambda, \nu}^X \in
  \mathcal{D}' (\mathbbm{R}^n - \{ 0 \})$ defined as in proof of proposition
  \ref{supp-R:prop-main}) and lemma \ref{supp-R:lem-supp-of-K} implies that
  $\Gamma \left( \frac{1 - \nu}{2} \right) K_{\lambda, \nu}$ is indeed
  holomorphic, whereas it is an element of $\mathcal{S} \tmop{ol}
  (\mathbbm{R}^n - \{ 0 \} ; \lambda, \nu)$ by proposition
  \ref{sol:prop-holocont}. Finally, as $\tmop{supp} \left( \frac{| x_p
  |^{\lambda + \nu - n}}{\Gamma \left( \frac{\lambda + \nu - n + 1}{2}
  \right)} \right) \subset \{ x_p = 0 \}$ we see that $\tmop{supp} \left(
  \frac{| x_p |^{\lambda + \nu - n}}{\Gamma \left( \frac{\lambda + \nu - n +
  1}{2} \right)} \cdot | Q |^{- \nu} \right) \subset \{ x_p = 0 \}$ by lemma
  \ref{KR-normalization-recur:lem-mult-smth}, while one the other hand $| Q
  |^{- \nu}$ is smooth nonzero on $\{ Q \neq 0 \}$, hence by lemma
  \ref{KR-normalization-recur:lem-mult-smth} again we have $\tmop{supp} \left(
  \frac{| x_p |^{\lambda + \nu - n}}{\Gamma \left( \frac{\lambda + \nu - n +
  1}{2} \right)} \cdot | Q |^{- \nu} \right) \supset \{ x_p = 0 \} \cap \{ Q
  \neq 0 \}$, whereas the conclusion.
\end{proof}

\subsection{Proofs}

\begin{proof}
  (of prop. \ref{supp-sing:prop-supp-mmm}) The information in proposition
  \ref{supp-R:prop-main} implies that under the given assumptions (that is,
  $(\lambda, \nu) \in \mid \mid \mid$) we have $K_{\lambda, \nu}^X \in
  \mathcal{D}' (\mathbbm{R}^n - \{ 0 \})$, as defined in proof of proposition
  \ref{supp-R:prop-main} extends to the homogeneous distribution on
  $\mathbbm{R}^n$, so the result about $\tmop{supp}
  (\widetilde{\tilde{K}}_{\lambda, \nu}^X) = \tmop{supp} (K_{\lambda, \nu}^X)$
  follows from lemma \ref{supp-R:lem-supp-of-K} for all cases except
  $(\lambda, \nu) \in \backslash\backslash \cap \mid \mid \cap / /$, when it
  follows from lemma \ref{supp-sing:lem-strangelove}.
\end{proof}

\begin{proof}
  (of prop. \ref{supp-sing:prop-supp-Y}) We will proceed case by case, based
  on the expression for $\tilde{K}_{\lambda, \nu}^Y$ given in the statement,
  spending a paragraph on every case. We note that we always implicity assume
  that $(\lambda, \nu) \in \backslash\backslash$ in this proof and whether the
  set (in)equality is written, like $A \supset B$, it is assumed to mean that
  $A \cap \backslash\backslash \supset B \cap \backslash\backslash$. First, we
  suppose $p = 1$ and $q$ is odd. We note that in this case $\mid \mid \mid
  =\mathbbm{C} \times (2\mathbbm{Z})$, $\mid \mid \mid \cap \mid \mid =
  \varnothing$ and $\mid \mid \cap / / = \mid \mid \mid \cap / / =
  \varnothing$. Hence, we have
  \[ \tmop{supp} (\tilde{K}_{\lambda, \nu}^X) = \left\{ \begin{array}{ll}
       Y, & (\lambda, \nu) \nin \mid \mid \sqcup / /,\\
       \varnothing, & (\lambda, \nu) \in \mid \mid (= \mid \mid - / /),\\
       \{ 0 \}, & (\lambda, \nu) \in / / .
     \end{array} \right. \]
  This immediately implies that $\tilde{K}_{\lambda, \nu}^X = 0$ whenever
  $(\lambda, \nu) \in \mid \mid$, hence the $\tilde{K}_{\lambda, \nu}^Y$ as
  defined in statement is holomorphic. It only remains to show that
  $\tmop{supp} (\tilde{K}_{\lambda, \nu}^Y) = Y$ for $(\lambda, \nu) \in \mid
  \mid (= \mid \mid - / /)$, but the latter readily follows from the fact that
  $\Gamma \left( \frac{\lambda - \nu}{2} \right) < \infty$ at these points and
  lemma \ref{supp-sing:lem-strangeelement}.
  
  Next, we go to the case $p = 1, q : \tmop{even}, \lambda + \nu > 0$. In
  particular, the latter implies that $/ / \cap (\mathbbm{C} \times
  \mathbbm{Z}_{\leqslant 0}) = \varnothing$, hence $/ / \cap \mid \mid \mid =
  / / \cap \mid \mid$. We also have $\mathbbm{C} \times
  (\mathbbm{Z}_{\leqslant 0} \cup (2\mathbbm{Z}_{\geqslant 0} + 1)) = \mid
  \mid \mid \supset \mid \mid$ and hence in this case
  \[ \tmop{supp} (\tilde{K}_{\lambda, \nu}^X) = \left\{ \begin{array}{ll}
       Y, & (\lambda, \nu) \nin \mid \mid \cup / /,\\
       \varnothing, & (\lambda, \nu) \in \mid \mid - / /,\\
       \varnothing, & (\lambda, \nu) \in / / \cap \mid \mid (= / / \cap \mid
       \mid \mid),\\
       \{ 0 \}, & (\lambda, \nu) \in / / - \mid \mid (= / / - \mid \mid \mid),
     \end{array} \right. = \left\{ \begin{array}{ll}
       Y, & (\lambda, \nu) \nin \mid \mid \cup / /,\\
       \varnothing, & (\lambda, \nu) \in \mid \mid,\\
       \{ 0 \}, & (\lambda, \nu) \in / / - \mid \mid .
     \end{array} \right. \]
  and then we continue as in the previous paragraph.
  
  Next, we got to the case $p = 1, q : \tmop{even}, \lambda + \nu \leqslant
  0$. The latter implies that $\mid \mid \subset / /$, hence
  \[ \tmop{supp} (\tilde{K}_{\lambda, \nu}^X) = \left\{ \begin{array}{ll}
       Y, & (\lambda, \nu) \nin / /,\\
       \varnothing, & (\lambda, \nu) \in / / \cap \mid \mid \mid,\\
       \{ 0 \}, & (\lambda, \nu) \in / / - \mid \mid \mid .
     \end{array} \right. \]
  This immediately implies that for $(\lambda, \nu) \in / /$ (which implies
  $\nu \in \mathbbm{Z}$) we have $\tilde{K}_{\lambda, \nu}^X = 0$, unless $\nu
  \in 2\mathbbm{Z}_{\geqslant 0}$, which implies that $\tilde{K}_{\lambda,
  \nu}^Y$ is holomorphic. It then remains to show that we have $\tmop{supp}
  (\tilde{K}_{\lambda, \nu}^Y) = \{ 0 \}$ for $(\lambda, \nu) \in / / \cap
  \mid \mid \mid$. Now, for $\nu \leqslant 0$ this follows by lemma
  \ref{supp-sing:lem-strangeelement}, while for $\nu > 0$ by lemma
  \ref{supp-sing:lem-strangelove-Y}.
  
  Next, we suppose $n \in 2\mathbbm{Z}, p > 1$, in which case we have $/ /
  \cap \mid \mid \mid = \varnothing$, hence $\tilde{K}_{\lambda, \nu}^X \neq
  0$, hence the conclusion.
  
  Next, we suppose $q \in 2\mathbbm{Z}+ 1, p \in 2\mathbbm{Z}_{\geqslant 1},
  \frac{\lambda + \nu}{2} : \tmop{odd}$. To show that $\tilde{K}_{\lambda .
  \nu}^Y$ is holomorphic, it suffices to show that $\tilde{K}_{\lambda, \nu}^X
  = 0$ for $(\lambda, \nu) \in \backslash\backslash \cap / /$, if $\frac{\nu -
  \lambda}{2} : \tmop{odd}$. For this we suppose $\lambda - \nu = - 2 l$ with
  $l : \tmop{odd}$. This implies that $\nu = \frac{\lambda + \nu}{2} -
  \frac{\lambda - \nu}{2} = \frac{\lambda + \nu}{2} - l$ is the difference of
  odd and odd, hence even, hence $(\lambda, \nu) \in \mid \mid \mid$, hence
  the conclusion. To compute the supports, we note that $\tilde{K}_{\lambda,
  \nu}^X$ and $\tilde{K}_{\lambda, \nu}^Y$ can have different supports only
  for $(\lambda, \nu) \in / /$ with $l : \tmop{odd}$. Now, as $\nu \equiv l +
  1 \tmop{mod} 2$, since $(\lambda + \nu) / 2 : \tmop{odd}$, we have that
  these points are not in $\mid \mid$, hence we may use lemma
  \ref{supp-sing:lem-strangeelement}. In this way we have
  \[ \tmop{supp} (\tilde{K}_{\lambda, \nu}^Y) = \left\{ \begin{array}{ll}
       Y, & (\lambda, \nu) \nin \mid \mid \cup / / = \mid \mid \cup / /^- \cup
       / /^+,\\
       Y \cap C, & (\lambda, \nu) \in \mid \mid - / / = \mid \mid - / /^-,\\
       Y, & / / \cap \mid \mid \mid = / /^+,\\
       \{ 0 \}, & / / - \mid \mid \mid = / /^-,
     \end{array} \right. = \left\{ \begin{array}{ll}
       Y, & (\lambda, \nu) \nin \mid \mid \cup / /^-,\\
       Y \cap C, & (\lambda, \nu) \in \mid \mid - / /^-,\\
       \{ 0 \}, & / /^- .
     \end{array} \right. \]
  
  
  Next, we handle the case $q : \tmop{odd}, p \in 2\mathbbm{Z}_{\geqslant 1},
  \frac{\lambda + \nu}{2} : \tmop{even}$ in the previous paragraph.
  
  Next, we consider the case $q : \tmop{even}, p \in 2\mathbbm{Z}_{\geqslant
  1} + 1, \lambda + \nu \leqslant 0$ (the latter implies that $\mid \mid
  \subset / /$). First, the holomorphicity is clear. Regarding the support, we
  have (using lemma \ref{supp-sing:lem-strangeelement})
  \begin{eqnarray}
    & \tmop{supp} (\tilde{K}_{\lambda, \nu}^Y) = \left\{ \begin{array}{ll}
      Y, & (\lambda, \nu) \nin \mid \mid \cup / / = / /,\\
      ?, & / / \cap \mid \mid \mid = \{ (\lambda, \nu) \in / / \mid \nu \in
      2\mathbbm{Z}_{\leqslant 0} - 1 \} \sqcup \{ (\lambda, \nu) \in / / \mid
      \nu \in 2\mathbbm{Z}_{\geqslant 0} + 1 \} \sqcup \{ (\lambda, \nu) \in /
      / \mid \nu \in 2\mathbbm{Z}_{\leqslant 0} \},\\
      \{ 0 \}, & / / - \mid \mid \mid = \{ (\lambda, \nu) \in / / \mid \nu \in
      2\mathbbm{Z}_{> 0} \},
    \end{array} \right. = &  \nonumber
  \end{eqnarray}
  now, we can use lemma \ref{supp-R:lem-supp-of-K} to resolve question marks
  for $\{ (\lambda, \nu) \in / / \mid \nu \in 2\mathbbm{Z}_{\leqslant 0} - 1
  \}, \{ (\lambda, \nu) \in / / \mid \nu \in 2\mathbbm{Z}_{\geqslant 0} + 1
  \}, \{ (\lambda, \nu) \in / / \mid \nu \in 2\mathbbm{Z}_{\leqslant 0} \}$,
  hence getting the
  \begin{eqnarray}
    & = \left\{ \begin{array}{ll}
      Y, & (\lambda, \nu) \nin / /,\\
      Y, & \{ (\lambda, \nu) \in / / \mid \nu \in 2\mathbbm{Z}_{\leqslant 0} -
      1 \},\\
      Y \cap C, & \{ (\lambda, \nu) \in / / \mid \nu \in
      2\mathbbm{Z}_{\geqslant 0} + 1 \},\\
      Y, & \{ (\lambda, \nu) \in / / \mid \nu \in 2\mathbbm{Z}_{\leqslant 0}
      \},\\
      \{ 0 \}, & \{ (\lambda, \nu) \in / / \mid \nu \in 2\mathbbm{Z}_{> 0} \},
    \end{array} \right. = \left\{ \begin{array}{ll}
      Y, & (\lambda, \nu) \in / /^c \cup (/ / \cap (\mathbbm{C} \times
      \mathbbm{Z}_{\leqslant 0})) = / / \cap (\mathbbm{C} \times
      \mathbbm{Z}_{> 0}) =,\\
      Y \cap C, & \{ (\lambda, \nu) \in / / \mid \nu \in
      2\mathbbm{Z}_{\geqslant 0} + 1 \} = \mid \mid,\\
      \{ 0 \}, & \{ (\lambda, \nu) \in / / \mid \nu \in 2\mathbbm{Z}_{> 0} \}
      = \{ (\lambda, \nu) \mid \nu \in 2\mathbbm{Z}_{> 0} \},
    \end{array} \right. &  \nonumber
  \end{eqnarray}
  Finally, we are left with two cases, when $q : \tmop{even}, p \in
  2\mathbbm{Z}_{\geqslant 1} + 1, (\lambda, \nu) \in \backslash\backslash^+$.
  We assume $\frac{\lambda + \nu}{2} :$odd, the other parity being handled in
  precisely the same way. Now, to show that the $\tilde{K}_{\lambda, \nu}^Y$
  is holomorphic, it suffices to show that if $(\lambda, \nu) \in
  \backslash\backslash$ and $\frac{\nu - \lambda}{2} : \tmop{even}$ implies
  $(\lambda, \nu) \in \mid \mid \mid$. The latter is clear, however, as $\nu =
  \frac{\lambda + \nu}{2} - \frac{\lambda - \nu}{2}$:odd under these
  assumptions (note that $\mathbbm{C} \times (2\mathbbm{Z}+ 1) \subset \mid
  \mid \mid$). This shows the holomorphicity. We also note that $(\lambda,
  \nu) \in / / \cap \mid \mid \mid$ iff $(\lambda, \nu) \in / / \cap \mid \mid
  \subseteq / /^-$ (as $(\lambda, \nu) \in / /$ and $\nu \leqslant 0$ would
  imply $\frac{\lambda + \nu}{2} = \frac{\lambda - \nu}{2} + \nu \leqslant 0$
  contradicting the $(\lambda, \nu) \in \backslash\backslash^+$ assumption),
  iff $(\lambda, \nu) \in / /^-$ (again, as $(\lambda, \nu) \in / /, \nu
  \leqslant 0$ would imply the contradiction). Keeping this in mind, we
  compute the support as
  \begin{eqnarray}
    & \tmop{supp} (\tilde{K}_{\lambda, \nu}^Y) = \left\{ \begin{array}{ll}
      Y, & (\lambda, \nu) \nin \mid \mid \cup / / = \mid \mid \sqcup / /^+,\\
      Y \cap C, & (\lambda, \nu) \in \mid \mid - / / = \mid \mid - / /^-,\\
      ?, & (\lambda, \nu) \in / / \cap \mid \mid \mid = / /^-,\\
      \{ 0 \}, & (\lambda, \nu) \in / / - \mid \mid \mid = / /^+,
    \end{array} \right. = &  \nonumber
  \end{eqnarray}
  Now, $/ /^- = \left\{ (\lambda, \nu) \in / / \mid \frac{\nu - \lambda}{2} :
  \tmop{even} \right\}$ (as $\frac{\lambda + \nu}{2}$:odd), hence we can use
  lemma \ref{supp-R:lem-supp-of-K} to resolve the question mark, thus
  continuing as
  \begin{eqnarray}
    & = \left\{ \begin{array}{ll}
      Y, & (\lambda, \nu) \nin \mid \mid \sqcup / /^+,\\
      Y \cap C, & (\lambda, \nu) \in \mid \mid - / /^-,\\
      Y \cap C, & (\lambda, \nu) \in / /^- = / /^- \cap \mid \mid,\\
      \{ 0 \}, & (\lambda, \nu) \in / /^+,
    \end{array} \right. = \left\{ \begin{array}{ll}
      Y, & (\lambda, \nu) \nin \mid \mid \sqcup / /^+,\\
      Y \cap C, & (\lambda, \nu) \in \mid \mid,\\
      \{ 0 \}, & (\lambda, \nu) \in / /^+ .
    \end{array} \right. &  \nonumber
  \end{eqnarray}
\end{proof}

\begin{proof}
  (of prop. \ref{supp-sing:prop-supp-C}) First, we consider the $p > 1$ case,
  when $\tilde{K}_{\lambda, \nu}^X$ vanishes iff $(\lambda, \nu) \in / / \cap
  \mid \mid \mid$. Now, if $q \in 2\mathbbm{Z}+ 1$, we see that $\mid \mid
  \cap (/ / \cap \mid \mid \mid) = \varnothing$, hence the answer. Conversely,
  if $q \in 2\mathbbm{Z}$, we have $\mid \mid \subset \mid \mid \mid$, hence
  $\tilde{K}_{\lambda, \nu}^C$ as defined in proposition is holomorphic, which
  implies that under the given assumptions $K_{\lambda, \nu}^X$ as defined in
  proof of proposition \ref{supp-R:prop-main} extends to $\mathbbm{R}^n$ and
  the result about support follows from lemma \ref{supp-R:lem-supp-of-K}.
  Thus, we proceed to $p = 1$ case.
  
  When $q \in 2\mathbbm{Z}+ 1$, we have $\mid \mid \cap \mid \mid \mid =
  \varnothing$, hence under these assumptions (ie. that $q \in 2\mathbbm{Z}+
  1, (\lambda, \nu) \in \mid \mid$) we have that $\tilde{K}_{\lambda, \nu}^X =
  0$ iff $(\lambda, \nu) \in \backslash\backslash$, hence $\tilde{K}_{\lambda,
  \nu}^C$ is holomorphic and due to the uniqueness of holomorphic extension
  implies that
  \[ \tilde{K}_{\lambda, \nu}^C = \tilde{\rho} \left( \frac{| Q |^{-
     \nu}}{\Gamma \left( \frac{1 - \nu}{2} \right)} \cdot | x_p |^{\lambda +
     \nu - n} \right) = \Gamma^{- 1} \left( \frac{\lambda - \nu}{2} \right)
     \rho \left( \frac{| Q |^{- \nu}}{\Gamma \left( \frac{1 - \nu}{2} \right)}
     \cdot | x_p |^{\lambda + \nu - n} \right) \]
  (where $\frac{| Q |^{- \nu}}{\Gamma \left( \frac{1 - \nu}{2} \right)} \cdot
  | x_p |^{\lambda + \nu - n} \in \mathcal{D}' (\mathbbm{R}^n - \{ 0 \})$ is
  as in lemma \ref{supp-sing:lem-strangeelement}) and as $\mid \mid \cap \mid
  \mid \mid = \varnothing$, we conclude that
  \[ \tmop{supp} (\tilde{K}_{\lambda, \nu}^C) = \left\{ \begin{array}{ll}
       \{ 0 \}, & (\lambda, \nu) \in / /\\
       \tmop{supp} \left( \frac{| Q |^{- \nu}}{\Gamma \left( \frac{1 - \nu}{2}
       \right)} \cdot | x_p |^{\lambda + \nu - n} \right) \cup \{ 0 \}, &
       (\lambda, \nu) \nin / /,
     \end{array} \right. \]
  whereas the answer follows from lemma \ref{supp-sing:lem-strangeelement}.
  
  Finally, we turn to the case $q \in 2\mathbbm{Z}$, when $\mid \mid \subset
  \mid \mid \mid$, whereas we have $\tilde{K}_{\lambda, \nu}^X = 0$ iff
  $(\lambda, \nu) \in / / \cup \backslash\backslash$ and as we have $q - 2 \nu
  \geqslant 0 \Rightarrow \mid \mid \cap / / \subset \mid \mid \cap
  \backslash\backslash$ and $q - 2 \nu < 0 \Rightarrow \mid \mid \cap
  \backslash\backslash \subset \mid \mid \cap / /$, we conclude that
  $\tilde{K}_{\lambda, \nu}^C$ as it is defined is holomorphic. Now, consider
  cases. First, if $q - 2 \nu \geqslant 0$, we have the information about
  support following as in the previous paragraph. On contrary, if $q - 2 \nu <
  0$, the definitions imply that $\tilde{K}_{\lambda, \nu}^C = \rho
  (K_{\lambda, \nu}^X)$ and the information about support follows from lemmas
  \ref{supp-R:lem-supp-of-K} and \ref{supp-sing:lem-strangelove}.
\end{proof}

\

\

\

\

\section{Differential symmetry breaking operators (B12)}\label{sec:diffSBO}

\subsection{Main results}

\begin{proposition}
  \label{diffSBO:prop-main}For $(\lambda, \nu) \in \mathbbm{C}^2$ we have
  \[ \mathcal{S} \tmop{ol}_{\{ 0 \}} (\mathbbm{R}^{p, q} ; \lambda, \nu) =
     \left\{ \begin{array}{ll}
       \mathbbm{C} \cdot \tilde{C}_{\nu - \lambda}^{\lambda - (n - 1) / 2} (-
       \tilde{\Delta} \tilde{\delta} (x), \delta (x_p)), & \lambda - \nu \in -
       2\mathbbm{Z}_{\geqslant 0}\\
       0, & \tmop{otherwise}
     \end{array} \right. \]
  where $\tilde{\Delta}$ is a $(p - 1, q)$-Laplacian, $\tilde{\delta}$ is
  dirac delta on all variables except of $x_p$ and $C^{\alpha}_k (\cdot,
  \cdot)$ is the 2-variable inflation of renormalized Gegenbauer polynomial,
  as in {\cite[(16.3)]{kobayashi2015symmetry}}. For $\lambda - \nu \in -
  2\mathbbm{Z}_{\geqslant 0}$ we will use the notation
  \[ \tilde{K}_{\lambda, \nu}^{\{ 0 \}} \assign \tilde{C}_{\nu -
     \lambda}^{\lambda - (n - 1) / 2} (- \tilde{\Delta} \tilde{\delta} (x),
     \delta (x_p)) . \]
\end{proposition}

\subsection{Auxiliary results}

\begin{lemma}
  \label{diffSBO:lem-aux}For $f \in \mathcal{D}'_{\{ 0 \}} (\mathbbm{R}^{p,
  q}) \assign \{ f \in \mathcal{D}' (\mathbbm{R}^{p, q}) | \tmop{supp} (f)
  \subset \{ 0 \} \}$ we have $f$ is $N_+'$-invariant on $\mathbbm{R}^{p, q}$
  (def. \ref{def-n-nots:def-n+invar}) iff it satisfies equations (\ref{Ndiff})
\end{lemma}

\begin{proof}
  As ``$\Rightarrow$'' part is clear, we just prove the converse. But it is
  readily implied by lemma \ref{supp-Q:lem-sing-q-4}.
\end{proof}

\subsection{Proofs}

\begin{proof}
  (of prop. \ref{diffSBO:prop-main}) We apply algebraic Fourier transform $F_c
  : \mathcal{D}'_{\{ 0 \}} (\mathfrak{n}_-) \rightarrow \tmop{Pol}
  (\mathfrak{n}_-^{\ast})^{}$ defined by formula $(F_c f) (\xi) \assign
  \hat{F} (\xi) \assign \int_{\mathfrak{n}_-}^{} f (x) \exp \langle x, \xi
  \rangle d x$. We further identify $\mathfrak{n}_-^{\ast}$ dual of
  $\mathfrak{n}_-$ with $\mathfrak{n}_+$, identification given via the pairing
  $\frac{1}{4} (X, Y) \mapsto \tmop{tr} (X Y)$.
  
  Lemmas \ref{diffSBO:lem-aux} and \ref{lem67:lem-homogImpliesE} tell us that
  $F \in \mathcal{D}'_{\{ 0 \}} (\mathbbm{R}^{p, q})$ belongs to $\mathcal{S}
  \tmop{ol}_{\{ 0 \}} (\mathbbm{R}^{p, q} ; \lambda, \nu)$ iff it satisfies
  \begin{eqnarray}
    & E F = (\lambda - \nu - n) F &  \nonumber\\
    & F (- x) = F (x) &  \nonumber\\
    & F (g \cdot) = F (\cdot), \quad \forall g \in O (p, q)_{e_p} \assign \{
    g \in O (p, q) | g \cdot x_p = x_p \} \simeq O (p - 1, q) &  \nonumber\\
    & \left[ (\lambda - n) \varepsilon_j x_j - \varepsilon_j x_j E +
    \frac{1}{2} Q (x) \frac{\partial}{\partial x_j} \right] F = 0, \quad 1
    \leqslant j \leqslant n, j \neq p &  \nonumber
  \end{eqnarray}
  Under the algebraic Fourier transform $F_c$ these change as follows (we note
  that $E$ under $F_c$ becomes $- E - n$, multiplication by $x_j$ becomes
  $\partial / \partial \zeta_j$ and $\partial / \partial x_j$ becomes
  multiplication by $- \zeta_j$):
  \begin{eqnarray}
    & E \hat{F} = (\nu - \lambda) \hat{F} &  \nonumber\\
    & \hat{F} (- x) = \hat{F} (x) &  \nonumber\\
    & \hat{F} (g \cdot) = \hat{F} (\cdot), \quad \forall g \in O (p, q)_{e_p}
    \assign \{ g \in O (p, q) | g \cdot x_p = x_p \} \simeq O (p - 1, q) & 
    \nonumber\\
    & \left[ - \frac{1}{2} \varepsilon_j \zeta_j \tilde{\Delta} + (\lambda +
    E) \frac{\partial}{\partial \zeta_j} \right] \hat{F} = 0, \quad 1
    \leqslant j \leqslant n, j \neq p, \quad \tilde{\Delta} \assign \sum_{i =
    1, i \neq p}^n \varepsilon_j \frac{\partial}{\partial \zeta_j} . & 
    \nonumber
  \end{eqnarray}
  Now, we make the following observations:
  \begin{enumerate}
    \item As $\hat{F}$ satisfies $E \hat{F} = (\nu - \lambda) \hat{F}
    \nocomma$, it should be homogeneous of degree $\lambda - \nu$. But as it
    is a polynomial, it's homogeneity degree can only be positive integer, so
    we immediately see that $\mathcal{S} \tmop{ol}_{\{ 0 \}} (\mathbbm{R}^{p,
    q} ; \lambda, \nu) = 0$ if $\lambda - \nu \nin -\mathbbm{Z}_{\geqslant
    0}$;
    
    \item Moreover, as $\hat{F}$ satisfies $\hat{F} (- x) = \hat{F} (x)$, it
    should be even and hence $\mathcal{S} \tmop{ol}_{\{ 0 \}} (\mathbbm{R}^{p,
    q} ; \lambda, \nu) = 0$ if $\lambda - \nu \nin - 2\mathbbm{Z}_{\geqslant
    0}$;
    
    \item The requirement $\hat{F} (g \cdot) = \hat{F} (\cdot), \quad \forall
    g \in O (p, q)_{e_p}$ implies that $\hat{F}$ is polynomial in $\tilde{Q}
    \assign \sum_{i = 1 ; i \neq p}^n \varepsilon_i x_i^2$ and $x_p$. Hence,
    using homogeneity with degree $a \assign \nu - \lambda \in
    2\mathbbm{Z}_{\geqslant 0}$ we can write then $\hat{F} = \tilde{Q}^{a / 2}
    g \left( x_{p - 1} / \sqrt{\tilde{Q}} \right)$ with $g (\cdot)$ being a
    polynomial;
    
    \item Substituting this into the $\left[ - \frac{1}{2} \varepsilon_j
    \zeta_j \tilde{\Delta} + (\lambda + E) \frac{\partial}{\partial \zeta_j}
    \right] \hat{F} = 0$ gives
    \[ g (t) (n - 1 - a - 2 \lambda) a - g' (t) \times (n - 2 - 2 \lambda) t
       + (t^2 + 1) g'' (t) = 0 ; \]
    \item Substituting $t = i s$ into the previous equation gives (we abuse
    notation and write $g (s)$ in place of $g (i s)$)
    \[ g (s) (n - 1 - 2 \lambda) a - g' (s) (n - 2 - 2 \lambda) s + (1 - s^2)
       g'' (s) = 0 ; \]
    hence $g (s)$ should be proportional to $C^{\lambda - (n - 1) / 2}_a (s)$
    (see {\cite[thm. 11.4]{kobayashi2015differential2}}). Indeed, other
    polynomial solution (say, $\tilde{C}$) of this equation arises only when
    $\alpha \assign \lambda - (n - 1) / 2 \in \mathbbm{Z}+ 1 / 2$ and $1 - 2 a
    \leqslant 2 \alpha \leqslant - a$. But then $\tilde{C}$ has its top term
    is a non-zero multiple of $t^{- (2 \alpha + a)}$, but as $\alpha \in
    \mathbbm{Z}+ 1 / 2$, we have $- (2 \alpha + a) \in 2\mathbbm{Z}+ 1$, which
    contradicts evenness of $\hat{F}$.
  \end{enumerate}
  The result now follows by taking the inverse Fourier transform.
\end{proof}

\section{Determination of $\mathcal{S} \tmop{ol}_C (\mathbbm{R}^n - \{ 0 \} ;\lambda, \nu)$ (B13)}\label{sec:uniq-c}


\subsection{Main results}

\begin{proposition}
  \label{sol-MO:prop-solCnonzero}The following holds:
  \[ \mathcal{S} \tmop{ol}_C (\mathbbm{R}^{p, q} \backslash \{ 0 \} ; \lambda,
     \nu) =\mathbbm{C} \left\{ \begin{array}{ll}
       \delta^{(\nu - 1)} (Q) \cdot | x_p |^{\lambda + \nu - n}, & p = 1,
       (\lambda, \nu) \in \mid \mid\\
       \delta^{(\nu - 1)} (Q) \cdot \frac{| x_p |^{\lambda + \nu - n}}{\Gamma
       ((\lambda + \nu - n + 1) / 2)}, & p > 1, (\lambda, \nu) \in \mid \mid\\
       0, & (\lambda, \nu) \nin \mid \mid
     \end{array} \right. . \]
\end{proposition}

\subsection{Auxiliary lemmas}

\begin{lemma}
  \label{uniq-c:lem-generic}Proposition \ref{sol-MO:prop-solCnonzero} holds
  for $(\lambda, \nu) \nin \mid \mid$.
\end{lemma}

\begin{proof}
  So let's suppose $\nu \nin 2\mathbbm{Z}_{\geqslant 0} + 1$ and $F \in
  \mathcal{S} \tmop{ol}_{\{ Q = 0 \}} (\mathbbm{R}^{p, q} ; \lambda, \nu)$. In
  the light of lemma \ref{lem:sing-q-3} it suffices to prove that $F
  \mid_{\{ (x, y) \in \R^{p, q} |x \neq 0, \hspace{0.75em} y \neq 0\}} =
  0$. Abusing notation, from now and till the end of the proof we will call
  this restriction $F$. We recall the $(\mu, s, \omega_{p - 1}, \omega_{q -
  1}) \in \mathbbm{R}_{> 0}^2 \times \mathbbm{S}^{p - 1} \times \mathbbm{S}^{q
  - 1}$ coordinates parametrizing $\{ x \neq 0 \} \cap \{ y \neq 0 \}$ that
  were introduced in lemma \ref{lem:sing-q-6}.
  
  As $\supp (F) \subset \{ Q = 0 \}$ and the latter subset in the $(\mu, s,
  \omega_{p - 1}, \omega_{q - 1})$ coordinates becomes $\{ \mu = 1 \}$, fact
  \ref{fact:sing-q-3} \ implies that in $(\mu, s, \omega_{p - 1}, \omega_{q -
  1})$ coordinates $F = \sum_{i \geq 0} \delta^{(i)}  (\mu - 1) u_i$ with the
  sum being locally finite. Now, lemma \ref{supp-Q:lem-operator} implies that
  $\mysbra{\nu + (\mu - 1)  \frac{\partial}{\partial \mu}} F = 0$ and
  substituting this in $F = \sum_{i \geq 0} \delta^{(i)}  (\mu - 1) u_i$ and
  keeping in mind the formula $(\mu - 1)  \frac{\partial}{\partial \mu}
  \delta^{(i)}  (\mu - 1) = - (i + 1) \delta^{(i)}  (\mu - 1)$ we get
  (invoking the uniqueness part of fact \ref{fact:sing-q-3}) $(\nu - (i + 1))
  u_i = 0$. This implies that $F = 0$, unless $\nu \in \Z_{> 0}$.
  
  So, it remains to handle the case $\nu \in 2\mathbbm{Z}_{> 0}$ now. But then
  the conclusion is implied by lemmas \ref{lem:sing-q-6} and
  \ref{lem:sing-q-7}.
\end{proof}

\begin{lemma}
  \label{lem:sing-q-3}If $F \in \D' (\{ x \neq 0 \} \cap \{ y \neq 0
  \})$\footnote{$\{ x \neq 0 \} \assign \{ (x, y) \in \mathbbm{R}^{p, q} | x
  \neq 0 \}$ and similarly $\{ y \neq 0 \}$ is defined}, $\supp (F) = \{ Q = 0
  \}$ , then $F$ extends to $\tilde{F} \in \D' (\R^{p, q} \setminus \{ 0 \})$
  with $\supp (\tilde{F}) = \{ Q = 0 \}$ and any two such extensions would
  coincide. Moreover, if $F \in \mathcal{S} \tmop{ol}_{\{ Q = 0 \}} (\{ x \neq
  0 \} \cap \{ y \neq 0 \} ; \lambda, \nu)$, then $\tilde{F} \in \mathcal{S}
  \tmop{ol}_{\{ Q = 0 \}} \left( \R^{p, q} \setminus \{ 0 \} ; \lambda, \nu
  \right)$.
\end{lemma}

\begin{proof}
  Note that $\R^{p, q} \setminus \{ 0 \} = A \cup B \assign \{ (x, y) \in
  \R^{p, q} \backslash \{ 0 \} |x \neq 0, \hspace{0.75em} y \neq 0\} \cup \{
  (x, y) \in \R^{p, q} \backslash \{ 0 \} | | x | \neq | y | \}$. If we define
  $\tilde{F}$ as $F$ on $A$ and 0 on $B$, the Fact \ref{fact:localization} \
  shows that $\tilde{F}$ is a well-defined extension, and this proves the
  existence part. Regarding the uniqueness part, whatever extension
  $\tilde{F}$ would be, it should be equal to $F$ on $A$ and 0 on $B$, hence
  the uniqueness part of Fact \ref{fact:localization} \ grants the uniqueness.
  This proves the conclusion in the first sentence.
  
  Now, assume that $F \in \mathcal{S} \tmop{ol} (A ; \lambda, \nu)$. Then both
  $\tilde{F} \mid_A = F \in \mathcal{S} \tmop{ol} (A ; \lambda, \nu)$ and
  $F \mid_B = 0 \in \mathcal{S} \tmop{ol} (B ; \lambda, \nu)$. This grants
  the desired conclusion.
\end{proof}

\begin{lemma}
  \label{supp-Q:lem-sing-q-4}Suppose $F \in \D' (\R^{p, q})$, $F
  \mid_{\R^{p, q} \setminus \{ 0 \}}$ is $N_+'$-invariant on $\R^{p, q}
  \setminus \{ 0 \}$ and $F$ satisfies (\ref{Ndiff}) on $\R^{p, q}$. Then, $F$
  is $N_+'$-invariant on $R^{p, q}$.
\end{lemma}

\begin{fact}
\label{fact:sing-q-2}{\proofexplanation{{\cite[Thm
2.1.3]{hormander1983analysis}}}} Suppose $\phi \in C^{\infty}  (X \times Y)$
where $X, \hspace{0.25em} Y \in \R^m, \hspace{0.25em} \R^n$ are open sets and
$\phi (x, y) = 0$ if $x \nin K \subset X$, where $K$ is compact. Then for any
$u \in \D' (X)$ $y \mapsto u (\phi (\cdot, y))$ is smooth function and $\left.
\left. \frac{\partial}{\partial y_i}  (y \mapsto u (\phi (\cdot, y)) \right) =
u \left( \frac{\partial}{\partial y_i} u \right( \cdot, y)
\right)$.
\end{fact}

\begin{proof}
  Fix arbitrary $b \in \R^{p, q}$ with $b_p = 0$ and let $c_b (x) \assign 1 -
  2 Q (b, x) + Q (b) Q (x)$, $\psi_b (x) \assign (x - Q (x) b) / c_b (x)$.
  What needs to be shown is that near $x = 0$ we have $F (x) = F (\psi_b (x))
  c_b^{\lambda - n} (x)$.
  
  Let us take open $\R^{p, q} \supset U \ni \{ 0 \}$ such that $\forall (t, x)
  \in (- 1 / 2, 3 / 2) \times U$ we have $c_{tb} (x) \neq 0$ and let us
  further fix arbitrary $\phi \in C_0^{\infty} (V)$. Define $(- 3 / 2, 1 / 2)
  \ni t \mapsto f (t) \assign \langle F (\psi_{tb} (x)) c_{tb}^{\lambda - n},
  \phi \rangle \in \R$. Recalling what $F (\psi_{tb} (x)) c_{tb}^{\lambda -
  n}$ means in a distribution sense, we may write $\langle F (\psi_{tb} (x))
  c_{tb}^{\lambda - n}, \phi \rangle = \langle F, \phi_t \rangle$ for $\phi_t
  \in C^{\infty} ((- 1 / 2, 3 / 2) \times V)$ the deformation of $\phi$.
  Further restricting if necessary $\supp (\phi)$ we may ensure that $\phi_t$
  vanishes outside $(- 1 / 2, 3 / 2) \times K$ for some compact $K \times V$.
  Fact \ref{fact:sing-q-2} \ then tells us that $f$ is smooth and $f' (t) =
  \langle D_b F, \phi_t \rangle$, where $D_b = \sum_j b_j \mysbra{2 \nu
  \varepsilon_j x_j + \frac{1}{2} Q (x) \frac{\partial}{\partial x_j}}$ But
  then the hypothesis of the lemma implies that $D_b F = 0$, hence $f' (t) =
  0$, hence $f (1) = f (0)$ and this implies the desired conclusion.
\end{proof}

\begin{fact}
{\proofexplanation{{\cite[Thm. 2.3.5]{hormander1983analysis}}}}
\label{fact:sing-q-3}Let $x = (x', x'')$ be a splitting of the variables in
$\R^n$ in two groups. If $u \in \D' (\R^n)$ has compact support and the latter
is contained in the plane $x' = 0$,
\[ u = \sum_{| \alpha | \leq k} u_{\alpha} \delta^{(\alpha)} (x') \]
with $u_{\alpha}$ is distribution with compact support in the $x''$
variables.
\end{fact}

\begin{remark}
  Using the cutoff function, this fact can be applied to distributions $u$ of
  non-compact support, the sum $u = \sum_{| \alpha | \leq k} u_{\alpha}
  \delta^{(\alpha)} (x')$ then becoming locally compact. Also, such extension
  is easily seen to be unique.
\end{remark}

\begin{definition}
  \label{n-nonequiv:def-solprime}It is possible to weaken the definition
  \ref{sol:def-sol} by replacing the item 4 of it with the differential of
  that (local) action of $N_+'$ on $\mathbbm{R}^{p, q}$. We will call the
  latter $\mathfrak{n}_+'$-invariance. Explicitly written, it becomes the
  system of differential equations
  \begin{equation}
    \left[ (\lambda - n) \varepsilon_j x_j - \varepsilon_j x_j E + \frac{1}{2}
    Q (x) \frac{\partial}{\partial x_j} \right] F = 0, \quad j = \{ 1, 2,
    \ldots, n \} \backslash \{ p \} \label{Ndiff}
  \end{equation}
  with $E \assign \sum_j x_j  \frac{\partial}{\partial x_j}$ and
  $\varepsilon_j = + 1$ for $1 \leq j \leq p$ and $= - 1$ otherwise.
  
  For $(\lambda, \nu) \in \mathbbm{C}^2$ and $U \subset \mathbbm{R}^{p, q}$
  open, we let $\mathcal{S} \tmop{ol}' (U ; \lambda, \nu)$ to denote set of
  distributions $u \in \mathcal{D}' (U)$, satisfying first three items of
  definition \ref{sol:def-sol} and equations (\ref{Ndiff}).
  
  We will also use the notation $\mathcal{S} \tmop{ol}_S' (U ; \lambda, \nu)
  \assign \{ u \in \mathcal{S} \tmop{ol} (U ; \lambda, \nu) | \tmop{supp} (u)
  \subset S \}$ for $S \subset U$ closed.
\end{definition}

\begin{remark}
  We note that $\mathcal{S} \tmop{ol}'_S (U ; \lambda, \nu) \supset
  \mathcal{S} \tmop{ol}_S (U ; \lambda, \nu)$.
\end{remark}

\begin{lemma}
  \label{lem:sing-q-6}For $\nu \in \mathbbm{Z}_{> 0}$ we have $\mathcal{S}
  \tmop{ol}_{\{ Q = 0 \}}' (\{ x \neq 0 \} \cap \{ y \neq 0 \} ; \lambda, \nu)
  =\mathbbm{C}K^{(p)}_{\lambda, \nu}$ with $\mathcal{S} \tmop{ol}'_S (U ;
  \lambda, \nu)$ as in definition \ref{n-nonequiv:def-solprime}.
\end{lemma}

\begin{remark}
  $K^{(p)}_{\lambda, \nu}$ is defined as follows. For $(m, \mu) \in
  \mathbbm{Z}_{> 0} \times \mathbbm{C}$ we let
  \[ \mathcal{D}' (\mathbbm{R}^m \backslash \{ 0 \}) \ni f_{\mu}^{(m)} \assign
     \left\{ \begin{array}{ll}
       | x |^{\mu}, & m = 1\\
       | x |^{\mu} / \Gamma ((\mu + 1) / 2), & m > 1
     \end{array} \right. \]
  (see {\cite[ch. III, sec. 3.2, 3.3]{gelfand1980distribution}} for definition
  of distribution $| x |^{\lambda}$).
  
  Fix $\nu \in \mathbbm{Z}_{> 0}$ and $g \in \D' (\R_{> 0} \times \Sp^{p -
  1})$ be pullback of $f^{(p)}_{\lambda + \nu - n} (x_p) \in \mathcal{D}'
  (\mathbbm{R}^p \backslash \{ 0 \})$ via the polar coordinates.
  
  We introduce the coordinate system to parametrize $\{(x, y) \in \R^{p, q} |x
  \neq 0, \hspace{0.75em} y \neq 0\}$. These will be $(\mu, s, \omega_{p - 1},
  \omega_{q - 1})$ coordinates given by $\{ (x, y) \in \R^{p, q} |x \neq 0,
  \hspace{0.75em} y \neq 0\} \ni (x, y) = (\sqrt{s} \omega_{p - 1}, \sqrt{\mu
  s} \omega_{q - 1}$ with $(\mu, s, \omega_{p - 1}, \omega_{q - 1}) \in \R_{>
  0} \times \R_{> 0} \times \Sp^{p - 1} \times \Sp^{q - 1}$.
  
  We define $K^{(p)}_{\lambda, \nu} \in \D' (\R_{> 0} \times \R_{> 0} \times
  \Sp^{p - 1} \times \Sp^{q - 1})$ to be the distribution $K^{(p)}_{\lambda,
  \nu} \assign \delta^{(\nu - 1)}  (\mu - 1) \otimes s^{- \nu} g \otimes
  1_{\Sp^{q - 1}}$ in variables $(\mu, s, \omega_{p - 1}, \omega_{q - 1}) \in
  \R_{> 0} \times \left( \R_{> 0} \times \Sp^{p - 1} \right) \times \Sp^{q -
  1}$.
\end{remark}

\begin{proof}
  Applying fact \ref{fact:sing-q-3}, in $(\mu, s, \omega_{p - 1}, \omega_{q -
  1})$ coordinates $F$ should be of the form $F = \sum_{i \geq 0} \delta^{(i)}
  (\mu - 1) \otimes u_i$ with sum being locally finite and $u_i \in \D'
  (\R_{> 0} \times \Sp^{p - 1} \times \mathbbm{S}^{q - 1})$ being independent
  of $\mu$. Similarly as in proof of lemma \ref{uniq-c:lem-generic}, we can
  conclude that $F = \delta^{(\nu - 1)}  (\mu - 1) u$ globally with $u \in \D'
  (\R_{> 0} \times \mathbbm{S}^{p - 1} \times \mathbbm{S}^{q - 1})$ being
  independent of $\mu$.
  
  Furthermore, being an element of $\mathcal{S} \tmop{ol}_{\{ Q = 0 \}} (\{ x
  \neq 0 \} \cap \{ y \neq 0 \} ; \lambda, \nu)$ implies homogeneity with
  degree $\lambda - \nu - n$ and hence that the Euler equation $EF = (\lambda
  - \nu - n) F$ should hold, with Euler operator $E$ in $(\mu, s, \omega_{p -
  1}, \omega_{q - 1})$ coordinates being written as $E = 2 s
  \frac{\partial}{\partial s}$, hence $\frac{\partial}{\partial s} u =
  \frac{\lambda - \nu - n}{2} u$. Hence, applying fact \ref{fact:sing-q-4} \
  we see that $s^{- \frac{\lambda - \nu - n}{2}} u$ is independent of $s$,
  hence we may write $u = s^{\frac{\lambda - \nu - n}{2}} v$ with $v \in \D'
  (\mathbbm{S}^{p - 1} \times \mathbbm{S}^{q - 1})$ independent of $s$.
  
  As $F$ has to be invariant under left multiplication by $O (q) \subset O (p,
  q)_{e_p}$, we see that in fact $v$ is independent of variables of
  $\mathbbm{S}^{q - 1}$, hence may be treated as $v \in \D' (\mathbbm{S}^{p -
  1})$. To finish the proof, it now suffices to show that under the polar
  coordinates transformation, $\tilde{v} \assign s^{\lambda + \nu - n} v \in
  \D' (\R_{> 0} \times \mathbbm{S}^{p - 1})$ becomes proportional to
  $f_{\lambda + \nu - n}^{(p)} (x_p)$. In the light of lemmas
  \ref{lem67:lem-tensor} and \ref{lem67:lem-homogR}, for this it suffices to
  show that $(\partial / \partial x_i)  \tilde{v} = 0$ and that $\tilde{v}$ is
  homogeneous of degree $\lambda + \nu - n$ (the latter is obvious from the
  above).
  
  To derive the required equalities we will need to understand how equations
  (\ref{Ndiff}) get written in $(\mu, s, \omega_{p - 1}, \omega_{q - 1})$
  coordinates and how $(\partial / \partial x_i)  \tilde{v} = 0$ get written.
  We assume that we fix two parametrizations $\R^{p - 1} \supset U \ni
  (z_i)_{i = 1}^{p - 1} \mapsto (\omega_{p - 1}^{(j)} (z))_{j = 1}^p \in
  \mathbbm{S}^{p - 1} \subset \R^p$ and $\R^{q - 1} \supset U \ni (w_i)_{i =
  1}^{q - 1} \mapsto (\omega_{q - 1}^{(j)} (w))_{j = 1}^q \in \mathbbm{S}^{q -
  1} \subset \R^q$. We will denote the first order derivatives of these by
  $D_{p - 1}$ and $D_{q - 1}$ respectively (these being $p - 1 \times p$ and
  $q - 1 \times q$ matrices respectively). We will also use shorthands to
  denote column-vectors: $\partial / \partial z \assign \left[
  \frac{\partial}{\partial z_i} \right]_i$ and $\partial / \partial w \assign
  \left[ \frac{\partial}{\partial w_j} \right]_j$
  
  We will start with the former task. Under the polar parametrization $x =
  \sqrt{s} \cdot \omega_{p - 1}$ the desired equality $(\partial / \partial
  x_i)  \tilde{v} = 0$ for $1 \leq i \leq p - 1$ gets written as
  \[ \left[ \frac{1}{\sqrt{s}} \omega_{p - 1}^{(i)} 2 s
     \frac{\partial}{\partial s} + \frac{1}{\sqrt{s}}  \left( J_{p - 1} 
     \frac{\partial}{\partial z} \right)_i \right]  \tilde{v} = 0, \quad 1
     \leq i \leq p - 1 \]
  as we further know that $\tilde{v}$ is homogeneous of degree $\lambda + \nu
  - n$ and Euler operator is written as $E = 2 s \frac{\partial}{\partial s}$,
  this can be rewritten as
  \begin{equation}
    \left[ \omega_{p - 1}^{(i)} (\lambda + \nu - n) + \left( J_{p - 1} 
    \frac{\partial}{\partial z} \right)_i \right]  \tilde{v} = 0, \quad 1 \leq
    i \leq p - 1 \label{eq:sing-q-dx}
  \end{equation}
  Furthermore, differentiating condition 2 of definition of $\mathcal{S}
  \tmop{ol} (\R^{p, q} ; \lambda, \nu)\{(x, y) \in \R^{p, q} |x \neq 0,
  \hspace{0.75em} y \neq 0\}$ for $F \in \mathcal{S} \tmop{ol} (\R^{p, q} ;
  \lambda, \nu)\{(x, y) \in \R^{p, q} |x \neq 0, \hspace{0.75em} y \neq 0\}$
  we see that (using the usual splitting $(x, y) \in \R^{p, q}$) we should
  have
  \[ [x_i \partial y_j + y_j \partial x_i] F = 0, \quad 1 \le i \le p - 1,
     \hspace{0.75em} 1 \leq j \leq q \]
  which in $(\mu, s, \omega_{p - 1}, \omega_{q - 1})$ gets written as
  \[ \sqrt{s} \omega^{(i)}_{p - 1}  \left[ \omega^{(j - 1)}_{q - 1}  \frac{2
     \sqrt{\mu}}{\sqrt{s}} \cdot \frac{\partial}{\partial \mu} +
     \frac{1}{\sqrt{s \mu}}  \left( J_{q - 1}  \frac{\partial}{\partial w}
     \right)_j \right] F + \]
  \[ \sqrt{s \mu} \omega^{(j)}_{q - 1}  \left[ \omega_{p - 1}^{(i)}  \left( -
     2 \mu / \sqrt{s}  \frac{\partial}{\partial \mu} + 2 \sqrt{s} 
     \frac{\partial}{\partial s} \right) + \frac{1}{\sqrt{s}}  \left( J_{p -
     1}  \frac{\partial}{\partial z} \right)_i \right] F = 0. \]
  As $F$ is independent of $w$ (as shown above) this gets rewritten as
  \[ \omega^{(i)}_{p - 1}  \left[ \omega^{(j - 1)}_{q - 1}  \frac{2}{\sqrt{s}}
     \cdot \frac{\partial}{\partial \mu} \right] F + \omega^{(j)}_{q - 1} 
     \left[ \omega_{p - 1}^{(i)}  \left( - 2 \mu / \sqrt{s} 
     \frac{\partial}{\partial \mu} + 2 \sqrt{s}  \frac{\partial}{\partial s}
     \right) + \frac{1}{\sqrt{s}}  \left( J_{p - 1}  \frac{\partial}{\partial
     z} \right)_i \right] F = 0. \]
  We also can choose $j$, so that $\omega^{(j)}_{q - 1} \neq 0$ locally, hence
  we can divide it and $\sqrt{s}$ out to get
  \[ \omega^{(i)}_{p - 1}  \left[ 2 \cdot \frac{\partial}{\partial \mu}
     \right] F + \left[ \omega_{p - 1}^{(i)}  \left( - 2 \mu
     \frac{\partial}{\partial \mu} + 2 s \frac{\partial}{\partial s} \right) +
     \left( J_{p - 1}  \frac{\partial}{\partial z} \right)_i \right] F = 0. \]
  where homogeneity of order $\lambda - \nu - n$ of $F$ renders this into
  \[ \omega^{(i)}_{p - 1} 2 (1 - \mu) \cdot \frac{\partial}{\partial \mu} F +
     \left[ \omega_{p - 1}^{(i)}  (\lambda - \nu - n) + \left( J_{p - 1} 
     \frac{\partial}{\partial z} \right)_i \right] F = 0. \]
  substituting here $F = \delta^{(\nu - 1)}  (\mu - 1) u$ and using $(\mu - 1)
  \frac{\partial}{\partial \mu} \delta^{(i)}  (\mu - 1) = - (i + 1)
  \delta^{(i + 1)}  (\mu - 1)$, we get
  \[ 2 \nu \omega^{(i)}_{p - 1} u + \left[ \omega_{p - 1}^{(i)}  (\lambda -
     \nu - n) + \left( J_{p - 1}  \frac{\partial}{\partial z} \right)_i
     \right] u = 0. \]
  and finally substituting $u = s^{- 2 \nu}  \tilde{v}$ one gets
  \[ \left[ \omega_{p - 1}^{(i)}  (\lambda + \nu - n) + \left( J_{p - 1} 
     \frac{\partial}{\partial z} \right)_i \right]  \tilde{v} = 0,
     \hspace{0.75em} 1 \leq i \leq p - 1. \]
  and as the latter is precisely (\ref{eq:sing-q-dx}), this finishes the
  proof.
\end{proof}

\begin{lemma}
  \label{supp-Q:lem-flip}For $x, b \in \mathbbm{R}^{p, q}$ we let $c_b (x)
  \assign 1 - 2 Q (x, b) + Q (x) Q (b)$ and $\psi_b (x) \assign (x - Q (x) b)
  / c_b (x)$. Then for every $x \in \mathbbm{R}^{p, q} \backslash \{ 0 \}$
  with $Q (x) = 0$ there exists $b \in \mathbbm{R}^{p, q}$ with $b_p = 0$ such
  that $c_b (x) < 0$.
\end{lemma}

\begin{proof}
  As under the hypothesis taken $c_b (x) = 1 - 2 Q (x, b)$, the statement
  follows.
\end{proof}

\begin{lemma}
  \label{supp-Q:lem-sing-q-7-aux}For arbitrary $x \in \mathbbm{R}^{p, q}
  \backslash \{ 0 \}$ with $Q (x) = 0$ there exists $b \in \mathbbm{R}^{p, q}$
  with $b_p = 0$ such that for $\tilde{x}$ near $x$
  \[ \left( \frac{Q_+^{\mu} - Q_-^{\mu}}{\Gamma ((\mu + 2) / 2)} \right)
     (\psi_b (\tilde{x})) = - \frac{1}{| c_b (\tilde{x}) |^{\mu}} \left(
     \frac{Q_+^{\mu} - Q_-^{\mu}}{\Gamma ((\mu + 2) / 2)} \right) (\tilde{x})
  \]
\end{lemma}

\begin{proof}
  Lemma \ref{supp-Q:lem-flip} gives us $b$ such that $b_p = 0$ and $c_b (x) <
  0$, and since the latter inequality is strict, it also holds for $\tilde{x}$
  near $x$. When seen as distributions on $\mathbbm{R}^{p, q} \backslash \{ 0
  \}$, both sides are holomorphic in $\mu \in \mathbbm{C}$, thus it suffices
  to show the equality for $\tmop{Re} (\mu) \gg 0$. Thus, we can forget about
  gamma-multipliers.
  
  Observe that $Q (\psi_b (\tilde{x})) = Q (\tilde{x}) / c_b (\tilde{x})
  \nocomma$, hence $| Q |^{\mu} (\psi_b (\tilde{x})) = | Q |^{\mu} (\tilde{x})
  / | c_b (\tilde{x}) |^{\mu} \nocomma$. For $\tilde{x} \in \{ Q = 0 \}$ the
  statement then holds trivially, as both sides vanish. When $\tilde{x} \in \{
  Q > 0 \}$ we have $\psi_b (\tilde{x}) \in \{ Q < 0 \}$ by the observation
  above, hence the right-hand side being equal to $- | Q |^{\mu}
  (\tilde{x}) / | c_b (\tilde{x}) |^{\mu}$, while left-hand side equals to $-
  | Q |^{\mu} (\psi_b (\tilde{x}))$ and the observation above grants the
  desired conclusion. Similarly, case $\tilde{x} \in \{ Q < 0 \}$ is handled.
\end{proof}

\begin{lemma}
  \label{lem:sing-q-7}With $K_{\lambda, \nu}^{(p)} \in \mathcal{D}' (\{ x \neq
  0 \} \cap \{ y \neq 0 \})$ as in lemma \ref{lem:sing-q-6}, we have for $\nu
  \in 2\mathbbm{Z}_{> 0}$ that $K_{\lambda, \nu}^{(p)}$ is not
  $N_+'$-invariant.
\end{lemma}

\begin{proof}
  We note that for $f_{\mu}^m \in \mathcal{D}' (\mathbbm{R}^m \backslash \{ 0
  \})$ as in lemma \ref{lem:sing-q-6}, we have product of distributions
  $(f_{\lambda + \nu - n}^p (x_p) \otimes 1_{\{ y \neq 0 \}}) \cdot
  \delta^{(\nu - 1)} (Q) \in \mathcal{D}' (\{ x \neq 0 \} \cap \{ y \neq 0
  \})$ being proportional to $K_{\lambda, \nu}^{(p)}$ by lemmas
  \ref{KR-normalization-recur:lem-mult-comm-pull} and
  \ref{KR-normalization-recur:lem-mult-distrib-tensor}. Then, it suffices to
  prove the statement for $(f_{\lambda + \nu - n}^p (x_p) \otimes 1_{\{ y \neq
  0 \}}) \cdot \delta^{(\nu - 1)} (Q)$, which we'll call $G$ (or $G_{\lambda,
  \nu}$) for brevity till the end of the proof. First we can fix $\nu \in
  2\mathbbm{Z}_{> 0}$ and assume that $\tmop{Re} (\lambda) \gg 0$ and thus
  ignore the $\Gamma (\cdot)$-multiple in $f_{\lambda + \nu - n}^p$, if it was
  there.
  
  The residue information obtained in {\cite[ch. III, sec
  2.2]{gelfand1980distribution}} tells us that restricted to $\R^{p, q}
  \setminus \{ 0 \}$ we have
  \[ \delta^{(2 k - 1)} (Q) = \frac{Q_+^{- \mu} - Q_-^{- \mu}}{\Gamma \left(
     \frac{- \mu + 2}{2} \right)} \mid_{\mu = 2 k}, \quad k \in \Z_{> 0}
  \]
  
  
  and right hand side is holomorphic in $\mu \in \mathbbm{C}$. Hence lemma
  similar to \ref{supp-n-waves:lem-|x|-holo-in} for $x_+^{\mu} - x_-^{\mu}$ in
  place of $| x |^{\mu}$ and lemmas
  \ref{holomorphicity-preserving:prop-pullback-holo} and
  \ref{holomorphicity-preserving:prop-tensor-holo} tell us that
  \[ G_{\lambda, \nu} = \lim_{\mu \to \nu = 2 k}  \frac{(Q_+^{- \mu} -
     Q_-^{- \mu}) | x_p |^{\lambda + \mu - n}}{\Gamma \left( \frac{- \mu +
     2}{2} \right)}, \quad k \in \Z_{> 0} . \]
  Thus we can call distribution under the last limit as $G_{\lambda, \mu}$ ,
  allowing $\mu \in \mathbbm{C}$.
  
  Now, we fix arbitrary $x \in \{ x \neq 0 \} \cap \{ y \neq 0 \}$ with $Q (x)
  \neq 0$. Lemma \ref{supp-Q:lem-sing-q-7-aux} gives us $b \in \mathbbm{R}^{p,
  q}$ with $b_p = 0$ for this $x$. As $x$ and $x - Q (x) b$ have the same
  $p$-th coordinate, conclusion of lemma \ref{supp-Q:lem-sing-q-7-aux} tells
  us that
  \[ G_{\lambda, \mu} (\psi_b (\tilde{x})) = - | c_b (\tilde{x}) |^{- (\lambda
     - n)} G_{\lambda, \mu} (\tilde{x}) \]
  holding for $\tilde{x}$ near $x$. Now, as both sides are holomorphic, we can
  take the $\mu \rightarrow \nu = 2 k \in 2\mathbbm{Z}_{> 0}$ limit to get
  \[ G_{\lambda, \nu} (\psi_b (\tilde{x})) = - | c_b (\tilde{x}) |^{- (\lambda
     - n)} G_{\lambda, \nu} (\tilde{x}) \]
  Furthermore, by analytic continuation this relation also holds for all
  $\lambda \in \mathbbm{C}$. Finally, assuming (in order to get a
  contradiction) that $G_{\lambda, \nu}$ is $N_+'$-invariant, we should have
  as well that
  \[ G_{\lambda, \nu} (\psi_b (\tilde{x})) = | 1 - 2 Q (b, \tilde{x}) + Q
     (\tilde{x}) Q (b) |^{- (\lambda - n)} G_{\lambda, \nu} (\tilde{x}) \]
  and together the latter two equalities imply that $G_{\lambda, \nu}
  (\tilde{x}) = 0$ near $x \in \{ Q = 0 \}$.
  
  To reach the contradiction it suffices now to show that $G_{\lambda, \nu}$
  is supported at least on $\{ Q = 0 \} \cap \{ x_p = 0 \}$ for $p > 1$ and at
  least on $\{ Q = 0 \}$ when $p = 1$. Now, the latter is implied by lemma
  \ref{supp-n-waves:lem-at-least} and the fact that for $p = 1$
  $(\mathbbm{R}^n \backslash \{ 0 \}) \cap \{ Q = 0 \} \subset \{ x_p \neq 0
  \}$ and $| x_p |^{\lambda + \nu - n}$ is smooth nonvanishing on this set. In
  turn, the former is implied for $(\lambda, \nu) \nin \backslash\backslash$
  by lemmas \ref{supp-n-waves:lem-at-least} and
  \ref{supp-n-waves:lem-supp-xp}, and for $(\lambda, \nu) \in
  \backslash\backslash$ by an argument similar to that of lemma
  \ref{supp-R:lem-supp-of-K}.
\end{proof}

\begin{lemma}
  \label{supp-Q:lem-operator}For $F \in \mathcal{S} \tmop{ol} (\{ (x, y) \in
  \R^{p, q} |x \neq 0, \hspace{0.75em} y \neq 0\}; \lambda, \nu)$ when
  transfering to $(\mu, s, \omega_{p - 1}, \omega_{q - 1})$ coordinates (with
  domain $(\mu, s, \omega_{p - 1}, \omega_{q - 1}) \in \mathbbm{R}_{> 0}^2
  \times \mathbbm{S}^{p - 1} \times \mathbbm{S}^{q - 1}$ and related to $(x,
  y) \in \{ (x, y) \in \R^{p, q} |x \neq 0, \hspace{0.75em} y \neq 0\}$ via $x
  = \sqrt{s} \omega_{p - 1}, y = \sqrt{\mu s} \omega_{q - 1}$) we have
  \[ \mysbra{\nu + (\mu - 1)  \frac{\partial}{\partial \mu}} F = 0. \]
\end{lemma}

\begin{proof}
  As $F \in \mathcal{S} \tmop{ol} (\{ (x, y) \in \R^{p, q} |x \neq 0,
  \hspace{0.75em} y \neq 0\}; \lambda, \nu)$, this in particular implies
  $N_+'$-invariance, hence equations (\ref{Ndiff}). Writing elements of
  $\R^{p, q}$ as $(x, y)$ the last $q$ of these equations get written as
  \[ \left[ - 2 \nu y_j + (| x |^2 - | y |^2) \frac{\partial}{\partial y_j}
     \right] F = 0, \quad 1 \leq j \leq q \]
  In turn, in bipolar coordinates $(x, y) = (r_1 \omega_{p - 1}, r_2 \omega_{q
  - 1})$, where $\partial y_j / \partial r = y_j / r_2$ these get written as
  \[ \mysbra{- 2 \nu \frac{y_j^2}{r_2} + (r_1^2 - r_2^2)  \frac{\partial
     y_j}{\partial r_2}  \frac{\partial}{\partial y_j}} F = 0, \quad 1 \leq j
     \leq q \]
  and summing these up and writing in $(\mu, s, \omega_{p - 1}, \omega_{q -
  1})$ we get
  \[ \mysbra{\nu + (\mu - 1)  \frac{\partial}{\partial \mu}} F = 0 \]
  and this ends the proof.
\end{proof}

\subsection{Proofs}

\begin{proof}
  (of prop. \ref{sol-MO:prop-solCnonzero}) The case $(\lambda, \nu) \nin \mid
  \mid$ is readily handled by lemma \ref{uniq-c:lem-generic}, so we assume the
  contrary in subsequent. Now, lemma \ref{supp-R:lem-supp-of-K} readily grants
  the $\supseteq$ part. Therefore, it suffices to show that $\dim (\mathcal{S}
  \tmop{ol}_C (\mathbbm{R}^n - \{ 0 \} ; \lambda, \nu)) \leqslant 1$. In the
  light of lemma \ref{lem:sing-q-3} and remark after the definition
  \ref{n-nonequiv:def-solprime}, it suffices to show that
  \[ \dim (\mathcal{S} \tmop{ol}_C' (\{ x \neq 0, y \neq 0 \} ; \lambda,
     \nu)) \leqslant 1. \]
  As the latter is readily granted by lemma \ref{lem:sing-q-6}, we are done.
\end{proof}

\section{Determination of $\mathcal{S} \tmop{ol} (\mathbbm{R}^n ; \lambda,\nu)$ (B14)}\label{sec:sol-MO}


In this section we explicitly determine the space $\mathcal{S} \tmop{ol}
(\mathbbm{R}^n ; \lambda, \nu)$ for every $(\lambda, \nu) \in \mathbbm{C}^2$.

\subsection{Main results}

\begin{proposition}
  \label{sol-MO:prop-solonnonzero}We have
  \begin{eqnarray}
    & \mathcal{S} \tmop{ol} (\mathbbm{R}^n - \{ 0 \} ; \lambda, \nu) =
    \left\{ \begin{array}{ll}
      \mathbbm{C} | Q |^{- \nu} \delta^{(2 k)} (x_p) \oplus \mathbbm{C} | x_p
      |^{\lambda + \nu - n} \delta^{(\nu - 1)} (Q), & p = 1, \; (\lambda, \nu)
      \in \mid \mid \cap \backslash\backslash, k : = - \frac{\lambda + \nu - n
      + 1}{2}\\
      \mathbbm{C} \frac{| x_p |^{\lambda + \nu - n}}{\Gamma \left(
      \frac{\lambda + \nu - n + 1}{2} \right)} \cdot \frac{| Q |^{-
      \nu}}{\Gamma \left( \frac{1 - \nu}{2} \right)}, & \tmop{otherwise}
    \end{array} \right. &  \nonumber
  \end{eqnarray}
\end{proposition}

\begin{proposition}
  \label{sol-MO:prop-main}For $p = 1$ we have
  \[ \mathcal{S} \tmop{ol} (\mathbbm{R}^n ; \lambda, \nu) = \left\{
     \begin{array}{ll}
       \tilde{K}_{\lambda, \nu}^{\mathbbm{R}^n}, & (\lambda, \nu) \in
       \mathbbm{C}^2 - (/ / \cap \mid \mid \mid) - (| | \cap
       \backslash\backslash)\\
       \widetilde{\tilde{K}}_{\lambda, \nu}^{\mathbbm{R}^n} \oplus
       \tilde{K}^{\{ 0 \}}_{\lambda, \nu}, & (\lambda, \nu) \in (/ / \cap \mid
       \mid \mid) - (| | \cap \backslash\backslash)\\
       \tilde{K}_{\lambda, \nu}^P \oplus \tilde{K}_{\lambda, \nu}^C, &
       (\lambda, \nu) \in (| | \cap \backslash\backslash) - / /\\
       \tilde{K}^{\{ 0 \}}_{\lambda, \nu}, & (\lambda, \nu) \in \mid
       \mid \cap \backslash\backslash \cap / /
     \end{array} \right. \]
  whereas for $p > 1$ we have
  \[ \mathcal{S} \tmop{ol} (\mathbbm{R}^n ; \lambda \comma \nu) = \left\{
     \begin{array}{ll}
       \widetilde{\tilde{K}}_{\lambda, \nu}^{\mathbbm{R}^n} \oplus
       \tilde{K}^{\{ 0 \}}_{\lambda, \nu}, & (\lambda, \nu) \in / / \cap \mid
       \mid \mid,\\
       \tilde{K}^{\mathbbm{R}^n}_{\lambda, \nu}, & \tmop{otherwise}
     \end{array} \right. \]
\end{proposition}

\subsection{Auxiliary results}

\begin{lemma}
  \label{sol-MO:lem-E2}Suppose $\{ 0 \} \in O \subset \mathbbm{R}^n$ is open
  and $M$ a manifold. Suppose further that for $f \in \mathcal{D}' (O \times
  M)$ we have $f$ supported inside $\{ 0 \} \times M$ and let $E = \sum x_i
  \frac{\partial}{\partial x_i}$ be an Euler operator on $\mathbbm{R}^n$ and
  $a \in \mathbbm{C}$. Then we have
  \[ (E - a)^2 f = 0 \Rightarrow (E - a) f = 0 \]
\end{lemma}

\begin{remark}
  This is a straightforward generalization of {\cite[lem.
  11.11]{kobayashi2015symmetry}} with proof done essentially in the same way.
\end{remark}

\begin{proof}
  As fact \ref{fact:sing-q-3} (together with the localization argument by fact
  \ref{fact:localization}) implies,
  \[ f = \sum_{\alpha} \delta^{(\alpha)} \otimes f_{\alpha}, \; f_{\alpha} \in
     \mathcal{D}' (M) . \]
  and hence
  \[ (E - a)^2 f = \sum_{\alpha} \delta^{(\alpha)} \otimes (| \alpha |^{} -
     a)^2 f_{\alpha} = 0 \]
  hence $\forall \alpha, \; (| \alpha |^{} - a)^2 f_{\alpha} = 0$. Now, for
  every fixed $\alpha$ the latter implies that either $(| \alpha | - a) = 0$
  or $f_{\alpha} = 0$, hence in any case $(| \alpha |^{} - a)^{} f_{\alpha} =
  0$ and thus $(E - a) f = 0$.
\end{proof}

\begin{lemma}
  \label{sol-MO:lem-zeromap-point}Suppose $S \subset \mathbbm{R}^n$ is closed
  such that $\dim (\mathcal{S} \tmop{ol}_S (\mathbbm{R}^{p, q} \backslash \{ 0
  \} ; \lambda, \nu)) \leqslant 1$. Suppose further $0 \in \Omega \subset
  \mathbbm{C}$ is an open set with $\lambda (\cdot), \nu (\cdot)$ holomorphic
  on $\Omega$ and such that $\lambda (\mu) - \nu (\mu) - \mu = \tmop{const}$
  on $\Omega$. Suppose further that $K^{(\mu)} \in \mathcal{S} \tmop{ol}_S
  (\mathbbm{R}^{p, q} ; \lambda (\mu), \nu (\mu))$, $K^{(0)}$ is supported
  at $\{ 0 \}$ and $(d / d \mu) \mid_{\mu = 0} K^{(\mu)}$ is supported
  at closed subset bigger than $\{ 0 \}$.
  
  Then, the restriction map $\mathcal{S} \tmop{ol}_S (\mathbbm{R}^{p, q} ;
  \lambda (0), \nu (0)) \rightarrow \mathcal{S} \tmop{ol}_S (\mathbbm{R}^{p,
  q} \backslash \{ 0 \} ; \lambda (0), \nu (0))$ is a zero map.
\end{lemma}

\begin{remark}
  This is essentially a slight generalization of {\cite[lemma
  11.8]{kobayashi2015symmetry}} with tecnhique of proof being exactly the
  same.
\end{remark}

\begin{proof}
  Indeed, suppose $F \in \mathcal{S} \tmop{ol}_S (\mathbbm{R}^{p, q} ; \lambda
  (0), \nu (0))$. We will show that $F \mid_{\mathbbm{R}^n \backslash \{ 0
  \}} = 0$. Indeed, we first note that expanding $K^{(\mu)}$ in Taylor
  series near $\mu = 0$ we have
  \[ K^{(\mu)} = K_0 + \mu \cdot K_1 + \mu^2 \cdot K_2 + \ldots \]
  and the hypothesis now implies that $K_0$ is supported at $\{ 0 \}$ and $K_1
  \mid_{\mathbbm{R}^n \backslash \{ 0 \}} \neq 0$.
  
  We also note that $K_1 \in \mathcal{S} \tmop{ol}_S (\mathbbm{R}^n
  \backslash \{ 0 \} ; \lambda, \nu)$, as $F^{(\mu)} \assign K^{(\mu)} / \mu
  \mid_{\mathbbm{R}^n \backslash \{ 0 \}} \in \mathcal{S} \tmop{ol}_S
  (\mathbbm{R}^{p, q} \backslash \{ 0 \} ; \lambda (\mu), \nu (\mu))$ for $\mu
  \neq 0$ and as $F^{(0)} = K_1$ and $F^{(\mu)}$ is holomorphic at $\mu = 0$,
  proposition \ref{sol:prop-holocont} implies that $K_1 \in \mathcal{S}
  \tmop{ol} (\mathbbm{R}^n \backslash \{ 0 \} ; \lambda, \nu)$. The fact
  that $K_1$ vanishes outside $S$ follows by continuity.
  
  Now, as $\dim (\mathcal{S} \tmop{ol}_S (\mathbbm{R}^{p, q} \backslash \{ 0
  \} ; \lambda, \nu)) \leqslant 1$ we should have
  \[ F \mid_{\mathbbm{R}^n \backslash \{ 0 \}} = c \cdot K_1
     \mid_{\mathbbm{R}^n \backslash \{ 0 \}} \]
  and it suffices to show that $c = 0$.
  
  Next, hypothesis $\lambda (\mu) + \nu (\mu) - \mu = \tmop{const}$ implies
  that for some $a \in \mathbbm{C}$ we have $\lambda (\mu) - \nu (\mu) - n = a
  + \mu$ and as member of $\mathcal{S} \tmop{ol}_S (\mathbbm{R}^n ; \lambda,
  \nu)$ should be homogeneous of degree $\lambda + \nu - n$, we should have
  \[ (E - a) K^{(\mu)} = \mu \cdot K^{(\mu)}, \]
  and therefore by {\cite[lem. 11.10]{kobayashi2015symmetry}}, we have
  \[ (E - a) K_0 = 0, \qquad (E - a) K_1 = K_0 . \]
  Moreover, as $F \in \mathcal{S} \tmop{ol}_S (\mathbbm{R}^{p, q} ; \lambda
  (0), \nu (0))$ by assumption, we should have $(E - a) F = 0$.
  
  Now, for $h \assign F - c \cdot K_1$ supported inside $\{ 0 \}$ distribution
  we should have $(E - a)^2 h = 0$, hence lemma \ref{sol-MO:lem-E2} implies
  that $(E - a) h = 0$ and therefore $0 = (E - a) F = c (E - a) K_1 = c \cdot
  K_0$, hence $c = 0$.
\end{proof}

\begin{lemma}
  \label{sol-MO:lem-zeromap}For exact sequence
  \[ 0 \rightarrow \mathcal{S} \tmop{ol}_C (\mathbbm{R}^n - \{ 0 \} ; \lambda,
     \nu) \rightarrow \mathcal{S} \tmop{ol} (\mathbbm{R}^n - \{ 0 \} ;
     \lambda, \nu) \xrightarrow{\pi} \mathcal{S} \tmop{ol} (\mathbbm{R}^n - C
     ; \lambda, \nu) \]
  and when
  \begin{eqnarray}
    & (\lambda, \nu) \in \left\{ \begin{array}{ll}
      \mid \mid, & p > 1\\
      \mid \mid \um \backslash\backslash, & p = 1
    \end{array} \right. &  \nonumber
  \end{eqnarray}
  with $\mid \mid$ and $\backslash\backslash$ as in proposition
  \ref{sol-MO:prop-solonnonzero}, we have $\pi = 0$.
\end{lemma}

\begin{proof}
  We assume
  \[ (\lambda_0, \nu_0) \in \left\{ \begin{array}{ll}
       \mid \mid, & p > 1\\
       \mid \mid \um \backslash\backslash, & p = 1
     \end{array} \right. \]
  and let
  \[ K^{(\nu)} \assign \frac{| x_p |^{\lambda_0 + \nu - n}}{\Gamma \left(
     \frac{\lambda_0 + \nu - n + 1}{2} \right)} \cdot \frac{| Q |^{-
     \nu}}{\Gamma \left( \frac{1 - \nu}{2} \right)} \in \mathcal{S} \tmop{ol}
     (\mathbbm{R}^n - \{ 0 \} ; \lambda_0, \nu) . \]
  Now, if we write the Taylor expansion $K^{(\nu)} = K_0 + (\nu - \nu_0)
  K_1 + (\nu - \nu_0)^2 K_2 + \ldots$, the hypothesis together with lemma
  \ref{supp-R:lem-supp-of-K} imply that $K_0 \neq 0$ is supported within $\{ Q
  = 0 \}$. Moreover, considering $K^{(\nu)} \mid_{\{ Q \neq 0 \}} / (\nu -
  \nu_0)$ which is holomorphic at $\nu_0$, we note that $K_1 \mid_{\{ Q
  \neq 0 \}} \in \mathcal{S} \tmop{ol} (\{ Q \neq 0 \} ; \lambda_0, \nu_0)$
  and is nonzero (the latter follows, since we notice that $K_1 \mid_{\{ Q
  \neq 0 \}}$ is nonzero multiple of $\frac{| x_p |^{\lambda_0 + \nu_0 -
  n}}{\Gamma \left( \frac{\lambda_0 + \nu_0 - n + 1}{2} \right)} \cdot | Q
  |^{- \nu_0} \mid_{\{ Q \neq 0 \}}$, which is nonzero).
  
  Now, we assume that $F \in \mathcal{S} \tmop{ol} (\mathbbm{R}^n \backslash
  \{ 0 \} ; \lambda_0, \nu_0)$ and show that $F \mid_{\{ Q \neq 0 \}}$=0,
  this will be sufficient to end the proof. As $\mathcal{S} \tmop{ol} (\{ Q
  \neq 0 \} ; \lambda_0, \nu_0)$ is at most-one dimensional (by proposition
  \ref{lem67:prop-67}) and $0 \neq K_1 \mid_{\{ Q \neq 0 \}} \in
  \mathcal{S} \tmop{ol} (\{ Q \neq 0 \} ; \lambda_0, \nu_0)$, we have $K_1
  \mid_{\{ Q \neq 0 \}} = c \cdot F \mid_{\{ Q \neq 0 \}}$ and it now
  suffices to show that $c = 0$.
  
  We now let $U \assign \{ (x, y) \in \mathbbm{R}^{p, q} | x \neq 0, y \neq 0
  \}$, $k^{(\nu)} \assign K^{(\nu)} \mid_U \in \mathcal{S} \tmop{ol} (U ;
  \lambda_0, \nu)$ with corresponding Taylor expansion $k^{(\nu)} = k_0 + (\nu
  - \nu_0) k_1 + \ldots$ with $k_i = K_i \mid_U$, and $f \assign F
  \mid_U \in \mathcal{S} \tmop{ol} (U ; \lambda_0, \nu_0)$. The above
  implies that $c \cdot k_1 \mid_{\{ Q \neq 0 \}} = f \mid_{\{ Q \neq
  0 \}}$. We also note that as $\mathcal{D}' (\mathbbm{R}^n \backslash \{ 0
  \}) \ni K_0 \neq 0$ and supported within $\{ Q = 0 \}$, we have $k_0 \neq 0$
  (as $\{ Q = 0 \}$ and $\mathbbm{R}^p \times \{ 0 \} \cup \{ 0 \} \times
  \mathbbm{R}^q$ are closed and disjoint inside $\mathbbm{R}^n \backslash \{ 0
  \}$). Now, lemma \ref{supp-Q:lem-operator} implies that (passing to $(\mu,
  s, \omega_{p - 1}, \omega_{q - 1})$ coordinates) we have $\left[ \nu_0 -
  (\mu - 1) \frac{\partial}{\partial \mu} \right] F = 0$ and $\left[ \nu -
  (\mu - 1) \frac{\partial}{\partial \mu} \right] k^{(\nu)} = 0$. Moreover,
  the latter and {\cite[lem. 11.10]{kobayashi2015symmetry}} imply than that
  \[ \left[ \nu_0 - (\mu - 1) \frac{\partial}{\partial \mu} \right] k_0 = 0,
     \quad \left[ \nu_0 - (\mu - 1) \frac{\partial}{\partial \mu} \right] k_1
     + k_0 = 0. \]
  Now, we have $h \assign F_0 - c K_1 \in \mathcal{D}' (U)$ being supported
  inside $\{ Q = 0 \}$ and moreover that $\left[ \nu_0 - (\mu - 1)
  \frac{\partial}{\partial \mu} \right]^2 h = 0$. Hence, lemma
  \ref{sol-MO:lem-E2} implies that $\left[ \nu_0 - (\mu - 1)
  \frac{\partial}{\partial \mu} \right]^{} h = 0$ and thus $c \left[ \nu_0 -
  (\mu - 1) \frac{\partial}{\partial \mu} \right] k_1 = c k_1 = 0$ and as $k_1
  \neq 0$, this implies that $c = 0$ and we are done.
\end{proof}

\subsection{Proofs}

\begin{proof}
  (of prop. \ref{sol-MO:prop-solonnonzero}) First of all, we recall that we
  have an exact sequence
  \[ 0 \rightarrow \mathcal{S} \tmop{ol}_C (\mathbbm{R}^n - \{ 0 \})
     \rightarrow \mathcal{S} \tmop{ol} (\mathbbm{R}^n - \{ 0 \})
     \xrightarrow{\pi} \mathcal{S} \tmop{ol} (\mathbbm{R}^n - C) \]
  and propositions \ref{sol-MO:prop-solCnonzero} and \ref{lem67:prop-67}
  respectively tell us that
  \begin{eqnarray}
    & \mathcal{S} \tmop{ol}_C (\mathbbm{R}^{p, q} \backslash \{ 0 \} ;
    \lambda, \nu) =\mathbbm{C} \left\{ \begin{array}{ll}
      \delta^{(\nu - 1)} (Q) \cdot | x_p |^{\lambda + \nu - n}, & p = 1, \nu
      \in 2\mathbbm{Z}_{\geqslant 0} + 1\\
      \delta^{(\nu - 1)} (Q) \cdot \frac{| x_p |^{\lambda + \nu - n}}{\Gamma
      ((\lambda + \nu - n + 1) / 2)}, & p > 1, \nu \in 2\mathbbm{Z}_{\geqslant
      0} + 1\\
      0, & \nu \nin 2\mathbbm{Z}_{\geqslant 0} + 1
    \end{array} \right. &  \nonumber\\
    & \mathcal{S} \tmop{ol} (\mathbbm{R}^n - C ; \lambda, \nu) =\mathbbm{C} |
    Q |^{- \nu} \frac{| x_p |^{\lambda + \nu - n}}{\Gamma \left( \frac{\lambda
    + \nu - n + 1}{2} \right)}, &  \nonumber
  \end{eqnarray}
  and so the thing is to know whether $\pi$ is onto or zero map for given
  $(\lambda, \nu) \in \mathbbm{C}^2$. Now, as we have (see proof of prop.
  \ref{supp-R:prop-main})
  \[ \frac{| x_p |^{\lambda + \nu - n}}{\Gamma \left( \frac{\lambda + \nu - n
     + 1}{2} \right)} \cdot \frac{| Q |^{- \nu}}{\Gamma \left( \frac{1 -
     \nu}{2} \right)} \in \mathcal{S} \tmop{ol} (\mathbbm{R}^n - \{ 0 \} ;
     \lambda, \nu), \]
  it is clear that $\pi$ is onto for $(\lambda, \nu) \nin \mid \mid$ and this
  gives the part of an answer. Moreover, if $p = 1$, we have lemma
  \ref{supp-sing:lem-strangeelement} (more specifically, the distribution $| Q
  |^{- \nu} \frac{| x_p |^{\lambda + \nu - n}}{\Gamma \left( \frac{\lambda +
  \nu - n + 1}{2} \right)}$ constructed in it) giving us that $\pi$ is onto as
  well when $(\lambda, \nu) \in \mid \mid \cap \backslash\backslash$. On the
  other hand, for other cases lemma \ref{sol-MO:lem-zeromap} tells us that
  $\pi = 0$ and this ends the proof.
\end{proof}

\begin{proof}
  (of prop. \ref{sol-MO:prop-main}) We employ an exact sequence
  \[ 0 \rightarrow \mathcal{S} \tmop{ol}_{\{ 0 \}} (\mathbbm{R}^n) \rightarrow
     \mathcal{S} \tmop{ol} (\mathbbm{R}^n) \xrightarrow{\pi} \mathcal{S}
     \tmop{ol} (\mathbbm{R}^n \um \{ 0 \}) \]
  and again as $\mathcal{S} \tmop{ol}_{\{ 0 \}} (\mathbbm{R}^n)$ and
  $\mathcal{S} \tmop{ol} (\mathbbm{R}^n \um \{ 0 \})$ are explicitly
  determined by propositions \ref{diffSBO:prop-main} and
  \ref{sol-MO:prop-solonnonzero} respectively, the thing is to know what is
  the image of $\pi$ for given $(\lambda, \nu) \in \mathbbm{C}^2$.
  
  We first consider the $p > 1$ case, when $\mathcal{S} \tmop{ol}
  (\mathbbm{R}^n - \{ 0 \} ; \lambda, \nu)$ is spanned by an element $\frac{|
  x_p |^{\lambda + \nu - n}}{\Gamma \left( \frac{\lambda + \nu - n + 1}{2}
  \right)} \cdot \frac{| Q |^{- \nu}}{\Gamma \left( \frac{1 - \nu}{2}
  \right)}$. For $(\lambda, \nu) \nin / /$ we have $\pi$ being onto, as the
  image of $\tilde{K}_{\lambda, \nu}^{\mathbbm{R}^n}$ covers the generating
  elements of $\mathcal{S} \tmop{ol} (\mathbbm{R}^n - \{ 0 \} ; \lambda,
  \nu)$. Now, for $(\lambda_0, \nu_0) \in / /$ with $\nu_0 \nin L$ we have
  $\tilde{K}_{\lambda_0, \nu_0}^{\mathbbm{R}^n}$ being nonzero and supported
  at $\{ 0 \}$ (hence it is a multiple of $\tilde{K}_{\lambda, \nu}^{\{ 0
  \}}$) and if we let $K_{\lambda} \assign \tilde{K}_{\lambda,
  \nu_0}^{\mathbbm{R}^n}$, we have that it satisfies the hypothesis of lemma
  \ref{sol-MO:lem-zeromap-point} with $S =\mathbbm{R}^n$ (in particular, the
  part about the support of the derivative is granted by lemma
  \ref{supp-R:lem-supp-of-K}) and hence $\pi = 0$ in this case. Finally, if
  $(\lambda_0, \nu_0) \in / /$ and $\nu_0 \in L$, the element $K_{\lambda}
  \assign \widetilde{\tilde{K}}_{\lambda, \nu_0}^{\mathbbm{R}^n}$ has its
  image covering generator of $\mathcal{S} \tmop{ol} (\mathbbm{R}^n \um \{ 0
  \})$.
  
  We now turn to $p = 1$ case. First of all, for $(\lambda, \nu) \in / / \cup
  (| | \cap \backslash\backslash)$ we have $\tilde{K}_{\lambda,
  \nu}^{\mathbbm{R}^n}$ being nonzero and $\mathcal{S} \tmop{ol}
  (\mathbbm{R}^n - \{ 0 \} ; \lambda \comma \nu)$ is one-dimensional, and we
  see that $\pi$ is onto in this case. Similaly, we have $\pi$ being onto for
  $(\lambda, \nu) \in (/ / \cap L) - (| | \cap \backslash\backslash)$ (this
  time however, the generator of $\mathcal{S} \tmop{ol} (\mathbbm{R}^n - \{ 0
  \} ; \lambda \comma \nu)$ is covered not by $\tilde{K}_{\lambda,
  \nu}^{\mathbbm{R}^n}$, but by $\widetilde{\tilde{K}}_{\lambda,
  \nu}^{\mathbbm{R}^n}$). \ Next, for $(\lambda, \nu) \in / / - L - (\mid
  \mid \cap \backslash\backslash)$ we have $\pi = 0$ by lemma
  \ref{sol-MO:lem-zeromap-point} (applied to $K_{\mu} = \tilde{K}_{\mu,
  \nu}^{\mathbbm{R}^n}$). This gives the first two rows of an answer for $p =
  2$ given in the statement.
  
  We next consider the case $(\lambda, \nu) \in (| | \cap
  \backslash\backslash) - / /$. Propositions \ref{supp-sing:prop-supp-Y} and
  \ref{supp-sing:prop-supp-C} tell us that $\tilde{K}_{\lambda, \nu}^Y$ and
  $\tilde{K}_{\lambda, \nu}^C$ are both nonzero and their respective images
  under $\pi$ cover the two generators of $\mathcal{S} \tmop{ol}
  (\mathbbm{R}^n - \{ 0 \} ; \lambda, \nu)$ listed in proposition
  \ref{sol-MO:prop-solonnonzero}. Hence, $\pi$ is onto in this case. Finally,
  let us assume that $(\lambda_0, \nu_0) \in \mid \mid \cap
  \backslash\backslash \cap / /$. As the latter set is nonempty only when $q
  \in 2\mathbbm{Z}$, we may assume it is so in subsequent. In this case
  $\tilde{K}_{\lambda, \nu}^P$ and $\tilde{K}_{\lambda, \nu}^C$ are both
  nonzero and supported at $\{ 0 \}$. We can introduce linear maps $\lambda,
  \lambda', \nu, \nu' : \mathbbm{C} \rightarrow \mathbbm{C}$ such that for any
  $\mu \in \mathbbm{C}$, we have $(\lambda (\mu), \nu (\mu)) \in \mid
  \mid$, $(\lambda' (\mu), \nu' (\mu)) \in \backslash\backslash$ and
  $(\lambda (0), \nu (0)) = (\lambda' (0), \nu' (0)) = (\lambda_0, \nu_0)$.
  We then let $K^C_{(\mu)} \assign \tilde{K}_{\lambda (\mu), \nu (\mu)}^C$ and
  $K^P_{(\mu)} \assign \tilde{K}^P_{\lambda' (\mu), \nu' (\mu)}$ and introduce
  the Taylor series expansions
  \[ K^C_{(\mu)} = K^C_0 + \mu K_1^C + \ldots, \qquad K^P_{(\mu)} = K_0^P +
     \mu K_1^P + \ldots \]
  It can be seen (inspecting the way we defined $K_{\lambda, \nu}^C$ and
  $K_{\lambda, \nu}^P$ and normalized them) that $K_1^C$ and $K_1^P$ have
  their supports equal to $C$ and $P$ (hence, linear independent) respectively
  and their restrictions to $\mathbbm{R}^n - \{ 0 \}$ are elements of
  $\mathcal{S} \tmop{ol} (\mathbbm{R}^n - \{ 0 \} ; \lambda_0, \nu_0)$, hence
  they span $\mathcal{S} \tmop{ol} (\mathbbm{R}^n - \{ 0 \} ; \lambda_0,
  \nu_0)$. We claim that $\pi = 0$. Indeed, suppose $F \in \mathcal{S}
  \tmop{ol} (\mathbbm{R}^n ; \lambda_0, \nu_0)$. We then have $F
  \mid_{\mathbbm{R}^n - \{ 0 \}} = a K_1^C \mid_{\mathbbm{R}^n - \{ 0
  \}} + b K_1^P \mid_{\mathbbm{R}^n - \{ 0 \}}$ for some $(a, b) \in
  \mathbbm{C}^2$ and then the proof of lemma \ref{sol-MO:lem-zeromap-point}
  goes through (with $K_{\mu} \assign a K_{(\mu)}^C + b K^P_{(\mu)}$) to show
  that $F \mid_{\mathbbm{R}^n - \{ 0 \}} = 0$. Since $F$ was arbitrary, we
  are done.
\end{proof}
\appendix
\newpage
\section{Dependence Diagram}
\pagenumbering{gobble}
\begin{sidewaysfigure}
		\centerline{
		\begin{dot2tex}[scale=0.49]
digraph G{
node[shape="box"]
splines=line;
subgraph cluster0 {
label="Section 9";
definition9d1;
definition9d2;
proposition9d4;
proposition9d5;
lemma9d7;
lemma9d8;
lemma9d9;
definition9d10;
definition9d11;
}
subgraph cluster1 {
label="Section 10";
proposition10d1;
definition10d2;
lemma10d3;
lemma10d4;
lemma10d5;
fact10d7;
fact10d8;
lemma10d9;
lemma10d10;
lemma10d12;
lemma10d13;
lemma10d14;
lemma10d16;
}
subgraph cluster2 {
label="Section 11";
proposition11d1;
lemma11d2;
lemma11d4;
lemma11d5;
}
subgraph cluster3 {
label="Section 12";
proposition12d1;
proposition12d2;
proposition12d3;
lemma12d4;
lemma12d5;
lemma12d6;
}
subgraph cluster4 {
label="Section 13";
proposition13d1;
lemma13d2;
}
subgraph cluster5 {
label="Section 14";
proposition14d1;
lemma14d2;
lemma14d3;
lemma14d4;
fact14d6;
definition14d8;
lemma14d10;
lemma14d12;
lemma14d13;
lemma14d14;
lemma14d15;
}
subgraph cluster6 {
label="Section 15";
proposition15d1;
proposition15d2;
lemma15d3;
lemma15d5;
lemma15d7;
}
"paper 1";
"paper 1" -> proposition9d4;
definition9d2 -> proposition9d4;
definition9d10 -> proposition9d4;
lemma9d8 -> proposition9d4;
definition9d11 -> proposition9d4;
definition9d1 -> proposition9d4;
definition9d2 -> proposition9d5;
lemma9d9 -> proposition9d5;
"paper 1" -> lemma9d7;
lemma9d7 -> lemma9d8;
"paper 1" -> lemma9d8;
"paper 1" -> lemma9d9;
"paper 1" -> definition9d10;
definition9d10 -> definition9d11;
"paper 1" -> definition9d11;
"paper 1" -> proposition10d1;
proposition9d5 -> proposition10d1;
lemma10d14 -> proposition10d1;
lemma10d16 -> proposition10d1;
lemma10d3 -> lemma10d4;
definition10d2 -> lemma10d5;
"paper 1" -> lemma10d5;
fact10d7 -> lemma10d5;
fact10d8 -> lemma10d5;
definition10d2 -> lemma10d9;
lemma10d5 -> lemma10d9;
fact10d7 -> lemma10d10;
"paper 1" -> lemma10d10;
"paper 1" -> lemma10d12;
definition9d2 -> lemma10d13;
proposition9d5 -> lemma10d14;
"paper 1" -> lemma10d14;
lemma10d4 -> lemma10d14;
lemma10d13 -> lemma10d14;
lemma10d9 -> lemma10d14;
lemma10d12 -> lemma10d14;
lemma10d10 -> lemma10d14;
"paper 1" -> proposition11d1;
proposition9d5 -> proposition11d1;
lemma11d2 -> proposition11d1;
lemma11d4 -> proposition11d1;
lemma11d5 -> proposition11d1;
"paper 1" -> lemma11d2;
proposition11d1 -> lemma11d4;
"paper 1" -> lemma11d4;
proposition11d1 -> lemma11d5;
lemma11d2 -> lemma11d5;
proposition11d1 -> proposition12d1;
lemma11d4 -> proposition12d1;
lemma12d4 -> proposition12d1;
lemma12d6 -> proposition12d2;
lemma12d5 -> proposition12d2;
lemma11d4 -> proposition12d2;
proposition11d1 -> proposition12d3;
lemma11d4 -> proposition12d3;
lemma12d6 -> proposition12d3;
lemma12d4 -> proposition12d3;
proposition11d1 -> lemma12d4;
proposition11d1 -> lemma12d5;
proposition11d1 -> lemma12d6;
lemma11d4 -> lemma12d6;
proposition9d5 -> lemma12d6;
"paper 1" -> lemma12d6;
lemma13d2 -> proposition13d1;
lemma10d12 -> proposition13d1;
definition9d1 -> lemma13d2;
"paper 1" -> lemma13d2;
lemma14d4 -> lemma13d2;
lemma14d2 -> proposition14d1;
lemma11d4 -> proposition14d1;
lemma14d3 -> proposition14d1;
definition14d8 -> proposition14d1;
lemma14d10 -> proposition14d1;
proposition14d1 -> lemma14d2;
lemma14d3 -> lemma14d2;
lemma14d10 -> lemma14d2;
fact14d6 -> lemma14d2;
lemma14d15 -> lemma14d2;
lemma14d14 -> lemma14d2;
"paper 1" -> lemma14d3;
"paper 1" -> lemma14d4;
definition9d2 -> definition14d8;
"paper 1" -> definition14d8;
definition14d8 -> lemma14d10;
fact14d6 -> lemma14d10;
lemma14d2 -> lemma14d10;
fact10d7 -> lemma14d10;
lemma10d9 -> lemma14d10;
lemma10d10 -> lemma14d10;
"paper 1" -> lemma14d10;
lemma14d12 -> lemma14d13;
lemma14d10 -> lemma14d14;
"paper 1" -> lemma14d14;
lemma14d13 -> lemma14d14;
lemma11d4 -> lemma14d14;
"paper 1" -> lemma14d15;
proposition14d1 -> proposition15d1;
proposition10d1 -> proposition15d1;
proposition11d1 -> proposition15d1;
lemma12d6 -> proposition15d1;
lemma15d7 -> proposition15d1;
proposition13d1 -> proposition15d2;
proposition15d1 -> proposition15d2;
lemma15d5 -> proposition15d2;
lemma11d4 -> proposition15d2;
proposition12d2 -> proposition15d2;
proposition12d3 -> proposition15d2;
fact14d6 -> lemma15d3;
"paper 1" -> lemma15d3;
proposition9d5 -> lemma15d5;
lemma15d3 -> lemma15d5;
proposition15d1 -> lemma15d7;
lemma11d4 -> lemma15d7;
proposition10d1 -> lemma15d7;
lemma14d15 -> lemma15d7;
lemma15d3 -> lemma15d7;
definition9d1 [shape="box",label="",texlbl="$\mbox{Def. \ref{def-n-nots:def-n+invar}}$"];
definition9d2 [shape="box",label="",texlbl="$\mbox{Def. \ref{sol:def-sol}}$"];
proposition9d4 [shape="box",label="",texlbl="$\mbox{Prop. \ref{sol:prop-sol}}$"];
proposition9d5 [shape="box",label="",texlbl="$\mbox{Prop. \ref{sol:prop-holocont}}$"];
lemma9d7 [shape="box",label="",texlbl="$\mbox{Lem. \ref{sol:lem-unfold}}$"];
lemma9d8 [shape="box",label="",texlbl="$\mbox{Lem. \ref{sol:lem-commdiag}}$"];
lemma9d9 [shape="box",label="",texlbl="$\mbox{Lem. \ref{sol:lem-holodep}}$"];
definition9d10 [shape="box",label="",texlbl="$\mbox{Def. \ref{sol:def-localaciton}}$"];
definition9d11 [shape="box",label="",texlbl="$\mbox{Def. \ref{sol:def-D'n}}$"];
proposition10d1 [shape="box",label="",texlbl="$\mbox{Prop. \ref{lem67:prop-67}}$"];
definition10d2 [shape="box",label="",texlbl="$\mbox{Def. \ref{lem67:def-SCSCC}}$"];
lemma10d3 [shape="box",label="",texlbl="$\mbox{Lem. \ref{lem67:lem-geom-aux}}$"];
lemma10d4 [shape="box",label="",texlbl="$\mbox{Lem. \ref{lem67:lem-geom}}$"];
lemma10d5 [shape="box",label="",texlbl="$\mbox{Lem. \ref{lem67:lem-tensor-aux}}$"];
fact10d7 [shape="box",label="",texlbl="$\mbox{Fact \ref{fact:sing-q-4}}$"];
fact10d8 [shape="box",label="",texlbl="$\mbox{Fact \ref{lem67:fact-pullback}}$"];
lemma10d9 [shape="box",label="",texlbl="$\mbox{Lem. \ref{lem67:lem-tensor}}$"];
lemma10d10 [shape="box",label="",texlbl="$\mbox{Lem. \ref{lem67:lem-homogR}}$"];
lemma10d12 [shape="box",label="",texlbl="$\mbox{Lem. \ref{lem67:lem-homogImpliesE}}$"];
lemma10d13 [shape="box",label="",texlbl="$\mbox{Lem. \ref{lem67:lem-eveninall}}$"];
lemma10d14 [shape="box",label="",texlbl="$\mbox{Lem. \ref{lem67:lem-Qpm}}$"];
lemma10d16 [shape="box",label="",texlbl="$\mbox{Lem. \ref{lem67:lem-flip}}$"];
proposition11d1 [shape="box",label="",texlbl="$\mbox{Prop. \ref{supp-R:prop-main}}$"];
lemma11d2 [shape="box",label="",texlbl="$\mbox{Lem. \ref{supp-R:lem-gelfand}}$"];
lemma11d4 [shape="box",label="",texlbl="$\mbox{Lem. \ref{supp-R:lem-supp-of-K}}$"];
lemma11d5 [shape="box",label="",texlbl="$\mbox{Lem. \ref{supp-R:lem-Kzero}}$"];
proposition12d1 [shape="box",label="",texlbl="$\mbox{Prop. \ref{supp-sing:prop-supp-mmm}}$"];
proposition12d2 [shape="box",label="",texlbl="$\mbox{Prop. \ref{supp-sing:prop-supp-Y}}$"];
proposition12d3 [shape="box",label="",texlbl="$\mbox{Prop. \ref{supp-sing:prop-supp-C}}$"];
lemma12d4 [shape="box",label="",texlbl="$\mbox{Lem. \ref{supp-sing:lem-strangelove}}$"];
lemma12d5 [shape="box",label="",texlbl="$\mbox{Lem. \ref{supp-sing:lem-strangelove-Y}}$"];
lemma12d6 [shape="box",label="",texlbl="$\mbox{Lem. \ref{supp-sing:lem-strangeelement}}$"];
proposition13d1 [shape="box",label="",texlbl="$\mbox{Prop. \ref{diffSBO:prop-main}}$"];
lemma13d2 [shape="box",label="",texlbl="$\mbox{Lem. \ref{diffSBO:lem-aux}}$"];
proposition14d1 [shape="box",label="",texlbl="$\mbox{Prop. \ref{sol-MO:prop-solCnonzero}}$"];
lemma14d2 [shape="box",label="",texlbl="$\mbox{Lem. \ref{uniq-c:lem-generic}}$"];
lemma14d3 [shape="box",label="",texlbl="$\mbox{Lem. \ref{lem:sing-q-3}}$"];
lemma14d4 [shape="box",label="",texlbl="$\mbox{Lem. \ref{supp-Q:lem-sing-q-4}}$"];
fact14d6 [shape="box",label="",texlbl="$\mbox{Fact \ref{fact:sing-q-3}}$"];
definition14d8 [shape="box",label="",texlbl="$\mbox{Def. \ref{n-nonequiv:def-solprime}}$"];
lemma14d10 [shape="box",label="",texlbl="$\mbox{Lem. \ref{lem:sing-q-6}}$"];
lemma14d12 [shape="box",label="",texlbl="$\mbox{Lem. \ref{supp-Q:lem-flip}}$"];
lemma14d13 [shape="box",label="",texlbl="$\mbox{Lem. \ref{supp-Q:lem-sing-q-7-aux}}$"];
lemma14d14 [shape="box",label="",texlbl="$\mbox{Lem. \ref{lem:sing-q-7}}$"];
lemma14d15 [shape="box",label="",texlbl="$\mbox{Lem. \ref{supp-Q:lem-operator}}$"];
proposition15d1 [shape="box",label="",texlbl="$\mbox{Prop. \ref{sol-MO:prop-solonnonzero}}$"];
proposition15d2 [shape="box",label="",texlbl="$\mbox{Prop. \ref{sol-MO:prop-main}}$"];
lemma15d3 [shape="box",label="",texlbl="$\mbox{Lem. \ref{sol-MO:lem-E2}}$"];
lemma15d5 [shape="box",label="",texlbl="$\mbox{Lem. \ref{sol-MO:lem-zeromap-point}}$"];
lemma15d7 [shape="box",label="",texlbl="$\mbox{Lem. \ref{sol-MO:lem-zeromap}}$"];
}
\end{dot2tex}

		\hspace{2cm}
		}
\end{sidewaysfigure}
\newpage
\bibliographystyle{alpha}
\bibliography{todai_master}
\end{document}
