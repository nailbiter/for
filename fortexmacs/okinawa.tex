\documentclass{article}
\usepackage[english]{babel}
\usepackage{amsmath,amssymb,graphicx,bbm}

%%%%%%%%%% Start TeXmacs macros
\newcommand{\assign}{:=}
\newcommand{\cdummy}{\cdot}
\newcommand{\nin}{\not\in}
\newcommand{\nobracket}{}
\newcommand{\tmop}[1]{\ensuremath{\operatorname{#1}}}
\newcommand{\tmtextbf}[1]{{\bfseries{#1}}}
\newcommand{\tmtextit}[1]{{\itshape{#1}}}
\newcommand{\tmtextmd}[1]{{\mdseries{#1}}}
\newcommand{\tmtextrm}[1]{{\rmfamily{#1}}}
\newcommand{\tmtextup}[1]{{\upshape{#1}}}
%%%%%%%%%% End TeXmacs macros

\begin{document}

\title{Symmetry breaking of indefinite orthogonal groups $O (p, q)$}

\maketitle

\section{Branching problem}

Suppose $G \supset G'$ are reductive groups and $\pi$ is irreducible
representation of $G$. If we restrict $\pi$ to $G'$, in general it is no
longer irreducible.

\begin{center}
  \resizebox{502px}{132px}{\includegraphics{okinawa-1.eps}}
\end{center}

\

{\underline{Branching problem}} (in a wider sense) =

= Understand $\pi \mid_{G'} .$

These are well-studied (e.g. combinatorial algorithm) for $\pi$
finitely-dimensional and $G$: compact. In this setting, $\pi$ always splits
into a direct sum
\[ \pi \mid_{G'} = \bigoplus_{\tau \in \widehat{G'}} m (\pi, \tau) \tau \]
of irreducibles $\tau$ of $G'$.

However, when $\dim (\pi) = \infty$ and $G, G'$ are non-compact, the situation
becomes much more involved and was not studied seriously before Kobayashi's
theory appeared in 90s. In particular, several examples that show that
behaviour is very wild in general were constructed {\cite{Kobayashi2008}}.

{\underline{SBOs:}} an idea to understand ``indecomposable'' restriction $\pi
\mid_{G'}$ is to compare it with irreducible representations $\tau$ of the
small group $G'$, i.e. to study the space
\[ \tmop{Hom}_{G'} (\pi \mid_{G'}, \tau) \]
of {\underline{symmetry breaking operators}} (SBOs, for short).

\section{$\mathcal{A}\mathcal{B}\mathcal{C}$ program for branching}

In {\cite{kobayashi2015program}} T. Kobayashi introduced the far-reaching
program for studying branching of noncompact groups, which can be summarized
as follows:
\begin{description}
  \item[($\mathcal{A}$)] $\mathcal{A}$bstract features of the representation
  (i.e. we want to find triples $(G, G', \pi)$, so that $\pi \mid_{G'}$ is
  manageable);
  
  \item[$(\mathcal{B})$] $\mathcal{B}$ranching law of $\pi \mid_{G'}$;
  
  \item[$(\mathcal{C})$] $\mathcal{C}$onstruction of SBOs.
\end{description}
The main theme of this work is part $\mathcal{C}$ for ``standard
representations'' with focus on:
\begin{description}
  \item[$(\mathcal{C}1)$] Construct SBOs;
  
  \item[$(\mathcal{C}2)$] Classify all SBOs;
  
  \item[$(\mathcal{C}3)$] Study functional equations among SBOs;
  
  \item[$(\mathcal{C}4)$] Find residue formulae for SBOs;
  
  \item[$(\mathcal{C}5)$] Find images of SBOs.
\end{description}
The subprogram $(\mathcal{C}1) - (\mathcal{C}5)$ was proposed by
Kobayashi-Speh in their book {\cite{kobayashi2015symmetry}}. Further, they
gave a complete answer to $(\mathcal{C}1) - (\mathcal{C}5)$ for real rank 1
pair $(G, G') = (O (n + 1, 1), O (n, 1))$.

\tmtextbf{Goal}: extend this to higher rank case $(G, G') = (O (p + 1, q + 1),
O (p, q + 1))$. The class of the ``standard'' representations we are working
with are \tmtextbf{degenerate spherical principal series representations}:
\begin{eqnarray}
  & I (\lambda) \assign \tmop{Ind}_P^G (\mathbbm{C}_{\lambda}), \quad \lambda
  \in \mathbbm{C}, &  \nonumber\\
  & J (\nu) \assign \tmop{Ind}_{P'}^{G'} (\mathbbm{C}_{\nu}), \quad \nu \in
  \mathbbm{C}. &  \nonumber
\end{eqnarray}
where $P \subset G$ is the maximal parabolic subgroup with the Levi part
\[ M A \simeq O (p, q) \times \{ \pm 1 \} \times \mathbbm{R}, \]
$P' = P \cap G'$ is a maximal parabolic of $G'$.

\fbox{Conformal viewpoint:} Geometrically $I (\lambda)$ arise from conformal
geometry:
\begin{eqnarray}
  & X \assign \mathbbm{S}^p \times \mathbbm{S}^q / \pm &  \nonumber\\
  & \rightsquigarrow \tmop{Conf} (X) = O (p + 1, q + 1) = G \curvearrowright
  &  \nonumber\\
  & \curvearrowright \mathcal{L}_{\lambda} \rightarrow X = G / P :
  \tmop{conformally} \tmop{equivariant} \tmop{line} \tmop{bundle} & 
  \nonumber\\
  & \rightsquigarrow G \curvearrowright I (\lambda) = C^{\infty} (X,
  \mathcal{L}_{\lambda}) . &  \nonumber
\end{eqnarray}
{\noindent}\tmtextbf{Remark . }Works {\cite{kobayashi2013finite}},
{\cite{kobayashi2014classification}} regarding the part $\mathcal{A}$ (a
priori estimate) for this setting, imply that $\dim (\tmop{Hom}_{G'} (I
(\lambda), J (\nu)))$ is uniformly bounded in $(\lambda, \nu) \in
\mathbbm{C}^2$.{\hspace*{\fill}}{\medskip}

\section{Towards classification of SBOs}

Distribution kernel for SBO:

{\noindent}\tmtextbf{Fact \tmtextup{1}. }\tmtextit{Applying the general
statement of {\cite{kobayashi2015symmetry}} to our concrete setting we get:

\begin{center}
  \resizebox{0.7\columnwidth}{!}{\includegraphics{okinawa-2.eps}}
\end{center}

\ }{\hspace*{\fill}}{\medskip}

{\noindent}\tmtextbf{Theorem . }\tmtextit{(description of double coset $P'
\backslash G / P$) We set
\begin{eqnarray}
  & X \assign G / P \simeq \mathbbm{S}^p \times \mathbbm{S}^q / \pm, \quad & 
  \nonumber\\
  & Y \assign \{ [\xi : \eta] \in G / P \simeq \mathbbm{S}^p \times
  \mathbbm{S}^q / \pm \mid \xi_p = 0 \} \simeq \mathbbm{S}^{p - 1} \times
  \mathbbm{S}^q / \pm, &  \nonumber\\
  & C \assign \{ [\xi : \eta] \in G / P \simeq \mathbbm{S}^p \times
  \mathbbm{S}^q / \pm \mid \xi_0 = \eta_q \}, &  \nonumber\\
  & \{ [o] \} \assign \{ [1 : 0_{p + q} : 1] \} . &  \nonumber
\end{eqnarray}
We then have

\begin{center}
  \resizebox{502px}{240px}{\includegraphics{okinawa-3.eps}}
\end{center}

\ }{\hspace*{\fill}}{\medskip}

{\noindent}\tmtextbf{Theorem . }\tmtextit{(construction of SBOs) For $S = X,
Y, C$, and $\{ [o] \}$, the following operators $R_{\lambda, \nu}^S$ and
$\tilde{R}_{\lambda, \nu}^X$ are symmetry breaking operators from $I (\lambda)
\mid_{G'}$ to $J (\nu)$, which depend holomorphically on $(\lambda, \nu) \in
D_S$. Moreover, $\mathcal{S} \tmop{upp} (R_{\lambda, \nu}^S) \subseteq S$ with
``='' holding generically and are determined explicitly.
\begin{center}
  \begin{tabular}{|l|l|l|l|}
    \hline
    $R_{\lambda, \nu}^S$ & $\tmop{Op} : \mathcal{S} \tmop{ol} (\mathbbm{R}^{p,
    q} ; \lambda, \nu) \rightarrow \tmop{Hom}_{G'} (I (\lambda), J (\nu))$ &
    $D_S \hspace{0.17em}$ & $\hspace{0.17em} \mathcal{S} \tmop{upp}
    (\cdummy)$\\
    \hline
    $R_{\lambda, \nu}^X =$ & $q_X (\lambda, \nu) \tmop{Op} (|x_p |^{\lambda +
    \nu - n} |Q_{p, q} |^{- \nu})$ & $\mathbbm{C}^2$ & $\nobracket \ldots
    \nobracket$\\
    \hline
    $\tilde{R}^X_{\lambda, \nu} =$ & $\Gamma \left( \frac{\lambda - \nu}{2}
    \right) q_X (\lambda, \nu) \tmop{Op} (|x_p |^{\lambda + \nu - n} |Q_{p, q}
    |^{- \nu})$ & $\mid \mid \mid$ & $\nobracket \ldots \nobracket$\\
    \hline
    $R_{\lambda, \nu}^Y =$ & $(- 1)^k k!q_Y (\lambda, \nu) \tmop{Op}
    (\delta^{(2 k)} (x_p) |Q_{p, q} |^{- \nu})$ & $\backslash\backslash$ &
    ...\\
    \hline
    $R_{\lambda, \nu}^C =$ & $(- 1)^m m!q_C (\lambda, \nu) \tmop{Op} (|x_p
    |^{\lambda + \nu - n} \delta^{(2 m)} (Q_{p, q}))$ & $\mid \mid$ &
    $\nobracket \ldots \nobracket$\\
    \hline
    $R_{\lambda, \nu}^{\{ [o] \}} =$ & $\tmop{Op} \left( \tilde{C}_{\nu -
    \lambda}^{\lambda - \frac{n - 1}{2}}  (- \Delta_{\mathbbm{R}^{p - 1, q}}
    \delta_{\mathbbm{R}^{p + q - 1}}, \delta (x_p)) \right)$ & $/ /$ & $\{ [o]
    \}$\\
    \hline
  \end{tabular}
  
  \ 
\end{center}
Let us explain the notation in the table.
\begin{itemize}
  \item for $(\lambda, \nu) \in \backslash\backslash$, $k \assign
  \frac{\lambda + \nu - n + 1}{2}$;
  
  \item for $(\lambda, \nu) \in \mid \mid,$ $m \assign \frac{\nu - 1}{2}$;
  
  \item $\mid \mid \mid \assign \{(\lambda, \nu) \in \mathbbm{C}^2 \mid \nu
  \in - 2 \mathbbm{N} \cup (q + 1 + 2 \mathbbm{Z})\}$;
  
  \item $\backslash \backslash \assign \{ (\lambda, \nu) \in \mathbbm{C}|
  \lambda + \nu - n + 1 \in - 2\mathbbm{C}\}$;
  
  \item $/ / \assign \{(\lambda, \nu) \in \mathbbm{C}^2 \mid \lambda - \nu \in
  - 2\mathbbm{N}\}$;
  
  \item $\mid \mid \assign \{ (\lambda, \nu) \in \mathbbm{C}^2 | \nu \in 1 +
  2\mathbbm{N}\}$;
  
  \item $\tilde{C} (s, t)$ is a polynomial of two-variable's, which obtained
  by inflation of the renormalized Gegenbauer polynomial, defined as in
  {\cite[(16.3)]{kobayashi2015symmetry}}.
  \[ q_Y (\lambda, \nu) \assign \left\{ \begin{array}{ll}
       \Gamma^{- 1} \left( \frac{\lambda - \nu}{2} \right), & q \in 2
       \mathbbm{Z} + 1 \; \tmop{or} \; \lambda + \nu \geqslant 0\\
       \Gamma^{- 1}  (- \nu), & \tmop{otherwise} .
     \end{array} \right. \]
  \[ q_C (\lambda, \nu) \assign \ldots, \quad q_X (\lambda, \nu) \assign
     \ldots \]
\end{itemize}}{\hspace*{\fill}}{\medskip}

{\noindent}\tmtextbf{Theorem . }\tmtextit{(dimension of SBO space) For all
$(\lambda, \nu) \in \mathbbm{C}^2$,
\[ \dim (\tmop{Hom}_{G'} (I (\lambda), J (\nu))) \in \{ 1, 2 \} .
\]}{\hspace*{\fill}}{\medskip}

{\noindent}\tmtextbf{Theorem . }\tmtextit{(classification of SBOs)
\[ p = 1 \Rightarrow \tmop{Hom}_{G'} (I (\lambda), J (\nu)) = \left\{
   \begin{array}{ll}
     \mathbbm{C} R^X_{\lambda, \nu}, & (\lambda, \nu) \in \mathbbm{C}^2 - (/ /
     \cap \mid \mid \mid) - (\mid \mid \cap \backslash \backslash),\\
     \mathbbm{C}  \tilde{R}^X_{\lambda, \nu} \oplus \mathbbm{C}
     R^{\{o\}}_{\lambda, \nu}, & (\lambda, \nu) \in (/ / \cap \mid \mid \mid)
     - (\mid \mid \cap \backslash \backslash),\\
     \mathbbm{C} R^P_{\lambda, \nu} \oplus \mathbbm{C} R^C_{\lambda, \nu}, &
     (\lambda, \nu) \in (\mid \mid \cap \backslash \backslash) - / /,\\
     \mathbbm{C} R^{\{o\}}_{\lambda, \nu}, & (\lambda, \nu) \in \mid \mid \cap
     \backslash \backslash \cap / / .
   \end{array} \right. \]
\[ p > 1 \Rightarrow \tmop{Hom}_{G'} (I (\lambda), J (\nu)) = \left\{
   \begin{array}{ll}
     \mathbbm{C}  \tilde{R}^X_{\lambda, \nu} \oplus \mathbbm{C}
     R^{\{o\}}_{\lambda, \nu \lambda, \nu}, & (\lambda, \nu) \in / / \cap \mid
     \mid \mid,\\
     \mathbbm{C} R^X_{\lambda, \nu}, & \tmop{otherwise} .
   \end{array} \right. \]}{\hspace*{\fill}}{\medskip}

\section{Answers to $(\mathcal{C}3) - (\mathcal{C}5)$}

{\noindent}\tmtextbf{Theorem . }\tmtextit{(spectrum for spherical vectors) Let
$n \assign p + q \hspace{0.27em} (p, q \ge 1)$ as before.
\[ R_{\lambda, \nu}^X \mathbbm{1}_{\lambda} = \frac{2^{1 - \lambda} \pi^{n /
   2}}{\Gamma \left( \frac{\lambda}{2} \right) \Gamma \left( \frac{\lambda + 1
   - q}{2} \right) \Gamma \left( \frac{q - \nu + 1}{2} \right)} 
   \mathbbm{1}_{\nu} . \]}{\hspace*{\fill}}{\medskip}

{\noindent}\tmtextbf{Remark . }Theorem \ref{thm:spherical} was known in
Bernstein-Reznikov 2004 for $p = q = 1$ (i.e. $G' \simeq \tmop{SL} (2,
\mathbbm{R})$), which was extended in {\cite[Thm. 1.1]{clerc2011generalized}}
for higher dimensional cases. The case $q = 0$ was proven in {\cite[Prop.
7.4]{kobayashi2015symmetry}}.{\hspace*{\fill}}{\medskip}

For $(\lambda, \nu) \nin / /$, we set
\begin{eqnarray}
  & K_{\lambda, \nu}^{\mathbb{R}^{p, q}} \assign \frac{| x_p |^{\lambda + \nu
  - n}}{\Gamma \left( \frac{\lambda + \nu - n + 1}{2} \right)} \times \frac{|
  Q_{p, q} |^{- \nu}}{\Gamma \left( \frac{1 - \nu}{2} \right)} \in \mathcal{S}
  \tmop{ol} (\mathbbm{R}^n ; \lambda, \nu) . &  \nonumber\\
  & \tmop{where} \; Q_{p, q} (x) = \sum_{i = 1}^p x_i^2 - \sum_{i = p + 1}^{p
  + q} x_i^2 &  \nonumber
\end{eqnarray}
Then $R_{\lambda, \nu}^X = \frac{1}{\Gamma \left( \frac{\lambda - \nu}{2}
\right)} \tmop{Op} (K_{\lambda, \nu}^X)$. We recall that the left-hand side
extends to a family of SBOs with holomorphic parameter $(\lambda, \nu) \in
\mathbbm{C}^2$.

{\noindent}\tmtextbf{Theorem . }\tmtextit{(residue formula) Let $n \assign p +
q \hspace{0.27em} (p, q \ge 1)$ as before. For $(\lambda, \nu) \in / /$, we
set $l \assign \frac{1}{2}  (\nu - \lambda) \in \mathbbm{N}$. Then we have
\[ R_{\lambda, \nu}^X = \frac{(- 1)^l l! \pi^{(n - 2) / 2}}{2^{\nu + 2 l - 1}}
   \cdot \frac{\sin \left( \frac{1 + q - \nu}{2} \pi \right)}{\Gamma \left(
   \frac{\nu}{2} \right)} R_{\lambda, \nu}^{\{ [o] \}}, \quad (\lambda, \nu)
   \in / / . \]}{\hspace*{\fill}}{\medskip}

{\noindent}\tmtextbf{Remark . }The residue formula in the case $q = 0$ was
given in {\cite[Thm. 12.2]{kobayashi2015symmetry}}.{\hspace*{\fill}}{\medskip}

{\noindent}\tmtextbf{Definition . }\tmtextit{The {\underline{Knapp-Stein
operator}} is a $G$-intertwining operator defined as
\begin{eqnarray}
  & \tilde{\mathbbm{T}}_{\lambda}^G : I (\lambda) \rightarrow I (n - \lambda)
  &  \nonumber\\
  & f \mapsto q_T (\lambda) (| Q_{p, q} |^{n - \lambda} \ast f) &  \nonumber
\end{eqnarray}
where $q_T (\lambda, \nu)$ is explicitly given by Gamma factors, so that we
have

\begin{center}
  \resizebox{0.3\columnwidth}{!}{\includegraphics{okinawa-4.eps}}
\end{center}

For $G' = O (p, q + 1)$ we similarly define $\tilde{\mathbbm{T}}_{\nu}^{G'} :
J (\nu) \rightarrow J (n - 1 - \nu)$, so that we have

\begin{center}
  \resizebox{0.3\columnwidth}{!}{\includegraphics{okinawa-5.eps}}
\end{center}

\ }{\hspace*{\fill}}{\medskip}

{\noindent}\tmtextbf{Theorem . }\tmtextit{(functional identities) Let $n
\assign p + q$ as before. We have: \begin{eqnarray}
  & \widetilde{\mathbbm{T}}^{G'}_{n - 1 - \nu} \circ R^X_{\lambda, n' - \nu}
  = q^{TX}_X (\lambda, \nu) R^X_{\lambda, \nu}, &  \nonumber\\
  & R_{n - \lambda, \nu}^X \circ \widetilde{\mathbbm{T}}^G_{\lambda} =
  q^{XT}_X (\lambda, \nu) R_{\lambda, \nu}^X, &  \nonumber
\end{eqnarray}

where
\begin{eqnarray}
  & q^{TX}_X (\lambda, \nu) \assign \frac{\pi^{\frac{n - 2}{2}} \sin \left(
  \frac{p - \nu}{2} \pi \right) 2^{1 - n + \nu}}{\Gamma \left( \frac{n - 1 -
  \nu}{2} \right)} \left\{ \begin{array}{ll}
    \Gamma \left( \frac{1 - \nu}{2} \right), & p = 1,\\
    1, & n \in 2 \mathbbm{Z},\\
    \ldots, & \ldots
  \end{array} \right. &  \nonumber\\
  & q^{XT}_X (\lambda, \nu) \assign \ldots &  \nonumber
\end{eqnarray}}{\hspace*{\fill}}{\medskip}

{\noindent}\tmtextbf{Remark . }The functional identities in the case $q = 0$
were proven in {\cite[Thm.
12.6]{kobayashi2015program}}.{\hspace*{\fill}}{\medskip}

{\noindent}\tmtextbf{Theorem . }\tmtextit{(images of SBOs) We can compute
images of every SBOs constructed above for every $(\lambda, \nu) \in
\mathbbm{C}^2$.}{\hspace*{\fill}}{\medskip}

\begin{thebibliography}{CK�P11}
  \bibitem[BR04]{bernstein2004estimates}J.~Bernstein  and 
  A.~Reznikov.{\newblock} Estimates of automorphic functions.{\newblock}
  \tmtextit{Mosc. Math. J}, \tmtextbf{4}(1):19--37, 2004.{\newblock}
  
  \bibitem[CK{\O}P11]{clerc2011generalized}J.-L.~Clerc, T.~Kobayashi,
  B.~{\O}rsted , and  M.~Pevzner.{\newblock} Generalized Bernstein--Reznikov
  integrals.{\newblock} \tmtextit{Mathematische Annalen}, 349(2):395--431,
  2011.{\newblock}
  
  \bibitem[KM14]{kobayashi2014classification}T.~Kobayashi  and 
  T.~Matsuki.{\newblock} Classification of finite-multiplicity symmetric
  pairs.{\newblock} In \tmtextit{\tmtextrm{\tmtextup{\tmtextmd{Special Issue
  in honour of Professor Dynkin for his 90th birthday}}}},  volume~19,  pages 
  457--493. Springer, 2014.{\newblock}
  
  \bibitem[KO13]{kobayashi2013finite}T.~Kobayashi  and  T.~Oshima.{\newblock}
  Finite multiplicity theorems for induction and restriction.{\newblock}
  \tmtextit{Advances in Mathematics}, 248:921--944, 2013.{\newblock}
  
  \bibitem[Kob08]{Kobayashi2008}Toshiyuki Kobayashi.{\newblock}
  \tmtextit{Multiplicity-free Theorems of the Restrictions of Unitary Highest
  Weight Modules with respect to Reductive Symmetric Pairs},  pages 
  45--109.{\newblock} Birkh{\"a}user Boston, Boston, MA, 2008.{\newblock}
  
  \bibitem[Kob15]{kobayashi2015program}T.~Kobayashi.{\newblock} A program for
  branching problems in the representation theory of real reductive
  groups.{\newblock} In \tmtextit{\tmtextrm{\tmtextup{\tmtextmd{Special issue
  in honor of Vogan's 60th years birthday}}}},  volume  312,  pages  277--322.
  Birkh{\"a}user, 2015.{\newblock}
  
  \bibitem[KS15]{kobayashi2015symmetry}T.~Kobayashi  and  B.~Speh.{\newblock}
  \tmtextit{Symmetry Breaking for Representations of Rank One Orthogonal
  Groups},  volume \tmtextbf{238} of \tmtextit{Memoirs of the Amer. Math.
  Soc}.{\newblock} American Mathematical Society, 2015.{\newblock}
\end{thebibliography}

\end{document}
