\documentclass[8pt]{article} % use larger type; default would be 10pt

\usepackage[margin=1in]{geometry}
\usepackage{graphicx}
\usepackage{float}
\usepackage{subfig}
\usepackage{amsmath}
\usepackage{amsfonts}
\usepackage{hyperref}
\usepackage{enumerate}
\usepackage{enumitem}
\usepackage{harpoon}
\usepackage{tikz}

\usepackage{mystyle}
\newcommand{\Z}{\mathbb{Z}}
\newcommand{\kZ}{\mathbb{Z}/k\mathbb{Z}}
\newcommand{\tZ}{\mathbb{Z}/2\mathbb{Z}}
\newcommand{\Tor}{\mbox{Tor}}
\newcommand{\Hom}{\mbox{Hom}}
\newcommand{\Ext}{\mbox{Ext}}

\title{Problem Set 6, MAT 5070}
\author{Alex Leontiev, 1155040702, CUHK}
\begin{document}
\maketitle
\begin{enumerate}[label=\bfseries \arabic*.]
	\item{{\it Exercise (26.18). }First statement, namely that orientability implies $H_{n-1}(M,\mathbb{Z})$ is torsion free is implied
		by $(26.9)$ of \cite{gh}, as it claims that torsion group of $H_{n-1}$ is isomorphic to torsion group of $H_0$, the latter
		is clearly torsion free (as it's equal to $\mathbb{Z}$). The second statement follows from the universal coefficient
		theorem, which implies that
		\[H_n(M,\mathbb{Z}/k\mathbb{Z})=\mybra{H_n(M,\mathbb{Z})\otimes\kZ}\oplus\Tor\mybra{H_{n-1}(M,\Z),\kZ}\]
		Now, $H_n(M,\mathbb{Z})=0$, due to non-orientability (implied by \cite[Theorem 3.26(b)]{hatcher})
		, hence first addend of the right hand side
		is zero (as for any abelian group $G$, $0\otimes G=0$). Now, as $M$ is non-orientable, torsion subgroup of $H_{n-1}(M,\Z)$
		is $\mathbb{Z}/2\mathbb{Z}$ by \cite[Corollary 3.28]{hatcher}, hence 
		\[\Tor\mybra{H_{n-1}(M,\Z),\kZ}=\Tor\mybra{F\oplus\mathbb{Z}/2\mathbb{Z},\kZ}=\Tor(F,\kZ)\oplus\Tor\mybra{\mathbb{Z}/2\mathbb{Z}
		,\kZ}=\Tor\mybra{\mathbb{Z}/2\mathbb{Z},\kZ}=0\]
		here linearity of $\Tor$ in the first element and the fact that $\Tor(F,\cdot)=0$ whenever $F$ is free are implied by points
		(2) and (3) respectively of \cite[Proposition 3A.5]{hatcher}, and the last step is implied by point (6) of the same proposition
		and the fact that if $k$ is even, $\kZ$ does not contain non-zero elements of order 2.

		Again, applying universal coefficient theorem we have
		\[H_{n-1}(M,\mathbb{Z}/2\mathbb{Z})=\mybra{H_{n-1}(M,\mathbb{Z})\otimes\tZ}\oplus\Tor\mybra{H_{n-2}(M,\Z),\tZ}\]
		by non-orientability (see \cite[Corollary 3.28]{hatcher}) $H_{n-1}(M,\Z)=\Z^r\oplus\tZ$, and since $\Z\otimes\tZ=\tZ\otimes\tZ=
		\tZ$, we have
		\[H_{n-1}(M,\mathbb{Z}/2\mathbb{Z})=\mybra{\tZ}^{r+1}\oplus\Tor\mybra{H_{n-2}(M,\Z),\tZ}\]
		so $H_{n-1}(M,\Z/\tZ)$ has subgroup $\tZ$.

		Finally, as $M$ is closed and $\tZ$-orientable, due to the Poincare duality (as stated in \cite[Theorem 3.30]{hatcher}), we have 
		\[H_1(M,\tZ)\simeq H^{n-1}(M,\tZ)\]
		and applying universal coefficient theorem for cohomology, we get
		\[H^{n-1}(M,\tZ)=\Ext(H_{n-2}(M,\Z),\tZ)\oplus\Hom(H_{n-1}(M,\Z),\tZ)\]
		and by \cite[Corollary 3.28]{hatcher} $H_{n-1}(M,\Z)$ contains element of order 2, hence $\Hom(H_{n-1}(M,\Z),\tZ)\neq0$, hence
		\[H_1(M,\tZ)\simeq H^{n-1}(M,\tZ)\neq0\]
		}
	\item{{\it Exercise. }Let's first show that $k\cdot\chi(Y)=[\Gamma,\mu_f]$. This is just a matter of computations. To facilitate the 
		notation, we'll denote $f\times id:X\times Y\to Y\times Y$ as $F$.
		\[\begin{array}{rr}
			[\Gamma,\mu_f]=&\myexplain{by Exercise 30.22 in \cite{gh}}\\
			=[\xi_X\times\xi_Y\cap\mu_f,\mu_f]=&\myexplain{by definition of $\mu_f$}\\
			=[\xi_X\times\xi_Y\cap F^*(\mu'_Y),F^*(\mu'_Y)]=&\myexplain{by definition of $F^*$}\\
			=[F_*(\xi_X\times\xi_Y\cap F^*(\mu'_Y)),\mu'_Y]=&\myexplain{by functoriality of $\cap$}\\
			=[F_*(\xi_X\times\xi_Y)\cap \mu'_Y,\mu'_Y]=&\myexplain{as $F=f\times id$ and $f^*(\xi_X)=k\cdot\xi_Y$, as $\deg f=k$}\\
			=k\cdot[\xi_Y\times\xi_Y\cap \mu'_Y,\mu'_Y]=&\myexplain{by Exercise (30.15) of \cite{gh}}\\
			=k\cdot[(H_*(d)\xi_X,\mu'_Y]=&\myexplain{by definition of $H_*(d)$}\\
			=k\cdot[\xi_X,H^*(d)\mu'_Y]=&\myexplain{by \cite[Corollary 30.10]{gh}}\\
			=k\cdot\chi(Y)&
		\end{array}\]
		
		Now, let us show that $[\Gamma,\mu_f]=\Gamma\cdot\Gamma$ indeed (due to previous paragraph, this'll imply $k\cdot\chi(Y)=\Gamma\cdot
		\Gamma$). To begin with, note that $g:X\ni x\mapsto g(x):=(x,f(x))\in X\times Y$ is an {\it inclusion} of $X$ into $X\times Y$
		and as $\Gamma=g_*(\xi_X)$ by definition, we see that $\Gamma$ is {\it represented} by $g(X)\subset X\times Y$. Now, as as it stands,
		intersection of $g(X)$ with itself is not transversal (so we cannot talk about intersection number), we assume $f$ can be so
		performed through the homotopy, yielding, say $\widetilde{f}$, so that resulting $\widetilde{g}(X)$ is transversal to $g(X)$. Anyway,
		homotopy does not changes $\mu_f$ (as it is defined through $F^*$ and homotopic maps induce same maps in cohomology)
		, hence does not change $\Gamma\in H_n(X\times Y)$ (due to Exercise (30.22) of \cite{gh}, $\Gamma=\xi_X\times\xi_Y\cap\mu_f$
		depends only on $\mu_f$, not on $f$ itself). Thus said, we can use \cite[Corollary (31.8)]{gh}, to get
		\[[\xi_{X\times Y},\mu_f\cup\mu_f]=(-1)^n\Gamma\cdot\Gamma\]
		as soon as we'll be able to show that $\mu_f=\mu'_{X\times Y}/\Gamma$. By \cite[(30.6)]{gh} it's equivalent to showing that
		$\xi_{X\times Y}\cap\mu_f=(-1)^{2n^2}\Gamma$. However, as by agreement, $\xi_{X\times Y}=\xi_X\times\xi_Y$, this is precisely
		the conclusion of Exercise 30.22. Hence, it remains to show
		\[[\xi_X\times\xi_Y,\mu_f\cup\mu_f]=(-1)^n[\Gamma,\mu_f]\]
		Again, by Exercise 30.22 we have
		\[(-1)^n[\Gamma,\mu_f]=(-1)^n[\xi_x\times\xi_Y\cap\mu_f,\mu_f]\]
		and the conclusion follows, by the relation $[a\cap b,c]=[a,b\cup c]$. {\bf Note}, that {\it seeming} inconsistency in sign is
		resolved by the fact that if $n$ is odd (that is the case we should worry about, as then $(-1)^n=-1$), then by Lemma 31.4 of \cite{gh
		} we have $\Gamma\cdot\Gamma=(-1)^{n^2}\Gamma\cdot\Gamma=-\Gamma\cdot\Gamma\implies\Gamma\cdot\Gamma=0$ and similarly $\chi(Y)=0$
		(as $Y$ then is odd-dimensional manifold, and everything cancels in the light of Poincare duality) and the equality still holds.
		}
\end{enumerate}
\begin{thebibliography}{9}
	\bibitem{gh} {\em Algebraic Topology, a first course}, Greenberg and Harper
	\bibitem{hatcher}{\em Algebraic Topology}, Allen Hatcher
	%\bibitem{lee} {\em Differential Topology}, Victor Guillemin , Alan Pollack
\end{thebibliography}
\end{document}
