\documentclass[8pt]{article} % use larger type; default would be 10pt

\usepackage[margin=1in]{geometry}
\usepackage{graphicx}
\usepackage{float}
\usepackage{subfig}
\usepackage{amsmath}
\usepackage{amsfonts}
\usepackage{hyperref}
\usepackage{enumerate}
\usepackage{enumitem}
\usepackage{harpoon}
\usepackage{tikz}

\usepackage{mystyle}
\newcommand{\Z}{\mathbb{Z}}
\newcommand{\kZ}{\mathbb{Z}/k\mathbb{Z}}
\newcommand{\tZ}{\mathbb{Z}/2\mathbb{Z}}
\newcommand{\Tor}{\mbox{Tor}}
\newcommand{\Hom}{\mbox{Hom}}
\newcommand{\Ext}{\mbox{Ext}}

\title{Problem Set 6, MAT 5070}
\author{Alex Leontiev, 1155040702, CUHK}
\begin{document}
\maketitle
\begin{enumerate}[label=\bfseries \arabic*.]
	\item{{\it Exercise (26.18). }First statement, namely that orientability implies $H_{n-1}(M,\mathbb{Z})$ is torsion free is implied
		by $(26.9)$ of \cite{gh}, as it claims that torsion group of $H_{n-1}$ is isomorphic to torsion group of $H_0$, the latter
		is clearly torsion free (as it's equal to $\mathbb{Z}$). The second statement follows from the universal coefficient
		theorem, which implies that
		\[H_n(M,\mathbb{Z}/k\mathbb{Z})=\mybra{H_n(M,\mathbb{Z})\otimes\kZ}\oplus\Tor\mybra{H_{n-1}(M,\Z),\kZ}\]
		Now, $H_n(M,\mathbb{Z})=0$, due to non-orientability (implied by \cite[Theorem 3.26(b)]{hatcher})
		, hence first addend of the right hand side
		is zero (as for any abelian group $G$, $0\otimes G=0$). Now, as $M$ is non-orientable, torsion subgroup of $H_{n-1}(M,\Z)$
		is $\mathbb{Z}/2\mathbb{Z}$ by \cite[Corollary 3.28]{hatcher}, hence 
		\[\Tor\mybra{H_{n-1}(M,\Z),\kZ}=\Tor\mybra{F\oplus\mathbb{Z}/2\mathbb{Z},\kZ}=\Tor(F,\kZ)\oplus\Tor\mybra{\mathbb{Z}/2\mathbb{Z}
		,\kZ}=\Tor\mybra{\mathbb{Z}/2\mathbb{Z},\kZ}=0\]
		here linearity of $\Tor$ in the first element and the fact that $\Tor(F,\cdot)=0$ whenever $F$ is free are implied by points
		(2) and (3) respectively of \cite[Proposition 3A.5]{hatcher}, and the last step is implied by point (6) of the same proposition
		and the fact that if $k$ is even, $\kZ$ does not contain non-zero elements of order 2.

		Again, applying universal coefficient theorem we have
		\[H_{n-1}(M,\mathbb{Z}/2\mathbb{Z})=\mybra{H_{n-1}(M,\mathbb{Z})\otimes\tZ}\oplus\Tor\mybra{H_{n-2}(M,\Z),\tZ}\]
		by non-orientability (see \cite[Corollary 3.28]{hatcher}) $H_{n-1}(M,\Z)=\Z^r\oplus\tZ$, and since $\Z\otimes\tZ=\tZ\otimes\tZ=
		\tZ$, we have
		\[H_{n-1}(M,\mathbb{Z}/2\mathbb{Z})=\mybra{\tZ}^{r+1}\oplus\Tor\mybra{H_{n-2}(M,\Z),\tZ}\]
		so $H_{n-1}(M,\Z/\tZ)$ has subgroup $\tZ$.

		Finally, as $M$ is closed and $\tZ$-orientable, due to the Poincare duality (as stated in \cite[Theorem 3.30]{hatcher}), we have 
		\[H_1(M,\tZ)\simeq H^{n-1}(M,\tZ)\]
		and applying universal coefficient theorem for cohomology, we get
		\[H^{n-1}(M,\tZ)=\Ext(H_{n-2}(M,\Z),\tZ)\oplus\Hom(H_{n-1}(M,\Z),\tZ)\]
		and by \cite[Corollary 3.28]{hatcher} $H_{n-1}(M,\Z)$ contains element of order 2, hence $\Hom(H_{n-1}(M,\Z),\tZ)\neq0$, hence
		\[H_1(M,\tZ)\simeq H^{n-1}(M,\tZ)\neq0\]
		}
	\item{{\it Exercise. }
		}
\end{enumerate}
\begin{thebibliography}{9}
	\bibitem{gh} {\em Algebraic Topology, a first course}, Greenberg and Harper
	\bibitem{hatcher}{\em Algebraic Topology}, Allen Hatcher
	%\bibitem{lee} {\em Differential Topology}, Victor Guillemin , Alan Pollack
\end{thebibliography}
\end{document}
