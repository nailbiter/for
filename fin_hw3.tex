\documentclass[8pt]{article} % use larger type; default would be 10pt

%\usepackage[utf8]{inputenc} % set input encoding (not needed with XeLaTeX)
\usepackage[10pt]{type1ec}          % use only 10pt fonts
\usepackage[T1]{fontenc}
%\usepackage{CJK}
\usepackage{graphicx}
\usepackage{float}
\usepackage{CJKutf8}
\usepackage{subfig}
\usepackage{amsmath}
\usepackage{amsfonts}
\usepackage{hyperref}
\usepackage{enumerate}
\usepackage{harpoon}
\usepackage{enumitem}
\usepackage{multicol}

\usepackage{mystyle}

\newcommand{\myexplain}[3]{#1\xrightarrow{\text{#2}}#3}
\newcommand{\myexplainf}[4]{#1\xrightarrow{\begin{subarray}{c}\text{#2}\\\text{#3}\end{subarray}}#4}
\newcommand{\myexplainfi}[5]{#1\xrightarrow{\begin{subarray}{c}\text{#2}\\\text{#3}\\\text{#4}\end{subarray}}#5}
\newcommand{\myfrac}[2]{^#1/_#2}

\title{Math 1540\\University Mathematics for Financial Studies\\2013-14 Term 1\\Suggested solutions for\\HW problems Sec 1.2 (Linear Algebra)}
\begin{document}
\maketitle
\begin{description}
\item[\# 3.]{{\it In each of the following, the augmented matrix is in reduced row-echelon form. In each case, find the solution
	set to the corresponding linear system.}
	\begin{enumerate}[label=(\alph*)]
		\setcounter{enumi}{2}
	\item{
		\[\left(\begin{array}{ccc|c}1&-3&0&2\\0&0&1&-2\\0&0&0&0\end{array}\right)\]
	The linear system represented by this augmented matrix is as follows (ignoring third vacuous row)
	\[\begin{cases}x_1-3x_2=2\\x_3=-2\end{cases}\]
	From this
	we immediately see that $x_3=-2$, while first equation can be rewritten as $x_1=2+3x_2$ and thus solutions can be parametrised
	by
	\[x_1=2+3t,\;x_2=t,\;x_3=-2\]
	}
	\item{
	\[\left(\begin{array}{cccc|c}1&2&0&1&5\\0&0&1&3&4\end{array}\right)\]
	The linear system represented by this augmented matrix is as follows
	\[\begin{cases}x_1+2x_2+x_4=5\\x_3+3x_4=4\end{cases}\]
	These can in turn be rewritten as
	\[\begin{cases}x_1=5-2x_2-x_4\\x_3=4-3x_4\end{cases}\]
	Thus solutions can be parametrized by
	\[x_1=5-2w-t,\;x_2=w,\;x_3=4-3t,\;x_4=t\]
	}
	\item{
	\[\left(\begin{array}{cccc|c}1&5&-2&0&3\\0&0&0&1&6\\0&0&0&0&0\\0&0&0&0&0\end{array}\right)\]
	The linear system represented by this augmented matrix is as follows (ignoring the last two vacuous rows)
	\[\begin{cases}x_1+5x_2-2x_3=3\\x_4=6\end{cases}\]
	The first equation can be rewritten as $x_1=3-5x_2+2x_3$, thus solutions can be parametrised as
	\[x_1=3-5v+2t,\;x_2=v,\;x_3=t,\;x_4=6\]
	}
	\item{
	\[\left(\begin{array}{ccc|c}0&1&0&2\\0&0&1&-1\\0&0&0&0\end{array}\right)\]
	The linear system represented by this augmented matrix is as follows (ignoring the last vacuous row)
	\[\begin{cases}x_2=2\\x_3=-1\end{cases}\]
	Note that $x_1$ does not appear in equations, though it has to be mentioned in solutions, which can be parametrised as 
	\[x_1=t,\;x_2=2,\;x_3=-1\]
	}
	\end{enumerate}
	}
%# 3cdef, 5fjl
\item[\# 5.]{
	{\it For each of the following system of equations, use Gaussian elimination to obtain an equivalent system whose coefficient
	matrix is in row echelon form. Indicate whether the system is consistent. If the system is consistent and involves no free variables,
	use back substitution to find the unique solution. If the system is consistent and there are free variables, transform it to reduced
	row echelon form and find all solutions.}
	\begin{enumerate}[label=(\alph*)]
	\setcounter{enumi}{5}\item{
	\[\begin{array}{rrrrrrr}
		x_1 & - {} & x_2 & {} + {}     &   2x_3 &  = {}& 4\\
		2x_1 & + {} & 3x_2 &  {} - {} & x_3 & = {}& 1\\
		7x_1 & + {} & 3x_2 & {} + {}  & 4x_3 &  = {}& 7
	\end{array}\]
	Rewriting this in the form of augmented matrix we have
	\[
	\left(\begin{array}{ccc|c}
		1&-1&2&4\\2&3&-1&1\\7&3&4&7
	\end{array}\right)
		\]
	The first row is already pivoted, so we straightly proceed to making everything under it vanish
	\[\myexplainf
		{\left(\begin{array}{ccc|c}1&-1&2&4\\2&3&-1&1\\7&3&4&7\end{array}\right)}
		{\textcircled{2}-2$*$\textcircled{1}}{\textcircled{3}-7$*$\textcircled{1}}
		{\left(\begin{array}{ccc|c}1&-1&2&4\\0&5&-5&-7\\0&10&-10&-21\end{array}\right)}
	\]
	Next we pivot the second row
	\[\myexplain
		{\left(\begin{array}{ccc|c}1&-1&2&4\\0&5&-5&-7\\0&10&-10&-21\end{array}\right)}
		{\textcircled{2}$/$5}
		{\left(\begin{array}{ccc|c}1&-1&2&4\\0&1&-1&-\myfrac{7}{5}\\0&10&-10&-21\end{array}\right)}
	\]
	and zero out everything under the pivot
	\[\myexplain
		{\left(\begin{array}{ccc|c}1&-1&2&4\\0&1&-1&-\myfrac{7}{5}\\0&10&-10&-21\end{array}\right)}
			{\textcircled{3}-10$*$\textcircled{2}}
		{\left(\begin{array}{ccc|c}1&-1&2&4\\0&1&-1&-\myfrac{7}{5}\\0&0&0&-7\end{array}\right)}
	\]
	System is inconsistent because of the third row and we stop here
	}
	\setcounter{enumi}{9}\item{
	\[\begin{array}{rrrrrrrrr}
		x_1 & + {} & 2x_2 & {} - {} &   3x_3 & + {} & x_4 & = & 1\\
		-x_1 & - {} & x_2 & {} + {} &   4x_3 & - {} & x_4 & = & 6\\
		-2x_1 & - {} & 4x_2 & {} + {} &   7x_3 & - {} & x_4 & = & 1
	\end{array}\]
	Rewriting this in the form of augmented matrix we have
	\[
	\left(\begin{array}{rrrr|r}1&2&-3&1&1\\-1&-1&4&-1&6\\-2&-4&7&-1&1\end{array}\right)
		\]
	Again, the first row already has pivot, so we do the next step -- eliminate everything under the pivot
	\[\myexplainf
		{\left(\begin{array}{rrrr|r}1&2&-3&1&1\\-1&-1&4&-1&6\\-2&-4&7&-1&1\end{array}\right)}
		{\textcircled{2}$+$\textcircled{1}}{\textcircled{3}$+2*$\textcircled{1}}
		{\left(\begin{array}{rrrr|r}1&2&-3&1&1\\0&1&1&0&7\\0&0&1&1&3\end{array}\right)}
	\]
	Accidentally, this brings system to row echelon form. However, it has free variables ($x_4$, corresponding to the fourth column
	is free, as there is no pivot in fourth column). Therefore, as problem requires, we bring the augmented matrix to reduced row echelon
	form. First, eliminate everything above the pivot of the third row
	\[\myexplainf
		{\left(\begin{array}{rrrr|r}1&2&-3&1&1\\0&1&1&0&7\\0&0&1&1&3\end{array}\right)}
		{\textcircled{2}$-$\textcircled{3}}{\textcircled{1}$+3*$\textcircled{3}}
		{\left(\begin{array}{rrrr|r}1&2&0&4&10\\0&1&0&-1&4\\0&0&1&1&3\end{array}\right)}
	\]
	Next, eliminate everything above the pivot of the second row
	\[\myexplain
		{\left(\begin{array}{rrrr|r}1&2&0&4&10\\0&1&0&-1&4\\0&0&1&1&3\end{array}\right)}
		{\textcircled{1}$-2*$\textcircled{2}}
		{\left(\begin{array}{rrrr|r}1&0&0&6&2\\0&1&0&-1&4\\0&0&1&1&3\end{array}\right)}
	\]
	From this, we see that all the solutions can be parametrized as
	\[x_1=2-6t,\;x_2=4+t,\;x_3=3-t,\;x_4=t\]
	}
	\setcounter{enumi}{11}\item{
	\[\begin{array}{rrrrrrr}
		x_1 & - {} & 3x_2 & {} + {}     &   x_3 &  = {}& 1\\
		2x_1 & + {} & x_2 & {} - {}     &   x_3 &  = {}& 2\\
		x_1 & + {} & 4x_2 & {} - {}     &   2x_3 &  = {}& 1\\
		5x_1 & - {} & 8x_2 & {} + {}     &  2x_3 &  = {}& 5
	\end{array}\]
	Rewriting this in the form of augmented matrix we have
	\[
	\left(\begin{array}{rrr|r}1&-3&1&1\\2&1&-1&2\\1&4&-2&1\\5&-8&2&5\end{array}\right)
		\]
	Again, the first row already has pivot, so we proceed directly to elimination
	\[\myexplainfi
		{\left(\begin{array}{rrr|r}1&-3&1&1\\2&1&-1&2\\1&4&-2&1\\5&-8&2&5\end{array}\right)}
		{\textcircled{2}$-2*$\textcircled{1}}
		{\textcircled{3}$-$\textcircled{1}}
		{\textcircled{4}$-5*$\textcircled{1}}
		{\left(\begin{array}{rrr|r}1&-3&1&1\\0&7&-3&0\\0&7&-3&0\\0&7&-3&0\end{array}\right)}
	\]
	We drop two last rows as they all are duplicates of the second row and do pivoting in the second row,
	bringing the matrix to row echelon form
	\[\myexplain
		{\left(\begin{array}{rrr|r}1&-3&1&1\\0&7&-3&0\end{array}\right)}
		{\textcircled{2}$/7$}
		{\left(\begin{array}{rrr|r}1&-3&1&1\\0&1&-\myfrac{3}{7}&0\end{array}\right)}
	\]
	Next, as we have free variables, we have to bring matrix to row echelon form. To achieve this, we just need to eliminate everything under
	the pivot of the second row
	\[\myexplain
		{\left(\begin{array}{rrr|r}1&-3&1&1\\0&1&-\myfrac{3}{7}&0\end{array}\right)}
		{\textcircled{1}$+3*$\textcircled{2}}
		{\left(\begin{array}{rrr|r}1&0&-\myfrac{2}{7}&1\\0&1&-\myfrac{3}{7}&0\end{array}\right)}
	\]
	Answer is then parametrized as \[x_1=1+\frac{2}{7}t,\;x_2=\frac{3}{7}t,\;x_3=t\]
	}
	\end{enumerate}
	}
\item[\# 9.]{{\it Consider a linear system whose augmented matrix is of the form}
	\[{\left(\begin{array}{rrr|r}1&2&1&0\\2&5&3&0\\-1&1&\beta&0\end{array}\right)}\]
	\begin{enumerate}[label=(\alph*)]
		\item{{\it Is it possible for the system to be inconsistent? Explan.} This system will be consistent for any value of a parameter
			$\beta$, as $x_1=x_2=x_3=0$ will be a solution for it, as can be seen by substitution.
			Therefore, there always be at least one solution.}
		\item{{\it For what values of $\beta$ will system have infinitely many solutions?} Let's perform Gaussian elimination to see
			when we will have free variables
			\[\myexplainf
			{\left(\begin{array}{rrr|r}1&2&1&0\\2&5&3&0\\-1&1&\beta&0\end{array}\right)}
			{\textcircled{2}$-2*$\textcircled{1}}
			{\textcircled{3}$+$\textcircled{1}}
			{\left(\begin{array}{rrr|r}1&2&1&0\\0&1&1&0\\0&3&\beta+1&0\end{array}\right)}
			\]
			\[\myexplain
			{\left(\begin{array}{rrr|r}1&2&1&0\\0&1&1&0\\0&3&\beta+1&0\end{array}\right)}
			{\textcircled{3}$-3*$\textcircled{2}}
			{\left(\begin{array}{rrr|r}1&2&1&0\\0&1&1&0\\0&0&\beta-2&0\end{array}\right)}
			\]
			As we want to have free variables, we would like to make the third row consisting of all zeros, so that it become vacuous.
			The only way to achieve this is to set \[\beta=2\].
			}
	\end{enumerate}
	}
\item[\# 11.]{{\it Given the linear systems}
	\begin{multicols}{2}\begin{enumerate}[label=(\alph*)]
		\item $\begin{array}[t]{rrrrr}
				x_1 & + {} & 2x_2 & = & 2\\
				3x_1 & + {} & 7x_2 & = & 8
			\end{array}$
		\item $\begin{array}[t]{rrrrr}
				x_1 & + {} & 2x_2 & = & 1\\
				3x_1 & + {} & 7x_2 & = & 7
			\end{array}$
	\end{enumerate}\end{multicols}
	{\it Solve both systems by incorporating the right-hand sides into a $2\times 2$ matrix $B$ and computing the reduced row echelon form of}
	\[\left(A\mid B\right)=\left(\begin{array}{rr|rr}1&2&2&1\\3&7&8&7\end{array}\right)\]
	The first row is already pivoted, so we proceed directly to elimination
	\[\myexplain
	{\left(\begin{array}{rr|rr}1&2&2&1\\3&7&8&7\end{array}\right)}
	{\textcircled{2}$-3*$\textcircled{1}}
	{\left(\begin{array}{rr|rr}1&2&2&1\\0&1&2&4\end{array}\right)}
	\]
	Accidentally, this makes second raw pivoted and we just clear everything above pivot of the second row
	\[\myexplain
	{\left(\begin{array}{rr|rr}1&2&2&1\\0&1&2&4\end{array}\right)}
	{\textcircled{1}$-2*$\textcircled{2}}
	{\left(\begin{array}{rr|rr}1&0&-2&-7\\0&1&2&4\end{array}\right)}
	\]
	From this we immediately see, that the answer to the first subproblem is
	\[x_1=-2,\;x_2=2\]
	whereas the answer to the second is
	\[x_1=-7,\;x_2=4\]
	}
\item[\# 17.]{{\it Liquid benzene burns in the atmosphere. If a cold object is placed directly over the benzene, water will condense on
	the object and a deposit of soot (carbon) will also form on the object. The chemical equation of this reaction is of the form}
	\[x_1\mbox{C}_6\mbox{H}_6+x_2\mbox{O}_2\to x_3\mbox{C}+x_4\mbox{H}_2\mbox{O}\]
	{\it Determine values of $x_1,x_2,x_3,$ and $x_4$ to balance the equation.}
	Number of atoms of each type should be equal no each side for the reaction to be balanced, so equating the number of 
	atoms of $\mbox{C}$ we get
	\[6x_1=x_3\implies 6x_1-x_3=0\]
	equating the number of atoms of $\mbox{H}$ gives
	\[6x_1=2x_4\implies 6x_1-2x_4=0\]
	and equating the number of atoms of $\mbox{O}$ gives
	\[2x_2=x_4\implies 2x_2-x_4=0\]
	Putting all these three equations in the form of augmented matrix gives us
	\[{\left(\begin{array}{rrrr|r}6&0&-1&0&0\\6&0&0&-2&0\\0&2&0&-1&0\end{array}\right)}\]
	As usual, we start by making pivot in the first row
	\[\myexplain
	{\left(\begin{array}{rrrr|r}6&0&-1&0&0\\6&0&0&-2&0\\0&2&0&-1&0\end{array}\right)}
	{\textcircled{1}$/6$}
	{\left(\begin{array}{rrrr|r}1&0&-\myfrac{1}{6}&0&0\\6&0&0&-2&0\\0&2&0&-1&0\end{array}\right)}
	\]
	Next, we eliminate everything under the pivot.
	\[\myexplain
	{\left(\begin{array}{rrrr|r}1&0&-\myfrac{1}{6}&0&0\\6&0&0&-2&0\\0&2&0&-1&0\end{array}\right)}
	{\textcircled{2}$-6*$\textcircled{1}}
	{\left(\begin{array}{rrrr|r}1&0&-\myfrac{1}{6}&0&0\\0&0&1&-2&0\\0&2&0&-1&0\end{array}\right)}
	\]
	We see that we cannot make pivot in the second row, while there are nonzero elements below the second row in the second column -- so we
	interchange the third and the second row.
	\[\myexplain
	{\left(\begin{array}{rrrr|r}1&0&-\myfrac{1}{6}&0&0\\0&0&1&-2&0\\0&2&0&-1&0\end{array}\right)}
	{\textcircled{3}$\leftrightarrow$\textcircled{2}}
	{\left(\begin{array}{rrrr|r}1&0&-\myfrac{1}{6}&0&0\\0&2&0&-1&0\\0&0&1&-2&0\end{array}\right)}
	\]
	Now we can make pivot in the second row
	\[\myexplain
	{\left(\begin{array}{rrrr|r}1&0&-\myfrac{1}{6}&0&0\\0&2&0&-1&0\\0&0&1&-2&0\end{array}\right)}
	{\textcircled{2}$/2$}
	{\left(\begin{array}{rrrr|r}1&0&-\myfrac{1}{6}&0&0\\0&1&0&-\myfrac{1}{2}&0\\0&0&1&-2&0\end{array}\right)}
	\]
	Matrix now is in row echelon form and we will do one more step to bring the matrix to reduced row echelon form, namely eliminate everything
	under the pivot of the third row
	\[\myexplain
	{\left(\begin{array}{rrrr|r}1&0&-\myfrac{1}{6}&0&0\\0&1&0&-\myfrac{1}{2}&0\\0&0&1&-2&0\end{array}\right)}
	{\textcircled{1}$+$\textcircled{3}$/6$}
	{\left(\begin{array}{rrrr|r}1&0&0&-\myfrac{1}{3}&0\\0&1&0&-\myfrac{1}{2}&0\\0&0&1&-2&0\end{array}\right)}
	\]
	From this we can readily parametrize the solution set as
	\[x_1=\frac{t}{3},\;x_2=\frac{t}{2},\;x_3=2t,\;x_4=t\]
	In particular (setting $t=6$ to make all the variables attain positive integer values), the values $x_1=2,\;x_2=3,\;x_3=12,\;x_4=6$ makes the 
	equation balanced.
	}
\end{description}
\end{document}
