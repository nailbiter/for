%% LyX 2.0.8.1 created this file.  For more info, see http://www.lyx.org/.
%% Do not edit unless you really know what you are doing.
\documentclass[english]{article}
\usepackage[T1]{fontenc}
\usepackage[latin9]{inputenc}
\usepackage{refstyle}
\usepackage{amsthm}
\usepackage{amsmath}
\usepackage{amssymb}

\makeatletter

%%%%%%%%%%%%%%%%%%%%%%%%%%%%%% LyX specific LaTeX commands.

\AtBeginDocument{\providecommand\propref[1]{\ref{prop:#1}}}
\AtBeginDocument{\providecommand\lemref[1]{\ref{lem:#1}}}
\RS@ifundefined{subref}
  {\def\RSsubtxt{section~}\newref{sub}{name = \RSsubtxt}}
  {}
\RS@ifundefined{thmref}
  {\def\RSthmtxt{theorem~}\newref{thm}{name = \RSthmtxt}}
  {}
\RS@ifundefined{lemref}
  {\def\RSlemtxt{lemma~}\newref{lem}{name = \RSlemtxt}}
  {}


%%%%%%%%%%%%%%%%%%%%%%%%%%%%%% Textclass specific LaTeX commands.
\theoremstyle{plain}
\newtheorem{thm}{\protect\theoremname}
  \theoremstyle{plain}
  \newtheorem{prop}[thm]{\protect\propositionname}
  \theoremstyle{plain}
  \newtheorem{lem}[thm]{\protect\lemmaname}

%%%%%%%%%%%%%%%%%%%%%%%%%%%%%% User specified LaTeX commands.
\usepackage{mystyle}

\makeatother

\usepackage{babel}
  \providecommand{\lemmaname}{Lemma}
  \providecommand{\propositionname}{Proposition}
\providecommand{\theoremname}{Theorem}

\begin{document}

\section{Settings and Notations}
\begin{enumerate}
\item We let $p,q>\mathbb{Z}_{>0}$
\item We let $K:=O(p+1)\times O(q+1)$ be the maximal compact subgroup of
$G:=O(p+1,q+1)$.
\item We will study kernel $K_{\lambda,\nu}^{C}\in\mathcal{D}'(\mathbb{R}^{p,q})$
of the singular symmetry breaking operator supported on $\{Q=0\}$
as defined in {[}todai\_masterreport\_15\_10\_14.pdf{]}. It was defined
for $\nu\in2\mathbb{Z}_{\geq0}+1$ and $\left\{ \lambda\in\mathbb{C}\;|\;\mbox{\mbox{Re}}(\lambda-\mu-n)\notin\mathbb{Z}_{\le1}\right\} $
\end{enumerate}

\section{Statements}
\begin{prop}
\label{prop:On-the-}On the $\mathfrak{n}_{-}\simeq\mathbb{R}^{p,q}$
the $K$-finite vectors are of the form \textbf{
\[
F_{M,N,n',m',\psi,\psi'}(r\omega_{p-1},s\omega_{q-1}):=
\]
}

\[
:=R(r,s)^{-\lambda/2}\left(\frac{r}{\sqrt{R(r,s)}}\right)^{N}\left(\frac{s}{\sqrt{R(r,s)}}\right)^{M}\left(\frac{1-(r^{2}-s^{2})}{\sqrt{R(r,s)}}\right)^{n'}\left(\frac{1+(r^{2}-s^{2})}{\sqrt{R(r,s)}}\right)^{m'}\psi(\omega_{p-1})\psi'(\omega_{q-1})
\]


where $m',n',M,N\in\mathbb{Z}_{\geq0}$, $m'+n'+M+N\in2\mathbb{Z}_{\geq0}$,
$R(r,s):=(1-r^{2}+s^{2})^{2}+4r^{2}$ and $(\psi,\psi')\in\mathcal{H}^{N}(\mathbb{S}^{p-1})\times\mathcal{H}^{M}(\mathbb{S}^{q-1})$.
\end{prop}

\begin{prop}
\label{prop:The-following-holds}The following holds\end{prop}
\begin{enumerate}
\item We have$\left\langle K_{\lambda,\nu}^{C},F_{M,N,n',m',\psi,\psi'}\right\rangle =0$
if $M>0$, $N$:odd or $\psi\perp\mathcal{H}^{N}(\mathbb{S}^{p-1})^{O(p-1)}$.
\item When $N$:even and $\psi(\omega_{p-1}):=\widetilde{\widetilde{C}}_{N}^{p/2-1}(\omega_{p-1}^{(p)})$,
we have
\[
\left\langle K_{\lambda,\nu}^{C},F_{M,N,n',m',\psi,1}\right\rangle =R\cdot S
\]
where 
\[
S\simeq\frac{\Gamma\left(\lambda+\nu-p+1\right)}{\Gamma\left(\frac{\lambda+\nu-p-N+2}{2}\right)\Gamma\left(\frac{\lambda+\nu+N}{2}\right)}
\]
and 
\[
R=\sum_{i=0}^{\nu-1}c_{i}(m')\left(\frac{q-2}{2}\right)_{\nu-1-i}\frac{\Gamma\left(\frac{\lambda+N-\nu}{2}-i\right)\Gamma\left(\frac{m'+n'+\nu-2i}{2}\right)}{\Gamma\left(\frac{\lambda+N+m'+n'}{2}\right)}
\]
\\
where $c_{i}(m')\in\mathbb{Q}[m']$ with $c_{0}=1$ and $(\alpha)_{j}=\alpha(\alpha-1)\dots(\alpha-j+1)$
denotes the falling factorial.\end{enumerate}
\begin{prop}
\label{prop:Define--as}Define $\widetilde{K}_{\lambda,\nu}^{C}$
as 
\[
\widetilde{K}_{\lambda,\nu}^{C}:=\begin{cases}
\frac{K_{\lambda,\nu}^{C}}{\Gamma\left(\frac{\lambda+\nu-q}{2}\right)\Gamma\left(\frac{\lambda+\nu-p+1}{2}\right)} & q\mbox{: even and }2\nu\ge q\\
\frac{K_{\lambda,\nu}^{C}}{\Gamma\left(\frac{\lambda-\nu}{2}\right)\Gamma\left(\frac{\lambda+\nu-p+1}{2}\right)} & \mbox{otherwise}
\end{cases}
\]


Then $\forall\nu\in2\mathbb{Z}_{\ge0}+1,\forall F_{M,N,n',m',\psi,\psi'},\;\lambda\mapsto\left\langle \widetilde{K}_{\lambda,\nu}^{C},F_{M,N,n',m',\psi,\psi'}\right\rangle $
is a holomorphic map and $\forall(\lambda,\nu)\in\mathbb{C}\times(2\mathbb{Z}_{\ge0}+1),\;\widetilde{K}_{\lambda,\nu}^{C}\ne0$.
\end{prop}

\begin{lem}
\label{lem:Suppose-there-exists}Suppose there exists $\left\{ c_{i}\right\} _{i=0}^{m},\alpha\in\mathbb{C}$
and increasing sequence $\left\{ k_{n}\right\} _{n}\subset\mathbb{Z}$
such that 
\[
\forall n\implies\sum_{i=0}^{m}c_{i}\Gamma(\alpha+i+k_{n})=0.
\]


Then all $c_{i}$ are zero.
\end{lem}

\begin{lem}
For $\lambda\in\mathbb{C}$, we have an isomorphism of Frechet$G$-representations
$\left(\pi_{\lambda,K},C^{\infty}(\mathbb{S}^{p}\times\mathbb{S}^{q})^{\mathbb{Z}_{2}}\right)\xrightarrow{\sim}\left(l_{\Xi},C_{-\lambda}^{\infty}(\Xi^{p+1,q+1})\right)$
given by $f\mapsto\left((x,y)\mapsto\right)$, where $\Xi^{p+1,q+1}:=\left\{ (x,y)\in\mathbb{R}^{p+1,q+1}\big|\left|x\right|_{p+1}=\left|y\right|_{q+1}\right\} \setminus\left\{ 0\right\} $,
$C_{-\lambda}^{\infty}(\Xi^{p+1,q+1}):=\left\{ f\in C^{\infty}(\Xi^{p+1,q+1})\big|\forall(t,x)\in\mathbb{R}^{\times}\times\Xi^{p+1,q+1},\; f(tx)=\left|t\right|^{-\lambda}f(x)\right\} $,
$C^{\infty}(\mathbb{S}^{p}\times\mathbb{S}^{q})^{\mathbb{Z}_{2}}:=\left\{ f\in C^{\infty}(\mathbb{S}^{p}\times\mathbb{S}^{q})\big|\forall(x,y)\in\mathbb{S}^{p}\times\mathbb{S}^{q},\; f(x,y)=f(-x,-y)\right\} $
$l_{\Xi}$ is left regular action induced by action of $G$ on $\Xi^{p+1,q+1}$
and $(\pi_{\lambda,K}(g)f)(x):=\left|g^{-1}\cdot x\right|^{-\lambda}f\left(\frac{g^{-1}\cdot x}{\left|g^{-1}\cdot x\right|}\right)$,
where $g^{-1}\cdot x$ refers to action of $G\curvearrowright\Xi^{p+1,q+1}$
and for $(x,y)\in\Xi^{p+1,q+1},\;\left|(x,y)\right|:=\left|x\right|_{p+1}$.
\end{lem}

\begin{lem}
Let $S\subset C^{\infty}(\mathbb{S}^{p}\times\mathbb{S}^{q})^{\mathbb{Z}_{2}}$
denote the subspace of $K$-finite vectors. Then,\end{lem}
\begin{enumerate}
\item $S=\sum_{\left\{ (l,k)\in\mathbb{Z}_{\ge0}^{2}\big|l+k\in2\mathbb{Z}_{\ge0}\right\} }\mathcal{H}^{l}(\mathbb{S}^{p})\otimes\mathcal{H}^{k}(\mathbb{S}^{q})$,
where $\sum$ denotes the span;
\item $S$ is spanned by functions of the form $(\eta_{0},\eta,\xi_{0},\xi)\mapsto\left|\eta\right|^{N}\psi\left(\frac{\eta}{\left|\eta\right|}\right)\eta_{0}^{m'}\left|\xi\right|^{M}\psi'\left(\frac{\xi}{\left|\xi\right|}\right)\xi^{n'}$
where $\left((\eta_{0},\eta),(\xi_{0},\xi)\right)\in\mathbb{S}^{p}\times\mathbb{S}^{q}$,
$M,N,m',n'\in\mathbb{Z}_{\geq0}$ with $M+N+m'+n'\in2\mathbb{Z}_{\geq0}$
and $\left(\psi,\psi\right)\in\mathcal{H}^{N}(\mathbb{S}^{p})\times\mathcal{H}^{M}(\mathbb{S}^{q})$.;
\end{enumerate}

\section{Proofs}
\begin{proof}
(of \propref{On-the-}) Recall that the $K$-finite vectors of $G\supset K=O(p+1)\times Q(q+1)$
acting on $C^{\infty}(\mathbb{S}^{p}\times\mathbb{S}^{q})$ are given
as $\bigoplus_{k,l=0}^{\infty}\mathcal{H}^{k}(\mathbb{S}^{p})\otimes\mathcal{H}^{l}(\mathbb{S}^{l})$,
as explained in \cite{howe1993homogeneous}. 
\end{proof}

\begin{proof}
(of \lemref{Suppose-there-exists}) Recall the formula $\Gamma(t+n)=t^{(n)}\cdot\Gamma(t)$
(here $t^{(n)}:=t(t+1)\dots(t+n-1)$ denotes the rising factioral).
It implies that $0=\sum_{i=0}^{m}c_{i}\Gamma(\alpha+i+k_{n})=\Gamma(\alpha+k_{n})\sum_{i=0}^{m}c_{i}(\alpha+k_{n})^{(i)}$.
As $\Gamma(\alpha+n)\neq0$ for $n$ big enough, this implies that
$\forall n>N,\;\sum_{i=0}^{m}c_{i}(\alpha+k_{n})^{(i)}=0$ and since
the latter expression is a polynomial in $k_{n}$, it implies that
$\forall x\in\mathbb{C},\;\sum_{i=0}^{m}c_{i}x^{(n)}=0$ and hence
that all $c_{i}$ should vanish.

\bibliographystyle{plain}
\bibliography{/home/u406/for/forlatex/todai_master}
\end{proof}

\end{document}
