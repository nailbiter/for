%% LyX 2.0.8.1 created this file.  For more info, see http://www.lyx.org/.
%% Do not edit unless you really know what you are doing.
\documentclass[english]{article}
\usepackage[T1]{fontenc}
\usepackage[latin9]{inputenc}
\usepackage{refstyle}
\usepackage{amsthm}
\usepackage{amsmath}
\usepackage{amssymb}
\usepackage{esint}

\makeatletter

%%%%%%%%%%%%%%%%%%%%%%%%%%%%%% LyX specific LaTeX commands.

\AtBeginDocument{\providecommand\propref[1]{\ref{prop:#1}}}
\AtBeginDocument{\providecommand\lemref[1]{\ref{lem:#1}}}
\RS@ifundefined{subref}
  {\def\RSsubtxt{section~}\newref{sub}{name = \RSsubtxt}}
  {}
\RS@ifundefined{thmref}
  {\def\RSthmtxt{theorem~}\newref{thm}{name = \RSthmtxt}}
  {}
\RS@ifundefined{lemref}
  {\def\RSlemtxt{lemma~}\newref{lem}{name = \RSlemtxt}}
  {}


%%%%%%%%%%%%%%%%%%%%%%%%%%%%%% Textclass specific LaTeX commands.
\theoremstyle{plain}
\newtheorem{thm}{\protect\theoremname}
  \theoremstyle{plain}
  \newtheorem{prop}[thm]{\protect\propositionname}
  \theoremstyle{plain}
  \newtheorem{lem}[thm]{\protect\lemmaname}
  \theoremstyle{remark}
  \newtheorem{rem}[thm]{\protect\remarkname}

%%%%%%%%%%%%%%%%%%%%%%%%%%%%%% User specified LaTeX commands.
\usepackage{mystyle}

\makeatother

\usepackage{babel}
  \providecommand{\lemmaname}{Lemma}
  \providecommand{\propositionname}{Proposition}
  \providecommand{\remarkname}{Remark}
\providecommand{\theoremname}{Theorem}

\begin{document}

\section{Settings and Notations}
\begin{enumerate}
\item We let $p,q>\mathbb{Z}_{>0}$
\item We let $K:=O(p+1)\times O(q+1)$ be the maximal compact subgroup of
$G:=O(p+1,q+1)$.
\item We will study kernel $K_{\lambda,\nu}^{C}\in\mathcal{D}'(\mathbb{R}^{p,q})$
of the singular symmetry breaking operator supported on $\{Q=0\}$
as defined in {[}todai\_masterreport\_15\_10\_14.pdf{]}. It was defined
for $\nu\in2\mathbb{Z}_{\geq0}+1$ and $\left\{ \lambda\in\mathbb{C}\;|\;\mbox{\mbox{Re}}(\lambda-\mu-n)\notin\mathbb{Z}_{\le1}\right\} $
\end{enumerate}

\section{Statements}
\begin{prop}
\label{prop:On-the-}On the $\mathfrak{n}_{-}\simeq\mathbb{R}^{p,q}$
the $K$-finite vectors are of the form \textbf{
\[
F_{N,M,n',m',\psi,\psi'}(r\omega_{p-1},s\omega_{q-1}):=
\]
}

\[
:=R(r,s)^{-\lambda/2}\left(\frac{r}{\sqrt{R(r,s)}}\right)^{N}\left(\frac{s}{\sqrt{R(r,s)}}\right)^{M}\left(\frac{1-(r^{2}-s^{2})}{\sqrt{R(r,s)}}\right)^{n'}\left(\frac{1+(r^{2}-s^{2})}{\sqrt{R(r,s)}}\right)^{m'}\psi(\omega_{p-1})\psi'(\omega_{q-1})
\]


where $m',n',M,N\in\mathbb{Z}_{\geq0}$, $m'+n'+M+N\in2\mathbb{Z}_{\geq0}$,
$R(r,s):=(1-r^{2}+s^{2})^{2}+4r^{2}$ and $(\psi,\psi')\in\mathcal{H}^{N}(\mathbb{S}^{p-1})\times\mathcal{H}^{M}(\mathbb{S}^{q-1})$.
\end{prop}

\begin{prop}
\label{prop:The-following-holds}The following holds\end{prop}
\begin{enumerate}
\item We have$\left\langle K_{\lambda,\nu}^{C},F_{M,N,n',m',\psi,\psi'}\right\rangle =0$
if $M>0$, $N$:odd or $\psi\perp\mathcal{H}^{N}(\mathbb{S}^{p-1})^{O(p-1)}$.
\item When $N$ and $m'+n'$:even and $\psi(\omega_{p-1}):=\widetilde{\widetilde{C}}_{N}^{p/2-1}(\omega_{p-1}^{(p)})$%
\footnote{recall that \emph{$\mathcal{H}^{N}(\mathbb{S}^{n-1})^{O(n-1)}=\mathbb{C}\mbox{-span}\tilde{\tilde{C}}_{N}^{\frac{n}{2}-1}(\omega_{n-1}^{(n)})$}%
}, we have
\[
\left\langle K_{\lambda,\nu}^{C},F_{N,0,n',m',\psi,1}\right\rangle =R\cdot S
\]
where 
\[
S\simeq\frac{\Gamma\left(\lambda+\nu-n+1\right)}{\Gamma\left(\frac{\lambda+\nu-n-N+2}{2}\right)\Gamma\left(\frac{\lambda-q+\nu+N}{2}\right)}
\]
(here $\simeq$ means proportionality up to a nonzero constant) and
\[
R\simeq\sum_{i=0}^{\nu-1}c_{i}(m')\left(\frac{q-2}{2}\right)_{\nu-1-i}\frac{\Gamma\left(\frac{\lambda+N-\nu}{2}+i\right)\Gamma\left(\frac{m'+n'+\nu-2i}{2}\right)}{\Gamma\left(\frac{\lambda+N+m'+n'}{2}\right)}
\]
\\
where $c_{i}(m')\in\mathbb{Q}[m']$ with $\deg(c_{i})=i$, $m'+(\nu-1)-2i<0\implies c_{i}(m')=0$
and $(\alpha)_{j}:=\alpha(\alpha-1)\dots(\alpha-j+1)$.
\end{enumerate}

\begin{prop}
\label{prop:Assume-that-1}Assume that $q$ is odd and define $\tilde{K}_{\lambda,\nu}^{C}:=K_{\lambda,\nu}^{C}/\Gamma\left(\frac{\lambda+\nu-n+1}{}\right)/\Gamma\left(\frac{\lambda-\nu}{2}\right)$.
Then $\forall(N,m',n')\in\mathbb{Z}_{\ge0}^{3}$ such that $N,\; m'+n'\in2\mathbb{Z}$
and $\psi:=\tilde{\tilde{C}}_{N}^{p/2-1}(\omega_{p-1}^{(p)})$, we
have $\lambda\mapsto\left\langle \tilde{K}_{\lambda,\nu}^{C},F_{N,0,n',m',\psi,1}\right\rangle $
being holomorphic function. Moreover, $\tilde{K}_{\lambda,\nu}^{C}\ne0$.
\end{prop}

\begin{prop}
\label{prop:Assume-that-2}Assume that $q\in2\mathbb{Z}_{>0}$, $q-2\nu\ge0$
and define $\tilde{K}_{\lambda,\nu}^{C}:=K_{\lambda,\nu}^{C}/\Gamma\left(\frac{\lambda+\nu-n+1}{2}\right)$.
Then $\forall(N,m',n')\in\mathbb{Z}_{\ge0}^{3}$ such that $N,\; m'+n'\in2\mathbb{Z}$
and $\psi:=\tilde{\tilde{C}}_{N}^{p/2-1}(\omega_{p-1}^{(p)})$, we
have $\lambda\mapsto\left\langle \tilde{K}_{\lambda,\nu}^{C},F_{N,0,n',m',\psi,1}\right\rangle $
being holomorphic function. Moreover, $\tilde{K}_{\lambda,\nu}^{C}\ne0$.
\end{prop}

\begin{prop}
\label{prop:Assume-that-3}Assume that $q\in2\mathbb{Z}_{>0}$, $q-2\nu<0$,
$q-1\ge\nu$ and define $\tilde{K}_{\lambda,\nu}^{C}:=K_{\lambda,\nu}^{C}/\Gamma\left(\frac{\lambda+\nu-n+1}{2}\right)$.
Then $\forall(N,m',n')\in\mathbb{Z}_{\ge0}^{3}$ such that $N,\; m'+n'\in2\mathbb{Z}$
and $\psi:=\tilde{\tilde{C}}_{N}^{p/2-1}(\omega_{p-1}^{(p)})$, we
have $\lambda\mapsto\left\langle \tilde{K}_{\lambda,\nu}^{C},F_{N,0,n',m',\psi,1}\right\rangle $
being holomorphic function. Moreover, $\tilde{K}_{\lambda,\nu}^{C}\ne0$.
\end{prop}

\begin{prop}
\label{prop:Assume-that-4}Assume that $q\in2\mathbb{Z}_{>0}$, $q-1<\nu$
and define $\tilde{K}_{\lambda,\nu}^{C}:=K_{\lambda,\nu}^{C}/\Gamma\left(\frac{\lambda+\nu-n+1}{2}\right)l\Gamma\left(\frac{\lambda-\nu+q-2}{2}\right)$.
Then $\forall(N,m',n')\in\mathbb{Z}_{\ge0}^{3}$ such that $N,\; m'+n'\in2\mathbb{Z}$
and $\psi:=\tilde{\tilde{C}}_{N}^{p/2-1}(\omega_{p-1}^{(p)})$, we
have $\lambda\mapsto\left\langle \tilde{K}_{\lambda,\nu}^{C},F_{N,0,n',m',\psi,1}\right\rangle $
being holomorphic function. Moreover, $\tilde{K}_{\lambda,\nu}^{C}\ne0$.
\end{prop}

\begin{lem}
\label{lem:Suppose-there-exists}Suppose there exists $\left\{ c_{i}\right\} _{i=0}^{m},\alpha\in\mathbb{C}$
and increasing unbounded sequence $\left\{ k_{n}\right\} _{n}\subset\mathbb{Q}$
such that 
\[
\forall n\implies\sum_{i=0}^{m}c_{i}\Gamma(\alpha+i+k_{n})=0.
\]


Then all $c_{i}$ are zero.
\end{lem}

\begin{lem}
\label{lem:For-,-we}For $\lambda\in\mathbb{C}$, we have an isomorphism
of Frechet$G$-representations $\left(\pi_{\lambda,K},C^{\infty}(\mathbb{S}^{p}\times\mathbb{S}^{q})^{\mathbb{Z}_{2}}\right)\xrightarrow{\sim}\left(l_{\Xi},C_{-\lambda}^{\infty}(\Xi^{p+1,q+1})\right)$
given by $f\mapsto\left((x,y)\mapsto\left|x\right|^{-\lambda}f\left[\frac{(x,y)}{\left|x\right|}\right]\right)$,
where 
\[
\Xi^{p+1,q+1}:=\left\{ (x,y)\in\mathbb{R}^{p+1,q+1}\big|\left|x\right|_{p+1}=\left|y\right|_{q+1}\right\} \setminus\left\{ 0\right\} 
\]
 
\[
C^{\infty}(\mathbb{S}^{p}\times\mathbb{S}^{q})^{\mathbb{Z}_{2}}:=\left\{ f\in C^{\infty}(\mathbb{S}^{p}\times\mathbb{S}^{q})\big|\forall(x,y)\in\mathbb{S}^{p}\times\mathbb{S}^{q},\; f(x,y)=f(-x,-y)\right\} 
\]
, $l_{\Xi}$ is left regular action induced by action of $G$ on $\Xi^{p+1,q+1}$
and $(\pi_{\lambda,K}(g)f)(x):=p\left(g^{-1}\cdot x\right)^{-\lambda}f\left(\frac{g^{-1}\cdot x}{p\left(g^{-1}\cdot x\right)}\right)$,
where $g^{-1}\cdot x$ refers to action of $G\curvearrowright\Xi^{p+1,q+1}$
and for $(x,y)\in\Xi^{p+1,q+1},\; p((x,y)):=\left|x\right|_{p+1}$.
Moreover, $\pi_{\lambda,K}\big|_{K}$ coincides with left-regular
action induced by action $K\curvearrowright\mathbb{S}^{p}\times\mathbb{S}^{q}$.
\end{lem}

\begin{lem}
\label{lem:Let--denote}Let $S\subset C^{\infty}(\mathbb{S}^{p}\times\mathbb{S}^{q})^{\mathbb{Z}_{2}}$
denote the subspace of $K$-finite vectors. Then,\end{lem}
\begin{enumerate}
\item $S=\sum_{\left\{ (l,k)\in\mathbb{Z}_{\ge0}^{2}\big|l+k\in2\mathbb{Z}_{\ge0}\right\} }\mathcal{H}^{l}(\mathbb{S}^{p})\otimes\mathcal{H}^{k}(\mathbb{S}^{q})$,
where $\sum$ denotes the span;
\item $S$ is spanned by functions of the form $(\eta_{0},\eta,\xi_{0},\xi)\mapsto\left|\eta\right|^{N}\psi\left(\frac{\eta}{\left|\eta\right|}\right)\eta_{0}^{n'}\left|\xi\right|^{M}\psi'\left(\frac{\xi}{\left|\xi\right|}\right)\xi_{0}^{m'}$
where $\left((\eta_{0},\eta),(\xi_{0},\xi)\right)\in\mathbb{S}^{p}\times\mathbb{S}^{q}$,
$M,N,m',n'\in\mathbb{Z}_{\geq0}$ with $M+N+m'+n'\in2\mathbb{Z}_{\geq0}$
and $\left(\psi,\psi\right)\in\mathcal{H}^{N}(\mathbb{S}^{p})\times\mathcal{H}^{M}(\mathbb{S}^{q})$.\end{enumerate}
\begin{lem}
\label{lem:Suppose-that-sets}Suppose that $\left\{ A_{N}\right\} _{N}$,
$\left\{ B_{N}\right\} _{N}$ and $\left\{ C_{N}\right\} _{N}$ are
three families of sets indexed with $N\in\Lambda$. Then, the following
holds:\end{lem}
\begin{enumerate}
\item If $\forall N,\; N'\in\Lambda$ we have $A_{N'}\cap B_{N}=\emptyset$,
then $\bigcap_{N\in\Lambda}\left(A_{N}\cup B_{N}\right)=\bigcap_{N\in\Lambda}A_{N}\cup\bigcup_{N\in\Lambda}B_{N}$.
\item If $\forall N,\; N'\in\Lambda$ we have $A_{N'}\cap B_{N}=\emptyset$,
$x\in\bigcap_{N\in\Lambda}\left(A_{N}\cup B_{N}\cup C_{N}\right)$
and $\exists N_{0}\in\Lambda$, such that $A_{N_{0}}\ni x$, then
$x\in\bigcap\left(A_{N}\cup C_{N}\right)$\end{enumerate}
\begin{rem}
Proof of this lemma is straightforward and will be omitted.
\end{rem}

\section{Proofs}
\begin{proof}
(of prop. \propref{On-the-}) Similarly to \cite{kobayashi2015symmetry,Kobayashi201489},
we have $I(\lambda)\simeq\left(l_{\Xi},C_{-\lambda}^{\infty}(\Xi^{p+1,q+1})\right)$,
so all we need to do is to calculate the pullback of the $K$-finite
vectors of $C_{-\lambda}^{\infty}(\Xi^{p+1,q+1})$ under the embedding
\[
\mathfrak{n}_{-}\simeq\mathbb{R}^{p,q}\ni(x,y)\mapsto(1-\left|x\right|^{2}+\left|y\right|^{2},\,2x,\,2y,\,1+\left|x\right|^{2}-\left|y\right|^{2})\in\Xi^{p+1,q+1}
\]


In turn, lemma \ref{lem:For-,-we} and the second item of \lemref{Let--denote}
tells us what is the form of $K$-finite vectors in $\left(l_{\Xi},C_{-\lambda}^{\infty}(\Xi^{p+1,q+1})\right)$,
and this gives the desired conclusion.
\end{proof}

\begin{proof}
(of prop. \ref{prop:The-following-holds}) Let us transfer to bipolar
coordinates, where

\[
\left\langle K_{\lambda,\nu}^{C},F_{N,M,n',m',\psi,\psi'}\right\rangle =
\]


\[
=\int_{r=0}^{\infty}\int_{s=0}^{\infty}\delta^{(\nu-1)}(\frac{s^{2}}{r^{2}}-1)R(r,s)^{-\lambda/2}\left(\frac{r}{\sqrt{R(r,s)}}\right)^{N}\left(\frac{s}{\sqrt{R(r,s)}}\right)^{M}\left(\frac{1-(r^{2}-s^{2})}{\sqrt{R(r,s)}}\right)^{n'}\left(\frac{1+(r^{2}-s^{2})}{\sqrt{R(r,s)}}\right)^{m'}\times
\]


\[
\times r^{\lambda+\nu-n}r^{p-1}s^{q-1}dr\, ds\times
\]


\[
\int_{\mathbb{S}^{p-1}}\psi(\omega_{p-1})\left|\omega_{p-1}^{(p)}\right|^{\lambda+\nu-n}d\omega_{p-1}\int_{\mathbb{S}^{q-1}}\psi'(\omega_{q-1})d\omega_{q-1}
\]


Now, as $\psi'\in\mathcal{H}^{M}(\mathbb{S}^{q-1})$ we see that $\int_{\mathbb{S}^{q-1}}\psi'(\omega_{q-1})d\omega_{q-1}=0$
unless $M=0$. Furthermore, as explained in \cite[lem 7.6]{kobayashi2015symmetry},
$\int_{\mathbb{S}^{p-1}}\psi(\omega_{p-1})\left|\omega_{p-1}^{(p)}\right|^{\lambda+\nu-n}d\omega_{p-1}=0$
if $N$ is odd or $\psi\perp\mathcal{H}^{N}(\mathbb{S}^{p-1})^{O(p-1)}$.
This proves the first item.

Regarding the second item, the $S$ factor comes from the shown in
\cite[lem. 7.6]{kobayashi2015symmetry} identity
\[
\int_{\mathbb{S}^{n-1}}\left|\omega_{n-1}^{(n)}\right|^{\lambda+\nu-n}\tilde{\tilde{C}}_{N}^{\frac{n}{2}-1}(\omega_{n-1}^{(n)})d\omega_{n-1}\simeq\frac{\Gamma\left(\lambda+\nu-n+1\right)}{\Gamma\left(\frac{\lambda+\nu-n-N+2}{2}\right)\Gamma\left(\frac{\lambda+\nu-N}{2}\right)}
\]


We next rewrite the integral

\[
\int_{r=0}^{\infty}\int_{s=0}^{\infty}\delta^{(\nu-1)}(\frac{s^{2}}{r^{2}}-1)R(r,s)^{-\lambda/2}\left(\frac{r}{\sqrt{R(r,s)}}\right)^{N}\left(\frac{1-(r^{2}-s^{2})}{\sqrt{R(r,s)}}\right)^{n'}\left(\frac{1+(r^{2}-s^{2})}{\sqrt{R(r,s)}}\right)^{m'}\times r^{\lambda+\nu-n}r^{p-1}s^{q-1}dr\, ds
\]


in coordinates $(x,y)$ given as 
\[
x:=\frac{1+(r^{2}-s^{2})}{\sqrt{R(r,s)}},\quad y:=\frac{1-(r^{2}-s^{2})}{\sqrt{R(r,s)}}
\]


The integration domain then becomes $\left\{ (x,y)\in[-1,1]\times[-1,1]\big|x+y>0\right\} $
and the whole integral becomes (up to proportionality)

\[
\int_{(x,y)\in D}\delta^{(\nu-1)}(x-y)(1-y^{2})^{\frac{\lambda+N+\nu-q-2}{2}}(1-x^{2})^{\frac{q-2}{2}}x^{m'}y^{n'}dx\, dy\simeq
\]


\[
\simeq\int_{y=0}^{1}(1-y^{2})^{\frac{\lambda+N+\nu-q-2}{2}}y^{n'}\left[\frac{\partial^{\nu-1}}{\partial x^{\nu-1}}\bigg|_{x=y}(1-x^{2})^{\frac{q-2}{2}}x^{m'}\right]dy
\]


Now, inductively one sees that 
\[
\frac{\partial^{\nu-1}}{\partial x^{\nu-1}}(1-x^{2})^{\frac{q-2}{2}}x^{m'}=\sum_{i=0}^{\nu-1}c_{i}(m')\left(\frac{q-2}{2}\right)_{\nu-1-i}(1-x^{2})^{\frac{q-2}{2}-(\nu-1)+i}x^{m'+(\nu-1)-2i}
\]


with $c_{i}$ having all the properties claimed in the proposition
statement. Now, using the equality 
\[
\int_{0}^{1}(1-y^{2})^{\alpha}y^{\beta}=\frac{\Gamma(\alpha+1)\Gamma\left(\frac{\beta+1}{2}\right)}{\Gamma\left(\frac{3}{2}+\alpha+\frac{\beta}{2}\right)}
\]


we have the latter integral being equal to

\[
\sum_{i=0}^{\nu-1}c_{i}(m')\left(\frac{q-2}{2}\right)_{\nu-1-i}\frac{\Gamma\left(\frac{\lambda+N-\nu}{2}+i\right)\Gamma\left(\frac{m'+n'+\nu-2i}{2}\right)}{\Gamma\left(\frac{\lambda+N+m'+n'}{2}\right)}
\]

\end{proof}

\begin{proof}
(of prop. \ref{prop:Assume-that-1}) In the light of proposition \ref{prop:The-following-holds}
we have 
\[
\left\langle K_{\lambda,\nu}^{C},F_{N,0,n',m',\psi,1}\right\rangle \simeq\sum_{i=0}^{\nu-1}c_{i}(m')\left(\frac{q-2}{2}\right)_{\nu-1-i}\frac{\Gamma\left(\frac{\lambda+N-\nu}{2}+i\right)\Gamma\left(\frac{m'+n'+\nu-2i}{2}\right)}{\Gamma\left(\frac{\lambda+N+m'+n'}{2}\right)}\cdot\frac{\Gamma\left(\lambda+\nu-n+1\right)}{\Gamma\left(\frac{\lambda+\nu-n-N+2}{2}\right)\Gamma\left(\frac{\lambda-q+\nu+N}{2}\right)}
\]


The factors $c_{i}(m')\Gamma\left(\frac{m'+n'+\nu-2i}{2}\right)$
cannot have poles due to the properties of $c_{i}$ (see prop. \ref{prop:The-following-holds}).
Therefore, it suffices to show that $\forall i$ with $0\leq i\leq\nu-1$
the expression

\[
\frac{\Gamma\left(\frac{\lambda+N-\nu}{2}+i\right)}{\Gamma\left(\frac{\lambda-\nu}{2}\right)}\cdot\frac{\Gamma\left(\lambda+\nu-n+1\right)}{\Gamma\left(\frac{\lambda+\nu-n-N+2}{2}\right)\Gamma\left(\frac{\lambda+\nu-n+1}{2}\right)}
\]

\end{proof}
is holomorphic as a function of $\lambda$. The first multiplicand
is obviously so. So is the second, once one recalls the Lagrange duplication
identity to write

\[
\frac{\Gamma\left(\lambda+\nu-n+1\right)}{\Gamma\left(\frac{\lambda+\nu-n-N+2}{2}\right)\Gamma\left(\frac{\lambda+\nu-n+1}{2}\right)}\simeq\frac{\Gamma\left(\frac{\lambda+\nu-n+2}{2}\right)}{\Gamma\left(\frac{\lambda+\nu-n-N+2}{2}\right)}
\]
\\
It remains therefore to show that $\tilde{K}_{\lambda,\nu}^{C}\ne0$.
Assume on contrary that for some fixed $(\lambda,\nu)$ it is not
so. Then we should have $\forall(N,m',n')\in\mathbb{Z}_{\ge0}^{3}$
such that $N,\; m'+n'\in2\mathbb{Z}$ we have

\[
\sum_{i=0}^{\nu-1}c_{i}(m')\left(\frac{q-2}{2}\right)_{\nu-1-i}\frac{\Gamma\left(\frac{\lambda+N-\nu}{2}+i\right)}{\Gamma\left(\frac{\lambda-\nu}{2}\right)}\cdot\frac{\Gamma\left(\frac{m'+n'+\nu-2i}{2}\right)}{\Gamma\left(\frac{\lambda+N+m'+n'}{2}\right)}\cdot\frac{1}{\Gamma\left(\frac{\lambda-q+\nu+N}{2}\right)}\cdot\frac{\Gamma\left(\frac{\lambda+\nu-n+2}{2}\right)}{\Gamma\left(\frac{\lambda+\nu-n-N+2}{2}\right)}=0
\]


We assume $N$ and $m'$ fixed for a moment and vary $n'$ only. Now,
as $\Gamma\left(\frac{\lambda+N+m'+n'}{2}\right)$ is finite for big
$n'$ we see that

\[
\sum_{i=0}^{\nu-1}c_{i}(m')\left(\frac{q-2}{2}\right)_{\nu-1-i}\frac{\Gamma\left(\frac{\lambda+N-\nu}{2}+i\right)}{\Gamma\left(\frac{\lambda-\nu}{2}\right)}\cdot\Gamma\left(\frac{m'+n'+\nu-2i}{2}\right)\cdot\frac{1}{\Gamma\left(\frac{\lambda-q+\nu+N}{2}\right)}\cdot\frac{\Gamma\left(\frac{\lambda+\nu-n+2}{2}\right)}{\Gamma\left(\frac{\lambda+\nu-n-N+2}{2}\right)}=0
\]


And then lemma \ref{lem:Suppose-there-exists} and the fact that $c_{i}\ne0$
imply that $\forall i$ with $0\leq i\leq\nu-1$ we have

\[
\left(\frac{q-2}{2}\right)_{\nu-1-i}\frac{\Gamma\left(\frac{\lambda+N-\nu}{2}+i\right)}{\Gamma\left(\frac{\lambda-\nu}{2}\right)}\cdot\frac{1}{\Gamma\left(\frac{\lambda-q+\nu+N}{2}\right)}\cdot\frac{\Gamma\left(\frac{\lambda+\nu-n+2}{2}\right)}{\Gamma\left(\frac{\lambda+\nu-n-N+2}{2}\right)}=0
\]


In particular, it holds for $i=0$ and thus

\[
\frac{\Gamma\left(\frac{\lambda+N-\nu}{2}\right)}{\Gamma\left(\frac{\lambda-\nu}{2}\right)}\cdot\frac{1}{\Gamma\left(\frac{\lambda-q+\nu+N}{2}\right)}\cdot\frac{\Gamma\left(\frac{\lambda+\nu-n+2}{2}\right)}{\Gamma\left(\frac{\lambda+\nu-n-N+2}{2}\right)}=0
\]


Then, if we let 
\[
A_{N}:=\left\{ \lambda\in\mathbb{C}\bigg|\frac{\Gamma\left(\frac{\lambda+N-\nu}{2}\right)}{\Gamma\left(\frac{\lambda-\nu}{2}\right)}=0\right\} =\left\{ \lambda=\nu-N+2j\right\} _{j=1}^{N/2}
\]


\[
B_{N}:=\left\{ \lambda\in\mathbb{C}\bigg|\frac{\Gamma\left(\frac{\lambda+\nu-n+2}{2}\right)}{\Gamma\left(\frac{\lambda+\nu-n-N+2}{2}\right)}=0\right\} =\left\{ \lambda=-\nu+n+2j\right\} _{j=0}^{N/2-1}
\]


\[
C_{N}:=\left\{ \lambda\in\mathbb{C}\bigg|\frac{1}{\Gamma\left(\frac{\lambda-q+\nu+N}{2}\right)}=0\right\} =\left\{ \lambda=-\nu+q-N-2j\right\} _{j=0}^{\infty}
\]


And $\lambda\in\bigcap_{N\in2\mathbb{Z}_{\ge0}}(A_{N}\cup B_{N}\cup C_{N})$
and as $C_{N}\subset-\nu+2\mathbb{Z}+1$, while $A_{N}\cup B_{N}\subset-\nu+2\mathbb{Z}$,
the first item of lemma \ref{lem:Suppose-that-sets} and the fact
that $B_{0}=A_{0}=\emptyset\implies\bigcap_{N\in2\mathbb{Z}_{\ge0}}(A_{N}\cup B_{N})=\emptyset$
imply that $\lambda\in\bigcap_{N\in2\mathbb{Z}_{\ge0}}C_{N}=\emptyset$.
This contradiction ends the proof.


\begin{proof}
(of prop. \ref{prop:Assume-that-2}) Let's start with proving the
holomorphicity. Similarly to the proof of prop. \ref{prop:Assume-that-1},
it suffices to show that $\forall i$ with $0\leq i\leq\nu-1$ and
$\forall N\in2\mathbb{Z}_{\ge0}$ the expression

\[
\frac{\Gamma\left(\frac{\lambda+N-\nu}{2}+i\right)}{\Gamma\left(\frac{\lambda+\nu-q+N}{2}\right)}\cdot\frac{\Gamma\left(\lambda+\nu-n+1\right)}{\Gamma\left(\frac{\lambda+\nu-n-N+2}{2}\right)\Gamma\left(\frac{\lambda+\nu-n+1}{2}\right)}
\]


is holomorphic as a function of $\lambda$. The second multiplicand
is obviously so, while the first is also so, as $-\nu+i\ge\nu-q$,
as implied by $q-2\nu\ge0$ hypothesis. Thus, it remains to show that
$\tilde{K}_{\lambda,\nu}^{C}\neq0$, so we on contrary assume that
for some $(\lambda,\nu)$ it is not so. Then, as in proof of prop.
\ref{prop:Assume-that-1}, this implies that $\forall N\in2\mathbb{Z}_{\ge0}$
we have
\[
\frac{\Gamma\left(\frac{\lambda+N-\nu}{2}\right)}{\Gamma\left(\frac{\lambda+\nu-q+N}{2}\right)}\cdot\frac{\Gamma\left(\lambda+\nu-n+1\right)}{\Gamma\left(\frac{\lambda+\nu-n-N+2}{2}\right)\Gamma\left(\frac{\lambda+\nu-n+1}{2}\right)}=0.
\]


Hence if we set
\[
A_{N}:=\left\{ \lambda\in\mathbb{C}\bigg|\frac{\Gamma\left(\frac{\lambda+N-\nu}{2}\right)}{\Gamma\left(\frac{\lambda+\nu-q+N}{2}\right)}\right\} =\left\{ \lambda=-N+\nu+2j\right\} _{j=1}^{q/2-\nu}
\]


\[
B_{N}:=\left\{ \lambda\in\mathbb{C}\bigg|\frac{\Gamma\left(\frac{\lambda+\nu-n+2}{2}\right)}{\Gamma\left(\frac{\lambda+\nu-n-N+2}{2}\right)}=0\right\} =\left\{ \lambda=-\nu+n+2j\right\} _{j=0}^{N/2-1}
\]


we have to conclude that $\lambda\in\bigcap_{N\in2\mathbb{Z}_{\ge0}}(A_{N}\cup B_{N})$.
Now, as $-N+\nu+q-2\nu<-\nu+n\iff q<n$, we have that $\forall N,N',\; A_{N'}\cap B_{N}=\emptyset$
and the first item of lemma \ref{lem:Suppose-that-sets} together
with the fact that $\bigcap_{N\in2\mathbb{Z}_{\ge0}}B_{N}\subset B_{0}=\emptyset$,
imply that $\lambda\in\bigcap_{N\in2\mathbb{Z}_{\ge0}}A_{N}=\emptyset$.
\end{proof}

\begin{proof}
(of prop. \ref{prop:Assume-that-3}) Let's start with proving the
holomorphicity. Similarly to the proof of prop. \ref{prop:Assume-that-1},
it suffices to show that the following expression is holomorphic as
a function of $\lambda$:
\[
\frac{\Gamma\left(\frac{\lambda+N-\nu}{2}+\frac{q-2}{2}\right)}{\Gamma\left(\frac{\lambda+\nu-q+N}{2}\right)}\cdot\frac{\Gamma\left(\lambda+\nu-n+1\right)}{\Gamma\left(\frac{\lambda+\nu-n-N+2}{2}\right)\Gamma\left(\frac{\lambda+\nu-n+1}{2}\right)}
\]


Now, the second multiplicand is clearly holomorphic, while the first
is also so, as $-\nu+q-2\ge-q+\nu\iff q-1\ge\nu$. Thus, it remains
to show that $\tilde{K}_{\lambda,\nu}^{C}\ne0$ and as in the proof
of prop. \ref{prop:Assume-that-1}, the contradicting assumption implies
that $\forall N\in2\mathbb{Z}_{\ge0}$ we have

\[
\frac{\Gamma\left(\frac{\lambda+N-\nu}{2}+\frac{q-2}{2}\right)}{\Gamma\left(\frac{\lambda+\nu-q+N}{2}\right)}\cdot\frac{\Gamma\left(\lambda+\nu-n+1\right)}{\Gamma\left(\frac{\lambda+\nu-n-N+2}{2}\right)\Gamma\left(\frac{\lambda+\nu-n+1}{2}\right)}=0
\]


and thus if we let 
\[
A_{N}:=\left\{ \lambda\in\mathbb{C}\bigg|\frac{\Gamma\left(\frac{\lambda+N-\nu}{2}+\frac{q-2}{2}\right)}{\Gamma\left(\frac{\lambda+\nu-q+N}{2}\right)}=0\right\} =\left\{ \lambda=q-\nu-N-2j\right\} _{j=0}^{q-\nu-1}
\]


\[
B_{N}:=\left\{ \lambda\in\mathbb{C}\bigg|\frac{\Gamma\left(\frac{\lambda+\nu-n+2}{2}\right)}{\Gamma\left(\frac{\lambda+\nu-n-N+2}{2}\right)}\right\} =\left\{ \lambda=-\nu+n+2j\right\} _{j=0}^{N/2-1}
\]


we have to conclude that $\lambda\in\bigcap_{N\in2\mathbb{Z}_{\ge0}}(A_{N}\cup B_{N})$.
Now, as $q-\nu-N<-\nu+n\iff q-N<n$, we have that $\forall N,N',\; A_{N'}\cap B_{N}=\emptyset$
and the first item of lemma \ref{lem:Suppose-that-sets} together
with the fact that $\bigcap_{N\in2\mathbb{Z}_{\ge0}}B_{N}\subset B_{0}=\emptyset$,
imply that $\lambda\in\bigcap_{N\in2\mathbb{Z}_{\ge0}}A_{N}=\emptyset$.
\end{proof}

\begin{proof}
(of prop. \ref{prop:Assume-that-4}) Let's start with proving the
holomorphicity. Similarly to the proof of prop. \ref{prop:Assume-that-1},
it suffices to show that the following expression is holomorphic as
a function of $\lambda$:
\[
\frac{\Gamma\left(\frac{\lambda+N-\nu}{2}+\frac{q-2}{2}\right)}{\Gamma\left(\frac{\lambda-\nu+q-2}{2}\right)}\cdot\frac{\Gamma\left(\lambda+\nu-n+1\right)}{\Gamma\left(\frac{\lambda+\nu-n-N+2}{2}\right)\Gamma\left(\frac{\lambda+\nu-n+1}{2}\right)}
\]


Both multilpicands are clearly holomorphic. Thus, it remains to show
that $\tilde{K}_{\lambda,\nu}^{C}\ne0$ and as in the proof of prop.
\ref{prop:Assume-that-1}, the contradicting assumption implies that
$\forall N\in2\mathbb{Z}_{\ge0}$ we have

\[
\frac{\Gamma\left(\frac{\lambda+N-\nu}{2}+\frac{q-2}{2}\right)}{\Gamma\left(\frac{\lambda-\nu+q-2}{2}\right)}\dot{\frac{1}{\Gamma\left(\frac{\lambda+\nu-q+N}{2}\right)}}\cdot\frac{\Gamma\left(\lambda+\nu-n+1\right)}{\Gamma\left(\frac{\lambda+\nu-n-N+2}{2}\right)\Gamma\left(\frac{\lambda+\nu-n+1}{2}\right)}=0
\]


and thus if we let 
\[
C_{N}:=\left\{ \lambda\in\mathbb{C}\bigg|\frac{\Gamma\left(\frac{\lambda+N-\nu}{2}+\frac{q-2}{2}\right)}{\Gamma\left(\frac{\lambda-\nu+q-2}{2}\right)}=0\right\} =\left\{ \lambda=(\nu-q+2)-2j\right\} _{j=0}^{N/2-1}
\]


\[
B_{N}:=\left\{ \lambda\in\mathbb{C}\bigg|\frac{\Gamma\left(\frac{\lambda+\nu-n+2}{2}\right)}{\Gamma\left(\frac{\lambda+\nu-n-N+2}{2}\right)}\right\} =\left\{ \lambda=-\nu+n+2j\right\} _{j=0}^{N/2-1}
\]
\[
A_{N}:=\left\{ \lambda\in\mathbb{C}\bigg|\dot{\frac{1}{\Gamma\left(\frac{\lambda+\nu-q+N}{2}\right)}}=0\right\} =\left\{ \lambda=-\nu+q-N-2j\right\} _{j=0}^{\infty}
\]


we have to conclude that $\lambda\in\bigcap_{N\in2\mathbb{Z}_{\ge0}}(A_{N}\cup B_{N}\cup C_{N})$.
Now, as $-\nu+q-N<-\nu+n\iff q-N<n$, we have that $\forall N,N',\; A_{N'}\cap B_{N}=\emptyset$
and the second item of lemma \ref{lem:Suppose-that-sets} (with $N_{0}=0$)
implies that $\lambda\in\bigcap_{N\in2\mathbb{Z}_{\ge0}}(A_{N}\cup C_{N})$.

As $\lambda\in A_{0}\cup C_{0}=A_{0}$, we can write $\lambda=-\nu+q-N'$
with $N'\in2\mathbb{Z}_{\ge0}$. Now, it should be that $\lambda\in A_{N'+2}\cup C_{N'+2}$
and hence $\lambda\in C_{N'+2}$ and thus $\lambda=-\nu+q-N'\geq\nu-q+N'+6$,
thus $-N'-3\ge\nu-q\ge0$ (the rightmost inequality is implied by
the hypothesis). The obtained contradiction ends the proof.
\end{proof}

\begin{proof}
(of \lemref{Suppose-there-exists}) Recall the formula $\Gamma(t+n)=t^{(n)}\cdot\Gamma(t)$
(here $t^{(n)}:=t(t+1)\dots(t+n-1)$ denotes the rising factorial).
It implies that $0=\sum_{i=0}^{m}c_{i}\Gamma(\alpha+i+k_{n})=\Gamma(\alpha+k_{n})\sum_{i=0}^{m}c_{i}(\alpha+k_{n})^{(i)}$.
As $\Gamma(\alpha+z)\neq0$ for, this implies that $\forall n>N,\;\sum_{i=0}^{m}c_{i}(\alpha+k_{n})^{(i)}=0$
and since the latter expression is a polynomial in $k_{n}$, it implies
that $\forall x\in\mathbb{C},\;\sum_{i=0}^{m}c_{i}x^{(n)}=0$ and
hence that all $c_{i}$ should vanish.
\end{proof}

\begin{proof}
(of lemma \ref{lem:For-,-we}) The last statement is obvious, as for
$g\in O(p+1)\times O(q+1)\subset O(p+1,q+1)$ and $x\in\Xi^{p+1,q+1}$we
have $\left|g\cdot x\right|=\left|x\right|$. Apart from this, there
are several things to be checked:\end{proof}
\begin{itemize}
\item It is true, that $\left(\pi_{\lambda,K},C^{\infty}(\mathbb{S}^{p}\times\mathbb{S}^{q})^{\mathbb{Z}_{2}}\right)$
and $\left(l_{\Xi},C_{-\lambda}^{\infty}(\Xi^{p+1,q+1})\right)$ are
indeed $G$-representations. That's straightforward;
\item It is true that for $f\in C^{\infty}(\mathbb{S}^{p}\times\mathbb{S}^{q})^{\mathbb{Z}_{2}}$
function $\widetilde{f}(x,y)$ on $\Xi^{p+1,q+1}$ defined as $\widetilde{f}(x,y):=\left|x\right|^{-\lambda}f\left[\frac{(x,y)}{\left|x\right|}\right]$
is in $C_{-\lambda}^{\infty}(\Xi^{p+1,q+1})$. Indeed, $\widetilde{f}(t\cdot(x,y))=\left|t\right|^{-\lambda}\left|x\right|^{-\lambda}f\left[\frac{t}{\left|t\right|}\cdot\frac{(x,y)}{\left|x\right|}\right]$.
Note that $t/\left|t\right|=\pm1$ and as $f\in C^{\infty}(\mathbb{S}^{p}\times\mathbb{S}^{q})^{\mathbb{Z}_{2}}$,
we have $\widetilde{f}(t\cdot(x,y))=\left|t\right|^{-\lambda}\left|x\right|^{-\lambda}f\left[\frac{t}{\left|t\right|}\cdot\frac{(x,y)}{\left|x\right|}\right]=\left|t\right|^{-\lambda}\left|x\right|^{-\lambda}f\left[\frac{(x,y)}{\left|x\right|}\right]=\left|t\right|^{-\lambda}\widetilde{f}(x,y)$;
\item It is true that map $f\mapsto\widetilde{f}$ is onto $C_{-\lambda}^{\infty}(\Xi^{p+1,q+1})$
is $G$-covariant. That's also straightforward;
\item It is true that map $f\mapsto\widetilde{f}$ is onto $C_{-\lambda}^{\infty}(\Xi^{p+1,q+1})$
has an inverse. Indeed, given $\widetilde{f}\in C_{-\lambda}^{\infty}(\Xi^{p+1,q+1})$
define $f:=\widetilde{f}\big|_{\mathbb{S}^{p}\times\mathbb{S}^{q}}$.
It is an element of $C(\mathbb{S}^{p}\times\mathbb{S}^{q})^{\mathbb{Z}_{2}}$
and straightforward computations show that $\widetilde{f}\mapsto f$
forms a desired inverse.
\end{itemize}

\begin{proof}
(of lemma \ref{lem:Let--denote}) Recall that the $K$-finite vectors
of $G\supset K=O(p+1)\times Q(q+1)$ acting on $C^{\infty}(\mathbb{S}^{p}\times\mathbb{S}^{q})$
are given as $\sum_{k,l=0}^{\infty}\mathcal{H}^{k}(\mathbb{S}^{p})\otimes\mathcal{H}^{l}(\mathbb{S}^{q})$
(where $\sum$ denotes span), as explained in \cite{howe1993homogeneous}.
As $C(\mathbb{S}^{p}\times\mathbb{S}^{q})^{\mathbb{Z}_{2}}$ is a
subrepresentation of $C(\mathbb{S}^{p}\times\mathbb{S}^{q})$, $K$-finite
vectors of $C(\mathbb{S}^{p}\times\mathbb{S}^{q})^{\mathbb{Z}_{2}}$
are precisely the $K$-finite vectors of $C(\mathbb{S}^{p}\times\mathbb{S}^{q})$
that belong to $C(\mathbb{S}^{p}\times\mathbb{S}^{q})^{\mathbb{Z}_{2}}$.
This proves the first item.

For the second item, recall the $O(n)\downarrow O(n-1)$ branching
rule mentioned in \cite{kobayashi2015symmetry}, namely that
\[
\mathcal{H}^{L}(\mathbb{S}^{n})\simeq\bigoplus_{N=0}^{L}\mathcal{H}^{N}(\mathbb{S}^{n-1})
\]


constructed explicitly with $I_{N}:\mathcal{H}^{N}(\mathbb{S}^{n-1})\to\mathcal{H}^{L}(\mathbb{S}^{n})$
given by
\[
(I_{N}\psi)(\eta_{0},\eta):=\left|\eta\right|^{-N}\psi\left(\frac{\eta}{\left|\eta\right|}\right)\tilde{\tilde{C}}_{L-N}^{\frac{n-1}{2}+N}(\eta_{0}).
\]


Hence, we see that 
\[
S=\sum_{\left\{ (L,K)\in\mathbb{Z}_{\ge0}^{2}\big|K+L\in2\mathbb{Z}_{\ge0}\right\} }\mathcal{H}^{L}(\mathbb{S}^{p})\otimes\mathcal{H}^{K}(\mathbb{S}^{q})=
\]


\[
=\sum_{\left\{ (L,K)\in\mathbb{Z}_{\ge0}^{2}\big|K+L\in2\mathbb{Z}_{\ge0}\right\} }\sum_{N,M=0}^{L,K}\left|\eta\right|^{N}\mathcal{H}^{N}\left(\frac{\eta}{\left|\eta\right|}\right)\tilde{\tilde{C}}_{L-N}^{\frac{p-1}{2}+N}(\eta_{0})\left|\xi\right|^{M}\mathcal{H}^{M}\left(\frac{\xi}{\left|\xi\right|}\right)\tilde{\tilde{C}}_{K-M}^{\frac{q-1}{2}+M}(\xi_{0})=
\]


\[
=\sum_{N,M=0}^{\infty}\sum_{\left\{ n',m'\ge0\big|N+M+n'+m'\in2\mathbb{Z}_{\ge0}\right\} }\left|\eta\right|^{N}\mathcal{H}^{N}\left(\frac{\eta}{\left|\eta\right|}\right)\tilde{\tilde{C}}_{n'}^{\frac{p-1}{2}+N}(\eta_{0})\left|\xi\right|^{M}\mathcal{H}^{M}\left(\frac{\xi}{\left|\xi\right|}\right)\tilde{\tilde{C}}_{m'}^{\frac{q-1}{2}+M}(\xi_{0})=
\]


\[
=\sum_{N,M=0}^{\infty}\sum_{\left\{ n',m'\ge0\big|N+M+n'+m'\in2\mathbb{Z}_{\ge0}\right\} }\left|\eta\right|^{N}\mathcal{H}^{N}\left(\frac{\eta}{\left|\eta\right|}\right)\eta_{0}^{n'}\left|\xi\right|^{M}\mathcal{H}^{M}\left(\frac{\xi}{\left|\xi\right|}\right)\xi_{0}^{m'}.
\]


The last equality holds true because $\tilde{\tilde{C_{m}^{\nu}}}$
are polynomials of degree $m$ whose highest term does not vanish,
unless $\nu\in\mathbb{Z}_{<0}$.

\bibliographystyle{plain}
\bibliography{/home/u406/for/forlatex/todai_master}
\end{proof}

\end{document}
