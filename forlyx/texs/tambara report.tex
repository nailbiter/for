%% LyX 2.0.8.1 created this file.  For more info, see http://www.lyx.org/.
%% Do not edit unless you really know what you are doing.
\documentclass[english]{beamer}
\usepackage[T1]{fontenc}
\usepackage[latin9]{inputenc}
\setcounter{secnumdepth}{3}
\setcounter{tocdepth}{3}
\usepackage{amssymb}
\usepackage{esint}
\PassOptionsToPackage{normalem}{ulem}
\usepackage{ulem}
\usepackage{nameref}

\makeatletter
%%%%%%%%%%%%%%%%%%%%%%%%%%%%%% Textclass specific LaTeX commands.
 % this default might be overridden by plain title style
 \newcommand\makebeamertitle{\frame{\maketitle}}%
 \AtBeginDocument{
   \let\origtableofcontents=\tableofcontents
   \def\tableofcontents{\@ifnextchar[{\origtableofcontents}{\gobbletableofcontents}}
   \def\gobbletableofcontents#1{\origtableofcontents}
 }
 \long\def\lyxframe#1{\@lyxframe#1\@lyxframestop}%
 \def\@lyxframe{\@ifnextchar<{\@@lyxframe}{\@@lyxframe<*>}}%
 \def\@@lyxframe<#1>{\@ifnextchar[{\@@@lyxframe<#1>}{\@@@lyxframe<#1>[]}}
 \def\@@@lyxframe<#1>[{\@ifnextchar<{\@@@@@lyxframe<#1>[}{\@@@@lyxframe<#1>[<*>][}}
 \def\@@@@@lyxframe<#1>[#2]{\@ifnextchar[{\@@@@lyxframe<#1>[#2]}{\@@@@lyxframe<#1>[#2][]}}
 \long\def\@@@@lyxframe<#1>[#2][#3]#4\@lyxframestop#5\lyxframeend{%
   \frame<#1>[#2][#3]{\frametitle{#4}#5}}
 \def\lyxframeend{} % In case there is a superfluous frame end

%%%%%%%%%%%%%%%%%%%%%%%%%%%%%% User specified LaTeX commands.
\newcommand{\R}{\mathbb{R}}
\setlength{\parindent}{0.5cm} % Default is 15pt.
\usepackage{amsmath}

\makeatother

\usepackage{babel}
\begin{document}

\lyxframeend{}\lyxframe{Goal}
\begin{theorem}%{}
[subrep thm]\label{thm:-Let-} Let $G$ be a \uline{connected}\emph{
}reductive group with $K$ -- maximal compact subgroup. Let $V$ be
an irreducible $(\mathfrak{g},K)$-module. Then, $V$ is isomorphic
to a submodule of quotient of $(H^{\sigma,\mu})_{K}$ for an appropriate
$\sigma\in\widehat{^{0}M}$%
\footnote{~~~$^{0}M$ denotes the $Z_{K}(\mathfrak{a})$%
} and $\mu\in\mathfrak{a}_{\mathbb{C}}^{*}$
\end{theorem}%{}


\uline{Remark.} In fact, the statement holds for disconnected groups
as well. The proof is just a bit more technical.


\lyxframeend{}


\lyxframeend{}\lyxframe{Definitions}
\begin{definition}%{}
[$(\mathfrak{g},K)$-module] Let $G$ be reductive subgroup with $K$
maximal compact.$(\mathfrak{g},K)$-module is a vector space $V$
which is $K$-module and $\mathfrak{g}$-module at the same time and
such that:\end{definition}%{}
\begin{enumerate}
\item $\forall(X,k,v)\in\mathfrak{g}\times K\times V$ we have $k\cdot X\cdot v=(Ad(k)X)\cdot k\cdot v$
\item $\forall v\in V$ set $K\cdot v$ span finitely-dimensional subspace
$W_{v}$ of $V$ and $K\curvearrowright W_{v}$ is continuous
\item $\forall(Y,v)\in\mathfrak{k}\times V$ we have $\lim_{t\to0}(e^{tY}\cdot v)=Y\cdot v$.
\end{enumerate}

\lyxframeend{}


\lyxframeend{}\lyxframe{Definitions}
\begin{definition}%{}
[$\mbox{Ind}_P^G(\sigma)$]($(\mathfrak{g},K)$-module) Let $G$ be
reductive group, with $P$ closed subgroup with property $PK=G$.
Given the Hilbert representation $(\sigma,H)$ of $P$ we let We define
subspace $(H^{\sigma})_{0}$ of $G\to H$ continuous functions $f$
that satisfy $\forall(p,g)\in P\times G,\; f(pg)=\delta_{P}(p)^{1/2}\sigma(p)f(g)$.%
\footnote{$\delta_{P}$ denotes the modular function of $P$%
} Finally, we define representation $H^{\sigma}$ as representation
of $G$ on the completion of $(H^{\sigma,\mu})_{0}$ with respect
to the norm $\left\langle f,g\right\rangle :=\int_{K}f\overline{g}$
and action $g\cdot f:=f(\cdot\ g)$. It will be denoted as $\mbox{Ind}_{P}^{G}(\sigma)$
and called \uline{induced representation}.
\end{definition}%{}

\begin{definition}%{}
[$H^{\sigma,\mu}$]$(\sigma,H)\in\widehat{^{0}M}$ and $\mu\in\mathfrak{a}_{\mathbb{C}}^{*}$.
We define the representation $\sigma_{\mu}$ of minimal parabolic
$P=MAN$ on $H$ by $\sigma_{\mu}(man):=e^{(\mu+\rho)(\log a)}\sigma(m)$%
\footnote{$\rho$ denotes half-sum of positive restricted roots with multiplicities%
}. The we will write $H^{\sigma,\mu}$ for $\mbox{Ind}_{P}^{G}(\sigma_{\mu})$
and call these \uline{principal series}.
\end{definition}%{}

\lyxframeend{}


\lyxframeend{}\lyxframe{Properties}
\begin{theorem}%{}
[$H^{\sigma,\mu}$] Let $G$ be reductive group, $K$ its maximal
compact subgroup\end{theorem}%{}
\begin{enumerate}
\item $\mbox{Ind}_{P}^{G}(\sigma)$ is a Hilbert representation, which is
unitary if $\sigma$ was so
\item For $H$ being a Hilbert rep of $G$, then $H_{K}:=\sum_{\gamma\in\widehat{K}}^{\oplus}H(\gamma)\cap H^{\infty}$
forms a $(\mathfrak{g},K)$-module, where $H(\gamma):=\overline{\sum_{W\subset H,W\in\widehat{K}}W}$%
\footnote{given Hilbert rep of a cpt gp, the inner product can be altered in
such a way that norm goes to equivalent one, while rep becomes unitary,
hence we may consider all Hilbert $G$-reps being $K$-unitary%
}
\end{enumerate}


\uline{Remark.} Hence, we may talk about $(H^{\sigma,\mu})_{K}$.


\lyxframeend{}


\lyxframeend{}\lyxframe{proof of {[}\nameref{thm:-Let-}{]}}

So let $V$ an irreducible $(\mathfrak{g},K)$-module be given. Let's
pick $\gamma\in\widehat{K}$ such that $Hom_{K}(\gamma,V)\neq0$.
We have the following
\begin{theorem}%{}
[$\mbox{Hom}_K(\gamma,V)$]\label{thm:Assume--is}Assume $V$ is an
irreducible $\left(\mathfrak{g},K\right)$-module and $\gamma\in\widehat{K}$
such that $\mbox{Ho\ensuremath{m_{K}}}(\gamma,V)\neq0$. Then\end{theorem}%{}
\begin{enumerate}
\item $\mbox{Hom}_{K}(\gamma,V)$ is an irreducible $U(\mathfrak{g})^{K}$-module
\item If $V'$ is another $\left(\mathfrak{g},K\right)$ module and $Y_{1}\subset Y_{2}\subset\mbox{Hom}_{K}(\gamma,V')$
are $U(\mathfrak{g})^{K}$ submodules such that $Y_{2}/Y_{1}\simeq\mbox{Hom}_{K}(\gamma,V)$,
then there are $Z_{1}\subset Z_{2}\subset V'$ $\left(\mathfrak{g},K\right)$-submodules
such that $V\simeq Z_{2}/Z_{1}$ (hence $V$ is isomorphic to submodule
of a quotient of $V'$)
\end{enumerate}
So we will denote the irreducible $U(\mathfrak{g})^{K}$-module $\mbox{Hom}_{K}(\gamma,V)$
by $\Omega$.


\lyxframeend{}


\lyxframeend{}\lyxframe{proof of {[}\nameref{thm:-Let-}{]}}
\begin{theorem}%{}
[$(\mathfrak{g},K)$-module] Let $V$ be a $\left(\mathfrak{g,}K\right)$-module.\end{theorem}%{}
\begin{enumerate}
\item $V=\sum_{\gamma\in\widehat{K}}^{\oplus}V(\gamma)$ where $V(\gamma)$
denotes the sum of all $K$-submodules of $V$ in class $\gamma$.
\item If $V$ is irreducible, it is admissible\end{enumerate}
\begin{definition}%{}
$\left(\mathfrak{g},K\right)$-module $V$ is called admissible if
$\forall\gamma\in\widehat{K}$ $V(\gamma)$ is finitely-dimensional.
\end{definition}%{}


Hence we see that $\mbox{Hom}_{K}(\gamma,V)=\Omega$ is not only irreducible,
but also finitely dimensional.


\lyxframeend{}


\lyxframeend{}\lyxframe{proof of {[}\nameref{thm:-Let-}{]}}

Given $H^{\sigma,\mu}$ principal series, we shall denote $U(\mathfrak{g})^{K}$
module $\mbox{Hom}_{K}(\gamma,(H^{\sigma,\mu})_{K})$ by $\mathcal{P}_{\sigma,\mu}$.
We also have the following
\begin{theorem}%{}
[$S$]\label{thm:S} Let $\Omega$ be such as defined above. Then
there does exist a finite set $S=\left\{ (\sigma_{i},\mu_{i})\right\} _{i}\subset\widehat{^{0}M}\times\mathfrak{a}_{\mathbb{C}}^{*}$
and $m\in\mathbb{Z}_{>0}$ such that $\mbox{Ann}\Omega\supset\left(\bigcap_{i}\mbox{Ann}\mathcal{P}_{\sigma,\mu}\right)^{m}$.
\end{theorem}%{}

\begin{theorem}%{}
[alg fact]\label{thm:We-have-the}We have the following\end{theorem}%{}
\begin{enumerate}
\item Let $A$ be associative left Noetherian algebra, $W$, $V$ and $X$
finite dimensional representations and $V$ is irreducible. Then if
$\mbox{Ann}V\supset\mbox{Ann}W\cdot\mbox{Ann}X$ then $\mbox{Ann}V$
should contain the annihilator of either $W$ or $X$.
\item Let $A$ be an associative algebra $\left\{ \sigma_{i}\right\} _{i}$
finite number of finite dimensional $A$-representations, $\sigma=\oplus_{i}\sigma_{i}$
and $\pi$ irrep of $A$. Then if $\mbox{Ann}\pi\supset\mbox{Ann}\sigma$,
there exists $i$ such that $\mbox{Ann}\pi\supset\mbox{Ann}\sigma_{i}$.
\end{enumerate}

\lyxframeend{}


\lyxframeend{}\lyxframe{proof of {[}\nameref{thm:-Let-}{]}}

As $U(\mathfrak{g})^{K}$ is Noetherian%
\footnote{since $U(\mathfrak{g})^{K}$ is so and then by symmetrization.%
}, these two theorems imply that $\mbox{Ann}\Omega\supset\mbox{Ann}\mathcal{P}_{\sigma,\mu}$
for some $(\sigma,\mu)\in\widehat{^{0}M}\times\mathfrak{a_{\mathbb{C}}^{*}}$.
Now, it can be shown that
\begin{theorem}%{}
[$H^{\sigma,\mu}$]\end{theorem}%{}
\begin{enumerate}
\item [3.]For every $\gamma\in\widehat{K}$ and principal series $H^{\sigma,\mu}$
we have $\mbox{Hom}_{K}\left(\gamma,(H^{\sigma,\mu})_{K}\right)$
being finitely dimensional
\end{enumerate}
\textrm{$\mathcal{P}_{\sigma,\mu}$ is finitely dimensional $U(\mathfrak{g})^{K}$
module, hence it is both Artinian and Noetherian, hence it has finite
composition series}

\[
\mathcal{P}_{\sigma,\mu}=H_{0}\supset H_{1}\supset\dots\supset H_{d}=0
\]


such that $H_{i}/H_{i+1}$ is irreducible $U(\mathfrak{g})^{K}$ module.


\lyxframeend{}


\lyxframeend{}\lyxframe{proof of {[}\nameref{thm:-Let-}{]}}

It is directly seen that as $\mbox{Ann}\Omega\supset\mbox{Ann}\mathcal{P}_{\sigma,\mu}$,
we have $\mbox{Ann}\Omega\supset\left(\bigcap_{i}\mbox{Ann}(H_{i}/H_{i+1})\right)^{d}$
and hence by applying theorem {[}\nameref{thm:We-have-the}{]} again
we see that for some $i$ $\mbox{Ann}\Omega\supset\mbox{Ann}(H_{i}/H_{i+1})$,
but as both modules are irreducible we have $\Omega\simeq H_{i}/H_{i+1}$
and hence by applying theorem {[}\nameref{thm:Assume--is}{]} we see
that the conclusion of theorem {[}\nameref{thm:-Let-}{]} follows.

We shall proceed by proving {[}\nameref{thm:S}{]}.


\lyxframeend{}


\lyxframeend{}\lyxframe{proof of {[}\nameref{thm:S}{]}}

We shall fix $\gamma\in\widehat{K}$ and$\mbox{Hom}_{K}(\gamma,V)=\mbox{Hom}_{K}(\gamma,V(\gamma))=:\Omega$
as above for irreducible $\left(\mathfrak{g},K\right)$-module $V$.
Now for every \textbf{$z\in Z(\mathfrak{g})$ }Schur's lemma implies
that $\Omega(z)=\chi(z)I$ and hence by Harish-Chandra theorem $\chi=\chi_{\Lambda}$
for some $\Lambda\in\mathfrak{h}_{\mathbb{C}}^{*}$. We set $S:=\{(\sigma,\mu)\in\widehat{^{0}M}\times\mathfrak{a}_{\mathbb{C}}^{*}\;\big|\;\mbox{Hom}_{^{0}M}(\sigma,\gamma)\ne0,\mu\in W(\mathfrak{g},\mathfrak{h})\Lambda\}$
-- it is clearly finite, and $m:=\left|W(\mathfrak{g},\mathfrak{h})\right|\cdot\mbox{dim}(\gamma)$.
\begin{definitions}%{}
We fix $\sigma\in\widehat{^{0}M}$ and $\mu\in\mathfrak{a}_{\mathbb{C}}^{*}$\end{definitions}%{}
\begin{enumerate}
\item We let $\beta_{\gamma}:U(\mathfrak{k})\to\mbox{End}_{\mathbb{C}}(\gamma)$
be induced by action $K\curvearrowright\gamma$.
\item We set $P_{\sigma}:\gamma\to\gamma(\sigma)$ to denote a projection
\item We let $p:U(\mathfrak{g})\to U(\mathfrak{a})\otimes U(\mathfrak{k})$
be projection induced by $U(\mathfrak{g})=U(\mathfrak{a})\otimes U(\mathfrak{k})\oplus\mathfrak{n}U(\mathfrak{g})$.
\end{enumerate}
\uline{Remark}. We put product structure on $U(\mathfrak{a})\otimes U(\mathfrak{k})$
and then one sees that $p(xy)=p(y)p(x)$ whenever $y\in U(\mathfrak{g})^{K}$.


\lyxframeend{}


\lyxframeend{}\lyxframe{proof of {[}\nameref{thm:S}{]}}
\begin{definitions}%{}
\end{definitions}%{}
\begin{enumerate}
\item $p_{\gamma}=(I\otimes\beta_{\gamma})p\,:U(\mathfrak{g})\to U(\mathfrak{a})\otimes\mbox{End}_{\mathbb{C}}(\gamma)$
\item $p_{\gamma,\sigma}:(I\otimes P_{\sigma}\beta_{\gamma}P_{\sigma})p\,:U(\mathfrak{g})\to U(\mathfrak{a})\otimes\mbox{End}_{\mathbb{C}}(\gamma(\sigma))$
\item $p_{\gamma,\sigma,\mu}:((\mu+\rho)\otimes P_{\sigma}\beta_{\gamma}P_{\sigma})p\,:U(\mathfrak{g})\to\mbox{End}_{\mathbb{C}}(\gamma(\sigma))$
\end{enumerate}
We recall the Harish-Chandra theory, that tells us that if we define
$q:U(\mathfrak{g})\to U(\mathfrak{h})$ be projection corresponding
to direct sum decomposition

$U(\mathfrak{g})=U(\mathfrak{h})\oplus(\mathfrak{n}U(\mathfrak{g})+U(\mathfrak{g})\mathfrak{n_{-}})$
and $\nu:U(\mathfrak{h})\to U(\mathfrak{h})$ be the extension of
$H\mapsto H+\delta(H)$ ($\delta$ denoting the half sum of positive
roots), then

$\nu\circ q:Z(\mathfrak{g})\to U(\mathfrak{h})^{W}$ is an isomorphism.


\lyxframeend{}


\lyxframeend{}\lyxframe{proof of {[}\nameref{thm:S}{]}}

We fix $u\in\bigcap_{(\sigma,\mu)\in S}\mbox{Ann}\mathcal{P}_{\sigma,\mu}\subset U(\mathfrak{g})^{K}$
and our goal is to show that $\Omega^{m}(u)=0$.
\begin{definitions}%{}
\end{definitions}%{}
\begin{enumerate}
\item $f_{\gamma,u}(T):=\Pi_{\sigma\in\widehat{^{0}M}}\mbox{det}(T\cdot I-p_{\gamma,\sigma}(u))\in U(\mathfrak{a})[T]\subset U(\mathfrak{h})[T]$
is polynomial of degree $\mbox{dim}(\gamma)$
\item $g_{\gamma,u}(T):=\nu^{-1}\Pi_{w\in W}wf_{\gamma,u}(T)\in U(\mathfrak{h})[T]$
is polynomial of degree $m$ and $\sum q(z_{j})T^{j}:=g_{\gamma,u}(T)$
for $q_{j}\in Z(\mathfrak{g})$ and $q_{m+1}=1$.
\end{enumerate}

\lyxframeend{}


\lyxframeend{}\lyxframe{proof of {[}\nameref{thm:S}{]}}

We have the following
\begin{theorem}%{}
[$v\in U(\mathfrak{g})\mbox{Ker}(\beta_{\gamma})$]\label{thm:For--we}For
$v:=\sum_{i}z_{i}u^{i}$ we have $v(u)\in U(\mathfrak{g})^{K}\cap U(\mathfrak{g})\mbox{Ker}(\beta_{\gamma})$.
\end{theorem}%{}


In the light of what $\Omega$ is, $\Omega(v)=0$. To get the desired%
\footnote{recall that $p(xy)=p(y)p(x)$ whenever $y\in U(\mathfrak{g})^{K}$%
} it suffices to show that for any $j\leq m$ $\Omega(z_{j})=\chi_{\Lambda}(z_{j})I=0$.
But indeed, 
\[
\chi_{\Lambda}(g_{\gamma,u})=\Lambda\Pi_{w\in W}wf_{\gamma,u}(T)=\Pi_{w}\Pi_{\sigma}w\Lambda\mbox{det}\left(T\cdot I-p_{\gamma,\sigma}(u)\right)=1
\]


as $(w\Lambda)\mbox{det}\left(T\cdot I-p_{\gamma,\sigma}(u)\right)=\mbox{det}\left(T\cdot I-p_{\gamma,\sigma,w\Lambda}(u)\right)=1$,
since \textrm{$p_{\gamma,\sigma,w\Lambda}(u)$ as granted by our choice
of $u$ and the following:}
\begin{theorem}%{}
[Ann=Ker($p_{\gamma,\sigma,\mu}$)] \label{thm:-For-}For $u\in U(\mathfrak{g})^{K}$
we have $u\in\mbox{Ann}\left[\mbox{Hom}_{K}(\gamma,(H^{\sigma,\mu})_{K})\right]\iff p_{\gamma,\sigma,\mu}(u)=0$.
\end{theorem}%{}

\lyxframeend{}


\lyxframeend{}\lyxframe{proof of {[}\nameref{thm:For--we}{]}}

So, we need to show that $v:=\sum_{i}z_{i}u^{i}$ we have $v(u)\in U(\mathfrak{g})^{K}\cap U(\mathfrak{g})\mbox{Ker}(\beta_{\gamma})$.
This follows from
\begin{theorem}%{}
[A]If $v\in U(\mathfrak{g})^{K}$ and $p_{\gamma}(u)=0$, then \textrm{$U(\mathfrak{g})^{K}\cap U(\mathfrak{g})\mbox{Ker}(\beta_{\gamma})$.}
\end{theorem}%{}
We have $p_{\gamma}(v(u))=\sum_{j}p_{\gamma}(z_{j})p_{\gamma}(u)^{j}$.
And then $p_{\gamma,\sigma}(v(u))=\sum_{j}p_{\gamma,\sigma}(z_{j})p_{\gamma,\sigma}(u)^{j}=\sum_{j}(I\otimes\Omega_{\sigma})q(z_{j})p_{\gamma,\sigma}(u)^{j}$%
\footnote{$\Omega_{\sigma}$ denotes the lowes weight in representation $\sigma$%
}%
\footnote{$\because(I\otimes P_{\sigma}\beta_{\gamma}P_{\sigma})p(z)$ should
be $U(\mathfrak{a})\otimes$scalar by Schur and we compute it by applying
to lowest vector.%
}.

Hence, $p_{\gamma,\sigma}(v(u))=g_{\gamma,u}(p_{\gamma,\sigma}(u))=0$.
Now 
\[
0=p_{\gamma,\sigma}(v(u))=(I\otimes P_{\sigma}\beta_{\gamma}P_{\sigma})p(v)=\sum_{i}a_{i}\otimes P_{\sigma}\beta_{\gamma}(k_{i})P_{\sigma}
\]


with $k_{i}\in U(\mathfrak{k})^{M}$ and hence $=\sum_{i}a_{i}\otimes\beta_{\gamma}(k_{i})P_{\sigma}$.%
\footnote{elements of $U(\mathfrak{k})^{M}$ stabilize $\gamma(\sigma)$ --
see $\gamma$ as $(\mathfrak{k},M)$-module%
}. Hence $p_{\gamma}(v)=0$.


\lyxframeend{}


\lyxframeend{}\lyxframe{proof of {[}\nameref{thm:For--we}{]}: {[}A{]}}
\begin{enumerate}
\item $T:S(\mathfrak{p})\otimes U(\mathfrak{k})\to U(\mathfrak{g})$ be
linear bijection defined by $p\otimes k\mapsto\mbox{symm}(p)k$%
\footnote{$\mbox{symm}(X_{I}):=\sum_{\sigma\in\mathfrak{S}}X_{\sigma(1)}\dots X_{\sigma(\left|I\right|)}$%
}
\item $q:=(\mbox{Res}_{\mathfrak{p}|\mathfrak{a}}\otimes I)T^{-1}:U(\mathfrak{g})\to S(\mathfrak{a})\otimes U(\mathfrak{k})$
\item $q_{\gamma}(u):=(I\otimes\beta_{\gamma})q:U(\mathfrak{g})\to S(\mathfrak{a})\otimes\mbox{End}_{\mathbb{C}}(\gamma)$. 
\item $S^{j}(\mathfrak{p})$ and $S_{j}(\mathfrak{p})$ denote grading and
filtration of $S(\mathfrak{p})$.
\item $U_{j}:=\mbox{symm}(S_{j}(\mathfrak{p}))U(\mathfrak{k})$
\end{enumerate}
We now assume two claims and prove the theorem.
\begin{theorem}%{}
\end{theorem}%{}
\begin{enumerate}
\item [B]$g\in U_{j}$, then $p(g)-q(g)\in U^{j-1}(\mathfrak{a})\otimes U(\mathfrak{k})$
\item [C]$v\in U(\mathfrak{g})^{K}$ and $q_{\gamma}(v)=0$, then $u\in U(\mathfrak{g})^{K}\cap U(\mathfrak{g})\mbox{Ker}(\beta_{\gamma})$
\end{enumerate}
Now given $v\in U(\mathfrak{g})$ with $p_{\gamma}(v)=0$ it suffices
to show $q_{\gamma}(v)=0$. Suppose not.


\lyxframeend{}


\lyxframeend{}\lyxframe{proof of {[}\nameref{thm:For--we}{]}: {[}A{]}, {[}C{]}}

Then $v=\sum_{i}v_{i},\; v_{i}\in\mbox{symm}(S^{i}(\mathfrak{p}))U(\mathfrak{k})$
and we can find $r$ such that $q_{\gamma}(v_{r})\neq0$ but $q_{\gamma}(v_{j})=0$
for all $j>r$. Then by {[}C{]} we have $p_{\gamma}(v_{j})=0$ for
all $j>r$%
\footnote{$U(\mathfrak{g})\mbox{Ker}\beta_{\gamma}=(U(\mathfrak{a})\otimes U(\mathfrak{k})\oplus\mathfrak{n}U(\mathfrak{g}))\mbox{Ker}\beta_{\gamma}$%
} and hence $0=p_{\gamma}(v)\equiv p_{\gamma}(v_{r})$ mod $U_{r-1}(\mathfrak{a})\otimes U(\mathfrak{k})$.
But $q_{\gamma}(v)\equiv q_{\gamma}(v_{r})\neq0$ mod $U_{r-1}(\mathfrak{a})\otimes U(\mathfrak{k})$
and this contradicts {[}B{]}.

Next we prove {[}C{]}. Let $k_{j}\in U(\mathfrak{k})$ be such that
$\beta_{\gamma}(k_{j})\in\mbox{End}_{\mathbb{C}}(\gamma)$ form a
basis. Then $v\equiv\sum_{j}\mbox{symm}(p_{j})k_{j}$ mod $U(\mathfrak{g})\mbox{Ker}\beta_{\gamma}$.
Also
\[
0=q_{\gamma}(u)=q_{\gamma}(Ad(k)u)\equiv\sum_{j}\mbox{Res}_{\mathfrak{p}|\mathfrak{a}}(Ad(k)p_{j})\cdot\left(Ad(k)k_{j}\right)
\]


This implies that $\forall k\in K$ \textrm{we have $\mbox{Res}_{\mathfrak{p}|\mathfrak{a}}(Ad(k)p_{j})=0$
and as $\mathfrak{p}=\bigcup_{k\in K}Ad(k)\mathfrak{a}$ this implies
$p_{j}=0$ for all $j$.}


\lyxframeend{}


\lyxframeend{}\lyxframe{proof of {[}\nameref{thm:For--we}{]}: {[}B{]}}

Finally, we prove {[}B{]}. Let $X_{j}$ be a basis for $\mathfrak{n}$.
Then $X_{j}-\theta X_{j}$ forms basis for $\mathfrak{a}^{\perp}\subset\mathfrak{p}$.
If $g\in U_{j}$ and modulo $U_{j-1}$%
\footnote{$g_{i}\in U_{j-1}$ and $X_{i}+\theta X_{i}\in\mathfrak{k}$%
} 
\[
g\equiv q(g)+\sum_{i}(X_{i}-\theta X_{i})g_{i}=q(g)+2\sum_{i}X_{i}g_{i}-\sum_{i}(X_{i}+\theta X_{i})g_{i}
\]
\[
g\equiv q(g)+2\sum_{i}X_{i}g_{i}
\]


and the result now follows by taking $p$ of both sides, as $p(q(g))=q(g)$,
$p(X_{i}g_{i})=0$ (since $X_{i}\in\mathfrak{n}$) and $p(U_{j})\subset U^{j}(\mathfrak{a})\otimes U(\mathfrak{k})$.


\lyxframeend{}


\lyxframeend{}\lyxframe{proof of {[}\nameref{thm:-For-}{]}}
\begin{theorem}%{}
[$H^{\sigma,\mu}$]\end{theorem}%{}
\begin{enumerate}
\item [4.]We have $(H^{\sigma,\mu})_{K}\simeq(\mbox{Ind}_{M}^{K}(\sigma))_{K}$
as $K$-modules via the $u\mapsto u\big|_{K}$ map
\item [5.]For $K$ compact group we have $\mbox{Hom}_{G}(\gamma,\mbox{Ind}_{M}^{K}(\sigma))\simeq\mbox{Hom}_{M}(\gamma,\sigma)$
with $T\mapsto(v\mapsto\widehat{T}(v):=T(v)(1)$
\end{enumerate}
Altogether these imply that we $T\mapsto(v\mapsto\widehat{T}(v):=T(v)(1)$
defines an isomorphism $\mbox{Hom}_{K}(\gamma,(H^{\sigma,\mu})_{K})\simeq\mbox{Hom}_{M}(\gamma,\sigma)$.

We seek to show that for $u\in U(\mathfrak{g})^{K}$ we have $u\in\mbox{Ann}\left[\mbox{Hom}_{K}(\gamma,(H^{\sigma,\mu})_{K})\right]\iff p_{\gamma,\sigma,\mu}(u)=0$.
More precisely, we'll show that $\widehat{uT}=\widehat{T}p_{\gamma,\sigma,\mu}$.


\lyxframeend{}


\lyxframeend{}\lyxframe{proof of {[}\nameref{thm:-For-}{]}}

We immediately have $\left(\mathfrak{n}(H^{\sigma,\mu})_{K}\right)(1)=0$
and $\left(\mathfrak{a}(H^{\sigma,\mu})_{K}\right)(1)=(\mu+\rho)(\mathfrak{a})$
and hence 
\[
\widehat{uT}(v)=(p(u)T(v))(1)=\sum_{i}\left(\left[a_{i}\otimes k_{i}\right]\cdot T(v)\right)(1)=
\]
\[
=\sum_{i}\left((\mu+\rho)(a_{i})k_{i}\cdot T(v)\right)(1)=\sum_{i}\left((\mu+\rho)(a_{i})\beta_{\gamma}(k_{i})P_{\sigma}\cdot T(v)\right)(1)=
\]


\[
=\left[T\left(\sum_{i}(\mu+\rho)(a_{i})\beta_{\gamma}(k_{i})v\right)\right](1)=T(p_{\gamma,\sigma,\mu}(u)v)(1)=
\]
\[
=\widehat{T}\left(p_{\gamma,\sigma,\mu}(u)v\right)
\]



\lyxframeend{}


\lyxframeend{}\lyxframe{appendix}
\begin{theorem}%{}

\end{theorem}%{}

\lyxframeend{}
\end{document}
