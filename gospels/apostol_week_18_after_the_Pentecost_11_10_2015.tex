\documentclass[10pt]{article} % use larger type; default would be 10pt
\usepackage[utf8]{inputenc}       % кодування документа; замість cp866nav
\usepackage{extsizes}
\usepackage[top=0.5in,bottom=0.5in,left=0.5in,right=0.5in]{geometry}
\usepackage[russian,english]{babel} % національна локалізація; може бути декілька
\usepackage{setspace}
\usepackage{CJKutf8}
\usepackage{mdframed}
\usepackage{extsizes}
\usepackage{setspace}
\title{Divine Liturgy\\Reading from The Epistles}
\author{Week 18 after the Pentecost\vspace{
-3ex%for author
}}
\date{\vspace{
-5ex%for date
}}
\begin{document}
\pagenumbering{gobble}
\begin{otherlanguage*}{russian}
\maketitle
\end{otherlanguage*}
\vspace*{\fill}
\Large%for eng/rus
\singlespacing %\\onehalfspacing \\doublespacing % for eng/rus
\framebox[\textwidth]{
\begin{minipage}[t]{0.45\textwidth}
\begin{otherlanguage*}{russian}
\textbf{2 Кор., 194 зач., XI, 31 - XII, 9.}\\
\Large%for eng/rus
Бог и Отец Господа нашего Иисуса Христа, благословенный во веки, знает, что я не лгу.
\\
В Дамаске областной правитель царя Ареты стерег город Дамаск, чтобы схватить меня; и я в корзине был спущен из окна по стене и избежал его рук.
\\
Не полезно хвалиться мне, ибо я приду к видениям и откровениям Господним.
\\
Знаю человека во Христе, который назад тому четырнадцать лет (в теле ли - не знаю, вне ли тела - не знаю: Бог знает) восхищен был до третьего неба.
\\
И знаю о таком человеке (только не знаю - в теле, или вне тела: Бог знает),
\\
что он был восхищен в рай и слышал неизреченные слова, которых человеку нельзя пересказать.
\\
Таким человеком могу хвалиться; собою же не похвалюсь, разве только немощами моими.
\\
Впрочем, если захочу хвалиться, не буду неразумен, потому что скажу истину; но я удерживаюсь, чтобы кто не подумал о мне более, нежели сколько во мне видит или слышит от меня.
\\
И чтобы я не превозносился чрезвычайностью откровений, дано мне жало в плоть, ангел сатаны, удручать меня, чтобы я не превозносился.
\\
Трижды молил я Господа о том, чтобы удалил его от меня.
\\
Но Господь сказал мне: "довольно для тебя благодати Моей, ибо сила Моя совершается в немощи". И потому я гораздо охотнее буду хвалиться своими немощами, чтобы обитала во мне сила Христова.
\\

\end{otherlanguage*}
\end{minipage}
\hfill
\begin{minipage}[t]{0.45\textwidth}

\textbf{2 Corinthians 11:31--12:9.}\\

\end{minipage}}
\vspace*{\fill}
\newpage
\Huge%for chi
\vspace*{\fill}
\begin{spacing}{1.3}%for chinese
\framebox[\textwidth]{
\begin{minipage}[t]{\textwidth}
\begin{CJK}{UTF8}{bsmi}
\textbf{哥林多後書 11:31--12:9.}\\
那永遠可稱頌之主耶穌的父 神、知道我不說謊。\\
在大馬色亞哩達王手下的提督、把守大馬色城要捉拿我.\\
我就從窗戶中、在筐子裡從城牆上被人縋下去、脫離了他的手。\\
我自誇固然無益、但我是不得已的.如今我要說到主的顯現和啟示。\\
我認得一個在基督裡的人、他前十四年被提到第三層天上去.或在身內、我不知道.或在身外、我也不知道.只有 神知道。\\
我認得這人、或在身內、或在身外、我都不知道.只有 神知道。\\
他被提到樂園裡、聽見隱秘的言語、是人不可說的。\\
為這人、我要誇口.但是為我自己、除了我的軟弱以外、我並不誇口。\\
我就是願意誇口、也不算狂.因為我必說實話.只是我禁止不說、恐怕有人把我看高了、過於他在我身上所看見所聽見的。\\
又恐怕我因所得的啟示甚大、就過於自高、所以有一根刺加在我肉體上、就是撒但的差役、要攻擊我、免得我過於自高。\\
為這事、我三次求過主、叫這刺離開我。\\
他對我說、我的恩典夠你用的.因為我的能力、是在人的軟弱上顯得完全.所以我更喜歡誇自己的軟弱、好叫基督的能力覆庇我.\\

\end{CJK}
\end{minipage}}
\end{spacing}
\vspace*{\fill}
\end{document}
