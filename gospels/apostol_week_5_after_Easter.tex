\documentclass[10pt]{article} % use larger type; default would be 10pt
\usepackage[utf8]{inputenc}       % кодування документа; замість cp866nav
\usepackage[top=0.0in,bottom=0.0in,left=0.0in,right=0.3in]{geometry}
\usepackage[russian,english]{babel} % національна локалізація; може бути декілька
\usepackage{setspace}
\usepackage{CJKutf8}
\usepackage{mdframed}
\usepackage{setspace}
\title{\vspace{-2ex}Divine Liturgy\\Reading from The Epistles\vspace{-1ex}}
\author{\vspace{-0ex}Week 5 after Easter\vspace{
-7ex%for author
}}
\date{\vspace{
-10ex%for date
}}
\begin{document}
\vspace{
-11ex%for date
}
\pagenumbering{gobble}
\begin{otherlanguage*}{russian}
\maketitle
\end{otherlanguage*}
\vspace*{\fill}
\Large%for eng/rus
\singlespacing %\\onehalfspacing \\doublespacing % for eng/rus
\framebox[\textwidth]{
\begin{minipage}[t]{0.45\textwidth}
\begin{otherlanguage*}{russian}
\textbf{Деян., 28 зач., XI, 19-26, 29-30.}\\
\Large%for eng/rus
Между тем рассеявшиеся от гонения, бывшего после Стефана, прошли до Финикии и Кипра и Антиохии, никому не проповедуя слово, кроме Иудеев.
\\
Были же некоторые из них Кипряне и Киринейцы, которые, придя в Антиохию, говорили Еллинам, благовествуя Господа Иисуса.
\\
И была рука Господня с ними, и великое число, уверовав, обратилось к Господу.
\\
Дошел слух о сем до церкви Иерусалимской, и поручили Варнаве идти в Антиохию.
\\
Он, прибыв и увидев благодать Божию, возрадовался и убеждал всех держаться Господа искренним сердцем;
\\
ибо он был муж добрый и исполненный Духа Святаго и веры. И приложилось довольно народа к Господу.
\\
Потом Варнава пошел в Тарс искать Савла и, найдя его, привел в Антиохию.
\\
Целый год собирались они в церкви и учили немалое число людей, и ученики в Антиохии в первый раз стали называться Христианами.
\\
Тогда ученики положили, каждый по достатку своему, послать пособие братьям, живущим в Иудее,
\\
что и сделали, послав собранное к пресвитерам через Варнаву и Савла.
\\

\end{otherlanguage*}
\end{minipage}
\hfill
\begin{minipage}[t]{0.45\textwidth}

\textbf{Acts 11:19--11:26, 11:29--11:30.}\\
Now they which were scattered abroad upon the persecution that arose about Stephen travelled as far as Phenice, and Cyprus, and Antioch, preaching the word to none but unto the Jews only.\\
And some of them were men of Cyprus and Cyrene, which, when they were come to Antioch, spake unto the Grecians, preaching the Lord Jesus.\\
And the hand of the Lord was with them: and a great number believed, and turned unto the Lord.\\
Then tidings of these things came unto the ears of the church which was in Jerusalem: and they sent forth Barnabas, that he should go as far as Antioch.\\
Who, when he came, and had seen the grace of God, was glad, and exhorted them all, that with purpose of heart they would cleave unto the Lord.\\
For he was a good man, and full of the Holy Ghost and of faith: and much people was added unto the Lord.\\
Then departed Barnabas to Tarsus, for to seek Saul:\\
And when he had found him, he brought him unto Antioch. And it came to pass, that a whole year they assembled themselves with the church, and taught much people. And the disciples were called Christians first in Antioch.\\
Then the disciples, every man according to his ability, determined to send relief unto the brethren which dwelt in Judaea:\\
Which also they did, and sent it to the elders by the hands of Barnabas and Saul.\\

\end{minipage}}
\vspace*{\fill}
\newpage
\Huge%for chi
\vspace*{\fill}
\begin{spacing}{1.1}%for chinese
\framebox[\textwidth]{
\begin{minipage}[t]{\textwidth}
\begin{CJK}{UTF8}{bsmi}
\textbf{使徒行傳 11:19--11:26, 11:29--11:30.}\\
那些因司提反的事遭患難四散的門徒、直走到腓尼基、和居比路、並安提阿.他們不向別人講道、只向猶太人講。\\
但內中有居比路、和古利奈人、他們到了安提阿、也向希利尼人傳講主耶穌。〔有古卷作也向說希利尼話的猶太人傳講主耶穌〕\\
主與他們同在、信而歸主的人就很多了。\\
這風聲傳到耶路撒冷教會人的耳中、他們就打發巴拿巴出去、走到安提阿為止.\\
他到了那裡、看見 神所賜的恩就歡喜、勸勉眾人、立定心志、恆久靠主。\\
這巴拿巴原是個好人、被聖靈充滿、大有信心.於是有許多人歸服了主。\\
他又往大數去找掃羅、\\
找著了、就帶他到安提阿去。他們足有一年的工夫、和教會一同聚集、教訓了許多人.門徒稱為基督徒、是從安提阿起首。\\
於是門徒定意、照各人的力量捐錢、送去供給住在猶太的弟兄。\\
他們就這樣行、把捐項託巴拿巴和掃羅、送到眾長老那裡。\\

\end{CJK}
\end{minipage}}
\end{spacing}
\vspace*{\fill}
\end{document}
