about the Pharisee and the Publican (Martyrs)

\singlespacing %\\onehalfspacing \\doublespacing % for eng/rus

\begin{spacing}{1.0}%for chinese

\large%for eng/rus

Рим., 99 зач., VIII, 28-39.

eng title

chi title

Притом знаем, что любящим Бога, призванным по Его изволению, все содействует ко благу.
Ибо кого Он предузнал, тем и предопределил быть подобными образу Сына Своего, дабы Он был первородным между многими братиями.
А кого Он предопределил, тех и призвал, а кого призвал, тех и оправдал; а кого оправдал, тех и прославил.
Что же сказать на это? Если Бог за нас, кто против нас?
Тот, Который Сына Своего не пощадил, но предал Его за всех нас, как с Ним не дарует нам и всего?
Кто будет обвинять избранных Божиих? Бог оправдывает их.
Кто осуждает? Христос Иисус умер, но и воскрес: Он и одесную Бога, Он и ходатайствует за нас.
Кто отлучит нас от любви Божией: скорбь, или теснота, или гонение, или голод, или нагота, или опасность, или меч? как написано:
за Тебя умерщвляют нас всякий день, считают нас за овец, обреченных на заклание.
Но все сие преодолеваем силою Возлюбившего нас.
Ибо я уверен, что ни смерть, ни жизнь, ни Ангелы, ни Начала, ни Силы, ни настоящее, ни будущее,
ни высота, ни глубина, ни другая какая тварь не может отлучить нас от любви Божией во Христе Иисусе, Господе нашем. 

eng text
eng text

chi text
chi text

\huge%for chi

-3ex%for author

-5ex%for date
