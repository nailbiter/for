1 of the Great Lent

\singlespacing %\\onehalfspacing \\doublespacing % for eng/rus

\begin{spacing}{1.0}%for chinese

\large%for eng/rus

Евр., 329 зач., XI, 24-26, 32 - XII, 2.

eng title

chi title

Верою Моисей, придя в возраст, отказался называться сыном дочери фараоновой,
и лучше захотел страдать с народом Божиим, нежели иметь временное греховное наслаждение,
и поношение Христово почел бо́льшим для себя богатством, нежели Египетские сокровища; ибо он взирал на воздаяние. 
И что еще скажу? Недостанет мне времени, чтобы повествовать о Гедеоне, о Вараке, о Самсоне и Иеффае, о Давиде, Самуиле и (других) пророках,
которые верою побеждали царства, творили правду, получали обетования, заграждали уста львов,
угашали силу огня, избегали острия меча, укреплялись от немощи, были крепки на войне, прогоняли полки чужих;
жены получали умерших своих воскресшими; иные же замучены были, не приняв освобождения, дабы получить лучшее воскресение;
другие испытали поругания и побои, а также узы и темницу,
были побиваемы камнями, перепиливаемы, подвергаемы пытке, умирали от меча, скитались в ми́лотях и козьих кожах, терпя недостатки, скорби, озлобления;
те, которых весь мир не был достоин, скитались по пустыням и горам, по пещерам и ущельям земли.
И все сии, свидетельствованные в вере, не получили обещанного,
потому что Бог предусмотрел о нас нечто лучшее, дабы они не без нас достигли совершенства. 
Посему и мы, имея вокруг себя такое облако свидетелей, свергнем с себя всякое бремя и запинающий нас грех и с терпением будем проходить предлежащее нам поприще,
взирая на начальника и совершителя веры Иисуса, Который, вместо предлежавшей Ему радости, претерпел крест, пренебрегши посрамление, и воссел одесную престола Божия. 

eng text
eng text

chi text
chi text

\huge%for chi

-3ex%for author

-5ex%for date
