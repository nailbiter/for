\documentclass[10pt]{article} % use larger type; default would be 10pt
\usepackage[utf8]{inputenc}       % кодування документа; замість cp866nav
\usepackage{extsizes}
\usepackage[top=0.1in,bottom=0.2in,left=0.0in,right=0.5in]{geometry}
\usepackage[russian,english]{babel} % національна локалізація; може бути декілька
\usepackage{setspace}
\usepackage{CJKutf8}
\usepackage{mdframed}
\usepackage{extsizes}
\usepackage{setspace}
\title{\vspace{-2ex}Divine Liturgy\\Reading from The Epistles}
\author{Week 12 after the Pentecost\vspace{
-3ex%for author
}}
\date{\vspace{
-5ex%for date
}}
\begin{document}
\pagenumbering{gobble}
\begin{otherlanguage*}{russian}
\maketitle
\end{otherlanguage*}
\vspace*{\fill}
\Large%for eng/rus
\singlespacing %\\onehalfspacing \\doublespacing % for eng/rus
\framebox[\textwidth]{
\begin{minipage}[t]{0.45\textwidth}
\begin{otherlanguage*}{russian}
\textbf{1 Кор., 158 зач., XV, 1-11.}\\
\Large%for eng/rus
Напоминаю вам, братия, Евангелие, которое я благовествовал вам, которое вы и приняли, в котором и утвердились,
\\
которым и спасаетесь, если преподанное удерживаете так, как я благовествовал вам, если только не тщетно уверовали.
\\
Ибо я первоначально преподал вам, что и сам принял, то есть, что Христос умер за грехи наши, по Писанию,
\\
и что Он погребен был, и что воскрес в третий день, по Писанию,
\\
и что явился Кифе, потом двенадцати;
\\
потом явился более нежели пятистам братий в одно время, из которых большая часть доныне в живых, а некоторые и почили;
\\
потом явился Иакову, также всем Апостолам;
\\
а после всех явился и мне, как некоему извергу.
\\
Ибо я наименьший из Апостолов, и недостоин называться Апостолом, потому что гнал церковь Божию.
\\
Но благодатию Божиею есмь то, что есмь; и благодать Его во мне не была тщетна, но я более всех их потрудился: не я, впрочем, а благодать Божия, которая со мною.
\\
Итак я ли, они ли, мы так проповедуем, и вы так уверовали.
\\

\end{otherlanguage*}
\end{minipage}
\hfill
\begin{minipage}[t]{0.45\textwidth}

\textbf{1 Corinthians 15:1--15:11.}\\
Moreover, brethren, I declare unto you the gospel which I preached unto you, which also ye have received, and wherein ye stand;\\
By which also ye are saved, if ye keep in memory what I preached unto you, unless ye have believed in vain.\\
For I delivered unto you first of all that which I also received, how that Christ died for our sins according to the scriptures;\\
And that he was buried, and that he rose again the third day according to the scriptures:\\
And that he was seen of Cephas, then of the twelve:\\
After that, he was seen of above five hundred brethren at once; of whom the greater part remain unto this present, but some are fallen asleep.\\
After that, he was seen of James; then of all the apostles.\\
And last of all he was seen of me also, as of one born out of due time.\\
For I am the least of the apostles, that am not meet to be called an apostle, because I persecuted the church of God.\\
But by the grace of God I am what I am: and his grace which was bestowed upon me was not in vain; but I laboured more abundantly than they all: yet not I, but the grace of God which was with me.\\
Therefore whether it were I or they, so we preach, and so ye believed.\\

\end{minipage}}
\vspace*{\fill}
\newpage
\Huge%for chi
\vspace*{\fill}
\begin{spacing}{1.3}%for chinese
\framebox[\textwidth]{
\begin{minipage}[t]{\textwidth}
\begin{CJK}{UTF8}{bsmi}
\textbf{哥林多前書 15:1--15:11.}\\
弟兄們、我如今把先前所傳給你們的福音、告訴你們知道、這福音你們也領受了、又靠著站立得住.\\
並且你們若不是徒然相信、能以持守我所傳給你們的、就必因這福音得救。\\
我當日所領受又傳給你們的、第一、就是基督照聖經所說、為我們的罪死了.\\
而且埋葬了.又照聖經所說、第三天復活了.\\
並且顯給磯法看.然後顯給十二使徒看.\\
後來一時顯給五百多弟兄看、其中一大半到如今還在、卻也有已經睡了的.\\
以後顯給雅各看.再顯給眾使徒看.\\
末了也顯給我看.我如同未到產期而生的人一般。\\
我原是使徒中最小的、不配稱為使徒、因為我從前逼迫 神的教會。\\
然而我今日成了何等人、是蒙 神的恩纔成的.並且他所賜我的恩、不是徒然的.我比眾使徒格外勞苦.這原不是我、乃是 神的恩與我同在。\\
不拘是我是眾使徒、我們如此傳、你們也如此信了。
\end{CJK}
\end{minipage}}
\end{spacing}
\vspace*{\fill}
\end{document}
