\documentclass[10pt]{article} % use larger type; default would be 10pt
\usepackage[utf8]{inputenc}       % кодування документа; замість cp866nav
\usepackage[top=0.2in,bottom=0.2in,left=0.2in,right=0.3in]{geometry}
\usepackage[russian,english]{babel} % національна локалізація; може бути декілька
\usepackage{setspace}
\usepackage{CJKutf8}
\usepackage{mdframed}
\usepackage{setspace}
\title{Divine Liturgy\\Reading from The Epistles\vspace{-1ex}}
\author{Week 5 after the Pentecost\vspace{
-6ex%for author
}}
\date{\vspace{
-7ex%for date
}}
\begin{document}
\pagenumbering{gobble}
\begin{otherlanguage*}{russian}
\maketitle
\end{otherlanguage*}
\vspace*{\fill}
\Large%for eng/rus
\singlespacing %\\onehalfspacing \\doublespacing % for eng/rus
\framebox[\textwidth]{
\begin{minipage}[t]{0.45\textwidth}
\begin{otherlanguage*}{russian}
\textbf{Рим., 103 зач., X, 1-10.}\\
\Large%for eng/rus
Братия! желание моего сердца и молитва к Богу об Израиле во спасение.
\\
Ибо свидетельствую им, что имеют ревность по Боге, но не по рассуждению.
\\
Ибо, не разумея праведности Божией и усиливаясь поставить собственную праведность, они не покорились праведности Божией,
\\
потому что конец закона - Христос, к праведности всякого верующего.
\\
Моисей пишет о праведности от закона: исполнивший его человек жив будет им.
\\
А праведность от веры так говорит: не говори в сердце твоем: кто взойдет на небо? то есть Христа свести.
\\
Или кто сойдет в бездну? то есть Христа из мертвых возвести.
\\
Но что говорит Писание? Близко к тебе слово, в устах твоих и в сердце твоем, то есть слово веры, которое проповедуем.
\\
Ибо если устами твоими будешь исповедовать Иисуса Господом и сердцем твоим веровать, что Бог воскресил Его из мертвых, то спасешься,
\\
потому что сердцем веруют к праведности, а устами исповедуют ко спасению.
\\

\end{otherlanguage*}
\end{minipage}
\hfill
\begin{minipage}[t]{0.45\textwidth}

\textbf{Romans 10:1--10:10.}\\
Brethren, my heart's desire and prayer to God for Israel is, that they might be saved.\\
For I bear them record that they have a zeal of God, but not according to knowledge.\\
For they being ignorant of God's righteousness, and going about to establish their own righteousness, have not submitted themselves unto the righteousness of God.\\
For Christ is the end of the law for righteousness to every one that believeth.\\
For Moses describeth the righteousness which is of the law, That the man which doeth those things shall live by them.\\
But the righteousness which is of faith speaketh on this wise, Say not in thine heart, Who shall ascend into heaven? (that is, to bring Christ down from above:)\\
Or, Who shall descend into the deep? (that is, to bring up Christ again from the dead.)\\
But what saith it? The word is nigh thee, even in thy mouth, and in thy heart: that is, the word of faith, which we preach;\\
That if thou shalt confess with thy mouth the Lord Jesus, and shalt believe in thine heart that God hath raised him from the dead, thou shalt be saved.\\
For with the heart man believeth unto righteousness; and with the mouth confession is made unto salvation.\\

\end{minipage}}
\vspace*{\fill}
\newpage
\Huge%for chi
\vspace*{\fill}
\begin{spacing}{1.3}%for chinese
\framebox[\textwidth]{
\begin{minipage}[t]{\textwidth}
\begin{CJK}{UTF8}{bsmi}
\textbf{羅馬書 10:1--10:10.}\\
弟兄們、我心裡所願的、向 神所求的、是要以色列人得救。\\
我可以證明他們向 神有熱心、但不是按著真知識.\\
因為不知道 神的義、想要立自己的義、就不服 神的義了。\\
律法的總結就是基督、使凡信他的都得著義。\\
摩西寫著說、『人若行那出於律法的義、就必因此活著。』\\
惟有出於信心的義如此說、『你不要心裡說、誰要升到天上去呢.就是要領下基督來.\\
誰要下到陰間去呢.就是要領基督從死裡上來。』\\
他到底怎麼說呢.他說、『這道離你不遠、正在你口裡、在你心裡。』就是我們所傳信主的道。\\
你若口裡認耶穌為主、心裡信 神叫他從死裡復活、就必得救.\\
因為人心裡相信、就可以稱義.口裡承認、就可以得救。\\

\end{CJK}
\end{minipage}}
\end{spacing}
\vspace*{\fill}
\end{document}
