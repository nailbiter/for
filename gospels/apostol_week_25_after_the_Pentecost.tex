\documentclass[14pt]{article} % use larger type; default would be 10pt
\usepackage[utf8]{inputenc}       % кодування документа; замість cp866nav
\usepackage{extsizes}
\usepackage[russian,english]{babel} % національна локалізація; може бути декілька
\usepackage[top=0.3in,bottom=0.3in,left=0.3in,right=0.5in]{geometry}
\usepackage{setspace}
\usepackage{CJKutf8}
\usepackage{mdframed}
\usepackage{setspace}
\title{Divine Liturgy\\Reading from The Epistles}
\author{Week 25 after the Pentecost\vspace{
-3ex%for author
}}
\date{\vspace{
-5ex%for date
}}
\begin{document}
\pagenumbering{gobble}
\begin{otherlanguage*}{russian}
\maketitle
\end{otherlanguage*}
\vspace*{\fill}
\Large%for eng/rus
\onehalfspacing %\\onehalfspacing \\doublespacing % for eng/rus
\framebox[\textwidth]{
\begin{minipage}[t]{0.45\textwidth}
\begin{otherlanguage*}{russian}
\textbf{Еф., 224 зач., IV, 1-6.}\\
\Large%for eng/rus
Итак я, узник в Господе, умоляю вас поступать достойно звания, в которое вы призваны,
\\
со всяким смиренномудрием и кротостью и долготерпением, снисходя друг ко другу любовью,
\\
стараясь сохранять единство духа в союзе мира.
\\
Одно тело и один дух, как вы и призваны к одной надежде вашего звания;
\\
один Господь, одна вера, одно крещение,
\\
один Бог и Отец всех, Который над всеми, и через всех, и во всех нас.
\end{otherlanguage*}
\end{minipage}
\hfill
\begin{minipage}[t]{0.45\textwidth}

\textbf{Ephesians 4:1--4:6.}\\
I therefore, the prisoner of the Lord, beseech you that ye walk worthy of the vocation wherewith ye are called,\\
With all lowliness and meekness, with longsuffering, forbearing one another in love;\\
Endeavouring to keep the unity of the Spirit in the bond of peace.\\
There is one body, and one Spirit, even as ye are called in one hope of your calling;\\
One Lord, one faith, one baptism,\\
One God and Father of all, who is above all, and through all, and in you all.
\end{minipage}}
\vspace*{\fill}
\newpage
\Huge%for chi
\vspace*{\fill}
\begin{spacing}{1.3}%for chinese
\framebox[\textwidth]{
\begin{minipage}[t]{\textwidth}
\begin{CJK}{UTF8}{bsmi}
%\fontsize{40}{48}\selectfont
\textbf{以弗所書 4:1--4:6.}\\
%\fontsize{40}{48}\selectfont
我為主被囚的勸你們、既然蒙召、行事為人就當與蒙召的恩相稱.\\
凡事謙虛、溫柔、忍耐、用愛心互相寬容、\\
用和平彼此聯絡、竭力保守聖靈所賜合而為一的心。\\
身體只有一個、聖靈只有一個、正如你們蒙召、同有一個指望、\\
一主、一信、一洗、\\
一 神、就是眾人的父、超乎眾人之上、貫乎眾人之中、也住在眾人之內。
\end{CJK}
\end{minipage}}
\end{spacing}
\vspace*{\fill}
\end{document}
