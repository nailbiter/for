\documentclass[10pt]{article} % use larger type; default would be 10pt
\usepackage[utf8]{inputenc}       % кодування документа; замість cp866nav
\usepackage[top=0.5in,bottom=0.5in,left=0.5in,right=0.5in]{geometry}
\usepackage[russian,english]{babel} % національна локалізація; може бути декілька
\usepackage{setspace}
\usepackage{CJKutf8}
\usepackage{mdframed}
\usepackage{setspace}
\title{Divine Liturgy\\Reading from The Epistles}
\author{Week 1 after the Pentecost\vspace{
-3ex%for author
}}
\date{\vspace{
-5ex%for date
}}
\begin{document}
\pagenumbering{gobble}
\begin{otherlanguage*}{russian}
\maketitle
\end{otherlanguage*}
\vspace*{\fill}
\Large%for eng/rus
\singlespacing %\\onehalfspacing \\doublespacing % for eng/rus
\framebox[\textwidth]{
\begin{minipage}[t]{0.45\textwidth}
\begin{otherlanguage*}{russian}
\textbf{Евр., 330 зач., XI, 33 - XII, 2.}\\
\Large%for eng/rus
которые верою побеждали царства, творили правду, получали обетования, заграждали уста львов,
\\
угашали силу огня, избегали острия меча, укреплялись от немощи, были крепки на войне, прогоняли полки чужих;
\\
жены получали умерших своих воскресшими; иные же замучены были, не приняв освобождения, дабы получить лучшее воскресение;
\\
другие испытали поругания и побои, а также узы и темницу,
\\
были побиваемы камнями, перепиливаемы, подвергаемы пытке, умирали от меча, скитались в милотях и козьих кожах, терпя недостатки, скорби, озлобления;
\\
те, которых весь мир не был достоин, скитались по пустыням и горам, по пещерам и ущельям земли.
\\
И все сии, свидетельствованные в вере, не получили обещанного,
\\
потому что Бог предусмотрел о нас нечто лучшее, дабы они не без нас достигли совершенства.
\\
Посему и мы, имея вокруг себя такое облако свидетелей, свергнем с себя всякое бремя и запинающий нас грех и с терпением будем проходить предлежащее нам поприще,
\\
взирая на начальника и совершителя веры Иисуса, Который, вместо предлежавшей Ему радости, претерпел крест, пренебрегши посрамление, и воссел одесную престола Божия.
\\

\end{otherlanguage*}
\end{minipage}
\hfill
\begin{minipage}[t]{0.45\textwidth}

\textbf{Hebrews 11:33--12:2.}\\
Who through faith subdued kingdoms, wrought righteousness, obtained promises, stopped the mouths of lions,\\
Quenched the violence of fire, escaped the edge of the sword, out of weakness were made strong, waxed valiant in fight, turned to flight the armies of the aliens.\\
Women received their dead raised to life again: and others were tortured, not accepting deliverance; that they might obtain a better resurrection:\\
And others had trial of cruel mockings and scourgings, yea, moreover of bonds and imprisonment:\\
They were stoned, they were sawn asunder, were tempted, were slain with the sword: they wandered about in sheepskins and goatskins; being destitute, afflicted, tormented;\\
(Of whom the world was not worthy:) they wandered in deserts, and in mountains, and in dens and caves of the earth.\\
And these all, having obtained a good report through faith, received not the promise:\\
God having provided some better thing for us, that they without us should not be made perfect.\\
Wherefore seeing we also are compassed about with so great a cloud of witnesses, let us lay aside every weight, and the sin which doth so easily beset us, and let us run with patience the race that is set before us,\\
Looking unto Jesus the author and finisher of our faith; who for the joy that was set before him endured the cross, despising the shame, and is set down at the right hand of the throne of God.\\

\end{minipage}}
\vspace*{\fill}
\newpage
\Huge%for chi
\vspace*{\fill}
\begin{spacing}{1.3}%for chinese
\framebox[\textwidth]{
\begin{minipage}[t]{\textwidth}
\begin{CJK}{UTF8}{bsmi}
\textbf{希伯來書 11:33--12:2.}\\
他們因著信、制伏了敵國、行了公義、得了應許、堵了獅子的口。\\
滅了烈火的猛勢、脫了刀劍的鋒刃、軟弱變為剛強、爭戰顯出勇敢、打退外邦的全軍。\\
有婦人得自己的死人復活、又有人忍受嚴刑、不肯苟且得釋放、〔釋放原文作贖〕為要得著更美的復活.\\
又有人忍受戲弄、鞭打、捆鎖、監禁、各等的磨煉、\\
被石頭打死、被鋸鋸死、受試探、被刀殺.披著綿羊山羊的皮各處奔跑、受窮乏、患難、苦害、\\
在曠野、山嶺、山洞、地穴、飄流無定.本是世界不配有的人。\\
這些人都是因信得了美好的證據、卻仍未得著所應許的.\\
因為 神給我們預備了更美的事、叫他們若不與我們同得、就不能完全。\\
我們既有這許多的見證人、如同雲彩圍著我們、就當放下各樣的重擔、脫去容易纏累我們的罪、存心忍耐、奔那擺在我們前頭的路程、\\
仰望為我們信心創始成終的耶穌.〔或作仰望那將真道創始成終的耶穌〕他因那擺在前面的喜樂、就輕看羞辱、忍受了十字架的苦難、便坐在 神寶座的右邊。\\

\end{CJK}
\end{minipage}}
\end{spacing}
\vspace*{\fill}
\end{document}
