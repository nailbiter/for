\documentclass[10pt]{article} % use larger type; default would be 10pt
\usepackage[utf8]{inputenc}       % кодування документа; замість cp866nav
\usepackage{extsizes}
\usepackage[top=0.5in,bottom=0.5in,left=0.5in,right=0.5in]{geometry}
\usepackage[russian,english]{babel} % національна локалізація; може бути декілька
\usepackage{setspace}
\usepackage{CJKutf8}
\usepackage{mdframed}
\usepackage{extsizes}
\usepackage{setspace}
\title{Divine Liturgy\\Reading from The Epistles}
\author{Week 14 after the Pentecost\vspace{
-3ex%for author
}}
\date{\vspace{
-5ex%for date
}}
\begin{document}
\pagenumbering{gobble}
\begin{otherlanguage*}{russian}
\maketitle
\end{otherlanguage*}
\vspace*{\fill}
\Large%for eng/rus
\singlespacing %\\onehalfspacing \\doublespacing % for eng/rus
\framebox[\textwidth]{
\begin{minipage}[t]{0.45\textwidth}
\begin{otherlanguage*}{russian}
\textbf{2 Кор., 170 зач., I, 21 - II, 4.}\\
\Large%for eng/rus
Утверждающий же нас с вами во Христе и помазавший нас есть Бог,
\\
Который и запечатлел нас и дал залог Духа в сердца наши.
\\
Бога призываю во свидетели на душу мою, что, щадя вас, я доселе не приходил в Коринф,
\\
не потому, будто мы берем власть над верою вашею; но мы споспешествуем радости вашей: ибо верою вы тверды.
\\
Итак я рассудил сам в себе не приходить к вам опять с огорчением.
\\
Ибо если я огорчаю вас, то кто обрадует меня, как не тот, кто огорчен мною?
\\
Это самое и писал я вам, дабы, придя, не иметь огорчения от тех, о которых мне надлежало радоваться: ибо я во всех вас уверен, что || моя радость есть радость и для всех вас.
\\
От великой скорби и стесненного сердца я писал вам со многими слезами, не для того, чтобы огорчить вас, но чтобы вы познали любовь, какую я в избытке имею к вам.
\\

\end{otherlanguage*}
\end{minipage}
\hfill
\begin{minipage}[t]{0.45\textwidth}

\textbf{2 Corinthians 1:21--2:4.}\\
Now he which stablisheth us with you in Christ, and hath anointed us, is God;\\
Who hath also sealed us, and given the earnest of the Spirit in our hearts.\\
Moreover I call God for a record upon my soul, that to spare you I came not as yet unto Corinth.\\
Not for that we have dominion over your faith, but are helpers of your joy: for by faith ye stand.\\
But I determined this with myself, that I would not come again to you in heaviness.\\
For if I make you sorry, who is he then that maketh me glad, but the same which is made sorry by me?\\
And I wrote this same unto you, lest, when I came, I should have sorrow from them of whom I ought to rejoice; having confidence in you all, that my joy is the joy of you all.\\
For out of much affliction and anguish of heart I wrote unto you with many tears; not that ye should be grieved, but that ye might know the love which I have more abundantly unto you.\\

\end{minipage}}
\vspace*{\fill}
\newpage
\Huge%for chi
\vspace*{\fill}
\begin{spacing}{1.3}%for chinese
\framebox[\textwidth]{
\begin{minipage}[t]{\textwidth}
\begin{CJK}{UTF8}{bsmi}
\textbf{哥林多後書 1:21--2:4.}\\
那在基督裡堅固我們和你們、並且膏我們的、就是 神.\\
他又用印印了我們、並賜聖靈在我們心裡作憑據。〔原文作質〕\\
我呼籲 神給我的心作見證、我沒有往哥林多去是為要寬容你們。\\
我們並不是轄管你們的信心、乃是幫助你們的快樂.因為你們憑信纔站立得住。\\
我自己定了主意、再到你們那裡去、必須大家沒有憂愁。\\
倘若我叫你們憂愁、除了我叫那憂愁的人以外、誰能叫我快樂呢。\\
我曾把這事寫給你們、恐怕我到的時候、應該叫我快樂的那些人、反倒叫我憂愁.我也深信、你們眾人都以我的快樂為自己的快樂。\\
我先前心裡難過痛苦、多多的流淚、寫信給你們.不是叫你們憂愁、乃是叫你們知道我格外的疼愛你們。\\

\end{CJK}
\end{minipage}}
\end{spacing}
\vspace*{\fill}
\end{document}
