\documentclass[10pt]{article} % use larger type; default would be 10pt
\usepackage[utf8]{inputenc}       % кодування документа; замість cp866nav
\usepackage[top=0.0in,bottom=0.0in,left=-0.15in,right=0.25in]{geometry}
\usepackage[russian,english]{babel} % національна локалізація; може бути декілька
\usepackage{setspace}
\usepackage{CJKutf8}
\usepackage{mdframed}
\usepackage[10pt]{moresize}
\usepackage{anyfontsize}
\title{Divine Liturgy\\Reading from The Gospels}
\author{Week 6 after Easter\vspace{
-3ex%for author
}}
\date{\vspace{
-7ex%for date
}}
\begin{document}
\pagenumbering{gobble}
\begin{otherlanguage*}{russian}
\maketitle
\end{otherlanguage*}
\vspace*{\fill}
\fontsize{8}{9.6}
\framebox[\textwidth]{
\begin{minipage}[t]{0.48\textwidth}
\begin{otherlanguage*}{russian}
\textbf{Ин., 34 зач., IX, 1-38.}\\
И, проходя, увидел человека, слепого от рождения.
\\
Ученики Его спросили у Него: Равви! кто согрешил, он или родители его, что родился слепым?
\\
Иисус отвечал: не согрешил ни он, ни родители его, но это для того, чтобы на нем явились дела Божии.
\\
Мне должно делать дела Пославшего Меня, доколе есть день; приходит ночь, когда никто не может делать.
\\
Доколе Я в мире, Я свет миру.
\\
Сказав это, Он плюнул на землю, сделал брение из плюновения и помазал брением глаза слепому,
\\
и сказал ему: пойди, умойся в купальне Силоам, что значит: посланный. Он пошел и умылся, и пришел зрячим.
\\
Тут соседи и видевшие прежде, что он был слеп, говорили: не тот ли это, который сидел и просил милостыни?
\\
Иные говорили: это он, а иные: похож на него. Он же говорил: это я.
\\
Тогда спрашивали у него: как открылись у тебя глаза?
\\
Он сказал в ответ: Человек, называемый Иисус, сделал брение, помазал глаза мои и сказал мне: пойди на купальню Силоам и умойся. Я пошел, умылся и прозрел.
\\
Тогда сказали ему: где Он? Он отвечал: не знаю.
\\
Повели сего бывшего слепца к фарисеям.
\\
А была суббота, когда Иисус сделал брение и отверз ему очи.
\\
Спросили его также и фарисеи, как он прозрел. Он сказал им: брение положил Он на мои глаза, и я умылся, и вижу.
\\
Тогда некоторые из фарисеев говорили: не от Бога Этот Человек, потому что не хранит субботы. Другие говорили: как может человек грешный творить такие чудеса? И была между ними распря.
\\
Опять говорят слепому: ты что скажешь о Нем, потому что Он отверз тебе очи? Он сказал: это пророк.
\\
Тогда Иудеи не поверили, что он был слеп и прозрел, доколе не призвали родителей сего прозревшего
\\
и спросили их: это ли сын ваш, о котором вы говорите, что родился слепым? как же он теперь видит?
\\
Родители его сказали им в ответ: мы знаем, что это сын наш и что он родился слепым,
\\
а как теперь видит, не знаем, или кто отверз ему очи, мы не знаем. Сам в совершенных летах; самого спросите; пусть сам о себе скажет.
\\
Так отвечали родители его, потому что боялись Иудеев; ибо Иудеи сговорились уже, чтобы, кто признает Его за Христа, того отлучать от синагоги.
\\
Посему-то родители его и сказали: он в совершенных летах; самого спросите.
\\
Итак, вторично призвали человека, который был слеп, и сказали ему: воздай славу Богу; мы знаем, что Человек Тот грешник.
\\
Он сказал им в ответ: грешник ли Он, не знаю; одно знаю, что я был слеп, а теперь вижу.
\\
Снова спросили его: что сделал Он с тобою? как отверз твои очи?
\\
Отвечал им: я уже сказал вам, и вы не слушали; что еще хотите слышать? или и вы хотите сделаться Его учениками?
\\
Они же укорили его и сказали: ты ученик Его, а мы Моисеевы ученики.
\\
Мы знаем, что с Моисеем говорил Бог; Сего же не знаем, откуда Он.
\\
Человек прозревший сказал им в ответ: это и удивительно, что вы не знаете, откуда Он, а Он отверз мне очи.
\\
Но мы знаем, что грешников Бог не слушает; но кто чтит Бога и творит волю Его, того слушает.
\\
От века не слыхано, чтобы кто отверз очи слепорожденному.
\\
Если бы Он не был от Бога, не мог бы творить ничего.
\\
Сказали ему в ответ: во грехах ты весь родился, и ты ли нас учишь? И выгнали его вон.
\\
Иисус, услышав, что выгнали его вон, и найдя его, сказал ему: ты веруешь ли в Сына Божия?
\\
Он отвечал и сказал: а кто Он, Господи, чтобы мне веровать в Него?
\\
Иисус сказал ему: и видел ты Его, и Он говорит с тобою.
\\
Он же сказал: верую, Господи! И поклонился Ему.
\\

\end{otherlanguage*}
\end{minipage}
\hfill
\begin{minipage}[t]{0.49\textwidth}
\fontsize{8}{9.6}\selectfont
\textbf{John 9:1--9:38.}\\
And as Jesus passed by, he saw a man which was blind from his birth.\\
And his disciples asked him, saying, Master, who did sin, this man, or his parents, that he was born blind?\\
Jesus answered, Neither hath this man sinned, nor his parents: but that the works of God should be made manifest in him.\\
I must work the works of him that sent me, while it is day: the night cometh, when no man can work.\\
As long as I am in the world, I am the light of the world.\\
When he had thus spoken, he spat on the ground, and made clay of the spittle, and he anointed the eyes of the blind man with the clay,\\
And said unto him, Go, wash in the pool of Siloam, (which is by interpretation, Sent.) He went his way therefore, and washed, and came seeing.\\
The neighbours therefore, and they which before had seen him that he was blind, said, Is not this he that sat and begged?\\
Some said, This is he: others said, He is like him: but he said, I am he.\\
Therefore said they unto him, How were thine eyes opened?\\
He answered and said, A man that is called Jesus made clay, and anointed mine eyes, and said unto me, Go to the pool of Siloam, and wash: and I went and washed, and I received sight.\\
Then said they unto him, Where is he? He said, I know not.\\
They brought to the Pharisees him that aforetime was blind.\\
And it was the sabbath day when Jesus made the clay, and opened his eyes.\\
Then again the Pharisees also asked him how he had received his sight. He said unto them, He put clay upon mine eyes, and I washed, and do see.\\
Therefore said some of the Pharisees, This man is not of God, because he keepeth not the sabbath day. Others said, How can a man that is a sinner do such miracles? And there was a division among them.\\
They say unto the blind man again, What sayest thou of him, that he hath opened thine eyes? He said, He is a prophet.\\
But the Jews did not believe concerning him, that he had been blind, and received his sight, until they called the parents of him that had received his sight.\\
And they asked them, saying, Is this your son, who ye say was born blind? how then doth he now see?\\
His parents answered them and said, We know that this is our son, and that he was born blind:\\
But by what means he now seeth, we know not; or who hath opened his eyes, we know not: he is of age; ask him: he shall speak for himself.\\
These words spake his parents, because they feared the Jews: for the Jews had agreed already, that if any man did confess that he was Christ, he should be put out of the synagogue.\\
Therefore said his parents, He is of age; ask him.\\
Then again called they the man that was blind, and said unto him, Give God the praise: we know that this man is a sinner.\\
He answered and said, Whether he be a sinner or no, I know not: one thing I know, that, whereas I was blind, now I see.\\
Then said they to him again, What did he to thee? how opened he thine eyes?\\
He answered them, I have told you already, and ye did not hear: wherefore would ye hear it again? will ye also be his disciples?\\
Then they reviled him, and said, Thou art his disciple; but we are Moses' disciples.\\
We know that God spake unto Moses: as for this fellow, we know not from whence he is.\\
The man answered and said unto them, Why herein is a marvellous thing, that ye know not from whence he is, and yet he hath opened mine eyes.\\
Now we know that God heareth not sinners: but if any man be a worshipper of God, and doeth his will, him he heareth.\\
Since the world began was it not heard that any man opened the eyes of one that was born blind.\\
If this man were not of God, he could do nothing.\\
They answered and said unto him, Thou wast altogether born in sins, and dost thou teach us? And they cast him out.\\
Jesus heard that they had cast him out; and when he had found him, he said unto him, Dost thou believe on the Son of God?\\
He answered and said, Who is he, Lord, that I might believe on him?\\
And Jesus said unto him, Thou hast both seen him, and it is he that talketh with thee.\\
And he said, Lord, I believe. And he worshipped him.\\
\end{minipage}}
\vspace*{\fill}
\newpage
\Large%for chi
\vspace*{\fill}
\begin{spacing}{0.93}
\framebox[\textwidth]{
\begin{minipage}[t]{\textwidth}
\begin{CJK}{UTF8}{bsmi}
\textbf{約翰福音 9:1--9:38.}\\
耶穌過去的時候、看見一個人生來是瞎眼的。\\
門徒問耶穌說、拉比、這人生來是瞎眼的、是誰犯了罪、是這人呢、是他父母呢。\\
耶穌回答說、也不是這人犯了罪、也不是他父母犯了罪、是要在他身上顯出 神的作為來。\\
趁著白日、我們必須作那差我來者的工.黑夜將到、就沒有人能作工了。\\
我在世上的時候、是世上的光。\\
耶穌說了這話、就吐唾沫在地上、用唾沫和泥抹在瞎子的眼睛上、\\
對他說、你往西羅亞池子裡去洗、(西羅亞翻出來、就是奉差遣)他去一洗、回頭就看見了。\\
他的鄰舍和那素常見他是討飯的、就說、這不是那從前坐著討飯的人麼。\\
有人說、是他.又有人說、不是、卻是像他.他自己說、是我。\\
他們對他說、你的眼睛是怎麼開的呢。\\
他回答說、有一個人名叫耶穌.他和泥抹我的眼睛、對我說、你往西羅亞池子去洗.我去一洗、就看見了。\\
他們說、那個人在那裡.他說、我不知道。\\
他們把從前瞎眼的人、帶到法利賽人那裡。\\
耶穌和泥開他眼睛的日子是安息日。\\
法利賽人也問他是怎麼得看見的。瞎子對他們說、他把泥抹在我的眼睛上、我去一洗、就看見了。\\
法利賽人中有的說、這個人不是從 神來的、因為他不守安息日。又有人說、一個罪人怎能行這樣的神蹟呢。他們就起了分爭。\\
他們又對瞎子說、他既然開了你的眼睛、你說他是怎樣的人呢。他說、是個先知。\\
猶太人不信他從前是瞎眼、後來能看見的、等到叫了他的父母來、\\
問他們說、這是你們的兒子麼.你們說他生來是瞎眼的、如今怎麼能看見了呢。\\
他父母回答說、他是我們的兒子、生來就瞎眼、這是我們知道的。\\
至於他如今怎麼能看見、我們卻不知道.是誰開了他的眼睛、我們也不知道.他已經成了人、你們問他罷.他自己必能說。\\
他父母說這話、是怕猶太人、因為猶太人已經商議定了、若有認耶穌是基督的、要把他趕出會堂。\\
因此他父母說、他已經成了人、你們問他罷。\\
所以法利賽人第二次叫了那從前瞎眼的人來、對他說、你該將榮耀歸給 神.我們知道這人是個罪人。\\
他說、他是個罪人不是、我不知道.有一件事我知道.從前我是眼瞎的、如今能看見了。\\
他們就問他說、他向你作甚麼、是怎麼開了你的眼睛呢。\\
他回答說、我方纔告訴你們、你們不聽.為甚麼又要聽呢.莫非你們也要作他的門徒麼。\\
他們就罵他說、你是他的門徒.我們是摩西的門徒。\\
 神對摩西說話、是我們知道的.只是這個人、我們不知道他從那裡來。\\
那人回答說、他開了我的眼睛、你們竟不知道他從那裡來、這真是奇怪。\\
我們知道 神不聽罪人.惟有敬奉神遵行他旨意的、 神纔聽他。\\
從創世以來、未曾聽見有人把生來是瞎子的眼睛開了。\\
這人若不是從 神來的、甚麼也不能作。\\
他們回答說、你全然生在罪孽中、還要教訓我們麼。於是把他趕出去了。\\
耶穌聽說他們把他趕出去.後來遇見他、就說、你信 神的兒子麼。\\
他回答說、主阿、誰是 神的兒子、叫我信他呢。\\
耶穌說、你已經看見他、現在和你說話的就是他。\\
他說、主阿、我信.就拜耶穌。\\

\end{CJK}
\end{minipage}}
\end{spacing}
\vspace*{\fill}
\end{document}
