\documentclass[10pt]{article} % use larger type; default would be 10pt
\usepackage[utf8]{inputenc}       % кодування документа; замість cp866nav
\usepackage{extsizes}
\usepackage[top=0.5in,bottom=0.5in,left=0.5in,right=0.5in]{geometry}
\usepackage[russian,english]{babel} % національна локалізація; може бути декілька
\usepackage{setspace}
\usepackage{CJKutf8}
\usepackage{mdframed}
\usepackage{extsizes}
\usepackage{setspace}
\title{Divine Liturgy\\Reading from The Epistles}
\author{Week 24 after the Pentecost\vspace{
-3ex%for author
}}
\date{\vspace{
-5ex%for date
}}
\begin{document}
\pagenumbering{gobble}
\begin{otherlanguage*}{russian}
\maketitle
\end{otherlanguage*}
\vspace*{\fill}
\Large%for eng/rus
\singlespacing %\\onehalfspacing \\doublespacing % for eng/rus
\framebox[\textwidth]{
\begin{minipage}[t]{0.45\textwidth}
\begin{otherlanguage*}{russian}
\textbf{Еф., 221 зач., II, 14-22.}\\
\Large%for eng/rus
Ибо Он есть мир наш, соделавший из обоих одно и разрушивший стоявшую посреди преграду,
\\
упразднив вражду Плотию Своею, а закон заповедей учением, дабы из двух создать в Себе Самом одного нового человека, устрояя мир,
\\
и в одном теле примирить обоих с Богом посредством креста, убив вражду на нем.
\\
И, придя, благовествовал мир вам, дальним и близким,
\\
потому что через Него и те и другие имеем доступ к Отцу, в одном Духе.
\\
Итак вы уже не чужие и не пришельцы, но сограждане святым и свои Богу,
\\
быв утверждены на основании Апостолов и пророков, имея Самого Иисуса Христа краеугольным камнем,
\\
на котором все здание, слагаясь стройно, возрастает в святый храм в Господе,
\\
на котором и вы устрояетесь в жилище Божие Духом.
\\

\end{otherlanguage*}
\end{minipage}
\hfill
\begin{minipage}[t]{0.45\textwidth}

\textbf{Ephesians 2:14--2:22.}\\
For he is our peace, who hath made both one, and hath broken down the middle wall of partition between us;\\
Having abolished in his flesh the enmity, even the law of commandments contained in ordinances; for to make in himself of twain one new man, so making peace;\\
And that he might reconcile both unto God in one body by the cross, having slain the enmity thereby:\\
And came and preached peace to you which were afar off, and to them that were nigh.\\
For through him we both have access by one Spirit unto the Father.\\
Now therefore ye are no more strangers and foreigners, but fellowcitizens with the saints, and of the household of God;\\
And are built upon the foundation of the apostles and prophets, Jesus Christ himself being the chief corner stone;\\
In whom all the building fitly framed together groweth unto an holy temple in the Lord:\\
In whom ye also are builded together for an habitation of God through the Spirit.\\

\end{minipage}}
\vspace*{\fill}
\newpage
\Huge%for chi
\vspace*{\fill}
\begin{spacing}{1.3}%for chinese
\framebox[\textwidth]{
\begin{minipage}[t]{\textwidth}
\begin{CJK}{UTF8}{bsmi}
\textbf{以弗所書 2:14--2:22.}\\
因他使我們和睦、〔原文作因他是我們的和睦〕將兩下合而為一、拆毀了中間隔斷的牆.\\
而且以自己的身體、廢掉冤仇、就是那記在律法上的規條.為要將兩下、藉著自己造成一個新人、如此便成就了和睦.\\
既在十字架上滅了冤仇、便藉這十字架、使兩下歸為一體、與 神和好了.\\
並且來傳和平的福音給你們遠處的人、也給那近處的人。\\
因為我們兩下藉著他被一個聖靈所感、得以進到父面前。\\
這樣、你們不再作外人、和客旅、是與聖徒同國、是 神家裡的人了.\\
並且被建造在使徒和先知的根基上、有基督耶穌自己為房角石.\\
各〔或作全〕房靠他聯絡得合式、漸漸成為主的聖殿.\\
你們也靠他同被建造成為 神藉著聖靈居住的所在。\\

\end{CJK}
\end{minipage}}
\end{spacing}
\vspace*{\fill}
\end{document}
