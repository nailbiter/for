\documentclass[10pt]{article} % use larger type; default would be 10pt
\usepackage[utf8]{inputenc}       % кодування документа; замість cp866nav
\usepackage[margin=0.5in]{geometry}
\usepackage[russian,english]{babel} % національна локалізація; може бути декілька
\usepackage{setspace}
\usepackage{CJKutf8}
\usepackage{mdframed}
\usepackage{setspace}
\title{Divine Liturgy\\Reading from The Epistles}
\author{Week about the Last Judgement\vspace{
-3ex%for author
}}
\date{\vspace{
-5ex%for date
}}
\begin{document}
\pagenumbering{gobble}
\begin{otherlanguage*}{russian}
\maketitle
\end{otherlanguage*}
\vspace*{\fill}
\Large%for eng/rus
\singlespacing %\\onehalfspacing \\doublespacing % for eng/rus
\framebox[\textwidth]{
\begin{minipage}[t]{0.45\textwidth}
\begin{otherlanguage*}{russian}
\textbf{1 Кор., 140 зач., VIII, 8 - IX, 2.}\\
\Large%for eng/rus
Пища не приближает нас к Богу: ибо, едим ли мы, ничего не приобретаем; не едим ли, ничего не теряем.\\
Берегитесь однако же, чтобы эта свобода ваша не послужила соблазном для немощных.\\
Ибо если кто-нибудь увидит, что ты, имея знание, сидишь за столом в капище, то совесть его, как немощного, не расположит ли и его есть идоложертвенное?\\
И от знания твоего погибнет немощный брат, за которого умер Христос.\\
А согрешая таким образом против братьев и уязвляя немощную совесть их, вы согрешаете против Христа.\\
И потому, если пища соблазняет брата моего, не буду есть мяса вовек, чтобы не соблазнить брата моего. \\
Не Апостол ли я? Не свободен ли я? Не видел ли я Иисуса Христа, Господа нашего? Не мое ли дело вы в Господе?\\
Если для других я не Апостол, то для вас Апостол; ибо печать моего апостольства - вы в Господе. \\
\end{otherlanguage*}
\end{minipage}
\hfill
\begin{minipage}[t]{0.45\textwidth}

\textbf{1 Corinthians 8:8 -- 9:2.}\\
But meat commendeth us not to God: for neither, if we eat, are we the better; neither, if we eat not, are we the worse.\\
But take heed lest by any means this liberty of yours become a stumblingblock to them that are weak.\\
For if any man see thee which hast knowledge sit at meat in the idol's temple, shall not the conscience of him which is weak be emboldened to eat those things which are offered to idols;\\
And through thy knowledge shall the weak brother perish, for whom Christ died?\\
But when ye sin so against the brethren, and wound their weak conscience, ye sin against Christ.\\
Wherefore, if meat make my brother to offend, I will eat no flesh while the world standeth, lest I make my brother to offend.\\
Am I not an apostle? am I not free? have I not seen Jesus Christ our Lord? are not ye my work in the Lord?\\
If I be not an apostle unto others, yet doubtless I am to you: for the seal of mine apostleship are ye in the Lord.\\

\end{minipage}}
\vspace*{\fill}
\newpage
\Huge%for chi
\vspace*{\fill}
\begin{spacing}{1.0}%for chinese
\framebox[\textwidth]{
\begin{minipage}[t]{\textwidth}
\begin{CJK}{UTF8}{bsmi}
\textbf{哥林多前書 8:8 -- 9:2.}\\
其實食物不能叫 神看中我們.因為我們不喫也無損、喫也無益。\\
只是你們要謹慎、恐怕你們這自由、竟成了那軟弱人的絆腳石。\\
若有人見你這有知識的、在偶像的廟裡坐席、這人的良心、若是軟弱、豈不放膽去喫那祭偶像之物麼。\\
因此、基督為他死的那軟弱弟兄、也就因你的知識沉淪了。\\
你們這樣得罪弟兄們、傷了他們軟弱的良心、就是得罪基督。\\
所以食物若叫我弟兄跌倒、我就永遠不喫肉、免得叫我弟兄跌倒了。\\
我不是自由的麼.我不是使徒麼.我不是見過我們的主耶穌麼.你們不是我在主裡面所作之工麼。\\
假若在別人我不是使徒、在你們我總是使徒.因為你們在主裡正是我作使徒的印證。\\

\end{CJK}
\end{minipage}}
\end{spacing}
\vspace*{\fill}
\end{document}
