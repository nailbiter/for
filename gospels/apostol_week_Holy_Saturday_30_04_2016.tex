%simple
\documentclass[10pt]{article} % use larger type; default would be 10pt
\usepackage[utf8]{inputenc}       % кодування документа; замість cp866nav
\usepackage{extsizes}
\usepackage[top=0.3in,bottom=0.3in,left=0.3in,right=0.5in]{geometry}
\usepackage[russian,english]{babel} % національна локалізація; може бути декілька
\usepackage{setspace}
\usepackage{CJKutf8}
\usepackage{mdframed}
\usepackage{extsizes}
\usepackage{setspace}
\title{Divine Liturgy\\Reading from The Epistles}
\author{Week Holy Saturday\vspace{
-3ex%for author
}}
\date{\vspace{
-5ex%for date
}}
\begin{document}
\pagenumbering{gobble}
\begin{otherlanguage*}{russian}
\maketitle
\end{otherlanguage*}
\vspace*{\fill}
\Large%for eng/rus
\singlespacing %\\onehalfspacing \\doublespacing % for eng/rus
\framebox[\textwidth]{
\begin{minipage}[t]{0.45\textwidth}
\begin{otherlanguage*}{russian}
\textbf{Рим., 91 зач., VI, 3-11.   }\\
\Large%for eng/rus
Неужели не знаете, что все мы, крестившиеся во Христа Иисуса, в смерть Его крестились?
\\
Итак мы погреблись с Ним крещением в смерть, дабы, как Христос воскрес из мертвых славою Отца, так и нам ходить в обновленной жизни.
\\
Ибо если мы соединены с Ним подобием смерти Его, то должны быть соединены и подобием воскресения,
\\
зная то, что ветхий наш человек распят с Ним, чтобы упразднено было тело греховное, дабы нам не быть уже рабами греху;
\\
ибо умерший освободился от греха.
\\
Если же мы умерли со Христом, то веруем, что и жить будем с Ним,
\\
зная, что Христос, воскреснув из мертвых, уже не умирает: смерть уже не имеет над Ним власти.
\\
Ибо, что Он умер, то умер однажды для греха; а что живет, то живет для Бога.
\\
Так и вы почитайте себя мертвыми для греха, живыми же для Бога во Христе Иисусе, Господе нашем.
\\

\end{otherlanguage*}
\end{minipage}
\hfill
\begin{minipage}[t]{0.45\textwidth}

\textbf{Romans 6:3--6:11.}\\
Know ye not, that so many of us as were baptized into Jesus Christ were baptized into his death?\\
Therefore we are buried with him by baptism into death: that like as Christ was raised up from the dead by the glory of the Father, even so we also should walk in newness of life.\\
For if we have been planted together in the likeness of his death, we shall be also in the likeness of his resurrection:\\
Knowing this, that our old man is crucified with him, that the body of sin might be destroyed, that henceforth we should not serve sin.\\
For he that is dead is freed from sin.\\
Now if we be dead with Christ, we believe that we shall also live with him:\\
Knowing that Christ being raised from the dead dieth no more; death hath no more dominion over him.\\
For in that he died, he died unto sin once: but in that he liveth, he liveth unto God.\\
Likewise reckon ye also yourselves to be dead indeed unto sin, but alive unto God through Jesus Christ our Lord.\\

\end{minipage}}
\vspace*{\fill}
\newpage
\Huge%for chi
\vspace*{\fill}
\begin{spacing}{1.3}%for chinese
\framebox[\textwidth]{
\begin{minipage}[t]{\textwidth}
\begin{CJK}{UTF8}{bsmi}
\textbf{羅馬書 6:3--6:11.}\\
豈不知我們這受洗歸入基督耶穌的人、是受洗歸入他的死麼。\\
所以我們藉著洗禮歸入死、和他一同埋葬.原是叫我們一舉一動有新生的樣式、像基督藉著父的榮耀、從死裡復活一樣。\\
我們若在他死的形狀上與他聯合、也要在他復活的形狀上與他聯合.\\
因為知道我們的舊人、和他同釘十字架、使罪身滅絕、叫我們不再作罪的奴僕.\\
因為已死的人、是脫離了罪。\\
我們若是與基督同死、就信必與他同活.\\
因為知道基督既從死裡復活、就不再死、死也不再作他的主了。\\
他死是向罪死了、只有一次.他活是向 神活著。\\
這樣、你們向罪也當看自己是死的.向 神在基督耶穌裡、卻當看自己是活的。\\

\end{CJK}
\end{minipage}}
\end{spacing}
\vspace*{\fill}
\end{document}
