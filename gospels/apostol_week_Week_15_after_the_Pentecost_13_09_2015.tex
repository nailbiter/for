\documentclass[10pt]{article} % use larger type; default would be 10pt
\usepackage[utf8]{inputenc}       % кодування документа; замість cp866nav
\usepackage{extsizes}
\usepackage[top=0.5in,bottom=0.5in,left=0.5in,right=0.5in]{geometry}
\usepackage[russian,english]{babel} % національна локалізація; може бути декілька
\usepackage{setspace}
\usepackage{CJKutf8}
\usepackage{mdframed}
\usepackage{extsizes}
\usepackage{setspace}
\title{Divine Liturgy\\Reading from The Epistles}
\author{Week Week 15 after the Pentecost\vspace{
-3ex%for author
}}
\date{\vspace{
-5ex%for date
}}
\begin{document}
\pagenumbering{gobble}
\begin{otherlanguage*}{russian}
\maketitle
\end{otherlanguage*}
\vspace*{\fill}
\Large%for eng/rus
\singlespacing %\\onehalfspacing \\doublespacing % for eng/rus
\framebox[\textwidth]{
\begin{minipage}[t]{0.45\textwidth}
\begin{otherlanguage*}{russian}
\textbf{2 Кор., 176 зач., IV, 6-15.}\\
\Large%for eng/rus
потому что Бог, повелевший из тьмы воссиять свету, озарил наши сердца, дабы просветить нас познанием славы Божией в лице Иисуса Христа.
\\
Но сокровище сие мы носим в глиняных сосудах, чтобы преизбыточная сила была приписываема Богу, а не нам.
\\
Мы отовсюду притесняемы, но не стеснены; мы в отчаянных обстоятельствах, но не отчаиваемся;
\\
мы гонимы, но не оставлены; низлагаемы, но не погибаем.
\\
Всегда носим в теле мертвость Господа Иисуса, чтобы и жизнь Иисусова открылась в теле нашем.
\\
Ибо мы живые непрестанно предаемся на смерть ради Иисуса, чтобы и жизнь Иисусова открылась в смертной плоти нашей,
\\
так что смерть действует в нас, а жизнь в вас.
\\
Но, имея тот же дух веры, как написано: я веровал и потому говорил, и мы веруем, потому и говорим,
\\
зная, что Воскресивший Господа Иисуса воскресит через Иисуса и нас и поставит перед Собою с вами.
\\
Ибо всё для вас, дабы обилие благодати тем большую во многих произвело благодарность во славу Божию.
\\

\end{otherlanguage*}
\end{minipage}
\hfill
\begin{minipage}[t]{0.45\textwidth}

\textbf{2 Corinthians 4:6--4:15.}\\
For God, who commanded the light to shine out of darkness, hath shined in our hearts, to give the light of the knowledge of the glory of God in the face of Jesus Christ.\\
But we have this treasure in earthen vessels, that the excellency of the power may be of God, and not of us.\\
We are troubled on every side, yet not distressed; we are perplexed, but not in despair;\\
Persecuted, but not forsaken; cast down, but not destroyed;\\
Always bearing about in the body the dying of the Lord Jesus, that the life also of Jesus might be made manifest in our body.\\
For we which live are alway delivered unto death for Jesus' sake, that the life also of Jesus might be made manifest in our mortal flesh.\\
So then death worketh in us, but life in you.\\
We having the same spirit of faith, according as it is written, I believed, and therefore have I spoken; we also believe, and therefore speak;\\
Knowing that he which raised up the Lord Jesus shall raise up us also by Jesus, and shall present us with you.\\
For all things are for your sakes, that the abundant grace might through the thanksgiving of many redound to the glory of God.\\

\end{minipage}}
\vspace*{\fill}
\newpage
\Huge%for chi
\vspace*{\fill}
\begin{spacing}{1.3}%for chinese
\framebox[\textwidth]{
\begin{minipage}[t]{\textwidth}
\begin{CJK}{UTF8}{bsmi}
\textbf{哥林多後書 4:6--4:15.}\\
那吩咐光從黑暗裡照出來的 神、已經照在我們心裡、叫我們得知 神榮耀的光、顯在耶穌基督的面上。\\
我們有這寶貝放在瓦器裡、要顯明這莫大的能力、是出於 神、不是出於我們。\\
我們四面受敵、卻不被困住.心裡作難、卻不至失望.\\
遭逼迫、卻不被丟棄.打倒了、卻不至死亡.\\
身上常帶著耶穌的死、使耶穌的生、也顯明在我們身上。\\
因為我們這活著的人、是常為耶穌被交於死地、使耶穌的生、在我們這必死的身上顯明出來。\\
這樣看來、死是在我們身上發動、生卻在你們身上發動。\\
但我們既有信心、正如經上記著說、『我因信、所以如此說話。』我們也信、所以也說話.\\
自己知道、那叫主耶穌復活的、也必叫我們與耶穌一同復活、並且叫我們與你們一同站在他面前。\\
凡事都是為你們、好叫恩惠因人多越發加增、感謝格外顯多、以致榮耀歸與 神。\\

\end{CJK}
\end{minipage}}
\end{spacing}
\vspace*{\fill}
\end{document}
