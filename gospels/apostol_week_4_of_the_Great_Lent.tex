\documentclass[10pt]{article} % use larger type; default would be 10pt
\usepackage[utf8]{inputenc}       % кодування документа; замість cp866nav
\usepackage[margin=0.5in]{geometry}
\usepackage[russian,english]{babel} % національна локалізація; може бути декілька
\usepackage{setspace}
\usepackage{CJKutf8}
\usepackage{mdframed}
\usepackage{setspace}
\title{Divine Liturgy\\Reading from The Epistles}
\author{Week 4 of the Great Lent\vspace{
-3ex%for author
}}
\date{\vspace{
-5ex%for date
}}
\begin{document}
\pagenumbering{gobble}
\begin{otherlanguage*}{russian}
\maketitle
\end{otherlanguage*}
\vspace*{\fill}
\Large%for eng/rus
\singlespacing %\\onehalfspacing \\doublespacing % for eng/rus
\framebox[\textwidth]{
\begin{minipage}[t]{0.45\textwidth}
\begin{otherlanguage*}{russian}
\textbf{Евр., 314 зач., VI, 13-20.}\\
\Large%for eng/rus
Бог, давая обетование Аврааму, как не мог никем высшим клясться, клялся Самим Собою,\\
говоря: истинно благословляя благословлю тебя и размножая размножу тебя.\\
И так Авраам, долготерпев, получил обещанное.\\
Люди клянутся высшим, и клятва во удостоверение оканчивает всякий спор их.\\
Посему и Бог, желая преимущественнее показать наследникам обетования непреложность Своей воли, употребил в посредство клятву,\\
дабы в двух непреложных вещах, в которых невозможно Богу солгать, твердое утешение имели мы, прибегшие взяться за предлежащую надежду,\\
которая для души есть как бы якорь безопасный и крепкий, и входит во внутреннейшее за завесу,\\
куда предтечею за нас вошел Иисус, сделавшись Первосвященником навек по чину Мелхиседека. \\
\end{otherlanguage*}
\end{minipage}
\hfill
\begin{minipage}[t]{0.45\textwidth}

\textbf{Hebrews 6:13--6:20.}\\
For when God made promise to Abraham, because he could swear by no greater, he sware by himself,\\
Saying, Surely blessing I will bless thee, and multiplying I will multiply thee.\\
And so, after he had patiently endured, he obtained the promise.\\
For men verily swear by the greater: and an oath for confirmation is to them an end of all strife.\\
Wherein God, willing more abundantly to shew unto the heirs of promise the immutability of his counsel, confirmed it by an oath:\\
That by two immutable things, in which it was impossible for God to lie, we might have a strong consolation, who have fled for refuge to lay hold upon the hope set before us:\\
Which hope we have as an anchor of the soul, both sure and stedfast, and which entereth into that within the veil;\\
Whither the forerunner is for us entered, even Jesus, made an high priest for ever after the order of Melchisedec.\\

\end{minipage}}
\vspace*{\fill}
\newpage
\Huge%for chi
\vspace*{\fill}
\begin{spacing}{1.0}%for chinese
\framebox[\textwidth]{
\begin{minipage}[t]{\textwidth}
\begin{CJK}{UTF8}{bsmi}
\textbf{希伯來書 6:13--6:20.}\\
當初 神應許亞伯拉罕的時候、因為沒有比自己更大可以指著起誓的、就指著自己起誓、說、\\
『論福、我必賜大福給你.論子孫、我必叫你的子孫多起來。』\\
這樣、亞伯拉罕既恆久忍耐、就得了所應許的。\\
人都是指著比自己大的起誓.並且以起誓為實據、了結各樣的爭論。\\
照樣、 神願意為那承受應許的人、格外顯明他的旨意是不更改的、就起誓為證.\\
藉這兩件不更改的事、 神決不能說謊、好叫我們這逃往避難所、持定擺在我們前頭指望的人、可以大得勉勵.\\
我們有這指望如同靈魂的錨、又堅固又牢靠、且通入幔內。\\
作先鋒的耶穌、既照著麥基洗德的等次、成了永遠的大祭司、就為我們進入幔內。\\

\end{CJK}
\end{minipage}}
\end{spacing}
\vspace*{\fill}
\end{document}
