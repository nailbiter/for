\documentclass[10pt]{article} % use larger type; default would be 10pt
\usepackage[utf8]{inputenc}       % кодування документа; замість cp866nav
\usepackage[top=0.5in,bottom=0.5in,left=0.5in,right=0.5in]{geometry}
\usepackage[russian,english]{babel} % національна локалізація; може бути декілька
\usepackage{setspace}
\usepackage{CJKutf8}
\usepackage{mdframed}
\usepackage{setspace}
\title{Divine Liturgy\\Reading from The Epistles}
\author{Week 3 of the Great Lent\vspace{
-3ex%for author
}}
\date{\vspace{
-5ex%for date
}}
\begin{document}
\pagenumbering{gobble}
\begin{otherlanguage*}{russian}
\maketitle
\end{otherlanguage*}
\vspace*{\fill}
\Large%for eng/rus
\singlespacing %\\onehalfspacing \\doublespacing % for eng/rus
\framebox[\textwidth]{
\begin{minipage}[t]{0.45\textwidth}
\begin{otherlanguage*}{russian}
\textbf{Евр., 311 зач., IV, 15 - V, 6.}\\
\Large%for eng/rus
Ибо мы имеем не такого первосвященника, который не может сострадать нам в немощах наших, но Который, подобно нам, искушен во всем, кроме греха.
\\
Посему да приступаем с дерзновением к престолу благодати, чтобы получить милость и обрести благодать для благовременной помощи.
\\
Ибо всякий первосвященник, из человеков избираемый, для человеков поставляется на служение Богу, чтобы приносить дары и жертвы за грехи,
\\
могущий снисходить невежествующим и заблуждающим, потому что и сам обложен немощью,
\\
и посему он должен как за народ, так и за себя приносить жертвы о грехах.
\\
И никто сам собою не приемлет этой чести, но призываемый Богом, как и Аарон.
\\
Так и Христос не Сам Себе присвоил славу быть первосвященником, но Тот, Кто сказал Ему: Ты Сын Мой, Я ныне родил Тебя;
\\
как и в другом месте говорит: Ты священник вовек по чину Мелхиседека.
\\

\end{otherlanguage*}
\end{minipage}
\hfill
\begin{minipage}[t]{0.45\textwidth}

\textbf{Hebrews 4:15--5:6.}\\
For we have not an high priest which cannot be touched with the feeling of our infirmities; but was in all points tempted like as we are, yet without sin.\\
Let us therefore come boldly unto the throne of grace, that we may obtain mercy, and find grace to help in time of need.\\
For every high priest taken from among men is ordained for men in things pertaining to God, that he may offer both gifts and sacrifices for sins:\\
Who can have compassion on the ignorant, and on them that are out of the way; for that he himself also is compassed with infirmity.\\
And by reason hereof he ought, as for the people, so also for himself, to offer for sins.\\
And no man taketh this honour unto himself, but he that is called of God, as was Aaron.\\
So also Christ glorified not himself to be made an high priest; but he that said unto him, Thou art my Son, to day have I begotten thee.\\
As he saith also in another place, Thou art a priest for ever after the order of Melchisedec.\\

\end{minipage}}
\vspace*{\fill}
\newpage
\Huge%for chi
\vspace*{\fill}
\begin{spacing}{1.0}%for chinese
\framebox[\textwidth]{
\begin{minipage}[t]{\textwidth}
\begin{CJK}{UTF8}{bsmi}
\textbf{希伯來書 4:15--5:6.}\\
因我們的大祭司、並非不能體恤我們的軟弱.他也曾凡事受過試探、與我們一樣.只是他沒有犯罪。\\
所以我們只管坦然無懼的、來到施恩的寶座前、為要得憐恤、蒙恩惠作隨時的幫助。\\
凡從人間挑選的大祭司、是奉派替人辦理屬 神的事、為要獻上禮物、和贖罪祭.〔或作要為罪獻上禮物和祭物〕\\
他能體諒那愚蒙的、和失迷的人、因為他自己也是被軟弱所困.\\
故此他理當為百姓和自己獻祭贖罪。\\
這大祭司的尊榮、沒有人自取、惟要蒙 神所召、像亞倫一樣。\\
如此、基督也不是自取榮耀作大祭司、乃是在乎向他說『你是我的兒子、我今日生你。』的那一位.\\
就如經上又有一處說、『你是照著麥基洗德的等次永遠為祭司。』\\

\end{CJK}
\end{minipage}}
\end{spacing}
\vspace*{\fill}
\end{document}
