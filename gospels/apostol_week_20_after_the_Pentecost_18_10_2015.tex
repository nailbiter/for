\documentclass[10pt]{article} % use larger type; default would be 10pt
\usepackage[utf8]{inputenc}       % кодування документа; замість cp866nav
\usepackage{extsizes}
\usepackage[top=0.2in,bottom=0.3in,left=0.0in,right=0.5in]{geometry}
\usepackage[russian,english]{babel} % національна локалізація; може бути декілька
\usepackage{setspace}
\usepackage{CJKutf8}
\usepackage{mdframed}
\usepackage{extsizes}
\usepackage{setspace}
\title{Divine Liturgy\\Reading from The Epistles}
\author{Week 20 after the Pentecost\vspace{
-3ex%for author
}}
\date{\vspace{
-5ex%for date
}}
\begin{document}
\pagenumbering{gobble}
\begin{otherlanguage*}{russian}
\maketitle
\end{otherlanguage*}
\vspace*{\fill}
\Large%for eng/rus
\singlespacing %\\onehalfspacing \\doublespacing % for eng/rus
\framebox[\textwidth]{
\begin{minipage}[t]{0.45\textwidth}
\begin{otherlanguage*}{russian}
\textbf{Гал., 200 зач., I, 11-19.}\\
\Large%for eng/rus
Возвещаю вам, братия, что Евангелие, которое я благовествовал, не есть человеческое,
\\
ибо и я принял его и научился не от человека, но через откровение Иисуса Христа.
\\
Вы слышали о моем прежнем образе жизни в Иудействе, что я жестоко гнал Церковь Божию, и опустошал ее,
\\
и преуспевал в Иудействе более многих сверстников в роде моем, будучи неумеренным ревнителем отеческих моих преданий.
\\
Когда же Бог, избравший меня от утробы матери моей и призвавший благодатью Своею, благоволил
\\
открыть во мне Сына Своего, чтобы я благовествовал Его язычникам,- я не стал тогда же советоваться с плотью и кровью,
\\
и не пошел в Иерусалим к предшествовавшим мне Апостолам, а пошел в Аравию, и опять возвратился в Дамаск.
\\
Потом, спустя три года, ходил я в Иерусалим видеться с Петром и пробыл у него дней пятнадцать.
\\
Другого же из Апостолов я не видел никого, кроме Иакова, брата Господня.
\\

\end{otherlanguage*}
\end{minipage}
\hfill
\begin{minipage}[t]{0.45\textwidth}

\textbf{Galatians 1:11--1:19.}\\
But I certify you, brethren, that the gospel which was preached of me is not after man.\\
For I neither received it of man, neither was I taught it, but by the revelation of Jesus Christ.\\
For ye have heard of my conversation in time past in the Jews' religion, how that beyond measure I persecuted the church of God, and wasted it:\\
And profited in the Jews' religion above many my equals in mine own nation, being more exceedingly zealous of the traditions of my fathers.\\
But when it pleased God, who separated me from my mother's womb, and called me by his grace,\\
To reveal his Son in me, that I might preach him among the heathen; immediately I conferred not with flesh and blood:\\
Neither went I up to Jerusalem to them which were apostles before me; but I went into Arabia, and returned again unto Damascus.\\
Then after three years I went up to Jerusalem to see Peter, and abode with him fifteen days.\\
But other of the apostles saw I none, save James the Lord's brother.\\

\end{minipage}}
\vspace*{\fill}
\newpage
\Huge%for chi
\vspace*{\fill}
\begin{spacing}{1.3}%for chinese
\framebox[\textwidth]{
\begin{minipage}[t]{\textwidth}
\begin{CJK}{UTF8}{bsmi}
\textbf{加拉太書 1:11--1:19.}\\
弟兄們、我告訴你們、我素來所傳的福音、不是出於人的意思。\\
因為我不是從人領受的、也不是人教導我的、乃是從耶穌基督啟示來的。\\
你們聽見我從前在猶太教中所行的事、怎樣極力逼迫殘害 神的教會。\\
我又在猶太教中、比我本國許多同歲的人更有長進、為我祖宗的遺傳更加熱心。\\
然而那把我從母腹裡分別出來、又施恩召我的 神、\\
既然樂意將他兒子啟示在我心裡、叫我把他傳在外邦人中、我就沒有與屬血氣的人商量、\\
也沒有上耶路撒冷去、見那些比我先作使徒的.惟獨往亞拉伯去.後又回到大馬色。\\
過了三年、纔上耶路撒冷去見磯法、和他同住了十五天。\\
至於別的使徒、除了主的兄弟雅各、我都沒有看見。
\end{CJK}
\end{minipage}}
\end{spacing}
\vspace*{\fill}
\end{document}
