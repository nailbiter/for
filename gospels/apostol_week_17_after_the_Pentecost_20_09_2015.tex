\documentclass[10pt]{article} % use larger type; default would be 10pt
\usepackage[utf8]{inputenc}       % кодування документа; замість cp866nav
\usepackage{extsizes}
\usepackage[top=0.5in,bottom=0.5in,left=0.5in,right=0.5in]{geometry}
\usepackage[russian,english]{babel} % національна локалізація; може бути декілька
\usepackage{setspace}
\usepackage{CJKutf8}
\usepackage{mdframed}
\usepackage{extsizes}
\usepackage{setspace}
\title{Divine Liturgy\\Reading from The Epistles}
\author{Week 17 after the Pentecost\vspace{
-3ex%for author
}}
\date{\vspace{
-5ex%for date
}}
\begin{document}
\pagenumbering{gobble}
\begin{otherlanguage*}{russian}
\maketitle
\end{otherlanguage*}
\vspace*{\fill}
\Large%for eng/rus
\singlespacing %\\onehalfspacing \\doublespacing % for eng/rus
\framebox[\textwidth]{
\begin{minipage}[t]{0.45\textwidth}
\begin{otherlanguage*}{russian}
\textbf{1 Кор., 125 зач., I, 18-24.}\\
\Large%for eng/rus
Ибо слово о кресте для погибающих юродство есть, а для нас, спасаемых,- сила Божия.
\\
Ибо написано: погублю мудрость мудрецов, и разум разумных отвергну.
\\
Где мудрец? где книжник? где совопросник века сего? Не обратил ли Бог мудрость мира сего в безумие?
\\
Ибо когда мир своею мудростью не познал Бога в премудрости Божией, то благоугодно было Богу юродством проповеди спасти верующих.
\\
Ибо и Иудеи требуют чудес, и Еллины ищут мудрости;
\\
а мы проповедуем Христа распятого, для Иудеев соблазн, а для Еллинов безумие,
\\
для самих же призванных, Иудеев и Еллинов, Христа, Божию силу и Божию премудрость;
\\

\end{otherlanguage*}
\end{minipage}
\hfill
\begin{minipage}[t]{0.45\textwidth}

\textbf{1 Corinthians 1:18--1:24.}\\
For the preaching of the cross is to them that perish foolishness; but unto us which are saved it is the power of God.\\
For it is written, I will destroy the wisdom of the wise, and will bring to nothing the understanding of the prudent.\\
Where is the wise? where is the scribe? where is the disputer of this world? hath not God made foolish the wisdom of this world?\\
For after that in the wisdom of God the world by wisdom knew not God, it pleased God by the foolishness of preaching to save them that believe.\\
For the Jews require a sign, and the Greeks seek after wisdom:\\
But we preach Christ crucified, unto the Jews a stumblingblock, and unto the Greeks foolishness;\\
But unto them which are called, both Jews and Greeks, Christ the power of God, and the wisdom of God.\\

\end{minipage}}
\vspace*{\fill}
\newpage
\Huge%for chi
\vspace*{\fill}
\begin{spacing}{1.3}%for chinese
\framebox[\textwidth]{
\begin{minipage}[t]{\textwidth}
\begin{CJK}{UTF8}{bsmi}
\textbf{哥林多前書 1:18--1:24.}\\
因為十字架的道理、在那滅亡的人為愚拙.在我們得救的人卻為 神的大能。\\
就如經上所記、『我要滅絕智慧人的智慧、廢棄聰明人的聰明。』\\
智慧人在那裡.文士在那裡.這世上的辯士在那裡. 神豈不是叫這世上的智慧變成愚拙麼。\\
世人憑自己的智慧、既不認識 神、 神就樂意用人所當作愚拙的道理、拯救那些信的人.這就是 神的智慧了。\\
猶太人是要神蹟、希利尼人是求智慧.\\
我們卻是傳釘十字架的基督、在猶太人為絆腳石、在外邦人為愚拙、\\
但在那蒙召的、無論是猶太人、希利尼人、基督總為 神的能力、 神的智慧。\\

\end{CJK}
\end{minipage}}
\end{spacing}
\vspace*{\fill}
\end{document}
