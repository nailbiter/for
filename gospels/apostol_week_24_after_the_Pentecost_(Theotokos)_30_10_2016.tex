%simple
\documentclass[10pt]{article} % use larger type; default would be 10pt
\usepackage[utf8]{inputenc}       % кодування документа; замість cp866nav
\usepackage{extsizes}
\usepackage[top=0.5in,bottom=0.5in,left=0.5in,right=0.5in]{geometry}
\usepackage[russian,english]{babel} % національна локалізація; може бути декілька
\usepackage{setspace}
\usepackage{CJKutf8}
\usepackage{mdframed}
\usepackage{extsizes}
\usepackage{setspace}
\title{Divine Liturgy\\Reading from The Epistles}
\author{Week 24 after the Pentecost (Theotokos)\vspace{
-3ex%for author
}}
\date{\vspace{
-5ex%for date
}}
\begin{document}
\pagenumbering{gobble}
\begin{otherlanguage*}{russian}
\maketitle
\end{otherlanguage*}
\vspace*{\fill}
\Large%for eng/rus
\singlespacing %\\onehalfspacing \\doublespacing % for eng/rus
\framebox[\textwidth]{
\begin{minipage}[t]{0.45\textwidth}
\begin{otherlanguage*}{russian}
\textbf{Евр., 320 зач., IX, 1-7.}\\
\Large%for eng/rus
И первый завет имел постановление о Богослужении и святилище земное:
\\
ибо устроена была скиния первая, в которой был светильник, и трапеза, и предложение хлебов, и которая называется Святое.
\\
За второю же завесою была скиния, называемая Святое Святых,
\\
имевшая золотую кадильницу и обложенный со всех сторон золотом ковчег завета, где были золотой сосуд с манною, жезл Ааронов расцветший и скрижали завета,
\\
а над ним херувимы славы, осеняющие очистилище; о чем не нужно теперь говорить подробно.
\\
При таком устройстве, в первую скинию всегда входят священники совершать Богослужение;
\\
а во вторую - однажды в год один только первосвященник, не без крови, которую приносит за себя и за грехи неведения народа.
\\

\end{otherlanguage*}
\end{minipage}
\hfill
\begin{minipage}[t]{0.45\textwidth}

\textbf{Hebrews 9:1--9:7.}\\
Then verily the first covenant had also ordinances of divine service, and a worldly sanctuary.\\
For there was a tabernacle made; the first, wherein was the candlestick, and the table, and the shewbread; which is called the sanctuary.\\
And after the second veil, the tabernacle which is called the Holiest of all;\\
Which had the golden censer, and the ark of the covenant overlaid round about with gold, wherein was the golden pot that had manna, and Aaron's rod that budded, and the tables of the covenant;\\
And over it the cherubims of glory shadowing the mercyseat; of which we cannot now speak particularly.\\
Now when these things were thus ordained, the priests went always into the first tabernacle, accomplishing the service of God.\\
But into the second went the high priest alone once every year, not without blood, which he offered for himself, and for the errors of the people:\\

\end{minipage}}
\vspace*{\fill}
\newpage
\Huge%for chi
\vspace*{\fill}
\begin{spacing}{1.3}%for chinese
\framebox[\textwidth]{
\begin{minipage}[t]{\textwidth}
\begin{CJK}{UTF8}{bsmi}
\textbf{希伯來書 9:1--9:7.}\\
原來前約有禮拜的條例、和屬世界的聖幕。\\
因為有預備的帳幕、頭一層叫作聖所.裡面有燈臺、桌子、和陳設餅。\\
第二幔子後、又有一層帳幕、叫作至聖所.\\
有金香爐、〔爐或作壇〕有包金的約櫃、櫃裡有盛嗎哪的金罐、和亞倫發過芽的杖、並兩塊約版.\\
櫃上面有榮耀基路伯的影罩著施恩座.〔施恩原文作蔽罪〕這幾件我現在不能一一細說。\\
這些物件既如此預備齊了、眾祭司就常進頭一層帳幕、行拜 神的禮.\\
至於第二層帳幕、惟有大祭司一年一次獨自進去、沒有不帶著血、為自己和百姓的過錯獻上.\\

\end{CJK}
\end{minipage}}
\end{spacing}
\vspace*{\fill}
\end{document}
