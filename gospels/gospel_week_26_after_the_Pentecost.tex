\documentclass[10pt]{article} % use larger type; default would be 10pt
\usepackage[utf8]{inputenc}       % кодування документа; замість cp866nav
\usepackage{extsizes}
\usepackage[top=0.5in,bottom=0.5in,left=0.5in,right=0.5in]{geometry}
\usepackage[russian,english]{babel} % національна локалізація; може бути декілька
\usepackage{setspace}
\usepackage{CJKutf8}
\usepackage{mdframed}
\title{Divine Liturgy\\Reading from The Gospels}
\author{Week 26 after the Pentecost\vspace{
-3ex%for author
}}
\date{\vspace{
-7ex%for date
}}
\begin{document}
\pagenumbering{gobble}
\begin{otherlanguage*}{russian}
\maketitle
\end{otherlanguage*}
\vspace*{\fill}
\large%for eng/rus
\singlespacing %\\onehalfspacing \\doublespacing % for eng/rus
\framebox[\textwidth]{
\begin{minipage}[t]{0.45\textwidth}
\begin{otherlanguage*}{russian}
\textbf{Лк., 71 зач., XIII, 10-17.}\\
В одной из синагог учил Он в субботу.
\\
Там была женщина, восемнадцать лет имевшая духа немощи: она была скорчена и не могла выпрямиться.
\\
Иисус, увидев ее, подозвал и сказал ей: женщина! ты освобождаешься от недуга твоего.
\\
И возложил на нее руки, и она тотчас выпрямилась и стала славить Бога.
\\
При этом начальник синагоги, негодуя, что Иисус исцелил в субботу, сказал народу: есть шесть дней, в которые должно делать; в те и приходите исцеляться, а не в день субботний.
\\
Господь сказал ему в ответ: лицемер! не отвязывает ли каждый из вас вола своего или осла от яслей в субботу и не ведет ли поить?
\\
сию же дочь Авраамову, которую связал сатана вот уже восемнадцать лет, не надлежало ли освободить от уз сих в день субботний?
\\
И когда говорил Он это, все противившиеся Ему стыдились; и весь народ радовался о всех славных делах Его.
\\

\end{otherlanguage*}
\end{minipage}
\hfill
\begin{minipage}[t]{0.45\textwidth}

\textbf{Luke 13:10--13:17.}\\
And he was teaching in one of the synagogues on the sabbath.\\
And, behold, there was a woman which had a spirit of infirmity eighteen years, and was bowed together, and could in no wise lift up herself.\\
And when Jesus saw her, he called her to him, and said unto her, Woman, thou art loosed from thine infirmity.\\
And he laid his hands on her: and immediately she was made straight, and glorified God.\\
And the ruler of the synagogue answered with indignation, because that Jesus had healed on the sabbath day, and said unto the people, There are six days in which men ought to work: in them therefore come and be healed, and not on the sabbath day.\\
The Lord then answered him, and said, Thou hypocrite, doth not each one of you on the sabbath loose his ox or his ass from the stall, and lead him away to watering?\\
And ought not this woman, being a daughter of Abraham, whom Satan hath bound, lo, these eighteen years, be loosed from this bond on the sabbath day?\\
And when he had said these things, all his adversaries were ashamed: and all the people rejoiced for all the glorious things that were done by him.\\

\end{minipage}}
\vspace*{\fill}
\newpage
\huge%for chi
\vspace*{\fill}
\begin{spacing}{1.0}
\framebox[\textwidth]{
\begin{minipage}[t]{\textwidth}
\begin{CJK}{UTF8}{bsmi}
\textbf{路加福音 13:10--13:17.}\\
安息日、耶穌在會堂裡教訓人。\\
有一個女人、被鬼附著病了十八年.腰彎得一點直不起來。\\
耶穌看見、便叫過他來、對他說、女人、你脫離這病了。\\
於是用兩隻手按著他.他立刻直起腰來、就歸榮耀與 神。\\
管會堂的、因為耶穌在安息日治病、就氣忿忿的對眾人說、有六日應當作工.那六日之內、可以來求醫、在安息日卻不可。\\
主說、假冒為善的人哪、難道你們各人在安息日不解開槽上的牛驢、牽去飲麼。\\
況且這女人本是亞伯拉罕的後裔、被撒但捆綁了這十八年、不當在安息日解開他的綁麼。\\
耶穌說這話、他的敵人都慚愧了.眾人因他所行一切榮耀的事、就都歡喜了。\\

\end{CJK}
\end{minipage}}
\end{spacing}
\vspace*{\fill}
\end{document}
