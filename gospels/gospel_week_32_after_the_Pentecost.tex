\documentclass[14pt]{article} % use larger type; default would be 10pt
\usepackage[utf8]{inputenc}       % кодування документа; замість cp866nav
\usepackage{extsizes}
\usepackage[top=0.3in,bottom=0.1in,left=0.1in,right=0.4in]{geometry}
\usepackage[russian,english]{babel} % національна локалізація; може бути декілька
\usepackage{setspace}
\usepackage{CJKutf8}
\usepackage{mdframed}
\title{\vspace{-5ex}Divine Liturgy\\Reading from The Gospels\vspace{-0.5ex}}
\author{\vspace{-2ex}Week 32 after the Pentecost\vspace{
-1ex%for author
}}
\date{\vspace{
-7ex%for date
}}
\begin{document}
\pagenumbering{gobble}
\begin{otherlanguage*}{russian}
\maketitle
\end{otherlanguage*}
\vspace*{\fill}
\large%for eng/rus
\singlespacing %\\onehalfspacing \\doublespacing % for eng/rus
\framebox[\textwidth]{
\begin{minipage}[t]{0.45\textwidth}
\begin{otherlanguage*}{russian}
\textbf{Мк., 1 зач., I, 1-8.}\\
Начало Евангелия Иисуса Христа, Сына Божия,
\\
как написано у пророков: вот, Я посылаю Ангела Моего пред лицем Твоим, который приготовит путь Твой пред Тобою.
\\
Глас вопиющего в пустыне: приготовьте путь Господу, прямыми сделайте стези Ему.
\\
Явился Иоанн, крестя в пустыне и проповедуя крещение покаяния для прощения грехов.
\\
И выходили к нему вся страна Иудейская и Иерусалимляне, и крестились от него все в реке Иордане, исповедуя грехи свои.
\\
Иоанн же носил одежду из верблюжьего волоса и пояс кожаный на чреслах своих, и ел акриды и дикий мед.
\\
И проповедовал, говоря: идет за мною Сильнейший меня, у Которого я недостоин, наклонившись, развязать ремень обуви Его;
\\
я крестил вас водою, а Он будет крестить вас Духом Святым.
\\

\end{otherlanguage*}
\end{minipage}
\hfill
\begin{minipage}[t]{0.45\textwidth}

\textbf{Mark 1:1--1:8.}\\
The beginning of the gospel of Jesus Christ, the Son of God;\\
As it is written in the prophets, Behold, I send my messenger before thy face, which shall prepare thy way before thee.\\
The voice of one crying in the wilderness, Prepare ye the way of the Lord, make his paths straight.\\
John did baptize in the wilderness, and preach the baptism of repentance for the remission of sins.\\
And there went out unto him all the land of Judaea, and they of Jerusalem, and were all baptized of him in the river of Jordan, confessing their sins.\\
And John was clothed with camel's hair, and with a girdle of a skin about his loins; and he did eat locusts and wild honey;\\
And preached, saying, There cometh one mightier than I after me, the latchet of whose shoes I am not worthy to stoop down and unloose.\\
I indeed have baptized you with water: but he shall baptize you with the Holy Ghost.\\

\end{minipage}}
\vspace*{\fill}
\newpage
\huge%for chi
\vspace*{\fill}
\begin{spacing}{1.0}
\framebox[\textwidth]{
\begin{minipage}[t]{\textwidth}
\begin{CJK}{UTF8}{bsmi}
\textbf{馬可福音 1:1--1:8.}\\
 神的兒子、耶穌基督福音的起頭、\\
正如先知以賽亞書上記著說、〔有古卷無以賽亞三字〕『看哪、我要差遣我的使者在你前面、預備道路.\\
在曠野有人聲喊著說、預備主的道、修直他的路。』\\
照這話、約翰來了、在曠野施洗、傳悔改的洗禮、使罪得赦。\\
猶太全地、和耶路撒冷的人、都出去到約翰那裡、承認他們的罪、在約但河裡受他的洗。\\
約翰穿駱駝毛的衣服、腰束皮帶、喫的是蝗蟲野蜜。\\
他傳道說、有一位在我以後來的、能力比我更大、我就是彎腰給他解鞋帶、也是不配的。\\
我是用水給你們施洗、他卻要用聖靈給你們施洗。\\

\end{CJK}
\end{minipage}}
\end{spacing}
\vspace*{\fill}
\end{document}
