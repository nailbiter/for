\documentclass[10pt]{article} % use larger type; default would be 10pt
\usepackage[utf8]{inputenc}       % кодування документа; замість cp866nav
\usepackage[top=0.1in,bottom=0.1in,left=0.1in,right=0.3in]{geometry}
\usepackage[russian,english]{babel} % національна локалізація; може бути декілька
\usepackage{setspace}
\usepackage{CJKutf8}
\usepackage{mdframed}
\usepackage{setspace}
\title{Divine Liturgy\\Reading from The Epistles}
\author{Pentecost\vspace{
-3ex%for author
}}
\date{\vspace{
-5ex%for date
}}
\begin{document}
\pagenumbering{gobble}
\begin{otherlanguage*}{russian}
\maketitle
\end{otherlanguage*}
\vspace*{\fill}
\Large%for eng/rus
\singlespacing %\\onehalfspacing \\doublespacing % for eng/rus
\framebox[\textwidth]{
\begin{minipage}[t]{0.48\textwidth}
\begin{otherlanguage*}{russian}
\textbf{Деян., 3 зач., II, 1-11.}\\
\Large%for eng/rus
При наступлении дня Пятидесятницы все они были единодушно вместе.
\\
И внезапно сделался шум с неба, как бы от несущегося сильного ветра, и наполнил весь дом, где они находились.
\\
И явились им разделяющиеся языки, как бы огненные, и почили по одному на каждом из них.
\\
И исполнились все Духа Святаго, и начали говорить на иных языках, как Дух давал им провещевать.
\\
В Иерусалиме же находились Иудеи, люди набожные, из всякого народа под небом.
\\
Когда сделался этот шум, собрался народ, и пришел в смятение, ибо каждый слышал их говорящих его наречием.
\\
И все изумлялись и дивились, говоря между собою: сии говорящие не все ли Галилеяне?
\\
Как же мы слышим каждый собственное наречие, в котором родились.
\\
Парфяне, и Мидяне, и Еламиты, и жители Месопотамии, Иудеи и Каппадокии, Понта и Асии,
\\
критяне и аравитяне, слышим их нашими языками говорящих о великих делах Божиих?
\\

\end{otherlanguage*}
\end{minipage}
\hfill
\begin{minipage}[t]{0.45\textwidth}

\textbf{Acts 2:1--2:11.}\\
And when the day of Pentecost was fully come, they were all with one accord in one place.\\
And suddenly there came a sound from heaven as of a rushing mighty wind, and it filled all the house where they were sitting.\\
And there appeared unto them cloven tongues like as of fire, and it sat upon each of them.\\
And they were all filled with the Holy Ghost, and began to speak with other tongues, as the Spirit gave them utterance.\\
And there were dwelling at Jerusalem Jews, devout men, out of every nation under heaven.\\
Now when this was noised abroad, the multitude came together, and were confounded, because that every man heard them speak in his own language.\\
And they were all amazed and marvelled, saying one to another, Behold, are not all these which speak Galilaeans?\\
And how hear we every man in our own tongue, wherein we were born?\\
Parthians, and Medes, and Elamites, and the dwellers in Mesopotamia, and in Judaea, and Cappadocia, in Pontus, and Asia,\\
Phrygia, and Pamphylia, in Egypt, and in the parts of Libya about Cyrene, and strangers of Rome, Jews and proselytes,\\
Cretes and Arabians, we do hear them speak in our tongues the wonderful works of God.\\

\end{minipage}}
\vspace*{\fill}
\newpage
\Huge%for chi
\vspace*{\fill}
\begin{spacing}{0.9}%for chinese
\framebox[\textwidth]{
\begin{minipage}[t]{\textwidth}
\begin{CJK}{UTF8}{bsmi}
\textbf{使徒行傳 2:1--2:11.}\\
五旬節到了、門徒都聚集在一處。\\
忽然從天上有響聲下來、好像一陣大風吹過、充滿了他們所坐的屋子。\\
又有舌頭如火焰顯現出來、分開落在他們各人頭上。\\
他們就都被聖靈充滿、按著聖靈所賜的口才、說起別國的話來。\\
那時、有虔誠的猶太人、從天下各國來、住在耶路撒冷。\\
這聲音一響、眾人都來聚集、各人聽見門徒用眾人的鄉談說話、就甚納悶.\\
都驚訝希奇說、看哪、這說話的不都是加利利人麼.\\
我們各人、怎麼聽見他們說我們生來所用的鄉談呢。\\
我們帕提亞人、瑪代人、以攔人、和住在米所波大米、猶太、加帕多家、本都、亞西亞、\\
弗呂家、旁非利亞、埃及的人、並靠近古利奈的呂彼亞一帶地方的人、從羅馬來的客旅中、或是猶太人、或是進猶太教的人、\\
革哩底和亞拉伯人、都聽見他們用我們的鄉談、講說 神的大作為。\\
\end{CJK}
\end{minipage}}
\end{spacing}
\vspace*{\fill}
\end{document}
