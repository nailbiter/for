\documentclass[10pt]{article} % use larger type; default would be 10pt
\usepackage[utf8]{inputenc}       % кодування документа; замість cp866nav
\usepackage[top=0.5in,bottom=0.5in,left=0.5in,right=0.5in]{geometry}
\usepackage[russian,english]{babel} % національна локалізація; може бути декілька
\usepackage{setspace}
\usepackage{CJKutf8}
\usepackage{mdframed}
\usepackage{setspace}
\title{Divine Liturgy\\Reading from The Epistles}
\author{Week 2 after the Pentecost\vspace{
-3ex%for author
}}
\date{\vspace{
-5ex%for date
}}
\begin{document}
\pagenumbering{gobble}
\begin{otherlanguage*}{russian}
\maketitle
\end{otherlanguage*}
\vspace*{\fill}
\Large%for eng/rus
\singlespacing %\\onehalfspacing \\doublespacing % for eng/rus
\framebox[\textwidth]{
\begin{minipage}[t]{0.45\textwidth}
\begin{otherlanguage*}{russian}
\textbf{Рим., 88 зач., V, 1-10.}\\
\Large%for eng/rus
Итак, оправдавшись верою, мы имеем мир с Богом через Господа нашего Иисуса Христа,
\\
через Которого верою и получили мы доступ к той благодати, в которой стоим и хвалимся надеждою славы Божией.
\\
И не сим только, но хвалимся и скорбями, зная, что от скорби происходит терпение,
\\
от терпения опытность, от опытности надежда,
\\
а надежда не постыжает, потому что любовь Божия излилась в сердца наши Духом Святым, данным нам.
\\
Ибо Христос, когда еще мы были немощны, в определенное время умер за нечестивых.
\\
Ибо едва ли кто умрет за праведника; разве за благодетеля, может быть, кто и решится умереть.
\\
Но Бог Свою любовь к нам доказывает тем, что Христос умер за нас, когда мы были еще грешниками.
\\
Посему тем более ныне, будучи оправданы Кровию Его, спасемся Им от гнева.
\\
Ибо если, будучи врагами, мы примирились с Богом смертью Сына Его, то тем более, примирившись, спасемся жизнью Его.
\\

\end{otherlanguage*}
\end{minipage}
\hfill
\begin{minipage}[t]{0.45\textwidth}

\textbf{Romans 5:1--5:10.}\\
Therefore being justified by faith, we have peace with God through our Lord Jesus Christ:\\
By whom also we have access by faith into this grace wherein we stand, and rejoice in hope of the glory of God.\\
And not only so, but we glory in tribulations also: knowing that tribulation worketh patience;\\
And patience, experience; and experience, hope:\\
And hope maketh not ashamed; because the love of God is shed abroad in our hearts by the Holy Ghost which is given unto us.\\
For when we were yet without strength, in due time Christ died for the ungodly.\\
For scarcely for a righteous man will one die: yet peradventure for a good man some would even dare to die.\\
But God commendeth his love toward us, in that, while we were yet sinners, Christ died for us.\\
Much more then, being now justified by his blood, we shall be saved from wrath through him.\\
For if, when we were enemies, we were reconciled to God by the death of his Son, much more, being reconciled, we shall be saved by his life.\\

\end{minipage}}
\vspace*{\fill}
\newpage
\Huge%for chi
\vspace*{\fill}
\begin{spacing}{1.3}%for chinese
\framebox[\textwidth]{
\begin{minipage}[t]{\textwidth}
\begin{CJK}{UTF8}{bsmi}
\textbf{羅馬書 5:1--5:10.}\\
我們既因信稱義、就藉著我們的主耶穌基督、得與 神相和。\\
我們又藉著他、因信得進入現在所站的這恩典中、並且歡歡喜喜盼望 神的榮耀。\\
不但如此、就是在患難中、也是歡歡喜喜的.因為知道患難生忍耐.\\
忍耐生老練.老練生盼望.\\
盼望不至於羞恥.因為所賜給我們的聖靈、將 神的愛澆灌在我們心裡。\\
因我們還軟弱的時候、基督就按所定的日期為罪人死。\\
為義人死、是少有的、為仁人死、或者有敢作的。\\
惟有基督在我們還作罪人的時候為我們死、 神的愛就在此向我們顯明了。\\
現在我們既靠著他的血稱義、就更要藉著他免去 神的忿怒。\\
因為我們作仇敵的時候、且藉著 神兒子的死、得與 神和好、既已和好、就更要因他的生得救了。\\

\end{CJK}
\end{minipage}}
\end{spacing}
\vspace*{\fill}
\end{document}
