\documentclass[10pt]{article} % use larger type; default would be 10pt
\usepackage[utf8]{inputenc}       % кодування документа; замість cp866nav
\usepackage[margin=0.5in]{geometry}
\usepackage[russian,english]{babel} % національна локалізація; може бути декілька
\usepackage{setspace}
\usepackage{CJKutf8}
\usepackage{mdframed}
\usepackage{setspace}
\title{Divine Liturgy\\Reading from The Epistles}
\author{Week about the Pharisee and the Publican (Martyrs)\vspace{
-3ex%for author
}}
\date{\vspace{
-5ex%for date
}}
\begin{document}
\pagenumbering{gobble}
\begin{otherlanguage*}{russian}
\maketitle
\end{otherlanguage*}
\vspace*{\fill}
\large%for eng/rus
\singlespacing %\\onehalfspacing \\doublespacing % for eng/rus
\framebox[\textwidth]{
\begin{minipage}[t]{0.45\textwidth}
\begin{otherlanguage*}{russian}
\textbf{Рим., 99 зач., VIII, 28-39.}\\
\large%for eng/rus
Притом знаем, что любящим Бога, призванным по Его изволению, все содействует ко благу.\\
Ибо кого Он предузнал, тем и предопределил быть подобными образу Сына Своего, дабы Он был первородным между многими братиями.\\
А кого Он предопределил, тех и призвал, а кого призвал, тех и оправдал; а кого оправдал, тех и прославил.\\
Что же сказать на это? Если Бог за нас, кто против нас?\\
Тот, Который Сына Своего не пощадил, но предал Его за всех нас, как с Ним не дарует нам и всего?\\
Кто будет обвинять избранных Божиих? Бог оправдывает их.\\
Кто осуждает? Христос Иисус умер, но и воскрес: Он и одесную Бога, Он и ходатайствует за нас.\\
Кто отлучит нас от любви Божией: скорбь, или теснота, или гонение, или голод, или нагота, или опасность, или меч? как написано:\\
за Тебя умерщвляют нас всякий день, считают нас за овец, обреченных на заклание.\\
Но все сие преодолеваем силою Возлюбившего нас.\\
Ибо я уверен, что ни смерть, ни жизнь, ни Ангелы, ни Начала, ни Силы, ни настоящее, ни будущее,\\
ни высота, ни глубина, ни другая какая тварь не может отлучить нас от любви Божией во Христе Иисусе, Господе нашем. \\
\end{otherlanguage*}
\end{minipage}
\hfill
\begin{minipage}[t]{0.45\textwidth}

\textbf{Romans 8:28 -- 8:39.}\\
And we know that all things work together for good to them that love God, to them who are the called according to his purpose.\\
For whom he did foreknow, he also did predestinate to be conformed to the image of his Son, that he might be the firstborn among many brethren.\\
Moreover whom he did predestinate, them he also called: and whom he called, them he also justified: and whom he justified, them he also glorified.\\
What shall we then say to these things? If God be for us, who can be against us?\\
He that spared not his own Son, but delivered him up for us all, how shall he not with him also freely give us all things?\\
Who shall lay any thing to the charge of God's elect? It is God that justifieth.\\
Who is he that condemneth? It is Christ that died, yea rather, that is risen again, who is even at the right hand of God, who also maketh intercession for us.\\
Who shall separate us from the love of Christ? shall tribulation, or distress, or persecution, or famine, or nakedness, or peril, or sword?\\
As it is written, For thy sake we are killed all the day long; we are accounted as sheep for the slaughter.\\
Nay, in all these things we are more than conquerors through him that loved us.\\
For I am persuaded, that neither death, nor life, nor angels, nor principalities, nor powers, nor things present, nor things to come,\\
Nor height, nor depth, nor any other creature, shall be able to separate us from the love of God, which is in Christ Jesus our Lord.\\

\end{minipage}}
\vspace*{\fill}
\newpage
\huge%for chi
\vspace*{\fill}
\begin{spacing}{1.0}%for chinese
\framebox[\textwidth]{
\begin{minipage}[t]{\textwidth}
\begin{CJK}{UTF8}{bsmi}
\textbf{羅馬書 8:28 -- 8:39.}\\
我們知道,萬事都互相效力,叫愛上帝的人得益處,就是按他旨意被召的人。 \\
因為他所預知的人,他也預定他們效法他兒子的榜樣,使他兒子在許多弟兄中作長子。 \\
他所預定的人,他又召他們來;所召來的人,他又稱他們為義;所稱為義的人,他又叫他們得榮耀。 \\
既是這樣,我們對這些事還要怎麼說呢?上帝若幫助我們,誰能抵擋我們呢? \\
上帝既不顧惜自己的兒子,為我們眾人捨了他,豈不也把萬物和他一同白白地賜給我們嗎? \\
誰能控告上帝所揀選的人呢?有上帝稱他們為義了。 \\
誰能定他們的罪呢?有基督耶穌 已經死了,而且復活了,現今在上帝的右邊,也替我們祈求。 \\
誰能使我們與基督的愛隔絕呢?難道是患難嗎?是困苦嗎?是迫害嗎?是飢餓嗎?是赤身露體嗎?是危險嗎?是刀劍嗎? \\
如經上所記: 「我們為你的緣故終日被殺; 人看我們如將宰的羊。」 \\
然而,靠着愛我們的主,在這一切的事上,我們已經得勝有餘了。 \\
因為我深信,無論是死,是活,是天使,是掌權的,是有權能的 ,是現在的事,是將來的事, \\
是高處的,是深處的,是別的受造之物,都不能使我們與上帝的愛隔絕,這愛是在我們的主基督耶穌裏的。 \\

\end{CJK}
\end{minipage}}
\end{spacing}
\vspace*{\fill}
\end{document}
