\documentclass[10pt]{article} % use larger type; default would be 10pt
\usepackage[utf8]{inputenc}       % кодування документа; замість cp866nav
\usepackage[top=0.2in,bottom=0.2in,left=0.1in,right=0.4in]{geometry}
\usepackage[russian,english]{babel} % національна локалізація; може бути декілька
\usepackage{setspace}
\usepackage{CJKutf8}
\usepackage{mdframed}
\title{Divine Liturgy\\Reading from The Gospels}
\author{Pentecost\vspace{
-3ex%for author
}}
\date{\vspace{
-7ex%for date
}}
\begin{document}
\pagenumbering{gobble}
\begin{otherlanguage*}{russian}
\maketitle
\end{otherlanguage*}
\vspace*{\fill}
\large%for eng/rus
\singlespacing %\\onehalfspacing \\doublespacing % for eng/rus
\framebox[\textwidth]{
\begin{minipage}[t]{0.45\textwidth}
\begin{otherlanguage*}{russian}
\textbf{Ин., 27 зач., VII, 37-52; VIII, 12.}\\
В последний же великий день праздника стоял Иисус и возгласил, говоря: кто жаждет, иди ко Мне и пей.
\\
Кто верует в Меня, у того, как сказано в Писании, из чрева потекут реки воды живой.
\\
Сие сказал Он о Духе, Которого имели принять верующие в Него: ибо еще не было на них Духа Святаго, потому что Иисус еще не был прославлен.
\\
Многие из народа, услышав сии слова, говорили: Он точно пророк.
\\
Другие говорили: это Христос. А иные говорили: разве из Галилеи Христос придет?
\\
Не сказано ли в Писании, что Христос придет от семени Давидова и из Вифлеема, из того места, откуда был Давид?
\\
Итак произошла о Нем распря в народе.
\\
Некоторые из них хотели схватить Его; но никто не наложил на Него рук.
\\
Итак служители возвратились к первосвященникам и фарисеям, и сии сказали им: для чего вы не привели Его?
\\
Служители отвечали: никогда человек не говорил так, как Этот Человек.
\\
Фарисеи сказали им: неужели и вы прельстились?
\\
Уверовал ли в Него кто из начальников, или из фарисеев?
\\
Но этот народ невежда в законе, проклят он.
\\
Никодим, приходивший к Нему ночью, будучи один из них, говорит им:
\\
судит ли закон наш человека, если прежде не выслушают его и не узнают, что он делает?
\\
На это сказали ему: и ты не из Галилеи ли? рассмотри и увидишь, что из Галилеи не приходит пророк.
\\
Опять говорил Иисус к народу и сказал им: Я свет миру; кто последует за Мною, тот не будет ходить во тьме, но будет иметь свет жизни.\\
\end{otherlanguage*}
\end{minipage}
\hfill
\begin{minipage}[t]{0.45\textwidth}

\textbf{John 7:37--7:52, 8:12.}\\
In the last day, that great day of the feast, Jesus stood and cried, saying, If any man thirst, let him come unto me, and drink.\\
He that believeth on me, as the scripture hath said, out of his belly shall flow rivers of living water.\\
(But this spake he of the Spirit, which they that believe on him should receive: for the Holy Ghost was not yet given; because that Jesus was not yet glorified.)\\
Many of the people therefore, when they heard this saying, said, Of a truth this is the Prophet.\\
Others said, This is the Christ. But some said, Shall Christ come out of Galilee?\\
Hath not the scripture said, That Christ cometh of the seed of David, and out of the town of Bethlehem, where David was?\\
So there was a division among the people because of him.\\
And some of them would have taken him; but no man laid hands on him.\\
Then came the officers to the chief priests and Pharisees; and they said unto them, Why have ye not brought him?\\
The officers answered, Never man spake like this man.\\
Then answered them the Pharisees, Are ye also deceived?\\
Have any of the rulers or of the Pharisees believed on him?\\
But this people who knoweth not the law are cursed.\\
Nicodemus saith unto them, (he that came to Jesus by night, being one of them,)\\
Doth our law judge any man, before it hear him, and know what he doeth?\\
They answered and said unto him, Art thou also of Galilee? Search, and look: for out of Galilee ariseth no prophet.\\
Then spake Jesus again unto them, saying, I am the light of the world: he that followeth me shall not walk in darkness, but shall have the 
light of life.\\
\end{minipage}}
\vspace*{\fill}
\newpage
\huge%for chi
\vspace*{\fill}
\begin{spacing}{1.0}
\framebox[\textwidth]{
\begin{minipage}[t]{\textwidth}
\begin{CJK}{UTF8}{bsmi}
\textbf{約翰福音 7:37--7:52, 8:12.}\\
節期的末日、就是最大之日、耶穌站著高聲說、人若渴了、可以到我這裡來喝。\\
信我的人、就如經上所說、從他腹中要流出活水的江河來。\\
耶穌這話是指著信他之人、要受聖靈說的、那時還沒有賜下聖靈來.因為耶穌尚未得著榮耀。\\
眾人聽見這話、有的說、這真是那先知。\\
有的說、這是基督.但也有的說、基督豈是從加利利出來的麼。\\
經上豈不是說、基督是大衛的後裔、從大衛本鄉伯利恆出來的麼。\\
於是眾人因著耶穌起了分爭。\\
其中有人要捉拿他.只是無人下手。\\
差役回到祭司長和法利賽人那裡.他們對差役說、你們為甚麼沒有帶他來呢。\\
差役回答說、從來沒有像他這樣說話的。\\
法利賽人說、你們也受了迷惑麼。\\
官長或是法利賽人、豈有信他的呢。\\
但這些不明白律法的百姓、是被咒詛的。\\
內中有尼哥底母、就是從前去見耶穌的、對他們說、\\
不先聽本人的口供、不知道他所作的事、難道我們的律法還定他的罪麼。\\
他們回答說、你也是出於加利利麼.你且去查考、就可知道加利利沒有出過先知。\\
耶穌又對眾人說、我是世界的光.跟從我的、就不在黑暗裡走、必要得著生命的光。\\
\end{CJK}
\end{minipage}}
\end{spacing}
\vspace*{\fill}
\end{document}
