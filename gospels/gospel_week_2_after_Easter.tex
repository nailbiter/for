\documentclass[10pt]{article} % use larger type; default would be 10pt
\usepackage[utf8]{inputenc}       % кодування документа; замість cp866nav
\usepackage[top=0.3in,bottom=0.3in,left=0.2in,right=0.5in]{geometry}
\usepackage[russian,english]{babel} % національна локалізація; може бути декілька
\usepackage{setspace}
\usepackage{CJKutf8}
\usepackage{mdframed}
\title{Divine Liturgy\\Reading from The Gospels}
\author{Week 2 after Easter\vspace{
-3ex%for author
}}
\date{\vspace{
-7ex%for date
}}
\begin{document}
\pagenumbering{gobble}
\begin{otherlanguage*}{russian}
\maketitle
\end{otherlanguage*}
\vspace*{\fill}
\large%for eng/rus
\singlespacing %\\onehalfspacing \\doublespacing % for eng/rus
\framebox[\textwidth]{
\begin{minipage}[t]{0.45\textwidth}
\begin{otherlanguage*}{russian}
\textbf{Ин., 65 зач., XX, 19-31.}\\
В тот же первый день недели вечером, когда двери дома, где собирались ученики Его, были заперты из опасения от Иудеев, пришел Иисус, и стал посреди, и говорит им: мир вам!
\\
Сказав это, Он показал им руки и ноги и ребра Свои. Ученики обрадовались, увидев Господа.
\\
Иисус же сказал им вторично: мир вам! как послал Меня Отец, так и Я посылаю вас.
\\
Сказав это, дунул, и говорит им: примите Духа Святаго.
\\
Кому простите грехи, тому простятся; на ком оставите, на том останутся.
\\
Фома же, один из двенадцати, называемый Близнец, не был тут с ними, когда приходил Иисус.
\\
Другие ученики сказали ему: мы видели Господа. Но он сказал им: если не увижу на руках Его ран от гвоздей, и не вложу перста моего в раны от гвоздей, и не вложу руки моей в ребра Его, не поверю.
\\
После восьми дней опять были в доме ученики Его, и Фома с ними. Пришел Иисус, когда двери были заперты, стал посреди них и сказал: мир вам!
\\
Потом говорит Фоме: подай перст твой сюда и посмотри руки Мои; подай руку твою и вложи в ребра Мои; и не будь неверующим, но верующим.
\\
Фома сказал Ему в ответ: Господь мой и Бог мой!
\\
Иисус говорит ему: ты поверил, потому что увидел Меня; блаженны невидевшие и уверовавшие.
\\
Много сотворил Иисус пред учениками Своими и других чудес, о которых не писано в книге сей.
\\
Сие же написано, дабы вы уверовали, что Иисус есть Христос, Сын Божий, и, веруя, имели жизнь во имя Его.
\\

\end{otherlanguage*}
\end{minipage}
\hfill
\begin{minipage}[t]{0.45\textwidth}

\textbf{John 20:19--20:31.}\\
Then the same day at evening, being the first day of the week, when the doors were shut where the disciples were assembled for fear of the Jews, came Jesus and stood in the midst, and saith unto them, Peace be unto you.\\
And when he had so said, he shewed unto them his hands and his side. Then were the disciples glad, when they saw the Lord.\\
Then said Jesus to them again, Peace be unto you: as my Father hath sent me, even so send I you.\\
And when he had said this, he breathed on them, and saith unto them, Receive ye the Holy Ghost:\\
Whose soever sins ye remit, they are remitted unto them; and whose soever sins ye retain, they are retained.\\
But Thomas, one of the twelve, called Didymus, was not with them when Jesus came.\\
The other disciples therefore said unto him, We have seen the Lord. But he said unto them, Except I shall see in his hands the print of the nails, and put my finger into the print of the nails, and thrust my hand into his side, I will not believe.\\
And after eight days again his disciples were within, and Thomas with them: then came Jesus, the doors being shut, and stood in the midst, and said, Peace be unto you.\\
Then saith he to Thomas, Reach hither thy finger, and behold my hands; and reach hither thy hand, and thrust it into my side: and be not faithless, but believing.\\
And Thomas answered and said unto him, My Lord and my God.\\
Jesus saith unto him, Thomas, because thou hast seen me, thou hast believed: blessed are they that have not seen, and yet have believed.\\
And many other signs truly did Jesus in the presence of his disciples, which are not written in this book:\\
But these are written, that ye might believe that Jesus is the Christ, the Son of God; and that believing ye might have life through his name.\\

\end{minipage}}
\vspace*{\fill}
\newpage
\huge%for chi
\vspace*{\fill}
\begin{spacing}{1.0}
\framebox[\textwidth]{
\begin{minipage}[t]{\textwidth}
\begin{CJK}{UTF8}{bsmi}
\textbf{約翰福音 20:19--20:31.}\\
那日(就是七日的第一日)晚上、門徒所在的地方、因怕猶太人、門都關了.耶穌來站在當中、對他們說、願你們平安。\\
說了這話、就把手和肋旁、指給他們看.門徒看見主、就喜樂了。\\
耶穌又對他們說、願你們平安.父怎樣差遣了我、我也照樣差遣你們。\\
說了這話、就向他們吹一口氣、說、你們受聖靈。\\
你們赦免誰的罪、誰的罪就赦免了.你們留下誰的罪、誰的罪就留下了。\\
那十二個門徒中、有稱為低土馬的多馬.耶穌來的時候、他沒有和他們同在。\\
那些門徒就對他說、我們已經看見主了。多馬卻說、我非看見他手上的釘痕、用指頭探入那釘痕、又用手探入他的肋旁、我總不信。\\
過了八日、門徒又在屋裡、多馬也和他們同在、門都關了.耶穌來站在當中說、願你們平安。\\
就對多馬說、伸過你的指頭來、摸(摸原文作看)我的手.伸出你的手來、探入我的肋旁.不要疑惑、總要信。\\
多馬說、我的主、我的 神。\\
耶穌對他說、你因看見了我纔信.那沒有看見就信的、有福了。\\
耶穌在門徒面前、另外行了許多神蹟、沒有記在這書上。\\
但記這些事、要叫你們信耶穌是基督、是 神的兒子.並且叫你們信了他、就可以因他的名得生命。\\

\end{CJK}
\end{minipage}}
\end{spacing}
\vspace*{\fill}
\end{document}
