\documentclass[10pt]{article} % use larger type; default would be 10pt
\usepackage[utf8]{inputenc}       % кодування документа; замість cp866nav
\usepackage[top=0.5in,bottom=0.5in,left=0.5in,right=0.5in]{geometry}
\usepackage[russian,english]{babel} % національна локалізація; може бути декілька
\usepackage{setspace}
\usepackage{CJKutf8}
\usepackage{mdframed}
\usepackage{setspace}
\title{Divine Liturgy\\Reading from The Epistles}
\author{Week 7 after the Pentecost\vspace{
-3ex%for author
}}
\date{\vspace{
-5ex%for date
}}
\begin{document}
\pagenumbering{gobble}
\begin{otherlanguage*}{russian}
\maketitle
\end{otherlanguage*}
\vspace*{\fill}
\Large%for eng/rus
\onehalfspacing %\\onehalfspacing \\doublespacing % for eng/rus
\framebox[\textwidth]{
\begin{minipage}[t]{0.45\textwidth}
\begin{otherlanguage*}{russian}
\textbf{Рим., 116 зач., XV, 1-7.}\\
\Large%for eng/rus
Мы, сильные, должны сносить немощи бессильных и не себе угождать.
\\
Каждый из нас должен угождать ближнему, во благо, к назиданию.
\\
Ибо и Христос не Себе угождал, но, как написано: злословия злословящих Тебя пали на Меня.
\\
А все, что писано было прежде, написано нам в наставление, чтобы мы терпением и утешением из Писаний сохраняли надежду.
\\
Бог же терпения и утешения да дарует вам быть в единомыслии между собою, по учению Христа Иисуса,
\\
дабы вы единодушно, едиными устами славили Бога и Отца Господа нашего Иисуса Христа.
\\
Посему принимайте друг друга, как и Христос принял вас в славу Божию.
\\

\end{otherlanguage*}
\end{minipage}
\hfill
\begin{minipage}[t]{0.45\textwidth}

\textbf{Romans 15:1--15:7.}\\
We then that are strong ought to bear the infirmities of the weak, and not to please ourselves.\\
Let every one of us please his neighbour for his good to edification.\\
For even Christ pleased not himself; but, as it is written, The reproaches of them that reproached thee fell on me.\\
For whatsoever things were written aforetime were written for our learning, that we through patience and comfort of the scriptures might have hope.\\
Now the God of patience and consolation grant you to be likeminded one toward another according to Christ Jesus:\\
That ye may with one mind and one mouth glorify God, even the Father of our Lord Jesus Christ.\\
Wherefore receive ye one another, as Christ also received us to the glory of God.\\

\end{minipage}}
\vspace*{\fill}
\newpage
\Huge%for chi
\vspace*{\fill}
\begin{spacing}{1.45}%for chinese
\framebox[\textwidth]{
\begin{minipage}[t]{\textwidth}
\begin{CJK}{UTF8}{bsmi}
\textbf{羅馬書 15:1--15:7.}\\
我們堅固的人、應該擔代不堅固人的軟弱、不求自己的喜悅。\\
我們各人務要叫鄰舍喜悅、使他得益處、建立德行。\\
因為基督也不求自己的喜悅、如經上所記、『辱罵你人的辱罵、都落在我身上。』\\
從前所寫的聖經、都是為教訓我們寫的、叫我們因聖經所生的忍耐和安慰、可以得著盼望。\\
但願賜忍耐安慰的 神、叫你們彼此同心、效法基督耶穌.\\
一心一口、榮耀 神、我們主耶穌基督的父。\\
所以你們要彼此接納、如同基督接納你們一樣、使榮耀歸與 神。\\

\end{CJK}
\end{minipage}}
\end{spacing}
\vspace*{\fill}
\end{document}
