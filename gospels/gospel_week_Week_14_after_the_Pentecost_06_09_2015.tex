\documentclass[10pt]{article} % use larger type; default would be 10pt
\usepackage[utf8]{inputenc}       % кодування документа; замість cp866nav
\usepackage{extsizes}
\usepackage[top=0.5in,bottom=0.5in,left=0.5in,right=0.5in]{geometry}
\usepackage[russian,english]{babel} % національна локалізація; може бути декілька
\usepackage{setspace}
\usepackage{CJKutf8}
\usepackage{mdframed}
\usepackage{extsizes}
\title{Divine Liturgy\\Reading from The Gospels}
\author{Week 14 after the Pentecost\vspace{
-3ex%for author
}}
\date{\vspace{
-7ex%for date
}}
\begin{document}
\pagenumbering{gobble}
\begin{otherlanguage*}{russian}
\maketitle
\end{otherlanguage*}
\vspace*{\fill}
\large%for eng/rus
\singlespacing %\\onehalfspacing \\doublespacing % for eng/rus
\framebox[\textwidth]{
\begin{minipage}[t]{0.45\textwidth}
\begin{otherlanguage*}{russian}
\textbf{Мф., 89 зач., XXII, 1-14.}\\
Иисус, продолжая говорить им притчами, сказал:
\\
Царство Небесное подобно человеку царю, который сделал брачный пир для сына своего
\\
и послал рабов своих звать званых на брачный пир; и не хотели прийти.
\\
Опять послал других рабов, сказав: скажите званым: вот, я приготовил обед мой, тельцы мои и что откормлено, заколото, и всё готово; приходите на брачный пир.
\\
Но они, пренебрегши то, пошли, кто на поле свое, а кто на торговлю свою;
\\
прочие же, схватив рабов его, оскорбили и убили их.
\\
Услышав о сем, царь разгневался, и, послав войска свои, истребил убийц оных и сжег город их.
\\
Тогда говорит он рабам своим: брачный пир готов, а званые не были достойны;
\\
итак пойдите на распутия и всех, кого найдете, зовите на брачный пир.
\\
И рабы те, выйдя на дороги, собрали всех, кого только нашли, и злых и добрых; и брачный пир наполнился возлежащими.
\\
Царь, войдя посмотреть возлежащих, увидел там человека, одетого не в брачную одежду,
\\
и говорит ему: друг! как ты вошел сюда не в брачной одежде? Он же молчал.
\\
Тогда сказал царь слугам: связав ему руки и ноги, возьмите его и бросьте во тьму внешнюю; там будет плач и скрежет зубов;
\\
ибо много званых, а мало избранных.
\\

\end{otherlanguage*}
\end{minipage}
\hfill
\begin{minipage}[t]{0.45\textwidth}

\textbf{Matthew 22:1--22:14.}\\
And Jesus answered and spake unto them again by parables, and said,\\
The kingdom of heaven is like unto a certain king, which made a marriage for his son,\\
And sent forth his servants to call them that were bidden to the wedding: and they would not come.\\
Again, he sent forth other servants, saying, Tell them which are bidden, Behold, I have prepared my dinner: my oxen and my fatlings are killed, and all things are ready: come unto the marriage.\\
But they made light of it, and went their ways, one to his farm, another to his merchandise:\\
And the remnant took his servants, and entreated them spitefully, and slew them.\\
But when the king heard thereof, he was wroth: and he sent forth his armies, and destroyed those murderers, and burned up their city.\\
Then saith he to his servants, The wedding is ready, but they which were bidden were not worthy.\\
Go ye therefore into the highways, and as many as ye shall find, bid to the marriage.\\
So those servants went out into the highways, and gathered together all as many as they found, both bad and good: and the wedding was furnished with guests.\\
And when the king came in to see the guests, he saw there a man which had not on a wedding garment:\\
And he saith unto him, Friend, how camest thou in hither not having a wedding garment? And he was speechless.\\
Then said the king to the servants, Bind him hand and foot, and take him away, and cast him into outer darkness; there shall be weeping and gnashing of teeth.\\
For many are called, but few are chosen.\\

\end{minipage}}
\vspace*{\fill}
\newpage
\huge%for chi
\vspace*{\fill}
\begin{spacing}{1.0}
\framebox[\textwidth]{
\begin{minipage}[t]{\textwidth}
\begin{CJK}{UTF8}{bsmi}
\textbf{馬太福音 22:1--22:14.}\\
耶穌又用比喻對他們說、\\
天國好比一個王、為他兒子擺設娶親的筵席。\\
就打發僕人去、請那些被召的人來赴席.他們卻不肯來。\\
王又打發別的僕人說、你們告訴那被召的人、我的筵席已經預備好了、牛和肥畜已經宰了、各樣都齊備.請你們來赴席。\\
那些人不理就走了.一個到自己田裡去.一個作買賣去.\\
其餘的拿住僕人、凌辱他們、把他們殺了。\\
王就大怒、發兵除滅那些兇手、燒燬他們的城。\\
於是對僕人說、喜筵已經齊備、只是所召的人不配。\\
所以你們要往岔路口上去、凡遇見的、都召來赴席。\\
那些僕人就出去到大路上、凡遇見的、不論善惡都召聚了來.筵席上就坐滿了客。\\
王進來觀看賓客、見那裡有一個沒有穿禮服的。\\
就對他說、朋友、你到這裡來、怎麼不穿禮服呢。那人無言可答。\\
於是王對使喚的人說、捆起他的手腳來、把他丟在外邊的黑暗裡.在那裡必要哀哭切齒了。\\
因為被召的人多、選上的人少。\\

\end{CJK}
\end{minipage}}
\end{spacing}
\vspace*{\fill}
\end{document}
