\documentclass[10pt]{article} % use larger type; default would be 10pt
\usepackage[utf8]{inputenc}       % кодування документа; замість cp866nav
\usepackage[margin=0.5in]{geometry}
\usepackage[russian,english]{babel} % національна локалізація; може бути декілька
\usepackage{setspace}
\usepackage{CJKutf8}
\usepackage{mdframed}
\title{Divine Liturgy\\Reading from The Epistles}
\author{Week 28 after the Pentecost\vspace{-3ex}}
\date{\vspace{-5ex}}
\begin{document}
\pagenumbering{gobble}
\begin{otherlanguage*}{russian}
\maketitle
\end{otherlanguage*}
\vspace*{\fill}
\normalsize%for eng/rus
\singlespacing %\\onehalfspacing \\doublespacing % for eng/rus
\framebox[\textwidth]{
\begin{minipage}[t]{0.45\textwidth}
\begin{otherlanguage*}{russian}
\textbf{Евр., 328 зач., XI, 9-10, 17-23, 32-40. }\\
\normalsize%for eng/rus
\textbf{\scriptsize 9} Верою обитал он на земле обетованной, как на чужой, и жил в шатрах с Исааком и Иаковом, сонаследниками того же обетования;\\
\textbf{\scriptsize 10} ибо он ожидал города, имеющего основание, которого художник и строитель Бог. \\
\textbf{\scriptsize 17} Верою Авраам, будучи искушаем, принес в жертву Исаака и, имея обетование, принес единородного,\\
\textbf{\scriptsize 18} о котором было сказано: в Исааке наречется тебе семя.\\
\textbf{\scriptsize 19} Ибо он думал, что Бог силен и из мертвых воскресить, почему и получил его в предзнаменование.\\
\textbf{\scriptsize 20} Верою в будущее Исаак благословил Иакова и Исава.\\
\textbf{\scriptsize 21} Верою Иаков, умирая, благословил каждого сына Иосифова и поклонился на верх жезла своего.\\
\textbf{\scriptsize 22} Верою Иосиф, при кончине, напоминал об исходе сынов Израилевых и завещал о костях своих.\\
\textbf{\scriptsize 23} Верою Моисей по рождении три месяца скрываем был родителями своими, ибо видели они, что дитя прекрасно, и не устрашились царского повеления. \\
\textbf{\scriptsize 32} И что еще скажу? Недостанет мне времени, чтобы повествовать о Гедеоне, о Вараке, о Самсоне и Иеффае, о Давиде, Самуиле и (других) пророках,\\
\textbf{\scriptsize 33} которые верою побеждали царства, творили правду, получали обетования, заграждали уста львов,\\
\textbf{\scriptsize 34} угашали силу огня, избегали острия меча, укреплялись от немощи, были крепки на войне, прогоняли полки чужих;\\
\textbf{\scriptsize 35} жены получали умерших своих воскресшими; иные же замучены были, не приняв освобождения, дабы получить лучшее воскресение;\\
\textbf{\scriptsize 36} другие испытали поругания и побои, а также узы и темницу,\\
\textbf{\scriptsize 37} были побиваемы камнями, перепиливаемы, подвергаемы пытке, умирали от меча, скитались в ми́лотях и козьих кожах, терпя недостатки, скорби, озлобления;\\
\textbf{\scriptsize 38} те, которых весь мир не был достоин, скитались по пустыням и горам, по пещерам и ущельям земли.\\
\textbf{\scriptsize 39} И все сии, свидетельствованные в вере, не получили обещанного,\\
\textbf{\scriptsize 40} потому что Бог предусмотрел о нас нечто лучшее, дабы они не без нас достигли совершенства. \\
\end{otherlanguage*}
\end{minipage}
\hfill
\begin{minipage}[t]{0.45\textwidth}

\textbf{Hebrews 11:9 -- 11:10, 11:17 -- 11:23, 11:32 -- 11:40.}\\
\textbf{\scriptsize 9} By faith he sojourned in the land of promise, as in a strange country, dwelling in tabernacles with Isaac and Jacob, the heirs with him of the same promise:\\
\textbf{\scriptsize 10} For he looked for a city which hath foundations, whose builder and maker is God.\\
\textbf{\scriptsize 17} By faith Abraham, when he was tried, offered up Isaac: and he that had received the promises offered up his only begotten son,\\
\textbf{\scriptsize 18} Of whom it was said, That in Isaac shall thy seed be called:\\
\textbf{\scriptsize 19} Accounting that God was able to raise him up, even from the dead; from whence also he received him in a figure.\\
\textbf{\scriptsize 20} By faith Isaac blessed Jacob and Esau concerning things to come.\\
\textbf{\scriptsize 21} By faith Jacob, when he was a dying, blessed both the sons of Joseph; and worshipped, leaning upon the top of his staff.\\
\textbf{\scriptsize 22} By faith Joseph, when he died, made mention of the departing of the children of Israel; and gave commandment concerning his bones.\\
\textbf{\scriptsize 23} By faith Moses, when he was born, was hid three months of his parents, because they saw he was a proper child; and they were not afraid of the king's commandment.\\
\textbf{\scriptsize 32} And what shall I more say? for the time would fail me to tell of Gedeon, and of Barak, and of Samson, and of Jephthae; of David also, and Samuel, and of the prophets:\\
\textbf{\scriptsize 33} Who through faith subdued kingdoms, wrought righteousness, obtained promises, stopped the mouths of lions,\\
\textbf{\scriptsize 34} Quenched the violence of fire, escaped the edge of the sword, out of weakness were made strong, waxed valiant in fight, turned to flight the armies of the aliens.\\
\textbf{\scriptsize 35} Women received their dead raised to life again: and others were tortured, not accepting deliverance; that they might obtain a better resurrection:\\
\textbf{\scriptsize 36} And others had trial of cruel mockings and scourgings, yea, moreover of bonds and imprisonment:\\
\textbf{\scriptsize 37} They were stoned, they were sawn asunder, were tempted, were slain with the sword: they wandered about in sheepskins and goatskins; being destitute, afflicted, tormented;\\
\textbf{\scriptsize 38} (Of whom the world was not worthy:) they wandered in deserts, and in mountains, and in dens and caves of the earth.\\
\textbf{\scriptsize 39} And these all, having obtained a good report through faith, received not the promise:\\
\textbf{\scriptsize 40} God having provided some better thing for us, that they without us should not be made perfect.\\
\end{minipage}}
\vspace*{\fill}
\newpage\Huge
\vspace*{\fill}
\singlespacing %for chinese
\framebox[\textwidth]{
\begin{minipage}[t]{\textwidth}
\begin{CJK}{UTF8}{bsmi}
\textbf{希伯來書 11:9 -- 11:10, 11:17 -- 11:23, 11:32 -- 11:40.}\\
\textbf{\scriptsize 9} 因着信,他就在所應許之地作客,好像在異鄉,居住在帳棚裏,與蒙同一個應許的 以撒和 雅各一樣。\\
\textbf{\scriptsize 10} 因為他等候着那座有根基的城,就是上帝所設計和建造的。 \\
\textbf{\scriptsize 17} 因着信, 亞伯拉罕被考驗的時候把 以撒獻上,這就是那領受了應許的人甘心把自己獨生的兒子獻上。\\
\textbf{\scriptsize 18} 論到這兒子,上帝曾說:「從 以撒生的才要稱為你的後裔。」\\
\textbf{\scriptsize 19} 他認為上帝甚至能使人從死人中復活,意味着他得回了他的兒子。\\
\textbf{\scriptsize 20} 因着信, 以撒指着將來的事給 雅各、 以掃祝福。\\
\textbf{\scriptsize 21} 因着信, 雅各臨死的時候給 約瑟的兩個兒子個別祝福,扶着枴杖敬拜上帝。\\
\textbf{\scriptsize 22} 因着信, 約瑟臨終的時候提到 以色列人將來要出 埃及,並為自己的骸骨留下遺言。\\
\textbf{\scriptsize 23} 因着信, 摩西生下來,他的父母見他是個俊美的孩子,把他藏了三個月,並不怕王的命令。 \\
\textbf{\scriptsize 32} 我還要說甚麼呢?若要一一細說 基甸、 巴拉、 參孫、 耶弗他、 大衛、 撒母耳和眾先知的事,時間就不夠了。\\
\textbf{\scriptsize 33} 他們藉着信,制伏了敵國,行了公義,得了應許,堵住了獅子的口,\\
\textbf{\scriptsize 34} 滅了烈火的威力,在鋒利的刀劍下逃生,從軟弱變為剛強,爭戰中顯出勇猛,打退外邦的全軍。\\
\textbf{\scriptsize 35} 有些婦人得回從死人中復活的親人。又有人忍受嚴刑,拒絕被釋放,為要得着更美好的復活。\\
\textbf{\scriptsize 36} 又有人忍受戲弄、鞭打、捆鎖、監禁、各等的磨煉;\\
\textbf{\scriptsize 37} 他們被石頭打死,被鋸鋸死,(註)被刀殺,披着綿羊山羊的皮各處奔跑,受貧窮、患難、虐待。\\
\textbf{\scriptsize 38} 這世界配不上他們,他們在曠野、山嶺、山洞、地穴,飄流無定。\\
\textbf{\scriptsize 39} 這些人都是因信獲得了讚許,卻仍未得着所應許的,\\
\textbf{\scriptsize 40} 因為上帝給我們預備了更美好的事,若沒有我們,他們就不能達到完全。 \\
\end{CJK}
\end{minipage}}
\vspace*{\fill}
\end{document}
