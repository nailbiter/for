\documentclass[10pt]{article} % use larger type; default would be 10pt
\usepackage[utf8]{inputenc}       % кодування документа; замість cp866nav
\usepackage{extsizes}
\usepackage[top=0.5in,bottom=0.5in,left=0.5in,right=0.5in]{geometry}
\usepackage[russian,english]{babel} % національна локалізація; може бути декілька
\usepackage{setspace}
\usepackage{CJKutf8}
\usepackage{mdframed}
\usepackage{extsizes}
\title{Divine Liturgy\\Reading from The Gospels}
\author{Week 23 after the Pentecost\vspace{
-3ex%for author
}}
\date{\vspace{
-7ex%for date
}}
\begin{document}
\pagenumbering{gobble}
\begin{otherlanguage*}{russian}
\maketitle
\end{otherlanguage*}
\vspace*{\fill}
\large%for eng/rus
\singlespacing %\\onehalfspacing \\doublespacing % for eng/rus
\framebox[\textwidth]{
\begin{minipage}[t]{0.45\textwidth}
\begin{otherlanguage*}{russian}
\textbf{Евр., 331 зач. (от полу́), XII, 6-13, 25-27.}\\
Ибо Господь, кого любит, того наказывает; бьет же всякого сына, которого принимает.
\\
Если вы терпите наказание, то Бог поступает с вами, как с сынами. Ибо есть ли какой сын, которого бы не наказывал отец?
\\
Если же остаетесь без наказания, которое всем обще, то вы незаконные дети, а не сыны.
\\
Притом, если мы, будучи наказываемы плотскими родителями нашими, боялись их, то не гораздо ли более должны покориться Отцу духов, чтобы жить?
\\
Те наказывали нас по своему произволу для немногих дней; а Сей - для пользы, чтобы нам иметь участие в святости Его.
\\
Всякое наказание в настоящее время кажется не радостью, а печалью; но после наученным через него доставляет мирный плод праведности.
\\
Итак укрепите опустившиеся руки и ослабевшие колени
\\
и ходите прямо ногами вашими, дабы хромлющее не совратилось, а лучше исправилось.
\\
Смотрите, не отвратитесь и вы от говорящего. Если те, не послушав глаголавшего на земле, не избегли наказания, то тем более не избежим мы, если отвратимся от Глаголющего с небес,
\\
Которого глас тогда поколебал землю, и Который ныне дал такое обещание: еще раз поколеблю не только землю, но и небо.
\\
Слова: "еще раз" означают изменение колеблемого, как сотворенного, чтобы пребыло непоколебимое.
\\

\end{otherlanguage*}
\end{minipage}
\hfill
\begin{minipage}[t]{0.45\textwidth}

\textbf{Hebrews 12:6--12:13, 12:25--12:27.}\\
For whom the Lord loveth he chasteneth, and scourgeth every son whom he receiveth.\\
If ye endure chastening, God dealeth with you as with sons; for what son is he whom the father chasteneth not?\\
But if ye be without chastisement, whereof all are partakers, then are ye bastards, and not sons.\\
Furthermore we have had fathers of our flesh which corrected us, and we gave them reverence: shall we not much rather be in subjection unto the Father of spirits, and live?\\
For they verily for a few days chastened us after their own pleasure; but he for our profit, that we might be partakers of his holiness.\\
Now no chastening for the present seemeth to be joyous, but grievous: nevertheless afterward it yieldeth the peaceable fruit of righteousness unto them which are exercised thereby.\\
Wherefore lift up the hands which hang down, and the feeble knees;\\
And make straight paths for your feet, lest that which is lame be turned out of the way; but let it rather be healed.\\
See that ye refuse not him that speaketh. For if they escaped not who refused him that spake on earth, much more shall not we escape, if we turn away from him that speaketh from heaven:\\
Whose voice then shook the earth: but now he hath promised, saying, Yet once more I shake not the earth only, but also heaven.\\
And this word, Yet once more, signifieth the removing of those things that are shaken, as of things that are made, that those things which cannot be shaken may remain.\\

\end{minipage}}
\vspace*{\fill}
\newpage
\huge%for chi
\vspace*{\fill}
\begin{spacing}{1.2}
\framebox[\textwidth]{
\begin{minipage}[t]{\textwidth}
\begin{CJK}{UTF8}{bsmi}
\textbf{希伯來書 12:6--12:13, 12:25--12:27.}\\
因為主所愛的他必管教、又鞭打凡所收納的兒子。』\\
你們所忍受的、是 神管教你們、待你們如同待兒子.焉有兒子不被父親管教的呢。\\
管教原是眾子所共受的、你們若不受管教、就是私子、不是兒子了。\\
再者、我們曾有生身的父管教我們、我們尚且敬重他、何況萬靈的父、我們豈不更當順服他得生麼。\\
生身的父都是暫隨己意管教我們.惟有萬靈的父管教我們、是要我們得益處、使我們在他的聖潔上有分。\\
凡管教的事、當時不覺得快樂、反覺得愁苦.後來卻為那經練過的人、結出平安的果子、就是義.\\
所以你們要把下垂的手、發酸的腿、挺起來.\\
也要為自己的腳把道路修直了、使瘸子不至歪腳、反得痊癒。〔歪腳或作差路〕\\
你們總要謹慎、不可棄絕那向你們說話的.因為那些棄絕在地上警戒他們的、尚且不能逃罪、何況我們違背那從天上警戒我們的呢。\\
當時他的聲音震動了地.但如今他應許說、『再一次我不單要震動地、還要震動天。』\\
這再一次的話、是指明被震動的、就是受造之物、都要挪去、使那不被震動的常存。\\

\end{CJK}
\end{minipage}}
\end{spacing}
\vspace*{\fill}
\end{document}
