\documentclass[10pt]{article} % use larger type; default would be 10pt
\usepackage[utf8]{inputenc}       % кодування документа; замість cp866nav
\usepackage[top=0.2in,bottom=0.0in,left=0.2in,right=0.5in]{geometry}
\usepackage[russian,english]{babel} % національна локалізація; може бути декілька
\usepackage{setspace}
\usepackage{CJKutf8}
\usepackage{mdframed}
\usepackage{setspace}
\title{Divine Liturgy\\Reading from The Epistles}
\author{Week 11 after the Pentecost\vspace{
-5ex%for author
}}
\date{\vspace{
-11ex%for date
}}
\begin{document}
\pagenumbering{gobble}
\begin{otherlanguage*}{russian}
\maketitle
\end{otherlanguage*}
\vspace*{\fill}
\Large%for eng/rus
\singlespacing %\\onehalfspacing \\doublespacing % for eng/rus
\framebox[\textwidth]{
\begin{minipage}[t]{0.45\textwidth}
\begin{otherlanguage*}{russian}
\textbf{1 Кор., 141 зач., IX, 2-12.}\\
\Large%for eng/rus
Если для других я не Апостол, то для вас Апостол; || ибо печать моего апостольства - вы в Господе.
\\
Вот мое защищение против осуждающих меня.
\\
Или мы не имеем власти есть и пить?
\\
Или не имеем власти иметь спутницею сестру жену, как и прочие Апостолы, и братья Господни, и Кифа?
\\
Или один я и Варнава не имеем власти не работать?
\\
Какой воин служит когда-либо на своем содержании? Кто, насадив виноград, не ест плодов его? Кто, пася стадо, не ест молока от стада?
\\
По человеческому ли только рассуждению я это говорю? Не то же ли говорит и закон?
\\
Ибо в Моисеевом законе написано: не заграждай рта у вола молотящего. О волах ли печется Бог?
\\
Или, конечно, для нас говорится? Так, для нас это написано; ибо, кто пашет, должен пахать с надеждою, и кто молотит, должен молотить с надеждою получить ожидаемое.
\\
Если мы посеяли в вас духовное, велико ли то, если пожнем у вас телесное?
\\
Если другие имеют у вас власть, не паче ли мы? Однако мы не пользовались сею властью, но все переносим, дабы не поставить какой преграды благовествованию Христову.
\end{otherlanguage*}
\end{minipage}
\hfill
\begin{minipage}[t]{0.45\textwidth}

\textbf{1 Corinthians 9:2--9:12.}\\
If I be not an apostle unto others, yet doubtless I am to you: for the seal of mine apostleship are ye in the Lord.\\
Mine answer to them that do examine me is this,\\
Have we not power to eat and to drink?\\
Have we not power to lead about a sister, a wife, as well as other apostles, and as the brethren of the Lord, and Cephas?\\
Or I only and Barnabas, have not we power to forbear working?\\
Who goeth a warfare any time at his own charges? who planteth a vineyard, and eateth not of the fruit thereof? or who feedeth a flock, and eateth not of the milk of the flock?\\
Say I these things as a man? or saith not the law the same also?\\
For it is written in the law of Moses, Thou shalt not muzzle the mouth of the ox that treadeth out the corn. Doth God take care for oxen?\\
Or saith he it altogether for our sakes? For our sakes, no doubt, this is written: that he that ploweth should plow in hope; and that he that thresheth in hope should be partaker of his hope.\\
If we have sown unto you spiritual things, is it a great thing if we shall reap your carnal things?\\
If others be partakers of this power over you, are not we rather? Nevertheless we have not used this power; but suffer all things, lest we should hinder the gospel of Christ.
\end{minipage}}
\vspace*{\fill}
\newpage
\Huge%for chi
\vspace*{\fill}
\begin{spacing}{1.1}%for chinese
\framebox[\textwidth]{
\begin{minipage}[t]{\textwidth}
\begin{CJK}{UTF8}{bsmi}
\textbf{哥林多前書 9:2--9:12.}\\
假若在別人我不是使徒、在你們我總是使徒.因為你們在主裡正是我作使徒的印證。\\
我對那盤問我的人、就是這樣分訴。\\
難道我們沒有權柄靠福音喫喝麼。\\
難道我們沒有權柄娶信主的姊妹為妻、帶著一同往來、彷彿其餘的使徒、和主的弟兄、並磯法一樣麼。\\
獨有我與巴拿巴沒有權柄不作工麼。\\
有誰當兵、自備糧餉呢.有誰栽葡萄園、不喫園裡的果子呢.有誰牧養牛羊、不喫牛羊的奶呢。\\
我說這話、豈是照人的意見.律法不也是這樣說麼。\\
就如摩西的律法記著說、『牛在場上踹榖的時候、不可籠住他的嘴。』難道 神所掛念的是牛麼.\\
不全是為我們說的麼.分明是為我們說的.因為耕種的當存著指望去耕種.打場的也當存得糧的指望去打場。\\
我們若把屬靈的種子撒在你們中間、就是從你們收割奉養肉身之物、這還算大事麼。\\
若別人在你們身上有這權柄、何況我們呢.然而我們沒有用過這權柄、倒凡事忍受、免得基督的福音被阻隔。
\end{CJK}
\end{minipage}}
\end{spacing}
\vspace*{\fill}
\end{document}
