\documentclass[10pt]{article} % use larger type; default would be 10pt
\usepackage[utf8]{inputenc}       % кодування документа; замість cp866nav
\usepackage{extsizes}
\usepackage[top=0.5in,bottom=0.5in,left=0.5in,right=0.5in]{geometry}
\usepackage[russian,english]{babel} % національна локалізація; може бути декілька
\usepackage{setspace}
\usepackage{CJKutf8}
\usepackage{mdframed}
\usepackage{extsizes}
\title{Divine Liturgy\\Reading from The Gospels}
\author{Week 15 after the Pentecost\vspace{
-3ex%for author
}}
\date{\vspace{
-7ex%for date
}}
\begin{document}
\pagenumbering{gobble}
\begin{otherlanguage*}{russian}
\maketitle
\end{otherlanguage*}
\vspace*{\fill}
\Large%for eng/rus
\singlespacing %\\onehalfspacing \\doublespacing % for eng/rus
\framebox[\textwidth]{
\begin{minipage}[t]{0.45\textwidth}
\begin{otherlanguage*}{russian}
\textbf{Мф., 92 зач., XXII, 35-46.}\\
И один из них, законник, искушая Его, спросил, говоря:
\\
Учитель! какая наибольшая заповедь в законе?
\\
Иисус сказал ему: возлюби Господа Бога твоего всем сердцем твоим и всею душею твоею и всем разумением твоим:
\\
сия есть первая и наибольшая заповедь;
\\
вторая же подобная ей: возлюби ближнего твоего, как самого себя;
\\
на сих двух заповедях утверждается весь закон и пророки.
\\
Когда же собрались фарисеи, Иисус спросил их:
\\
что вы думаете о Христе? чей Он сын? Говорят Ему: Давидов.
\\
Говорит им: как же Давид, по вдохновению, называет Его Господом, когда говорит:
\\
сказал Господь Господу моему: седи одесную Меня, доколе положу врагов Твоих в подножие ног Твоих?
\\
Итак, если Давид называет Его Господом, как же Он сын ему?
\\
И никто не мог отвечать Ему ни слова; и с того дня никто уже не смел спрашивать Его.
\\

\end{otherlanguage*}
\end{minipage}
\hfill
\begin{minipage}[t]{0.45\textwidth}

\textbf{Matthew 22:35--22:46.}\\
Then one of them, which was a lawyer, asked him a question, tempting him, and saying,\\
Master, which is the great commandment in the law?\\
Jesus said unto him, Thou shalt love the Lord thy God with all thy heart, and with all thy soul, and with all thy mind.\\
This is the first and great commandment.\\
And the second is like unto it, Thou shalt love thy neighbour as thyself.\\
On these two commandments hang all the law and the prophets.\\
While the Pharisees were gathered together, Jesus asked them,\\
Saying, What think ye of Christ? whose son is he? They say unto him, The Son of David.\\
He saith unto them, How then doth David in spirit call him Lord, saying,\\
The LORD said unto my Lord, Sit thou on my right hand, till I make thine enemies thy footstool?\\
If David then call him Lord, how is he his son?\\
And no man was able to answer him a word, neither durst any man from that day forth ask him any more questions.\\
\end{minipage}}
\vspace*{\fill}
\newpage
\Huge%for chi
\vspace*{\fill}
\begin{spacing}{1.0}
\framebox[\textwidth]{
\begin{minipage}[t]{\textwidth}
\begin{CJK}{UTF8}{bsmi}
\textbf{馬太福音 22:35--22:46.}\\
內中有一個人是律法師、要試探耶穌、就問他說、\\
夫子、律法上的誡命、那一條是最大的呢。\\
耶穌對他說、你要盡心、盡性、盡意、愛主你的 神。\\
這是誡命中的第一、且是最大的。\\
其次也相倣、就是要愛人如己。\\
這兩條誡命、是律法和先知一切道理的總綱。\\
法利賽人聚集的時候、耶穌問他們說、\\
論到基督、你們的意見如何.他是誰的子孫呢。他們回答說、是大衛的子孫。\\
耶穌說、這樣、大衛被聖靈感動、怎麼還稱他為主.說、\\
『主對我主說、你坐在我的右邊、等我把你仇敵、放在你的腳下。』\\
大衛既稱他為主、他怎麼又是大衛的子孫呢。\\
他們沒有一個人能回答一言.從那日以後、也沒有人敢再問他甚麼。\\

\end{CJK}
\end{minipage}}
\end{spacing}
\vspace*{\fill}
\end{document}
