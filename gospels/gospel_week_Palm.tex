\documentclass[10pt]{article} % use larger type; default would be 10pt
\usepackage[utf8]{inputenc}       % кодування документа; замість cp866nav
\usepackage[top=0.0in,bottom=0.0in,left=0.0in,right=0.3in]{geometry}
\usepackage[russian,english]{babel} % національна локалізація; може бути декілька
\usepackage{setspace}
\usepackage{CJKutf8}
\usepackage{mdframed}
\title{\vspace{-2.5ex}Divine Liturgy\\Reading from The Gospels\vspace{-1ex}}
\author{Palm Sunday\vspace{
-11ex%for author
}}
\date{\vspace{
-8ex%for date
}}
\begin{document}
\pagenumbering{gobble}
\begin{otherlanguage*}{russian}
\maketitle
\end{otherlanguage*}
\vspace*{\fill}
\large%for eng/rus
\singlespacing %\\onehalfspacing \\doublespacing % for eng/rus
\framebox[\textwidth]{
\begin{minipage}[t]{0.5\textwidth}
\begin{otherlanguage*}{russian}
\textbf{Ин., 41 зач., XII, 1-18.}\\
За шесть дней до Пасхи пришел Иисус в Вифанию, где был Лазарь умерший, которого Он воскресил из мертвых.
\\
Там приготовили Ему вечерю, и Марфа служила, и Лазарь был одним из возлежавших с Ним.
\\
Мария же, взяв фунт нардового чистого драгоценного мира, помазала ноги Иисуса и отерла волосами своими ноги Его; и дом наполнился благоуханием от мира.
\\
Тогда один из учеников Его, Иуда Симонов Искариот, который хотел предать Его, сказал:
\\
Для чего бы не продать это миро за триста динариев и не раздать нищим?
\\
Сказал же он это не потому, чтобы заботился о нищих, но потому что был вор. Он имел при себе денежный ящик и носил, что туда опускали.
\\
Иисус же сказал: оставьте ее; она сберегла это на день погребения Моего.
\\
Ибо нищих всегда имеете с собою, а Меня не всегда.
\\
Многие из Иудеев узнали, что Он там, и пришли не только для Иисуса, но чтобы видеть и Лазаря, которого Он воскресил из мертвых.
\\
Первосвященники же положили убить и Лазаря,
\\
потому что ради него многие из Иудеев приходили и веровали в Иисуса.
\\
На другой день множество народа, пришедшего на праздник, услышав, что Иисус идет в Иерусалим,
\\
взяли пальмовые ветви, вышли навстречу Ему и восклицали: осанна! благословен грядущий во имя Господне, Царь Израилев!
\\
Иисус же, найдя молодого осла, сел на него, как написано:
\\
Не бойся, дщерь Сионова! се, Царь твой грядет, сидя на молодом осле.
\\
Ученики Его сперва не поняли этого; но когда прославился Иисус, тогда вспомнили, что так было о Нем написано, и это сделали Ему.
\\
Народ, бывший с Ним прежде, свидетельствовал, что Он вызвал из гроба Лазаря и воскресил его из мертвых.
\\
Потому и встретил Его народ, ибо слышал, что Он сотворил это чудо.
\\

\end{otherlanguage*}
\end{minipage}
\hfill
\begin{minipage}[t]{0.45\textwidth}

\textbf{John 12:1--12:18.}\\
Then Jesus six days before the passover came to Bethany, where Lazarus was which had been dead, whom he raised from the dead.\\
There they made him a supper; and Martha served: but Lazarus was one of them that sat at the table with him.\\
Then took Mary a pound of ointment of spikenard, very costly, and anointed the feet of Jesus, and wiped his feet with her hair: and the house was filled with the odour of the ointment.\\
Then saith one of his disciples, Judas Iscariot, Simon's son, which should betray him,\\
Why was not this ointment sold for three hundred pence, and given to the poor?\\
This he said, not that he cared for the poor; but because he was a thief, and had the bag, and bare what was put therein.\\
Then said Jesus, Let her alone: against the day of my burying hath she kept this.\\
For the poor always ye have with you; but me ye have not always.\\
Much people of the Jews therefore knew that he was there: and they came not for Jesus' sake only, but that they might see Lazarus also, whom he had raised from the dead.\\
But the chief priests consulted that they might put Lazarus also to death;\\
Because that by reason of him many of the Jews went away, and believed on Jesus.\\
On the next day much people that were come to the feast, when they heard that Jesus was coming to Jerusalem,\\
Took branches of palm trees, and went forth to meet him, and cried, Hosanna: Blessed is the King of Israel that cometh in the name of the Lord.\\
And Jesus, when he had found a young ass, sat thereon; as it is written,\\
Fear not, daughter of Sion: behold, thy King cometh, sitting on an ass's colt.\\
These things understood not his disciples at the first: but when Jesus was glorified, then remembered they that these things were written of him, and that they had done these things unto him.\\
The people therefore that was with him when he called Lazarus out of his grave, and raised him from the dead, bare record.\\
For this cause the people also met him, for that they heard that he had done this miracle.\\

\end{minipage}}
\vspace*{\fill}
\newpage
\huge%for chi
\vspace*{\fill}
\begin{spacing}{1.0}
\framebox[\textwidth]{
\begin{minipage}[t]{\textwidth}
\begin{CJK}{UTF8}{bsmi}
\textbf{約翰福音 12:1--12:18.}\\
逾越節前六日、耶穌來到伯大尼、就是他叫拉撒路從死裡復活之處。\\
有人在那裡給耶穌預備筵席.馬大伺候、拉撒路也在那同耶穌坐席的人中。\\
馬利亞就拿著一斤極貴的真哪噠香膏、抹耶穌的腳、又用自己頭髮去擦.屋裡就滿了膏的香氣。\\
有一個門徒、就是那將要賣耶穌的加略人猶大、\\
說、這香膏為甚麼不賣三十兩銀子賙濟窮人呢。\\
他說這話、並不是掛念窮人、乃因他是個賊、又帶著錢囊、常取其中所存的。\\
耶穌說、由他罷、他是為我安葬之日存留的。\\
因為常有窮人和你們同在.只是你們不常有我。\\
有許多猶太人知道耶穌在那裡、就來了、不但是為耶穌的緣故、也是要看他從死裡所復活的拉撒路。\\
但祭司長商議連拉撒路也要殺了.\\
因有好些猶太人、為拉撒路的緣故、回去信了耶穌。\\
第二天有許多上來過節的人、聽見耶穌將到耶路撒冷、\\
就拿著棕樹枝、出去迎接他、喊著說、和散那、奉主名來的以色列王、是應當稱頌的。\\
耶穌得了一個驢駒、就騎上.如經上所記的說、\\
『錫安的民哪、〔民原文作女子〕不要懼怕、你的王騎著驢駒來了。』\\
這些事門徒起先不明白.等到耶穌得了榮耀以後、纔想起這話是指著他寫的、並且眾人果然向他這樣行了。\\
當耶穌呼喚拉撒路叫他從死復活出墳墓的時候、同耶穌在那裡的眾人、就作見證。\\
眾人因聽見耶穌行了這神蹟、就去迎接他。\\

\end{CJK}
\end{minipage}}
\end{spacing}
\vspace*{\fill}
\end{document}
