\documentclass[10pt]{article} % use larger type; default would be 10pt
\usepackage[utf8]{inputenc}       % кодування документа; замість cp866nav
\usepackage{extsizes}
\usepackage[top=0.3in,bottom=0.5in,left=0.3in,right=0.5in]{geometry}
\usepackage[russian,english]{babel} % національна локалізація; може бути декілька
\usepackage{setspace}
\usepackage{CJKutf8}
\usepackage{mdframed}
\usepackage{setspace}
\title{Divine Liturgy\\Reading from The Epistles}
\author{Week 27 after the Pentecost\vspace{
-3ex%for author
}}
\date{\vspace{
-5ex%for date
}}
\begin{document}
\pagenumbering{gobble}
\begin{otherlanguage*}{russian}
\maketitle
\end{otherlanguage*}
\vspace*{\fill}
\Large%for eng/rus
\singlespacing %\\onehalfspacing \\doublespacing % for eng/rus
\framebox[\textwidth]{
\begin{minipage}[t]{0.45\textwidth}
\begin{otherlanguage*}{russian}
\textbf{Еф., 233 зач., VI, 10-17.}\\
\Large%for eng/rus
Наконец, братия мои, укрепляйтесь Господом и могуществом силы Его.
\\
Облекитесь во всеоружие Божие, чтобы вам можно было стать против козней диавольских,
\\
потому что наша брань не против крови и плоти, но против начальств, против властей, против мироправителей тьмы века сего, против духов злобы поднебесных.
\\
Для сего приимите всеоружие Божие, дабы вы могли противостать в день злой и, все преодолев, устоять.
\\
Итак станьте, препоясав чресла ваши истиною и облекшись в броню праведности,
\\
и обув ноги в готовность благовествовать мир;
\\
а паче всего возьмите щит веры, которым возможете угасить все раскаленные стрелы лукавого;
\\
и шлем спасения возьмите, и меч духовный, который есть Слово Божие.
\\

\end{otherlanguage*}
\end{minipage}
\hfill
\begin{minipage}[t]{0.45\textwidth}

\textbf{Ephesians 6:10--6:17.}\\
Finally, my brethren, be strong in the Lord, and in the power of his might.\\
Put on the whole armour of God, that ye may be able to stand against the wiles of the devil.\\
For we wrestle not against flesh and blood, but against principalities, against powers, against the rulers of the darkness of this world, against spiritual wickedness in high places.\\
Wherefore take unto you the whole armour of God, that ye may be able to withstand in the evil day, and having done all, to stand.\\
Stand therefore, having your loins girt about with truth, and having on the breastplate of righteousness;\\
And your feet shod with the preparation of the gospel of peace;\\
Above all, taking the shield of faith, wherewith ye shall be able to quench all the fiery darts of the wicked.\\
And take the helmet of salvation, and the sword of the Spirit, which is the word of God:\\

\end{minipage}}
\vspace*{\fill}
\newpage
\Huge%for chi
\vspace*{\fill}
\begin{spacing}{1.3}%for chinese
\framebox[\textwidth]{
\begin{minipage}[t]{\textwidth}
\begin{CJK}{UTF8}{bsmi}
\textbf{以弗所書 6:10--6:17.}\\
我還有末了的話、你們要靠著主、倚賴他的大能大力、作剛強的人。\\
要穿戴 神所賜的全副軍裝、就能抵擋魔鬼的詭計。\\
因我們並不是與屬血氣的爭戰、乃是與那些執政的、掌權的、管轄這幽暗世界的、以及天空屬靈氣的惡魔爭戰。〔兩爭戰原文都作摔跤〕\\
所以要拿起 神所賜的全副軍裝、好在磨難的日子、抵擋仇敵、並且成就了一切、還能站立得住。\\
所以要站穩了、用真理當作帶子束腰、用公義當作護心鏡遮胸.\\
又用平安的福音、當作預備走路的鞋穿在腳上.\\
此外又拿著信德當作籐牌、可以滅盡那惡者一切的火箭.\\
並戴上救恩的頭盔、拿著聖靈的寶劍、就是 神的道.\\

\end{CJK}
\end{minipage}}
\end{spacing}
\vspace*{\fill}
\end{document}
