\documentclass[10pt]{article} % use larger type; default would be 10pt
\usepackage[utf8]{inputenc}       % кодування документа; замість cp866nav
\usepackage[left=0.2in,right=0.4in,top=0.0in,bottom=0.1in]{geometry}
\usepackage[russian,english]{babel} % національна локалізація; може бути декілька
\usepackage{setspace}
\usepackage{CJKutf8}
\usepackage{mdframed}
\title{Divine Liturgy\\Reading from The Gospels}
\author{Week about the Prodigal Son\vspace{
-12ex%for author
}}
\date{\vspace{
-7ex%for date
}}
\begin{document}
\pagenumbering{gobble}
\begin{otherlanguage*}{russian}
\maketitle
\end{otherlanguage*}
\vspace*{\fill}
\normalsize%for eng/rus
\singlespacing %\\onehalfspacing \\doublespacing % for eng/rus
\framebox[\textwidth]{
\begin{minipage}[t]{0.45\textwidth}
\begin{otherlanguage*}{russian}
\textbf{Лк., 79 зач., XV, 11-32.}\\
Еще сказал: у некоторого человека было два сына;\\
и сказал младший из них отцу: отче! дай мне следующую мне часть имения. И отец разделил им имение.\\
По прошествии немногих дней младший сын, собрав всё, пошел в дальнюю сторону и там расточил имение свое, живя распутно.\\
Когда же он прожил всё, настал великий голод в той стране, и он начал нуждаться;\\
и пошел, пристал к одному из жителей страны той, а тот послал его на поля свои пасти свиней;\\
и он рад был наполнить чрево свое рожк\'{а}ми, которые ели свиньи, но никто не давал ему.\\
Придя же в себя, сказал: сколько наемников у отца моего избыточествуют хлебом, а я умираю от голода;\\
встану, пойду к отцу моему и скажу ему: отче! я согрешил против неба и пред тобою\\
и уже недостоин называться сыном твоим; прими меня в число наемников твоих.\\
Встал и пошел к отцу своему. И когда он был еще далеко, увидел его отец его и сжалился; и, побежав, пал ему на шею и целовал его.\\
Сын же сказал ему: отче! я согрешил против неба и пред тобою и уже недостоин называться сыном твоим.\\
А отец сказал рабам своим: принесите лучшую одежду и оденьте его, и дайте перстень на руку его и обувь на ноги;\\
и приведите откормленного теленка, и заколите; станем есть и веселиться!\\
ибо этот сын мой был мертв и ожил, пропадал и нашелся. И начали веселиться.\\
Старший же сын его был на поле; и возвращаясь, когда приблизился к дому, услышал пение и ликование;\\
и, призвав одного из слуг, спросил: что это такое?\\
Он сказал ему: брат твой пришел, и отец твой заколол откормленного теленка, потому что принял его здоровым.\\
Он осердился и не хотел войти. Отец же его, выйдя, звал его.\\
Но он сказал в ответ отцу: вот, я столько лет служу тебе и никогда не преступал приказания твоего, но ты никогда не дал мне и козлёнка, чтобы мне повеселиться с друзьями моими;\\
а когда этот сын твой, расточивший имение своё с блудницами, пришел, ты заколол для него откормленного теленка.\\
Он же сказал ему: сын мой! ты всегда со мною, и всё мое твое,\\
а о том надобно было радоваться и веселиться, что брат твой сей был мертв и ожил, пропадал и нашелся. \\
\end{otherlanguage*}
\end{minipage}
\hfill
\begin{minipage}[t]{0.45\textwidth}

\textbf{Luke 15:11 -- 15:32.}\\
And he said, A certain man had two sons:\\
And the younger of them said to his father, Father, give me the portion of goods that falleth to me. And he divided unto them his living.\\
And not many days after the younger son gathered all together, and took his journey into a far country, and there wasted his substance with riotous living.\\
And when he had spent all, there arose a mighty famine in that land; and he began to be in want.\\
And he went and joined himself to a citizen of that country; and he sent him into his fields to feed swine.\\
And he would fain have filled his belly with the husks that the swine did eat: and no man gave unto him.\\
And when he came to himself, he said, How many hired servants of my father's have bread enough and to spare, and I perish with hunger!\\
I will arise and go to my father, and will say unto him, Father, I have sinned against heaven, and before thee,\\
And am no more worthy to be called thy son: make me as one of thy hired servants.\\
And he arose, and came to his father. But when he was yet a great way off, his father saw him, and had compassion, and ran, and fell on his neck, and kissed him.\\
And the son said unto him, Father, I have sinned against heaven, and in thy sight, and am no more worthy to be called thy son.\\
But the father said to his servants, Bring forth the best robe, and put it on him; and put a ring on his hand, and shoes on his feet:\\
And bring hither the fatted calf, and kill it; and let us eat, and be merry:\\
For this my son was dead, and is alive again; he was lost, and is found. And they began to be merry.\\
Now his elder son was in the field: and as he came and drew nigh to the house, he heard musick and dancing.\\
And he called one of the servants, and asked what these things meant.\\
And he said unto him, Thy brother is come; and thy father hath killed the fatted calf, because he hath received him safe and sound.\\
And he was angry, and would not go in: therefore came his father out, and intreated him.\\
And he answering said to his father, Lo, these many years do I serve thee, neither transgressed I at any time thy commandment: and yet thou never gavest me a kid, that I might make merry with my friends:\\
But as soon as this thy son was come, which hath devoured thy living with harlots, thou hast killed for him the fatted calf.\\
And he said unto him, Son, thou art ever with me, and all that I have is thine.\\
It was meet that we should make merry, and be glad: for this thy brother was dead, and is alive again; and was lost, and is found.\\

\end{minipage}}
\vspace*{\fill}
\newpage
\Large%for chi
\vspace*{\fill}
\begin{spacing}{1.3}
\framebox[\textwidth]{
\begin{minipage}[t]{\textwidth}
\begin{CJK}{UTF8}{bsmi}
\textbf{路加福音 15:11 -- 15:32.}\\
耶穌又說:「一個人有兩個兒子。 \\
小兒子對父親說:『父親,請你把我應得的家業分給我。』他父親就把財產分給他們。 \\
過了不多幾天,小兒子把他一切所有的都收拾起來,往遠方去了。在那裏,他任意放蕩,浪費錢財。 \\
他耗盡了一切所有的,又恰逢那地方有大饑荒,就窮困起來。 \\
於是他去投靠當地的一個居民,那人打發他到田裏去放豬。 \\
他恨不得拿豬所吃的豆莢充飢,也沒有人給他甚麼吃的。 \\
他醒悟過來,就說:『我父親有多少雇工,糧食有餘,我倒在這裏餓死嗎? \\
我要起來,到我父親那裏去,對他說:父親!我得罪了天,又得罪了你, \\
從今以後,我不配稱為你的兒子,把我當作一個雇工吧。』 \\
於是他起來,往他父親那裏去。相離還遠,他父親看見,就動了慈心,跑去擁抱着他,連連親他。 \\
兒子對他說:『父親!我得罪了天,又得罪了你,從今以後,我不配稱為你的兒子。』 \\
父親卻吩咐僕人:『快把那上好的袍子拿出來給他穿,把戒指戴在他指頭上,把鞋穿在他腳上, \\
把那肥牛犢牽來宰了,我們來吃喝慶祝; \\
因為我這個兒子是死而復活,失而復得的。』他們就開始慶祝。 \\
「那時,大兒子正在田裏。他回來,離家不遠時,聽見奏樂跳舞的聲音, \\
就叫一個僮僕來,問是甚麼事。 \\
僮僕對他說:『你弟弟回來了,你父親因為他無災無病地回來,把肥牛犢宰了。』 \\
大兒子就生氣,不肯進去,他父親出來勸他。 \\
他對父親說:『你看,我服侍你這麼多年,從來沒有違背過你的命令,而你從來沒有給我一隻小山羊,叫我和朋友們一同快樂。 \\
但你這個兒子和娼妓吃光了你的財產,他一回來,你倒為他宰了肥牛犢。』 \\
父親對他說:『兒啊!你常和我同在,我所有的一切都是你的; \\
可是你這個弟弟是死而復活,失而復得的,所以我們理當歡喜慶祝。』」 \\

\end{CJK}
\end{minipage}}
\end{spacing}
\vspace*{\fill}
\end{document}
