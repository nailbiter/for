%simple
\documentclass[10pt]{article} % use larger type; default would be 10pt
\usepackage[utf8]{inputenc}       % кодування документа; замість cp866nav
\usepackage{extsizes}
\usepackage[top=0.4in,bottom=0.5in,left=0.5in,right=0.5in]{geometry}
\usepackage[russian,english]{babel} % національна локалізація; може бути декілька
\usepackage{setspace}
\usepackage{CJKutf8}
\usepackage{mdframed}
\usepackage{extsizes}
\usepackage{setspace}
\title{Divine Liturgy\\Reading from The Epistles}
\author{Week before the Great Lent\vspace{
-3ex%for author
}}
\date{\vspace{
-5ex%for date
}}
\begin{document}
\pagenumbering{gobble}
\begin{otherlanguage*}{russian}
\maketitle
\end{otherlanguage*}
\vspace*{\fill}
\Large%for eng/rus
\singlespacing %\\onehalfspacing \\doublespacing % for eng/rus
\framebox[\textwidth]{
\begin{minipage}[t]{0.45\textwidth}
\begin{otherlanguage*}{russian}
\textbf{Рим., 112 зач., XIII, 11 - XIV, 4.}\\
\Large%for eng/rus
Так поступайте, зная время, что наступил уже час пробудиться нам от сна. || Ибо ныне ближе к нам спасение, нежели когда мы уверовали.
\\
Ночь прошла, а день приблизился: итак отвергнем дела тьмы и облечемся в оружия света.
\\
Как днем, будем вести себя благочинно, не предаваясь ни пированиям и пьянству, ни сладострастию и распутству, ни ссорам и зависти;
\\
но облекитесь в Господа нашего Иисуса Христа, и попечения о плоти не превращайте в похоти.
\\
Немощного в вере принимайте без споров о мнениях.
\\
Ибо иной уверен, что можно есть все, а немощный ест овощи.
\\
Кто ест, не уничижай того, кто не ест; и кто не ест, не осуждай того, кто ест, потому что Бог принял его.
\\
Кто ты, осуждающий чужого раба? Перед своим Господом стоит он, или падает. И будет восставлен, ибо силен Бог восставить его.
\\

\end{otherlanguage*}
\end{minipage}
\hfill
\begin{minipage}[t]{0.45\textwidth}

\textbf{Romans 13:11--14:4.}\\
And that, knowing the time, that now it is high time to awake out of sleep: for now is our salvation nearer than when we believed.\\
The night is far spent, the day is at hand: let us therefore cast off the works of darkness, and let us put on the armour of light.\\
Let us walk honestly, as in the day; not in rioting and drunkenness, not in chambering and wantonness, not in strife and envying.\\
But put ye on the Lord Jesus Christ, and make not provision for the flesh, to fulfil the lusts thereof.\\
Him that is weak in the faith receive ye, but not to doubtful disputations.\\
For one believeth that he may eat all things: another, who is weak, eateth herbs.\\
Let not him that eateth despise him that eateth not; and let not him which eateth not judge him that eateth: for God hath received him.\\
Who art thou that judgest another man's servant? to his own master he standeth or falleth. Yea, he shall be holden up: for God is able to make him stand.
\end{minipage}}
\vspace*{\fill}
\newpage
\Huge%for chi
\vspace*{\fill}
\begin{spacing}{1.3}%for chinese
\framebox[\textwidth]{
\begin{minipage}[t]{\textwidth}
\begin{CJK}{UTF8}{bsmi}
\textbf{羅馬書 13:11--14:4.}\\
再者、你們曉得現今就是該趁早睡醒的時候、因為我們得救、現今比初信的時候更近了。\\
黑夜已深、白晝將近.我們就當脫去暗昧的行為、帶上光明的兵器。\\
行事為人要端正、好像行在白晝.不可荒宴醉酒.不可好色邪蕩.不可爭競嫉妒。\\
總要披戴主耶穌基督、不要為肉體安排、去放縱私慾。\\
信心軟弱的、你們要接納、但不要辯論所疑惑的事。\\
有人信百物都可喫.但那軟弱的、只喫蔬菜。\\
喫的人不可輕看不喫的人.不喫的人不可論斷喫的人.因為 神已經收納他了。\\
你是誰、竟論斷別人的僕人呢。他或站住、或跌倒、自有他的主人在.而且他也必要站住.因為主能使他站住。
\end{CJK}
\end{minipage}}
\end{spacing}
\vspace*{\fill}
\end{document}
