\documentclass[10pt]{article} % use larger type; default would be 10pt
\usepackage[utf8]{inputenc}       % кодування документа; замість cp866nav
\usepackage{extsizes}
\usepackage[top=0.5in,bottom=0.5in,left=0.4in,right=0.4in]{geometry}
\usepackage[russian,english]{babel} % національна локалізація; може бути декілька
\usepackage{setspace}
\usepackage{CJKutf8}
\usepackage{mdframed}
\usepackage{setspace}
\title{Divine Liturgy\\Reading from The Epistles}
\author{Week 31 after the Pentecost\vspace{
-3ex%for author
}}
\date{\vspace{
-5ex%for date
}}
\begin{document}
\pagenumbering{gobble}
\begin{otherlanguage*}{russian}
\maketitle
\end{otherlanguage*}
\vspace*{\fill}
\normalfont%for eng/rus
\singlespacing %\\onehalfspacing \\doublespacing % for eng/rus
\framebox[\textwidth]{
\begin{minipage}[t]{0.45\textwidth}
\begin{otherlanguage*}{russian}
\textbf{Евр., 328 зач., XI, 9-10, 17-23, 32-40.}\\
\normalfont%for eng/rus
Верою обитал он на земле обетованной, как на чужой, и жил в шатрах с Исааком и Иаковом, сонаследниками того же обетования;
\\
ибо он ожидал города, имеющего основание, которого художник и строитель Бог.
\\
Верою Авраам, будучи искушаем, принес в жертву Исаака и, имея обетование, принес единородного,
\\
о котором было сказано: в Исааке наречется тебе семя.
\\
Ибо он думал, что Бог силен и из мертвых воскресить, почему и получил его в предзнаменование.
\\
Верою в будущее Исаак благословил Иакова и Исава.
\\
Верою Иаков, умирая, благословил каждого сына Иосифова и поклонился на верх жезла своего.
\\
Верою Иосиф, при кончине, напоминал об исходе сынов Израилевых и завещал о костях своих.
\\
Верою Моисей по рождении три месяца скрываем был родителями своими, ибо видели они, что дитя прекрасно, и не устрашились царского повеления.
\\
И что еще скажу? Недостанет мне времени, чтобы повествовать о Гедеоне, о Вараке, о Самсоне и Иеффае, о Давиде, Самуиле и (других) пророках,
\\
которые верою побеждали царства, творили правду, получали обетования, заграждали уста львов,
\\
угашали силу огня, избегали острия меча, укреплялись от немощи, были крепки на войне, прогоняли полки чужих;
\\
жены получали умерших своих воскресшими; иные же замучены были, не приняв освобождения, дабы получить лучшее воскресение;
\\
другие испытали поругания и побои, а также узы и темницу,
\\
были побиваемы камнями, перепиливаемы, подвергаемы пытке, умирали от меча, скитались в милотях и козьих кожах, терпя недостатки, скорби, озлобления;
\\
те, которых весь мир не был достоин, скитались по пустыням и горам, по пещерам и ущельям земли.
\\
И все сии, свидетельствованные в вере, не получили обещанного,
\\
потому что Бог предусмотрел о нас нечто лучшее, дабы они не без нас достигли совершенства.
\end{otherlanguage*}
\end{minipage}
\hfill
\begin{minipage}[t]{0.45\textwidth}

\textbf{Hebrews 11:9--11:10, 11:17--11:23, 11:32--11:40.}\\
By faith he sojourned in the land of promise, as in a strange country, dwelling in tabernacles with Isaac and Jacob, the heirs with him of the same promise:\\
For he looked for a city which hath foundations, whose builder and maker is God.\\
By faith Abraham, when he was tried, offered up Isaac: and he that had received the promises offered up his only begotten son,\\
Of whom it was said, That in Isaac shall thy seed be called:\\
Accounting that God was able to raise him up, even from the dead; from whence also he received him in a figure.\\
By faith Isaac blessed Jacob and Esau concerning things to come.\\
By faith Jacob, when he was a dying, blessed both the sons of Joseph; and worshipped, leaning upon the top of his staff.\\
By faith Joseph, when he died, made mention of the departing of the children of Israel; and gave commandment concerning his bones.\\
By faith Moses, when he was born, was hid three months of his parents, because they saw he was a proper child; and they were not afraid of the king's commandment.\\
And what shall I more say? for the time would fail me to tell of Gedeon, and of Barak, and of Samson, and of Jephthae; of David also, and Samuel, and of the prophets:\\
Who through faith subdued kingdoms, wrought righteousness, obtained promises, stopped the mouths of lions,\\
Quenched the violence of fire, escaped the edge of the sword, out of weakness were made strong, waxed valiant in fight, turned to flight the armies of the aliens.\\
Women received their dead raised to life again: and others were tortured, not accepting deliverance; that they might obtain a better resurrection:\\
And others had trial of cruel mockings and scourgings, yea, moreover of bonds and imprisonment:\\
They were stoned, they were sawn asunder, were tempted, were slain with the sword: they wandered about in sheepskins and goatskins; being destitute, afflicted, tormented;\\
(Of whom the world was not worthy:) they wandered in deserts, and in mountains, and in dens and caves of the earth.\\
And these all, having obtained a good report through faith, received not the promise:\\
God having provided some better thing for us, that they without us should not be made perfect.
\end{minipage}}
\vspace*{\fill}
\newpage
\LARGE%for chi
\vspace*{\fill}
\begin{spacing}{1.1}%for chinese
\framebox[\textwidth]{
\begin{minipage}[t]{\textwidth}
\begin{CJK}{UTF8}{bsmi}
\textbf{希伯來書 11:9--11:10, 11:17--11:23, 11:32--11:40.}\\
他因著信、就在所應許之地作客、好像在異地居住帳棚、與那同蒙一個應許的以撒、雅各一樣。\\
因為他等候那座有根基的城、就是 神所經營所建造的。\\
亞伯拉罕因著信、被試驗的時候、就把以撒獻上.這便是那歡喜領受應許的、將自己獨生的兒子獻上。\\
論到這兒子曾有話說、『從以撒生的纔要稱為你的後裔.』\\
他以為 神還能叫人從死裡復活.他也彷彿從死中得回他的兒子來。\\
以撒因著信、就指著將來的事、給雅各以掃祝福。\\
雅各因著信、臨死的時候、給約瑟的兩個兒子各自祝福、扶著杖頭敬拜 神。\\
約瑟因著信、臨終的時候、提到以色列族將來要出埃及、並為自己的骸骨留下遺命。\\
摩西生下來、他的父母見他是個俊美的孩子、就因著信把他藏了三個月、並不怕王命。\\
我又何必再說呢.若要一一細說、基甸、巴拉、參孫、耶弗他、大衛、撒母耳、和眾先知的事、時候就不夠了。\\
他們因著信、制伏了敵國、行了公義、得了應許、堵了獅子的口。\\
滅了烈火的猛勢、脫了刀劍的鋒刃、軟弱變為剛強、爭戰顯出勇敢、打退外邦的全軍。\\
有婦人得自己的死人復活、又有人忍受嚴刑、不肯苟且得釋放、〔釋放原文作贖〕為要得著更美的復活.\\
又有人忍受戲弄、鞭打、捆鎖、監禁、各等的磨煉、\\
被石頭打死、被鋸鋸死、受試探、被刀殺.披著綿羊山羊的皮各處奔跑、受窮乏、患難、苦害、\\
在曠野、山嶺、山洞、地穴、飄流無定.本是世界不配有的人。\\
這些人都是因信得了美好的證據、卻仍未得著所應許的.\\
因為 神給我們預備了更美的事、叫他們若不與我們同得、就不能完全。
\end{CJK}
\end{minipage}}
\end{spacing}
\vspace*{\fill}
\end{document}
