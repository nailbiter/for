\documentclass[12pt]{article} % use larger type; default would be 10pt

\usepackage[T1]{fontenc}
\usepackage{graphicx}
\usepackage{float}
\usepackage{CJKutf8}
\usepackage{subfig}
\usepackage{amsmath}
\usepackage{amsfonts}
\usepackage{hyperref}
\usepackage{enumerate}
\usepackage{enumitem}
\usepackage[T1,T2A]{fontenc}
\usepackage[utf8]{inputenc}
\usepackage[english,ukrainian]{babel}

%theorem environments configuration
\newtheorem{problem}{Задача}
\newenvironment{solution}%
{\par\textbf{Розв'язок}\space }%
{\par}

%custom theorems for saving typing
\renewcommand{\P}{\mathbb{P}}
\newcommand{\mymod}{\mathrm{mod}\;}

\title{Завдання на залік з Прикладної Алгебри\\Зимова Сессія 2013}
\author{Студент 4го курсу\\Механіко-математичного факультету КНУ\\Заочної форми\\Олексій Леонтьєв}

\begin{document}
\maketitle
\begin{problem}
	За допомогою китайської теореми про остачі розв’яжіть систему лінійних конгруенцій
	\[x\equiv 1\quad(\mymod 7);\quad x\equiv 2\quad(\mymod 9);\quad x\equiv 3\quad(\mymod 11).\]
\end{problem}
\begin{solution}
	Оскільки 7, 9 та 11 є взаємно простими, теорема застосовується без модифікацій і гарантує існування єдиного за модулем $7\cdot 9\cdot 11=
	693$ розв’язка. Крім того, оскільки ${(9\cdot 11)}^{-1}\equiv 1\;(\mymod 7)$,
	${(11\cdot7)}^{-1}\equiv 2\;(\mymod 9)$ і ${(7\cdot 9)}^{-1}\equiv 7\;(\mymod 11)$, розв’язок (знову ж таки, за китайською
	теоремою про лишки) записується як
	\[1\cdot1\cdot 9\cdot 11+2\cdot 2\cdot 11\cdot 7+3\cdot 7\cdot 7\cdot 9\equiv 344\quad(\mymod 693).\]
	Таким чином, розв’язками є всі $x=344+693k,\;k\in\mathbb{Z}$ і лише вони.
\end{solution}
\begin{problem}
Нехай
\[G =\begin{pmatrix}
1 & 0 & 1 & 1 & 0\\
1 & 1 & 0 & 0 & 0\\
0 & 1 & 1 & 1 & 0\end{pmatrix}
\]
породжуюча матриця деякого лiнiйного бiнарного $(n, k)$-коду $\mathcal{C}$. Чому дорiвнюють $n$
та $k$? Знайдiть перевiрочну матрицю $H$ цього коду. В яке кодове слово буде кодуватися слово $a = (1, 0, 1)$ кодом $\mathcal{C}$?
\end{problem}
\begin{solution}
	Перш за все зазначимо, що оскільки ми працюємо з \textit{бінарним} кодом, ми працюємо в полі $\mathbb{F}_2$. Матриця $H$ в цьому полі має лише два незалежні рядки,
	(попри те, що в означенні сказано, що рядки мають бути базисом і одже бути незалежними). Через це, ми просто будемо ігнорувати третій рядок в подальших розрахунках --
	виключення його зробить рядки матриці базисом (що генерує той же підпростір, що і оригінальні три рядки).

	З означення породжуючої матриці, $n$ і $k$ є кількістю стовпчиків і рядків відповідно, тому $n=5$, $k=2$.

	Оскільки ми працюємо з \textit{бінарним} кодом, ми працюємо в полі $\mathbb{F}_2$ і щоб знайти перевірочну матрицю, за означенням, необхідно просто знайти базис 
	ортогонального доповнення простору, згенерованого рядками матриці $G$. Це просто. За стандартною процедурою, ми спочатку зводимо $G$ до рядкової ступінчастої форми,
	після цього розв'язуємо відповідну систему лінійних рівнянь (3 вільні змінні, 2 залежні) і виражаємо розв'язки як лінійну комбінацію певних векторів, що і дають базис
	для підпростору розв'язків системи. Таким чином, перевірочна матриця
	\[H =\begin{pmatrix}
	0 & 0 & 0 & 0 & 1\\
	1 & 1 & 0 & 1 & 0\\
	1 & 1 & 1 & 0 & 0\\
	\end{pmatrix}
	\]

	Щоб порахувати, у що кодується дане кодове слово, робимо
	\[aG=\begin{pmatrix}1 & 0 & 1\end{pmatrix}\cdot
	\begin{pmatrix}
1 & 0 & 1 & 1 & 0\\
1 & 1 & 0 & 0 & 0\\
	0 & 1 & 1 & 1 & 0\end{pmatrix}=\begin{pmatrix}1 & 1 & 0 & 0& 0\end{pmatrix}\]
\end{solution}
\begin{problem}Розшифруйте повідомлення \rm{\texttt{ETYKL}} , яке зашифроване за допомогою афінного перетворення $X\mapsto 19X + 13$ .
\end{problem}
\begin{solution}
	Розв’язок ми оформимо як таблицю. Зауважимо, що $Y\equiv 19X+13\;(\mymod 26)\Leftrightarrow Y-13\equiv 19X\;(\mymod 26)
	\Leftrightarrow 11Y+13\equiv X\;(\mymod 26)$ (19 та 26 є взаємно простими). Таким чином, зворотня трансформація, яка буде
	використовуватися при розшифровці - $Y\mapsto 11Y+13\;(\mymod 26)$\\
	\begin{tabular}{|c|c|c|c|}
		\hline
		Літера шифру & Алфавітний код & Алфавітний код літери оригіналу & Літера оригіналу\\
		\hline
		E & 4 & 5 & F\\
		\hline
		T & 19 & 14 & O\\
		\hline
		Y & 24 & 17 & R\\
		\hline
		K & 10 & 19 & T\\
		\hline
		L & 11 & 4 & E\\
		\hline
	\end{tabular}\\
	Таким чином, повідомлення розшифровується як \texttt{FORTE}.
\end{solution}
\begin{problem}
	Випишіть всі елементи поля $\mathbb{F}_8 = \mathbb{Z}_2 [x]/(f)$ та обчисліть суму $f_1 + f_2$ і
добуток $f_1 \cdot f_2$ , якщо $f = x^3 + x^2 + 1$, а $f_1 = x^2 + x + 1$, $f_2 = x^2 + 1$. Перевірте, чи є
$f_1$ твірним елементом $\mathbb{F}^*_8 $.
\end{problem}
\begin{solution}
	Всі елементи поля $\mathbb{F}_8 $ матимуть форму $f_i(x)=a_0+a_1x+a_2x,\;a_0,a_1,a_2\in\mathbb{Z}_2$. Їх можна виписати і явно
	\[\mathbb{F}_8=\left\{0,1,x,x+1,x^2,x^2+1,x^2+x,x^2+x+1\right\}\]

	У $\mathbb{Z}_2[x]$ маємо наступне
	\[f_1+ f_2=x\equiv x\quad(\mymod x^3+x^2+1);\]
	\[f_1\cdot f_2=x^4+x^3+x+1\equiv 1\quad(\mymod x^3+x^2+1).\]

	Стосовно того, чи є $f_1$ твірним, варто зазначити, що поле $\mathbb{F}_8$ має в собі 8 елементів, тому $\mathbb{F}^*_8$ є циклічною групою з 7 елементів. Таким чином, в ній \textit{кожний} елемент (окрім, звісно, 1) є
	твірним і тому зокрема $f_1$ є твірним.
\end{solution}
\begin{problem}
 Сформулюйте означення або твердження.
 \begin{enumerate}[label=\arabic*)]
	 \item{Лiнiйний код.}
	 \item{Породжуюча матриця коду.}
	 \item{Характеристика поля.}
	 \item{Твердження про кiлькiсть елементiв скiнченного поля.}
 \end{enumerate}
\end{problem}
\begin{solution}
	\begin{enumerate}[label=\arabic*)]
		\item{\textit{Лінійним $(n,k)$-кодом} називається $k$-вимірний підростір $n$-вимірного простору над скінченним полем.
			
			Ця концепція застосовується в кодах, напрямлених на виправлення помилок, що виникають
			при передачі інформації по каналу з шумом. При передачі по такому каналу деякі частини оригінального повідомлення можуть бути пошкоджені (змінені). В такій ситуації ми згодні додати до повідомлення
			певну надлишкову інформацію, яка дасть можливість відновити його при пошкодженні деякої (невеликої) частини або ж хоча б виявити сам факт пошкодження. Таким чином, ми розглядаємо множину можливих оригінальних
			повідомлень як $k$-вимірний простір над скінченним полем і намагаємось вкласти його в більший простір (таким чином, зробивши \textit{підпростором}).
			Однією з переваг використання лінійних просторів є те, що повідомлення натурально записується як послідовність символів -- вектор. До
			того ж, трансформації можна представити як лінійні функції, що звісно ж прискорює розрахунки.}
		\item{\textit{Породжуючою матрицею} лінійного $(n,k)$-коду називається матриця розмірності $n\times k$, рядки якої є базисом для даного лінійного коду (нагадаємо, що він є підпростором). 
			
			Таким чином, ця матриця, розглянута як лінійна функція, здійснює вкладення множини повідомлень в множину повідомлень з надлишковою інформацією.}
		\item{\textit{Характеристикою} поля $F$ (позначається як $\text{char}(F)$) називається таке мінімальне число $k$, що $\underbrace{1_F+1_F+\ldots+1_F}_{n \text{ разів}}=0_F$, де $1_F$ і $0_F$ позначають одиницю та нуль поля $F$ відповідно
			. Якщо ж такого $k$ не існує (тобто,
			додаючи $1_F$ до себе ми отримуємо все нові і нові елементи $F$) означують $\text{char}(F):=0$.

			Легко показати, що ненульова характеристика може бути лише простим числом.
			}
		\item{\textit{Твердження про кількість елементів скінченного поля формулюється наступним чином.} Нехай $F$ -- скінченне поле. Тоді його характеристика
			є простим позитивним числом $p$ і більше того, $|F|=p^n$, тобто кількість елементів є степінь простого число, степінь характеристики. }
	\end{enumerate}
\end{solution}
\end{document}
