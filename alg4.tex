\documentclass[8pt]{article} % use larger type; default would be 10pt

%\usepackage[utf8]{inputenc} % set input encoding (not needed with XeLaTeX)
%\usepackage{CJK}
\usepackage[margin=1in]{geometry}
\usepackage{graphicx}
\usepackage{float}
\usepackage{subfig}
\usepackage{amsmath}
\usepackage{amsfonts}
\usepackage{hyperref}
\usepackage{enumerate}
\usepackage{enumitem}
\usepackage{harpoon}

\usepackage{mystyle}

\title{Homework 4, Math 5111}
\author{Alex Leontiev, 1155040702, CUHK}
\begin{document}
\maketitle
\begin{enumerate}[label=\bfseries Problem \arabic*.]
	\newcommand{\Frac}{\mbox{Frac}}
	\item First of all, let us show that $\Frac S^{-1}R$ can be considered as a subfield of $\Frac R$ by exhibiting field homomorphism
		(there's no need to prove injectivity as any field homomorphism is so).
		Let us denote the set of equivalence classes that form $\Frac S^{-1}R$ as $[x,y]'$ for $x,y\in S^{-1}R$ (where $[x,y]'=[x',y']'$ if
		$xy'=x'y$) and the equivalence classes that form $S^{-1}R$ as $[r,s]$ for $r\in R,s\in S$ (where $[r,s]=[r',s']$ if $s^*(rs'-sr')=0$
		for $s^*\in S$). By the way, note that as $R$ is an integral domain by hypothesis, $s^*(rs'-sr')=0$ for $s^*\in S$ means that
		$rs-sr'=0$, as $0\notin S$ (by definition of multiplicative subset). Now, let us map $[[r,s],[r',s']]'\in\Frac S^{-1}R$ to $
		F([[r,s],[r',s']]):=[
		rs',r's]''$ (we denote the equivalence classes that form $\Frac R$ by $[r,t]''$ for $r,t\in R$, where $[r,t]''=[r',t']''$ whether
		$rt'=tr'$). The mapping we have exhibited is well defined, as if $[[t,u],[t',u']]'=[[r,s],[r',s']]'$, then it is
		mapped to $[tu',t'u]$ and
		\[[t',u']\cdot[r,s]=[t,u]\cdot[r',s']\implies [t'r,u's]=[tr',us']\implies t'rus'=u'str'\], which in turn implies that $[tu',t'u]=
		[rs',r's]$, which means that both $[[t,u],[t',u']]'$ and $[[r,s],[r',s']]'$ are mapped to the same element of $\Frac R$, hence
		the mapping is well-defined. Furthermore, it is well-behaved with respect to addition, as \[F([[r,s],[r',s']]'+[[t,u],[t',u']]')=
		F([[r,s]\cdot[t',u']+[r',s']\cdot[t,u],[r',s']\cdot[t',u']]')=\]
		\[=F([[rt',su']+[r't,s'u],[r't',s'u']]')=F([[rt's'u+su'r't,su's'u],[r't',s'u']]')=\]\[=[(rt's'u+su'r't)s'u',su's'ur't']''=
		[rt's'u+su'r't',sur't']''\]
		while \[F([[r,s],[r',s']]')+F([[t,u],[t',u']]')=[rs',sr']''+[tu',ut']''=[rs'ut'+sr'tu',sr'ut']''\]
		which is the same thing. Similarly, $F$ is well-behaved under multiplication, as
		\[F([[r,s],[r',s']]')\cdot F([[t,u],[t',u']]')=[rs',sr']''\cdot[tu',ut']''=\]
		\[=[rs'tu',sr'ut']''=F([[rt,su],[r't',s'u']]')=F([[r,s],[r',s']]'\cdot[[t,u],[t',u']]')\]
		and besides, identity is mapped to identity, as
		\[F([[1,1],[1,1]]')=[1\cdot1,1\cdot1]''=1\in\Frac R\]
		having this, we will treat $\Frac S^{-1}R$ as a subfield of $\Frac R$ from now on.

		Now, suppose $x=\frac{rs'}{r's}\in\Frac S^{-1}R$ and is algebraic over $S^{-1}R$ (following justified above convention,
		we treat elements of $\Frac S^{-1}R$ as elements of $\Frac R$), so 
		\[\left(\frac{rs'}{r's}\right)^n+\frac{r_{n-1}}{s_{n-1}}\left(\frac{rs'}{r's}\right)^{n-1}+\hdots+
		\frac{r_1}{s_1}\left(\frac{rs'}{r's}\right)+\frac{r_0}{s_0}=0\]
		Let us multiply both sides by $T:=(s_{n-1}\cdot\hdots\cdot s_1s_0)^n$. This shall give us
		\[\left(\frac{Trs'}{r's}\right)^n+R_{n-1}\left(\frac{Trs'}{r's}\right)^{n-1}+\hdots+
		R_1\left(\frac{Trs'}{r's}\right)+R_0=0\]
		for $R_i\in R$. As $R$ is integrally closed, \[\frac{Trs'}{r's}\in R\implies x=\frac{rs'}{r's}\in S^{-1}R\]
	\item \begin{enumerate}[label=(\arabic*).]
			\item Let us prove the required properties one-by-one
				\begin{description}
					\item[$\mathbf{T^2=T}$] Let us show first that $\mbox{Im}(T)\subset R^G$. Indeed, given $f\in R$ and
						$g'\in G$ arbitrary, we have
						\[g'\cdot T(f)=g'\cdot\frac{1}{\myabs{G}}\sum_{g\in G}g\cdot f=\frac{1}{\myabs{G}}\sum_{g\in G}
						g'g\cdot f=\frac{1}{\myabs{G}}\sum_{g'g\in G} g'g\cdot f=T(f)\]
						(in other words, the mapping $G\ni g\mapsto g'g\in G$ is a permutation of $G$, and as sum is 
						finite, it's invariant under the permutation of addends). Therefore, as $g'\in G$ was arbitrary,
						$T(f)\in R^G$ whenever $f\in R$. Next, let us show that for $f\in R^G$ we have $T(f)=f$. Indeed,
						\[T(f)=\frac{1}{\myabs{G}}\sum_{g\in G} g\cdot f=\frac{1}{\myabs{G}}\sum_{g\in G}f=\frac{\myabs{G}
						\cdot f}{\myabs{G}}=f\]
						altogether, these two claims give the desired $T^2=T$.
					\item[$\mathbf{\mbox{Im}(T)=R^G}$] As we have shown above that $\mbox{Im}(T)\subset R^G$, it just
						remains to show the inverse inclusion or in other words, surjectivity. But for $f\in R^G\subset R$
						we have shown above that $T(f)=f$, hence every $f\in R^G$ is an image of some element in $R$, hence
						we are done.
					\item[$\mathbf{T}$ is $\mathbf{R^G}$-linear] This follows from the fact that every $g$ is algebra 
						homomorphism (as linear combination of objects) and $T$ fixes $R^G$. Indeed, for $f_i\in R^G$ and
						$g_i\in R$ we have
					\[\begin{array}{rr}
					T(f_1\cdot g_1+f_1\cdot g_1+\hdots+f_n\cdot g_n)= &\mbox{ (\textit{applying definition}) }\\\\
				\frac{1}{\myabs{G}}\sum_{g\in G}g\cdot(f_1\cdot
				g_1+f_2\cdot g_2 \hdots+f_n\cdot g_n)= &\mbox{ (\textit{$g$ is algebra homomorphism}) }\\\\
				\frac{1}{\myabs{G}}\sum_{g\in G}g(f_1)\cdot g(g_1)
				+g(f_2)\cdot g(g_2)+\hdots+g(f_n)\cdot g(g_n)= &\mbox{ (\textit{as $f_i\in R^G$}) }\\\\
				f_1\cdot\frac{1}{\myabs{G}}\sum_{g\in G}g\cdot g_1+
				f_2\cdot\frac{1}{\myabs{G}}\sum_{g\in G}g\cdot g_2+\hdots+
				f_n\cdot\frac{1}{\myabs{G}}\sum_{g\in G}g\cdot g_n= &f_1T(g_1)+f_2T(g_2)+\hdots+f_nT(g_n)
					\end{array}\]
				\end{description}
			\item As $F$ is a field, it is Noetherian ring in particular. Therefore, by Hilbert's basis theorem, $R=F[x_1,\hdots,x_n]$
				is also Noetherian, hence every ideal in $R$, in particular $I$, is finitely generated. Therefore, for some
				$F_1,F_2,\hdots,F_m\in I$ we have that \[I=RF_1+\hdots+RF_m\] Now, given any $f\in R^G\subset I$ we have
				\[f=T(f)=T(r_1F_1+\hdots+r_mF_m)=T(r_1)T(F_1)+\dots+T(r_m)T(F_m)\]
				thus every element of $R^G$ can be written as linear combinations of $f_i:=T(F_i)\in R^G$ over $R$ and as
				in turn $I$ consists of linear combinations of elements of $R^G$ over $R$, we see that the whole $I$ is generated
				by $f_i\in R^G$ in fact. It should be noted, that even if $I$ would be defined as an ideal generated by elements
				of $R^G$ of \textit{positive} degree, the above proof would still work literally
				, except that now we would have that $\forall 1\leq i\leq m,\;\deg(f_i)>0$
			\item
		\end{enumerate}
	\item \begin{enumerate}[label=(\arabic*).]
			\newcommand{\Span}{\mbox{Span}}
			\newcommand{\F}{\mathbb{C}(z)}
			\item Before we start anything, let us not that as $\mathbb{C}(z)$ is in fact a field, all the modules mentioned in a
				problem are in fact vector spaces, which shall be used. Besides, we will introduce the (perhaps, 
				infinitely-dimensional) vector space $V$ of all meromorphic functions over the field $\mathbb{C}(z)$.\\
				$(\Rightarrow)$. If $f(z)$ is algebraic, then for some $n\geq 0$, $r_i(z)\in\mathbb{C}(z)$ we have
				\[f^{(n)}(z)=-r_{n-1}(z)f^{(n-1)}(z)-\hdots-r_1(z)f'(z)-r_0(z)f(z)\]
				Taking derivatives successive derivatives of this expression, we can generate all of the elements in
				$\mbox{Span}_{\F}(f(z),f'(z),\dots)$ as linear combinations of first $n-1$ derivatives of $f$ with coefficients
				in $\F$. Thus, $\mbox{Span}_{\F}(f(z),f'(z),\dots)$ is finitely dimensional (i.e. finitely generated)
				$(\Leftarrow)$. Suppose $\mbox{Span}_{\F}(f(z),f'(z),\dots)$ is finitely generated, hence finitely dimensional
				of dimension, say $n$. Then, the $n+1$ elements $f(z),f'(z),\hdots,f^{(n+1)}(z)$ should be linearly dependent
				over $\F$ which exactly lead to the conclusion that $f$ is algebraic.
			\item Given, $f$ and $g$ algebraic, let us denote by $V_f:=\Span{\F}(f,f',\hdots)\subset V$ and 
				$V_g:=\Span(g,g',\hdots)\subset V$. As $f$ and $g$ are both algebraic, both spaces are finitely dimensional,
				hence so is their sum $V_f+V_g:=\mysetn{v+u}{v\in V_f,\;u\in V_g}$ and hence so is
				\[V_{f+g}:=\Span_{\F}(f+g,f'+g',\dots)\subset V_f+V_g\]
				hence, by previous subproblem, $f+g$ is algebraic.

				Next, let us consider the subspace $V_f\cdot V_g$ defined to be a subspace, generated by all the products
				$\mysetn{vu}{v\in V_f,\;u\in V_g}$. It is finitely dimensional, as if $\mycbra{f_i}_{i=1}^m$ and
				$\mycbra{g_j}_{j=1}^n$ generate $V_f$ and $V_g$ respectively, then $V_f\cdot V_g$ is generated by
				$\mycbra{f_i\cdot g_j}_{i,j=1}^{m,n}$. Then,
				\[V_{fg}=\Span_{\F}\left(fg,(fg)'=f'g+gf',(fg)''=f''g+2f'g'+fg'',\hdots\right)\subset V_f\cdot V_g\]
				is also finitely dimensional, as a subspace of a finitely dimensional space.

				Unfortunately, however, algebraic meromoprhic functions are \textbf{not} closed under division. That is, if
				$f$ and $g$ are algebraic meromorphic, and $g$ is not constantly zero, their quotient still as well might be
				non algebraic. For example, consider $f(z)\equiv 1$ and $g(z)=\sin(z)$ both algebraic meromorphic. However,
				their quotient $h(z)=1/\sin(z)$ is not, as every new derivative of it has form (can be shown by induction)
				\[h^{(2n)}(z)=\cos(z)P_{2n}\left(\frac{1}{\sin(z)}\right)\]
				\[h^{(2n-1)}(z)=P_{2n-1}\left(\frac{1}{\sin(z)}\right)\]
				where $P_i$ are polynomials and $\deg(P_i)=i$.
				Hence, nontrivial linear dependence over $\F$ between them would lead to nontrivial
				linear dependence over $\mathbb{C}[z]$ of the degrees of $\sin(z)$ 
				and hence to nontrivial linear dependence over $\mathbb{C}[z]$ between $e^{niz}$ for $n\in\mathbb{Z}$. But the
				latter cannot be true.
		\end{enumerate}
\end{enumerate}
\end{document}
