\documentclass[8pt]{article} % use larger type; default would be 10pt

\usepackage[margin=1in]{geometry}
\usepackage{graphicx}
\usepackage{float}
\usepackage{subfig}
\usepackage{amsmath}
\usepackage{amsfonts}
\usepackage{hyperref}
\usepackage{enumerate}
\usepackage{enumitem}
\usepackage{harpoon}
\usepackage{tikz}

\usepackage{mystyle}

\title{Final Exam, MATH 5070}
\author{Alex Leontiev, 1155040702, CUHK}
\begin{document}
\maketitle
\begin{enumerate}[label=\bfseries \arabic*.]
	\item Let us start with homology and cohomology, as it is simpler. As computations in \cite[page 107]{gh} show, the homology groups are
		\[H_q(S^3-K);\mathbb{Z}=\left\{\begin{array}{ll}\mathbb{Z},&\mbox{$q$=1 or 0}\\0,&\mbox{otherwise}\end{array}\right.\]
		and cohomology are correspondingly just reversed.

		Let us proceed to homotopy computations. First, of all, note that treefoil can be realized as an image of a map
		$S^1\ni e^{i\theta}\mapsto(e^{2\cdot2\pi i\theta},e^{2\cdot3\pi i\theta})\in S^1\times S^1\subset\mathbb{S}^3$. Then, as
		show in \cite[Theorem 2.3.3]{martin} (whose proof is just an application of van Kampen theorem from \cite{gh} and won't be repeated
		to save typing), we have
		\[\pi_1(S^3-K)=\mysetn{x,y}{x^2=y^3}\]
		(where r.h.s. denotes group generated by $x$ and $y$, subject to $x^2=y^3$).
	\item The identification is essentially identity, as both unit quaternions and $S^3$ are the subsets of $\mathbb{R}^4$ (at least,
		we understand the quaternions as subset of $\mathbb{R}^4$, in the lack of more careful explanation in the statement).

		The computation of Lefschetz number goes as follows. We claim that it is equal to 1. First, let us show that the {\it only}
		identity quaternion, such that $z^2=z$ is $z=1$. Indeed, viewing the quaternion $z$ as $z=(a,b,c,d)$ we have that $z=z^2$ is written
		as
		\[(a,b,c,d)=z=z^2=(a^2-b^2-c^2-d^2,2ab,2ac,2ad)\]
		From the equality of second to fourth coordinates we see that {\it either} $b=c=d=0$, or $a=1/2$. However, the latter
		cannot happen, as then $a>a^2\geq a^2-b^2-c^2-d^2$, so we cannot have an equality in the first coordinate. Hence, $b=c=d=0$
		and thus $a=\pm 1$. However, as $a=a^2\geq0$, we are led to the conclusion $a=1$ and $b=c=d=0$.

		Now, let us consider the map $F:\mathbb{R}^4\ni(a,b,c,d)\mapsto(a^2-b^2-c^2-d^2,2ab,2ac,2ad)\in\mathbb{R}^4$ near the point
		$(1,0,0,0)$. It's differential is
		\[\begin{bmatrix}2a&2b&2c&2d\\-2b&2a&0&0\\-2c&0&2a&0\\-2d&0&0&2a\end{bmatrix}\]
		which is proportional to identity at $(a,b,c,d)=(1,0,0,0)$, hence, we see that the tangent space of $\mys{3}$ at
		$(1,0,0,0)$ is mapped isomorphically onto itself with orientation preserving. Thus, $1$ has index 1 and Lefschetz number is 1.
	\item To begin with, we will show the special case, namely that the map $f:X\to S^n$ of degree 0, where
		$X$ is non-orientable, is homotopic to a constant. Indeed, let's take a point $y\in S^n$ which is a regular image of $f$. Then,
		$f^{-1}(y)$ has even number of points, say $x_1,x_2,\hdots,x_{2k}$. Now, taking $x_1$, as $X$ is non-orientable, there exists
		a $\gamma$ loop from $x_1$ to itself, whose tubular neighborhood $T$ is diffeomorphic to cylinder $S^{n-1}\times[0,1]$ with top
		and bottom identified through the orientation-changing (i.e. $(x,0)\sim(y,1)\iff x=\pi(y)$, where $\pi:S^{n-1}\to S^{n-1}$ is
		orientation-reversing). Now, by taking any point of $\gamma$ in-between its endpoints and connecting it with $x_2$,
		then expanding a bit we can ensure that $\gamma$ passes through $x_2$.\\
		\mypic{0.6}{tour}

		Now, having $x_0,x_1\in T$, which is diffeomorphic to cylinder modulo $\pi$, we may induce arbitrary orientation on $x_0$ and
		notice the following: for every $\epsilon>0$ both $T\supset S^{n-1}\times [0,1-\epsilon]/\sim$ and $T\supset S^{n-1}\times
		[\epsilon,1]/\sim$ are orientable (as they are just open cylinders) and hence both induce orientation on $x_1$. Now, these
		two orientations induced should be different (otherwise, we can orient the whole $T$, which is impossible by construction). Without
		loss of generality, assume that the map $f$ restricted to $S^{n-1}\times[0,1-\epsilon]/\sim$ has oriented intersection number $0$
		(otherwise, the choice $[\epsilon,1]$ would give the desired result). By Hopf degree theorem, we can modify $f$ inside $T$ (without
		affecting values outside), such that $f$ does not hit $y$ on $T$. Repeating this procedure $k$ times, we get that 
		$f$ does not hit $y$ at all, hence $f:X\to S^n$ is not onto, and can be homotyped to constant, as was required.

		Now, if degree of $f$ and $g$ are 1, consider the manifold $X\times[0,1]$ with map on $\partial(X\times[0,1])=X\sqcup X$
		defined to be $f$ on one copy of $X$ and $g$ on another one. As in the proof of Hopf degree theorem in \cite{gp}, we may extend
		this to map $F:X\times[0,1]\to\mathbb{R}^{n+1}$, which is of un-oriented degree $0$ and hence (as $X\times[0,1]$ is compact and
		also non-orientable), as in the proof of previous result, and step 8 in proof of Hopf degree theorem in \cite{gp}, we can
		repair $F$, without altering it on boundary, such that it does not hit some point near $0\in\mathbb{R}^{n+1}$, and
		thus $F$ can be taken to be $X\times[0,1]\to S^n$, thus giving homotopy required.
	\item To begin with, as the problem does not clearly says what topology should we impose on $C^{\infty}(M,N)$ for "open and dense" to make
		sense, we assume that it's the topology induced by the metric (we'll call it $C^1$ metric in subsequent)
		$d(f,g):=\sup_{x\in M}\mycbra{d(f(x)-g(x))_N,\mynorm{df(x)
		-dg(x)}}$, where $d(\cdot,\cdot)_N$ is some metric on $N$ and $\nabla f$ is taken to be a matrix (resulted after chart
		mapping by same fixed atlas) and $\mynorm{\cdot,\cdot}$ is any matrix norm.

		The scheme is as follows.
		\begin{enumerate}[label=\arabic*.]
			\item We will show that injective mappings are open dense in $C^{\infty}(M,N)$.
			\item We will define the set of maps that the problem asks about.
			\item We will show that this set has required properties, i.e. no $(k+1)$-fold points and $k$-fold form submanifold.
			\item We will show that this set is open dense in the topological space of injective maps (endowed
				with induced topology as subset of $C^{\infty}(M,N)$). This will end the proof,
				as being open dense is transitive relation.
		\end{enumerate}
		Let us proceed directly to the first item. The fact that open mappings form open subset of $C^{\infty}(M,N)$ is evident,
		as if $f_0$ is an immersion, then $M\ni x\mapsto\inf_{\mynorm{v}=1}\mynorm{df(x)(v)}\in\mathbb{R}_{>0}$ is continuous, hence
		has positive minimum on $M$ (compactness of $M$ is used). Then, as $\inf_{\mynorm{v}}{Av}$ continuous depends on $A$, small (in
		$C^1$ metric) perturbations of immersion are immersions.

		In subsequent, we'll use the notation at \cite[p. 209]{uchida} and all the Theorems will be referred from there. First, we
		assume Lemma 2.5, which is just linear algebra. Now, we define the set of maps the problem asks as {\bf immersions $f$, such that
		for all $k+1\geq
		p\geq 2$, $f^{(p)}:M^{(p)}-\Delta_{(p)}M\to N^{(p)}$ is transversal regular to $\Delta^p N\subset N$} (we take property in
		Lemma 3.1 as definition of transverse regular). Note, that $M^{(p)}-\Delta_{(p)}M\subset M^{(p)}$ is open (as $\Delta_{(p)}$ is 
		closed), hence is a submanifold.
		Now, we'll show that the set of such $f$ has required properties. First,
		by the statement of problem, $\dim M\cdot(k+1)-k\dim N<0\implies (k+1)N>(k+1)\dim M+N$, hence if 
		$f^{(k+1)}:M^{(k+1)}-\Delta_{(k+1)}M\to N^{(k+1)}$ intersects $\Delta^{k+1} N$, intersection {\it cannot} be transversal, hence
		maps $f$, such that $f^{(k+1)}:M^{(k+1)}-\Delta_{(k+1)}M\to N^{(k+1)}$ is transversal to $\Delta^{k+1} N$, in fact {\it does not}
		intersect $\Delta^{k+1} N$, hence no $k+1$-fold points. 
		Second, transversality for $1\leq p\leq k$, by Lemma 3.1 (which we will prove below), implies that
		for every $(x_1,x_2,\hdots,x_k)\in M^k$ all distinct and with equal value $f$, subspaces $df_{x_i}T_{x_i}M\subset T_{f(x_i)}N$
		are in general position (as $(b)\implies(a)$ in Lemma 2.5)
		, hence as $(a)\implies(c)$ in Lemma 2.5, orthogonal complements of $df_{x_i}T_{x_i}M$ are all linearly
		independent, hence set of $k$-fold points near $x_1$ is cut by $k(n-m)$ independent conditions, hence image of $k$-fold points
		has codimension $k(n-m)$ (hence, dimension $n-k(n-m)=km-(k-1)n$)
		in $N$, and as near $x_1$ $f$ is diffeomorphism onto its image (as $f$ is immersions), $k$-fold points in $M$ are
		a manifold of dimension $km-(k-1)n$ in $M$, as required.

		Finally, the fact that the family of maps defined in previous paragraph is open and dense can be shown for each $k$ separately
		(as finite intersection of open dense sets is open dense in any metric space). Openness follows from the fact that transversal
		(to given submanifold) maps are open set in $C^1$ metric (topology). Denseness follows from transversality theorem in \cite{gp}.

		The proof of Lemma 3.1 (at least, the "only if" part) is easy, as if $f^{(p)}$ is transversal, we have that
		\[\bigoplus_{i=1}^k df_{x_i}T_{x_i}M+d_yN=\bigoplus_{i=1}^k d_yN\]
		projecting on $i=1$, and using the fact $df_{x_i}T_{x_i}M\subset d_yN$, we get
\end{enumerate}
\begin{thebibliography}{9}
	\bibitem{gh} {\em Algebraic Topology, a first course}, Greenberg and Harper
	\bibitem{gp} {\em Differential Topology}, Victor Guillemin , Alan Pollack
	\bibitem{martin} {\em Introductory Knot Theory}, \url{http://goo.gl/S0dJ6P}
	\bibitem{uchida} {\em Cobordism groups of immersions}, Fuichi Uchida, \url{http://goo.gl/ERgvJm}
\end{thebibliography}
\end{document}
%4-->3-->1-->2
