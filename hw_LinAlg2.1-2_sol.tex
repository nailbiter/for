\documentclass[8pt]{article} % use larger type; default would be 10pt

%\usepackage[utf8]{inputenc} % set input encoding (not needed with XeLaTeX)
\usepackage{graphicx}
\usepackage{float}
\usepackage{subfig}
\usepackage{amsmath}
\usepackage{amsfonts}
\usepackage{hyperref}
\usepackage{harpoon}
\usepackage{enumitem}
\usepackage{multicol}
\usepackage[neverdecrease]{paralist}
\usepackage{enumerate}
\usepackage{cancel}
\usepackage{ulem}

\usepackage{mystyle}

\newcommand{\myexplain}[3]{#1\xrightarrow{\text{#2}}#3}
\newcommand{\myexplainf}[4]{#1\xrightarrow{\begin{subarray}{c}\text{#2}\\\text{#3}\end{subarray}}#4}
\newcommand{\myexplainfi}[5]{#1\xrightarrow{\begin{subarray}{c}\text{#2}\\\text{#3}\\\text{#4}\end{subarray}}#5}
\newcommand{\myfrac}[2]{^#1/_#2}

\title{Math 1540\\University Mathematics for Financial Studies\\2013-14 Term 1\\Suggested solutions for\\HW problems Sec. 2.1-2.2 (Linear Algebra)}
\begin{document}
\maketitle
\section{Section 2.1}
\begin{description}
	\newcommand{\mydet}[2]{\left|\begin{array}{#1}#2\end{array}\right|}
	\item[\# 3.]{{\it Evaluate the following determinants.\\\\}
	\begin{inparaenum}[(a)]
		\setcounter{enumi}{4}
		\item $\left|\begin{array}{rrr}1&3&2\\4&1&-2\\2&1&3\end{array}\right|$\qquad
		\setcounter{enumi}{6}
		\item $\left|\begin{array}{rrrr}2&0&0&1\\0&1&0&0\\1&6&2&0\\1&1&-2&3\end{array}\right|$
	  \end{inparaenum}\\\\
	We simply apply the recurrent formula for the determinant\\
	\begin{enumerate}[(a)]
		\setcounter{enumi}{4}
		\item\[\left|\begin{array}{rrr}1&3&2\\4&1&-2\\2&1&3\end{array}\right|=1\mydet{rr}{1&-2\\1&3}-3\mydet{rr}{4&-2\\2&3}
			+2\mydet{rr}{4&1\\2&1}=\]\[=1\cdot(1\cdot 3-1\cdot(-2))-3\cdot(4\cdot 3-2\cdot (-2))+2\cdot(4\cdot 1-2\cdot 1)=-39\]
		\setcounter{enumi}{6}
	\item Here we shall use expansion along the second row first, as it contains the biggest number of zeros
		\[\mydet{rrrr}{2&0&0&1\\0&1&0&0\\1&6&2&0\\1&1&-2&3}=1\cdot\mydet{rrr}{2&0&1\\1&2&0\\1&-2&3}=2\mydet{rr}{2&0\\-2&3}+1\cdot\mydet{rr}
		{1&2\\1&-2}=\]
		\[=2\cdot(2\cdot3-(-2)\cdot0)+1\cdot(1\cdot(-2)-1\cdot2)=8\]
	\end{enumerate}
	}
\item[\# 4.]{{\it Evaluate the following determinants by inspection.\\\\}
	\begin{inparaenum}[(a)]
		\setcounter{enumi}{1}
		\item $\mydet{rrr}{2&0&0\\4&1&0\\7&3&-2}$\qquad
		\item $\mydet{rrr}{3&0&0\\2&1&1\\1&2&2}$\qquad
		\item $\mydet{rrrr}{4&0&2&1\\5&0&4&2\\2&0&3&4\\1&0&2&3}$
	  \end{inparaenum}\\\\
	  \begin{enumerate}[(a)]
		  \item As the matrix is lower triangular, determinant is equal to the product of the elements on a diagonal
			  \[\mydet{rrr}{2&0&0\\4&1&0\\7&3&-2}=2\cdot1\cdot(-2)=-4\]
			  \item First we may apply the recurrent formula along the first row. This can be done by inspection, as first row has
				  only one nonzero element
				  \[\mydet{rrr}{3&0&0\\2&1&1\\1&2&2}=3\cdot\mydet{rr}{1&1\\2&2}\]
				  Now the $2\times2$ determinant is zero, as two rows of the matrix are proportional, hence it is singular. 
				  Thus
				  \[\mydet{rrr}{3&0&0\\2&1&1\\1&2&2}=3\cdot\mydet{rr}{1&1\\2&2}=0\]
			  \item The determinant is zero, as matrix contains zero column, hence is singular
				  \[\mydet{rrrr}{4&0&2&1\\5&0&4&2\\2&0&3&4\\1&0&2&3}=0\]
	  \end{enumerate}
	}
\item[\# 6.]{{\it Find all the values of $\lambda$ for which the following determinant will equal 0.}
	\[\mydet{cc}{2-\lambda&4\\3&3-\lambda}\]
	First we write the determinant of above $2\times2$ matrix
	\[\mydet{cc}{2-\lambda&4\\3&3-\lambda}=(2-\lambda)\cdot(3-\lambda)-3\cdot4=\lambda^2-5\lambda-6\]
	As we want this quantity to be zero, we end up with solving quadratic equation. The solutions are $\lambda=-1$ and $\lambda=6$.
	}
\item[\# 10.]{{\it Use mathematical induction to prove that if $A$ is an $(n+1)\times(n+1)$ matrix with two identical rows, then $\det(A)=0$.}
	We shall use the induction on the $n$ in the statement. Base case is $n=1$. In this case the statement can be verified directly
	\[\mydet{cc}{a&b\\a&b}=ab-ab=0\]
	Now, let us assume that it is correct for some $n\geq 1$ and try to show that it will also hold for $n+1$ then. For this purpose, let
	$A$ be $(n+2)\times(n+2)$ matrix with rows, say $\mathbf{v_1},\mathbf{v_2},\dots,\mathbf{v_{n+2}}$, of which two, say $\mathbf{v_i}$
	and $\mathbf{v_j}$ are equal. Now, applying recurrent formula along any row, $\mathbf{v_k}=(v_k^1,v_k^2,\dots,v_k^{n+2})$,
	such that $k\neq i$ and $k\neq j$ (which is possible as $n+2\geq 3$, we get
	\[\myabs{A}=v_k^1\cdot\myabs{A_1}-v_k^2\myabs{A_2}\pm\dots+(-1)^{n+1}\cdot v_k^{n+2}\myabs{A_{n+2}}\]
	where $A_l$ is matrix $A$ with $k$-th row and $l$-th column removed for $1\leq l\leq n+2$. Now, for any $A_l$ it will still contain two 
	identical rows, as each of $\mathbf{v_i}$ and $\mathbf{v_j}$ will still be present, although cut by one element (corresponding to $l$-th
	column removed). Hence, by inductive assumption, $\myabs{A_1}=\myabs{A_2}=\dots=\myabs{A_{n+2}}=0$ and therefore
	\[\myabs{A}=v_k^1\cdot\myabs{A_1}-v_k^2\myabs{A_2}\pm\dots+(-1)^{n+1}\cdot v_k^{n+2}\myabs{A_{n+2}}=0\]
	which justifies inductive step and finishes the proof.
	}
\item[\# 11.]{{\it Let $A$ and $B$ be $2\times2$ matrices.}
	\begin{enumerate}[(a)]
		\item{\it Does $\det(A+B)=\det(A)+\det(B)$?}
		\item{\it Does $\det(AB)=\det(A)\det(B)$?}
		\item{\it Does $\det(AB)=\det(BA)$?}
	\end{enumerate}
	\begin{enumerate}[(a)]
		\item Sometimes, yes, e.g.
			\[\mydet{rr}{0&0\\0&2}=\mydet{rr}{1&0\\0&1}+\mydet{rr}{-1&0\\0&1}\]
			and sometimes, no
			\[4=\mydet{rr}{2&0\\0&2}\neq\mydet{rr}{1&0\\0&1}+\mydet{rr}{1&0\\0&1}=2\]
			Some less trivial examples of when $\det(A+B)=\det(A)+\det(B)$ may hold, see the solution of Problem 12 below.
		\item{
			\newcommand{\uuuline}[1]{\dashuline{#1}}
			\newcommand{\uuuuline}[1]{\dotuline{#1}}
			Yes. This can be verified easily just by writing down the explicit formulae
			\[\mydet{rr}{a&b\\c&d}\cdot\mydet{rr}{a'&b'\\c'&d'}=(ad-bc)(a'd'-b'c')=\underline{ada'd'}
			-\underline{\underline{adb'c'}}-\uuuline{bca'd'}+\uuuuline{bcb'c'}=\]
			\[=\cancel{aa'cb'}+\underline{aa'dd'}
			+\uuuuline{bc'cb'}+\xcancel{bc'dd'}-\cancel{ca'ab'}-\uuuline{ca'bd'}-\underline{\underline{dc'ab'}}-\xcancel{dc'bd'}=\]
			\[=\mydet{cc}{aa'+bc'&ab'+bd'\\ca'+dc'&cb'+dd'}
			=\det\mybra{\begin{bmatrix}a&b\\c&d\end{bmatrix}\cdot\begin{bmatrix}a'&b'\\c'&d'\end{bmatrix}}\]
				}
		\item Yes and this is an implication of what we have proven in a previous item, as
			\[\det(AB)=\det(A)\det(B)=\det(B)\det(A)=\det(BA)\]
	\end{enumerate}
	}
\item[\# 12.]{{\it Let $A$ and $B$ be $2\times2$ matrices and let}
	\[C=\begin{pmatrix}a_{11}&a_{12}\\b_{21}&b_{22}
	\end{pmatrix},\quad D=\begin{pmatrix}b_{11}&b_{12}\\a_{21}&a_{22}\end{pmatrix},\quad E=\begin{pmatrix}0&\alpha\\\beta&0\end{pmatrix}\]
	\begin{enumerate}[(a)]
		\item{\it Show that $\det(A+B)=\det(A)+\det(B)+\det(C)+\det(D)$.}
		\item{\it Show that if $B=EA$ then $\det(A+B)=\det(A)+\det(B)$.}
	\end{enumerate}
	\begin{enumerate}[(a)]
		\item{
			This is easy, as
			\[\det(A+B)=\begin{vmatrix}a_{11}+b_{11}&a_{12}+b_{12}\\a_{21}+b_{21}&a_{22}+b_{22}\end{vmatrix}=\]
				\[=\underline{a_{11}a_{22}}+\underline{\underline{a_{11}b_{22}}}+\dashuline{b_{11}a_{22}}+
				\dashuline{\dashuline{b_{11}b_{22}}}-\dotuline{a_{21}a_{12}}-\dotuline{\dotuline{a_{21}b_{12}}}
				-\uwave{b_{21}a_{12}}-\uwave{\uwave{b_{21}b_{12}}}=\]
				\[=(\uline{a_{11}a_{22}}-\dotuline{a_{12}a_{21}})+(\dashuline{\dashuline{b_{11}b_{22}}}-
				\uwave{\uwave{b_{12}b_{21}}})+(\uuline{a_{11}b_{22}}-\dotuline{a_{12}b_{21}})+(\dashuline{
				\dashuline{b_{11}a_{22}}}-\dotuline{\dotuline{a_{21}b_{12}}})=\]
				\[=\begin{vmatrix}a_{11}&a_{12}\\a_{21}&a_{22}\end{vmatrix}+\begin{vmatrix}b_{11}&b_{12}\\b_{21}&b_{22}
				\end{vmatrix}+\begin{vmatrix}a_{11}&a_{12}\\b_{21}&b_{22}
				\end{vmatrix}+\begin{vmatrix}b_{11}&b_{12}\\a_{21}&a_{22}\end{vmatrix}=\det(A)+\det(B)+\det(C)+\det(D)\]
				}
		\item{To see that $\det(A+B)=\det(A)+\det(B)$ when $B=EA$, not that in this case 
			Hence
			\[C=\begin{vmatrix}a_{11}&a_{12}\\b_{21}&b_{22}\end{vmatrix}=\begin{vmatrix}\end{vmatrix}\]
			\[D=\begin{vmatrix}b_{11}&b_{12}\\a_{21}&a_{22}\end{vmatrix}=\det(A)+\det(B)+\det(C)+\det(D)\]
			Hence using the result of a previous subproblem, we have

	\end{enumerate}
	}
\item[\# 13.]{{\it Let $A$ by symmetric tridiagonal matrix }
	}
	
%Sec 2.1 #3eg, 4bcd, 6, 10, 11, 12, 13
\section{Section 2.2}
%Sec 2.2 #4, 5
\end{description}
\end{document}
