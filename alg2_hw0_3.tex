\documentclass[8pt]{article} % use larger type; default would be 10pt

%\usepackage[utf8]{inputenc} % set input encoding (not needed with XeLaTeX)
\usepackage[10pt]{type1ec}          % use only 10pt fonts
\usepackage[T1]{fontenc}
%\usepackage{CJK}
\usepackage{graphicx}
\usepackage{float}
\usepackage{CJKutf8}
\usepackage{subfig}
\usepackage{amsmath}
\usepackage{amsfonts}
\usepackage{hyperref}
\usepackage{enumerate}
\usepackage{enumitem}

%theorem environments configuration
\newtheorem{theorem}{Theorem}
\newtheorem{claim}{Claim}
\newtheorem{lemma}[theorem]{Lemma}
\newtheorem{proposition}[theorem]{Proposition}
\newtheorem{corollary}[theorem]{Corollary}
\newenvironment{proof}[1][Proof]{\begin{trivlist}
\item[\hskip \labelsep {\bfseries #1}]}{\qed\end{trivlist}}
\newenvironment{definition}[1][Definition]{\begin{trivlist}
\item[\hskip \labelsep {\bfseries #1}]}{\end{trivlist}}
\newenvironment{example}[1][Example]{\begin{trivlist}
\item[\hskip \labelsep {\bfseries #1}]}{\end{trivlist}}
\newenvironment{remark}[1][Remark]{\begin{trivlist}
\item[\hskip \labelsep {\bfseries #1}]}{\end{trivlist}}
\newcommand{\qed}{\nobreak \ifvmode \relax \else
\ifdim\lastskip<1.5em \hskip-\lastskip
\hskip1.5em plus0em minus0.5em \fi \nobreak
  \vrule height0.75em width0.5em depth0.25em\fi}

\newtheorem{prob}{Problem}

\newenvironment{solution}%
{\par\textbf{Solution}\space }%
{\par}

%custom commands to save typing
\newcommand{\F}{\mathbb{F}_2(u)}
\newcommand{\charac}{\text{char}}

\title{Algebra II\\Homework 0.2}
\author{Oleksii (Alex) Leontiev\\歐立思\\3035078276\\Exchange Student (BSc)\\4th grade\\
Original School: \href{http://www.nctu.edu.tw/}{NCTU}, Taiwan}

\begin{document}
\begin{CJK}{UTF8}{bsmi}
\maketitle
\end{CJK}
\begin{claim}
	Suppose $F$ is infinite field and $F(\alpha)$ is its algebraic extension. Then, the number of intermediate fields
	$F\subset E\subset F(\alpha)$ is finite.
\end{claim}
\begin{proof}
	First, given such intermediate field $E$ let us associate with it $p_E(x)\in F(\alpha)[x]$ which will denote \textit{irreducible
	polynomial of $\alpha$ over $E$}. We claim that such correspondence is injective. Indeed, let $F\subset E_1,E_2\subset F(\alpha)$
	and $p_{E_1}(x)=p_{E_2}(x)=:p(x)$. Then, since $p\in E_1[x],E_2[x]$ we have $p(x)\in E[x]$, where $E:=E_1\cap E_2$. Then,
	$p(x)$ is also irreducible over $E\subset E_1$ (since it is irreducible over $E_1$) and sends $\alpha$ to 0, thus showing
	$p=p_E$ and giving us $[E(\alpha):E]=
	[E_1(\alpha):E_1]$. But since $E(\alpha)=E_1(\alpha)=F(\alpha)$,
	this implies $[E_1:E]=[F(\alpha):E]/[F(\alpha):E_1]=
	1\implies E_1=E$. Similarly, $E_2=E$ and hence $E_1=E_2$. Hence, $E\mapsto p_E$ is injective.

	Now, for any intermediate field $E$, $p_E|p_F$ and hence, number of such $p_E$ is finite. Hence, by injectivity of $E\mapsto p_E$
	number of intermediate fields $E$ is also finite, as claimed.
\end{proof}
\end{document}
