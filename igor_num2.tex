\documentclass[8pt]{article} % use larger type; default would be 10pt

%\usepackage[utf8]{inputenc} % set input encoding (not needed with XeLaTeX)
\usepackage[10pt]{type1ec}          % use only 10pt fonts
\usepackage[T1]{fontenc}
%\usepackage{CJK}
\usepackage{graphicx}
\usepackage{float}
\usepackage{CJKutf8}
\usepackage{subfig}
\usepackage{amsmath}
\usepackage{amsfonts}
\usepackage{hyperref}
\usepackage{enumerate}
\usepackage{enumitem}

\newtheorem{prob}{Problem}

\newenvironment{solution}%
{\par\textbf{Solution}\space }%
{\par}

\title{Numerical Analysis\\Homework 1}
\author{Igor Tereshkov\\9722056\\Department of Applied Mathematics\\National Chiao Tung University}
\begin{document}
\begin{CJK}{UTF8}{bsmi}
\maketitle
\end{CJK}

\begin{prob}
\end{prob}
\begin{solution}
	Given inequality may be rewritten in the form \[2^{-q}x\leq x-y \leq 2^{-p}y\]
	Therefore, we see that, for example, $x-y \geq 2^{-q}$, that is the difference is no smaller than the $x$ shifted on the right by $q$
	bits. Therefore, in subtraction there will be no more than $q$ significant digits of $x$ lost. Similarly, there will be no less than 
	$p$ significant digits lost.
\end{solution}
\begin{prob}
\end{prob}
\begin{solution}
	In subsequent, we will denote the floating-point form for $y\in\mathbb{R}$ by $fl(y)$. We will also write
	$fl(x_0+x_1+\dots+x_n)$ as a shorthand for $fl(fl(\dots fl(fl(x_0+x_1)+x_2)\dots+x_{n-1})+x_n)$
	We will proceed by induction on $n$. According to the definition of the machine unit roundoff,
	\[|\frac{fl(x_0+x_1)-(x_0+x_1)}{x_0+x_1}|\leq \epsilon_M=(1+\epsilon_M)^1-1\]
	Which precisely gives the base case. Now, assume that
	\[|\frac{fl(x_0+x_1+\dots+x_n)-(x_0+x_1+\dots+x_n)}{x_0+x_1+\dots+x_n}|\leq (1+\epsilon_M)^n-1\]
	We want to show that
	\[fl(fl(x_0+x_1+\dots+x_n)+x_{n+1})=(x_0+\dots+x_{n+1})(1+z''),\;|z''|\leq (1+\epsilon_M)^{n+1}-1\]
	Notice, that we know that $x_i>0$, therefore their sum (and its floating-point form) both will be non-negative, therefore $1+z''\geq0$.
	From the assumption above,
	\[fl(x_0+x_1+\dots+x_n)=(x_0+x_1+\dots+x_n)\times(1+z),\;|z|\leq (1+\epsilon_M)^n-1\]
	As a consequence,
	\[fl(fl(x_0+x_1+\dots+x_n)+x_{n+1})=(fl(x_0+x_1+\dots+x_n)+x_{n+1})(1+z'),\;|z'|\leq \epsilon_M\]
	\[(fl(x_0+x_1+\dots+x_n)+x_{n+1})(1+z')=((x_0+x_1+\dots+x_n)(1+z)+x_{n+1})(1+z')\]
	Hence,
	\[|z''|=|\frac{((x_0+x_1+\dots+x_n)(1+z)+x_{n+1})(1+z')}{x_0+x_1+\dots+x_{n+1}}-1|=\]
	\[=|\frac{z'(x_0+x_1+\dots+x_{n+1})+z(1+z')(x_0+\dots+x_n)}{{x_0+x_1+\dots+x_{n+1}}}|\leq |z'|+|z|(1+|z'|)\leq\] 
	\[\leq\epsilon_M+(1+\epsilon_M)((1+\epsilon_M)^n-1)=(1+\epsilon_M)^{n+1}-1\]
\end{solution}
\begin{prob}
\end{prob}
\begin{solution}
	As we know, $x(y+z)$ will be evaluated as
	\[fl(fl(x)\times fl(fl(y)+fl(z)))\]
	Note, that in general, $\alpha=\infty$, for in the case when $y$, $z$ have opposite signs, are unequal and very small in magnitude,
	we will have $fl(y)=fl(z)=0$, therefore $fl(x(y+z))=0$, while $x(y+z)\neq 0$.\\ Therefore, in subsequent we will assume that $x,y,z>0$.
	We further assume that $\epsilon_M$ is small.
	Also we know $fl(x)=x(1+z_x),\;fl(y)=y(1+z_y),\;fl(z)=z(1+z_z)$ and $|z_x|,|z_y|,|z_z|<\epsilon_M$. Now,
	\[fl(fl(y)+fl(z))=(fl(y)+fl(z))(1+z_{y+z})\leq (y+z)(1+\epsilon_M)(1+\epsilon_M)<(y+z)(1+3\epsilon_M)\]
	Hence,
	\[fl(fl(x)fl(fl(y)+fl(z))) \leq x(1+z_x)(y+z)(1+3\epsilon_M)\leq x(y+z)(1+5\epsilon_M)\]
	Lower bound proved similarly, $\alpha=\epsilon_M$
\end{solution}
\begin{prob}
\end{prob}
\begin{solution}
	Definitely, this function will be nonzero for non-positive numbers. So, let's restrict ourselves to positive floating point numbers.
	In the real numbers the solution to this equation is $x=1/3$ which cannot be represented in IEEE arithmetic. Thus, in IEEE arithmetic
	this function does not have zeros.
\end{solution}
\begin{prob}
\end{prob}
\begin{solution}
	\begin{enumerate}[label=(\alph*)]
		\item{\[\sqrt{x+1}-1=\frac{x}{\sqrt{x+1}+1}\]}
		\item{\[\sin x-\sin y=2 \sin(\frac{x-y}{2})\cos(\frac{x+y}{2})\]}
		\item{\[x^2-y^2=(x-y)\times(x+y)\]}
		\item{\[(1-\cos x)/\sin x=\frac{2\sin^2 (x/2)}{2\sin (x/2)\cos (x/2)}=\tan (\frac{x}{2})\]}
		\item{\[c=(a^2+b^2-2ab\cos \theta)^{\frac{1}{2}}=((a-b)^2+2ab(1-\cos \theta))^{\frac{1}{2}}=
			((a-b)^2+4ab(\sin^2(\theta/2)))^{\frac{1}{2}}\]}
	\end{enumerate}
\end{solution}

\end{document}
