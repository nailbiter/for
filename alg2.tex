\documentclass[8pt]{article} % use larger type; default would be 10pt

%\usepackage[utf8]{inputenc} % set input encoding (not needed with XeLaTeX)
%\usepackage{CJK}
\usepackage[margin=1in]{geometry}
\usepackage{graphicx}
\usepackage{float}
\usepackage{subfig}
\usepackage{amsmath}
\usepackage{amsfonts}
\usepackage{hyperref}
\usepackage{enumerate}
\usepackage{enumitem}

\usepackage{mystyle}

\title{Homework 2, Math 5111}
\author{Alex Leontiev, 1155040702, CUHK}
\begin{document}
\maketitle
\begin{enumerate}[label=\bfseries Problem \arabic*.]
	\item{\begin{enumerate}[label=(\arabic*).]
			\item{There's not that much to check, in fact following \cite[subsection 10.1.1]{tb} we just verify
				\begin{enumerate}[label=(\roman*)]
					\item{{\it Associativity: } Let $A,B,C,D\in ob(Cor)$, $h\in hom_{Cor}(A,B),g\in
						hom_{Cor}(B,C),f\in hom_{Cor}(C,D)$. We need to show that $f\circ(g\circ h)=(f\circ g)\circ h$.
						As both left-hand side and right-hand side are by definition subsets of $A\times D$, we need to check
						that these two sets are equal.
						
						First, assume $(a,d)\in f\circ(g\circ h)$. By definition of composition in this category, this means
						that there is $c\in C$, such that $(a,c)\in g\circ h$ and $(c,d)\in f$. Consequently, by definition
						the fact that $(a,c)\in g\circ h$ means that for some $b\in B$, $(a,b)\in h,\;(b,c)\in g$. Thus
						as $(b,c)\in g,(c,d)\in f$ we have that $(b,d)\in f\circ g$. As moreover $(a,b)\in h$ we have that
						$(a,d)\in (f\circ g)\circ h$ and thus $f\circ(g\circ h)\subset(f\circ g)\circ h$.
						
						Conversely, suppose $(a,d)\in (f\circ g)\circ h$. This means that for some $b\in B$,
						$(a,b)\in h$ and $(b,d)\in f\circ g$. The latter consequently means that for some
						$c\in C$ we have $(b,c)\in g$ and $(c,d)\in f$. Thus, $(a,c)\in g\circ h$, because
						$(a,b)\in h$ and $(b,c)\in g$, and consequently $(a,d)\in f\circ(g\circ h)$ (because
						$(c,d)\in f$ and $(a,c)\in g\circ h$). Thus, $f\circ(g\circ h)\supset(f\circ g)\circ h$,
						finishing the proof.
						}
					\item{{\it Identity: } For $A\in ob(Cor)$ we shall denote $1_A:=\mysetn{(a,a)}{a\in A}\in hom_{Cor}(A,A)$.
						We need to show that for $f\in hom_{Cor}(A,B)$, $f=f\circ 1_A$ and for $g\in hom_{Cor}(B,A),\;
						g=1_A\circ g$. Let's start with the first one.
						
						To begin with, if $(a,b)\in f$, then by definition $(a,a)\in 1_A$ and thus $(a,b)\in f\circ 1_A$,
						hence $f\subset f\circ 1_A$.
						Conversely, if $(a,b)\in f\circ 1_A$, by definition of composition this means that for some $a'
						\in A$, we have $(a,a')\in 1_A$ and $(a',b)\in f$. But from the definition of $1_A,
						\;(a',a)\in 1_A\implies a'=a\implies (a',b)=
						(a,b)\in f$. Thus $f\supset f\circ 1_A$ and two sets are equal.
						
						Secondly, let us show that $g=1_A\circ g$ for arbitrary $g\in hom_{Cor}(B,A)$. Similarly,
						to above, if $(b,a)\in g$, then as $(a,a)\in 1_A$ by definition, $(b,a)\in 1_A\circ g$, thus
						$g\subset 1_A\circ g$. Conversely, if $(b,a)\in 1_A\circ g$, then for some $a'\in A$ we have
						$(b,a')\in g$ and $(a',a)\in 1_A\implies a'=a$ (by definition of $1_A$), thus $(b,a)=(b,a')\in 
						g$, hence $1_A\circ g\subset g$ and two sets are equal.
						}
				\end{enumerate}
				}
			\item{Recall how the morphisms on $SET$ are defined. For $A,B\in ob(SET)$ the $\hom_{SET}$ by definition consists of
				{\it functions} from $A$ to $B$, where {\it function} is in turn defined to be a subset $f$ of $A\times B$ such that
				$\forall a\in A\;\exists! b\in B$ such that $(a,b)\in f$. Therefore, we naturally have
				$\hom_{SET}(A,B)\subset \hom_{COR}(A,B)=2^{A\times B}$ (here by $2^X$ we shall denote the set of set of all subsets
				of set $X$). In the light of this and the fact that $ob(SET)=ob(COR)=\{\mbox{sets}\}$ we may defined
				$F:ob(SET)\ni S\mapsto F(S)=S\in ob(COR)$ and $F: hom_{SET}(A,B)\ni f\mapsto F(f)\in hom_{COR}(F(A),F(B))
				= hom_{COR}(A,B)$ to be simply inclusion. As inclusion is injective by definition, all that still remains to be
				done is show that $F$ is indeed a functor. For this, again following \cite[subsection 10.3.1]{tb}, we just need
				to verify the {\it functorial property}.

				First, let $A,B,C\in ob(SET)=ob(COR)$ and $h\in hom_{SET}(A,B),\;g\in hom_{SET}(B,C)$ we want to show that
				$F(g\circ h)=F(g)\circ F(h)$. Let us recall, how the composition for functions is defined. It is defined to be
				\[g\circ f=\mysetn{(a,c)\in A\times C}{\exists b\in B,\;(a,b)\in h,\;(b,c)\in g}\]
				As this is exactly the same as composition in $COR$, functor respects composition.

				Second, we need to show that for $A\in ob(SET)=ob(COR)$, $F(1_A)=1_{F(A)}=1_A$. But this directly follows
				from the way we define $1_A$ in both categories.
				}
		\end{enumerate}
		}
	\item{First, let us show the existence of products. Given family of sets $\left\{A_i\right\}_{i\in I}$ 
		together with corresponding $\mycol{f_i:A_i\mapsto S}{i\in I}$, we define
		the product of its elements as $f:S\mapsto X$,
		where $X:=\mysetn{a\in\Pi_{i\in A}A_i}{\forall i,j\in I,\;f_i(\pi_i(a))=f_j(\pi_j(a))}$ and
		$f(a)=f_i(\pi_i(a))$ ($i\in I$ in definition of $f$ can be taken arbitrary and any choice will
		give the same result by definition of $X$).
		Morphisms $p_i:X\mapsto A_i$ are in turn defined as restrictions of $\pi_i$ to $X$. To show that these
		are morphisms we need to verify $\forall i\in I,\;f=f_i\circ p_i$, but this is directly follows from definition
		of $f$ and the way we defined $X$. To prove the universal property, let $g:Y\mapsto S$ be another object in $SET_S$
		and $g_i:Y\mapsto A_i$ (so that $\forall i\in I,\;g=f_i\circ g_i$). Let's take arbitrary $y\in Y$, we have then that
		$\forall i\in I,\;f_i(g_i(y))=g(y)$ and therefore $\left(g_i(y)\right)_{i\in I}\in X$. Mapping $\overline{f}:Y\ni y
		\mapsto \overline{f}(y):=\left(g_i(y)\right)_{i\in I}\in X$ is a morphism (as $f\circ\overline{f}=g$ because
		$\forall y\in Y,\;f(\overline{f}(y))=f\left(\left(g_i(y)\right)_{i\in I}\right)=f_i(g_i(y))$ 
		for arbitrary chosen $i\in I$ by definition of $f$, and $f_i(g_i(y))=g(y)$ as $f_i\circ g_i=g$, because $g_i$ is a
		morphism in $SET_S$) and $\forall i\in I,\;f_i\circ\overline{f}=g_i$ as for arbitrary $y\in Y$ $f_i(\overline{f}(y))=
		\pi_i\left(\left(g_i(y)\right)_{i\in I}\right)=g_i(y)$.
		. Finally, such $\overline{f}$ is unique, as we should have $\forall i\in I,\;g_i=f_i\circ\overline{f}\implies
		\forall i\in I\forall y\in Y,\;g_i(y)=\pi_i(\overline{f}(y))\implies \overline{f}(y)=\left(g_i(y)\right)_{i\in I}
		\in X$.

		Second, let's show that coproduct of sets $\left\{A_i\right\}_{i\in I}$ 
		together with corresponding $\mycol{f_i:A_i\mapsto S}{i\in I}$ exists. We define their coproduct simply as
		$f:X\ni(a,i)\mapsto f_i(a)\in S$, where $X:=\mysetn{(a,i)}{i\in I,\;a\in A_i}$ together with morphisms 
		$p_i:A_i\ni a\mapsto (a,i)\in X$. The latter is indeed a morphism for each $i\in I$, for $\forall a\in A_i,\;
		f(p_i(a))=f(a,i)=f_i(a)\implies f\circ p_i=f_i$. Finally, let us verify the universal property. Assume
		$g:Y\mapsto S$ is an object in $SET_S$ together with morphisms $g_i:A_i\mapsto Y$. Let us construct the morphism
		$\overline{f}:X\ni (a,i)\mapsto \overline{f}(a,i):=g_i(a)\in Y$. First, this is a morphism, for $\forall (a,i)\in X,\;
		g(\overline{f}(a,i))=g(g_i(a))=f_i(a)$ (for $g_i:A_i\mapsto Y$ is a morphism), thus $\forall (a,i)\in X
		g(\overline{f}(a,i))=f_i(a)=f(a,i)$ and $g\circ\overline{f}=f$. Second, $\forall i\in I$ we have $\overline{f}\circ
		f_i=g_i$, because for fixed arbitrary $i\in I$ and arbitrary $a\in A_i$ we have $\overline{f}(f_i(a))=\overline{f}(a,i
		)=g_i(a)$, thus $\overline{f}\circ f_i=g_i$ as required.
		}
	\item{\begin{enumerate}[label=(\arabic*).]
				\renewcommand{\char}{\mbox{char}}
				\renewcommand{\deg}{\mbox{deg}}
			\item{In the field $\overline{F}$ (algebraic closure of $F$) consider the set $K$ of elements satisfying polynomial equation
				$x^{q^2}-x=0$ with coefficients in $F$. To begin with, $K$ is a field. Indeed, $0,1\in K$
				(here we mean zero and one as elements of $F$), and if $a,b\in K$ we have
				$(a+b)^{q^2}-(a+b)=a^{q^2}-a+b^{q^2}-b=0\implies a+b\in K$ (recall that as $\char A=\char F=p$ for some
				prime $p\mid q$
				we have that in $\forall a,b\in A,\;(a+b)^p=a^p+b^p\implies (a+b)^q=a^q+b^q$). Similarly, if $a,b\in K$
				we have $(ab)^{q^2}-ab=a^{q^2}b^{q^2}-ab=ab-ab=0$ and if $a\neq 0$ we also have $\left(\frac{1}{a}\right)
				^{q^2}-\frac{1}{a}
				=\frac{a-a^{q^2}}{q^{q^2+1}}=0$ and hence $K$ contains zero and one and is closed under additions, multiplication
				and taking the inverse. Hence, it is a field. Moreover, $K\supset F$, as if $F\ni a\neq 0$ then $a^{q-1}=1\implies
				a^{q^2}-a=a(\left(a^{q-1}\right)^{q+1}-1)=a(1^{q+1}-1)=0$ and $0\in K$.

				Let us further compute the size of $K$. Note, that the formal derivative of $x^{q^2}-x$ is
				$\frac{d}{dx}(x^{q^2}-x)=-1$ (recall that $\char K=\char F=p\mid q$) and hence it has no repeated roots. Moreover,
				as $A$ is algebraically closed, $x^{q^2}-x$ completely splits in $A$.
				Therefore, $\myabs{K}= \deg\left(x^{q^2}-x\right)=q^2$. Hence, $[K:F]=2$.

				Let us factor $x^{q^2}-x$ as a product of monic irreducible polynomials over $F$. 
				Assume some irreducible $f(x)$ divides $x^{q^2}-x$.
				Then splitting field of $f(x)$ is an intermediate field of $K$ and $F$. However, as $[K:F]=2$, this intermediate
				field should be either $K$ (in this case it has degree 2 as is degree of extension
				) or $F$ (in which case $f(x)$ splits over $F$ and as $f(x)$ was irreducible by
				assumption, it should be linear). Conversely, every linear polynomial $x-a,\;a\in F$ divides $x^{q^2}-x$, as $
				a^{q^2}-a=0$ for every $a\in F$, as was shown above. Hence, every linear polynomial appears in aforementioned
				decomposition and it can appear only once (as $x^{q^2}-x$ has no repeated roots).

				Similarly, if irreducible $f(x)$ divides $x^{q^2}-x$ and has degree 2, it cannot appear twice in decomposition,
				as $x^{q^2}-x$ has no repeated roots. Let us show that {\it every} monic irreducible over $F$ divides $x^{q^2}-x$.
				Let $f(x)$ be irreducible over $F$, $E\subset A$ it's splitting field and $\alpha\in E$ it's root. As
				$\deg f=2$, $[E:F]=2\implies \myabs{E}=q^2\implies \alpha^{q^2-1}=1$ (as $\alpha\neq 0\in F$) and thus
				$\alpha^{q^2}-\alpha=0$ and since $\alpha$ was arbitrary, we get $f\mid x^{q^2-q}$.

				Now we have that all monic linear (there are $q$ of them)
				and monic quadratic irreducible polynomials over $F$ appear in decomposition of
				$x^{q^2}-x$ in irreducible factors, none appears twice and no other polynomials appear there (as was shown above, 
				irreducible factor of $x^{q^2}-x$ can have only degree 1 or 2). Thus, summing their degrees should give us
				degree of $x^{q^2}-x$. Finally, denoting number of irreducible monic quadratic polynomials over $F$ as $k$ we get
				$2k+q=q^2\implies k=\frac{q^2-q}{2}$ as required.
				}
			\item{Given $A\in M_2(F)$ we shall denote it's characteristic polynomial by $p_A$. Note, that similar
				matrices have the same characteristic polynomials, so it makes sense to talk about the characteristic
				polynomial of equivalence class (by the way, we shall denote them by $[A]$ for $A\in M_2(F)$).
				First, let us compute the number of equivalence
				classes, whose characteristic polynomial is irreducible. We shall show, that when restricted to
				these classes the mapping of class to characteristic polynomial is bijective, hence computations
				in previous item show that the number of classes with irreducible characteristic polynomials is
				\[N_1=\frac{q^2-q}{2}\]

				The mapping mentioned above is clearly onto, as for $p(x)=x^2+px+q$ the matrix
				\[A:=\begin{pmatrix}-p&-q\\1&0\end{pmatrix}\]will clearly have $p_A=p$. Let us proceed to show
				it is injective. Suppose for some $A,\;B\in M_2(F)$, $p_A=p_B\in F[x]$
				is monic irreducible and let us show that $[A]=[B]$. 

				\begin{lemma}\end{lemma}
				\begin{proof}Let me apologize to begin with\qed\end{proof}
				}
		\end{enumerate}
		}
	\item{
		\newcommand{\m}{\mathfrak{m}}
		In subsequent we shall denote equivalence classes of $\mathfrak{m}/\mathfrak{m}^2$ and $R/\mathfrak{m}$ as $[\cdot]$ and
		$[\cdot]'$ respectively. Now, given $[r]'\in R/\m$ and $[u],[v]\in \m^2/\m$ let us define addition and scalar multiplication
		simply as $[r]'\cdot[u]:=[ru]$ (notation $[ru]$ here makes sense, for $ru\in\m$ if $r\in R,\;u\in\m$ as $\m$ is and 
		ideal in $R$)
		and $[u]+[v]:=[u+v]$ respectively. We just need to show that these are well-defined.

		As a brief detour, let us not that $\m^2$ is an additive subgroup of $R$ by definition and thus all its elements
		have form of finite sums $\sum_{i=1}^n a_ib_i,\;a_i,b_i\in\m$. Besides, as for any $r\in R$ we have $r\cdot
		\sum_{i=1}^n a_ib_i=\sum_{i=1}^n a_ib_i\cdot r=\sum_{i=1}^n a_ib_i'$, where $b_i':=b_i\in r\in\m$ as $\m$ is
		an ideal, hence $r\cdot\sum_{i=1}^n a_ib_i\in\m^2$ and this latter is a (two-sided) ideal.

		Let's start with showing that scalar multiplication is well-defined.
		Let $[s]'=[r]'\in R/\m$ and $[u]=[v]\in \m^2/\m$ we need to show that
		$[su]=[rv]$. As $[s]'=[r]'$ we have $r-s=m\in\m$ and similarly $v-u=q\in\m^2$. Then $[rv]=[(r+m)(u+q)]=
		[ru+mu+rq+mq]=[ru]$, as $mu\in\m^2\implies[mu]=[0]$ (because $m,u\in\m$), similarly $rq\in\m^2$ (as $q\in \m^2$ and
		$\m^2$ is an ideal, as shown above) and $\m q\in\m^2$ (as $q\in \m^2$ and $\m^2$ is an ideal, as shown above).

		Second, let's go to addition. Let $[u]=[u'],\;[v]=[v']$ Then $u'=u+d,\;v'=v+d'$, for some $d,d'\in\m^2$ and hence
		$[u'+v']=[u+v+(d+d')]=[u+v]$, as $d+d'\in\m^2$, since latter is closed under addition by definition.

		Finally, having that both scalar multiplication and addition well-defined it just remains to verify the 
		axioms of a vector space.
		\begin{itemize}
			\item{$[u]+([v]+[w])=([u]+[v])+[w]$ -- this follows from the associativity of addition in $R$.}
			\item{$[u]+[v]=[v]+[u]$ -- this follows from the commutativity of addition in $R$.}
			\item{$[0]+[u]=[u]$ -- by definition of addition $[0]+[u]=[0+u]=[u]$ (this, $[0]$ is a zero-vector
				in vector space $Spec(R)$).}
			\item{$[-v]+[v]=[0]$ -- again, this follows from how we defined addition. It can only be mentioned that
				$-v\in\m$ whenever $v\in\m$, as $\m$ is an additive subgroup of $R$.
				}
			\item{$[s]'([u]+[v])=[s]'[u]+[s]'[v]$ -- applying definitions of addition and scalar multiplication to both
				sides we get that equivalently we need to show that $[s(u+v)]=[su+sv]$. This follows from distributive
				law in $R$.}
			\item{$([s]'+[t]')[u]=[s]'[u]+[t]'[u]$ -- again, 
				applying definitions of addition and scalar multiplication
				to both sides we get that equivalently we need to show that $[(s+t)u]=[su+tu]$ and again
				this follows from distributive law in $R$.}
			\item{$[s]'([t]'[u])=([s]'[t]')[u]$ -- 
				applying definitions of addition and scalar multiplication
				to both sides we get equivalent statement $[stu]=[stu]$, which is true.}
			\item{$[1]'[v]=[v]$ -- from the way we defined scalar multiplication, $[1]'[v]=[1\cdot v]=[v]$.}
		\end{itemize}
		}
	\item{\begin{enumerate}[label=(\arabic*).]
			\newcommand{\m}{\mathfrak{m}}
		\item{We shall denote the equivalence classes in $k_{\m}$ as $[x]$ for $x\in C^{\infty}(U)$. Let us define a mapping
			from $F:k_{\m}\ni[f]\mapsto F([f]):=f(0)\in\mathbb{R}$
			. It is well defined, as if $[f]=[g]$ this means that $g=f+h,\;h\in\m\implies
			h(0)=0\implies F(g)=(f+h)(0)=f(0)=F(f)$. In subsequent, we will show $F$ is in fact field isomorphism. 
			
			First, $F$ respects addition, for
			\[F([f]+[g])=F([f+g])=(f+g)(0)=F([f])+F([g])\]
			and also multiplication, for
			\[F([f][g])=F([fg])=(fg)(0)=F(f)F(g)\]
			it also maps $[1]\in k_{\m}$ (which is an identity element in $k_{\m}$) to 1, as
			\[F([1])=1\]
			moreover, it is onto, as for $\forall a\in\mathbb{R}$ holds
			\[F([a])=a\]
			and bijective, for $F([f])=F([g])\implies f(0)=g(0)\implies (f-g)(0)=0\implies f-g\in\m\implies [f]=[g]$. Altogether
			these properties show that $F$ is indeed a field isomorphism between $k_{\m}$ and $\mathbb{R}$.
			}
		\item{Abusing notation from the previous items, we will denote the equivalence classes in $\m/\m^2$ again by $[f]$ for
			$f\in\m$. 
			Let us further denote coordinate functions of $\mathbb{R}^n$ by
			$\left\{f_i:\mathbb{R}^n\supset U\mapsto\mathbb{R}\right\}_{i=1}^n$
			, that is $f_i(x)=x_i$ for
			$x=(x_1,x_2,\dots,x_n)\in\mathbb{R}^n$. We shall show that $\left\{[f_i]\right\}_{i=1}^n$
			form a basis for $k_{\m}$ and since the cardinality of the basis is $n$, so is the dimension of $\m/\m^2$.}

			First, let us show $\left\{[f_i]\right\}_{i=1}^n$ are linearly independent. For assume for some $\left\{\lambda_i\right\}_{
			i=1}^n\subset\mathbb{R}$ we have $[0]=\sum_{i=1}^n\lambda_i[f_i]=\left[\sum_{i=1}^n\lambda_if_i\right]\implies
			\sum_{i=1}^n\lambda_if_i\in\m^2$. By the observation from the previous problem this means that $\sum_{i=1}^n\lambda_if_i=
			\sum_{j=1}^m a_jb_j$ for some $\myfincol{a_i}{m},\;\myfincol{b_i}{m}\subset\m$. Now, let us define linear operators
			$D_i:\m\ni f\mapsto \frac{\partial f}{\partial x_i}\mid_{x=0}\in\mathbb{R}$ for $1\leq i\leq n$. They are clearly linear
			and moreover have the property that $D_i(f_j)=\delta_{ij}$ (by $\delta_{ij}$ we denote Kronecker's symbol) and
			$\forall 1\leq i\leq n,\;\forall f,g\in\m,\;D_i(fg)=\frac{\partial (fg)}{\partial x_i}\mid_{x=0}=f(0)D_i(g)+g(0)D_i(f)$,
			so in particular $\forall 1\leq i\leq n,\;D_i(ab)=0$ if $a,b\in\m$. Therefore, as $D_i$ are all linear, we have
			$\forall 1\leq j\leq n,\;D_j(\sum_{i=1}^n\lambda_if_i)=0$. However, $D_j(\sum_{i=1}^n\lambda_if_i)=
			\sum_{i=1}^n\lambda_iD_j(f_i)=\sum_{i=1}^n\lambda_i\delta_{ij}=\lambda_j\implies \forall 1\leq j\leq n,\;\lambda_j=0$ hence
			$\left\{[f_i]\right\}_{i=1}^n$ are linearly independent as was claimed.

			Let us further show that they span $\m/\m^2$. We shall use without a proof the following equality that holds for all
			$f\in C^{\infty}(U)$
			\[f(x_1,x_2,\dots,x_n)=f(0,0,\dots,0)+\sum_{i=1}^nf_i(x)\int_0^1\frac{\partial f}{\partial x_i}(tx_1,tx_2,\dots,tx_n)dt\]
			we shall denote $\int_0^1\frac{\partial f}{\partial x_i}(tx_1,tx_2,\dots,tx_n)dt=:g_i(x)$ and note that
			$(g-g(0))(0)=g(0)-g(0)=0\implies g-g(0)\in\m$, hence for $f\in\m$
			\[[f]=\sum_{i=1}^n[f_i][g_i]=\sum_{i=1}^n[f_i][g_i(0)]+\sum_{i=1}^n[f_i][g_i-g_i(0)]\]
			as $g_i-g_i(0)\in\m$ and $f_i\in\m$ we have $f_i(g_i-g_i(0))\in\m^2\implies
			[f]=\sum_{i=1}^n[f_i][g_i(0)]$ and hence $\left\{[f_i]\right\}_{i=1}^n$ span $\m/\m^2$ as was claimed.
		\end{enumerate}
		}
\end{enumerate}
\begin{thebibliography}{9}
	\bibitem{tb} Robert B. Ash, {\em Abstract Algebra: The basic Graduate Year}, digital version provided by author can be found at
		\url{http://www.math.uiuc.edu/~r-ash/Algebra.html}
\end{thebibliography}
\end{document}
