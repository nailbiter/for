\documentclass[8pt]{article} % use larger type; default would be 10pt

%\usepackage[utf8]{inputenc} % set input encoding (not needed with XeLaTeX)
\usepackage{graphicx}
\usepackage{float}
\usepackage{subfig}
\usepackage{amsmath}
\usepackage{amsfonts}
\usepackage{hyperref}
\usepackage{enumerate}
\usepackage{harpoon}
\usepackage{enumitem}
\usepackage{multicol}

\usepackage{mystyle}

\newcommand{\myexplain}[3]{#1\xrightarrow{\text{#2}}#3}
\newcommand{\myexplainf}[4]{#1\xrightarrow{\begin{subarray}{c}\text{#2}\\\text{#3}\end{subarray}}#4}
\newcommand{\myexplainfi}[5]{#1\xrightarrow{\begin{subarray}{c}\text{#2}\\\text{#3}\\\text{#4}\end{subarray}}#5}
\newcommand{\myfrac}[2]{^#1/_#2}

\title{Math 1540\\University Mathematics for Financial Studies\\2013-14 Term 1\\Suggested solutions for\\HW problems Sec 1.3-1.4 (Linear Algebra)}
\begin{document}
\maketitle
\section{Section 1.3}
\begin{description}
\item[\# 4.]{{\it Write each of the following systems of equations as a matrix equation.}
	\begin{multicols}{3}\begin{enumerate}[label=(\alph*)]
		\item $\arraycolsep=1.4pt\def\arraystretch{2.2}
			\begin{array}[t]{rrrrr}
				3x_1 & + {} & 2x_2 & = & 1\\
				2x_1 & - {} & 3x_2 & = & 5
			\end{array}$
		\item $\arraycolsep=1.4pt\def\arraystretch{2.2}
			\begin{array}[t]{rrrrrrr}
				x_1 & + {} & x_2 & & & = & 5\\
				2x_1 & + {} & x_2 & - {} & x_3 & = & 6\\
				3x_1 & - {} & 2x_2 & + {} & 2x_3 & = & 7
			\end{array}$
		\item $\arraycolsep=1.4pt\def\arraystretch{2.2}
			\begin{array}[t]{rrrrrrr}
				2x_1 & + {} & x_2 & + {} & x_3 & = & 4\\
				x_1 & - {} & x_2 & + {} & 2x_3 & = & 2\\
				3x_1 & - {} & 2x_2 & - {} & x_3 & = & 0
			\end{array}$
	\end{enumerate}\end{multicols}
	It is difficult to give some comments on this completely mechanical exercise, so we will just write down the answers
		\begin{enumerate}[label=(\alph*)]
		\item $\begin{pmatrix}3&2\\2&-3\end{pmatrix}\begin{pmatrix}x_1\\x_2\end{pmatrix}=\begin{pmatrix}1\\5\end{pmatrix}$
		\item $\begin{pmatrix}1&1&0\\2&1&-1\\3&-2&2\end{pmatrix}\begin{pmatrix}x_1\\x_2\\x_3
			\end{pmatrix}=\begin{pmatrix}5\\6\\7\end{pmatrix}$
		\item $\begin{pmatrix}2&1&1\\1&-1&2\\3&-2&-1\end{pmatrix}\begin{pmatrix}x_1\\x_2\\x_3
			\end{pmatrix}=\begin{pmatrix}4\\2\\0\end{pmatrix}$
	\end{enumerate}
	}
\item[\# 14.]{
	\renewcommand{\b}{\mathbf{b}}
	\newcommand{\x}{\mathbf{x}}
	{\it For each of the following choices of $A$ and $\b$ determine whether the system $A\x=\b$ is consistent by examining how $\b$ relates
	to the column vectors of $A$. Explain your answer in each case.}
	\begin{multicols}{2}\begin{enumerate}[label=(\alph*)]
		\item $A=\begin{pmatrix}2&1\\-2&-1\end{pmatrix},\quad\b=\begin{pmatrix}3\\1\end{pmatrix}$
		\item $A=\begin{pmatrix}1&4\\2&3\end{pmatrix},\quad\b=\begin{pmatrix}5\\5\end{pmatrix}$
	\end{enumerate}\end{multicols}
	\begin{enumerate}[label=(\alph*)]
		\setcounter{enumi}{2}
		\item $A=\begin{pmatrix}3&2&1\\3&2&1\\3&2&1\end{pmatrix},\quad\b=\begin{pmatrix}1\\0\\-1\end{pmatrix}$
	\end{enumerate}
	Here is how we shall proceed
	\begin{enumerate}[label=(\alph*)]
			\newcommand{\icv}[1]{$\begin{pmatrix}#1\end{pmatrix}$}
		\item{Note that both of the columns of $A$ in this case are multiples of $\begin{pmatrix}1\\-1\end{pmatrix}$, any linear
			combination of column vectors will be a multiple
			of $\begin{pmatrix}1\\-1\end{pmatrix}$. Now since $\begin{pmatrix}3\\1\end{pmatrix}$ is clearly not
			a multiple of $\begin{pmatrix}1\\-1\end{pmatrix}$, it cannot be represented as a linear combination of 
			columns of $A$, thus system is inconsistent.}
		\item{As $\begin{pmatrix}5\\5\end{pmatrix}=\begin{pmatrix}1\\2\end{pmatrix}+\begin{pmatrix}4\\3\end{pmatrix}$, $\b$ is the
				sum of columns of $A$, thus it is a linear combination of column vectors of $A$, thus system is consistent.}
		\item As all the column vectors are equal to \icv{3\\2\\1}, the space spanned by columns will consist of the multiples of this 
			vector. As $\b=\begin{pmatrix}1\\0\\-1\end{pmatrix}$ is clearly not of this type, system is inconsistent.
	\end{enumerate}
	}
\item[\# 22.]{{\it Find a $2\times2$ matrices $A$ and $B$ that both are not zero matrix for which $AB=O$.}
	One of the simplest examples is probably
	\[A=\begin{pmatrix}1&0\\0&0\end{pmatrix},\quad B=\begin{pmatrix}0&0\\0&1\end{pmatrix}\]
	As \[AB=\begin{pmatrix}1&0\\0&0\end{pmatrix}\begin{pmatrix}0&0\\0&1\end{pmatrix}=\begin{pmatrix}
		1\cdot0+0\cdot0&1\cdot0+0\cdot1\\0\cdot0+0\cdot0&0\cdot0+0\cdot0
	\end{pmatrix}=\begin{pmatrix}0&0\\0&0\end{pmatrix}=O\]
	}
\item[\# 23.]{{\it Find nonzero matrices $A,\;B,\;C$ such that}
	\[AC=BC\quad\mbox{ and }\quad A\neq B\]
	Note that $AC=BC\iff (A-B)C=0$, so this example is intimately related to the previous one and may be constructed based on it. Thus we will
	take\[A=\begin{pmatrix}2&0\\0&0\end{pmatrix},\quad B=\begin{pmatrix}1&0\\0&0\end{pmatrix},\quad C=\begin{pmatrix}0&0\\0&1\end{pmatrix}\]
	so that $A-B=\bigl(\begin{smallmatrix}1&0\\0&1\end{smallmatrix}\bigr)$ and indeed, this gives
	\[AC=\begin{pmatrix}2&0\\0&0\end{pmatrix}\begin{pmatrix}0&0\\0&1\end{pmatrix}=\begin{pmatrix}0&0\\0&0\end{pmatrix}=
	\begin{pmatrix}1&0\\0&0\end{pmatrix}\begin{pmatrix}0&0\\0&1\end{pmatrix}=BC\]
	while clearly $A\neq B$.
	}
\end{description}
	\section{Section 1.4}
%\newenvironment{name}[num]{before}{after}
	\begin{description}
	\item[\# 4.]{\textit{For each of the following pairs of matrices, find an elementary matrix $E$ such that $AE=B$.}
		\begin{enumerate}[label=(\alph*)]
			\setcounter{enumi}{2}
		\item $A=\left(\begin{array}{rrr}4&-2&3\\-2&4&2\\6&1&-2\end{array}\right),
				\quad B=\left(\begin{array}{rrr}2&-2&3\\-1&4&2\\3&1&-2\end{array}\right)$
		\end{enumerate}
		Notice, that if we will divide the first column of $A$ by 2, we will get precisely $B$. Matrix $E$ which we are required to produce
		is obtained from identity by applying this very operation -- division of the first column by two.
		Hence \[E=\left(\begin{array}{rrr}^1/_2&0&0\\0&1&0\\0&0&1\end{array}\right)\]
		}
	\item[\# 5.]{
		\newenvironment{mymat}{\left(\begin{array}{rrr}}{\end{array}\right)}
		{\it Given
		\[A=\begin{mymat}1&2&4\\2&1&3\\1&0&2\end{mymat},\quad B=\begin{mymat}1&2&4\\2&1&3\\2&2&6\end{mymat},\quad 
		C=\begin{mymat}1&2&4\\0&-1&-3\\2&2&6\end{mymat}\]
		\begin{enumerate}[label=(\alph*)]
			\item Find an elementary matrix $E$ such that $EA=B$.
			\item Find an elementary matrix $F$ such that $FB=C$.
			\item Is $C$ row equivalent to $A$? Explain.
		\end{enumerate}
		}
		\begin{enumerate}[label=(\alph*)]
			\item If we will add the first row of $A$ to its third row, we will get precisely $B$. Therefore, $E$ should be obtained
				from the identity matrix by adding its first row to its third row, hence
				\[E=\begin{mymat}1&0&0\\0&1&0\\1&0&1\end{mymat}\]
			\item If we subtract the third row of of $B$ from its second row, we will get precisely $C$. Hence,
				if we'll apply this operation (subtraction of the third row from the second one) to the identity 
				matrix, we will get $F$. Consequently
				\[F=\begin{mymat}1&0&0\\0&1&-1\\0&0&1\end{mymat}\]
			\item Yes, $C$ and $A$ are row equivalent. As $B=EA$ and $C=FB$ we have $C=(FE)A$ and since $FE$ is a product
				of elementary matrices (as both $F$ and $E$ are products of elementary matrices), $C$ and $A$ are row equivalent.
		\end{enumerate}
		}
	\item[\# 7.]{{\it Given \[A=\begin{pmatrix}2&1\\6&4\end{pmatrix}\]
			\begin{enumerate}[label=(\alph*)]
				\item Express $A$ as a product of elementary matrices.
				\item Express $A^{-1}$ as a product of elementary matrices.
			\end{enumerate}
		}
		\begin{enumerate}[label=(\alph*)]
		\newcommand{\mymat}[1]{\left(\begin{array}{rr}#1\end{array}\right)}
			\item We will just bring $A$ to reduced row echelon form (which will inevitably be identity matrix) and
				encode each operation we do as an elementary matrix. The product of these will give us an answer we
				seek. So, let's start with making a pivot in the first row.
				\[\myexplain
				{\mymat{2&1\\6&4}}
				{\textcircled{1}$/2$}
				{\mymat{1&\myfrac{1}{2}\\6&4}}
				\]
				This operation corresponds to the elementary matrix \[E_1=\mymat{\myfrac{1}{2}&0\\0&1}\]
				Next we eliminate everything under the pivot
				\[\myexplain
				{\mymat{1&\myfrac{1}{2}\\6&4}}
				{\textcircled{2}$-6*$\textcircled{1}}
				{\mymat{1&\myfrac{1}{2}\\0&1}}
				\]
				This corresponds to elementary matrix \[E_2=\mymat{1&0\\-6&1}\]
				As second row is accidentally get pivoted, we eliminate everything above pivot of the second row
				\[\myexplain
				{\mymat{1&\myfrac{1}{2}\\0&1}}
				{\textcircled{2}$-\frac{1}{2}*$\textcircled{1}}
				{\mymat{1&0\\0&1}}
				\]
				This corresponds to elementary matrix \[E_3=\mymat{1&-\myfrac{1}{2}\\0&1}\]
				And hence we get
				\[E_3E_2E_1A=\mymat{1&0\\0&1}\]
				Hence $A=E_1^{-1}E_2^{-1}E_3^{-1}$. Fortunately, inverting elementary matrices is easy and we can
				directly write the answer as
				\[A=\mymat{2&0\\0&1}\mymat{1&0\\6&1}\mymat{1&\myfrac{1}{2}\\0&1}\]
			\item In the light of the obtained above equality 
				\[E_3E_2E_1A=\mymat{1&0\\0&1}\]
				Therefore, we have have $E_3E_2E_1=A^{-1}$ and hence
				\[A^{-1}=\mymat{\myfrac{1}{2}&0\\0&1}\mymat{1&0\\-6&1}\mymat{1&-\myfrac{1}{2}\\0&1}\]
		\end{enumerate}
		}
	\item[\# 10.]{
		\newcommand{\mymat}[1]{\left(\begin{array}[t]{rr}#1\end{array}\right)}
		{\it Find the inverse of each of the following matrices.
		\begin{enumerate}[label=(\alph*)]
			\setcounter{enumi}{3}
			\renewcommand{\mymat}[1]{\left(\begin{array}{rr}#1\end{array}\right)}
			\item $\mymat{3&0\\9&3}$
			\setcounter{enumi}{6}
			\renewcommand{\mymat}[1]{\left(\begin{array}{rrr}#1\end{array}\right)}
			\item $\mymat{-1&-3&-3\\2&6&1\\3&8&3}$
		\end{enumerate}
			}
		We shall employ the following method, which will be justified later in problem 20. To invert the matrix (call it $A$) we
		will augment it with identity matrix to get $\left[A\mid I\right]$ (where $I$ is the square identity matrix of a
		corresponding size) and bring it to reduced row echelon form via the row operations. As the result we will get
		$\left[I\mid C\right]$ for some square matrix $C$, which will be exactly the inverse of $C$, as we will prove below in
		problem 20. Let's proceed.
		\begin{enumerate}[label=(\alph*)]
			\setcounter{enumi}{3}
			\renewcommand{\mymat}[1]{\left(\begin{array}{rr|rr}#1\end{array}\right)}
			\item First, we construct the augmented matrix
				\[\mymat{3&0&1&0\\9&3&0&1}\]
			Then, make the pivot in the first row
			\[\myexplain
				{\mymat{3&0&1&0\\9&3&0&1}}
				{\textcircled{1}$/3$}
				{\mymat{1&0&\myfrac{1}{3}&0\\9&3&0&1}}
			\]
			Eliminate everything under the pivot of the first row
			\[\myexplain
				{\mymat{3&0&1&0\\9&3&0&1}}
				{\textcircled{2}$-9*$\textcircled{1}}
				{\mymat{1&0&\myfrac{1}{3}&0\\0&3&-3&1}}
			\]
			Then, make the pivot in the second row
			\[\myexplain
				{\mymat{1&0&\myfrac{1}{3}&0\\0&3&-3&1}}
				{\textcircled{2}$/3$}
				{\mymat{1&0&\myfrac{1}{3}&0\\0&1&-1&\myfrac{1}{3}}}
			\]
			Hence, \[A^{-1}=\left(\begin{array}{rr}\myfrac{1}{3}&0\\-1&\myfrac{1}{3}\end{array}\right)\]
			\setcounter{enumi}{6}
			\renewcommand{\mymat}[1]{\left(\begin{array}{rrr|rrr}#1\end{array}\right)}
			\item Again, we begin by constructing the augmented matrix.
				\[\mymat{-1&-3&-3&1&0&0\\2&6&1&0&1&0\\3&8&3&0&0&1}\]
				Then, pivot the first row
				\[\myexplain
				{\mymat{-1&-3&-3&1&0&0\\2&6&1&0&1&0\\3&8&3&0&0&1}}
				{\textcircled{1}$*(-1)$}
				{\mymat{1&3&3&-1&0&0\\2&6&1&0&1&0\\3&8&3&0&0&1}}
				\]
				Eliminate everything under the pivot
				\[\myexplainf
				{\mymat{1&3&3&-1&0&0\\2&6&1&0&1&0\\3&8&3&0&0&1}}
				{\textcircled{2}$-2*$\textcircled{1}}
				{\textcircled{3}$-3*$\textcircled{1}}
				{\mymat{1&3&3&-1&0&0\\0&0&-5&2&1&0\\0&-1&-6&3&0&1}}
				\]
				As we cannot pivot the second row as is, we bring the third on its place. That is, we exchange the second
				and the third row
				\[\myexplain
				{\mymat{1&3&3&-1&0&0\\0&0&-5&2&1&0\\0&-1&-6&3&0&1}}
				{\textcircled{2}$\leftrightarrow$\textcircled{3}}
				{\mymat{1&3&3&-1&0&0\\0&-1&-6&3&0&1\\0&0&-5&2&1&0}}
				\]
				Pivot the second row
				\[\myexplain
				{\mymat{1&3&3&-1&0&0\\0&-1&-6&3&0&1\\0&0&-5&2&1&0}}
				{\textcircled{2}$*(-1)$}
				{\mymat{1&3&3&-1&0&0\\0&1&6&-3&0&-1\\0&0&-5&2&1&0}}
				\]
				Pivot the third row next, as there is nothing to eliminate under the pivot of the second row
				\[\myexplain
				{\mymat{1&3&3&-1&0&0\\0&1&6&-3&0&-1\\0&0&-5&2&1&0}}
				{\textcircled{3}$*\left(-\frac{1}{5}\right)$}
				{\mymat{1&3&3&-1&0&0\\0&1&6&-3&0&-1\\0&0&1&-\myfrac{2}{5}&-\myfrac{1}{5}&0}}
				\]
				Eliminate everything above the pivot of the third row
				\[\myexplainf
				{\mymat{1&3&3&-1&0&0\\0&1&6&-3&0&-1\\0&0&1&-\myfrac{2}{5}&-\myfrac{1}{5}&0}}
				{\textcircled{2}$-6*$\textcircled{3}}
				{\textcircled{1}$-3*$\textcircled{3}}
				{\mymat{1&3&0&\myfrac{1}{5}&\myfrac{3}{5}&0\\0&1&0&-\myfrac{3}{5}&\myfrac{6}{5}
				&-1\\0&0&1&-\myfrac{2}{5}&-\myfrac{1}{5}&0}}
				\]
				Eliminate everything above the pivot of the second row
				\[\myexplain
				{\mymat{1&3&0&\myfrac{1}{5}&\myfrac{3}{5}&0\\0&1&0&-\myfrac{3}{5}&\myfrac{6}{5}
				&-1\\0&0&1&-\myfrac{2}{5}&-\myfrac{1}{5}&0}}
				{\textcircled{1}$-3*$\textcircled{2}}
				{\mymat{1&0&0&2&-3&3\\0&1&0&-\myfrac{3}{5}&\myfrac{6}{5}
				&-1\\0&0&1&-\myfrac{2}{5}&-\myfrac{1}{5}&0}}
				\]
				Hence the answer is
				\[A^{-1}=\left(\begin{array}{rrr}2&-3&3\\-\myfrac{3}{5}&\myfrac{6}{5}&-1\\-\myfrac{2}{5}&-\myfrac{1}{5}&0
				\end{array}\right)\]
		\end{enumerate}
			}
	\item[\# 17.]{
		\newcommand{\x}{\mathbf{x}}
		{\it Let $A$ and $B$ be $n\times n$ matrices and let $C=A-B$. Show that if $A\x_0=B\x_0$ and $\x_0\neq\mathbf{0}$,
		then $C$ must be singular.} Indeed, $C\x_0=(A-B)\x_0=A\x_0-B\x_0=\mathbf{0}$ and since $\x_0\neq\mathbf{0}$, $C$ is singular.
			}
	\item[\# 18.]{
		\newcommand{\x}{\mathbf{x}}
		{\it Let $A$ and $B$ be $n\times n$ matrices and let $C=AB$. Prove that if $B$ is singular, than $C$ must be singular.}
		Indeed, if $B$ is singular there is vector $\x_0\neq\mathbf{0}$, such that $B\x_0=\mathbf{0}$. Then, $C\x_0=A(B\x_0)=A\mathbf{0}
		=\mathbf{0}$ and hence $C$ is singular, as $\x_0\neq\mathbf{0}$.
		}
	\item[\# 20.]{
		\newcommand{\x}{\mathbf{x}}
		{\it Let $A$ be a nonsingular $n\times n$ matrix and let $B$ be $n\times r$ matrix. Show that the reduced row echelon form of 
		$(A\mid B)$ is $(I\mid C)$, where $C=A^{-1}B$.}
		As $A$ is nonsingular by assumption, its reduced row echelon form is $I$ and hence there is matrix $E$, which is the product
		of elementary matrices, representing row operations needed to bring $A$ to reduced row echelon form, such that $EA=I$. Then,
		applying the same sequence of row operations to $(A\mid B)$ we get $E(A\mid B)=(I\mid C)$. Note, that $(I\mid C)$ is the reduced
		row echelon form of $(A\mid B)$, as they are row equivalent (one obtained from the other one via the sequence of row operations)
		and $(I\mid C)$ is in reduced row echelon form -- it has pivot in each row, they form "ladder" pattern and everything above and
		below every pivot is zero. Now, as $E(A\mid B)=(I\mid C)$ we get $EA=I$ and $EB=C$ (as row operations represented by $E$ can
		be applied to $A$ and $B$ separately). Now, the first equality gives us that $E=A^{-1}$, hence the second can be written as
		$A^{-1}B=C$.
		}
	\item[\# 27.]{
		\newcommand{\x}{\mathbf{x}}
		\renewcommand{\c}{\mathbf{c}}
		\newcommand{\y}{\mathbf{y}}
		\newcommand{\z}{\mathbf{0}}
		{\it Given a vector $\x\in R^{n+1},$ the $(n+1)\times (n+1)$ matrix $V$ defined by}
		\[v_{ij}=\begin{cases}1 & \mbox{ if }j=1\\x_i^{j-1} & \mbox{ for } j=2,\dots,n+1\end{cases}\]
		{\it is called Vandermonde matrix.
		\begin{enumerate}[label=(\alph*)]
			\item Show that if \[V\c=\y\] and \[p(x)=c_1+c_2x+\dots+c_{n+1}x^n\] then \[p(x_i)=y_i,\quad i=1,2,\dots,n+1\]
			\item Suppose that $x_1,x_2,\dots,x_{n+1}$ are all distinct. Show that if $\c$ is a solution to $V\x=\z,$ then the
				coefficients $c_1,c_2,\dots,c_n$ must all be zero, and hence $V$ must be nonsingular.
		\end{enumerate}
		}
		\begin{enumerate}[label=(\alph*)]
			\item There is a direct consequence of the way we defined $V$. Indeed, as $V\c=\y$ we have that for arbitrary $i=1,2,\dots,
				n+1$ $(V\c)_i=\y_i$. Writing $(V\c)_i$ explicitly according to the way matrices are multiplied we get
				\[y_i=(V\c)_i=\sum_{j=1}^{n+1}v_{ij}c_j=c_1+c_2x_i+c_3x_i^2+\dots+c_{n+1}x_i^n=p(x_i)\]
			\item Assuming we have $V\c=\z$, then previous item would imply that for $p(x):=c_1+c_2x+\dots+c_{n+1}x^n$ we have
				$p(x_i)=0$ for every $i=1,2,\dots,n+1$ and hence, as all $x_i$ are distinct, $p(x)$ has $n+1$ distinct zeros
				in $R$. However, as $p(x)$ has degree no more than $n$, the only possibility is that $p(x)\equiv 0$ and hence
				$c_1,c_2,\dots,c_n$ are all zero.
		\end{enumerate}
		}
%Sec. 1.4 #4c, 5, 7, 10dg, 17, 18 20, 27
	\end{description}
\end{document}
