\documentclass[8pt]{article} % use larger type; default would be 10pt

%\usepackage[utf8]{inputenc} % set input encoding (not needed with XeLaTeX)
\usepackage[10pt]{type1ec}          % use only 10pt fonts
\usepackage[T1]{fontenc}
%\usepackage{CJK}
\usepackage{graphicx}
\usepackage{float}
\usepackage{CJKutf8}
\usepackage{subfig}
\usepackage{amsmath}
\usepackage{amsfonts}
\usepackage{hyperref}
\usepackage{enumerate}
\usepackage{harpoon}
\usepackage{enumitem}
\usepackage{multicol}

\usepackage{mystyle}

\newcommand{\myexplain}[3]{#1\xrightarrow{\text{#2}}#3}
\newcommand{\myexplainf}[4]{#1\xrightarrow{\begin{subarray}{c}\text{#2}\\\text{#3}\end{subarray}}#4}
\newcommand{\myexplainfi}[5]{#1\xrightarrow{\begin{subarray}{c}\text{#2}\\\text{#3}\\\text{#4}\end{subarray}}#5}
\newcommand{\myfrac}[2]{^#1/_#2}

\title{Math 1540\\University Mathematics for Financial Studies\\2013-14 Term 1\\Suggested solutions for\\HW problems Sec 1.3-1.4 (Linear Algebra)}
\begin{document}
\maketitle
%Sec. 1.4 #4c, 5, 7, 10dg, 17, 18 20, 27   
\section{Section 1.3}
\begin{description}
\item[\# 4.]{{\it Write each of the following systems of equations as a matrix equation.}
	\begin{multicols}{3}\begin{enumerate}[label=(\alph*)]
		\item $\arraycolsep=1.4pt\def\arraystretch{2.2}
			\begin{array}[t]{rrrrr}
				3x_1 & + {} & 2x_2 & = & 1\\
				2x_1 & - {} & 3x_2 & = & 5
			\end{array}$
		\item $\arraycolsep=1.4pt\def\arraystretch{2.2}
			\begin{array}[t]{rrrrrrr}
				x_1 & + {} & x_2 & & & = & 5\\
				2x_1 & + {} & x_2 & - {} & x_3 & = & 6\\
				3x_1 & - {} & 2x_2 & + {} & 2x_3 & = & 7
			\end{array}$
		\item $\arraycolsep=1.4pt\def\arraystretch{2.2}
			\begin{array}[t]{rrrrrrr}
				2x_1 & + {} & x_2 & + {} & x_3 & = & 4\\
				x_1 & - {} & x_2 & + {} & 2x_3 & = & 2\\
				3x_1 & - {} & 2x_2 & - {} & x_3 & = & 0
			\end{array}$
	\end{enumerate}\end{multicols}
	It is difficult to give some comments on this completely mechanical exercise, so we will just write down the answers
		\begin{enumerate}[label=(\alph*)]
		\item $\begin{pmatrix}3&2\\2&-3\end{pmatrix}\cdot\begin{pmatrix}x_1\\x_2\end{pmatrix}=\begin{pmatrix}1\\5\end{pmatrix}$
		\item $\begin{pmatrix}1&1&0\\2&1&-1\\3&-2&2\end{pmatrix}\cdot\begin{pmatrix}x_1\\x_2\\x_3
			\end{pmatrix}=\begin{pmatrix}5\\6\\7\end{pmatrix}$
		\item $\begin{pmatrix}2&1&1\\1&-1&2\\3&-2&-1\end{pmatrix}\cdot\begin{pmatrix}x_1\\x_2\\x_3
			\end{pmatrix}=\begin{pmatrix}4\\2\\0\end{pmatrix}$
	\end{enumerate}
	}
\item[\# 14.]{{\it }
	}
\item[\# 22.]{{\it }
	}
\item[\# 23.]{{\it }
	}
\end{description}
%Sec. 1.3 #4, 14, 22, 23
\section{Section 1.4}
\begin{description}
\item[\# 5.]{Helo}
\end{description}
\end{document}
