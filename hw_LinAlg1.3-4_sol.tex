\documentclass[8pt]{article} % use larger type; default would be 10pt

%\usepackage[utf8]{inputenc} % set input encoding (not needed with XeLaTeX)
\usepackage{graphicx}
\usepackage{float}
\usepackage{subfig}
\usepackage{amsmath}
\usepackage{amsfonts}
\usepackage{hyperref}
\usepackage{enumerate}
\usepackage{harpoon}
\usepackage{enumitem}
\usepackage{multicol}

\usepackage{mystyle}

\newcommand{\myexplain}[3]{#1\xrightarrow{\text{#2}}#3}
\newcommand{\myexplainf}[4]{#1\xrightarrow{\begin{subarray}{c}\text{#2}\\\text{#3}\end{subarray}}#4}
\newcommand{\myexplainfi}[5]{#1\xrightarrow{\begin{subarray}{c}\text{#2}\\\text{#3}\\\text{#4}\end{subarray}}#5}
\newcommand{\myfrac}[2]{^#1/_#2}

\title{Math 1540\\University Mathematics for Financial Studies\\2013-14 Term 1\\Suggested solutions for\\HW problems Sec 1.3-1.4 (Linear Algebra)}
\begin{document}
\maketitle
\section{Section 1.3}
\begin{description}
\item[\# 4.]{{\it Write each of the following systems of equations as a matrix equation.}
	\begin{multicols}{3}\begin{enumerate}[label=(\alph*)]
		\item $\arraycolsep=1.4pt\def\arraystretch{2.2}
			\begin{array}[t]{rrrrr}
				3x_1 & + {} & 2x_2 & = & 1\\
				2x_1 & - {} & 3x_2 & = & 5
			\end{array}$
		\item $\arraycolsep=1.4pt\def\arraystretch{2.2}
			\begin{array}[t]{rrrrrrr}
				x_1 & + {} & x_2 & & & = & 5\\
				2x_1 & + {} & x_2 & - {} & x_3 & = & 6\\
				3x_1 & - {} & 2x_2 & + {} & 2x_3 & = & 7
			\end{array}$
		\item $\arraycolsep=1.4pt\def\arraystretch{2.2}
			\begin{array}[t]{rrrrrrr}
				2x_1 & + {} & x_2 & + {} & x_3 & = & 4\\
				x_1 & - {} & x_2 & + {} & 2x_3 & = & 2\\
				3x_1 & - {} & 2x_2 & - {} & x_3 & = & 0
			\end{array}$
	\end{enumerate}\end{multicols}
	It is difficult to give some comments on this completely mechanical exercise, so we will just write down the answers
		\begin{enumerate}[label=(\alph*)]
		\item $\begin{pmatrix}3&2\\2&-3\end{pmatrix}\begin{pmatrix}x_1\\x_2\end{pmatrix}=\begin{pmatrix}1\\5\end{pmatrix}$
		\item $\begin{pmatrix}1&1&0\\2&1&-1\\3&-2&2\end{pmatrix}\begin{pmatrix}x_1\\x_2\\x_3
			\end{pmatrix}=\begin{pmatrix}5\\6\\7\end{pmatrix}$
		\item $\begin{pmatrix}2&1&1\\1&-1&2\\3&-2&-1\end{pmatrix}\begin{pmatrix}x_1\\x_2\\x_3
			\end{pmatrix}=\begin{pmatrix}4\\2\\0\end{pmatrix}$
	\end{enumerate}
	}
\item[\# 14.]{
	\renewcommand{\b}{\mathbf{b}}
	\newcommand{\x}{\mathbf{x}}
	{\it For each of the following choices of $A$ and $\b$ determine whether the system $A\x=\b$ is consistent by examining how $\b$ relates
	to the column space of $A$. Explain your answer in each case.}
	\begin{multicols}{2}\begin{enumerate}[label=(\alph*)]
		\item $A=\begin{pmatrix}2&1\\-2&-1\end{pmatrix},\quad\b=\begin{pmatrix}3\\1\end{pmatrix}$
		\item $A=\begin{pmatrix}1&4\\2&3\end{pmatrix},\quad\b=\begin{pmatrix}5\\5\end{pmatrix}$
	\end{enumerate}\end{multicols}
	\begin{enumerate}[label=(\alph*)]
		\setcounter{enumi}{2}
		\item $A=\begin{pmatrix}3&2&1\\3&2&1\\3&2&1\end{pmatrix},\quad\b=\begin{pmatrix}1\\0\\-1\end{pmatrix}$
	\end{enumerate}
	Here is how we shall proceed
	\begin{enumerate}[label=(\alph*)]
			\newcommand{\icv}[1]{$\begin{pmatrix}#1\end{pmatrix}^T$}
		\item{Note that both of the columns of $A$ in this case are multiples of $\begin{pmatrix}1&-1\end{pmatrix}^T$, so the whole
			column space is the space generated by this sole vector. Hence, the column space of $A$ consists of multiples
			of $\begin{pmatrix}1&-1\end{pmatrix}^T$. Now since $\begin{pmatrix}3&1\end{pmatrix}^T$ is clearly not
			a multiple of $\begin{pmatrix}1&-1\end{pmatrix}^T$, it does not belong to a column space of $A$, thus system
			is inconsistent.}
		\item{As $\begin{pmatrix}5&5\end{pmatrix}^T=\begin{pmatrix}1&2\end{pmatrix}^T+\begin{pmatrix}4&3\end{pmatrix}^T$, $\b$ is the
				sum of columns of $A$, thus it belongs to a column space of $A$, thus system is consistent.}
		\item As all the column vectors are equal to \icv{3&2&1}, the whole column space will consist of the multiples of this vector.
			As $\b=\begin{pmatrix}1&0&-1\end{pmatrix}^T$ is clearly not of this type, system is inconsistent.
	\end{enumerate}
	}
\item[\# 22.]{{\it Find a $2\times2$ matrices $A$ and $B$ that both are not zero matrix for which $AB=O$.}
	One of the simplest examples is probably
	\[A=\begin{pmatrix}1&0\\0&0\end{pmatrix},\quad B=\begin{pmatrix}0&0\\0&1\end{pmatrix}\]
	As \[AB=\begin{pmatrix}1&0\\0&0\end{pmatrix}\begin{pmatrix}0&0\\0&1\end{pmatrix}=\begin{pmatrix}
		1\cdot0+0\cdot0&1\cdot0+0\cdot1\\0\cdot0+0\cdot0&0\cdot0+0\cdot0
	\end{pmatrix}=\begin{pmatrix}0&0\\0&0\end{pmatrix}=O\]
	}
\item[\# 23.]{{\it Find nonzero matrices $A,\;B,\;C$ such that}
	\[AC=BC\quad\mbox{ and }\quad A\neq B\]
	Note that $AC=BC\iff (A-B)C=0$, so this example is intimately related to the previous one and may be constructed based on it. Thus we will
	take\[A=\begin{pmatrix}2&0\\0&0\end{pmatrix},\quad B=\begin{pmatrix}1&0\\0&0\end{pmatrix},\quad C=\begin{pmatrix}0&0\\0&1\end{pmatrix}\]
	so that $A-B=\bigl(\begin{smallmatrix}1&0\\0&1\end{smallmatrix}\bigr)$ and indeed, this gives
	\[AC=\begin{pmatrix}2&0\\0&0\end{pmatrix}\begin{pmatrix}0&0\\0&1\end{pmatrix}=\begin{pmatrix}0&0\\0&0\end{pmatrix}=
	\begin{pmatrix}1&0\\0&0\end{pmatrix}\begin{pmatrix}0&0\\0&1\end{pmatrix}=BC\]
	while clearly $A\neq B$.
	}
\end{description}
	\section{Section 1.4}
%\newenvironment{name}[num]{before}{after}
	\begin{description}
	\item[\# 4.]{\textit{For each of the following pairs of matrices, find an elementary matrix $E$ such that $AE=B$.}
		\begin{enumerate}[label=(\alph*)]
			\setcounter{enumi}{2}
		\item $A=\left(\begin{array}{rrr}4&-2&3\\-2&4&2\\6&1&-2\end{array}\right),
				\quad B=\left(\begin{array}{rrr}2&-2&3\\-1&4&2\\3&1&-2\end{array}\right)$
		\end{enumerate}
		Notice, that if we will divide the first column of $A$ by 2, we will get precisely $B$. Matrix $E$ which we are required to produce
		is obtained from identity by applying this very operation -- division of the first column by two.
		Hence \[E=\left(\begin{array}{rrr}^1/_2&0&0\\0&1&0\\0&0&1\end{array}\right)\]
		}
	\item[\# 5.]{
		\newenvironment{mymat}{\left(\begin{array}{rrr}}{\end{array}\right)}
		{\it Given
		\[A=\begin{mymat}1&2&4\\2&1&3\\1&0&2\end{mymat},\quad B=\begin{mymat}1&2&4\\2&1&3\\2&2&6\end{mymat},\quad 
		C=\begin{mymat}1&2&4\\0&-1&-3\\2&2&6\end{mymat}\]
		\begin{enumerate}[label=(\alph*)]
			\item Find an elementary matrix $E$ such that $EA=B$.
			\item Find an elementary matrix $F$ such that $FB=C$.
			\item Is $C$ row equivalent to $A$? Explain.
		\end{enumerate}
		}
		\begin{enumerate}[label=(\alph*)]
			\item If we will add the first row of $A$ to its third row, we will get precisely $B$. Therefore, $E$ should be obtained
				from the identity matrix by adding its first row to its third row, hence
				\[E=\begin{mymat}1&0&0\\0&1&0\\1&0&1\end{mymat}\]
			\item If we subtract the third row of of $B$ from its second row, we will get precisely $C$. Hence,
				if we'll apply this operation (subtraction of the third row from the second one) to the identity 
				matrix, we will get $F$. Consequently
				\[F=\begin{mymat}1&0&0\\0&1&-1\\0&0&1\end{mymat}\]
			\item Yes, $C$ and $A$ are row equivalent. As $B=EA$, $A$ and $B$ are row equivalent -- they have the same
				row space.
				. Similarly, $B$ and $C$
				are row equivalent and have the same row space. Hence, $A$ and $C$ also have the same row space,
				equal to that of $B$. In other words, they are row equivalent.
		\end{enumerate}
		}
	\item[\# 7.]{{\it Given \[A=\begin{pmatrix}2&1\\6&4\end{pmatrix}\]
			\begin{enumerate}[label=(\alph*)]
				\item Express $A$ as a product of elementary matrices.
				\item Express $A^{-1}$ as a product of elementary matrices.
			\end{enumerate}
		}
		\begin{enumerate}[label=(\alph*)]
		\newcommand{\mymat}[1]{\left(\begin{array}{rr}#1\end{array}\right)}
			\item We will just bring $A$ to reduced row echelon form (which will inevitably be identity matrix) and
				encode each operation we do as an elementary matrix. The product of these will give us an answer we
				seek. So, let's start with making a pivot in the first row.
				\[\myexplain
				{\mymat{2&1\\6&4}}
				{\textcircled{1}$/2$}
				{\mymat{1&\myfrac{1}{2}\\6&4}}
				\]
				This operation corresponds to the elementary matrix \[E_1=\mymat{\myfrac{1}{2}&0\\0&1}\]
				Next we eliminate everything under the pivot
				\[\myexplain
				{\mymat{1&\myfrac{1}{2}\\6&4}}
				{\textcircled{2}$-6*$\textcircled{1}}
				{\mymat{1&\myfrac{1}{2}\\0&1}}
				\]
				This corresponds to elementary matrix \[E_2=\mymat{1&0\\-6&1}\]
				As second row is accidentally get pivoted, we eliminate everything above pivot of the second row
				\[\myexplain
				{\mymat{1&\myfrac{1}{2}\\0&1}}
				{\textcircled{2}$-\frac{1}{2}*$\textcircled{1}}
				{\mymat{1&0\\0&1}}
				\]
				This corresponds to elementary matrix \[E_3=\mymat{1&-\myfrac{1}{2}\\0&1}\]
				And hence we get
				\[E_3E_2E_1A=\mymat{1&0\\0&1}\]
				Hence $A=E_1^{-1}E_2^{-1}E_3^{-1}$. Fortunately, inverting elementary matrices is easy and we can
				directly write the answer as
				\[A=\mymat{2&0\\0&1}\mymat{1&0\\6&1}\mymat{1&\myfrac{1}{2}\\0&1}\]
			\item In the light of the obtained above equality 
				\[E_3E_2E_1A=\mymat{1&0\\0&1}\]
				Therefore, we have have $E_3E_2E_1=A^{-1}$ and hence
				\[A^{-1}=\mymat{\myfrac{1}{2}&0\\0&1}\mymat{1&0\\-6&1}\mymat{1&-\myfrac{1}{2}\\0&1}\]
		\end{enumerate}
		}
	\item[\# 10.]{{\it Find the inverse of each of the following matrices.
			}
			}
%Sec. 1.4 #4c, 5, 7, 10dg, 17, 18 20, 27
	\end{description}
\end{document}
