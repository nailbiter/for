\documentclass[8pt]{article} % use larger type; default would be 10pt

%\usepackage[utf8]{inputenc} % set input encoding (not needed with XeLaTeX)
\usepackage[10pt]{type1ec}          % use only 10pt fonts
\usepackage[T1]{fontenc}
%\usepackage{CJK}
\usepackage{graphicx}
\usepackage{float}
\usepackage{CJKutf8}
\usepackage{subfig}
\usepackage{amsmath}
\usepackage{amsfonts}
\usepackage{hyperref}
\usepackage{enumerate}
\usepackage{enumitem}

\newtheorem{prob}{Problem}

\newenvironment{solution}%
{\par\textbf{Solution}\space }%
{\par}

\title{Literature and Films\\Midterm Report\\The Nun's Story}
\author{歐立思\\9822058\\Department of Applied Mathematics\\National Chiao Tung University}

\begin{document}
\begin{CJK}{UTF8}{bsmi}
\maketitle
\end{CJK}
To begin with, it was not easy to come up with a film to review. Each of the five had its own attraction to me. To be honest, I love watching films. "The nun's story" was the oldest one, the least known to me and my natural curiosity
took upon. The final decision was taken when I've accidentally bumped into the trailer of "The nun's story". I decided that I've never seen anything similar, after all I do not watch films made before 1980s often, so why not to give
it a chance? Anyway, now after 150 minutes I can truly say that I watched it with pleasure. I hope, you will read what is written below with pleasure.\\
The film is based on a novel of 1956 written by Kathryn Hulme, who in turn based it on an experience of her friend, Marie Louise Habets who was a nurse and an ex-nun. It should be mentioned briefly, that this novel reached
Number One in New York Times best-seller list.
"The nun's story" begins with a young Belgium girl, Gabrielle van der Mal, who is deciding to be a nun, despite the moderate disagree of her father, famous surgeon,
who thinks it will be a difficult journey for such a stubborn person who she thinks Gabby is. Later on, the story follows the admission, first vows and service
of a young nun. We may see the deep inner struggle of this young, but very strong soul, who becomes known as Sister Luke upon entering the convent. Although the secret dream of a Sister Luke is going to the Congo in order to help 
operating natives in a remote hospital, it does not happen for a while. Newly-admitted Sister first has to struggle with vows of obedience and humility. One of the crucial moments is happening when she is making her exams
for nursing together with other nuns. During that time Sister Luke realizes that she cannot make friends with other nurse, Sister Polin, who despite being in Congo for 7 years has considerably lower qualification. 
Indeed, Sister Luke, being a famous surgeon's daughter, was a skilled nurse already before entering the convent. Upon coming to the
principal in order to make a confession, she is proposed to fail her tests in order to give Sister Polin chance not to feel embarrassed, thus demonstrating humility. We see the deep struggle in the heart of a young nun, but in the end
she still makes an exam perfectly. However, she is not allowed to go to Kong, but rather has to serve in a local Belgium hospital for demented. Despite very serious disappointment, she obeys still.
While being there she is at some moment almost killed by finding herself into the one cage with a dangerous lunatic, who claimed herself to be an archangel. After some time in the local hospital, Sister Luke shows herself as an excellent
nurse and is admitted to go to Congo. There the most of the film goes. Upon arrival in the hospital the young nun seems to be very curious and happy, just till the point that she learns that she will not work in the hospital for natives
with other sisters, but rather work in a "White Hospital" (hospital for Europeans) and assisting Dr. Fortunati, who runs the latter. From a brief conversation with an older nurse Sister Luke learns that Dr. Fortunati is a genius,
who works hard, exhausts his nurses as well and is a stubborn unbeliever. Despite tense relationships, which give a lot of trouble to Sister Luke, Dr. Fortunati soon realizes how brilliant and indispensable nurse she is. Life in Kongo
exhausts Gabrielle and soon she finds herself having tuberculosis, which means almost surely that she will have to go to the Europe, much to Dr. Fortunati and others chagrin. However, due to the nearly magical cure of a doctor's genius,
Sister Luke find herself recovered after 6 months. Life in Congo goes its own way, when the young nun learns that she will have to go to Belgium in order to accompany an important patient on a trip from Congo. After a conversation
with Dr. Fortunati, who still claims that Sister Luke will never be a nun, as she simply different, Gabrielle leaves. In Belgium she soon realizes that the was with Nazi started, Belgium surrenders and enemy (who Sister Luke cannot
help but perceives as an enemy) slowly enters convent and hospital life. 
Upon the outbreak of World War II, Sister Luke tries to honor the edicts of her order and not take sides, but this becomes impossible when her father (Dean Jagger) is killed by the Nazis. Realizing that she cannot remain true to her vows, Sister Luke leaves the order and returns to "civilian" life.
One day Gabrielle receives these news that serves as a turning point in her inner struggle with her vows of obedience - Nazi have killed her father while he was
operating on a people. The young nun cannot endure anymore, she asks for permission to live the convent, that is stop being a nurse. With great difficulties, the permission as granted and Gabrielle van der Mal leaves the cathedral
silently. "The Nun's Story" ends with a long, silent sequence in which Sister Luke divests herself of her religious robes, dons street garb, and walks out to an uncertain future. There is no background music: director Fred Zinnemann decided that "triumphant" music would indicate that Sister Luke's decision was the right one, while "tragic" music would suggest that she is doing wrong. Rather than make an editorial comment, the director decided against music, allowing the audience members to fill in the blanks themselves.\\
First of all, "The Nun's Story" will be interesting as it exposes the close, yet unknown to the most realm - the monastic life. What is more, the perfect play of Audrey Hepburn allows us to look into the inner world of a struggling
nurse. Most Hollywood movies featuring nuns sentimentalize them. Zinnemann's movie is a very serious portrayal of one who is a great nurse and a nun who has no difficulties with her vows of chastity and poverty, but great difficulties with humility (the sin of pride)
The audience seeing this thin realm of eternal struggle between what we perceive as our destiny and obedience - the struggle that I believe every talented individual has experienced not only once. A main objective in the monastic life is to polish the heart in humility and to transcend the ego, which is puffed up with pride. But for many serious servants of God, this may lead to an intense internal battle. The screenplay by Robert Anderson does a remarkable job avoiding religious stereotypes about nuns. Particularly heartening is that none of the women superiors who are in charge of spiritual training in the convent come across as a tyrant or an unfeeling authoritarian. On contrary, each of them is depicted as a
kind and understanding mother. Therefore, the conflict in its traditional, Hollywood understanding is absent from this film. Everything is much more interesting - the main conflict of the film goes inside the heart of the main 
character. In the convent, Gabrielle has a hard time with the practice of silence and adjusting to the fact that she must give up all her possessions and any semblance of a personal life. Another sister who entered the order at the same time, leaves after becoming convinced that she will not be able to follow the rigorous standards for the religious life. However, Sister Luke endures. \\
From the technical viewpoint, the film is also more than interesting. 
Fred Zinnemann (1907-1997) directed some of the iconic 1950s movies (From Here to Eternity, High Noon, Hatful of Rain, Oklahoma!), as well as such lauded and honored later movies as "A Man for All Seasons," "Julia," "The Sundowners," and "The Day of the Jackal." Having started with documentary movies, Zinnemann's movie often showed different settings and occupations in meticulous detail (the procedures in "The Day of the Jackal," in particular) and showed very determined (some would call them stubborn) protagonists (Prewitt in "Eternity," Thomas More in "Seasons," Julia, the Alpine climbers in "Five Days One Summer," the assassin and his hunter in "Jackal"). 
Zinnemann won two Oscars and was nominated for six others. He also won four "best director" New York Film Critics Awards, and directed 19 different actors in Oscar-nominated performance (7 of whom won the award). Although one could not say he was obscure or unhonored during his long life, he tends to be forgotten by those canonizing "auteurs." Perhaps, he was too tasteful and too unobtrusive at his craft to get his due. Or, perhaps, his movies were too successful at the box office (though most had modest budgets). 
The film gives the audience quite a few beautiful pictures of an internal life of a monastic community, the knowledge which in fact is not so easy to get for an 
uninitiated. Throughout the film, the devotional practices of the sisters is presented with utter respect. The viewer gets a real feel for what cloistered life was like in the 1930s and 1940s in Europe. Depicted on this not very
shiny or pictorial background the whole complexity of Sister Luke's multidimensional inner struggle is seen with an utmost clearance. It is worthwhile to emphasize once more these great shots of a nun's life that are exposed to us
by a creator. Despite this, "The Nun’s Story" certainly does not offer the positive depiction of religious life common in 1950s Hollywood, but it’s not an anti-religious or anti-Catholic depiction either.
There’s no effort to depict all nuns as warped or frustrated; there are certainly bad apples, but also warm, sympathetic, apparently well-adjusted human beings. 
One of the more insidious moments involves a twisted mother superior suggesting that Sr. Luke (Gabby’s name in religion) display her humility by deliberately failing a nursing exam — but another superior later
confirms that this advice was wrong-headed.\\
The film is mostly taken in dark and moderate colors in the first part, which changes to bright in vivid during the middle (in Congo) and goes back to a moderate during the last part. For more than two hours, covering about a decade, she struggles mightily to be a good nun. The movie is in no rush as it shows the ceremonies and the daily grind of her career. Because of the intensity of Audrey Hepburn's performance (widely regarded as her greatest one), this unfolding is not boring. Peter Finch's surgeon who sees much that Sister Luce has suppressed from consciousness injects the equivalent of a shot of adrenaline of the way through. Hepburn showed she could be mesmerizing without Givenchy couture (and with no visible coiffure). Yes, this perhaps is the most modes roles of a fair Audrey Hepburn, indeed, during the most of the film we hardly ever see even her hair. Despite this, she and her play are just great. Again, 
I can truly say that I watched this film with pleasure. I hope, what is written above seems reasonable enough.\\
\end{document}
