\documentclass[12pt]{article} % use larger type; default would be 10pt

\usepackage{mathtext}                 % підключення кирилиці у математичних формулах
                                          % (mathtext.sty входить в пакет t2).
\usepackage[T1,T2A]{fontenc}         % внутрішнє кодування шрифтів (може бути декілька);
                                          % вказане останнім діє по замовчуванню;
                                          % кириличне має співпадати з заданим в ukrhyph.tex.
\usepackage[utf8]{inputenc}       % кодування документа; замість cp866nav
                                          % може бути cp1251, koi8-u, macukr, iso88595, utf8.
\usepackage[english,russian,ukrainian]{babel} % національна локалізація; може бути декілька
                                          % мов; остання з переліку діє по замовчуванню. 
\usepackage{amsthm}
\usepackage{amsmath}
\usepackage{amsfonts}
\usepackage{graphicx}
\usepackage[pdftex]{hyperref}
\usepackage{caption}
\usepackage{subfig}
\usepackage{fancyhdr}
\usepackage{cancel}

\newtheorem{prob}{Завдання}
\newcommand{\ds}{\;ds}
\newcommand{\dt}{\;dt}
\newcommand{\dx}{\;dx}
\newcommand{\dta}{\;d\tau}
\let\oldint\int
\renewcommand{\int}{\oldint\limits}
\let\phi\varphi

\usepackage{mystyle}

\newtheorem{myulem}[mythm]{Лема}

\renewenvironment{myproof}[1][Доведення]{\begin{trivlist}
\item[\hskip \labelsep {\bfseries #1}]}{\myqed\end{trivlist}}

\title{Контрольна робота з функціонального аналізу (9 семестр)\\Вар. 1}
\author{Олексій Леонтьєв}

\begin{document}
\maketitle
\begin{prob}Знайти оператор, спряжений до $A:l_2\mapsto l_2$\[Ax=(\frac{1}{2}x_2,x_3,x_4,\dots)\]\end{prob}
	Нагадаємо, що нам потрібно знайти (єдиний, за теоремою Ріса) оператор, що для всіх $x,y\in l_2$ задовольняв би рівності
	\[\mysca{Ax}{y}=\mysca{x}{A^*y}\]
	Ми стверджуємо, що $A^*:l_2\mapsto l_2$ заданий як
	\[A^*y=(0,\frac{1}{2}y_1,y_2,\dots)\]
	задовольняє цій умові. Це легко перевірити, адже $A^*$ є неперервним оператором на $l_2$ і до того ж ми маємо (всі суми нижче збіжні)
	\[\mysca{Ax}{y}=
	\frac{1}{2}x_2\overline{y_1}+x_3\overline{y_2}+x_4\overline{y_3}+\dots=
	x_1\cdot 0+x_2\cdot\overline{\frac{1}{2}y_1}+x_3\overline{y_2}+x_4\overline{y_3}+\dots=\]
	\[\mysca{(x_1,x_2,x_3,\dots)}{(0,\frac{1}{2}y_1,y_2,y_3,\dots)}=\mysca{x}{A^*y}\]
\begin{prob}Довести, що оператор $A:B\mapsto B$ є скінченновимірним, де $B=C[0,\pi]$, $(Ax)(t)=\int_0^{\pi}\sin(t+\tau)x(\tau)d\tau,\;t\in[0,\pi]$.
	Чи буде $A$ компактним оператором?\end{prob}
	Помітимо, що
	\[\forall x\in B,\;(Ax)(t)=\int_0^{\pi}\sin(t+\tau)x(\tau)d\tau=\int_0^{\pi}\left(\sin t\cos\tau+\cos t\sin\tau\right)x(\tau)d\tau=\]
	\[\sin t\int_0^{\pi}\cos\tau x(\tau)d\tau+\cos t\int_0^{\pi}\sin\tau x(\tau)d\tau\in \left<\left\{\sin t,\cos t\right\}\right>\]
	Оскільки $\left<\left\{\sin t,\cos t\right\}\right>:=\mysetn{\alpha\sin t+\beta\cos t}{\alpha,\beta\in\mathbb{C}}$ є скінченновимірним
	підпростором $B$ (розмірності 2, адже $\sin t$ і $\cos t$ лінійно незалежні в $B$)
	, скінченновимірним є і множина значень $A$, таким чином останній є скінченновимірним.\\
	Більше того, $A$ є компактним оператором. Дійсно, якщо $X\subset B$ - обмежена множина, $A(X)\subset
	\left<\left\{\sin t,\cos t\right\}\right>$ також буде обмеженою в $\left<\left\{\sin t,\cos t\right\}\right>\simeq \mathbb{C}^2$
	, через обмеженість оператора $A$, а отже замикання $A(X)$ в $B$ буде міститися в $\mathbb{C}^2\simeq
	\left<\left\{\sin t,\cos t\right\}\right>$
	(адже останній є скінченновимірним підпростором $B$ і тому замкненою підмножиною $B$) і буде замкненою обмеженою множиною. Проте, 
	кожна замкнена обмежена множина в $\mathbb{C}^2\simeq
	\left<\left\{\sin t,\cos t\right\}\right>$ є компактною, а отже $A(X)$ буде прекомпактною в $B$ і тому $A$ - компактний оператор.
\begin{prob}Знайти спектр, власні числа, норму і спектральний радіус оператора $A:l_2\mapsto l_2$, $Ax=(x_1,x_1-x_2,x_3,x_4,\dots)$\end{prob}
	Нехай $S\subset\mathbb{C}$ позначає спектр $A$. Покажемо, що $S\subseteq\{1,-1\}$. Дійсно, зафіксуємо довільне $\lambda\in\mathbb{C}\setminus
	\{1,-1\}$ і покажемо, що $\lambda I-A$ є неперервно оборотнім - з цього випливатиме бажане $\lambda\notin S$. За критерієм неперервної
	оборотності достатньо показати, що для певного $m>0$ виконується $\forall x\in l_2,\;\mynorm{(\lambda I-A)x}\geq m\mynorm{x}$. Через лінійність
	оператора і однорідність норми достатньо показати, що $\forall x\in\mysetn{x\in l_2}{\mynorm{x}=1},\;\mynorm{(\lambda I-A)x}>m$. Помітимо, що
	якщо $\mynorm{x}=1$ виконується
	\begin{gather*}
	\mynorm{(\lambda I-A)x}^2=\myabs{\lambda-1}^2(\myabs{x_1}^2+\myabs{x_3}^2+\myabs{x_4}^2+\dots)+\myabs{\lambda x_2+x_2-x_1}^2=\\
	=\myabs{\lambda-1}^2(1-\myabs{x_2}^2)+\myabs{\lambda x_2+x_2-x_1}^2
	\end{gather*}
	і таким чином
	\[\inf_{\mynorm{x}=1}\mynorm{(\lambda I-A)x}^2=\inf_{\substack{x_1,x_2\in\mathbb{C}\\\myabs{x_1}^2+\myabs{x_2}^2\leq 1}}
	\myabs{\lambda-1}^2(1-\myabs{x_2}^2)+\myabs{\lambda x_2+x_2-x_1}^2\]
	Більше того, оскільки $\mysetn{x_1,x_2\in\mathbb{C}}{\myabs{x_1}^2+\myabs{x_2}^2\leq 1}$ є компактною підмножиною $\mathbb{C}^2$, а
	$\myabs{\lambda-1}^2(1-\myabs{x_2}^2)+\myabs{\lambda x_2+x_2-x_1}^2$ - неперервна функція, точна нижня межа досягається. Таким чином,
	якщо ми підемо від супротивного і припустимо, що $\inf_{\mynorm{x}=1}\mynorm{(\lambda I-A)x}^2=0$, це буде означати, що для $a,b\in\mathbb{C}
	,\;\myabs{a}^2+\myabs{b}^2\leq 1$ виконується $\myabs{\lambda-1}^2(1-\myabs{b}^2)+\myabs{\lambda b+b-a}^2=0$. Таким чином
	$\myabs{\lambda-1}^2(1-\myabs{b}^2)=0$ і оскільки $\lambda\neq 1$ має виконуватися $\myabs{b}=1$, а отже $\myabs{a}=0$. Проте в цьому 
	разі	$\myabs{\lambda b+b-a}=\myabs{\lambda+1}>0$, оскільки $\lambda\neq -1$, що і дає протиріччя та доводить бажане $S\subseteq\{1,-1\}$.

	Продовжуючи, $\lambda=1$ та $\lambda=-1$ є власними значеннями $A$ ($x=(1,0,0,\dots)$ та $x=(0,1,0,0,\dots)$ - власні вектори відповідно).
	Оскільки власні вектори належать спектру, з цього випливає, що $S\supseteq\{1,-1\}$ і тому $S=\{1,-1\}$, а отже $\lambda=\pm 1$ - єдині
	власні числа $A$. Спектральний радіус, відповідно, $\rho=1$.

	Покажемо наостанок, що $\mynorm{A}=\frac{\sqrt{5}+3}{2}$. Дійсно, з одного боку, якщо $x\in l_2$ і $\mynorm{x}=1$ маємо
	\[\mynorm{Ax}^2=\myabs{x_1}^2+\myabs{x_1-x_2}^2+\myabs{x_3}^2+\myabs{x_4}^2+\dots\leq\]
	\[\leq \myabs{x_1}^2+(\myabs{x_1}+\myabs{x_2})^2+\myabs{x_3}^2+\myabs{x_4}^2+\dots=1+2\myabs{x_1}\myabs{x_2}+\myabs{x_1}^2\]
	За нерівністю між середнім геометричним і середнім арифметичним маємо
	\[\myabs{x_1}\myabs{x_2}\leq\frac{\left({\frac{\sqrt{5}-1}{2}}\right)\myabs{x_1}^2+\left(\frac{\sqrt{5}-1}{2}\right)^{-1}\myabs{x_2}^2}{2}\]
	відповідно,
	\[1+2\myabs{x_1}\myabs{x_2}+\myabs{x_1}^2
	\leq1+\left({\frac{\sqrt{5}-1}{2}}\right)\myabs{x_1}^2+\left(\frac{\sqrt{5}-1}{2}\right)^{-1}\myabs{x_2}^2+\myabs{x_1}^2=\]
	\[=1+\frac{\sqrt{5}+1}{2}\myabs{x_1}^2+\frac{\sqrt{5}+1}{2}\myabs{x_2}^2\leq 1+\frac{\sqrt{5}+1}{2}=\frac{\sqrt{5}+3}{2}\]
	З іншого боку, для $l_2\ni x_0:=(\sqrt{\frac{\sqrt{5}+1}{2\sqrt{5}}},-\sqrt{\frac{\sqrt{5}-1}{2\sqrt{5}}},0,0,\dots)$ виконується
	по-перше
	\[\mynorm{x}^2=\frac{\sqrt{5}+1}{2\sqrt{5}}+\frac{\sqrt{5}-1}{2\sqrt{5}}=\frac{2\sqrt{5}}{2\sqrt{5}}=1\]
	і по-друге
	\[\mynorm{Ax}=\frac{\sqrt{5}+1}{2\sqrt{5}}+\left(\sqrt{\frac{\sqrt{5}+1}{2\sqrt{5}}}+\sqrt{\frac{\sqrt{5}-1}{2\sqrt{5}}}\right)^2=\]\[=
	\frac{\sqrt{5}+1}{2\sqrt{5}}+\frac{\sqrt{5}+1}{2\sqrt{5}}+{\frac{\sqrt{5}-1}{2\sqrt{5}}}+2\sqrt{
	\frac{\sqrt{5}+1}{2\sqrt{5}}\cdot\frac{\sqrt{5}-1}{2\sqrt{5}}}=\]
	\[=\frac{\sqrt{5}+1}{2\sqrt{5}}+1+2\frac{2}{2\sqrt{5}}=\frac{\sqrt{5}+5}{2\sqrt{5}}+1=\frac{1+\sqrt{5}}{2}+1=\frac{\sqrt{5}+3}{2}\]
	що і завершує доведення бажаної рівності $\mynorm{A}=\frac{\sqrt{5}+3}{2}$
\begin{prob}Знайти спектр, власні числа і власні функції оператора $A\in L(H)$, $H=L_2[0,2\pi]$, $(Ax)(t)=\int_0^{2\pi}\cos^2(t-\tau)x(\tau)
	d\tau,\;t\in[0,2\pi]$
\end{prob}
Для початку, помітимо, що
\newcommand{\myint}[1]{\int_0^{2\pi}#1x(\tau)\;d\tau}
\newcommand{\mysint}[1]{\int_0^{2\pi}#1\;d\tau}
\[(Ax)(t)=\int_0^{2\pi}\cos^2(t-\tau)x(\tau)d\tau=\int_0^{2\pi}(\cos\tau\cos t+\sin\tau\sin t)^2x(\tau)d\tau=\]
\[=\cos^2 t\myint{\cos^2(\tau)}+\sin^2 t\myint{\sin^2(\tau)}+\sin 2t\myint{\frac{\sin (2\tau)}{2}}\]
Таким чином, $A$ є скінченновимірним оператором, а отже, згідно із прикладом 1.2 в главі IX \cite{tb}, 
$A$ є компактним оператором. Таким чином, згідно з теоремою 4.1 з глави IX у \cite{tb} всі ненульові елементи спектру $A$ будуть його власними
значеннями, а $0$ належить спектру як наслідок Зауваження 1.5 глави IX \cite{tb}.

Також, нуль є власним числом $A$, адже як буде показано нижче $\cos^2t$, $\sin^2t$ та $\sin2t$ є лінійно незалежними, а отже $Ax=0\iff
\myint{\cos^2(\tau)}=\myint{\sin^2(\tau)}=\myint{\sin(2\tau)}=0$. Таким чином,
\[\emptyset\neq
\mbox{Ker} A=\mysetn{x\in L_2[0,2\pi]}{Ax=0}=\]\[=\mysetn{x\in L_2[0,2\pi]}{\myint{\cos^2(\tau)}=\myint{\sin^2(\tau)}=\myint{\sin(2\tau)}=0}=\]
\[\mysetn{\alpha\sin^2(\tau)+\beta\cos^2(\tau)+\gamma\sin(2\tau)
}{\alpha,\beta,\gamma\in\mathbb{C}}^\perp=:\left<\left\{\sin^2(\tau),\cos^2(\tau),\sin(2\tau)\right\}\right>^\perp=\]
\[=\left<\left\{1,\cos(2\tau),\sin(2\tau)\right\}\right>^\perp=\myabra{\mycbra{\cos(n\tau),\sin(n\tau)}_{n>0,\;n\neq2}}\]
адже $\mycbra{\cos(n\tau),\sin(n\tau)}_{n>0}\cup\mycbra{1}$ утворюють ортонормований базис для $H$.

Припустимо, що $0\neq\lambda\in\mathbb{C}$ належить спектру $A$, отже для $0\neq x\in H$ виконується $Ax=\lambda x$, тому $\lambda x\in\Im A
\implies x\in\Im A\implies x(t)=a\cos^2t+b\sin^2t+c\sin2t$. Підстановка дає
\begin{equation}\label{Eigenvectors}\begin{gathered}
A(a\cos^2t+b\sin^2t+c\sin2t)(t)=\\
=\cos^2 t\myint{\cos^2(\tau)}+\sin^2 t\myint{\sin^2(\tau)}+\sin 2t\myint{\frac{\sin (2\tau)}{2}}=\\
=\lambda a\cos^2t+\lambda b\sin^2t+\lambda c\sin2t
\end{gathered}\end{equation}
Більше того, позначаючи $e_1:=\cos^2t\in H$, $e_2:=\sin^2t\in H$, $e_3:=\sin2t\in H$ і $H\times H\ni(x,y)\mapsto \mysca{x}{y}:=\mysint{x(\tau)
\overline{y(\tau)}}\in\mathbb{C}$ -- скалярний добуток на $H$, можемо записати
\[\mysca{e_1}{e_1}=\mysint{\cos^4\tau}=\mysint{\cos^2(\tau)\sin^2(\tau)}=\mysint{\mybra{\frac{\cos2t+1}{2}}^2}=\]
\[=\frac{1}{4}\mysint{\cos^2(2\tau)}+\frac{1}{2}\underbrace{\mysint{\cos2\tau}}_{=0}+\frac{\pi}{2}=\]
\[=\frac{1}{4}\mysint{{\frac{\cos4\tau+1}{2}}}+\frac{\pi}{2}=\frac{1}{8}\underbrace{\mysint{\cos4\tau}}_{=0}+\frac{\pi}{4}+\frac{\pi}{2}=\frac{3\pi}
{4}\]
\[\mysca{e_1}{e_2}=\mysint{(1-\cos^2(\tau))\cos^2(\tau)}=\mysint{\cos^2(\tau)}-\underbrace{\mysint{\cos^4(\tau)}}_{=3\pi/4}=\]
\[=\mysint{\frac{\cos2\tau+1}{2}}-\frac{3\pi}{4}=\frac{1}{2}\mysint{\cos2\tau}+\pi-\frac{3\pi}{4}=\frac{\pi}{4}\]
\[\mysca{e_1}{e_3}=\mysint{\cos^2(\tau)\sin2\tau}=0,\mbox{ бо підінтегральний вираз має властивість $f(x+\pi)=-f(x)$}\]
\[\mysca{e_2}{e_2}=\mysint{\sin^4(\tau)}=\mysint{(1-\cos^2\tau)^2}=2\pi-2\underbrace{\mysint{\cos^2\tau}}_{=\pi}
+\underbrace{\mysint{\cos^4(\tau)}}_{=3\pi/4}=\frac{3\pi}{4}\]
\[\mysca{e_2}{e_3}=\mysint{\sin^2(\tau)\sin2\tau}=0,\mbox{ бо підінтегральний вираз має властивість $f(x+\pi)=-f(x)$}\]
\[\mysca{e_3}{e_3}=\mysint{\sin^22\tau}=\mysint{\frac{1-\cos4\tau}{2}}=\pi-\frac{1}{2}\underbrace{\mysint{\cos4\tau}}_{=0}=\pi\]
З цих розрахунків можна зробити декілька висновків. По-перше, $e_1$, $e_2$ та $e_3$ є лінійно незалежними над $\mathbb{C}$. Дійсно,
рівність $\alpha e_1+\beta e_2+\gamma e_3$ (де грецькі літери позначають комплексні константи) можна скалярно домножити на $e_1$, $e_2$ та
$e_3$ справа, щоб отримати три рівності, які в матричній формі можна записати як
\[\begin{bmatrix}
	\mysca{e_1}{e_1}&\mysca{e_2}{e_1}&\mysca{e_3}{e_1}\\
	\mysca{e_1}{e_2}&\mysca{e_2}{e_2}&\mysca{e_3}{e_2}\\
	\mysca{e_1}{e_3}&\mysca{e_2}{e_3}&\mysca{e_3}{e_3}\\
\end{bmatrix}\begin{bmatrix}\alpha\\\beta\\\gamma\end{bmatrix}=\begin{bmatrix}0\\0\\0\end{bmatrix}\]
	а оскільки
\[
\begin{bmatrix}
	\mysca{e_1}{e_1}&\mysca{e_2}{e_1}&\mysca{e_3}{e_1}\\
	\mysca{e_1}{e_2}&\mysca{e_2}{e_2}&\mysca{e_3}{e_2}\\
	\mysca{e_1}{e_3}&\mysca{e_2}{e_3}&\mysca{e_3}{e_3}\\
\end{bmatrix}=
\begin{bmatrix}
	3\pi/4&\pi/4&0\\\pi/4&3\pi/4&0\\0&0&\pi
\end{bmatrix}
\]
не є сингулярною матрицею, з рівності вище випливає $\alpha=\beta=\gamma=0$, а з цього, в свою чергу, лінійна незалежність $e_1$, $e_2$ та $e_3$
над $\mathbb{C}$. Відповідно, рівність \ref{Eigenvectors} вище дає нам як наслідок
\[\begin{cases}\myint{\cos^2(\tau)}=\lambda a\\
\myint{\sin^2(\tau)}=\lambda b\\
\myint{\frac{\sin2\tau}{2}}=\lambda c\end{cases}\]
Використовуючи наші знання скалярних добутків векторів $e_i$, це переписується як
\[\begin{bmatrix}
	3\pi/4&\pi/4&0\\\pi/4&3\pi/4&0\\0&0&\pi
\end{bmatrix}\begin{bmatrix}a\\b\\c\end{bmatrix}=\lambda\begin{bmatrix}a\\b\\c\end{bmatrix}\]
Розраховуючи власні значення і відповідні власні вектори для матриці $3\times3$, ми маємо, що власними значеннями є: 
$\pi$ (з відповідними незалежними векторами $(0,0,1)$ та $(1,1,0)$) та $\pi/2$ (з вектором $(1,-1,0)$).

У підсумку маємо наступне. Спектром $A$ є $\mycbra{0,\pi,\pi/2}$, що співпадає з множиною власних чисел, а відповідними підпросторами власних
векторів є $\myabra{\mycbra{\cos(n\tau),\sin(n\tau)}_{n>0,\;n\neq2}}$, $\myabra{\cos^2t+\sin^2t,\sin(2t)}$ та $\myabra{\cos^2t-\sin^2t}$.
\begin{prob}Чи можуть наступні множини бути спектром деякого компактного оператора в $l_2$?\end{prob}
	У випадку, коли нам треба буде довести \textit{негативний} результат (тобто, показати, що певна множина \textit{не є}
	спектром жодного компактного оператора, ми будемо використовувати \textbf{Теорему 4.1} з глави IX
	\cite{tb}, яка перелічує деякі властивості, які
	притаманні спектру компактного оператора. Таким чином, якщо дана множина не буде задовольняти якийсь з цих властивостей, вона автоматично
	не може бути спектром компактного оператора.

	Якщо ж ми забажаємо довести, що певна множина є спектром декого компактного оператора на $l_2$, ми просто сконструюємо такий оператор. У 
	нагоді буде наступна лема.
	\begin{myulem}\label{CompactSpectrumLemma}
		Нехай $\mycbra{a_n}_{n=1}^{\infty}$ послідовність комплексних чисел, що збігається до нуля і така, що хоча б один її член
		є нулем.
		Тоді множина значень $\mycbra{a_n}_{n=1}^{\infty}$ є спектром компактного оператора
	$A:l_2\ni(x_1,x_2,\hdots)\mapsto(a_1x_1,a_2x_2,\hdots)\in l_2$.\end{myulem}
	\begin{myproof} Нам потрібно довести три твердження зростаючої складності: по-перше, треба показати, що $A$ означене як у твердженні
		дійсно буде оператором на $l_2$, 
		по-друге, що він буде компактним
		і по-третє, що його спектром дійсно буде $\mycbra{a_n}_{n=1}^{\infty}$.
		$A$ дійсно оператор, адже для $(x_1,x_2,\hdots)\in l_2$ маємо $\mynorm{A(x_1,x_2,\hdots)}=\sqrt{\sum_{i=1}^{\infty}
		\myabs{a_i}^2\myabs{x_i}^2}\leq\sqrt{\sum_{i=1}^{\infty}B^2\myabs{x_i}^2}=B\mynorm{x}<\infty$ (
		існування $B\in\mathbb{R}$ такого, що $\forall i\geq 1 \myabs{x_i}<B$, тобто обмеженість послідовності
		$\mycbra{a_n}_{n=1}^{\infty}$ випливає з її збіжності), адже $(x_1,x_2,\hdots)\in l_2\implies
		\mynorm{x}<\infty$ і отже $(a_1x_1,a_2x_2,\hdots)=A(x_1,x_2,\hdots)\in l_2$. Лінійність випливає з означення $A$.
		
		Для того, щоб показати компактність оператора $A$, розглянемо послідовність операторів $A_n(x_1,x_2,\dots)=(a_1x_1,a_2x_2,\dots,
		a_nx_n,0,0,0,\dots)$. Кожен із них є компактним, адже множина значень кожного є скінченновимірним підпростором $l_2$ (див. приклад
		1.2 з глави IX \cite{tb}). Більше того, для оскільки $a_n\to0$, для кожного довільно малого $\epsilon>0$ існує таке $N$,
		що $\forall i>N,\;\myabs{a_i}<\epsilon$ і тому $\forall i>N,\;\forall x\in l_2,\;\mynorm{x}\leq1$ маємо $\mynorm{(A-A_i)(x)}=
		\sqrt{\sum{j=i+1}^{\infty}\myabs{a_j\cdot x_j}}<\sqrt{\sum_{j=i+1}^{\infty}\epsilon^2\myabs{x_j}^2}\leq
		\sqrt{\sum_{j=1}^{\infty}\epsilon^2\myabs{x_j}^2}=\epsilon\mynorm{x}\leq\epsilon$. Таким чином, $\forall i>N,\;\mynorm{A-A_i}<
		\epsilon$ і отже $A_n\to A$ за нормою в $l_2$, тому $A$ є компактним як послідовність компактних операторів.

		Далі, кожне з комплексних чисел $a_1,a_2,\hdots$ безумовно належить спектру оператора $A$, адже кожне є його власним значенням,
		що випливає прямо з означення (зокрема, власним вектором для $a_i$ буде $e_i$). Залишається показати, що $\forall i,\;a_i\neq a\implies a\notin\sigma(A)$.
		Для цього, припустимо, що $a$ не співпадає з жодним з $a_i$, але $a\in\sigma(A)$.
		Тоді зокрема $a\neq 0$ (оскільки за гіпотезою $\exists i\geq 1,\mid a_i=0$) і тому за вищезгаданою Теоремою 4.1 з
		глави IX \cite{tb}, $a\neq0$ має бути власним значенням $A$, а отже $\exists l_2\ni v\neq0\mid Av=av\implies\forall i\geq 1
		(Av)_i=a_iv_i=av_i$. Оскільки $\forall i\geq 1,\;a_i\neq a$, маємо $v_i=0$ і тому $v=0$, і отримане протиріччя завершує доведення.
	\end{myproof}
\begin{enumerate}
	\renewcommand{\labelenumi}{\myralph{enumi})}
\item{$[-1,1]$ {\it не може бути спектром жодного компактного оператора}
	, адже за Теоремою 4.1 спектр компактного оператора може бути не більше ніж зліченним.}
	\item{Так, множина $\mycbra{-1,0,1}$ {\it є спектром компактного оператора}. Щоб побачити це, ми застосуємо доведену вище лему
		\ref{CompactSpectrumLemma} до
		послідовності $\mycbra{-1,1,0,0,\hdots}$, що дасть нам оператор $A(x_1,x_2,x_3,\hdots)=(-x_1,x_2,0,0,\hdots)$. За лемою, $A$
		дійсно компактний оператора на $l_2$ зі спектром $\mycbra{-1,0,1}$.}
	\item{За Теоремою 4.1, спектр компактного оператора має містити 0. Оскільки $0\notin\mycbra{\frac{1}{\sqrt{n}},\;n\geq 1}$, ця множина
		{\it не може бути спектром компактного оператора.}}
	\item{Знову ж таки, за лемою \ref{CompactSpectrumLemma} вище, множина значень послідовності
		$\mycbra{0,\frac{1}{\sqrt{1}},\frac{1}{\sqrt{2}},\frac{1}{\sqrt{3}},\hdots}$ {\it є спектром компактного оператора.}}
	\item{За Теоремою 4.1, якщо спектр компактного оператора нескінченний, єдиною граничною точкою його є нуль. Проте $1\neq 0$ являється
		граничною точкою множини $\mycbra{1-\sqrt{1}{\sqrt{n}},\;n\geq 1}$ і тому ця множина {\it 
		не може бути спектром компактного оператора.}}
\end{enumerate}
\begin{prob}За допомогою повторних ядер побудувати резольвенту інтегрального рівняння $x(t)=\lambda
	\displaystyle\int_{-1}^{1}e^{\tau-t}x(\tau)\;d\tau+
	y(t),\;t\in[-1;1]$, і знайти його розв’язок при $\lambda=1+e,\;y(t)=\sin\pi t$.\end{prob}
	Ми скористуємося методом повторних ядер, як викладено в \S IX.6 підручника
	\cite{tb}. Для цього ми введемо $K(t,\tau):=e^{\tau-t}\in
	C([-1;1]\times[-1;1])$ (тому метод можна застосувати, адже $[-1;1]\subset\mathbb{R}$
	компактна множина). Ми позначимо $K^{(1)}(t,
	\tau):=K(t,\tau)$ і рекурентно введемо
	\[K^{(n+1)}(t,\tau):=\int_{-1}^1 K(t,s)K^{(n)}(s,\tau)\ds\]
	За допомогою математичної індукції бачимо, що $K^{(n)}(t,\tau)=2^{n-1}e^{\tau-t}$. Дійсно, рівність виконується для
	$n=1$, а далі маємо
	\[K^{(n+1)}(t,\tau):=\int
	_{-1}^1 K(t,s)K^{(n)}(s,\tau)\ds=\int_{-1}^12^{n-1}e^{s-t}e^{\tau-s}\ds=2^{n-1}e^{\tau-t}\int_{-1}^1\ds=2^ne^{\tau-t}\]
	Далі, згідно з \cite{tb}, резольвента являється інтегральним оператором із ядром
	\[\mathcal{R}(t,\tau;\lambda)=\sum_{n=1}^\infty \lambda^{n-1}K^{(n)}(t,\tau)=\sum_{n=1}^\infty e^{\tau-t}(2\lambda)^{n-1}
	=e^{\tau-t}\frac{1}{1-2\lambda}\]

	Таким чином, для даних $\lambda$ і $y(t)$ розв’язок записується як
	\[x(t)=y(t)+\lambda\int_{-1}^1\mathcal{R}(t,\tau;\lambda)y(\tau)\;d\tau=\sin\pi t+(1+e)\int_{-1}^1e^{\tau-t}\frac{1}{(-1-2e)}
	\sin\pi\tau\;d\tau=\]
	\[=\sin\pi t-\frac{(e+1)e^{-t}}{1+2e}\int_{-1}^1e^\tau\sin\pi\tau\;d\tau=\sin\pi t-e^{-t}\frac{e+1}{1+2e}\cdot
	\frac{(e^2-1)\pi}{e(1+\pi^2)}\]
\begin{prob}Звести до системи алгебраїчних рівнянь і розв’язати інтегральне рівняння
	\[x(t)=\lambda\int_{-1}^1(t^2-t\tau)x(\tau)\;d\tau+y(t),\;t\in[-1;1],\;\mbox{при }y(t)=t^2+t,\;\lambda\in\mathbb{C}\]
\end{prob}
Як показано в \S 3.3 глави IX підручника \cite{tb}, для інтегрального рівняння Фредгольма другого роду $\int_RK(t,\tau)x(\tau)\;d\tau
-x(t)=-y(t)$ з виродженим ядром $K(x,t)=\sum_{j=1}^n a_j(t)b_j(\tau)$ і для чисел
\[a_{jk}:=\int_Rпa_k(t)b_j(t)\dt,\;y_j:=-\int_Ry(t)b_j(t)\dt\]
маємо, що $\mybra{x_j}_{j=1}^n\in\mathbb{C}^n$ є розв’язком лінійної системи
$\sum_{j=1}^na_{ij}x_j-x_i=y_i$ тоді і лише тоді, коли $x(t):=\sum_{k=1}^nx_ka_k(t)+y(t)$ є розв’язком $\int_RK(t,\tau)x(\tau)\;d\tau
-x(t)=-y(t)$ і більше того,
розв’язки відповідного інтегрального рівняння можуть набувати {\it лише} виду $x(t)=\sum_{k=1}^nx_ka_k(t)+y(t)$.

У нашому випадку маємо $a_1(t)=\lambda t^2,\;a_2(t)=-\lambda t$ (помітімо, що оскільки при $\lambda=0$
$a_1$ та $a_2$ не є лінійно незалежними, ми проводимо всі розрахунки нижче за неявної додаткової умови
$\lambda\neq0$, проте отриманий висновок в кінці є вірним очевидно і для $\lambda=0$, адже у цьому
випадку інтегральне рівняння вироджується просто в $x(t)=y(t)$, що звісно ж 
має єдиний розв’язок), а також $b_1(\tau)=1,\;b_2(\tau)=\tau$ і відповідно
\[\mysbra{a_{jk}}_{j,k=1}^2=\lambda\begin{bmatrix}
	\int_{-1}^1t^2\dt&\int_{-1}^1(-t)\dt\\
	\int_{-1}^1t^3\dt&\int_{-1}^1(-t^2)\dt
\end{bmatrix}=\lambda\begin{bmatrix}\myfrac{2}{3}&0\\0&-\myfrac{2}{3}\end{bmatrix}\]
	\[\begin{bmatrix}y_1\\y_2\end{bmatrix}=
		\begin{bmatrix}-\int_{-1}^1(t^2+t)\dt\\-\int_{-1}^1(t^2+t)t\dt\end{bmatrix}=
			\begin{bmatrix}-\myfrac{2}{3}\\-\myfrac{2}{3}
		\end{bmatrix}
		\]
Таким чином, $\sum_{j=1}^na_{ij}x_j-x_i=y_i$ перетворюється в лінійну систему
\[\begin{bmatrix}\frac{2}{3}\lambda-1&0\\0&-\frac{2}{3}\lambda-1\end{bmatrix}\begin{bmatrix}x_1\\x_2\end{bmatrix}=
	\begin{bmatrix}-\myfrac{2}{3}\\-\myfrac{2}{3}\end{bmatrix}\]
		Ми бачимо, що система має єдиний розв’язок при $\lambda\notin\mycbra{\pm\myfrac{2}{3}}$ і відповідно єдиний розв’язок матиме
		рівняння  $x(t)=\lambda\int_{-1}^1(t^2-t\tau)x(\tau)\;d\tau+y(t)$.

		У випадку ж $\lambda\in\mycbra{\pm\myfrac{2}{3}}$ лінійна система
		\[\begin{bmatrix}\frac{2}{3}\lambda-1&0\\0&-\frac{2}{3}\lambda-1\end{bmatrix}\begin{bmatrix}x_1\\x_2\end{bmatrix}=
	\begin{bmatrix}-\myfrac{2}{3}\\-\myfrac{2}{3}\end{bmatrix}\]
		не має розв’язку, а отже не має їх і відповідне інтегральне рівняння.

		Підводячи підсумок, інтегральне рівняння
		\[x(t)=\lambda\int_{-1}^1(t^2-t\tau)x(\tau)\;d\tau+t^2+t\]
		має єдиний розв’язок $x(t):=\frac{-\myfrac{2}{3}}{\myfrac{2}{3}\lambda-1}\lambda t^2-\frac{-\myfrac{2}{3}}{-\myfrac{2}{3}\lambda-1}
		\lambda t+t^2+t$ при $\lambda\notin\mycbra{\pm\myfrac{2}{3}}$ і не має розв’язків при $\lambda\in\mycbra{\pm\myfrac{2}{3}}$.
\begin{prob}За допомогою альтернативи Фредгольма знайти всі $\lambda\in\mathbb{C}$, при яких наступне інтегральне рівняння
	має єдиний розв’язок при всіх $y\in C[0,2\pi]$:
	\[x(t)=\lambda\int_0^{2\pi}\cos(2t-\tau)x(\tau)\;d\tau+y(t),\;t\in[0,2\pi]\]
\end{prob}
Альтернатива Фредгольма (для інтегральних рівнянь) стверджує, що рівняння 
\[x(t)=\lambda\int_0^{2\pi}\cos(2t-\tau)x(\tau)\;d\tau+y(t)\]
має єдиний розв’язок для кожного $y\in C[0,2\pi]$ тоді і лише тоді, коли
\[x(t)=\lambda\int_0^{2\pi}\cos(2t-\tau)x(\tau)\;d\tau\]
має лише тривіальний розв’язок. Альтернативу Фредгольма можна застосувати, адже $K(t,\tau)=\cos(2t-\tau)$ є неперервною функцією 
(і відповідно $x(t)\mapsto\lambda\int_0^{2\pi}\cos(2t-\tau)x(\tau)\;d\tau$ є компактним оператором). Друге рівняння, в свою чергу, має вироджене
ядро, а отже має нетривіальні розв’язки тоді і лише тоді, коли їх має система
\[\begin{bmatrix}\lambda\int_0^{2\pi}\cos(2t)\cos(t)\;dt-1&0\\0&\lambda\int_0^{2\pi}\sin(2t)\sin(t)\;dt-1\end{bmatrix}
	\begin{bmatrix}x_1\\x_2\end{bmatrix}=\begin{bmatrix}0\\0\end{bmatrix}\]
		що в свою чергу відбувається коли і тільки коли $1/\lambda\in\mycbra{{\int_0^{2\pi}\cos(2t)\cos(t)\;dt},
		{\int_0^{2\pi}\sin(2t)\sin(t)\;dt}}=\mycbra{0,0}$, а отже для всіх $\lambda\in\mathbb{C}$, рівняння
		\[x(t)=\lambda\int_0^{2\pi}\cos(2t-\tau)x(\tau)\;d\tau+y(t)\]
		має єдиний розв’язок для довільного $y\in C[0,2\pi]$. 
\begin{prob}
	Знайти характеристичні числа, відповідні власні функції та розв’язки інтегрального рівняння
	\[x(t)=\lambda\int_{-\pi}^{\pi}\sin t\sin\tau x(\tau)\;d\tau-\sin t+\cos t,\;t\in[-\pi,\pi]\]
\end{prob}
	Відповідне однорідне рівняння
	\[x(t)=\lambda\int_{-\pi}^{\pi}\sin t\sin\tau x(\tau)\;d\tau\]
	як і оригінальне неоднорідне, є рівнянням з виродженим ядром, а отже має нетривіальні розв’язки тоді і лише тоді, коли їх має лінійне
	рівняння
	\[x_1\mybra{\lambda\int_{-\pi}^\pi\sin^2 t\dt-1}=0\]
	\[x_1\mybra{\lambda\pi-1}=0\]
	Відповідно, $\lambda=\myfrac{1}{\pi}$ є єдиним власним числом, а $x(t):=\sin(t)/\sqrt{\pi}$ -- відповідної власною функцією.

	В свою чергу, розв’язки рівняння
	\[x(t)=\lambda\int_{-\pi}^{\pi}\sin t\sin\tau x(\tau)\;d\tau-\sin t+\cos t,\;t\in[-\pi,\pi]\]
	взаємно-відповідні розв’язкам системи
	\[x_1\mybra{{\lambda}{\pi}-1}=\int_{-\pi}^\pi(\sin t-\cos t)\sin t\dt=\pi\]
	Таким чином, рівняння
	\[x(t)=\lambda\int_{-\pi}^{\pi}\sin t\sin\tau x(\tau)\;d\tau-\sin t+\cos t,\;t\in[-\pi,\pi]\]
	має розв’язок \[x(t):=-\sin t+\cos t+\frac{\pi}{\lambda\pi-1}\lambda\sin t\]
	при $\lambda\neq\myfrac{1}{\pi}$ і жодних розв’язків в іншому разі.
\begin{prob}
	Довести, що функціонал є узагальненою функцією	\[f(\phi)=\int_{\mathbb{R}}e^{-x}\phi'(x)\dx,\;\phi\in\mathcal{D}(\mathbb{R})\]
	Чи буде вона регулярною?
\end{prob}
\newcommand{\supp}{\mbox{supp }}
Лінійність очевидна, залишається лише довести неперервність. Нехай $\mathcal{D}(\mathbb{R})\ni\phi_n\to\phi$. За означенням
збіжності в $\mathcal{D}(\mathbb{R})$, $\exists r>0,\;
\forall n\;\widetilde{B_r}(0)\supset\supp\phi_n$ і таким чином, оскільки $\supp\phi'_n\subset\supp\phi_n$, маємо
\[f(\phi_n)=\int_{B_r(0)}e^{-x}\phi'_n(x)\dx\to\int_{B_r(0)}=\int_{B_r(0)}e^{-x}\phi'(x)\dx=\int_{\mathbb{R}}e^{-x}\phi'(x)\dx\]
за теоремою Лебега про граничний перехід під знаком інтеграла,
оскільки збіжність $\phi_n\to\phi$ є рівномірною на $B_r(0)$ за означенням збіжності в $\mathcal{D}(\mathbb{R})$.

Щодо регулярності, якщо $\phi\in\mathcal{D}(\mathbb{R})$ і $\supp\phi'\subset\supp\phi\subset[-A,A]$, причому $\phi(\pm A)=0$, використовуючи
інтегрування частинами, маємо
\[f(\phi)=\int_{-A}^Ae^{-x}\phi'(x)\dx=e^{-x}\phi(x)\bigg|_{-A}^A-\int_{-A}^A(-e^{-x})\phi(x)\dx=\int_{-A}^Ae^{-x}\phi(x)\dx=\int_\mathbb{R}
e^{-x}\phi(x)\dx\]
і $f$ є регулярною загальною функцією.
\begin{prob}
	\[f(x)=\left\{\begin{array}{ll}\cos x,\;x\leq0\\1,\;x>0\end{array}\right.,\;\mbox{$f',\;f''$--?{ в }$\mathcal{D}'(\mathbb{R})$}\]
\end{prob}
За означенням,
\[f'(\phi)=-\int_{-\infty}^\infty f(x)\phi'(x)\dx=-\int_{-\infty}^0\cos(x)\phi'(x)\dx-\int_0^{\infty}\phi'(x)\dx=\]
\[=-\cancel{\cos(0)\phi(0)}+\cos(-\infty)\underbrace{\phi(-\infty)}
_{=0}-\int_{-\infty}^0\sin(x)\phi(x)\dx-\underbrace{\phi(+\infty)}_{=0}+\cancel{\phi(0)}=\int_{-\infty}^\infty f'(x)\phi(x)\dx\]
де $f'(x)$ визначено як
\[f'(x):=\left\{\begin{array}{ll}-\sin(x),&x<0\\0,&x\geq0\\\end{array}\right.\]
і відповідно 
\[f''(\phi)=-\int_{-\infty}^{0}(-\sin(x))\phi'(x)\dx=\underbrace{\sin(0)}_{=0}\phi(0)
-\sin(-\infty)\underbrace{\phi(-\infty)}_{=0}-\int_{-\infty}^{0}\cos(x)\phi(x)\dx=\]\[=\int_{-\infty}^\infty f''(x)\phi(x)\dx\]
де $f''(x)$ визначено як
\[f''(x):=\left\{\begin{array}{ll}-\cos(x),&x<0\\0,&x\geq0\\\end{array}\right.\]
\begin{prob}
	Довести
	\[F[\frac{1}{2i}(\delta_h-\delta_{-h})]=\frac{1}{\sqrt{2\pi}}\sin(hy),\;h\in\mathbb{R},\;\mbox{де}\]
	$F$ -- перетворення Фур’є
\end{prob}
Нехай $\psi\in\mathcal{S}(\mathbb{R})$. Нам потрібно показати, що
\[\myabra{\frac{1}{2i}(\delta_h-\delta_{-h}),F(\psi)}=\myabra{\frac{1}{\sqrt{2\pi}}\sin(hy),\psi}\]
переписуючи ліву і праву частину, маємо
\[\frac{1}{2i}\cdot\frac{1}{\sqrt{2\pi}}{\int_{\mathbb{R}}\psi(t)(e^{iht}-e^{-iht})\dt}=\frac{1}{\sqrt{2\pi}}\int_{\mathbb{R}}\psi(t)
\sin(ht)\dt\]
оскільки $e^{iht}-e^{-iht}=2i\sin(ht)$, рівність доведено.
\begin{thebibliography}{9}
\bibitem{tb}
Березанський Ю. М., Ус Г. Ф., Шефтель З. Г.
Митропольський Ю. А., Самойленко А. М., Кулик В. Л.
\emph{Функціональний аналіз}.
Київ, "Вища школа"{}, 1990, російською мовою, 600 с.
\end{thebibliography}
\end{document}
