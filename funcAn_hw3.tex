\documentclass[12pt]{article} % use larger type; default would be 10pt

\usepackage{mathtext}                 % підключення кирилиці у математичних формулах
                                          % (mathtext.sty входить в пакет t2).
\usepackage[T1,T2A]{fontenc}         % внутрішнє кодування шрифтів (може бути декілька);
                                          % вказане останнім діє по замовчуванню;
                                          % кириличне має співпадати з заданим в ukrhyph.tex.
\usepackage[utf8]{inputenc}       % кодування документа; замість cp866nav
                                          % може бути cp1251, koi8-u, macukr, iso88595, utf8.
\usepackage[english,russian,ukrainian]{babel} % національна локалізація; може бути декілька
                                          % мов; остання з переліку діє по замовчуванню. 
\usepackage{amsthm}
\usepackage{amsmath}
\usepackage{amsfonts}
\usepackage{graphicx}
\usepackage[pdftex]{hyperref}
\usepackage{caption}
\usepackage{subfig}
\usepackage{fancyhdr}

\newtheorem{prob}{Завдання}

\usepackage{mystyle}

\newtheorem{myulem}[theorem]{Лема}

\renewenvironment{myproof}[1][Доведення]{\begin{trivlist}
\item[\hskip \labelsep {\bfseries #1}]}{\myqed\end{trivlist}}

\title{Контрольна робота з функціонального аналізу (9 семестр)\\Вар. 1}
\author{Олексій Леонтьєв}

\begin{document}
\maketitle
\begin{prob}Знайти оператор, спряжений до $A:l_2\mapsto l_2$\[Ax=(\frac{1}{2}x_2,x_3,x_4,\dots)\]\end{prob}
	Нагадаємо, що нам потрібно знайти (єдиний, за теоремою Ріса) оператор, що для всіх $x,y\in l_2$ задовольняв би рівності
	\[\mysca{Ax}{y}=\mysca{x}{A^*y}\]
	Ми стверджуємо, що $A^*:l_2\mapsto l_2$ заданий як
	\[A^*y=(0,\frac{1}{2}y_1,y_2,\dots)\]
	задовольняє цій умові. Це легко перевірити, адже $A^*$ є неперервним оператором на $l_2$ і до того ж ми маємо (всі суми нижче збіжні)
	\[\mysca{Ax}{y}=
	\frac{1}{2}x_2\overline{y_1}+x_3\overline{y_2}+x_4\overline{y_3}+\dots=
	x_1\cdot 0+x_2\cdot\overline{\frac{1}{2}y_1}+x_3\overline{y_2}+x_4\overline{y_3}+\dots=\]
	\[\mysca{(x_1,x_2,x_3,\dots)}{(0,\frac{1}{2}y_1,y_2,y_3,\dots)}=\mysca{x}{A^*y}\]
\begin{prob}Довести, що оператор $A:B\mapsto B$ є скінченновимірним, де $B=C[0,\pi]$, $(Ax)(t)=\int_0^{\pi}\sin(t+\tau)x(\tau)d\tau,\;t\in[0,\pi]$.
	Чи буде $A$ компактним оператором?\end{prob}
	Помітимо, що
	\[\forall x\in B,\;(Ax)(t)=\int_0^{\pi}\sin(t+\tau)x(\tau)d\tau=\int_0^{\pi}\left(\sin t\cos\tau+\cos t\sin\tau\right)x(\tau)d\tau=\]
	\[\sin t\int_0^{\pi}\cos\tau x(\tau)d\tau+\cos t\int_0^{\pi}\sin\tau x(\tau)d\tau\in \left<\left\{\sin t,\cos t\right\}\right>\]
	Оскільки $\left<\left\{\sin t,\cos t\right\}\right>:=\mysetn{\alpha\sin t+\beta\cos t}{\alpha,\beta\in\mathbb{C}}$ є скінченновимірним
	підпростором $B$ (розмірності 2, адже $\sin t$ і $\cos t$ лінійно незалежні в $B$)
	, скінченновимірним є і множина значень $A$, таким чином останній є скінченновимірним.\\
	Більше того, $A$ є компактним оператором. Дійсно, якщо $X\subset B$ - обмежена множина, $A(X)\subset
	\left<\left\{\sin t,\cos t\right\}\right>$ також буде обмеженою в $\left<\left\{\sin t,\cos t\right\}\right>\simeq \mathbb{C}^2$
	, через обмеженість оператора $A$, а отже замикання $A(X)$ в $B$ буде міститися в $\mathbb{C}^2\simeq
	\left<\left\{\sin t,\cos t\right\}\right>$
	(адже останній є скінченновимірним підпростором $B$ і тому замкненою підмножиною $B$) і буде замкненою обмеженою множиною. Проте, 
	кожна замкнена обмежена множина в $\mathbb{C}^2\simeq
	\left<\left\{\sin t,\cos t\right\}\right>$ є компактною, а отже $A(X)$ буде прекомпактною в $B$ і тому $A$ - компактний оператор.
\begin{prob}Знайти спектр, власні числа, норму і спектральний радіус оператора $A:l_2\mapsto l_2$, $Ax=(x_1,x_1-x_2,x_3,x_4,\dots)$\end{prob}
	Нехай $S\subset\mathbb{C}$ позначає спектр $A$. Покажемо, що $S\subseteq\{1,-1\}$. Дійсно, зафіксуємо довільне $\lambda\in\mathbb{C}\setminus
	\{1,-1\}$ і покажемо, що $\lambda I-A$ є неперервно оборотнім - з цього випливатиме бажане $\lambda\notin S$. За критерієм неперервної
	оборотності достатньо показати, що для певного $m>0$ виконується $\forall x\in l_2,\;\mynorm{(\lambda I-A)x}\geq m\mynorm{x}$. Через лінійність
	оператора і однорідність норми достатньо показати, що $\forall x\in\mysetn{x\in l_2}{\mynorm{x}=1},\;\mynorm{(\lambda I-A)x}>m$. Помітимо, що
	якщо $\mynorm{x}=1$ виконується
	\begin{gather*}
	\mynorm{(\lambda I-A)x}^2=\myabs{\lambda-1}^2(\myabs{x_1}^2+\myabs{x_3}^2+\myabs{x_4}^2+\dots)+\myabs{\lambda x_2+x_2-x_1}^2=\\
	=\myabs{\lambda-1}^2(1-\myabs{x_2}^2)+\myabs{\lambda x_2+x_2-x_1}^2
	\end{gather*}
	і таким чином
	\[\inf_{\mynorm{x}=1}\mynorm{(\lambda I-A)x}^2=\inf_{\substack{x_1,x_2\in\mathbb{C}\\\myabs{x_1}^2+\myabs{x_2}^2\leq 1}}
	\myabs{\lambda-1}^2(1-\myabs{x_2}^2)+\myabs{\lambda x_2+x_2-x_1}^2\]
	Більше того, оскільки $\mysetn{x_1,x_2\in\mathbb{C}}{\myabs{x_1}^2+\myabs{x_2}^2\leq 1}$ є компактною підмножиною $\mathbb{C}^2$, а
	$\myabs{\lambda-1}^2(1-\myabs{x_2}^2)+\myabs{\lambda x_2+x_2-x_1}^2$ - неперервна функція, точна нижня межа досягається. Таким чином,
	якщо ми підемо від супротивного і припустимо, що $\inf_{\mynorm{x}=1}\mynorm{(\lambda I-A)x}^2=0$, це буде означати, що для $a,b\in\mathbb{C}
	,\;\myabs{a}^2+\myabs{b}^2\leq 1$ виконується $\myabs{\lambda-1}^2(1-\myabs{b}^2)+\myabs{\lambda b+b-a}^2=0$. Таким чином
	$\myabs{\lambda-1}^2(1-\myabs{b}^2)=0$ і оскільки $\lambda\neq 1$ має виконуватися $\myabs{b}=1$, а отже $\myabs{a}=0$. Проте в цьому 
	разі	$\myabs{\lambda b+b-a}=\myabs{\lambda+1}>0$, оскільки $\lambda\neq -1$, що і дає протиріччя та доводить бажане $S\subseteq\{1,-1\}$.

	Продовжуючи, $\lambda=1$ та $\lambda=-1$ є власними значеннями $A$ ($x=(1,0,0,\dots)$ та $x=(0,1,0,0,\dots)$ - власні вектори відповідно).
	Оскільки власні вектори належать спектру, з цього випливає, що $S\supseteq\{1,-1\}$ і тому $S=\{1,-1\}$, а отже $\lambda=\pm 1$ - єдині
	власні числа $A$. Спектральний радіус, відповідно, $\rho=1$.

	Покажемо наостанок, що $\mynorm{A}=\frac{\sqrt{5}+3}{2}$. Дійсно, з одного боку, якщо $x\in l_2$ і $\mynorm{x}=1$ маємо
	\[\mynorm{Ax}^2=\myabs{x_1}^2+\myabs{x_1-x_2}^2+\myabs{x_3}^2+\myabs{x_4}^2+\dots\leq\]
	\[\leq \myabs{x_1}^2+(\myabs{x_1}+\myabs{x_2})^2+\myabs{x_3}^2+\myabs{x_4}^2+\dots=1+2\myabs{x_1}\myabs{x_2}+\myabs{x_1}^2\]
	За нерівністю між середнім геометричним і середнім арифметичним маємо
	\[\myabs{x_1}\myabs{x_2}\leq\frac{\left({\frac{\sqrt{5}-1}{2}}\right)\myabs{x_1}^2+\left(\frac{\sqrt{5}-1}{2}\right)^{-1}\myabs{x_2}^2}{2}\]
	відповідно,
	\[1+2\myabs{x_1}\myabs{x_2}+\myabs{x_1}^2
	\leq1+\left({\frac{\sqrt{5}-1}{2}}\right)\myabs{x_1}^2+\left(\frac{\sqrt{5}-1}{2}\right)^{-1}\myabs{x_2}^2+\myabs{x_1}^2=\]
	\[=1+\frac{\sqrt{5}+1}{2}\myabs{x_1}^2+\frac{\sqrt{5}+1}{2}\myabs{x_2}^2\leq 1+\frac{\sqrt{5}+1}{2}=\frac{\sqrt{5}+3}{2}\]
	З іншого боку, для $l_2\ni x_0:=(\sqrt{\frac{\sqrt{5}+1}{2\sqrt{5}}},-\sqrt{\frac{\sqrt{5}-1}{2\sqrt{5}}},0,0,\dots)$ виконується
	по-перше
	\[\mynorm{x}^2=\frac{\sqrt{5}+1}{2\sqrt{5}}+\frac{\sqrt{5}-1}{2\sqrt{5}}=\frac{2\sqrt{5}}{2\sqrt{5}}=1\]
	і по-друге
	\[\mynorm{Ax}=\frac{\sqrt{5}+1}{2\sqrt{5}}+\left(\sqrt{\frac{\sqrt{5}+1}{2\sqrt{5}}}+\sqrt{\frac{\sqrt{5}-1}{2\sqrt{5}}}\right)^2=\]\[=
	\frac{\sqrt{5}+1}{2\sqrt{5}}+\frac{\sqrt{5}+1}{2\sqrt{5}}+{\frac{\sqrt{5}-1}{2\sqrt{5}}}+2\sqrt{
	\frac{\sqrt{5}+1}{2\sqrt{5}}\cdot\frac{\sqrt{5}-1}{2\sqrt{5}}}=\]
	\[=\frac{\sqrt{5}+1}{2\sqrt{5}}+1+2\frac{2}{2\sqrt{5}}=\frac{\sqrt{5}+5}{2\sqrt{5}}+1=\frac{1+\sqrt{5}}{2}+1=\frac{\sqrt{5}+3}{2}\]
	що і завершує доведення бажаної рівності $\mynorm{A}=\frac{\sqrt{5}+3}{2}$
\begin{prob}Знайти спектр, власні числа і власні функції оператора $A\in L(H)$, $H=L_2[0,2\pi]$, $(Ax)(t)=\int_0^{2\pi}\cos^2(t-\tau)x(\tau)
	d\tau,\;t\in[0,2\pi]$
\end{prob}
Для початку, помітимо, що
\newcommand{\myint}[1]{\int_0^{2\pi}#1x(\tau)d\tau}
\[(Ax)(t)=\int_0^{2\pi}\cos^2(t-\tau)x(\tau)d\tau=\int_0^{2\pi}(\cos\tau\cos t+\sin\tau\sin t)^2x(\tau)d\tau=\]
\[=\cos^2 t\myint{\cos^2(\tau)}+\sin^2 t\myint{\sin^2(\tau)}+\sin 2t\myint{\frac{\sin (2\tau)}{2}}\]
Таким чином, ми бачимо, що $A$ є скінченновимірним оператором, а отже, згідно із прикладом 1.2 в главі IX \cite{tb}, 
$A$ є компактним оператором. Таким чином, згідно з теоремою 4.1 з глави IX у \cite{tb} всі ненульові елементи спектру $A$ будуть його власними
значеннями.

Покажемо, що $\lambda=0$ є власним числом $A$ і знайдемо відповідні власні функції. 
\[\mbox{Ker} A=\mysetn{x\in L_2[0,2\pi]}{Ax=0}=\]\[=\mysetn{x\in L_2[0,2\pi]}{\myint{\cos^2(\tau)}=\myint{\sin^2(\tau)}=\myint{\sin(2\tau)}=0}=\]
\[\mysetn{\alpha\sin^2(\tau)+\beta\cos^2(\tau)+\gamma\sin(2\tau)
}{\alpha,\beta,\gamma\in\mathbb{C}}^\perp=:\left<\left\{\sin^2(\tau),\cos^2(\tau),\sin(2\tau)\right\}\right>^\perp\]
Згадаємо тепер, що 
\begin{prob}Чи можуть наступні множини бути спектром деякого компактного оператора в $l_2$?\end{prob}
	У випадку, коли нам треба буде довести \textit{негативний} результат (тобто, показати, що певна множина \textit{не є}
	спектром жодного компактного оператора, ми будемо використовувати \textbf{Теорему 4.1} з глави IX
	\cite{tb}, яка перелічує деякі властивості, які
	притаманні спектру компактного оператора. Таким чином, якщо дана множина не буде задовольняти якийсь з цих властивостей, вона автоматично
	не може бути спектром компактного оператора.

	Якщо ж ми забажаємо довести, що певна множина є спектром декого компактного оператора на $l_2$, ми просто сконструюємо такий оператор. У 
	нагоді буде наступна лема.
	\begin{myulem}\label{CompactSpectrumLemma}
		Нехай $\mycbra{a_n}_{n=1}^{\infty}$ послідовність комплексних чисел, що збігається до нуля і така, що хоча б один її член
		є нулем.
		Тоді множина значень $\mycbra{a_n}_{n=1}^{\infty}$ є спектром компактного оператора
	$A:l_2\ni(x_1,x_2,\hdots)\mapsto(a_1x_1,a_2x_2,\hdots)\in l_2$.\end{myulem}
	\begin{myproof} Нам потрібно довести три твердження зростаючої складності: по-перше, треба показати, що $A$ означене як у твердженні
		дійсно буде оператором на $l_2$, 
		по-друге, що він буде компактним
		і по-третє, що його спектром дійсно буде $\mycbra{a_n}_{n=1}^{\infty}$.
		$A$ дійсно оператор, адже для $(x_1,x_2,\hdots)\in l_2$ маємо $\mynorm{A(x_1,x_2,\hdots)}=\sqrt{\sum_{i=1}^{\infty}
		\myabs{a_i}^2\myabs{x_i}^2}\leq\sqrt{\sum_{i=1}^{\infty}B^2\myabs{x_i}^2}=B\mynorm{x}<\infty$ (
		існування $B\in\mathbb{R}$ такого, що $\forall i\geq 1 \myabs{x_i}<B$, тобто обмеженість послідовності
		$\mycbra{a_n}_{n=1}^{\infty}$ випливає з її збіжності), адже $(x_1,x_2,\hdots)\in l_2\implies
		\mynorm{x}<\infty$ і отже $(a_1x_1,a_2x_2,\hdots)=A(x_1,x_2,\hdots)\in l_2$. Лінійність випливає з означення $A$.
		
		Для того, щоб показати компактність оператора $A$, розглянемо послідовність операторів $A_n(x_1,x_2,\dots)=(a_1x_1,a_2x_2,\dots,
		a_nx_n,0,0,0,\dots)$. Кожен із них є компактним, адже множина значень кожного є скінченновимірним підпростором $l_2$ (див. приклад
		1.2 з глави IX \cite{tb}). Більше того, для оскільки $a_n\to0$, для кожного довільно малого $\epsilon>0$ існує таке $N$,
		що $\forall i>N,\;\myabs{a_i}<\epsilon$ і тому $\forall i>N,\;\forall x\in l_2,\;\mynorm{x}\leq1$ маємо $\mynorm{(A-A_i)(x)}=
		\sqrt{\sum{j=i+1}^{\infty}\myabs{a_j\cdot x_j}}<\sqrt{\sum_{j=i+1}^{\infty}\epsilon^2\myabs{x_j}^2}\leq
		\sqrt{\sum_{j=1}^{\infty}\epsilon^2\myabs{x_j}^2}=\epsilon\mynorm{x}\leq\epsilon$. Таким чином, $\forall i>N,\;\mynorm{A-A_i}<
		\epsilon$ і отже $A_n\to A$ за нормою в $l_2$, тому $A$ є компактним як послідовність компактних операторів.

		Далі, кожне з комплексних чисел $a_1,a_2,\hdots$ безумовно належить спектру оператора $A$, адже кожне є його власним значенням,
		що випливає прямо з означення (зокрема, власним вектором для $a_i$ буде $e_i$). Залишається показати, що $\forall i,\;a_i\neq a\implies a\notin\sigma(A)$.
		Для цього, припустимо, що $a$ не співпадає з жодним з $a_i$, але $a\in\sigma(A)$.
		Тоді зокрема $a\neq 0$ (оскільки за гіпотезою $\exists i\geq 1,\mid a_i=0$) і тому за вищезгаданою Теоремою 4.1 з
		глави IX \cite{tb}, $a\neq0$ має бути власним значенням $A$, а отже $\exists l_2\ni v\neq0\mid Av=av\implies\forall i\geq 1
		(Av)_i=a_iv_i=av_i$. Оскільки $\forall i\geq 1,\;a_i\neq a$, маємо $v_i=0$ і тому $v=0$, і отримане протиріччя завершує доведення.
	\end{myproof}
\begin{enumerate}
	\renewcommand{\labelenumi}{\myralph{enumi})}
\item{$[-1,1]$ {\it не може бути спектром жодного компактного оператора}
	, адже за Теоремою 4.1 спектр компактного оператора може бути не більше ніж зліченним.}
	\item{Так, множина $\mycbra{-1,0,1}$ {\it є спектром компактного оператора}. Щоб побачити це, ми застосуємо доведену вище лему
		\ref{CompactSpectrumLemma} до
		послідовності $\mycbra{-1,1,0,0,\hdots}$, що дасть нам оператор $A(x_1,x_2,x_3,\hdots)=(-x_1,x_2,0,0,\hdots)$. За лемою, $A$
		дійсно компактний оператора на $l_2$ зі спектром $\mycbra{-1,0,1}$.}
	\item{За Теоремою 4.1, спектр компактного оператора має містити 0. Оскільки $0\notin\mycbra{\frac{1}{\sqrt{n}},\;n\geq 1}$, ця множина
		{\it не може бути спектром компактного оператора.}}
	\item{Знову ж таки, за лемою \ref{CompactSpectrumLemma} вище, множина значень послідовності
		$\mycbra{0,\frac{1}{\sqrt{1}},\frac{1}{\sqrt{2}},\frac{1}{\sqrt{3}},\hdots}$ {\it є спектром компактного оператора.}}
	\item{За Теоремою 4.1, якщо спектр компактного оператора нескінченний, єдиною граничною точкою його є нуль. Проте $1\neq 0$ являється
		граничною точкою множини $\mycbra{1-\sqrt{1}{\sqrt{n}},\;n\geq 1}$ і тому ця множина {\it 
		не може бути спектром компактного оператора.}}
\end{enumerate}
\begin{thebibliography}{9}
\bibitem{tb}
Березанський Ю. М., УС Г. Ф., Шефтель З. Г.
Митропольський Ю. А., Самойленко А. М., Кулик В. Л.
\emph{Функціональний аналіз}.
Київ, "Вища школа"{}, 1990, російською мовою, 600 с.
\end{thebibliography}
\end{document}
%TODO: reference link
