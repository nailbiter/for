\documentclass[10pt]{article} % use larger type; default would be 10pt

\usepackage{enumerate}
\usepackage{amsmath}
\usepackage{ulem}
\usepackage{hyperref}
\usepackage{circuitikz}
\def\Dp{\mathcal{D}'}
\def\R{\mathbb{R}}


\newcommand{\NOR}{\operatorname{NOR}}
\newcommand{\NAND}{\operatorname{NAND}}

\title{Homework 13;
30.00/50.00 (60.00\%)
}
\begin{document}

\maketitle
\secscore{1}{0}{10}
NOT ok
\secscore{2}{10}{10}
\begin{enumerate}[(a)]
  \item OK
  \item OK
  \item OK
\end{enumerate}
\secscoreFootnote{3}{0}{10}{13.10,13.11}
\begin{enumerate}[(a)]
  \item NOT ok.
    When ${Q_1,Q_2}=01,X=0$, \begin{equation*}
      \begin{array}[]{c}
        J_1=\NAND(\NAND(X,Q_2'),\NAND(X',Q_2))=\NAND(\NAND(0,0),\NAND(1,1))=\NAND(1,0)=1,\\
        K_1=X'Q_2=1,\\
        J_2=K_2=\NOR(X',Q_1)=0,\\
        Q_1^+=J_1Q_1'+K_1'Q_1=1,\\
        Q_2^+=J_2Q_2'+K_2'Q_2=1,\\
      \end{array}
    \end{equation*}
    Hence, when $X=1$, we should have $s_1\to s_3$ transition, thus contradicting your state table.
  \item NOT ok
  \item NOT ok
\end{enumerate}
\secscore{4}{10}{10}
OK
\secscore{5}{10}{10}
waveform diagram is OK
\end{document}
