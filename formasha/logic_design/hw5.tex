\documentclass[10pt]{article} % use larger type; default would be 10pt

\usepackage{enumerate}
\usepackage{amsmath}
\usepackage{ulem}
\usepackage{hyperref}
\usepackage{circuitikz}
\def\Dp{\mathcal{D}'}
\def\R{\mathbb{R}}


\title{Homework 5;
}
\begin{document}

\maketitle
\secscore{1}{5}{10}
\begin{enumerate}[(a)]
	\item OK
	\item not ok; function can be realized as
		\begin{equation*}
			(A+B(D+C))(FG+E)
		\end{equation*}
		12 gates
\end{enumerate}
\secscoreFootnote{2}{5}{10}
\begin{enumerate}[(a)]
	\item NOT ok. The function can be implemented as
		\begin{equation*}
			a'bc+a'bd'+ab'c'+ab'd'
		\end{equation*}
		this has the same number of gates, but less inputs than your
		version
	\item ok
\end{enumerate}
\secscoreFootnote{8}{5}{10}{}
\begin{enumerate}[(a)]
	\item NOT ok
	\item ok
\end{enumerate}
\secscoreFootnote{12}{0}{10}{}
NOT ok. The first function can be implemented as\begin{equation*}
	(b'+d)(a+c+d)(a+b+c)
\end{equation*}
with less gates and inputs as your version. Similar for other functions.

Also, your circuit has more than 9 gates.
\secscoreFootnote{42}{0}{10}{}

\end{document}
