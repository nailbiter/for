\documentclass[10pt]{article} % use larger type; default would be 10pt

\usepackage{enumerate}
\usepackage{amsmath}
\usepackage{ulem}
\usepackage{hyperref}
\usepackage{hyperref}
\def\Dp{\mathcal{D}'}
\def\R{\mathbb{R}}


\title{Homework 1;
 35.00/50.00 (70.00\%)
}
\begin{document}

\maketitle

\section*{Problem 1 \score{10}{10}}
ok
\section*{Problem 2 \score{0}{10} \footnote{similar exercises: \url{https://nailbiter.github.io/threadmill-for-logic-design/} and choose ``exercise 2'' in section ``homework 1''}}
NOT ok (because of rounding)
\section*{Problem 3 \score{10}{10}}
using 2's complement:
\begin{enumerate}[(a)]
  \item OK
  \item OK
  \item OK
  \item OK
\end{enumerate}
using 1's complement:
\begin{enumerate}[(a)]
  \item OK
  \item OK
  \item OK
  \item OK
\end{enumerate}
\section*{Problem 4 \score{10}{10}}
OK
\section*{Problem 5 \score{5}{10} \footnote{similar problems: write the second condition}}
\begin{itemize}
  \item OK
  \item 
    not OK, as the second condition you wrote only for 2's complement. What about 1's complement?
\end{itemize}
\end{document}
