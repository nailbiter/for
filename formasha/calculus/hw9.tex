\documentclass[10pt]{article} % use larger type; default would be 10pt

\usepackage{enumerate}
\usepackage{amsmath}
\usepackage{ulem}
\usepackage{mathtools}
\usepackage{centernot}
\def\Dp{\mathcal{D}'}
\def\R{\mathbb{R}}


\title{Homework 10 (Chap.~4.4),
68.00/110.00 (61.82\%)
}
\begin{document}
\maketitle

\secscore{1d}{10}{10}
OK, but please write \textbf{clearly} whether this is indeterminate form or not.
\secscore{2c}{10}{10}
OK, but please write \textbf{clearly} whether this is indeterminate form or not.
\secscore{3b}{10}{10}
ok
\secscore{4a}{10}{10}
ok
\secscore{9}{10}{10}
OK
\secscoreFootnote{14}{0}{10}{15,16}
NOT ok
\begin{equation*}
	\left( \tan 3x \right)'\neq \frac{1}{\cos^23x},\quad
	\left( \sin 2x \right)'\neq \cos 2x.
\end{equation*}
\secscoreFootnote{25}{0}{10}{26,27}
\begin{equation*}
	\left( 
	\sqrt{1+2x}-\sqrt{1-4x}
	\right)'\neq\frac{1}{2\sqrt{1+2x}}-\frac{1}{2\sqrt{1-4x}}
\end{equation*}
\secscore{37}{10}{10}
ok
\secscoreFootnote{48}{0}{10}{49,50}
NOT ok
\begin{equation*}
	\lim_{x\to\infty}x^{3/2}\sin\left( \frac{1}{x} \right)
	\neq\lim_{x\to\infty}\frac{x^{3/2}}{\sin\frac{1}{x}}\left( 
		=\lim_{x\to\infty}\frac{x^{3/2}}{\left( \sin\frac{1}{x} \right)^{-1}}
	\right)<++>
\end{equation*}<++>
\secscoreFootnote{51}{0}{10}{52,53}
NOT ok
\begin{equation*}
	\lim_{x\to1}\frac{\left( \ln x \right)'}{\left( \ln x+1-\frac{1}{x} \right)}\neq
	\lim_{x\to1}\frac{\frac{1}{x}}{\frac{1}{x}-\frac{1}{x^2}}\left( 
		\lim_{x\to1}\frac{\frac{1}{x}}{\frac{1}{x}+\frac{1}{x^2}}
	\right)
\end{equation*}
\secscore{59}{10}{10}
ok
\secscore{63}{10}{10}
ok
\begin{enumerate}
	\item 
		Where did you check that denominator is differentiable around 0?
	\item 
		Where did you check that  enumerator is differentiable around 0?
	\item Where did you check that denominator's derivative is nonzero near 0?
\end{enumerate}
\secscore{67}{10}{10}
ok
\begin{enumerate}
	\item 
		Where did you check that denominator is differentiable around 0?
	\item 
		Where did you check that  enumerator is differentiable around 0?
	\item Where did you check that denominator's derivative is nonzero near 0?
\end{enumerate}
\secscore{84}{10}{10}
\begin{enumerate}
	\item Where did you check that denominator's derivative is nonzero near $\theta=0$?
\end{enumerate}
\secscoreFootnote{87}{8}{10}{compute $\lim_{x\to0}\frac{f(3-3x)-f(3+4x)+f(3-7x^2)}{x}$ with $f'$ 
continuous, $f(3)=0,f'(3)=-5$; compute $\lim_{x\to0}\frac{-f(-3+5x)+f(-3+x)+f(-3)}{x}$ with $f'$ continuous,
$f(-3)=0,f'(-3)=-3$}
\begin{enumerate}
	\item 
		Where did you check that denominator is differentiable around 0?
	\item 
		Where did you check that  enumerator is differentiable around 0?
	\item Where did you check that denominator's derivative is nonzero near 0?
\end{enumerate}
\end{document}
