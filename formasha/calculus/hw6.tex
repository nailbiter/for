\documentclass[10pt]{article} % use larger type; default would be 10pt

\usepackage{enumerate}
\usepackage{amsmath}
\usepackage{ulem}
\usepackage{mathtools}
\usepackage{centernot}
\def\Dp{\mathcal{D}'}
\def\R{\mathbb{R}}


\title{Homework 7 (Chap.~3.9),
78.00/110.00 (70.91\%)
}
\begin{document}

\maketitle

\subsection*{Problem 1 \score{10}{10}}
ok
\subsection*{Problem 7 \score{10}{10}}
ok
\subsection*{Problem 11 \score{10}{10}}
ok
\subsection*{Problem 15 \score{0}{10}\footnote{similar problems: 14,16}}
\begin{equation*}
  \frac{25}{3}\neq 6\frac{1}{3}
\end{equation*}
\subsection*{Problem 17 \score{8}{10}\footnote{similar problems: 18,19}}
\begin{equation*}
  z=\sqrt{x^2+y^2}=t\sqrt{\left( \frac{dx}{dt}^2
  +\frac{dy}{dt}^2\right)}
\end{equation*}
this is not explained very well\footnote{cf. what 
  you wrote in Problem 23: $y=\frac{dy}{dt}t,x=\frac{dx}{dt}$; I think that is better}.
\subsection*{Problem 21 \score{10}{10}}
ok
\subsection*{Problem 23 \score{10}{10}}
ok
\subsection*{Problem 29 \score{10}{10}}
ok
\subsection*{Problem 31 \score{10}{10}}
ok
\subsection*{Problem 39 \score{0}{10}\footnote{similar problems: 40,41}}
\begin{equation*}
  \begin{array}[]{c}
    \displaystyle R=\frac{R_1R_2}{R_1+R_2}\\
    \mbox{{$\displaystyle \frac{\left( \frac{dR_1}{dt}R_2\mbox{${+}$}R_1\frac{dR_2}{dt} \right)\left( R_1+R_2 \right)
    \mbox{\xout{$+$}}
    R_1R_2\left( \frac{dR_1}{dt}+\frac{dR_2}{dt} \right)}{\left( R_1+R_2 \right)^2}$}}
  \end{array}
\end{equation*}
  (should be
  \begin{equation*}
    \left.\frac{\left( \frac{dR_1}{dt}R_2\mbox{${+}$}R_1\frac{dR_2}{dt} \right)\left( R_1+R_2 \right)
    -
    R_1R_2\left( \frac{dR_1}{dt}+\frac{dR_2}{dt} \right)}{\left( R_1+R_2 \right)^2}\right).
  \end{equation*}
  \subsection*{Problem 47 \score{0}{10}\footnote{similar problems: 50,46}}
\begin{align*}
  z^2=\left( h+x\sin\theta \right)^2+x^2\cos^2\theta\\
  2z\frac{dz}{dt}=2\left( h+x\sin\theta \right)\cdot\mbox{\xout{$\displaystyle\frac{dx}{dt}$}}+
  2x\cos^2\theta\frac{dx}{dt}
\end{align*}
(should be\begin{equation*}
  \left.
  2z\frac{dz}{dt}=2\left( h+x\sin\theta \right)\cdot
  \mbox{{$\displaystyle\frac{dx}{dt}$}}\sin\theta
  +
  2x\cos^2\theta\frac{dx}{dt}
  \right)
\end{equation*}
\section*{make-up \#1}
\secscore{14}{10}{10}
ok
\secscore{16}{0}{10}
NOT ok
\begin{equation*}
	\frac{dz}{dt}=\frac{(35\cdot4-\underbrace{\mbox{\sout{100}}}_{L=150})
	\cdot35+25\cdot4}{\sqrt{(150-35\cdot4)^2+100^2}}
\end{equation*}
\secscore{18}{0}{10}
NOT ok
\begin{equation*}
	\frac{dh}{dt}=\mbox{\sout{$-LH\frac{1}{(L-x)^2}$}}
\end{equation*}
where is $dx/dt$?
\secscore{19}{10}{10}
ok
\secscore{40}{0}{10}
NOT ok
\begin{equation*}
	0.007\cdot\frac{2}{3}\cdot\left( 0.12L^{2.53} \right)^{-\frac{1}{3}}
	\cdot0.12\cdot2.53\cdot L^{1.53}\cdot\frac{L_2-L_1}{\Delta t}
	\neq 2.872\cdot10^{-3}
	L^{\mbox{\sout{2.26}}}
	\cdot\frac{L_2-L_1}{\Delta t}
\end{equation*}
\secscore{41}{0}{10}
NOT ok
\begin{align*}
	c^2&=& a^2+b^2-2ab\cos\theta\\
	2c\frac{dc}{dt}&=& 0+0-\mbox{\sout{$2ab\left( -\sin\theta \right)$}}
\end{align*}
where is $d\theta/dt$?
\secscore{46}{10}{10}
ok
\secscore{50}{4}{10}
\sout{mintute} -- minute

NOT ok -- you forgot the minus at the end.
\end{document}
