\documentclass[10pt]{article} % use larger type; default would be 10pt

\usepackage{enumerate}
\usepackage{hyperref}
\usepackage{amsmath}
\usepackage{ulem}
\usepackage{mathtools}
\usepackage{centernot}
\usepackage{amsfonts}
%%\usepackage{underbrace}
\def\Dp{\mathcal{D}'}
\def\R{\mathbb{R}}


\title{Homework 21 (Chap.~11.4),
67.00/80.00 (83.75\%)
}
\begin{document}
\maketitle
\secscore{24}{10}{10}
good
\secscoreFootnote{32}{9}{10}{33,34}
why $\max\left( \sqrt[n]{n} \right)=\sqrt{2}$?
\secscoreFootnote{38}{6}{10}{36,37}
In $p>1$, the writing $\forall n$ is incorrect (does not hold, e.g. for $n=2$, since $\ln2 < 1$).

In $p=1$ you did not check hypothesis of Integral test (e.g. continuity and monotonicity).

In $p<1$ what you have written is \textbf{wrong}. Why $\frac{1}{\ln n}>\frac{1}{n^{\frac{p-1}{2}}}$ for big $n$? (note that $\frac{p-1}{2}<0$ 
when $p<1$)
\secscore{39}{10}{10}
good
\secscore{40}{10}{10}
\begin{enumerate}[(a)]
	\item good
	\item good
\end{enumerate}
\secscore{41}{10}{10}
\begin{enumerate}[(a)]
	\item good
	\item good
\end{enumerate}
\secscore{43}{2}{10}
Why do you claim that ``for large numbers $n$ $\left\{na_n\right\}$ is approaching some limit $L$''? What if this sequence has
NO limit?
\secscore{45}{10}{10}
good
\end{document}
