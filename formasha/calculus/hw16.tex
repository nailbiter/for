\documentclass[10pt]{article} % use larger type; default would be 10pt

\usepackage{enumerate}
\usepackage{hyperref}
\usepackage{amsmath}
\usepackage{ulem}
\usepackage{mathtools}
\usepackage{centernot}
\def\Dp{\mathcal{D}'}
\def\R{\mathbb{R}}


\title{Homework 17 (Chap.~6.5),
48.00/60.00 (80.00\%)
}
\begin{document}
\maketitle
\secscoreFootnote{6}{0}{10}{7,8}
NOT ok
\begin{equation*}
	\int_{-1}^1\frac{x^2}{(x^3+3)^2}dx\neq \int_{2}^4\frac{du}{u^2}\left( =\frac{1}{3}\int_2^4\frac{du}{u^2} 
	\right)
\end{equation*}
\secscoreFootnote{10}{9}{10}{11,12}
\begin{enumerate}[(a)]
	\item OK
	\item OK
	\item OK, but strictly speaking, the phrase ``according to the MVT for integrals $\int_{1}^3\frac{1}{x}dx=
		f(c)\cdot2$'' is incorrect. In this particular situation, this equality holds not because of
		MVT, but because of your choice of $c$.
\end{enumerate}
\secscore{13}{10}{10}
OK
\secscore{14}{10}{10}
OK
\secscoreFootnote{22}{9}{10}{23,24}
\begin{enumerate}[(a)]
	\item OK
	\item OK, but you wrote $s=vt\implies ds=gt dt$. Where is $v$ defined
		and what does it equal to?
\end{enumerate}
\secscore{25}{10}{10}
OK
\end{document}
