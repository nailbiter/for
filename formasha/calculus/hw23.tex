\documentclass[10pt]{article} % use larger type; default would be 10pt

\usepackage{enumerate}
\usepackage{hyperref}
\usepackage{amsmath}
\usepackage{ulem}
\usepackage{mathtools}
\usepackage{centernot}
\usepackage{amsfonts}
%%\usepackage{underbrace}
\def\Dp{\mathcal{D}'}
\def\R{\mathbb{R}}


\title{Homework 23 (Chap.~11.8),
59.00/70.00 (84.29\%)
}
\begin{document}
\maketitle
\secscore{6}{10}{10}
good
\secscoreFootnote{27}{9}{10}{28,29}
If $b_n=\frac{n!2^n}{1\cdot3\cdot\dots\cdot(2n-1)}$, why $\lim_{n\to\infty}\neq0$?
\secscore{30}{10}{10}
\secscore{31}{10}{10}
good
\secscore{33}{10}{10}
good
\secscoreFootnote{37}{0}{10}{38,39}
You can NOT use 41 here, as the statement in 41 does NOT hold when two series have same radius of convergence.

In other words, in general it is NOT true that if series $\sum_{n\ge0}a_nx^n$ and $\sum_{n\ge0}b_nb^n$ have
the same radius of convergence $R$, then $\sum_{n\ge0}(a_n+b_n)x^n$ has also radius of convergence $R$ (take $a_n=1, b_n=-1$).

Also, were did you find an explicit form of $f(x)$?
\secscore{41}{10}{10}
good
\end{document}
