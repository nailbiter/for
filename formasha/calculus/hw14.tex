\documentclass[10pt]{article} % use larger type; default would be 10pt

\usepackage{enumerate}
\usepackage{hyperref}
\usepackage{amsmath}
\usepackage{ulem}
\usepackage{mathtools}
\usepackage{centernot}
\def\Dp{\mathcal{D}'}
\def\R{\mathbb{R}}


\title{Homework 15 (Chap.~6.2),
84.00/100.00 (84.00\%)
}
\begin{document}
\maketitle
\secscoreFootnote{10}{0}{10}{11,12}
you forgot the $\pi$ in your answer.
\secscore{14}{10}{10}
OK
\secscore{42}{10}{10}
OK
\secscoreFootnote{45}{5}{10}{46,44}
\begin{enumerate}[(a)]
	\item OK
	\item NOT ok. $270.03*\pi\neq810.09\left( =848.32 \right)$
		Also, writing $\sum_{i=1}^4\left( \bar{x}_o^2 -\bar{x}_i^2\right)$ is very bad, since
		\begin{enumerate}
			\item it does not show dependence on summation index $i$
			\item it (wrongly) suggests that $i$ in $x_i$ denotes summation index, while
				it denotes ``i'' for \textit{i}inner.
		\end{enumerate}
		notation like $\sum_{i=1}^4\left( \bar{x}_o^{(i)}-\bar{x}_{i}^{(i)} \right)$ would be better
\end{enumerate}
\secscore{47}{10}{10}
OK
\secscoreFootnote{49}{9}{10}{50,51}
NOT completely ok
\begin{equation*}
	r^2=y^2+r^2-2rx+x^2\not\implies
	y=\sqrt{2x(r-x)}\left( \implies y=\sqrt{x(2r-x)} \right).
\end{equation*}<++>
\secscore{56}{10}{10}
OK
\secscore{57}{10}{10}
OK
\secscore{60}{10}{10}
OK
\secscore{66}{10}{10}
OK
\end{document}
