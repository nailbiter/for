\documentclass[10pt]{article} % use larger type; default would be 10pt

\usepackage{enumerate}
\usepackage{hyperref}
\usepackage{amsmath}
\usepackage{ulem}
\usepackage{mathtools}
\usepackage{centernot}
%%\usepackage{underbrace}
\def\Dp{\mathcal{D}'}
\def\R{\mathbb{R}}


\title{Homework 18 (Chap.~11.3),
55.00/90.00 (61.11\%)
}
\begin{document}
\maketitle
%%\secscoreFootnote{6}{0}{10}{7,8}
\secscoreFootnote{22}{9}{10}{20,21}
OK, but where did you check the requirements for Integral Test (e.g. the fact that $f(x)$ is continuous, positive
and decreasing)?

Also, series \textit{converges} (not \textit{converge}).
\secscoreFootnote{24}{9}{10}{19,23}
OK, but where did you check the requirements for Integral Test (e.g. the fact that $f(x)$ is continuous, positive
and decreasing)?
\secscoreFootnote{26}{9}{10}{18,25}
Same as in previous problem.
\secscoreFootnote{32}{0}{10}{31,33}
Answer is wrong. For example, this series converges when $p=3/2<2$:
\begin{equation*}
	\sum_{n=1}^\infty \frac{\ln n}{n^{3/2}}=
	\underbrace{\sum_{n=1}^\infty \frac{1}{n^{5/4}}}_{\mbox{convergent}} 
	\quad\cdot\underbrace{\frac{\ln n}{n^{1/4}}}_{
		\mbox{$<1$ for $n$ big}}
\end{equation*}
\secscoreFootnote{39}{6}{10}{39,40}
Note that the fact that reminder is less that $10^{-6}$ does NOT in general guarantee correctness to fifth decimal
places. For example, take\begin{equation*}
	a=1.9999999,\quad b=0.0000001<10^{-6},
\end{equation*}
but we still have $a+b=2$, so $a$ and $a+b$ have different fifth decimal sign.
\secscore{43}{10}{10}
good
\secscore{45}{10}{10}
good
\secscoreFootnote{46}{2}{10}{44,42}<++>
Answer is wrong. For example, for $c=0\le1$ series diverges.
\end{document}
