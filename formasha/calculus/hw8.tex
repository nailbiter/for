\documentclass[10pt]{article} % use larger type; default would be 10pt

\usepackage{enumerate}
\usepackage{amsmath}
\usepackage{ulem}
\usepackage{mathtools}
\usepackage{centernot}
\def\Dp{\mathcal{D}'}
\def\R{\mathbb{R}}


\title{Homework 8 (Chap.~4.1),
	91.50/100.00 (91.50\%)
}
\begin{document}

\maketitle

\secscore{6}{10}{10}
ok
\secscoreFootnote{7}{9}{10}{8,9}
on your picture, local minima is clearly located \textit{a bit on the left} to 2.
\secscore{10}{10}{10}
ok, but not that you also have critical point at 2.
\secscoreFootnote{27}{7.5}{10}{28, also do the same for function defined as $f(x)=\sin x$ on $[-\pi/2,0]$
and $f(x)=-2+4x$ on $(0,4]$}
$f(0)=0$ is not a local minimum, since for $x\to0+,f(x)\to2$.
\secscoreFootnote{37}{5}{10}{38,39}
\begin{equation*}
	\frac{3}{2}=\frac{1}{\sqrt{t}}\centernot{\implies }t=\sqrt{\frac{2}{3}}
\end{equation*}
\secscore{41}{10}{10}
ok
\secscore{51}{10}{10}
ok
\secscore{54}{10}{10}
ok
\secscore{57}{10}{10}
ok
\secscore{76}{10}{10}
ok
\end{document}
