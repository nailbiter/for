\documentclass[10pt]{article} % use larger type; default would be 10pt

\usepackage{enumerate}
\usepackage{hyperref}
\usepackage{amsmath}
\usepackage{ulem}
\usepackage{mathtools}
\usepackage{centernot}
\usepackage{amsfonts}
%%\usepackage{underbrace}
\def\Dp{\mathcal{D}'}
\def\R{\mathbb{R}}


\title{Homework 24 (Chap.~13.1),
}
\begin{document}
\maketitle
\secscore{2}{10}{10}
good
\secscoreFootnote{6}{0}{10}{7,8}
It's not $\sin\frac{1}{t}$, but $t\sin\frac{1}{t}$.
\secscore{12}{10}{10}
good
\secscore{21}{10}{10}
good
\secscore{22}{10}{10}
good
\secscore{23}{10}{10}
good
\secscore{24}{10}{10}
good
\secscore{25}{10}{10}
good
\secscore{26}{10}{10}
good
\secscoreFootnote{32}{0}{10}{33,34}
why do you have $\sin^{-1}$?
\secscore{41}{10}{10}
good
\secscoreFootnote{45}{9}{10}{46,47}
where did you prove that \textit{every} point of intersection can be represented in this way?
\end{document}
