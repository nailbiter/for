\documentclass[10pt]{article} % use larger type; default would be 10pt

\usepackage{enumerate}
\usepackage{hyperref}
\usepackage{amsmath}
\usepackage{ulem}
\usepackage{mathtools}
\usepackage{centernot}
\usepackage{amsfonts}
%%\usepackage{underbrace}
\def\Dp{\mathcal{D}'}
\def\R{\mathbb{R}}


\title{Homework 19 (Chap.~11.10),
}
\begin{document}
\maketitle
\secscore{2}{10}{10}
good.
\secscore{8}{10}{10}
good.
\secscore{25}{10}{10}
good, but you were NOT asked to find convergence radius.
\secscore{28}{10}{10}
good, but in general the notation
\begin{equation*}
	\mbox{some predicate of $x$}\quad\forall x
\end{equation*}
is discouraged and should be replaced by
\begin{equation*}
 	\forall x,\quad
	\mbox{some predicate of $x$}
\end{equation*}
also, it's better to write explicitly $\forall x\in\mathbb{R}$.
\secscoreFootnote{34}{5}{10}{35,36}
You only proved that radius of convergence is $\ge1$.
\secscore{43}{10}{10}
good
\secscore{55}{10}{10}
good
\secscore{66}{10}{10}
good
\secscore{69}{10}{10}
good
\secscore{69}{10}{10}
good
\secscore{74}{10}{10}
good
\secscoreFootnote{84}{4}{10}{85,86}
How (where?) did you prove that derivatives of all orders are equal to zero? (you checked only first and second)

Where is part b)?
\end{document}
