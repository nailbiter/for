\documentclass[10pt]{article} % use larger type; default would be 10pt

\usepackage{enumerate}
\usepackage{hyperref}
\usepackage{amsmath}
\usepackage{ulem}
\usepackage{mathtools}
\usepackage{centernot}
\usepackage{amsfonts}
%%\usepackage{underbrace}
\def\Dp{\mathcal{D}'}
\def\R{\mathbb{R}}


\title{Homework 22 (Chap.~11.7),
57.00/60.00 (95.00\%)
}
\begin{document}
\maketitle
\secscoreFootnote{9}{7}{10}{10,11}
Strictly speaking, the fact that $\frac{\pi^2}{(2n+2)(2n+1)}<1\;\forall n\in\mathbb{N}$ does NOT imply that $b_n$ is decreasing.
(what about $n=0$?)

Also, why $\lim_{n\to\infty}\frac{\pi^2}{(2n)!}=0$? I do not quite understand your explanation.
\secscore{14}{10}{10}
good
\secscore{24}{10}{10}
good
\secscore{25}{10}{10}
good
\secscore{33}{10}{10}
good
\secscore{36}{10}{10}
good
\end{document}
