\documentclass[8pt]{article} % use larger type; default would be 10pt

\usepackage{graphicx}
\usepackage{float}
\usepackage{subfig}
\usepackage{amsmath}
\usepackage{amsfonts}
\usepackage{hyperref}
\usepackage{enumerate}
\usepackage{enumitem}
\usepackage{cancel}
\usepackage{uline}

\usepackage{mystyle}
\newcommand{\myfrac}[2]{^{#1}/_{#2}}

\title{Math 1540\\University Mathematics for Financial Studies\\2013-14 Term 1\\Suggested problems for tutorial on\\October 15, 2013}
\date{}
\begin{document}
\maketitle
In this tutorial we will practice in computing the inverse of a matrix via the explicit formulas provided by the adjoint matrix concept, as well as solving systems of linear
equations, again using the explicit formulas given by the Cramer Rule. For simplicity, we shall restrict ourselves to the matrices of size $3\times3$.

\section{Inverse Matrix via the Adjoint Matrix (theory)}
Essentially, the process is done in a four steps, as outlined below
\[A=\begin{pmatrix}a_{11}&a_{12}&\hdots&a_{1n}\\a_{21}&a_{22}&\hdots&a_{2n}\\\vdots&\vdots&\ddots&\vdots\\a_{n1}&a_{n2}&\hdots&a_{nn}\end{pmatrix}
	\xrightarrow{\text{cofactor matrix}}
	\begin{pmatrix}\det M_{11}&\det M_{12}&\hdots&\det M_{1n}\\\det M_{21}&\det M_{22}&
		\hdots&\det M_{2n}\\\vdots&\vdots&\ddots&\vdots\\\det M_{n1}&\det M_{n2}&\hdots&\det M_{nn}\end{pmatrix}\]
	\[\xrightarrow{\text{change signs in chessboard pattern}}
	\begin{pmatrix}+\det M_{11}&-\det M_{12}&\hdots&(-1)^{1+n}\det M_{1n}\\-\det M_{21}&+\det M_{22}&
		\hdots&(-1)^{2+n}\det M_{2n}\\\vdots&\vdots&\ddots&\vdots\\(-1)^{n+1}\det M_{n1}&(-1)^{n+2}\det M_{n2}&\hdots&+\det M_{nn}\end{pmatrix}
		\]
	\[\xrightarrow{\text{transpose}}
	\begin{pmatrix}+\det M_{11}&-\det M_{21}&\hdots&(-1)^{n+1}\det M_{n1}\\-\det M_{12}&+\det M_{22}&
		\hdots&(-1)^{n+2}\det M_{n2}\\\vdots&\vdots&\ddots&\vdots\\(-1)^{1+n}\det M_{1n}&(-1)^{2+n}\det M_{2n}&\hdots&+\det M_{nn}\end{pmatrix}
		\]
	\[\xrightarrow{\text{divide by $\det(A)$}}
	\frac{1}{\det(A)}\begin{pmatrix}+\det M_{11}&-\det M_{21}&\hdots&(-1)^{n+1}\det M_{n1}\\-\det M_{12}&+\det M_{22}&
		\hdots&(-1)^{n+2}\det M_{n2}\\\vdots&\vdots&\ddots&\vdots\\(-1)^{1+n}\det M_{1n}&(-1)^{2+n}\det M_{2n}&\hdots&+\det M_{nn}\end{pmatrix}
		=A^{-1}\]
	To summarize,
	\begin{enumerate}
		\item As you see, on a first step we replace each $a_{ij}$ with $\det M_{ij}$, where $M_{ij}$ denotes the matrix that is obtained
			from $A$ by cutting \textit{$i$-th row and $j$-th column}. So, for example for 
			\[A=\begin{pmatrix}1&2&3\\4&5&6\\7&8&9\end{pmatrix}\] we have 
				\[M_{23}=\begin{pmatrix}1&2&\cancel{3}\\
					\cancel{4}&\cancel{5}&\cancel{6}\\
					7&8&\cancel{9}\end{pmatrix}=
					\begin{pmatrix}1&2\\7&8\end{pmatrix}\]
		\item Next, we change the signs in a chessboard pattern, so for example
			\[A=\begin{pmatrix}1&2&3\\4&5&6\\7&8&9\end{pmatrix}\] becomes
			\[A=\begin{pmatrix}+1&-2&+3\\-4&+5&-6\\+7&-8&+9\end{pmatrix}\]
		\item Transposition goes next
		\item And finally we divide by $\det A$.
	\end{enumerate}
Before proceeding to the examples, let us make one remark.\\
\textbf{Remark. } The algorithm above should \textit{never} be used for computing inverse in real life. It is \textit{extremely} slow when
compared to the row reduction method (i.e. write augmented matrix $(A\mid I)$ and bring it to reduced row echelon form to get $(I\mid A^{-1})$).
It is rather a nice theoretical tool which allows one to prove some statements and is almost useful for practical people.
\section{Inverse Matrix via the Adjoint Matrix (example)}
To illustrate the discussion above, let us work through the two examples
\begin{myeg}
\[A=\left(\begin{array}{rrr}
1&2&3\\
0&4&5\\
1&0&6\\
\end{array}\right)
	\xrightarrow{\text{cofactor matrix}}\left(\begin{array}{rrr}
24&-5&-4\\
12&3&-2\\
-2&5&4\\
\end{array}\right)
		\]
	\[\xrightarrow{\text{change signs in chessboard pattern}}
\left(\begin{array}{rrr}24&5&-4\\
-12&3&2\\
-2&-5&4\\
\end{array}\right)
		\]
	\[\xrightarrow{\text{transpose}}
\left(\begin{array}{rrr}
24&-12&-2\\
5&3&-5\\
-4&2&4\\
\end{array}\right)
		\]
		\[\xrightarrow{\text{divide by $\det(A)$}}\frac{1}{22}
\left(\begin{array}{rrr}
24&-12&-2\\
5&3&-5\\
-4&2&4\\
\end{array}\right)
		=A^{-1}\]
\textbf{Remark. } Note, that for matrices of high dimension (e.g. with {more }than 2 rows) the computation of the determinant via recursive
formula $\sum_{i=1}^n a_{1i}\det(M_{1i})$ becomes \textit{extremely inefficient and slow
} when compared to computation via the row reduction. Hence, recursive expansion
\textit{should be avoided} when matrix has more than 2 rows in favor of row reduction
. As an illustration, let us compute $\det(A)$ via the row reduction
\[
\det(A)=
\left|\begin{array}{rrr}
1&2&3\\
0&4&5\\
1&0&6\\
\end{array}\right|=
\left|\begin{array}{rrr}
1&2&3\\
0&4&5\\
0&-2&3\\
\end{array}\right|=
\left|\begin{array}{rrr}
1&2&3\\
0&4&5\\
0&0&\myfrac{11}{2}\\
\end{array}\right|=1\cdot4\cdot\frac{11}{2}=22
\]
\end{myeg}
\begin{myeg}
\[A=
\left(\begin{array}{rrr}
1&2&3\\
0&1&4\\
5&6&0\\
\end{array}\right)
	\xrightarrow{\text{cofactor matrix}}
\left(\begin{array}{rrr}
-24&-20&-5\\
-18&-15&-4\\
5&4&1\\
\end{array}\right)
		\]
	\[\xrightarrow{\text{change signs in chessboard pattern}}
\left(\begin{array}{rrr}
-24&20&-5\\
18&-15&4\\
5&-4&1\\
\end{array}\right)
		\]
	\[\xrightarrow{\text{transpose}}
\left(\begin{array}{rrr}
-24&18&5\\
20&-15&-4\\
-5&4&1\\
\end{array}\right)
		\]
		\[\xrightarrow{\text{divide by $\det(A)$}}\frac{1}{1}
\left(\begin{array}{rrr}
-24&18&5\\
20&-15&-4\\
-5&4&1\\
\end{array}\right)
		=A^{-1}\]
\end{myeg}
\section{Cramer's Rule (theory)}
Recall, that similarly to how the concept of adjoint matrix gives us an explicit formula to compute the inverse of a matrix, Cramer's Rule gives
us an explicit formulas to compute the solution of linear systems. Before, we shall proceed, let us make one remark, similar in spirit to those
we've made for determinant and adjoint matrix above.\\
\textbf{Remark. }Again, Cramer's Rule is rather a tool in theoretical construction, not a feasible way to solve real linear systems. It is
\textit{much } slower than the usual Gaussian elimination, when implemented naively.

Given linear system \[Ax=b\]
with non-singular matrix $A$, Cramer's Rule tells us that the solution to it is given by
\[x_i=\frac{\det(A_i)}{\det(A)},\;1\leq i\leq n\]
where $A_i$ denotes matrix $A$ with $i$-th column replaced by $b$.

For example, for \[A=\begin{pmatrix}1&2&3\\4&5&6\\7&8&9\end{pmatrix},\quad b=\begin{pmatrix}10\\11\\12\end{pmatrix}\]
we have
\[A_3=\begin{pmatrix}1&2&\uwave{10}\u\\4&5&6\mbox{\textcircled{11}}\\7&8&9\end{pmatrix}\]

\section{Cramer's Rule (example)}
\section{Bonus}
\end{document}
