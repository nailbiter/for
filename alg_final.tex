\documentclass[8pt]{article} % use larger type; default would be 10pt

%\usepackage[utf8]{inputenc} % set input encoding (not needed with XeLaTeX)
%\usepackage{CJK}
\usepackage[margin=1in]{geometry}
\usepackage{graphicx}
\usepackage{float}
\usepackage{subfig}
\usepackage{amsmath}
\usepackage{amsfonts}
\usepackage{hyperref}
\usepackage{enumerate}
\usepackage{enumitem}
\usepackage{harpoon}

\usepackage{mystyle}

\title{Final Exam for Math 5111, Fall 2013\\Solutions}
\author{Alex Leontiev, 1155040702, CUHK}
\begin{document}
\maketitle
\begin{enumerate}[label=\bfseries Problem \arabic*.]
	%\newcommand{\dim}{\mbox{dim}}
	%\newcommand{\Im}{\mbox{Im}}
	\setcounter{enumi}{1}
	\item Indeed, what we have to show is $R$ has an identity,
		and that every nonzero element of $R$ has an inverse in $R$ (because $R$, being an integral
		domain, is automatically a commutative ring). We claim that the identity of $R$ is $1\in F\subset R$ (the identity of $F$)
		and will begin with proving a latter statement, regarding invertibility in $R$. Indeed, let $r\in R$ be arbitrary. Then,
		mapping $r^*:
		R\ni x\mapsto r^*(x):=x\cdot r=r\cdot x\in R$ is $F$-linear, and maps nonzero elements to nonzero elements (due to the fact, that
		$R$ is an integral domain), hence its kernel is $0$, hence $\dim(\Im R)=\dim R$, thus $r^*$ is onto $R$ and there exists
		$r^{-1}:=(r^*)^{-1}(1)\in R$ -- the inverse of $r$.

		Now, to show that $1\in F\subset R$ indeed is the identity in $R$, let $r\in R$ be arbitrary, $r^{-1}$ its inverse, and we have
		\[1\cdot r=r\cdot r^{-1}\cdot r=r\
\end{enumerate}
\end{document}
