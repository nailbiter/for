\documentclass[8pt]{article} % use larger type; default would be 10pt

%\usepackage[utf8]{inputenc} % set input encoding (not needed with XeLaTeX)
%\usepackage{CJK}
\usepackage[margin=1in]{geometry}
\usepackage{graphicx}
\usepackage{float}
\usepackage{subfig}
\usepackage{amsmath}
\usepackage{amsfonts}
\usepackage{hyperref}
\usepackage{enumerate}
\usepackage{enumitem}
\usepackage{harpoon}
\usepackage{cancel}

\usepackage{mystyle}

\title{Final Exam for Math 5111, Fall 2013\\Solutions}
\author{Alex Leontiev, 1155040702, CUHK}
\begin{document}
\maketitle
\begin{enumerate}[label=\bfseries Problem \arabic*.]
	%\newcommand{\dim}{\mbox{dim}}
	%\newcommand{\Im}{\mbox{Im}}
	\setcounter{enumi}{1}
	\item Indeed, what we have to show is $R$ has an identity,
		and that every nonzero element of $R$ has an inverse in $R$ (because $R$, being an integral
		domain, is automatically a commutative ring). We claim that the identity of $R$ is $1\in F\subset R$ (the identity of $F$)
		and will begin with proving a latter statement, regarding invertibility in $R$. Indeed, let $r\in R$ be arbitrary. Then,
		mapping $r^*:
		R\ni x\mapsto r^*(x):=x\cdot r=r\cdot x\in R$ is $F$-linear, and maps nonzero elements to nonzero elements (due to the fact, that
		$R$ is an integral domain), hence its kernel is $0$, hence $\dim(\Im R)=\dim R$, thus $r^*$ is onto $R$ and there exists
		$r^{-1}:=(r^*)^{-1}(1)\in R$ -- the inverse of $r$.

		Now, to show that $1\in F\subset R$ indeed is the identity in $R$, let $r\in R$ be arbitrary. Then, by axioms of
		a vector space, $1\cdot r=r$, and hence $1$ is an identity of $E$ indeed.
	\item Problem 3
	\item \begin{enumerate}[label=(\arabic*).]
				\newcommand{\Ker}{\mbox{Ker\,}}
			\item Given exact sequence $0\to A_n\to A_{n-1}\to\cdots\to A_1\to A_0\to 0$, let us call the mapping 
				$A_i\to A_{i-1}$ as $\partial_i: A_i\to A_{i-1}$ (as a special case, $A_{n+1}:=A_{-1}:=0$, hence
				$\partial_{n+1}:0\to A_n$ and $\partial_0:A_0\to 0$ are identity and constant map respectively).
				Then, according to fundamental theorem on homomorphisms (for abelian groups), $\forall 0\leq i\leq n,
				\;A_i/\Ker \partial_i\simeq\Im\partial_i$, and since all 3 groups are finite (as kernel and image
				are subgroups of finite groups), we have
				$\myabs{\Im\partial_i}=\myabs{A_i/\Ker\partial_i}=\frac{\myabs{A_i}}
				{\myabs{\Ker\partial_i}}\implies \myabs{A_i}=\myabs{\Im\partial_i}\cdot\myabs{\Ker\partial_i}$.
				Finally, we claim that $\forall 0\leq k\leq n$ we have 
				\[\Pi_{i=0}^k \myabs{A_i}^{(-1)^i}=\myabs{\Ker\partial_k}^{(-1)^k}\]
				Note, that this claim will imply the desired statement, as for $k=n$, by exactness
				$\myabs{\Ker\partial_n}=\myabs{\Im\partial_{n+1}}=\myabs{\mycbra{0}}=1\implies
				\Pi_{i=0}^n \myabs{A_i}^{(-1)^i}=\myabs{\Ker\partial_n}^{(-1)^n}=1^{(-1)^n}=1$, which is precisely
				the desired statement. Now, we will prove the claim by induction. Base case $k=0$ is true, as
				$A_i=\Ker\partial_0$ (by the definition of $\partial_0$). Now, assuming the statement holds
				for $k$, we have
		\[\begin{array}{rr}
			\Pi_{i=0}^{k+1}\myabs{A_i}^{(-1)^i}=\Pi_{i=0}^{k}\myabs{A_i}^{(-1)^i}\cdot\myabs{A_{k+1}}^{(-1)^{k+1}}=
			&\mbox{ (\textit{by inductive assumption}) }\\
			=\myabs{\Ker\partial_k}^{(-1)^k}\cdot\myabs{A_{k+1}}^{(-1)^{k+1}}=
			&\mbox{ (\textit{as $\myabs{A_{k+1}}=\myabs{\Im\partial_{k+1}}\cdot\myabs{\Ker\partial_{{k+1}}}$}) }\\
			={\myabs{\Ker\partial_k}^{(-1)^k}}\cdot{\myabs{\Im\partial_{k+1}}^{(-1)^{k+1}}}
				\cdot\myabs{\Ker\partial_{{k+1}}}^{(-1)^{k+1}}=
			&\mbox{ (\textit{as ${\Im\partial_{k+1}}\simeq{\Ker\partial_k}$ by exactness}) }\\
			=\cancel{\myabs{\Ker\partial_k}^{(-1)^k}}\cdot\cancel{\myabs{\Im\partial_{k+1}}^{(-1)^{k+1}}}
				\cdot\myabs{\Ker\partial_{{k+1}}}^{(-1)^{k+1}}=
			&\myabs{\Ker\partial_{{k+1}}}^{(-1)^{k+1}}
		\end{array}\]
		This finishes the induction step, proves claim, and gives the desired statement.
			\item Again, we can show by induction that 
		\end{enumerate}
\end{enumerate}
\end{document}
