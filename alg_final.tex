\documentclass[8pt,fleqn]{article} % use larger type; default would be 10pt

\usepackage[margin=1in]{geometry}
\usepackage{graphicx}
\usepackage{float}
\usepackage{subfig}
\usepackage{amsmath}
\usepackage{amsfonts}
\usepackage{hyperref}
\usepackage{enumerate}
\usepackage{enumitem}
\usepackage{harpoon}
\usepackage{cancel}

\usepackage{mystyle}

\newcommand{\myexplain}[1]{\mbox{ (\textit{#1})}}

\title{Final Exam for Math 5111, Fall 2013\\Solutions}
\author{Alex Leontiev, 1155040702, CUHK}
\begin{document}
\maketitle
\begin{enumerate}[label=\bfseries Problem \arabic*.]
	%\newcommand{\dim}{\mbox{dim}}
	%\newcommand{\Im}{\mbox{Im}}
	\setcounter{enumi}{1}
	\item Indeed, what we have to show is $R$ has an identity,
		and that every nonzero element of $R$ has an inverse in $R$ (because $R$, being an integral
		domain, is automatically a commutative ring). We claim that the identity of $R$ is $1\in F\subset R$ (the identity of $F$)
		and will begin with proving a latter statement, regarding invertibility in $R$. Indeed, let $r\in R$ be arbitrary. Then,
		mapping $r^*:
		R\ni x\mapsto r^*(x):=x\cdot r=r\cdot x\in R$ is $F$-linear, and maps nonzero elements to nonzero elements (due to the fact, that
		$R$ is an integral domain), hence its kernel is $0$, hence $\dim(\Im R)=\dim R$, thus $r^*$ is onto $R$ and there exists
		$r^{-1}:=(r^*)^{-1}(1)\in R$ -- the inverse of $r$.

		Now, to show that $1\in F\subset R$ indeed is the identity in $R$, let $r\in R$ be arbitrary. Then, by axioms of
		a vector space, $1\cdot r=r$, and hence $1$ is an identity of $E$ indeed.
	\item 
	\item \begin{enumerate}[label=(\arabic*).]
				\newcommand{\Ker}{\mbox{\normalfont Ker\,}}
			\item Given exact sequence $0\to A_n\to A_{n-1}\to\cdots\to A_1\to A_0\to 0$, let us call the mapping 
				$A_i\to A_{i-1}$ as $\partial_i: A_i\to A_{i-1}$ (as a special case, $A_{n+1}:=A_{-1}:=0$, hence
				$\partial_{n+1}:0\to A_n$ and $\partial_0:A_0\to 0$ are identity and constant map respectively).
				Then, according to fundamental theorem on homomorphisms (for abelian groups), $\forall 0\leq i\leq n,
				\;A_i/\Ker \partial_i\simeq\Im\partial_i$, and since all 3 groups are finite (as kernel and image
				are subgroups of finite groups), we have
				$\myabs{\Im\partial_i}=\myabs{A_i/\Ker\partial_i}=\frac{\myabs{A_i}}
				{\myabs{\Ker\partial_i}}\implies \myabs{A_i}=\myabs{\Im\partial_i}\cdot\myabs{\Ker\partial_i}$.
				Finally, we claim that $\forall 0\leq k\leq n$ we have 
				\[\Pi_{i=0}^k \myabs{A_i}^{(-1)^i}=\myabs{\Ker\partial_k}^{(-1)^k}\]
				Note, that this claim will imply the desired statement, as for $k=n$, by exactness
				$\myabs{\Ker\partial_n}=\myabs{\Im\partial_{n+1}}=\myabs{\mycbra{0}}=1\implies
				\Pi_{i=0}^n \myabs{A_i}^{(-1)^i}=\myabs{\Ker\partial_n}^{(-1)^n}=1^{(-1)^n}=1$, which is precisely
				the desired statement. Now, we will prove the claim by induction. Base case $k=0$ is true, as
				$A_i=\Ker\partial_0$ (by the definition of $\partial_0$). Now, assuming the statement holds
				for $k$, we have
		\[\begin{array}{rr}
			\Pi_{i=0}^{k+1}\myabs{A_i}^{(-1)^i}=\Pi_{i=0}^{k}\myabs{A_i}^{(-1)^i}\cdot\myabs{A_{k+1}}^{(-1)^{k+1}}=
			&\mbox{ (\textit{by inductive assumption}) }\\
			=\myabs{\Ker\partial_k}^{(-1)^k}\cdot\myabs{A_{k+1}}^{(-1)^{k+1}}=
			&\mbox{ (\textit{as $\myabs{A_{k+1}}=\myabs{\Im\partial_{k+1}}\cdot\myabs{\Ker\partial_{{k+1}}}$}) }\\
			={\myabs{\Ker\partial_k}^{(-1)^k}}\cdot{\myabs{\Im\partial_{k+1}}^{(-1)^{k+1}}}
				\cdot\myabs{\Ker\partial_{{k+1}}}^{(-1)^{k+1}}=
			&\mbox{ (\textit{as ${\Im\partial_{k+1}}\simeq{\Ker\partial_k}$ by exactness}) }\\
			=\cancel{\myabs{\Ker\partial_k}^{(-1)^k}}\cdot\cancel{\myabs{\Im\partial_{k+1}}^{(-1)^{k+1}}}
				\cdot\myabs{\Ker\partial_{{k+1}}}^{(-1)^{k+1}}=
			&\myabs{\Ker\partial_{{k+1}}}^{(-1)^{k+1}}
		\end{array}\]
		This finishes the induction step, proves claim, and gives the desired statement.
			\item We will use the same notation for $\partial_i$, as in previous subproblem, that is $\partial_i:A_i\to
				A_{i-1}$ is the complex mapping from $A_i$ to $A_{i-1}$ for $0\leq i\leq n+1$, with $A_{n+1}:=A_{-1}:=
				=0$. Now, we will show by math induction that for any $0\leq k\leq n$ we have
				\[\Pi_{i=0}^k\myabs{A_i}^{(-1)^i}=\myabs{\Im\partial_{k+1}}^{(-1)^{k}}\cdot
				\Pi_{i=0}^k\myabs{H_i}^{(-1)^i}\]
				This statement will imply the desired, as for $k=n$, we have $\Im\partial_{n+1}=\mycbra{0}\implies
				\myabs{\Im\partial_{n+1}}^{(-1)^{n+1}}=1^{(-1)^{n+1}}=1$. Now, base case $i=0$ holds, as
		\[\begin{array}{rr}
			\myabs{H_0}=\
			&\mbox{ (\textit{as} $H_0:=\Ker\partial_0/\Im\partial_{1}$ {\it and by finiteness}) }\\
			=\frac{\Ker\partial_0}{\myabs{\Im\partial_1}}=
			&\mbox{ ({$\partial_0\equiv0\implies\Ker\partial_0=A_0$}) }\\
			=\frac{\myabs{A_0}}{\myabs{\Im\partial_1}}\implies&\myabs{A_0}=\myabs{\Im\partial_1}^{(-1)^0}\cdot\myabs{H_0}
		\end{array}\]
		So it remains to do the induction step. Assuming that statement holds for some $k$, for $k+1$ we have
		\noindent\[\begin{array}{rr}
		\Pi_{i=0}^{k+1}\myabs{A_i}^{(-1)^i}=\Pi_{i=0}^k\myabs{A_i}^{(-1)^i}\cdot\myabs{A_{k+1}}^{(-1)^{k+1}}=
		&\myexplain{by inductive assumption}\\
		=\myabs{\Im\partial_{k+1}}^{(-1)^{k}}\cdot\Pi_{i=0}^k\myabs{H_i}^{(-1)^i}\cdot\myabs{A_{k+1}}^{(-1)^{k+1}}=
		&\myexplain{as $A_{k+1}/\Ker\partial_{k+1}\simeq\Im\partial_{k+1}$ and by finiteness}\\
		=\cancel{\myabs{\Im\partial_{k+1}}^{(-1)^{k}}}
		\cdot\Pi_{i=0}^k\myabs{H_i}^{(-1)^i}\cdot\myabs{\Ker\partial_{k+1}}^{(-1)^{k+1}}
		\cancel{\myabs{\Im\partial_{k+1}}^{(-1)^{k+1}}}=
		&\myexplain{{as} $H_{k+1}:=\Ker\partial_{k+1}/\Im\partial_{k+2}$ {and by finiteness}}\\
		=\Pi_{i=0}^k\myabs{H_i}^{(-1)^i}\cdot\myabs{H_{k+1}}^{(-1)^{k+1}}\cdot\myabs{\Im\partial_{k+2}}^{(-1)^{k+1}}=
		&\Pi_{i=0}^{k+1}\myabs{H_i}^{(-1)^i}\cdot\myabs{\Im\partial_{k+2}}^{(-1)^{k+1}}\end{array}\]
		This finishes the induction step, proves claim, and gives the desired statement.
		\end{enumerate}
		\setcounter{enumi}{6}
	\item\begin{enumerate}[label=(\arabic*).]
				\newcommand{\crz}{C(\mathbb{R}/\mathbb{Z})}
				\newcommand{\gc}{\Gamma_{\mathbb{C}}}
			\item Indeed, given any two $u,v\in\Gamma$ and $x\in\mathbb{R}$ we have $uv(x+1)=u(x+1)v(x+1)=-u(x)\cdot(-v(x)
				)=uv(x)$ and $v^2(x+1)=(-v(x))^2=v(x)$, hence $uv,v^2\in\crz$ and as $uv\cdot u+(-u^2)\cdot v=0$,
				we see that $u$ and $v$ are linearly dependent over $\crz$.
			\item $\Gamma$ {\it cannot} be a free module of rank bigger than 1, as it does not contain two independent
				elements (as was shown in previous subproblem). It remains therefore to show that it cannot be a free
				module of rank 1 (we rule out the trivially wrong possibility that $\Gamma$ is a free module of rank
				0, as it has more than 1 element in it, say $\sin\pi x$ and $\cos\pi x$). Assume,
				seeking contradiction, that $\Gamma$ is
				a free module of rank 1, then there is an element $h\in \Gamma$, such that every other element
				in $\Gamma$ would be a multiple of $h$ and element in $\crz$. Now, as $h(1)=-h(0)$, we have that
				either $h(1)=0$, or $h(0)$ and $h(1)$ have opposite signs, hence as $h$ is continuous, there is
				$0<c<1$, such that $h(c)=0$. In any case, $\exists c\in\mathbb{R},\;h(c)=0$. Now, take
				$v(x):=\cos(\pi x-\pi c)\in\Gamma$,
				we have $v(c)=1\neq 0$, while $h(c)=0$, so we cannot have $v=a\cdot h$ for $a\in\crz$. The obtained
				contradiction shows that $\Gamma$ cannot be a free module of rank 1 and finishes the proof.
			\item We will show that $\gc$ is a free module of rank 2 by exhibiting two elements of it, which are 
				independent over $\crz$ and together span the whole $\gc$. We claim that 
				$u(x):=e^{i\pi x}$ and $v(x):=ie^{i\pi x}$
				are such elements. First, we show their independence over $\crz$. Indeed, assume for
				some $a,b\in\crz$ we have $a\cdot u+b\cdot v=0$, so that $(a(x)+ib(x))e^{i\pi x}=0$. Now,
				as $\myabs{e^{i\pi x}}=1$ its nonzero, hence $a(x)+ib(x)=0$. Now, as both $a(x)$ and $b(x)$ are real,
				this implies $a\equiv b\equiv0$.This shows independence.

				Now, we will show that $\gc$ is spanned by $u(x)$ and $v(x)$ over $\crz$. Given $f\in\gc$, note that
				$f(x+2)=-f(x+1)=f(x)$, hence $f$ is periodic and continuous, hence bounded. Thus, for $A\geq 0$
				big enough $\myabs{Ae^{i\pi x}+f(x)}\geq \myabs{Ae^{i\pi x}}-\myabs{f(x)}=A-\myabs{f(x)}>0$ for
				all $x$ and thus $g(x):=Ae^{i\pi x}+f(x)$ is nonzero, while still in $\gc$. It is enough therefore
				to show that $g(x)$ is spanned by $u$ and $v$, as constant function with value
				$A\in\mathbb{R}$ is contained in $\crz$. Now, similarly to the first subproblem of this
				problem, 
				\[gu\cdot g+(-g^2)\cdot u=0\]
				As $u$ and $h$ both nowhere zero, division is allowed, yielding
				\[g=\frac{g}{u}\cdot u\]
				and $\mybra{\frac{g}{u}}(x+1)=\mybra{\frac{g}{u}}(x)$, as both $gu$ and $-g^2$ have this property
				similarly to the first subproblem of this problem. Finally, writing 
				\[\frac{g}{u}=a+ib\]
				for real-valued functions $a$ and $b$, we see that they satisfy the property $a(x+1)=a(x)$ and $b(x+1)
				=b(x)$, as $g/u$ did so. Then, $a,b\in\crz$ and hence
				\[g=(a+ib)u=au+bv\]
			\item Indeed, let's define $i\Gamma$ as the ring of purely complex-valued functions (that is $f(x):\mathbb{R}
				\to\mysetn{z\in\mathbb{C}}{\Re z=0}$) such that $f(x+1)=-f(x)$. Then, every $g\in\gc$ can be written
				as sum $g=\Re g+i\Im z$, where $\Re g\in\Gamma$ and $i\Im z\in i\Gamma$. Finally,
				if $h(x)=\Gamma\cap i\Gamma$, then $h(x)\equiv0$, as $h(x)$ should be both real and purely complex
				for every $x$, hence $\gc=\Gamma\oplus i\Gamma$, finishing the proof.
			\item Finally, as was show in subproblem (3), every element of $\gc$ is spanned by $e^{i\pi x}$
				and $ie^{i\pi x}$. As $\Gamma\in\gc$, this means that $\forall h\in\Gamma$, we have
				$h(x)=a(x)e^{i\pi x}+b(x)ie^{i\pi x}$ for $a,b\in\crz$ and taking the real part of both sides
				and using the fact that $a(x),b(x)\in\mathbb{R}$, we have $h(x)=a(x)\cos\pi x-b(x)\sin \pi x$,
				which shows that $\cos\pi x$ and $\sin\pi x$ span $\Gamma$ over $\crz$.
		\end{enumerate}
\end{enumerate}
\end{document}
%7-->3, canto9, cantoEx
